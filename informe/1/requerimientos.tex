\chapter{Especificación de requerimientos}

	\section{Introducción}
	El presente documento tiene como propósito definir las especificaciones
	funcionales y no funcionales para el desarrollo de una biblioteca de
	software que proveerá implementaciones de técnicas de registración, fusión
	y relleno de huecos de mallas tridimensionales.

	\section{Descripción}
		Se dispone de un sistema cámara-superficie giratoria, cuyas posiciones
		se encuentran fijas en el espacio y el eje de giro de la superficie se
		encuentra alineado con el eje vertical del dispositivo de captura.
		El objeto de interés se ubicará sobre la superficie giratoria, y se
		realizarán capturas a intervalos de giro regulares
		hasta totalizar una vuelta completa (360\textdegree).

		A partir de estas suposiciones, se desarrollarán algoritmos de
		registración de las capturas parciales,
		fusión de las mallas resultantes
		y relleno de huecos para conseguir una superficie cerrada.

		Las entradas serán nubes de puntos con valores de posición (x,y,z).
		No se dispondrá de información de textura, normales o conectividades. %[PANCHO] no contradice lo dicho en el anteproyecto?

		Como salida se tendrá una superficie cerrada triangulada.

		\subsection{Interfaces con software}
			Se utilizará la \emph{Point Cloud Library} (PCL)
			para realizar las operaciones sobre las mallas y nubes de puntos.
			Debido a esto, se utilizará el lenguaje de programación C++.

		%Volar, mandar todo a requerimientos
		%\subsection{Funciones del producto}
		%	Registración de mallas:
		%		El usuario dispondrá de funciones para
		%		determinar la transformación que lleve una malla
		%		a su alineación en el espacio de otra.

		%	Métricas:
		%		El usuario dispondrá de funciones para
		%		la obtención de métricas que le permitan
		%		evaluar el resultado de una registración

		%	Corrección de bucle:
		%		El usuario dispondrá de funciones para
		%		corregir el error propagado durante la registración.

		%	Reconstrucción de la superficie:
		%		El usuario dispondrá de funciones para
		%		obtener una triangulación a partir de mallas alienadas.

		%	Relleno de huecos:
		%		El usuario dispondrá de funciones para
		%		estimar la superficie en zonas carente de datos observados.

		\subsection{Suposiciones}
		%Se tiene un ambiente controlado, es la forma de trabajo diseñada
			El ángulo máximo entre dos mallas no podrá exceder los 60\textdegree.

			La cámara no se encontrará demasiado elevada respecto a la
			superficie giratoria. En ningún caso deberá superar el punto más alto del objeto. %[PANCHO]
			\clearpage

	\section{Requerimientos funcionales}    %[PANCHO] dejales un espacio entre medio, move la seccion asi queda en una pagina (sin cortarse por uno o dos renglones, es por facha, si no queres no hay drama) 
		El sistema deberá proveer las siguientes funcionalidades:

		\Requerimiento
			{Outliers}
			{Se debe disponer de funciones para la detección y eliminación de
			puntos considerados outliers.}

		\Requerimiento
			{Registración Inicial}
			{Se debe disponer de funciones que dadas dos mallas calculen una
			transformación que las acerque lo suficiente como para poder
			utilizar luego ICP.}

		\Requerimiento
			{Área solapada}
			{Se debe disponer de funciones que establezcan los puntos en común (o una buena
			aproximación) entre dos mallas ya alineadas burdamente.}

		\Requerimiento
			{Métricas}
			{Se debe disponer de funciones para evaluar la calidad de una registración.}

		\Requerimiento
			{Corrección de bucle}
			{Se debe disponer de funciones para corregir el error propagado durante la registración
			una vez que se haya realizado una vuelta con las capturas.}

		\Requerimiento
			{Combinación de nubes}
			{Se debe disponer de funciones para ajustar los puntos y sus normales,
			según la información provista por cada malla.}
			%[PANCHO] "según la información de cada malla de donde provengan" reformular? "segun la informacion proveniente de cada malla"

		\Requerimiento
			{Triangulación}
			{Se debe disponer de funciones para obtener una triangulación dada una nube de puntos tridimensional.}

		\Requerimiento
			{Relleno}
			{Se debe disponer de funciones para lograr que una triangulación sea cerrada. Se
			estimará una superficie en las zonas donde se carezca de
			información.}

	\section{Requerimientos no funcionales}
		Se identificaron los siguientes requerimientos no funcionales:
	
	    %[PANCHO] Agrega introduccion como la seccion anterior, y lista los requerimientos.
	
		\Requerimiento{Tiempo de ejecución}
		{No se espera una ejecución a tiempo real de los algoritmos implementados.}
		%Si bien no se busca una ejecución en tiempo real de los algoritmos,
		%éstos deberán correr en una PC estándar en tiempo de minutos. % [PANCHO] no prometas nada a esta altura, con aclarar que no esta destinado a tiempo real esta bien, estas cosas las usa gaston para joder despues.

		%El consumo de memoria no deberá ser muy elevado. %[PANCHO] Cambia este.
		
		% [PANCHO] Requerimientos no funcionales (ejemplos): - Desarrollado en lenguaje C++. - Empleando las bibliotecas XXXXX.
		\Requerimiento{Lenguaje de programación}
		{El producto se desarrollará en el lenguaje C++.}


		\Requerimiento{Sistemas operativos}
		{El producto desarrollado estará destinado a utilizarse en los sistemas
		operativos Windows y Linux.}
