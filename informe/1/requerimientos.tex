\documentclass{pfc}
\title{Especificación de requerimientos}
\author{Walter Bedrij}
\date{\today}

\newcommand{\Requerimiento}[2]{
	{\bfseries #1}\\
	#2
}

\begin{document}
	\maketitle
	\tableofcontents
	\section{Introducción}
	El presente documento tiene como propósito definir las especificaciones
	funcionales y no funcionales para el desarrollo de una biblioteca de
	software que proveerá implementaciones de técnicas de registración, fusión
	y relleno de huecos de mallas tridimensionales.

	\section{Descripción}
		Se dispone de un sistema cámara-superficie giratoria, cuyas posiciones
		se encuentran fijas en el espacio, y estando alineado el eje de giro de
		la superficie con el eje vertical del dispositivo de captura.
		El objeto de interés se ubicará sobre la superficie giratoria, y se
		realizarán capturas a intervalos de giro regulares
		hasta totalizar una vuelta completa (360\textdegree).

		A partir de estas restricciones, se desarrollarán algoritmos de
		registración de las capturas parciales,
		fusión de las mallas resultantes
		y relleno de huecos para conseguir una superficie cerrada.

		Las entradas serán nubes de puntos con valores de posición (x,y,z).
		No se dispondrá de información de textura, normales o conectividades.

		Como salida se tendrá una superficie cerrada triangulada.

		\subsection{Interfaces con software}
			Se utilizará la \emph{Point Cloud Library} (PCL)
			para realizar las operaciones sobre las mallas y nubes de puntos.
			Debido a esto, se utilizará el lenguaje de programación C++.

		\subsection{Funciones del producto}
			Registración de mallas:
				El usuario dispondrá de funciones para
				determinar la transformación que lleve una malla
				a su alineación en el espacio de otra.

			Métricas:
				El usuario dispondrá de funciones para
				la obtención de métricas que le permitan
				evaluar el resultado de una registración

			Corrección de bucle:
				El usuario dispondrá de funciones para
				corregir el error propagado durante la registración.

			Reconstrucción de la superficie:
				El usuario dispondrá de funciones para 
				obtener una triangulación a partir de mallas alienadas.

			Relleno de huecos:
				El usuario dispondrá de funciones para 
				estimar la superficie en zonas carente de datos observados.

		\subsection{Restricciones}
			El ángulo máximo entre dos mallas no podrá exceder los 60\textdegree.

			La cámara no se encontrará demasiado elevada respecto a la
			superficie giratoria, en ningún caso deberá superar el punto más alto del objeto.

	\section{Requerimientos}
		\Requerimiento
			{Outliers}
			{Funciones para la detección y eliminación de puntos considerados
			outliers.}

		\Requerimiento
			{Registración Inicial}
			{Se debe disponer de funciones que dadas dos mallas calculen una
			transformación que las acerque lo suficiente como para poder
			utilizar luego ICP}

		\Requerimiento
			{Área solapada}
			{Funciones que establezcan los puntos en común (o una buena
			aproximación) entre dos mallas ya alineadas burdamente}

		\Requerimiento
			{Ajuste de puntos}
			{Funciones para ajustar los puntos y sus normales según la
			información de cada malla de donde provengan}

		\Requerimiento
			{Triangulación}
			{Funciones para obtener una triangulación dada una nube de puntos tridimensional}

		\Requerimiento
			{Relleno}
			{Funciones para lograr que una triangulación sea cerrada. Se
			estimará una superficie en las zonas donde se carezca de
			información.}

\end{document}
