\documentclass{pfc}
\title{Relevamiento del estado del arte}
\author{Walter Bedrij}
\date{\today}

\begin{document}
%FIXME: alineación / registración
%\section{Informe bibliográfico}
	\section{General}

	\section{Herramientas}
		\subsection{C++}
			PCL se encuentra disponible para ser usada en C++.
			Existen proyectos para portarla a Python y Java,
			pero no se encuentran suficientemente avanzados.
		\subsection{Open Source Computer Vision Library (OpenCV)}
			Es una biblioteca de código abierto de
			visión computacional y aprendizaje maquinal.
			Cuenta con módulos de
			procesamiento de imágenes de profundidad
			y registración

			%No usé OpenCV porque eran imágenes rgb-d
			%y yo recibía nube de puntos
			%queda PCL

		\subsection{The Point Cloud Library (PCL)}
			\url{http://www.pointclouds.org}
			Es un framework de código abierto multiplataforma para el procesado de imágenes 2D/3D y nubes de puntos.
			Provee numerosos algoritmos state-of-the-art %FIXME
			para reducción de ruido, extracción de puntos salientes,
			cálculo de descriptores, registración,
			reconstrucción de superficies, entre otros.

			La documentación incluye tutoriales para cada módulo de la biblioteca
			y además se cuenta con listas de correos
			y canales de IRC para brindar soporte.
			%Registration with the Point Cloud Library A Modular Framework for Aligning in 3-D
			%3d_is_here_point_cloud_library_pcl.txt

		\subsection{CloudCompare, Meshlab}
			Son programas de procesamiento y edición de mallas de puntos 3D.
			Presentan herramientas de registración semiautomática (a partir de
			puntos seleccionados por el usuario), y cuentan con una
			implementación de poisson_surface_reconstruction
			para reconstrucción de superficies.

			Se utilizarán especialmente para visualización
			y comparación de resultados.

		\subsection{The Stanford 3D Scanning Repository}
			\url{http://graphics.stanford.edu/data/3Dscanrep/}

			%FIXME
			%The purpose of this repository is to make some
			%range data and detailed reconstructions available to the public.
			Repositorio público de escaneos y sus reconstrucciones 3D.

			Se utilizarán 5 modelos: armadillo, bunny, dragon, drill y happy.
			Estos cuentan con las transformaciones necesarias para la
			registración de cada captura y con reconstrucciones libres de ruido
			a diversas resoluciones y con método zippered %citar
			y volumétrico
			%zippered_polygon_meshes_from_range_images
			%a_volumetric_method_for_building_complex_models_from_range_images

		kinectfusion:
			%kinectfusion_real-time_3d_reconstruction_and_interaction_using_a_moving_depth_camera
			Poca diferencia entre las posiciones de la cámara
			Utiliza ICP para alinear
			Reconstrucción volumétrica en gpu

	\section{Alineación}
		%PCL_Registration_Tutorial
		La registración entre dos nubes de puntos se suele resolver mediante
		alguna de las variantes del algoritmo Iterative Closest Point (ICP).
		Sin embargo, para evitar caer en mínimos locales,
		se debe contar con una buena aproximación inicial.
		Se buscarán algorimos para conseguir esta aproximación inicial.

		%TODO:
			pipeline: keypoints -> features -> correspondences -> restrictions -> estimation


		Iterative closest point:
			Busca la transformación que minimice el error de alineación
			entre los puntos de las mallas
			%efficient_variants_of_the_icp_algorithm.txt
			%generalized_icp.txt
			implementado en PCL

		%PCL_Registration_Tutorial
		Keypoints:
			no buscar saliencias: Uniforme, random,
			basado en procesamiento de imágenes: harris, brisk
			específico 3d: ISS
			persistencia multiescala de features: fpfh
			%Fast Point Feature Histograms (FPFH) for 3D Registration

		Features:
			ISS, USC, SHOT:
				por cada keypoint se obtiene 
				una representación independiente de la vista,
					lo cual nos permite establecer las correspondencias;
				y un marco de referencia, que nos permite calcular la transformación con tan sólo dos puntos

		Correción del bucle:
			Debido a que la registración se hace de a pares el error aumenta
			con cada nueva malla, se tendrá entonces un corrimiento de los
			bordes que deberían cerrar el bucle.
				Ajustar según el corrimiento, y propagar el ajuste a las alineaciones anteriores
				%in-hand_scanning_with_online_loop_closure
				%embedded_deformation_for_shape_manipulation
				Realizan las correcciones "de la forma más rígida posible"

	\section{Fusión}
		Volumetric: superficie implícita %a_volumetric_method_for_building_complex_models_from_range_images.txt
			usado por kinectfusion
			arreglo de vóxeles (en gpu kf)
			raycasting desde la cámara hacia la superficie,
				rellenando con valores de distancia a la superficie
			extraer superficie D=0

			rellena huecos automáticamente

			%ver implementación pcl::TSDFVolume
		Zippered: %zippered_polygon_meshes_from_range_images.txt
			malla poligonal
			en dos pasos:
				1. aproximar la topología: reducir el área solapada eligiendo puntos de una de las mallas
				2. refinar mediante un promedio ponderado: mover por la normal, pesar según *confianza*
			La confianza depende de la distancia del punto al centro de la nube, y del ángulo de su normal respecto al del eye-target de la cámara

		Surfel: %in-hand_scanning_with_online_loop_closure.txt
			en lugar de triángulos, representación de discos en los vértices
			facilidad de actualización / agregado / eliminación de puntos
			propone manejo de algunos outliers

		Poisson: %poisson_surface_reconstruction
			requiere de puntos y normales
			transforma el problema en una ecuación de Poisson %(soporte local, ¿fem?)
			sensible al ruido
			rellena huecos

			implementado en cloudcompare
			% ver implementación pcl::Poisson


	\section{Relleno de huecos}
		volumetric rellena huecos automáticamente
			hueco: frontera entre vóxeles unseen/empty (nunca alcanzados por raycasting / parte externa de la superficie)
			%filling_holes_in_comple_surfaces_using_volumetric_diffusion
			convertir a una versión volumétrica y realizar una difusión
				utilizar línea de vista (raycasting a la cámara)
				(sólo en los alrededores de los huecos)
		Surface fitting
		Poisson rellena huecos automáticamente
			%Be careful; poisson reconstruction method is not a hole filling algorithm; its output is deformed compared to the original cloud!
			%poisson_surface_reconstruction
		Adhoc

\section{Bibliografía}
\end{document}
