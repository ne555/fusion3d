\documentclass{pfc}
\title{Relevamiento del estado del arte}

\begin{document}
%FIXME: alineación / registración
%\section{Informe bibliográfico}
	\maketitle
	\section{General}

	\section{Herramientas}
		\subsection{Open Source Computer Vision Library (OpenCV)}
			Es una biblioteca de código abierto de
			visión computacional y aprendizaje maquinal.
			Cuenta con módulos de
			procesamiento de imágenes de profundidad
			y registración.

			%No usé OpenCV porque eran imágenes rgb-d
			%y yo recibía nube de puntos
			%además de que tampoco tenían color
			%queda PCL

		\subsection{The Point Cloud Library (PCL)}
			\url{http://www.pointclouds.org}
			Es un framework de código abierto multiplataforma para el procesado de imágenes 2D/3D y nubes de puntos.
			Provee numerosos algoritmos state-of-the-art %FIXME
			para reducción de ruido, extracción de puntos salientes,
			cálculo de descriptores, registración,
			reconstrucción de superficies, entre otros.

			La documentación incluye tutoriales para cada módulo de la biblioteca
			y además se cuenta con listas de correos
			y canales de IRC para brindar soporte.
			%Registration with the Point Cloud Library A Modular Framework for Aligning in 3-D
			%3d_is_here_point_cloud_library_pcl.txt

			PCL se encuentra disponible para ser usada en C++.
			Existen proyectos para portarla a Python y Java,
			pero no se encuentran suficientemente avanzados.

		\subsection{CloudCompare, Meshlab}
			Son programas de procesamiento y edición de mallas de puntos 3D.
			Presentan herramientas de registración semiautomática (a partir de
			puntos seleccionados por el usuario), y cuentan con una
			implementación del algoritmo \emph{Poisson Surface Reconstruction}
			para reconstrucción de superficies.

			Se utilizarán especialmente para visualización
			y comparación de resultados.

		\subsection{The Stanford 3D Scanning Repository}
			\url{http://graphics.stanford.edu/data/3Dscanrep/}

			Este proyecto surge debido a la falta de disponibilidad
			por parte de los investigadores
			de acceso a modelos poligonales densos
				o a la infraestructura para obtener los mismos.
			Por lo cual se crea un repositorio público de
			escaneos tridimensionales y sus correspondientes reconstrucciones.

			Los modelos fueron escaneados mediante un escáner Cyberware 3030~MS,
			el cual es un escáner láser de barrido.
			Se provee de un archivo de configuración que lista por cada captura
			las transformaciones necesarias para alinearla en un sistema de
			referencia global.

			Las capturas fueron combinadas para producir una única malla
			triangular utilizando el método de \emph{zippering} o bien
			\emph{volumetric merging}, ambos métodos desarrollados en Stanford.


			Se seleccinonarion estos 5 modelos para ser utilizados: armadillo, bunny, dragon, drill y happy.

		\subsection{KinectFusion}
			%kinectfusion_real-time_3d_reconstruction_and_interaction_using_a_moving_depth_camera
			Debido a que uno de los objetivos era lograr
			una implementación en tiempo real,
			el algoritmo de registración requiere de
			poca variación entre capturas de
			la posición relativa cámara-objeto.

			Para realizar la combinación utiliza una variación del método de
			\emph{volumetric merging} sobre GPU.

	\section{Registración}
		%PCL_Registration_Tutorial
		La registración entre dos nubes de puntos se suele resolver mediante
		alguna de las variantes del algoritmo \emph{Iterative Closest Point (ICP)}. 
		Sin embargo, para evitar caer en mínimos locales,
		se debe contar con una buena aproximación inicial.
		Por esto, es necesario desarrollar algorimos para conseguir esta
		aproximación inicial.

		Iterative closest point:
			Busca la transformación que minimice el error de alineación
			entre los puntos de las mallas
			%efficient_variants_of_the_icp_algorithm.txt
			%generalized_icp.txt
			implementado en PCL


		% efficient_variants_of_the_icp_algorithm
		Los algoritmos de registración se pueden dividir en los siguientes pasos:
		\begin{enumerate}
			\item Selección de puntos de la entrada (\emph{keypoints}).
			\item Utilizar descriptores para establecer correspondencias entre los puntos de las nubes.
			\item Rechazar correspondencias para reducir los \emph{outliers}.
			\item Alineación.
		\end{enumerate}

		En cuanto a la selección de puntos se tienen como opciones:
		\begin{itemize}
			\item No buscar saliencias: utilizar todos los puntos o realizar un submuestreo uniforme o aleatorio.
			\item Algoritmos basados en procesamiento de imágenes: como harris y brisk.
			\item Algoritmos específicos para puntos 3D: como ISS.
			\item Búsqueda de puntos cuyos descriptores sean persistentes a varias escalas: se requiere de formas eficientes de calcular los descriptores, como FPFH.
		\end{itemize}

		Descriptores:
			Por cada \emph{keypoint} se calculará un descriptor que nos
			permitirá determinar las correspondencias entre las dos nubes de
			puntos.
			Un descriptor es una representación compacta
			de la vecindad de un punto,
			siendo importante además establecer el límite de esta vecindad.

			Para puntos en el espacio, los descriptores utilizarán las
			posiciones relativas de los vecinos o los ángulos entre sus
			normales, ponderándolos según la distancia al punto de interés.
			Por ejemplo: el descriptor 3DSC define un histograma tridimensional
			cuantizando la distancia, azimuth y elevación entre los puntos;
			el descriptor FPFH realiza algo similar con las normales.

			Ciertos descriptores, como ISS y SHOT, establecen un marco de
			referencia que permite obtener una estimación de la transformación
			entre las nubes a partir de solamente dos puntos.

		\subsection{Correción del bucle}
			Debido a que la registración se hace de a pares el error aumenta
			con cada nueva malla.
			Este error generará artefactos que se evidenciarán sobre todo como
			un corrimiento apreciable entre los bordes de la primera y última
			malla de la lista.

			Para ajustar este corrimiento se podrá perturbar la última
			registración y luego propagar esta perturbación en las alineaciones
			anteriores.
			se tendrá entonces un corrimiento de los
			bordes que deberían cerrar el bucle.

			%in-hand_scanning_with_online_loop_closure
			%embedded_deformation_for_shape_manipulation
			Realizan las correcciones "de la forma más rígida posible"

	\section{Fusión}
		Volumetric: superficie implícita %a_volumetric_method_for_building_complex_models_from_range_images.txt
			usado por kinectfusion
			arreglo de vóxeles (en gpu kf)
			raycasting desde la cámara hacia la superficie,
				rellenando con valores de distancia a la superficie
			extraer superficie D=0

			rellena huecos automáticamente

			%ver implementación pcl::TSDFVolume
		Zippered: %zippered_polygon_meshes_from_range_images.txt
			malla poligonal
			en dos pasos:
				1. aproximar la topología: reducir el área solapada eligiendo puntos de una de las mallas
				2. refinar mediante un promedio ponderado: mover por la normal, pesar según *confianza*
			La confianza depende de la distancia del punto al centro de la nube, y del ángulo de su normal respecto al del eye-target de la cámara

		Surfel: %in-hand_scanning_with_online_loop_closure.txt
			en lugar de triángulos, representación de discos en los vértices
			facilidad de actualización / agregado / eliminación de puntos
			propone manejo de algunos outliers

		Poisson: %poisson_surface_reconstruction
			requiere de puntos y normales
			transforma el problema en una ecuación de Poisson %(soporte local, ¿fem?)
			sensible al ruido
			rellena huecos

			implementado en cloudcompare
			% ver implementación pcl::Poisson


	\section{Relleno de huecos}
		volumetric rellena huecos automáticamente
			hueco: frontera entre vóxeles unseen/empty (nunca alcanzados por raycasting / parte externa de la superficie)
			%filling_holes_in_comple_surfaces_using_volumetric_diffusion
			convertir a una versión volumétrica y realizar una difusión
				utilizar línea de vista (raycasting a la cámara)
				(sólo en los alrededores de los huecos)
		Surface fitting
		Poisson rellena huecos automáticamente
			%Be careful; poisson reconstruction method is not a hole filling algorithm; its output is deformed compared to the original cloud!
			%poisson_surface_reconstruction
		Adhoc

\section{Bibliografía}
\end{document}
