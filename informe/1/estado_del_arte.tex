\chapter{Relevamiento del estado del arte}
	Con el fin de identificar requerimientos y obtener el marco teórico
	necesario para el desarrollo del proyecto se analizaron herramientas de
	software utilizadas para la reconstrucción tridimensional y textos
	científicos acordes a la temática.

	El proceso de reconstrucción se dividirá en tres etapas:
	\begin{enumerate}
		\item Registración: donde se determinan las transformaciones necesarias
			para llevar cada vista a su correcta posición en un marco de
			referencia global.
		\item Fusión: donde se unifica el aporte de cada vista para obtener una
			superficie que las englobe.
		\item Relleno de huecos: donde se asegura que la superficie global sea
			cerrada, es decir, que encierre un volumen.
	\end{enumerate}


%FIXME: alineación / registración
%\section{Informe bibliográfico}
	%\section{Introducción}

	%	Iterative closest point:
	%		Busca la transformación que minimice el error de alineación
	%		entre los puntos de las mallas
	%		%efficient_variants_of_the_icp_algorithm.txt
	%		%generalized_icp.txt
	%		implementado en PCL

	\section{Registración}
		La registración entre dos nubes de puntos se suele resolver mediante
		alguna de las variantes del algoritmo \emph{Iterative Closest Point (ICP)}. 
		Sin embargo, para evitar caer en mínimos locales,
		se debe contar con una buena aproximación inicial.
		Por esto, es necesario desarrollar algoritmos para conseguir esta
		aproximación inicial.\cite{7271006}
		%Registration with the Point Cloud Library A Modular Framework for Aligning in 3-D



		% efficient_variants_of_the_icp_algorithm
		Los algoritmos de registración se pueden dividir en los siguientes pasos:
		\begin{enumerate}
			\item Selección de puntos de la entrada (\emph{keypoints}).
			\item Utilizar descriptores para establecer correspondencias entre los puntos de las nubes.
			\item Rechazar correspondencias para reducir los \emph{outliers}.
			\item Alineación.\cite{conf/3dim/RusinkiewiczL01}
		\end{enumerate}

		En cuanto a la selección de puntos se tienen como opciones:
		\begin{itemize}
			\item No buscar saliencias: utilizar todos los puntos o realizar un submuestreo uniforme o aleatorio.
			\item Algoritmos basados en procesamiento de imágenes: como harris y brisk.
			\item Algoritmos específicos para puntos 3D: como ISS.
			\item Búsqueda de puntos cuyos descriptores sean persistentes a varias escalas: se requiere de formas eficientes de calcular los descriptores, como FPFH.
		\end{itemize}

		%Descriptores:
			Por cada \emph{keypoint} se calculará un descriptor que nos
			permitirá determinar las correspondencias entre las dos nubes de
			puntos.
			Un descriptor es una representación compacta
			de la vecindad de un punto,
			siendo importante además establecer el límite de esta vecindad.

			Para puntos en el espacio, los descriptores utilizarán las
			posiciones relativas de los vecinos o los ángulos entre sus
			normales, ponderándolos según la distancia al punto de interés.
			Por ejemplo: el descriptor 3DSC define un histograma tridimensional
			cuantizando la distancia, azimut y elevación entre los puntos;
			el descriptor FPFH realiza algo similar con las normales.

			Ciertos descriptores, como ISS y SHOT, establecen un marco de
			referencia que permite obtener una estimación de la transformación
			entre las nubes a partir de solamente dos puntos.

		\subsection{Corrección del bucle}
			Debido a que la registración se hace de a pares el error aumenta
			con cada nueva malla.
			Este error generará artefactos que se evidenciarán sobre todo como
			un corrimiento apreciable entre los bordes de la primera y última
			malla de la lista.

			Para ajustar este corrimiento se podrá perturbar la última
			registración y luego propagar esta perturbación en las alineaciones
			anteriores.

			Estas perturbaciones provocarán una deformación de la malla.
			%in-hand_scanning_with_online_loop_closure
			%FIXME: describir el algoritmo
			En \cite{5457479}
			se plantea un algoritmo para el cálculo
			y la propagación de estas perturbaciones, de forma que la deformación sea ``lo más rígida posible''.

	\section{Fusión}
		Una vez alineadas las superficies éstas deben combinarse en una única malla resultante

		\subsection{Volumetric merge}
		Se divide el espacio en un arreglo de vóxeles que contienen la
		distancia con signo desde el centro del vóxel a la superficie (la
		distancia será positiva si el vóxel se encuentra en la parte exterior y
		negativa si se encuentra en el interior). De esta forma, en los vóxeles
		donde la distancia es 0, se tiene definida de forma implícita a la
		superficie, la cual puede extraerse con, por ejemplo, \emph{marching cubes}.

		En cada nueva registración se actualizarán las distancias de los
		vóxeles mediante raycasting
		desde la posición estimada de la cámara hacia la superficie,
		realizando un promedio ponderado con los valores anteriormente calculados.\cite{Curless:1996:VMB:237170.237269} %a_volumetric_method_for_building_complex_models_from_range_images.txt
		Debido a que sólo nos interesan las zonas donde la distancia es cercana a 0 es posible utilizar una representación rala para los vóxeles.\cite{Steinbrucker:2013:LMS:2586117.2586926} % large-scale_multi-resolution_surface_reconstruction_from_rgb-d_sequences

		%PCL provee la clase TSDFVolume para las operaciones sobre el arreglo de vóxeles.

		\subsection{Zippered}
		%zippered_polygon_meshes_from_range_images.txt
		Se obtendrá como resultado una malla poligonal.

		El algoritmo trabaja en dos pasos:
		\begin{enumerate}
			\item Aproximación de la topología: se reduce el área solapada
				entre las mallas seleccionando una de las representaciones para
				cada punto en común.
			\item Refinado: se ajusta la posición de cada punto moviéndolo
				según su normal mediante un promedio ponderado dependiente del
				nivel de \emph{confianza} que posee el punto en cada vista.\cite{Turk:1994:ZPM:192161.192241}
		\end{enumerate}

		\subsection{Representación de surfel}
		%in-hand_scanning_with_online_loop_closure.txt
		En lugar de realizar una triangulación, en cada punto se establece un disco de radio variable orientado según la normal.
		Esto facilita la actualización, el agregado y la eliminación de puntos en cada nueva registración y además permite detectar ciertos \emph{outliers}.\cite{5457479}

		\subsection{Reconstrucción Poisson}
		%poisson_surface_reconstruction
		Utiliza una función indicadora $\chi$ que recibe el valor 1 para puntos
		dentros del modelo y 0 para puntos en el exterior.
		La superficie del objeto queda determinada por la frontera entre estos
		valores, y el gradiente de la función indicadora se corresponde con las
		normales de los puntos del objeto $\vec{n}$.

		El problema consiste en encontrar la function $\chi$ cuyo gradiente
		aproxime el campo vectorial definido por las normales. Aplicando el
		operador de divergencia, se obtiene un problema de Poisson: calcular la
		función escalar $\chi$ cuyo laplaciano equivale a la divergencia del
		campo vectorial $\vec{n}$.\cite{Kazhdan:2006:PSR:1281957.1281965}

		\[\Delta\chi \equiv \nabla \cdot\nabla\chi = \nabla \vec{n}\]

		El algoritmo rellena huecos automáticamente, pero puede resultar demasiado agresivo, uniendo porciones que deberían permanecer separadas.


	\section{Relleno de huecos}
		Los huecos son regiones que ninguna de las vistas logró capturar.
		La existencia de huecos elimina la propiedad de \emph{watertight} de la
		malla, la cual es necesaria para su impresión 3D.

		El método de fusión \emph{volumetric merge} rellena huecos
		automáticamente al definirlos como la frontera entre vóxeles externos y
		aquellos nunca alcanzados por el raycasting.

		En el caso de una malla poligonal,
		pueden identificarse fácilmente los vértices que limitan el hueco
		como aquellos que definen una arista que corresponde a sólo un elemento.
		Entonces puede operarse solamente en la vecindad del hueco.

		Una opción es convertir la superficie en una
		representación volumétrica y luego realizar la difusión de la función
		de distancia en la zona del hueco.
		Este proceso se itera hasta que no se detectan cambios significativos en la superficie.\cite{fillingholes}
		%filling_holes_in_comple_surfaces_using_volumetric_diffusion

		Otra opción es transformar el problema al de una interpolación.
		\begin{itemize}
			\item Por cada hueco se ajusta un plano a los puntos del borde y
				sus vecinos.
			\item Se proyectan los puntos a este plano, obteniéndose un mapa de
				altura del contorno del hueco.
			\item Utilizando el algoritmo de \emph{Moving Least Squares} se
				ajusta una superficie a este mapa de altura.
			\item Se realiza un muestreo sobre esta superficie para obtener
				puntos que rellenen el hueco.
		\end{itemize}
		De esta forma el parche de reconstrucción se unirá suavemente a la malla original.\cite{Filling_holes_on_locally_smooth_surfaces}
			%Filling holes on locally smooth surfaces reconstructed from point clouds





	\section{Herramientas}
	%Intro

		\subsection{Open Source Computer Vision Library (OpenCV)}
			Es una biblioteca de código abierto de
			visión computacional y aprendizaje maquinal.
			%Cuenta con módulos de
			%procesamiento de imágenes de profundidad
			%y registración.

			En un principio se consideró utilizar la información de textura de
			las capturas para poder lograr la registración, pero la base de
			datos utilizada sólo contenía información geométrica.
			Debido a esto, no se utilizaron las funcionalidades de esta biblioteca.

			%No usé OpenCV porque eran imágenes rgb-d
			%y yo recibía nube de puntos
			%además de que tampoco tenían color
			%queda PCL

		\subsection{The Point Cloud Library (PCL)}
			%\url{http://www.pointclouds.org}

			Es un framework de código abierto multiplataforma para el procesado de imágenes 2D/3D y nubes de puntos.
			Provee numerosos algoritmos state-of-the-art %FIXME
			para reducción de ruido, extracción de puntos salientes,
			cálculo de descriptores, registración,
			reconstrucción de superficies, entre otros.

			La documentación incluye tutoriales para cada módulo de la biblioteca
			y además se cuenta con listas de correos
			y canales de IRC para brindar soporte.
			%Registration with the Point Cloud Library A Modular Framework for Aligning in 3-D
			%3d_is_here_point_cloud_library_pcl.txt

			%Listar algunas funciones
			%tiene para registración, fusión volumétrica, poisson, etc

			PCL se encuentra disponible para ser usada en C++.
			Existen proyectos para portarla a Python y Java,
			pero no se encuentran suficientemente avanzados.

		\subsection{CloudCompare, Meshlab}
			Son programas de procesamiento y edición de mallas de puntos 3D.
			Presentan herramientas de registración semiautomática (a partir de
			puntos seleccionados por el usuario), y cuentan con una
			implementación del algoritmo \emph{Poisson Surface Reconstruction}
			para reconstrucción de superficies.

			Se utilizarán especialmente para visualización
			y comparación de resultados.

		\subsection{The Stanford 3D Scanning Repository}
			%\url{http://graphics.stanford.edu/data/3Dscanrep/}

			Este proyecto surge debido a la falta de disponibilidad
			por parte de los investigadores
			de acceso a modelos poligonales densos
				o a la infraestructura para obtener los mismos.
			Por lo cual, se crea un repositorio público de
			escaneos tridimensionales y sus correspondientes reconstrucciones.

			Los modelos fueron escaneados mediante un escáner Cyberware 3030~MS,
			el cual es un escáner láser de barrido.
			Se provee de un archivo de configuración que lista por cada captura
			las transformaciones necesarias para alinearla en un sistema de
			referencia global.

			Las capturas fueron combinadas para producir una única malla
			triangular utilizando el método de \emph{zippering} o bien
			\emph{volumetric merging}, ambos métodos desarrollados en Stanford.\cite{StanfordScanRep}


			Se seleccinonarion estos 5 modelos: armadillo, bunny, dragon, drill y happy.

		\subsection{KinectFusion}
			Es el algoritmo desarrollado por Microsoft para lograr reconstrucciones tridimensionales utilizando el dispositivo Kinect.

			%kinectfusion_real-time_3d_reconstruction_and_interaction_using_a_moving_depth_camera
			Debido a que uno de los objetivos era lograr
			una implementación en tiempo real,
			el algoritmo de registración requiere de
			poca variación entre capturas de
			la posición relativa cámara-objeto.

			Para realizar la combinación utiliza una variación del método de
			\emph{volumetric merging} sobre GPU.\cite{Izadi:2011:KRR:2047196.2047270}

%\nocite{*}
\bibliographystyle{alpha}
\bibliography{biblio}
