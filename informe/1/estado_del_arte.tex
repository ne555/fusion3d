\documentclass{pfc}
\title{Relevamiento del estado del arte}
\author{Walter Bedrij}
\date{\today}

\begin{document}
%FIXME: alineación / registración
%\section{Informe bibliográfico}
	\section{General}

	\section{Herramientas}
		\subsection{C++}
			PCL se encuentra disponible para ser usada en c++
			Existen proyectos para portarla a python y java,
			pero no se encuentran suficientemente avanzados.
		\subsection{Open Source Computer Vision Library (OpenCV)}
			Es una biblioteca de código abierto de
			visión computacional y aprendizaje maquinal.
			Cuenta con módulos de
			procesamiento de imágenes de profundidad
			y registración

			%No usé OpenCV porque eran imágenes rgb-d
			%y yo recibía nube de puntos
			%queda PCL

		\subsection{The Point Cloud Library (PCL)}
			\url{http://www.pointclouds.org}
			Es un framework de código abierto multiplataforma para el procesado de imágenes 2D/3D y nubes de puntos.
			Provee numerosos algoritmos state-of-the-art %%FIXME
			para reducción de ruido, extracción de puntos salientes,
			cálculo de descriptores, registración,
			reconstrucción de superficies, entre otros.

			La documentación incluye tutoriales para cada módulo de la biblioteca
			y además se cuenta con listas de correos
			y canales de IRC para brindar soporte.



			%Registration with the Point Cloud Library A Modular Framework for Aligning in 3-D
			%3d_is_here_point_cloud_library_pcl.txt
		\subsection{CloudCompare, Meshlab}
			Son programas de procesamiento y edición de mallas de puntos 3D.
			Presentan herramientas de registración semiautomática (a partir de
			puntos seleccionados por el usuario), y cuentan con una
			implementación de \cite{poisson-reconstruction} para reconstrucción
			de superficies.

			Se utilizarán especialmente para visualización
			y comparación de resultados.

		\subsection{The Stanford 3D Scanning Repository}
			\url{http://graphics.stanford.edu/data/3Dscanrep/}

			%FIXME
			%The purpose of this repository is to make some
			%range data and detailed reconstructions available to the public.
			Repositorio público de escaneos y sus reconstrucciones 3D.
			Las capturas se realizaron mediante un escáner de barrido,
			el cual brinda información de las conectividades de los puntos.
			Sin embargo, esta información se descartó ya que
			\cite{Pancho} no la provee.

			Se utilizarán 5 modelos: armadillo, bunny, dragon, drill y happy.
			Estos cuentan con las transformaciones necesarias para la registración de cada captura y con reconstrucciones libres de ruido a diversas resoluciones y con método zippered %citar
			y volumétrico

			%the scanner typically examines a high-resolution image of the
			%reflected laser line, deciding from its profile in this image
			%whether two adjacent vertices should be connected by a surface or
			%not. Cyberware laser scanners do this. Thus, if you discard mesh
			%connectivity, you are discarding real and possibly useful
			%information about the underlying surface.


		%TODO
		kinfu:
		kinectfusion:

	\section{Preproceso}

	\section{Alineación}
	%Sacar de PCL_Registration_Tutorial.pdf
		La registración entre dos nubes de puntos se suele resolver mediante
		alguna de las variantes del algoritmo Iterative Closest Point (ICP).
		%citar, citar, citar
		Sin embargo, para evitar caer en mínimos locales,
		se debe contar con una buena aproximación inicial. %citar
		Este trabajo buscará algorimos para conseguir esta aproximación inicial.

		%TODO:
			pipeline: keypoints -> features -> correspondences -> restrictions -> estimation
		%Iterative closest point:

		Keypoints:
			no buscar saliencias: Uniforme, random,
			basado en procesamiento de imágenes: harris, brisk
			persistencia multiescala:

		Features:
			Intrinsic shape signatures:
				Por keypoint se tiene
				una representación independiente de la vista,
					lo cual nos permite establecer las correspondencias;
				y un marco de referencia, que nos permite calcular la transformación.

		Correción del bucle:
			Debido a que la registración se hace de a pares el error aumenta
			con cada nueva malla, se tendrá entonces un corrimiento de los
			bordes que deberían cerrar el bucle.
				ajusta según el corrimiento, y propaga el ajuste a las alineaciones anteriores
				%in-hand_scanning_with_online_loop_closure.pdf
				%embedded_deformation_for_shape_manipulation.pdf

	\section{Resolver bucle}
		Debido a que la registración se realiza entre pares de nubes de puntos,
		el error es propagado en una dirección, por lo cual es necesario distribuírlo sobre todas las registraciones
		%Embedded Deformation for Shape Manipulation
		%in-hand_scanning_with_online_loop_closure.txt

	\section{Fusión}
		Volumetric: superficie implícita%a_volumetric_method_for_building_complex_models_from_range_images.txt
			usado por kinectfusion
			arreglo de vóxeles (en gpu kf)
			raycasting desde la cámara hacia la superficie,
				rellenando con valores de distancia a la superficie
			extraer superficie D=0

			rellena huecos automáticamente
			%pcl::TSDFVolume
		Zippered: %zippered_polygon_meshes_from_range_images.txt
			malla poligonal
			en dos pasos:
				1. aproximar la topología: reducir el área solapada eligiendo puntos de una de las mallas
				2. refinar mediante un promedio ponderado: mover por la normal, pesar según *confianza*

		Surfel: %in-hand_scanning_with_online_loop_closure.txt
			representación de discos en los vértices
			facilidad de actualización / agregado / eliminación de puntos (sobre malla poligonal, asegura consistencia)
			manejo de algunos outliers
			%ver diferencia con zippered
		Poisson: %poisson_surface_reconstruction.pdf
			requiere de puntos y normales
			transforma el problema en una ecuación de Poisson %(¿soporte local, fem?)
			sensible al ruido
			rellena huecos
			implementado en pcl, también en cloudcompare
			usar para comparar con el propio


	\section{Relleno de huecos}
		volumetric rellena huecos automáticamente
			hueco: frontera entre unseen - empty
			backdrop %(¿en la adquisición?)
			%filling_holes_in_comple_surfaces_using_volumetric_diffusion
			convertir a una versión volumétrica y realizar una difusión
				utilizar línea de vista
		Surface fitting
		Poisson rellena huecos automáticamente
			%poisson_surface_reconstruction.pdf
		Adhoc

\section{Bibliografía}

% ¿esto estaba?
%Definiciones de las técnicas de
%	alineación
%	y fusión de mallas
%	y rellenado de huecos.
\end{document}
