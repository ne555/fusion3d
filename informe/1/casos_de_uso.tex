\chapter{Casos de uso}


	%verlo como unit testing
	%ver formato
	\CasoUso{Preprocesar nube}
		\CUField{Actor}{Usuario}
		\CUNormal
		\begin{enumerate}
			\item El usuario introduce el nombre de archivo de la nube de puntos.
			\item El sistema lee la malla del disco.
			\item El sistema realiza una reducción de ruidos y eliminación de outliers.
			\item El sistema calcula normales y vecindades de los puntos de la nube.
			\item El sistema devuelve una nube de puntos preprocesada
		\end{enumerate}

	\CasoUso{Alinear dos nubes de puntos}
		\CUField{Actor}{Usuario}
		\CUField{Descripción}{Se calcula la transformación que posiciona la nube \emph{fuente}
		en el sistema de referencia de la nube \emph{objetivo}.}
		\CUNormal
		\begin{enumerate}
			\item El usuario introduce las nubes preprocesadas \emph{fuente} y la nube \emph{objetivo}. \IncCU{Preprocesar nube}
			\item El usuario selecciona los parámetros del algoritmo.
			\item El sistema selecciona puntos salientes.
			\item El sistema calcula descriptores y sus correspondencias.
			\item El sistema elimina correspondencias espurias.
			\item El sistema devuelve la transformación de alineación.
		\end{enumerate}

	\CasoUso{Obtener métrica de alineación}
		\CUField{Actor}{Usuario}
		\CUField{Descripción}{Se calculan medidas sobre las nubes alineadas para
		reflejar la calidad de la alineación.}
		\CUNormal
		\begin{enumerate}
			\item El usuario introduce las nubes preprocesadas \emph{fuente} y la nube \emph{objetivo}. \IncCU{Preprocesar nube}
			\item El usuario introduce una \emph{transformación}.
			\item El sistema aplica la \emph{transformación} a la nube \emph{fuente}.
			\item El sistema determina los puntos que corresponden al área común entre las nubes (área solapada).
			\item El sistema calcula medidas sobre el área solapada:
				cantidad de puntos en relación a las nubes originales,
				distancia punto a punto entre las nubes.
		\end{enumerate}


	\CasoUso{Corregir error de bucle}
		\CUField{Actor}{Usuario}
		\CUField{Descripción}{Se ajustan las transformaciones de las nubes para lograr que el bucle cierre correctamente.}
		\CUField{Requisitos}{}
		\begin{itemize}
			\item Las nubes describen una vuelta completa.
			\item Las nubes se encuentran alineadas.
		\end{itemize}
		\CUNormal
		\begin{enumerate}
			\item El usuario introduce una lista de nubes preprocesadas. \IncCU{Preprocesar nube}
			\item El sistema calcula la transformación de la última nube para lograr que la superficie cierre. \UseCU{Alinear dos nubes de puntos}
			\item El sistema propaga el ajuste hacia las otras nubes.
			\item El sistema devuelve una lista de las transformaciones.
		\end{enumerate}

	%\CasoUso{Rechazar malla}
	%	Actor: {Usuario}

	\CasoUso{Extraer superficie}
		\CUField{Actor}{Usuario}
		\CUField{Descripción}{Se obtendrá una malla poligonal que represente una colección de nubes de puntos.}
		\CUField{Requisito}{Las nubes se encuentran alineadas.}
		\CUNormal
		\begin{enumerate}
			\item El usuario introduce una lista de nubes.
			\item El sistema une todas las nubes en una sola.
			\item El sistema filtra el ruido y elimina los outliers de la nube.
			\item El sistema calcula las vecindades de los puntos y realiza una triangulación.
			\item El sistema calcula las normales para cada triángulo.
		\end{enumerate}

	\CasoUso{Identificar huecos}
		\CUField{Actor}{Usuario}
		\CUField{Descripción}{Se determinarán las zonas donde existen huecos.}
		\CUNormal
		\begin{enumerate}
			\item El usuario introduce una malla poligonal.
			\item El sistema identifica los vértices que forman parte de un hueco.
			\item El sistema calcula los contornos de los huecos.
			\item El sistema devuelve una lista con cada hueco y los vértices de su contorno.
		\end{enumerate}

	\CasoUso{Rellenar huecos}
		\CUField{Actor}{Usuario}
		\CUField{Descripción}{Se estimarán las posiciones de puntos dentro de las zonas
		donde existen huecos.}
		\CUNormal
		\begin{enumerate}
			\item El usuario introduce una malla poligonal.
			\item El usuario identifica un hueco. \IncCU{Identificar huecos}
			\item El sistema agrega puntos en el interior del hueco.
			\item El sistema integra los nuevos puntos a la malla.
		\end{enumerate}
