\chapter{Módulo de fusión}
	Al agregar una vista se obtendrá nueva información del objeto en aquellos
	puntos que no eran antes visibles.
	Además, en las zonas solapadas confirmará o refutará la información ya presente.

	El módulo de fusión se encargará de combinar esta información para obtener
	finalmente una malla que represente la porción observada del objeto.

	Cada punto de una vista tendrá asociado un valor de confianza que indica la
	probabilidad de que el punto no sea un outlier.
	Este valor se inicializa según el ángulo de su normal respecto a la línea
	de la cámara y su distancia al centro de la captura.

	Cuando se agrega un nuevo punto, se busca el más cercano en la nube global
	Si la distancia supera un umbral, se considera que es un nuevo punto. Caso
	contrario, se ajustan la posición y normal del punto en la nube global
	según los niveles de confianza de ambos.  Además, se contará desde cuántas
	vistas es observado cada punto.


	Finalmente se descartarán aquellos puntos que hayan sido observados desde sólo una vista y tengan bajo nivel de confianza.

	%diagramas de clase

	%informes de prueba

	%casos de uso/funcionalidades resueltas
