\documentclass{pfc}
\usepackage{algorithmicx}
\usepackage{algorithm}
\usepackage[noend]{algpseudocode}

\makeatletter
 \renewcommand{\ALG@name}{Algoritmo}
\makeatother
\renewcommand{\algorithmicfunction}{}

\title{Diseño y desarrollo}

\begin{document}
	\maketitle
	El proceso de reconstrucción tridimensional generalmente se divide en tres etapas:
	\begin{enumerate}
		\item Registración: donde se determinan las transformaciones necesarias
			para llevar cada vista a su correcta posición en un marco de
			referencia global.
		\item Fusión: donde se unifica el aporte de cada vista para obtener una
			superficie que las englobe.
		\item Relleno de huecos: donde se asegura que la superficie global sea
			cerrada, es decir, que encierre un volumen.
	\end{enumerate}
	Cada una de estas etapas se corresponde con un módulo del sistema.
	A continuación se describirán los detalles de diseño e implementación de
	los métodos propuestos para cada módulo, junto con sus resultados
	preliminares.

	\chapter{Módulo de registración}
	%¿Qué es la registración?
	Dado un conjunto de nubes de puntos correspondientes a distintas vistas de
	un objeto, la registración consiste en calcular las transformaciones de
	rotación y translación que lleven a cada vista a un sistema global de forma
	que las zonas comunes encajen perfectamente.

	Se tomará la primera captura como marco de referencia, cada nueva vista se
	alineará con la anterior hasta completar una vuelta sobre el objeto.
	El problema, entonces, se resuelven encontrando para cada punto de la nube
	A su posición, si es que es visible, en la nube B, para luego estimar la
	transformación que alinee todos estos puntos.

	Las coordenadas espaciales de un punto no nos suministran suficiente
	información para poder identificarlo en otra vista, es necesario utilizar
	sus relaciones con otros puntos en una vecindad.
	Así, podrá describirse al punto mediante las posiciones relativas, la
	densidad o la orientación, para luego corresponderlo con el más parecido en
	la otra vista.
	Sin embargo, los puntos en zonas homogéneas de la nube serán descriptos de
	forma muy similar, con lo que se corre el riesgo de realizar
	correspondencias equívocas que producirán una alineación errónea.

	Para solventar este último problema, se puede tomar un subconjunto de los
	puntos de entrada, aquellos puntos que por sus características de vecindad
	sea más probable que tengan un descriptor único (\emph{keypoints}).

	Entonces, los métodos de registración seguirán los siguientes pasos:
	\begin{enumerate}
		\item Selección de puntos de la entrada (\emph{keypoints}).
		\item Cálculo de descriptores y determinación de correspondencias.
		\item Rechazo de correspondencias.
		\item Alineación.
	\end{enumerate}
	Variaciones en estos pasos permiten la implementación de diversos métodos.

		%pseudocódigo
		%ref{FPFH}
	%FIXME: queda muy colgado
	Como medida de reducción de ruido y para evitar considerar puntos outliers
	durante la alineación, se realizó un preproceso de las nubes de entrada.
	%ver que suavice y que se llame así
	Primeramente se ajustaron los puntos a una superficie estimada mediante el
	método de mínimos cuadrados móviles y luego se descartaron puntos de poca
	confianza, como ser aquellos correspondientes a los bordes o aquellos cuyas
	normales resulten ortogonales a la cámara.

	\section{Diagrama de clases}
		En la figura~\ref{fig:align_class} se presentan las clases principales y sus interacciones.
		A continuación se presenta una breve descripción de las mismas.
		\begin{figure}
			%\Imagen{uml/align.png}
			%\centering
			%\def\svgwidth{\linewidth}
			%\chapter{Módulo de registración}
	%¿Qué es la registración?
	Dado un conjunto de nubes de puntos correspondientes a distintas vistas de
	un objeto, la registración consiste en calcular las transformaciones de
	rotación y translación que lleven a cada vista a un sistema global de forma
	que las zonas comunes encajen perfectamente.

	Se tomará la primera captura como marco de referencia, cada nueva vista se
	alineará con la anterior hasta completar una vuelta sobre el objeto.
	El problema, entonces, se resuelven encontrando para cada punto de la nube
	A su posición, si es que es visible, en la nube B, para luego estimar la
	transformación que alinee todos estos puntos.

	Las coordenadas espaciales de un punto no nos suministran suficiente
	información para poder identificarlo en otra vista, es necesario utilizar
	sus relaciones con otros puntos en una vecindad.
	Así, podrá describirse al punto mediante las posiciones relativas, la
	densidad o la orientación, para luego corresponderlo con el más parecido en
	la otra vista.
	Sin embargo, los puntos en zonas homogéneas de la nube serán descriptos de
	forma muy similar, con lo que se corre el riesgo de realizar
	correspondencias equívocas que producirán una alineación errónea.

	Para solventar este último problema, se puede tomar un subconjunto de los
	puntos de entrada, aquellos puntos que por sus características de vecindad
	sea más probable que tengan un descriptor único (\emph{keypoints}).

	Entonces, los métodos de registración seguirán los siguientes pasos:
	\begin{enumerate}
		\item Selección de puntos de la entrada (\emph{keypoints}).
		\item Cálculo de descriptores y determinación de correspondencias.
		\item Rechazo de correspondencias.
		\item Alineación.
	\end{enumerate}
	Variaciones en estos pasos permiten la implementación de diversos métodos.

		%pseudocódigo
		%ref{FPFH}
	%FIXME: queda muy colgado
	Como medida de reducción de ruido y para evitar considerar puntos outliers
	durante la alineación, se realizó un preproceso de las nubes de entrada.
	%ver que suavice y que se llame así
	Primeramente se ajustaron los puntos a una superficie estimada mediante el
	método de mínimos cuadrados móviles y luego se descartaron puntos de poca
	confianza, como ser aquellos correspondientes a los bordes o aquellos cuyas
	normales resulten ortogonales a la cámara.

	\section{Diagrama de clases}
		En la figura~\ref{fig:align_class} se presentan las clases principales y sus interacciones.
		A continuación se presenta una breve descripción de las mismas.
		\begin{figure}
			%\Imagen{uml/align.png}
			%\centering
			%\def\svgwidth{\linewidth}
			%\chapter{Módulo de registración}
	%¿Qué es la registración?
	Dado un conjunto de nubes de puntos correspondientes a distintas vistas de
	un objeto, la registración consiste en calcular las transformaciones de
	rotación y translación que lleven a cada vista a un sistema global de forma
	que las zonas comunes encajen perfectamente.

	Se tomará la primera captura como marco de referencia, cada nueva vista se
	alineará con la anterior hasta completar una vuelta sobre el objeto.
	El problema, entonces, se resuelven encontrando para cada punto de la nube
	A su posición, si es que es visible, en la nube B, para luego estimar la
	transformación que alinee todos estos puntos.

	Las coordenadas espaciales de un punto no nos suministran suficiente
	información para poder identificarlo en otra vista, es necesario utilizar
	sus relaciones con otros puntos en una vecindad.
	Así, podrá describirse al punto mediante las posiciones relativas, la
	densidad o la orientación, para luego corresponderlo con el más parecido en
	la otra vista.
	Sin embargo, los puntos en zonas homogéneas de la nube serán descriptos de
	forma muy similar, con lo que se corre el riesgo de realizar
	correspondencias equívocas que producirán una alineación errónea.

	Para solventar este último problema, se puede tomar un subconjunto de los
	puntos de entrada, aquellos puntos que por sus características de vecindad
	sea más probable que tengan un descriptor único (\emph{keypoints}).

	Entonces, los métodos de registración seguirán los siguientes pasos:
	\begin{enumerate}
		\item Selección de puntos de la entrada (\emph{keypoints}).
		\item Cálculo de descriptores y determinación de correspondencias.
		\item Rechazo de correspondencias.
		\item Alineación.
	\end{enumerate}
	Variaciones en estos pasos permiten la implementación de diversos métodos.

		%pseudocódigo
		%ref{FPFH}
	%FIXME: queda muy colgado
	Como medida de reducción de ruido y para evitar considerar puntos outliers
	durante la alineación, se realizó un preproceso de las nubes de entrada.
	%ver que suavice y que se llame así
	Primeramente se ajustaron los puntos a una superficie estimada mediante el
	método de mínimos cuadrados móviles y luego se descartaron puntos de poca
	confianza, como ser aquellos correspondientes a los bordes o aquellos cuyas
	normales resulten ortogonales a la cámara.

	\section{Diagrama de clases}
		En la figura~\ref{fig:align_class} se presentan las clases principales y sus interacciones.
		A continuación se presenta una breve descripción de las mismas.
		\begin{figure}
			%\Imagen{uml/align.png}
			%\centering
			%\def\svgwidth{\linewidth}
			%\input{uml/align.pdf_tex}
			\Imagen{uml/align.pdf}
			\caption{\label{fig:align_class}Diagrama de clases del módulo de registración}
		\end{figure}

		\begin{itemize}
			\item {\bfseries Registración:} se encarga de obtener la \emph{Transformación} que
				permita alinear dos \emph{Nubes} entre sí.  Para esto establece
				correspondencias entre los \emph{Anclaje}.
			\item {\bfseries Anclaje:} a partir de puntos salientes de \emph{Nube} calcula
				\emph{descriptores}  que permitan asociarlos y un
				\emph{marco\_de\_referencia} para obtener una estimación de la
				\emph{Transformación}.
			\item {\bfseries Nube:} representa una vista del objeto que se desea alinear.
				Es una colección de \emph{Puntos} sin organización. Clase
				provista por PCL.
			\item {\bfseries Punto:} contiene las coordenadas $xyz$ obtenidas por el
				dispositivo de captura. El algoritmo estimará las normales.
			\item {\bfseries Transformación:} representa una transformación rígida
				(rotación y translación) que será aplicada a una \emph{Nube}
				para alinearla.
		\end{itemize}

	\section{Método A}
		\subsection{Selección de keypoints}
			Basándonos en los resultados obtenidos por %ref ISS
			se consideró utilizar el algoritmo de detección de keypoints basado en \emph{Intrinsic Shape Signatures} (algoritmo~\ref{alg:iss}),
			el cual se haya implementado en PCL en la clase \texttt{ISSKeypoint3D}, permitiendo
			definir el radio de la esfera y el nivel de disimilitud.

			%ver bien el problema
			Sin embargo, no pudieron encontrarse los parámetros adecuados.
			Los keypoints resultaban en toda la superficie y al calcular luego
			los descriptores se generaban demasiadas correspondencias erróneas
			produciendo una alineación incorrecta.
			%FIXME: gráficos

			\begin{algorithm}
				\begin{algorithmic}[1]
					\Function{ISS Keypoints}{nube}
						\State keypoints $\gets\emptyset$
						\ForAll{$p \in \mbox{nube.puntos}$}
							\State vecinos $\gets$ obtener puntos cercanos(nube, p, $r_1$)
							\State m $\gets$ matriz de covarianza(vecinos)
							\State $\lambda$ = eigenvalores(m)
							\If{$\lambda_1/\lambda_2 > \mbox{umbral}_1$ and $\lambda_2/\lambda_3 > \mbox{umbral}_2$}
								\State keypoints.insert(p)
							\EndIf
						\EndFor
						\State\Return Non-Max Suppression(keypoints, $r_2$)
					\EndFunction
				\end{algorithmic}
				\caption{\label{alg:iss}Determinación de los keypoints mediante ISS}
			\end{algorithm}


			%A partir de acá ref Rusu FPFH
			Se procedió entonces a cambiar el método de selección de keypoints, eligiendo ahora un análisis de persistencia multiescala.
			\begin{enumerate}
				\item Por cada punto de la nube se calcula su descriptor para distintos tamaños de vecindad (escala).
				\item A partir de todos los descriptores en todas las escalas se estima una distribución gaussiana que los aproxime.
				\item Los keypoints quedan definidos como aquellos puntos cuyos descriptores se encuentran alejados de la media.
			\end{enumerate}
			El algoritmo se encuentra implementado en PCL en la clase
			\texttt{Multiscale\-Feature\-Persistence} permitiendo ajustar las
			escalas a utilizar, el umbral para ser considerado saliente y la
			función descriptora a utilizar.

			Debido a que es necesario calcular el descriptor para cada punto, y además en diferentes escalas, se eligió como descriptor 
			\emph{Fast Point Feature Histograms} (FPFH), el cual es lineal en la cantidad de puntos de la vecindad.
			Este descriptor calcula un histograma de los ángulos entre las normales del punto y sus vecinos. %ref FPFH

			Los keypoints se encontraban ahora agrupados, formando líneas en zonas de cambio brusco de curvatura.
			%FIXME: gráficos




		\subsection{Alineación}
			Se utilizó el algoritmo de \emph{sample consensus initial alignment (SAC-IA)} para la alineación, el cual consiste en:
			\begin{enumerate}
				\item Seleccionar al azar \emph{m} puntos de la nube A
				\item Por cada punto, buscar aquellos con descriptores similares en B. Seleccionar uno al azar.
				\item Calcular la transformación definida por estos puntos y
					sus correspondencias. Calcular, además, una medida del
					error de transformación.
				\item Repetir varias veces y devolver aquella transformación que produjo el menor error.
			\end{enumerate}

			\subsection{Resultados}
			Mediante este método, se obtuvieron buenos resultados en la mayoría de las
			capturas de \texttt{happy}
			donde los ángulos eran cercanos a $25^{\circ}$.
			Sin embargo, algunas alineaciones presentaban problemas de deslizamiento.

			En el caso de \texttt{bunny}, cuyos ángulos eran cercanos a $45^{\circ}$,
			se obtuvo una buena alineación en el caso de las capturas 315--000 y 000--045,
			sin embargo, para los otros casos las correspondencias fueron completamente erróneas.

			%gráficos


	%El que tengo ahora
	\section{Método B}
		En este caso no se seleccionaron keypoints, simplemente se realizó un submuestreo de
		los puntos de la nube para reducir el costo computacional.
		Además del descriptor, en cada punto se estableció un marco de referencia utilizando los
		eigenvectores de la matriz de covarianza de la vecindad del punto. Este
		marco de referencia nos permite establecer
		la transformación de alineación considerando solamente dos puntos. %ref ISS

		Para establecer las correspondencias se utilizó el descriptor FPFH
		comparando los histogramas mediante la distancia $\chi^2$.  Luego se
		procedió a eliminar correspondencias erróneas utilizando los marcos de
		referencia y las suposiciones de ubicación de la cámara en la obtención
		de las capturas.  Por esto, se descartan aquellas correspondencias que
		requieren un movimiento en $y$ excesivo o una rotación sobre un eje no
		vertical. 

		Cada correspondencia entonces define un ángulo de giro $\theta$ sobre
		el eje $y$ y una translación en el plano $xz$.  Se observará entonces
		una agrupación de los parámetros de estas transformaciones, mediante el
		algoritmo de k-means se buscará el centroide del clúster más grande.
		%gráfico

		\subsection{Resultados}
			Se realizaron pruebas sobre los objetos \texttt{happy} y \texttt{bunny}.

			En el caso de \texttt{bunny} se ignoró la captura
			\texttt{bun180.ply} ya que presentaba un salto de aproximadamente
			$90^{\circ}$ respecto a su anterior y siguiente.  Las otras
			capturas presentaban un desvío no mayor a $2^{\circ}$ respecto al
			\emph{ground truth}.

			Para \texttt{happy} se tuvo un desvío medio de poco más de
			$2^{\circ}$, con un máximo de $6^{\circ}$ respecto al \emph{ground truth}.

			De esta forma, se logró acercar las capturas lo suficiente como
			para intentar alinearlas por ICP.


	\section{Refinamiento}
	Una vez obtenida la alineación inicial, se procedió a realizar una segunda
	alineación mediante ICP.
	Se consideraron sólo las áreas solapadas y se restringió el
	espacio de búsqueda de las correspondencias.

	Luego, para reducir el error propagado por cada alineación, se propuso una
	corrección de bucle.
	Se ajustó la última captura para que correspondiese con la primera, y
	esta transformación se agregó a las otras alineaciones con un peso
	proporcional a su posición en el bucle.


	%informes de prueba
	%gráficos

	%casos de uso/funcionalidades resueltas

			\Imagen{uml/align.pdf}
			\caption{\label{fig:align_class}Diagrama de clases del módulo de registración}
		\end{figure}

		\begin{itemize}
			\item {\bfseries Registración:} se encarga de obtener la \emph{Transformación} que
				permita alinear dos \emph{Nubes} entre sí.  Para esto establece
				correspondencias entre los \emph{Anclaje}.
			\item {\bfseries Anclaje:} a partir de puntos salientes de \emph{Nube} calcula
				\emph{descriptores}  que permitan asociarlos y un
				\emph{marco\_de\_referencia} para obtener una estimación de la
				\emph{Transformación}.
			\item {\bfseries Nube:} representa una vista del objeto que se desea alinear.
				Es una colección de \emph{Puntos} sin organización. Clase
				provista por PCL.
			\item {\bfseries Punto:} contiene las coordenadas $xyz$ obtenidas por el
				dispositivo de captura. El algoritmo estimará las normales.
			\item {\bfseries Transformación:} representa una transformación rígida
				(rotación y translación) que será aplicada a una \emph{Nube}
				para alinearla.
		\end{itemize}

	\section{Método A}
		\subsection{Selección de keypoints}
			Basándonos en los resultados obtenidos por %ref ISS
			se consideró utilizar el algoritmo de detección de keypoints basado en \emph{Intrinsic Shape Signatures} (algoritmo~\ref{alg:iss}),
			el cual se haya implementado en PCL en la clase \texttt{ISSKeypoint3D}, permitiendo
			definir el radio de la esfera y el nivel de disimilitud.

			%ver bien el problema
			Sin embargo, no pudieron encontrarse los parámetros adecuados.
			Los keypoints resultaban en toda la superficie y al calcular luego
			los descriptores se generaban demasiadas correspondencias erróneas
			produciendo una alineación incorrecta.
			%FIXME: gráficos

			\begin{algorithm}
				\begin{algorithmic}[1]
					\Function{ISS Keypoints}{nube}
						\State keypoints $\gets\emptyset$
						\ForAll{$p \in \mbox{nube.puntos}$}
							\State vecinos $\gets$ obtener puntos cercanos(nube, p, $r_1$)
							\State m $\gets$ matriz de covarianza(vecinos)
							\State $\lambda$ = eigenvalores(m)
							\If{$\lambda_1/\lambda_2 > \mbox{umbral}_1$ and $\lambda_2/\lambda_3 > \mbox{umbral}_2$}
								\State keypoints.insert(p)
							\EndIf
						\EndFor
						\State\Return Non-Max Suppression(keypoints, $r_2$)
					\EndFunction
				\end{algorithmic}
				\caption{\label{alg:iss}Determinación de los keypoints mediante ISS}
			\end{algorithm}


			%A partir de acá ref Rusu FPFH
			Se procedió entonces a cambiar el método de selección de keypoints, eligiendo ahora un análisis de persistencia multiescala.
			\begin{enumerate}
				\item Por cada punto de la nube se calcula su descriptor para distintos tamaños de vecindad (escala).
				\item A partir de todos los descriptores en todas las escalas se estima una distribución gaussiana que los aproxime.
				\item Los keypoints quedan definidos como aquellos puntos cuyos descriptores se encuentran alejados de la media.
			\end{enumerate}
			El algoritmo se encuentra implementado en PCL en la clase
			\texttt{Multiscale\-Feature\-Persistence} permitiendo ajustar las
			escalas a utilizar, el umbral para ser considerado saliente y la
			función descriptora a utilizar.

			Debido a que es necesario calcular el descriptor para cada punto, y además en diferentes escalas, se eligió como descriptor 
			\emph{Fast Point Feature Histograms} (FPFH), el cual es lineal en la cantidad de puntos de la vecindad.
			Este descriptor calcula un histograma de los ángulos entre las normales del punto y sus vecinos. %ref FPFH

			Los keypoints se encontraban ahora agrupados, formando líneas en zonas de cambio brusco de curvatura.
			%FIXME: gráficos




		\subsection{Alineación}
			Se utilizó el algoritmo de \emph{sample consensus initial alignment (SAC-IA)} para la alineación, el cual consiste en:
			\begin{enumerate}
				\item Seleccionar al azar \emph{m} puntos de la nube A
				\item Por cada punto, buscar aquellos con descriptores similares en B. Seleccionar uno al azar.
				\item Calcular la transformación definida por estos puntos y
					sus correspondencias. Calcular, además, una medida del
					error de transformación.
				\item Repetir varias veces y devolver aquella transformación que produjo el menor error.
			\end{enumerate}

			\subsection{Resultados}
			Mediante este método, se obtuvieron buenos resultados en la mayoría de las
			capturas de \texttt{happy}
			donde los ángulos eran cercanos a $25^{\circ}$.
			Sin embargo, algunas alineaciones presentaban problemas de deslizamiento.

			En el caso de \texttt{bunny}, cuyos ángulos eran cercanos a $45^{\circ}$,
			se obtuvo una buena alineación en el caso de las capturas 315--000 y 000--045,
			sin embargo, para los otros casos las correspondencias fueron completamente erróneas.

			%gráficos


	%El que tengo ahora
	\section{Método B}
		En este caso no se seleccionaron keypoints, simplemente se realizó un submuestreo de
		los puntos de la nube para reducir el costo computacional.
		Además del descriptor, en cada punto se estableció un marco de referencia utilizando los
		eigenvectores de la matriz de covarianza de la vecindad del punto. Este
		marco de referencia nos permite establecer
		la transformación de alineación considerando solamente dos puntos. %ref ISS

		Para establecer las correspondencias se utilizó el descriptor FPFH
		comparando los histogramas mediante la distancia $\chi^2$.  Luego se
		procedió a eliminar correspondencias erróneas utilizando los marcos de
		referencia y las suposiciones de ubicación de la cámara en la obtención
		de las capturas.  Por esto, se descartan aquellas correspondencias que
		requieren un movimiento en $y$ excesivo o una rotación sobre un eje no
		vertical. 

		Cada correspondencia entonces define un ángulo de giro $\theta$ sobre
		el eje $y$ y una translación en el plano $xz$.  Se observará entonces
		una agrupación de los parámetros de estas transformaciones, mediante el
		algoritmo de k-means se buscará el centroide del clúster más grande.
		%gráfico

		\subsection{Resultados}
			Se realizaron pruebas sobre los objetos \texttt{happy} y \texttt{bunny}.

			En el caso de \texttt{bunny} se ignoró la captura
			\texttt{bun180.ply} ya que presentaba un salto de aproximadamente
			$90^{\circ}$ respecto a su anterior y siguiente.  Las otras
			capturas presentaban un desvío no mayor a $2^{\circ}$ respecto al
			\emph{ground truth}.

			Para \texttt{happy} se tuvo un desvío medio de poco más de
			$2^{\circ}$, con un máximo de $6^{\circ}$ respecto al \emph{ground truth}.

			De esta forma, se logró acercar las capturas lo suficiente como
			para intentar alinearlas por ICP.


	\section{Refinamiento}
	Una vez obtenida la alineación inicial, se procedió a realizar una segunda
	alineación mediante ICP.
	Se consideraron sólo las áreas solapadas y se restringió el
	espacio de búsqueda de las correspondencias.

	Luego, para reducir el error propagado por cada alineación, se propuso una
	corrección de bucle.
	Se ajustó la última captura para que correspondiese con la primera, y
	esta transformación se agregó a las otras alineaciones con un peso
	proporcional a su posición en el bucle.


	%informes de prueba
	%gráficos

	%casos de uso/funcionalidades resueltas

			\Imagen{uml/align.pdf}
			\caption{\label{fig:align_class}Diagrama de clases del módulo de registración}
		\end{figure}

		\begin{itemize}
			\item {\bfseries Registración:} se encarga de obtener la \emph{Transformación} que
				permita alinear dos \emph{Nubes} entre sí.  Para esto establece
				correspondencias entre los \emph{Anclaje}.
			\item {\bfseries Anclaje:} a partir de puntos salientes de \emph{Nube} calcula
				\emph{descriptores}  que permitan asociarlos y un
				\emph{marco\_de\_referencia} para obtener una estimación de la
				\emph{Transformación}.
			\item {\bfseries Nube:} representa una vista del objeto que se desea alinear.
				Es una colección de \emph{Puntos} sin organización. Clase
				provista por PCL.
			\item {\bfseries Punto:} contiene las coordenadas $xyz$ obtenidas por el
				dispositivo de captura. El algoritmo estimará las normales.
			\item {\bfseries Transformación:} representa una transformación rígida
				(rotación y translación) que será aplicada a una \emph{Nube}
				para alinearla.
		\end{itemize}

	\section{Método A}
		\subsection{Selección de keypoints}
			Basándonos en los resultados obtenidos por %ref ISS
			se consideró utilizar el algoritmo de detección de keypoints basado en \emph{Intrinsic Shape Signatures} (algoritmo~\ref{alg:iss}),
			el cual se haya implementado en PCL en la clase \texttt{ISSKeypoint3D}, permitiendo
			definir el radio de la esfera y el nivel de disimilitud.

			%ver bien el problema
			Sin embargo, no pudieron encontrarse los parámetros adecuados.
			Los keypoints resultaban en toda la superficie y al calcular luego
			los descriptores se generaban demasiadas correspondencias erróneas
			produciendo una alineación incorrecta.
			%FIXME: gráficos

			\begin{algorithm}
				\begin{algorithmic}[1]
					\Function{ISS Keypoints}{nube}
						\State keypoints $\gets\emptyset$
						\ForAll{$p \in \mbox{nube.puntos}$}
							\State vecinos $\gets$ obtener puntos cercanos(nube, p, $r_1$)
							\State m $\gets$ matriz de covarianza(vecinos)
							\State $\lambda$ = eigenvalores(m)
							\If{$\lambda_1/\lambda_2 > \mbox{umbral}_1$ and $\lambda_2/\lambda_3 > \mbox{umbral}_2$}
								\State keypoints.insert(p)
							\EndIf
						\EndFor
						\State\Return Non-Max Suppression(keypoints, $r_2$)
					\EndFunction
				\end{algorithmic}
				\caption{\label{alg:iss}Determinación de los keypoints mediante ISS}
			\end{algorithm}


			%A partir de acá ref Rusu FPFH
			Se procedió entonces a cambiar el método de selección de keypoints, eligiendo ahora un análisis de persistencia multiescala.
			\begin{enumerate}
				\item Por cada punto de la nube se calcula su descriptor para distintos tamaños de vecindad (escala).
				\item A partir de todos los descriptores en todas las escalas se estima una distribución gaussiana que los aproxime.
				\item Los keypoints quedan definidos como aquellos puntos cuyos descriptores se encuentran alejados de la media.
			\end{enumerate}
			El algoritmo se encuentra implementado en PCL en la clase
			\texttt{Multiscale\-Feature\-Persistence} permitiendo ajustar las
			escalas a utilizar, el umbral para ser considerado saliente y la
			función descriptora a utilizar.

			Debido a que es necesario calcular el descriptor para cada punto, y además en diferentes escalas, se eligió como descriptor 
			\emph{Fast Point Feature Histograms} (FPFH), el cual es lineal en la cantidad de puntos de la vecindad.
			Este descriptor calcula un histograma de los ángulos entre las normales del punto y sus vecinos. %ref FPFH

			Los keypoints se encontraban ahora agrupados, formando líneas en zonas de cambio brusco de curvatura.
			%FIXME: gráficos




		\subsection{Alineación}
			Se utilizó el algoritmo de \emph{sample consensus initial alignment (SAC-IA)} para la alineación, el cual consiste en:
			\begin{enumerate}
				\item Seleccionar al azar \emph{m} puntos de la nube A
				\item Por cada punto, buscar aquellos con descriptores similares en B. Seleccionar uno al azar.
				\item Calcular la transformación definida por estos puntos y
					sus correspondencias. Calcular, además, una medida del
					error de transformación.
				\item Repetir varias veces y devolver aquella transformación que produjo el menor error.
			\end{enumerate}

			\subsection{Resultados}
			Mediante este método, se obtuvieron buenos resultados en la mayoría de las
			capturas de \texttt{happy}
			donde los ángulos eran cercanos a $25^{\circ}$.
			Sin embargo, algunas alineaciones presentaban problemas de deslizamiento.

			En el caso de \texttt{bunny}, cuyos ángulos eran cercanos a $45^{\circ}$,
			se obtuvo una buena alineación en el caso de las capturas 315--000 y 000--045,
			sin embargo, para los otros casos las correspondencias fueron completamente erróneas.

			%gráficos


	%El que tengo ahora
	\section{Método B}
		En este caso no se seleccionaron keypoints, simplemente se realizó un submuestreo de
		los puntos de la nube para reducir el costo computacional.
		Además del descriptor, en cada punto se estableció un marco de referencia utilizando los
		eigenvectores de la matriz de covarianza de la vecindad del punto. Este
		marco de referencia nos permite establecer
		la transformación de alineación considerando solamente dos puntos. %ref ISS

		Para establecer las correspondencias se utilizó el descriptor FPFH
		comparando los histogramas mediante la distancia $\chi^2$.  Luego se
		procedió a eliminar correspondencias erróneas utilizando los marcos de
		referencia y las suposiciones de ubicación de la cámara en la obtención
		de las capturas.  Por esto, se descartan aquellas correspondencias que
		requieren un movimiento en $y$ excesivo o una rotación sobre un eje no
		vertical. 

		Cada correspondencia entonces define un ángulo de giro $\theta$ sobre
		el eje $y$ y una translación en el plano $xz$.  Se observará entonces
		una agrupación de los parámetros de estas transformaciones, mediante el
		algoritmo de k-means se buscará el centroide del clúster más grande.
		%gráfico

		\subsection{Resultados}
			Se realizaron pruebas sobre los objetos \texttt{happy} y \texttt{bunny}.

			En el caso de \texttt{bunny} se ignoró la captura
			\texttt{bun180.ply} ya que presentaba un salto de aproximadamente
			$90^{\circ}$ respecto a su anterior y siguiente.  Las otras
			capturas presentaban un desvío no mayor a $2^{\circ}$ respecto al
			\emph{ground truth}.

			Para \texttt{happy} se tuvo un desvío medio de poco más de
			$2^{\circ}$, con un máximo de $6^{\circ}$ respecto al \emph{ground truth}.

			De esta forma, se logró acercar las capturas lo suficiente como
			para intentar alinearlas por ICP.


	\section{Refinamiento}
	Una vez obtenida la alineación inicial, se procedió a realizar una segunda
	alineación mediante ICP.
	Se consideraron sólo las áreas solapadas y se restringió el
	espacio de búsqueda de las correspondencias.

	Luego, para reducir el error propagado por cada alineación, se propuso una
	corrección de bucle.
	Se ajustó la última captura para que correspondiese con la primera, y
	esta transformación se agregó a las otras alineaciones con un peso
	proporcional a su posición en el bucle.


	%informes de prueba
	%gráficos

	%casos de uso/funcionalidades resueltas

	\section{Módulo de fusión}
	%¿citas?
	%en el de loop correction (surfel)
	El módulo de fusión tiene como objetivo obtener un modelo que
	describa la geometría del objeto escaneado.
	Para esto, el modelo se inicializa a partir de una captura cualquiera
	y se procesan una a una las capturas correspondientes a las otras vistas.
	En cada nueva vista se agrega información del objeto en zonas que no eran antes visibles
	y además se confirma o refuta la información ya presente en el modelo.
	Al combinar esta información, el módulo de fusión obtiene finalmente una malla
	que represente a la totalidad del objeto.

	\subsection{Método}
	Se utiliza una representación de \emph{surfel} para cada punto similar a la propuesta en \cite{5457479} %ref in-hand scanning
	debido a la facilidad de implementación de las funciones de actualización de los puntos.
	Como es necesario integrar la información de posición y normal de cada punto,
	se idearon funciones de conversión del formato de nube usado en el módulo de registración.

	Cada surfel tiene asociado un valor de confianza y las vistas en las que
	fue observado.  El valor de confianza nos indica la probabilidad de que sus
	valores de posición y normal no sean producto de un error de muestreo.
	Este valor se inicializa según el ángulo de su normal respecto a la línea
	de la cámara (figura~\ref{fig:confianza_surfel}).
	%y a su distancia al centro de la captura %terminé usando sólo las normales

	\begin{figure}
		\Imagen{img/bunny_confianza}
		\caption[Visualización de los valores de confianza de los súrfeles]{\label{fig:confianza_surfel}Visualización de los valores de confianza de los súrfeles para la captura \texttt{bun000}.}
	\end{figure}

	El algoritmo~\ref{alg:surfel} describe el agregado de una nueva vista.
	Por cada punto de la vista se busca qué surfel lo contiene y se actualiza
	su posición y normal ponderando según el nivel de confianza.
	\begin{eqnarray*}
		\hat{P}_k = \frac{\sum_{j} \alpha_j P^{(j)}_k}{\sum_{j} \alpha_j} \\
		\hat{n}_k = \frac{\sum_{j} \alpha_j n^{(j)}_k}{\sum_{j} \alpha_j}
	\end{eqnarray*}
	En caso de que haya caído fuera del dominio de la reconstrucción actual, se
	considera que es un nuevo punto y se lo agrega.
	Si la distancia de proximidad elegida es demasiado pequeña,
	nunca se realizará la actualización ya que todos los puntos serán considerados como nuevos surfels.
	En cambio, si la distancia es muy grande, se considerarán puntos
	que deberían haber sido descartados como atípicos.

	\begin{algorithm}
		\begin{algorithmic}[1]
			\Function{Agregar vista}{vista, reconstrucción}
				\ForAll{$p \in \mbox{vista.puntos}$}
					\State surfel $\gets$ reconstrucción.buscar(p)
					\If {surfel = $\emptyset$}
						\State reconstrucción.agregar(p)
					\Else
						\State surfel.actualizar(p)
					\EndIf
				\EndFor
			\EndFunction
		\end{algorithmic}
		\caption[Actualización de la reconstrucción al agregar una nueva vista]{\label{alg:surfel}Actualización de la reconstrucción al agregar una nueva vista.}
	\end{algorithm}

	Al terminar de procesar todas las vistas, habrá surfels que hayan sido
	observados desde sólo una posición y tengan un nivel de confianza bajo.
	Estos son considerados como ruido y eliminados de la
	reconstrucción.

	Finalmente, se procede a la obtención de una triangulación de los surfels.
	Para esto se empleó el algoritmo de \emph{Greedy Projection Triangulation} que provee PCL,
	el cual realiza una proyección de la vecindad de un punto en el plano definido por su normal
	y procede a conectar los puntos de modo que los triángulos resultantes
	respeten umbrales de longitud y ángulos definidos por el usuario (figura~\ref{fig:surface}).

	\begin{figure}
		\Imagen{img/bun_fusion}

		%\Imagen{img/bun_fusion_conf}
		\caption[Superficie reconstruida]{\label{fig:surface}Superficie reconstruida. En verde se destacan los bordes de los huecos.}
	\end{figure}

	\chapter{Módulo de rellenado de huecos}
	%Poisson
	La clase pcl::Poisson provee algoritmos de reconstrucción basados en %cite Poisson surface reconstruction
	siendo el principal parámetro la profundidad del octree utilizado,
	impactando directamente en la resolución de la malla resultante.


	Se transforma el problema en una ecuación de Poisson
	\[\Delta\chi \equiv \nabla \cdot\nabla\chi = \nabla \vec{n}\]

	Se realiza una discretización del dominio mediante un octree, ya que sólo
	interesa la solución de la función en la proximidad de la superficie a
	reconstruir.
	Luego se definen funciones de soporte local que aproximen una gaussiana.

	Una vez resuelto el problema en el dominio, se extrae la isosuperficie mediante una variante de marching cubes.

	Como resultado se tendrá una malla triangular que no presenta huecos.


	%mi método
	%ver paper
	advancing front
	Definir isla
	Eliminar islas

	Dado un boundary a rellenar
		Buscar los segmentos que generen el menor ángulo
		caso \alpha < 75
			unir los extremos
		\alpha < 135
			agregar un punto en la bisectriz
		\alpha < pi
			agregar un punto en un tercio del ángulo
		\alpha >= pi
			no debería ocurrir

		Verificar que el nuevo punto no caiga muy cerca de otro ya existente
		En ese caso usar el existente
			Dividir el boundary en dos, procesar cada sección de forma independiente.

		De esta forma se tiene un frente que avanza desde el contorno hacia adentro, terminando por encontrarse con un frente opuesto y cerrando el hueco.


	Ubicación del nuevo punto
		Una vez determinado los tres puntos que definen el menor ángulo: 
			\theta = \widehat{PCN}

		se define el plano \alpha mediante los tres puntos.
		se crea un nuevo punto Q sobre \alpha, ubicado en la bisectriz del ángulo, a una distancia step de C
		se define el plano \beta como aquel cuya normal es el promedio de las normales de los tres puntos y pasa por el promedio de los tres puntos
		\[
		O = \frac{P+C+N}{3}
		n = \frac{P_n + C_n + N_n} {|P_n + C_n + N_n|}
		\]

		se proyecta Q en \beta

		`step' es constante en todo el proceso, siendo el promedio de las longitudes de los segmentos que forman el contorno del hueco


		Con este método se pueden rellenar agujeros pequeños, obteniéndose una malla bastante regular.
		Sin embargo, debido a la localidad con la que se generan nuevos puntos, el frente puede diverger o pretender unirse a puntos que no forman parte del contorno del hueco, perdiendo la propiedad de manifold

	%\chapter{Conclusiones y trabajos futuros}
\section{Del producto}
	En este proyecto se realizó el desarrollo de una biblioteca de software para
	realizar la reconstrucción tridimensional de un objeto a partir de capturas de
	vistas parciales.
	Para esto, se dividió el problema en tres módulos: registración, fusión y
	rellenado de huecos, y se implementaron diversos algoritmos.

	Al trabajar directamente con las nubes de puntos no se restringió el
	dispositivo de captura a un hardware en particular. Sin embargo, las
	restricciones impuestas de base giratoria y limitar la cantidad de capturas
	requeridas fueron planteadas considerando una integración futura con
	el trabajo realizado por \TODO{cite{Pancho}}.

	Las registraciones se obtuvieron en tiempos razonables, sin requerir hardware especial.
	Si bien se observa un efecto de «inflado» debido a la propagación de los errores de registración,
	este se encuentra suficientemente acotado respecto al tamaño del objeto.

	Al realizar las capturas solamente sobre una base giratoria,
	Debido a que las capturas se realizaron solamente sobre una base giratoria,
	no se lograron resolver todas las oclusiones, lo que genera la aparición de huecos de
	tamaño considerable.
	Aún así, el resultado es una malla cerrada, y la
	superficie estimada en las zonas sin información se une suavemente al resto de
	la malla.

\section{Del proceso}
La elección de la metodología en cascada modificada fue incorrecta.
La investigación bibliográfica se alargó demasiado y se desperdiciaron recursos
al abordar temas que tuvieron que ser descartados luego (como el uso de la
información de textura).
Además, se produjo un desfasaje entre la adquisición de los conocimientos y la
implementación de los mismos.

Hubiese sido mejor utilizar directamente una metodología incremental en todo el
proceso, con más incrementos de menor tamaño, como ser agregar un primer módulo
de preproceso que contenga la reducción de ruido y la operatoria básica con las
nubes de puntos.

En cuanto al desarrollo, uno de los principales problemas fue la definición de
métricas para evaluar los algoritmos y establecer qué niveles de error eran
aceptables y cuáles requerían una corrección.
Esto se dificulta, además, al considerar que los resultados producidos en una etapa
serán la entrada de otra, de la cual se desconoce su sensibilidad.
Muchas evaluaciones fueron primeramente visuales, resultando en un proceso lento
que en ocasiones fallaba en detectar errores considerables.



%\subsection{Riesgos efectivizados}
%Ausencia de repositorio de mallas tridimensionales
%(copiar base de datos)
%
%Falla en los equipos de trabajo
%(copiar parte de pcl)
%Meshlab: no se logró instalarlo en el nuevo equipo, se cambió a CloudCompare



\section{Trabajos futuros}
En esta sección se describirán actividades que excedieron el alcance de este proyecto
y podrían ser abordadas en una etapa posterior.

\begin{itemize}
	\item Combinar escaneos cilíndricos del objeto en varias posiciones sobre la base giratoria,
		buscando de esta forma eliminar huecos y reducir la propagación del error de alineación.
		Esto requerirá eliminar la restricción del eje de giro en la
		registración, por lo que deberán evaluarse otros métodos para detectar
		correspondencias erróneas.
	\item Ajustar los métodos para trabajar con el volumen de puntos generados
		por \TODO{cite{Pancho}}.  Las capturas de los modelos con los cuales se
		realizaron las pruebas contenían a lo sumo ochenta mil puntos. Las
		reconstrucciones obtenidas en \TODO{cite{Pancho}} pueden llegar a los
		dos millones, es decir, 25 veces más.
		Es necesario entonces, analizar la escalabilidad de los métodos
		propuestos y optimizarlos o reemplazarlos según resulte conveniente.
		En particular, puede plantearse la ejecución sobre GPU.
	\item Implementar métricas de calidad del mallado que consideren las
		características de la impresión 3D.
		En este proyecto solamente se consideraron las condiciones de que la malla
		resultase cerrada y no presente intersecciones consigo misma.
	%\item Mejorar la detección de \emph{outliers} en el módulo de fusión.
	%\item Implementar métodos de suavizado de mallas.
	\item Modificar el método de \emph{advancing front} para que utilice una
		superficie de soporte para establecer la posición de los nuevos puntos,
		asegurando de esta forma la convergencia del método y la suavidad del
		parche generado.
	\item Modificar el método de \emph{advancing front} para que considere las islas.
		Esto requerirá detectar dentro de qué hueco se encuentra cada isla.
\end{itemize}

	\bibliographystyle{alpha}
	\bibliography{biblio}
\end{document}
