\documentclass{pfc}
\usepackage{algorithmicx}
\usepackage{algorithm}
\usepackage[noend]{algpseudocode}

\makeatletter
 \renewcommand{\ALG@name}{Algoritmo}
\makeatother
\renewcommand{\algorithmicfunction}{}

\title{Diseño y desarrollo}

\begin{document}
	\maketitle
	El proceso de reconstrucción tridimensional generalmente se divide en tres etapas:
	\begin{enumerate}
		\item Registración: donde se determinan las transformaciones necesarias
			para llevar cada vista a su correcta posición en un marco de
			referencia global.
		\item Fusión: donde se unifica el aporte de cada vista para obtener una
			superficie que las englobe.
		\item Relleno de huecos: donde se asegura que la superficie global sea
			cerrada, es decir, que encierre un volumen.
	\end{enumerate}
	Cada una de estas etapas se corresponde con un módulo del sistema.
	A continuación se describirán los detalles de diseño e implementación de
	los métodos propuestos para cada módulo, junto con sus resultados
	preliminares.

	\chapter{Módulo de registración}
	%¿Qué es la registración?
	Dado un conjunto de nubes de puntos correspondientes a distintas vistas de
	un objeto, la registración consiste en calcular las transformaciones de
	rotación y translación que lleven a cada vista a un sistema global de forma
	que las zonas comunes encajen perfectamente.

	Los métodos de registración seguirán los siguientes pasos:
	\begin{enumerate}
		\item Selección de puntos de la entrada (\emph{keypoints}).
		\item Cálculo de descriptores y determinación de correspondencias.
		\item Rechazo de correspondencias.
		\item Alineación.
	\end{enumerate}
	Variaciones en estos pasos permiten la implementación de diversos métodos.

		%pseudocódigo
		%ref{FPFH}
	Como medida de reducción de ruido y para evitar considerar puntos outliers
	durante la alineación, se realizó un preproceso de las nubes de entrada.
	%ver que suavice y que se llame así
	Primeramente se ajustaron los puntos a una superficie estimada mediante el
	método de mínimos cuadrados móviles y luego se descartaron puntos de poca
	confianza, como ser aquellos cerca de los bordes o cuyas normales sean
	ortogonales a la cámara.

	\section{Keypoints}
		ISS:
		%descripción
		Por cada punto $p$ de la nube se calcula la matriz de covarianza con
		todos los puntos que caen dentro de una esfera de radio $r$ centrada en
		$p$.
		Un punto es detectado como keypoint si los eigenvalores de esta matriz
		de covarianza son suficientemente disímiles y además presenta la mayor disimiluted entre sus vecinos.

		La clase pcl::ISSKeypoint3D permite identificar estos keypoints, permitiendo definir el radio de la esfera y el nivel de disimilitud.
		%ver bien el problema
		Sin embargo, no pudieron encontrarse los parámetros adecuados.
		Se obtuvieron muy pocos keypoints, imposibilitando la detección de outliers,
		y además estos no resultaban representativos ya que sus correspondientes en la otra nube se encontraban demasiado alejados.

		Multiscale persistence:
		%descripción
		Por cada punto de la nube se calcula su descriptor para distintos tamaños de vecindad (escala).
		A partir de todos los descriptores en todas las escalas se estima una distribución gaussiana que los aproxime.
		Los keypoints quedan definidos como aquellos puntos cuyos descriptores se encuentran alejados de la media.

		Se utilizó Fast Point Feature Histograms (FPFH) como descriptor debido a su bajo costo computacional.

		all:
		Se realiza un submuestreo de los puntos de la nube simplemente para
		reducir el costo computacional.
		Nuevamente, y por las mismas razones, el descriptor elegido fue FPFH.
		El algoritmo de rechazo de correspondencias se encargará de eliminar outliers y asegurar la convergencia.

	\section{Alineación}
	En el caso de los keypoints de multiscale persistence se utilizó el
	algoritmo de sample consensus initial alignment (SAC-IA) para realizar la
	primera alineación.

	Este algoritmo selecciona de forma aleatoria keypoints de la nube, busca
	puntos cercanos en la otra nube, seleccionando nuevamente de forma
	aleatoria, y calcula la transformación que los alinee, obteniéndose una
	medida del error de la transformación.

	Este proceso se repite varias veces, quedándose con la transformación que
	produjo el menor error de alineación.

	Mediante este método, se obtuvieron buenos resultados en la mayoría de las
	capturas de `happy' %enf
	donde los ángulos eran cercanos a 25. %grados 
	Sin embargo, algunas alineaciones presentaban problemas de deslizamiento.
	En el caso de `bunny', cuyos ángulos eran cercanos a 45, %grados
	los resultados no fueron satisfactorios, presentándose confusión en las
	correspondencias del cuerpo e ignorando completamente las orejas.


	%El que tengo ahora
	Se utilizan todos los puntos de la nube.

	En cada punto se calcula un descriptor FPFH que se corresponderá con el más
	cercano utilizando una distancia de chi cuadrado.

	Luego se procede al rechazo de correspondencias erróneas.
	Por cada punto se calcula el marco de referencia provisto por el algoritmo de ISS. %citar
	Este marco de referencia nos permite calcular la transformación de
	alineación por cada correspondencia.
	Entonces, utilizando las suposiciones de ubicación de la cámara en la
	obtención de las capturas, se descartan aquellas correspondencias que
	requieren un movimiento en $y$ excesivo o una rotación por un eje no
	vertical. 

	Cada correspondencia entonces define un ángulo de giro $\theta$ sobre el
	eje $y$ y una translación en el plano $xz$.
	Se observará entonces una agrupación de los parámetros de estas
	transformaciones, mediante el algoritmo de k-means se buscará el centroide
	del clúster más grande.

	Las pruebas sobre `happy' y `bunny' dieron una distancia de aproximadamente
	5 %grados
	comparados al \emph{ground truth}.


	\section{Refinamiento}
	Una vez obtenida la alineación inicial, se procedió a realizar una segunda
	alineación mediante ICP.
	Se consideraron sólo las áreas solapadas y se restringió el
	espacio de búsqueda de las correspondencias.

	Luego, para reducir el error propagado por cada alineación, se propuso una
	corrección de bucle.
	Se ajustó la última captura para que correspondiese con la primera, y
	esta transformación se agregó a las otras alineaciones con un peso
	proporcional a su posición en el bucle.


	%diagramas de clase

	%informes de prueba

	%casos de uso/funcionalidades resueltas

	\chapter{Módulo de fusión}
	%¿citas?
	%en el de loop correction (surfel)
	Al agregar una vista se obtendrá nueva información del objeto en aquellos
	puntos que no eran antes visibles.
	Además, en las zonas solapadas confirmará o refutará la información ya presente.

	El módulo de fusión se encargará de combinar esta información para obtener
	finalmente una malla que represente la porción observada del objeto.

	Cada punto de una vista tendrá asociado un valor de confianza que indica la
	probabilidad de que el punto no sea un outlier.
	Este valor se inicializa según el ángulo de su normal respecto a la línea
	de la cámara y su distancia al centro de la captura.

	Cuando se agrega un nuevo punto, se busca el más cercano en la nube global
	Si la distancia supera un umbral, se considera que es un nuevo punto. Caso
	contrario, se ajustan la posición y normal del punto en la nube global
	según los niveles de confianza de ambos.  Además, se contará desde cuántas
	vistas es observado cada punto.


	Finalmente se descartarán aquellos puntos que hayan sido observados desde sólo una vista y tengan bajo nivel de confianza.

	%diagramas de clase
	\section{Diagrama de clases}
		En la figura~\ref{fig:fusion_class} se presentan las clases principales y sus interacciones.
		A continuación se presenta una breve descripción de las mismas.
		\begin{figure}
			\Imagen{uml/fusion.png}
			\caption{\label{fig:fusion_class}Diagrama de clases del módulo de fusión}
		\end{figure}

		\begin{itemize}
			\item {\bfseries Fusión:} unirá las \emph{Nubes} ya alineadas para
				obtener una superficie global que represente al objeto.
				Esto supone corregir la posición de los puntos, descartar
				aquellos considerados como ruido y triangular la superficie.
			\item {\bfseries Malla:} triangulación de la \emph{Nube} representada por un grafo de conectividades (\emph{DCEL}).
		\end{itemize}

	%informes de prueba

	%casos de uso/funcionalidades resueltas

	\chapter{Módulo de rellenado de huecos}
	Al finalizar el algoritmo de fusión se obtuvo una malla triangular a partir
	de la información proveniente de cada vista.
	Sin embargo, esta malla no es cerrada ya que existen zonas que ninguna
	vista pudo capturar y por lo tanto carecen de puntos, produciendo huecos en la misma.
	Además, es posible observar la presencia de islas, es decir, puntos dentro de los huecos que no lograron conectarse con el resto de la malla.

	El módulo de rellenado de huecos se encargará de estimar de forma automática la superficie del
	objeto en estas zonas para así obtener finalmente una malla cerrada.

	\section{Diagrama de clases}
		En la figura~\ref{fig:filling_class} se presentan las clases principales y sus interacciones.
		A continuación se presenta una breve descripción de las mismas.
		\begin{figure}
			\Imagen{uml/hole_filling.pdf}
			\caption{\label{fig:filling_class}Diagrama de clases del módulo de registración}
		\end{figure}

		\begin{itemize}
			\item {\bfseries Relleno\_de\_huecos:} La clase se encargará de estimar
				nuevos puntos en zonas donde se carece de información (huecos)
				y triangularlos para que la \emph{Malla} sea cerrada.
			\item {\bfseries Borde:} Es una colección de puntos ordenados
				que representa un borde de un hueco en la \emph{Malla}.
		\end{itemize}

	\section{Método A}
		Con el fin de simplificar la identificación de los huecos primeramente se eliminaron
		todas las islas al quedarse únicamente con la componente conectada que
		contenía la mayor cantidad de puntos.
		De esta forma, una arista que defina sólo un triángulo formará parte de un hueco,
		y podrá obtenerse el contorno del mismo recorriendo el grafo de conectividades. 

		Para realizar el rellenado se implementó una variante del método de \emph{advancing front}. %citar
		\begin{algorithm}
			\begin{algorithmic}[1]
				\Function{Advancing front}{Contorno}
					\State AF $\gets$ Contorno
					\Repeat
					\State $\alpha = \widehat{PCN} =$ ángulo mínimo(Contorno)
					\Until $\mbox{Contorno} \neq \emptyset$
				\EndFunction
			\end{algorithmic}
		\end{algorithm}


	%mi método
	%ver paper
	advancing front
	Definir isla
	Eliminar islas

	Dado un boundary a rellenar
		Buscar los segmentos que generen el menor ángulo
		caso $\alpha < 75$
			unir los extremos
		$\alpha < 1354$
			agregar un punto en la bisectriz
		$\alpha < \pi$
			agregar un punto en un tercio del ángulo
		$\alpha \geq \pi$
			no debería ocurrir

		Verificar que el nuevo punto no caiga muy cerca de otro ya existente
		En ese caso usar el existente
			Dividir el boundary en dos, procesar cada sección de forma independiente.

		De esta forma se tiene un frente que avanza desde el contorno hacia adentro, terminando por encontrarse con un frente opuesto y cerrando el hueco.


	Ubicación del nuevo punto
		Una vez determinado los tres puntos que definen el menor ángulo: 
			$\theta = \widehat{PCN}$

		se define el plano $\alpha$ mediante los tres puntos.
		se crea un nuevo punto Q sobre $\alpha$, ubicado en la bisectriz del ángulo, a una distancia step de C
		se define el plano $\beta$ como aquel cuya normal es el promedio de las normales de los tres puntos y pasa por el promedio de los tres puntos
		\[
		O = \frac{P+C+N}{3}
		n = \frac{P_n + C_n + N_n} {|P_n + C_n + N_n|}
		\]

		se proyecta Q en $\beta$

		`step' es constante en todo el proceso, siendo el promedio de las longitudes de los segmentos que forman el contorno del hueco


		Con este método se pueden rellenar agujeros pequeños, obteniéndose una malla bastante regular.
		Sin embargo, debido a la localidad con la que se generan nuevos puntos, el frente puede diverger o pretender unirse a puntos que no forman parte del contorno del hueco, perdiendo la propiedad de manifold

	\section{Método B}
	%Poisson
	La clase pcl::Poisson provee algoritmos de reconstrucción basados en %cite Poisson surface reconstruction
	siendo el principal parámetro la profundidad del octree utilizado,
	impactando directamente en la resolución de la malla resultante.


	Se transforma el problema en una ecuación de Poisson
	\[\Delta\chi \equiv \nabla \cdot\nabla\chi = \nabla \vec{n}\]

	Se realiza una discretización del dominio mediante un octree, ya que sólo
	interesa la solución de la función en la proximidad de la superficie a
	reconstruir.
	Luego se definen funciones de soporte local que aproximen una gaussiana.

	Una vez resuelto el problema en el dominio, se extrae la isosuperficie mediante una variante de marching cubes.

	Como resultado se tendrá una malla triangular que no presenta huecos.



	%\chapter{Conclusiones y trabajos futuros}
%TODO: introducción

\section{Conclusiones del producto}
%explayarse más
	En este proyecto se realizó el desarrollo de una biblioteca de software para lograr
	la reconstrucción tridimensional de un objeto a partir de capturas de
	vistas parciales.
	Para esto, se dividió el problema en tres módulos: registración, fusión y
	rellenado de huecos, y se implementaron diversos algoritmos.

	A pesar de que las capturas no contenían información de textura,
	el algoritmo de registración fue exitoso en casi todos los casos sin requerir
	ajustes a su conjunto de parámetros. 
	Además, se cuenta con medidas de la calidad de la alineación,
	que permiten detectar fallas durante esta etapa sin requerir de una inspección visual.

	%Al trabajar directamente con las nubes de puntos no se restringió el
	%dispositivo de captura a un hardware en particular.  Sin embargo, las
	%restricciones impuestas de base giratoria y limitar la cantidad de capturas
	%requeridas fueron planteadas considerando una integración futura con
	%el trabajo realizado por \cite{Pancho}.

	Las reconstrucciones fueron obtenidas en tiempos razonables y sin requerir hardware especial.
	Si bien se observa un efecto de «inflación/deflación» debido a la propagación de los errores de registración,
	este se encuentra suficientemente acotado respecto al tamaño del objeto.

	Debido a que se consideró solamente una posición del objeto sobre la base giratoria,
	se presenta una gran cantidad de oclusiones,
	lo que genera la aparición de huecos de tamaño considerable
	en la superficie reconstruida luego de la fusión.
	Aún así, el resultado final es una malla cerrada (a excepción de la base), y la
	superficie estimada en las zonas sin información se une suavemente al resto.

\section{Conclusiones del proceso}
%redacción informal
Se tuvieron problemas al implementar la metodología seleccionada.
El tiempo invertido en la etapa de investigación bibliográfica fue demasiado extenso
y se desperdiciaron recursos al abordar el tratamiento de la información de textura,
que finalmente debió ser descartada al no contar con un repositorio propio.
Además, se produjo un desfasaje temporal entre la adquisición de los conocimientos y la
implementación de los mismos, requiriendo un nuevo análisis.
Estos inconvenientes se hubieran resuelto
al utilizar directamente una metodología incremental en todo el proceso,
con más incrementos de menor tamaño, como ser agregar un primer módulo
de preproceso que contenga la reducción de ruido y la operatoria básica con las
nubes de puntos.

En cuanto al desarrollo, uno de los principales problemas fue la definición de métricas
para evaluar los algoritmos y establecer los niveles de error aceptables en cada etapa.
%Esto se dificulta, además, al considerar que los resultados producidos en una etapa
%serán la entrada de otra, de la cual se desconoce su sensibilidad.
Muchas evaluaciones fueron primeramente visuales, resultando en un proceso lento
que en ocasiones fallaba en detectar errores considerables.

Durante el desarrollo se efectivizó otro de los riesgos identificados para el proyecto:
la falla en los equipos de trabajo requiriendo su reemplazo.
Gracias a las copias de respaldo periódicas, fue posible recuperar fácilmente el trabajo realizado hasta ese momento.
Sin embargo, debido a que el nuevo equipo de trabajo contaba con otro sistema operativo (Clear Linux),
se requirió de un largo proceso de configuración para instalar la biblioteca PCL a partir de sus archivos fuentes.
%Primeramente, se contaba con un sistema operativo Arch Linux, donde la
%instalación de la biblioteca PCL se realiza mediante un script
%\texttt{PKGBUILD} que resuelve las dependencias y configura los módulos.  Fue
%necesario compilar los fuentes, pero fuera del tiempo requerido, no se tuvieron
%mayores inconvenientes.  En el nuevo equipo, se contaba esta vez con un sistema
%Clear Linux.  Ahora la instalación resultó más problemática.  Se requerían
%demasiados recursos de memoria, por lo que el sistema operativo detenía el
%proceso.  Fue necesario un largo proceso de configuración y prueba para lograr
%la instalación exitosa de la biblioteca.

%\subsection{Riesgos efectivizados}
%Ausencia de repositorio de mallas tridimensionales
%(copiar base de datos)
%
%Falla en los equipos de trabajo
%(copiar parte de pcl)
%Meshlab: no se logró instalarlo en el nuevo equipo, se cambió a CloudCompare



\section{Trabajos futuros}
%\TODO{fusión, mejora de la confianza. Confianza según cercanía al borde, promedio de confianza}

En esta sección se describen actividades que excedieron el alcance de este proyecto
y podrían ser abordadas en una etapa posterior.

\begin{itemize}
	\item Ajustar los métodos de registración para combinar escaneos del objeto
		en varias posiciones sobre la base giratoria,
		buscando de esta forma eliminar huecos y reducir la propagación del error de alineación.
		Esto requerirá eliminar la restricción del eje de giro en la registración
		y ajustar el algoritmo de corrección de bucle.
	\item Ajustar los métodos para trabajar con el volumen de puntos generados
		por \cite{Pancho}.
		Es necesario analizar la robustez del algoritmo al submuestreo de la entrada,
		realizar una selección de keypoints de las nubes de entrada
		y utilizar un método más eficiente para la búsqueda de correspondencias.
	\item \TODO{Intentar la paralelización de los métodos desarrollados.}
	\item \TODO{Paso a GPU.}
	\item Implementar métricas de calidad del mallado que consideren las
		características de la impresión 3D.
		En este proyecto solamente se consideraron las condiciones de que la malla
		resultase cerrada y no presente intersecciones consigo misma.
	%\item Mejorar la detección de \emph{outliers} en el módulo de fusión.
	%\item Implementar métodos de suavizado de mallas.
	\item \TODO{Cerrar la base Poisson}
	\item Modificar el método de \emph{advancing front} para que utilice una
		superficie de soporte para establecer la posición de los nuevos puntos,
		asegurando de esta forma la convergencia del método y la suavidad del
		parche generado.
	\item Modificar el método de \emph{advancing front} para que considere las islas.
		Esto requerirá detectar dentro de qué hueco se encuentra cada isla.
\end{itemize}

	\bibliographystyle{alpha}
	\bibliography{biblio}
\end{document}
