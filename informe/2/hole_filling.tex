\chapter{Módulo de rellenado de huecos}
	Al finalizar el algoritmo de fusión se obtuvo una malla triangular a partir
	de la información proveniente de cada vista.
	Sin embargo, esta malla no es cerrada ya que existen zonas que ninguna
	vista pudo capturar y por lo tanto carecen de puntos, produciendo huecos en la misma.
	Además, es posible observar la presencia de islas, es decir, puntos dentro de los huecos que no lograron conectarse con el resto de la malla.

	El módulo de rellenado de huecos se encargará de estimar de forma automática la superficie del
	objeto en estas zonas para así obtener finalmente una malla cerrada.

	\section{Diagrama de clases}
		En la figura~\ref{fig:filling_class} se presentan las clases principales y sus interacciones.
		A continuación se presenta una breve descripción de las mismas.
		\begin{figure}
			\Imagen{uml/hole_filling.pdf}
			\caption{\label{fig:filling_class}Diagrama de clases del módulo de registración}
		\end{figure}

		\begin{itemize}
			\item {\bfseries Relleno\_de\_huecos:} La clase se encargará de estimar
				nuevos puntos en zonas donde se carece de información (huecos)
				y triangularlos para que la \emph{Malla} sea cerrada.
			\item {\bfseries Borde:} Es una colección de puntos ordenados
				que representa un borde de un hueco en la \emph{Malla}.
		\end{itemize}

	\section{Método A}
		Con el fin de simplificar la identificación de los huecos primeramente se eliminaron
		todas las islas al quedarse únicamente con la componente conectada que
		contenía la mayor cantidad de puntos.
		De esta forma, una arista que defina sólo un triángulo formará parte de un hueco,
		y podrá obtenerse el contorno del mismo recorriendo el grafo de conectividades. 

		Para realizar el rellenado se implementó una variante del método de \emph{advancing front}. %citar
		\begin{algorithm}
			\begin{algorithmic}[1]
				\Function{Advancing front}{Contorno}
					\State AF $\gets$ Contorno
					\Repeat
					\State $\alpha = \widehat{PCN} =$ ángulo mínimo(Contorno)
					\Until $\mbox{Contorno} \neq \emptyset$
				\EndFunction
			\end{algorithmic}
		\end{algorithm}


	%mi método
	%ver paper
	advancing front
	Definir isla
	Eliminar islas

	Dado un boundary a rellenar
		Buscar los segmentos que generen el menor ángulo
		caso $\alpha < 75$
			unir los extremos
		$\alpha < 1354$
			agregar un punto en la bisectriz
		$\alpha < \pi$
			agregar un punto en un tercio del ángulo
		$\alpha \geq \pi$
			no debería ocurrir

		Verificar que el nuevo punto no caiga muy cerca de otro ya existente
		En ese caso usar el existente
			Dividir el boundary en dos, procesar cada sección de forma independiente.

		De esta forma se tiene un frente que avanza desde el contorno hacia adentro, terminando por encontrarse con un frente opuesto y cerrando el hueco.


	Ubicación del nuevo punto
		Una vez determinado los tres puntos que definen el menor ángulo: 
			$\theta = \widehat{PCN}$

		se define el plano $\alpha$ mediante los tres puntos.
		se crea un nuevo punto Q sobre $\alpha$, ubicado en la bisectriz del ángulo, a una distancia step de C
		se define el plano $\beta$ como aquel cuya normal es el promedio de las normales de los tres puntos y pasa por el promedio de los tres puntos
		\[
		O = \frac{P+C+N}{3}
		n = \frac{P_n + C_n + N_n} {|P_n + C_n + N_n|}
		\]

		se proyecta Q en $\beta$

		`step' es constante en todo el proceso, siendo el promedio de las longitudes de los segmentos que forman el contorno del hueco


		Con este método se pueden rellenar agujeros pequeños, obteniéndose una malla bastante regular.
		Sin embargo, debido a la localidad con la que se generan nuevos puntos, el frente puede diverger o pretender unirse a puntos que no forman parte del contorno del hueco, perdiendo la propiedad de manifold

	\section{Método B}
	%Poisson
	La clase pcl::Poisson provee algoritmos de reconstrucción basados en %cite Poisson surface reconstruction
	siendo el principal parámetro la profundidad del octree utilizado,
	impactando directamente en la resolución de la malla resultante.


	Se transforma el problema en una ecuación de Poisson
	\[\Delta\chi \equiv \nabla \cdot\nabla\chi = \nabla \vec{n}\]

	Se realiza una discretización del dominio mediante un octree, ya que sólo
	interesa la solución de la función en la proximidad de la superficie a
	reconstruir.
	Luego se definen funciones de soporte local que aproximen una gaussiana.

	Una vez resuelto el problema en el dominio, se extrae la isosuperficie mediante una variante de marching cubes.

	Como resultado se tendrá una malla triangular que no presenta huecos.


