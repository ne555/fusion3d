\chapter{Módulo de registración}
	%¿Qué es la registración?
	Dado un conjunto de nubes de puntos correspondientes a distintas vistas de
	un objeto, la registración consiste en calcular las transformaciones de
	rotación y translación que lleven a cada vista a un sistema global de forma
	que las zonas comunes encajen perfectamente.

	Los métodos de registración seguirán los siguientes pasos:
	\begin{enumerate}
		\item Selección de puntos de la entrada (\emph{keypoints}).
		\item Cálculo de descriptores y determinación de correspondencias.
		\item Rechazo de correspondencias.
		\item Alineación.
	\end{enumerate}
	Variaciones en estos pasos permiten la implementación de diversos métodos.

		%pseudocódigo
		%ref{FPFH}
	Como medida de reducción de ruido y para evitar considerar puntos outliers
	durante la alineación, se realizó un preproceso de las nubes de entrada.
	%ver que suavice y que se llame así
	Primeramente se ajustaron los puntos a una superficie estimada mediante el
	método de mínimos cuadrados móviles y luego se descartaron puntos de poca
	confianza, como ser aquellos cerca de los bordes o cuyas normales sean
	ortogonales a la cámara.

	\section{Diagrama de clases}
		En la figura~\ref{fig:align_class} se presentan las clases principales y sus interacciones.
		A continuación se presenta una breve descripción de las mismas.
		\begin{figure}
			%\Imagen{uml/align.png}
			%\centering
			%\def\svgwidth{\linewidth}
			%\chapter{Módulo de registración}
	%¿Qué es la registración?
	Dado un conjunto de nubes de puntos correspondientes a distintas vistas de
	un objeto, la registración consiste en calcular las transformaciones de
	rotación y translación que lleven a cada vista a un sistema global de forma
	que las zonas comunes encajen perfectamente.

	Se tomará la primera captura como marco de referencia, cada nueva vista se
	alineará con la anterior hasta completar una vuelta sobre el objeto.
	El problema, entonces, se resuelven encontrando para cada punto de la nube
	A su posición, si es que es visible, en la nube B, para luego estimar la
	transformación que alinee todos estos puntos.

	Las coordenadas espaciales de un punto no nos suministran suficiente
	información para poder identificarlo en otra vista, es necesario utilizar
	sus relaciones con otros puntos en una vecindad.
	Así, podrá describirse al punto mediante las posiciones relativas, la
	densidad o la orientación, para luego corresponderlo con el más parecido en
	la otra vista.
	Sin embargo, los puntos en zonas homogéneas de la nube serán descriptos de
	forma muy similar, con lo que se corre el riesgo de realizar
	correspondencias equívocas que producirán una alineación errónea.

	Para solventar este último problema, se puede tomar un subconjunto de los
	puntos de entrada, aquellos puntos que por sus características de vecindad
	sea más probable que tengan un descriptor único (\emph{keypoints}).

	Entonces, los métodos de registración seguirán los siguientes pasos:
	\begin{enumerate}
		\item Selección de puntos de la entrada (\emph{keypoints}).
		\item Cálculo de descriptores y determinación de correspondencias.
		\item Rechazo de correspondencias.
		\item Alineación.
	\end{enumerate}
	Variaciones en estos pasos permiten la implementación de diversos métodos.

		%pseudocódigo
		%ref{FPFH}
	%FIXME: queda muy colgado
	Como medida de reducción de ruido y para evitar considerar puntos outliers
	durante la alineación, se realizó un preproceso de las nubes de entrada.
	%ver que suavice y que se llame así
	Primeramente se ajustaron los puntos a una superficie estimada mediante el
	método de mínimos cuadrados móviles y luego se descartaron puntos de poca
	confianza, como ser aquellos correspondientes a los bordes o aquellos cuyas
	normales resulten ortogonales a la cámara.

	\section{Diagrama de clases}
		En la figura~\ref{fig:align_class} se presentan las clases principales y sus interacciones.
		A continuación se presenta una breve descripción de las mismas.
		\begin{figure}
			%\Imagen{uml/align.png}
			%\centering
			%\def\svgwidth{\linewidth}
			%\chapter{Módulo de registración}
	%¿Qué es la registración?
	Dado un conjunto de nubes de puntos correspondientes a distintas vistas de
	un objeto, la registración consiste en calcular las transformaciones de
	rotación y translación que lleven a cada vista a un sistema global de forma
	que las zonas comunes encajen perfectamente.

	Se tomará la primera captura como marco de referencia, cada nueva vista se
	alineará con la anterior hasta completar una vuelta sobre el objeto.
	El problema, entonces, se resuelven encontrando para cada punto de la nube
	A su posición, si es que es visible, en la nube B, para luego estimar la
	transformación que alinee todos estos puntos.

	Las coordenadas espaciales de un punto no nos suministran suficiente
	información para poder identificarlo en otra vista, es necesario utilizar
	sus relaciones con otros puntos en una vecindad.
	Así, podrá describirse al punto mediante las posiciones relativas, la
	densidad o la orientación, para luego corresponderlo con el más parecido en
	la otra vista.
	Sin embargo, los puntos en zonas homogéneas de la nube serán descriptos de
	forma muy similar, con lo que se corre el riesgo de realizar
	correspondencias equívocas que producirán una alineación errónea.

	Para solventar este último problema, se puede tomar un subconjunto de los
	puntos de entrada, aquellos puntos que por sus características de vecindad
	sea más probable que tengan un descriptor único (\emph{keypoints}).

	Entonces, los métodos de registración seguirán los siguientes pasos:
	\begin{enumerate}
		\item Selección de puntos de la entrada (\emph{keypoints}).
		\item Cálculo de descriptores y determinación de correspondencias.
		\item Rechazo de correspondencias.
		\item Alineación.
	\end{enumerate}
	Variaciones en estos pasos permiten la implementación de diversos métodos.

		%pseudocódigo
		%ref{FPFH}
	%FIXME: queda muy colgado
	Como medida de reducción de ruido y para evitar considerar puntos outliers
	durante la alineación, se realizó un preproceso de las nubes de entrada.
	%ver que suavice y que se llame así
	Primeramente se ajustaron los puntos a una superficie estimada mediante el
	método de mínimos cuadrados móviles y luego se descartaron puntos de poca
	confianza, como ser aquellos correspondientes a los bordes o aquellos cuyas
	normales resulten ortogonales a la cámara.

	\section{Diagrama de clases}
		En la figura~\ref{fig:align_class} se presentan las clases principales y sus interacciones.
		A continuación se presenta una breve descripción de las mismas.
		\begin{figure}
			%\Imagen{uml/align.png}
			%\centering
			%\def\svgwidth{\linewidth}
			%\chapter{Módulo de registración}
	%¿Qué es la registración?
	Dado un conjunto de nubes de puntos correspondientes a distintas vistas de
	un objeto, la registración consiste en calcular las transformaciones de
	rotación y translación que lleven a cada vista a un sistema global de forma
	que las zonas comunes encajen perfectamente.

	Se tomará la primera captura como marco de referencia, cada nueva vista se
	alineará con la anterior hasta completar una vuelta sobre el objeto.
	El problema, entonces, se resuelven encontrando para cada punto de la nube
	A su posición, si es que es visible, en la nube B, para luego estimar la
	transformación que alinee todos estos puntos.

	Las coordenadas espaciales de un punto no nos suministran suficiente
	información para poder identificarlo en otra vista, es necesario utilizar
	sus relaciones con otros puntos en una vecindad.
	Así, podrá describirse al punto mediante las posiciones relativas, la
	densidad o la orientación, para luego corresponderlo con el más parecido en
	la otra vista.
	Sin embargo, los puntos en zonas homogéneas de la nube serán descriptos de
	forma muy similar, con lo que se corre el riesgo de realizar
	correspondencias equívocas que producirán una alineación errónea.

	Para solventar este último problema, se puede tomar un subconjunto de los
	puntos de entrada, aquellos puntos que por sus características de vecindad
	sea más probable que tengan un descriptor único (\emph{keypoints}).

	Entonces, los métodos de registración seguirán los siguientes pasos:
	\begin{enumerate}
		\item Selección de puntos de la entrada (\emph{keypoints}).
		\item Cálculo de descriptores y determinación de correspondencias.
		\item Rechazo de correspondencias.
		\item Alineación.
	\end{enumerate}
	Variaciones en estos pasos permiten la implementación de diversos métodos.

		%pseudocódigo
		%ref{FPFH}
	%FIXME: queda muy colgado
	Como medida de reducción de ruido y para evitar considerar puntos outliers
	durante la alineación, se realizó un preproceso de las nubes de entrada.
	%ver que suavice y que se llame así
	Primeramente se ajustaron los puntos a una superficie estimada mediante el
	método de mínimos cuadrados móviles y luego se descartaron puntos de poca
	confianza, como ser aquellos correspondientes a los bordes o aquellos cuyas
	normales resulten ortogonales a la cámara.

	\section{Diagrama de clases}
		En la figura~\ref{fig:align_class} se presentan las clases principales y sus interacciones.
		A continuación se presenta una breve descripción de las mismas.
		\begin{figure}
			%\Imagen{uml/align.png}
			%\centering
			%\def\svgwidth{\linewidth}
			%\input{uml/align.pdf_tex}
			\Imagen{uml/align.pdf}
			\caption{\label{fig:align_class}Diagrama de clases del módulo de registración}
		\end{figure}

		\begin{itemize}
			\item {\bfseries Registración:} se encarga de obtener la \emph{Transformación} que
				permita alinear dos \emph{Nubes} entre sí.  Para esto establece
				correspondencias entre los \emph{Anclaje}.
			\item {\bfseries Anclaje:} a partir de puntos salientes de \emph{Nube} calcula
				\emph{descriptores}  que permitan asociarlos y un
				\emph{marco\_de\_referencia} para obtener una estimación de la
				\emph{Transformación}.
			\item {\bfseries Nube:} representa una vista del objeto que se desea alinear.
				Es una colección de \emph{Puntos} sin organización. Clase
				provista por PCL.
			\item {\bfseries Punto:} contiene las coordenadas $xyz$ obtenidas por el
				dispositivo de captura. El algoritmo estimará las normales.
			\item {\bfseries Transformación:} representa una transformación rígida
				(rotación y translación) que será aplicada a una \emph{Nube}
				para alinearla.
		\end{itemize}

	\section{Método A}
		\subsection{Selección de keypoints}
			Basándonos en los resultados obtenidos por %ref ISS
			se consideró utilizar el algoritmo de detección de keypoints basado en \emph{Intrinsic Shape Signatures} (algoritmo~\ref{alg:iss}),
			el cual se haya implementado en PCL en la clase \texttt{ISSKeypoint3D}, permitiendo
			definir el radio de la esfera y el nivel de disimilitud.

			%ver bien el problema
			Sin embargo, no pudieron encontrarse los parámetros adecuados.
			Los keypoints resultaban en toda la superficie y al calcular luego
			los descriptores se generaban demasiadas correspondencias erróneas
			produciendo una alineación incorrecta.
			%FIXME: gráficos

			\begin{algorithm}
				\begin{algorithmic}[1]
					\Function{ISS Keypoints}{nube}
						\State keypoints $\gets\emptyset$
						\ForAll{$p \in \mbox{nube.puntos}$}
							\State vecinos $\gets$ obtener puntos cercanos(nube, p, $r_1$)
							\State m $\gets$ matriz de covarianza(vecinos)
							\State $\lambda$ = eigenvalores(m)
							\If{$\lambda_1/\lambda_2 > \mbox{umbral}_1$ and $\lambda_2/\lambda_3 > \mbox{umbral}_2$}
								\State keypoints.insert(p)
							\EndIf
						\EndFor
						\State\Return Non-Max Suppression(keypoints, $r_2$)
					\EndFunction
				\end{algorithmic}
				\caption{\label{alg:iss}Determinación de los keypoints mediante ISS}
			\end{algorithm}


			%A partir de acá ref Rusu FPFH
			Se procedió entonces a cambiar el método de selección de keypoints, eligiendo ahora un análisis de persistencia multiescala.
			\begin{enumerate}
				\item Por cada punto de la nube se calcula su descriptor para distintos tamaños de vecindad (escala).
				\item A partir de todos los descriptores en todas las escalas se estima una distribución gaussiana que los aproxime.
				\item Los keypoints quedan definidos como aquellos puntos cuyos descriptores se encuentran alejados de la media.
			\end{enumerate}
			El algoritmo se encuentra implementado en PCL en la clase
			\texttt{Multiscale\-Feature\-Persistence} permitiendo ajustar las
			escalas a utilizar, el umbral para ser considerado saliente y la
			función descriptora a utilizar.

			Debido a que es necesario calcular el descriptor para cada punto, y además en diferentes escalas, se eligió como descriptor 
			\emph{Fast Point Feature Histograms} (FPFH), el cual es lineal en la cantidad de puntos de la vecindad.
			Este descriptor calcula un histograma de los ángulos entre las normales del punto y sus vecinos. %ref FPFH

			Los keypoints se encontraban ahora agrupados, formando líneas en zonas de cambio brusco de curvatura.
			%FIXME: gráficos




		\subsection{Alineación}
			Se utilizó el algoritmo de \emph{sample consensus initial alignment (SAC-IA)} para la alineación, el cual consiste en:
			\begin{enumerate}
				\item Seleccionar al azar \emph{m} puntos de la nube A
				\item Por cada punto, buscar aquellos con descriptores similares en B. Seleccionar uno al azar.
				\item Calcular la transformación definida por estos puntos y
					sus correspondencias. Calcular, además, una medida del
					error de transformación.
				\item Repetir varias veces y devolver aquella transformación que produjo el menor error.
			\end{enumerate}

			\subsection{Resultados}
			Mediante este método, se obtuvieron buenos resultados en la mayoría de las
			capturas de \texttt{happy}
			donde los ángulos eran cercanos a $25^{\circ}$.
			Sin embargo, algunas alineaciones presentaban problemas de deslizamiento.

			En el caso de \texttt{bunny}, cuyos ángulos eran cercanos a $45^{\circ}$,
			se obtuvo una buena alineación en el caso de las capturas 315--000 y 000--045,
			sin embargo, para los otros casos las correspondencias fueron completamente erróneas.

			%gráficos


	%El que tengo ahora
	\section{Método B}
		En este caso no se seleccionaron keypoints, simplemente se realizó un submuestreo de
		los puntos de la nube para reducir el costo computacional.
		Además del descriptor, en cada punto se estableció un marco de referencia utilizando los
		eigenvectores de la matriz de covarianza de la vecindad del punto. Este
		marco de referencia nos permite establecer
		la transformación de alineación considerando solamente dos puntos. %ref ISS

		Para establecer las correspondencias se utilizó el descriptor FPFH
		comparando los histogramas mediante la distancia $\chi^2$.  Luego se
		procedió a eliminar correspondencias erróneas utilizando los marcos de
		referencia y las suposiciones de ubicación de la cámara en la obtención
		de las capturas.  Por esto, se descartan aquellas correspondencias que
		requieren un movimiento en $y$ excesivo o una rotación sobre un eje no
		vertical. 

		Cada correspondencia entonces define un ángulo de giro $\theta$ sobre
		el eje $y$ y una translación en el plano $xz$.  Se observará entonces
		una agrupación de los parámetros de estas transformaciones, mediante el
		algoritmo de k-means se buscará el centroide del clúster más grande.
		%gráfico

		\subsection{Resultados}
			Se realizaron pruebas sobre los objetos \texttt{happy} y \texttt{bunny}.

			En el caso de \texttt{bunny} se ignoró la captura
			\texttt{bun180.ply} ya que presentaba un salto de aproximadamente
			$90^{\circ}$ respecto a su anterior y siguiente.  Las otras
			capturas presentaban un desvío no mayor a $2^{\circ}$ respecto al
			\emph{ground truth}.

			Para \texttt{happy} se tuvo un desvío medio de poco más de
			$2^{\circ}$, con un máximo de $6^{\circ}$ respecto al \emph{ground truth}.

			De esta forma, se logró acercar las capturas lo suficiente como
			para intentar alinearlas por ICP.


	\section{Refinamiento}
	Una vez obtenida la alineación inicial, se procedió a realizar una segunda
	alineación mediante ICP.
	Se consideraron sólo las áreas solapadas y se restringió el
	espacio de búsqueda de las correspondencias.

	Luego, para reducir el error propagado por cada alineación, se propuso una
	corrección de bucle.
	Se ajustó la última captura para que correspondiese con la primera, y
	esta transformación se agregó a las otras alineaciones con un peso
	proporcional a su posición en el bucle.


	%informes de prueba
	%gráficos

	%casos de uso/funcionalidades resueltas

			\Imagen{uml/align.pdf}
			\caption{\label{fig:align_class}Diagrama de clases del módulo de registración}
		\end{figure}

		\begin{itemize}
			\item {\bfseries Registración:} se encarga de obtener la \emph{Transformación} que
				permita alinear dos \emph{Nubes} entre sí.  Para esto establece
				correspondencias entre los \emph{Anclaje}.
			\item {\bfseries Anclaje:} a partir de puntos salientes de \emph{Nube} calcula
				\emph{descriptores}  que permitan asociarlos y un
				\emph{marco\_de\_referencia} para obtener una estimación de la
				\emph{Transformación}.
			\item {\bfseries Nube:} representa una vista del objeto que se desea alinear.
				Es una colección de \emph{Puntos} sin organización. Clase
				provista por PCL.
			\item {\bfseries Punto:} contiene las coordenadas $xyz$ obtenidas por el
				dispositivo de captura. El algoritmo estimará las normales.
			\item {\bfseries Transformación:} representa una transformación rígida
				(rotación y translación) que será aplicada a una \emph{Nube}
				para alinearla.
		\end{itemize}

	\section{Método A}
		\subsection{Selección de keypoints}
			Basándonos en los resultados obtenidos por %ref ISS
			se consideró utilizar el algoritmo de detección de keypoints basado en \emph{Intrinsic Shape Signatures} (algoritmo~\ref{alg:iss}),
			el cual se haya implementado en PCL en la clase \texttt{ISSKeypoint3D}, permitiendo
			definir el radio de la esfera y el nivel de disimilitud.

			%ver bien el problema
			Sin embargo, no pudieron encontrarse los parámetros adecuados.
			Los keypoints resultaban en toda la superficie y al calcular luego
			los descriptores se generaban demasiadas correspondencias erróneas
			produciendo una alineación incorrecta.
			%FIXME: gráficos

			\begin{algorithm}
				\begin{algorithmic}[1]
					\Function{ISS Keypoints}{nube}
						\State keypoints $\gets\emptyset$
						\ForAll{$p \in \mbox{nube.puntos}$}
							\State vecinos $\gets$ obtener puntos cercanos(nube, p, $r_1$)
							\State m $\gets$ matriz de covarianza(vecinos)
							\State $\lambda$ = eigenvalores(m)
							\If{$\lambda_1/\lambda_2 > \mbox{umbral}_1$ and $\lambda_2/\lambda_3 > \mbox{umbral}_2$}
								\State keypoints.insert(p)
							\EndIf
						\EndFor
						\State\Return Non-Max Suppression(keypoints, $r_2$)
					\EndFunction
				\end{algorithmic}
				\caption{\label{alg:iss}Determinación de los keypoints mediante ISS}
			\end{algorithm}


			%A partir de acá ref Rusu FPFH
			Se procedió entonces a cambiar el método de selección de keypoints, eligiendo ahora un análisis de persistencia multiescala.
			\begin{enumerate}
				\item Por cada punto de la nube se calcula su descriptor para distintos tamaños de vecindad (escala).
				\item A partir de todos los descriptores en todas las escalas se estima una distribución gaussiana que los aproxime.
				\item Los keypoints quedan definidos como aquellos puntos cuyos descriptores se encuentran alejados de la media.
			\end{enumerate}
			El algoritmo se encuentra implementado en PCL en la clase
			\texttt{Multiscale\-Feature\-Persistence} permitiendo ajustar las
			escalas a utilizar, el umbral para ser considerado saliente y la
			función descriptora a utilizar.

			Debido a que es necesario calcular el descriptor para cada punto, y además en diferentes escalas, se eligió como descriptor 
			\emph{Fast Point Feature Histograms} (FPFH), el cual es lineal en la cantidad de puntos de la vecindad.
			Este descriptor calcula un histograma de los ángulos entre las normales del punto y sus vecinos. %ref FPFH

			Los keypoints se encontraban ahora agrupados, formando líneas en zonas de cambio brusco de curvatura.
			%FIXME: gráficos




		\subsection{Alineación}
			Se utilizó el algoritmo de \emph{sample consensus initial alignment (SAC-IA)} para la alineación, el cual consiste en:
			\begin{enumerate}
				\item Seleccionar al azar \emph{m} puntos de la nube A
				\item Por cada punto, buscar aquellos con descriptores similares en B. Seleccionar uno al azar.
				\item Calcular la transformación definida por estos puntos y
					sus correspondencias. Calcular, además, una medida del
					error de transformación.
				\item Repetir varias veces y devolver aquella transformación que produjo el menor error.
			\end{enumerate}

			\subsection{Resultados}
			Mediante este método, se obtuvieron buenos resultados en la mayoría de las
			capturas de \texttt{happy}
			donde los ángulos eran cercanos a $25^{\circ}$.
			Sin embargo, algunas alineaciones presentaban problemas de deslizamiento.

			En el caso de \texttt{bunny}, cuyos ángulos eran cercanos a $45^{\circ}$,
			se obtuvo una buena alineación en el caso de las capturas 315--000 y 000--045,
			sin embargo, para los otros casos las correspondencias fueron completamente erróneas.

			%gráficos


	%El que tengo ahora
	\section{Método B}
		En este caso no se seleccionaron keypoints, simplemente se realizó un submuestreo de
		los puntos de la nube para reducir el costo computacional.
		Además del descriptor, en cada punto se estableció un marco de referencia utilizando los
		eigenvectores de la matriz de covarianza de la vecindad del punto. Este
		marco de referencia nos permite establecer
		la transformación de alineación considerando solamente dos puntos. %ref ISS

		Para establecer las correspondencias se utilizó el descriptor FPFH
		comparando los histogramas mediante la distancia $\chi^2$.  Luego se
		procedió a eliminar correspondencias erróneas utilizando los marcos de
		referencia y las suposiciones de ubicación de la cámara en la obtención
		de las capturas.  Por esto, se descartan aquellas correspondencias que
		requieren un movimiento en $y$ excesivo o una rotación sobre un eje no
		vertical. 

		Cada correspondencia entonces define un ángulo de giro $\theta$ sobre
		el eje $y$ y una translación en el plano $xz$.  Se observará entonces
		una agrupación de los parámetros de estas transformaciones, mediante el
		algoritmo de k-means se buscará el centroide del clúster más grande.
		%gráfico

		\subsection{Resultados}
			Se realizaron pruebas sobre los objetos \texttt{happy} y \texttt{bunny}.

			En el caso de \texttt{bunny} se ignoró la captura
			\texttt{bun180.ply} ya que presentaba un salto de aproximadamente
			$90^{\circ}$ respecto a su anterior y siguiente.  Las otras
			capturas presentaban un desvío no mayor a $2^{\circ}$ respecto al
			\emph{ground truth}.

			Para \texttt{happy} se tuvo un desvío medio de poco más de
			$2^{\circ}$, con un máximo de $6^{\circ}$ respecto al \emph{ground truth}.

			De esta forma, se logró acercar las capturas lo suficiente como
			para intentar alinearlas por ICP.


	\section{Refinamiento}
	Una vez obtenida la alineación inicial, se procedió a realizar una segunda
	alineación mediante ICP.
	Se consideraron sólo las áreas solapadas y se restringió el
	espacio de búsqueda de las correspondencias.

	Luego, para reducir el error propagado por cada alineación, se propuso una
	corrección de bucle.
	Se ajustó la última captura para que correspondiese con la primera, y
	esta transformación se agregó a las otras alineaciones con un peso
	proporcional a su posición en el bucle.


	%informes de prueba
	%gráficos

	%casos de uso/funcionalidades resueltas

			\Imagen{uml/align.pdf}
			\caption{\label{fig:align_class}Diagrama de clases del módulo de registración}
		\end{figure}

		\begin{itemize}
			\item {\bfseries Registración:} se encarga de obtener la \emph{Transformación} que
				permita alinear dos \emph{Nubes} entre sí.  Para esto establece
				correspondencias entre los \emph{Anclaje}.
			\item {\bfseries Anclaje:} a partir de puntos salientes de \emph{Nube} calcula
				\emph{descriptores}  que permitan asociarlos y un
				\emph{marco\_de\_referencia} para obtener una estimación de la
				\emph{Transformación}.
			\item {\bfseries Nube:} representa una vista del objeto que se desea alinear.
				Es una colección de \emph{Puntos} sin organización. Clase
				provista por PCL.
			\item {\bfseries Punto:} contiene las coordenadas $xyz$ obtenidas por el
				dispositivo de captura. El algoritmo estimará las normales.
			\item {\bfseries Transformación:} representa una transformación rígida
				(rotación y translación) que será aplicada a una \emph{Nube}
				para alinearla.
		\end{itemize}

	\section{Método A}
		\subsection{Selección de keypoints}
			Basándonos en los resultados obtenidos por %ref ISS
			se consideró utilizar el algoritmo de detección de keypoints basado en \emph{Intrinsic Shape Signatures} (algoritmo~\ref{alg:iss}),
			el cual se haya implementado en PCL en la clase \texttt{ISSKeypoint3D}, permitiendo
			definir el radio de la esfera y el nivel de disimilitud.

			%ver bien el problema
			Sin embargo, no pudieron encontrarse los parámetros adecuados.
			Los keypoints resultaban en toda la superficie y al calcular luego
			los descriptores se generaban demasiadas correspondencias erróneas
			produciendo una alineación incorrecta.
			%FIXME: gráficos

			\begin{algorithm}
				\begin{algorithmic}[1]
					\Function{ISS Keypoints}{nube}
						\State keypoints $\gets\emptyset$
						\ForAll{$p \in \mbox{nube.puntos}$}
							\State vecinos $\gets$ obtener puntos cercanos(nube, p, $r_1$)
							\State m $\gets$ matriz de covarianza(vecinos)
							\State $\lambda$ = eigenvalores(m)
							\If{$\lambda_1/\lambda_2 > \mbox{umbral}_1$ and $\lambda_2/\lambda_3 > \mbox{umbral}_2$}
								\State keypoints.insert(p)
							\EndIf
						\EndFor
						\State\Return Non-Max Suppression(keypoints, $r_2$)
					\EndFunction
				\end{algorithmic}
				\caption{\label{alg:iss}Determinación de los keypoints mediante ISS}
			\end{algorithm}


			%A partir de acá ref Rusu FPFH
			Se procedió entonces a cambiar el método de selección de keypoints, eligiendo ahora un análisis de persistencia multiescala.
			\begin{enumerate}
				\item Por cada punto de la nube se calcula su descriptor para distintos tamaños de vecindad (escala).
				\item A partir de todos los descriptores en todas las escalas se estima una distribución gaussiana que los aproxime.
				\item Los keypoints quedan definidos como aquellos puntos cuyos descriptores se encuentran alejados de la media.
			\end{enumerate}
			El algoritmo se encuentra implementado en PCL en la clase
			\texttt{Multiscale\-Feature\-Persistence} permitiendo ajustar las
			escalas a utilizar, el umbral para ser considerado saliente y la
			función descriptora a utilizar.

			Debido a que es necesario calcular el descriptor para cada punto, y además en diferentes escalas, se eligió como descriptor 
			\emph{Fast Point Feature Histograms} (FPFH), el cual es lineal en la cantidad de puntos de la vecindad.
			Este descriptor calcula un histograma de los ángulos entre las normales del punto y sus vecinos. %ref FPFH

			Los keypoints se encontraban ahora agrupados, formando líneas en zonas de cambio brusco de curvatura.
			%FIXME: gráficos




		\subsection{Alineación}
			Se utilizó el algoritmo de \emph{sample consensus initial alignment (SAC-IA)} para la alineación, el cual consiste en:
			\begin{enumerate}
				\item Seleccionar al azar \emph{m} puntos de la nube A
				\item Por cada punto, buscar aquellos con descriptores similares en B. Seleccionar uno al azar.
				\item Calcular la transformación definida por estos puntos y
					sus correspondencias. Calcular, además, una medida del
					error de transformación.
				\item Repetir varias veces y devolver aquella transformación que produjo el menor error.
			\end{enumerate}

			\subsection{Resultados}
			Mediante este método, se obtuvieron buenos resultados en la mayoría de las
			capturas de \texttt{happy}
			donde los ángulos eran cercanos a $25^{\circ}$.
			Sin embargo, algunas alineaciones presentaban problemas de deslizamiento.

			En el caso de \texttt{bunny}, cuyos ángulos eran cercanos a $45^{\circ}$,
			se obtuvo una buena alineación en el caso de las capturas 315--000 y 000--045,
			sin embargo, para los otros casos las correspondencias fueron completamente erróneas.

			%gráficos


	%El que tengo ahora
	\section{Método B}
		En este caso no se seleccionaron keypoints, simplemente se realizó un submuestreo de
		los puntos de la nube para reducir el costo computacional.
		Además del descriptor, en cada punto se estableció un marco de referencia utilizando los
		eigenvectores de la matriz de covarianza de la vecindad del punto. Este
		marco de referencia nos permite establecer
		la transformación de alineación considerando solamente dos puntos. %ref ISS

		Para establecer las correspondencias se utilizó el descriptor FPFH
		comparando los histogramas mediante la distancia $\chi^2$.  Luego se
		procedió a eliminar correspondencias erróneas utilizando los marcos de
		referencia y las suposiciones de ubicación de la cámara en la obtención
		de las capturas.  Por esto, se descartan aquellas correspondencias que
		requieren un movimiento en $y$ excesivo o una rotación sobre un eje no
		vertical. 

		Cada correspondencia entonces define un ángulo de giro $\theta$ sobre
		el eje $y$ y una translación en el plano $xz$.  Se observará entonces
		una agrupación de los parámetros de estas transformaciones, mediante el
		algoritmo de k-means se buscará el centroide del clúster más grande.
		%gráfico

		\subsection{Resultados}
			Se realizaron pruebas sobre los objetos \texttt{happy} y \texttt{bunny}.

			En el caso de \texttt{bunny} se ignoró la captura
			\texttt{bun180.ply} ya que presentaba un salto de aproximadamente
			$90^{\circ}$ respecto a su anterior y siguiente.  Las otras
			capturas presentaban un desvío no mayor a $2^{\circ}$ respecto al
			\emph{ground truth}.

			Para \texttt{happy} se tuvo un desvío medio de poco más de
			$2^{\circ}$, con un máximo de $6^{\circ}$ respecto al \emph{ground truth}.

			De esta forma, se logró acercar las capturas lo suficiente como
			para intentar alinearlas por ICP.


	\section{Refinamiento}
	Una vez obtenida la alineación inicial, se procedió a realizar una segunda
	alineación mediante ICP.
	Se consideraron sólo las áreas solapadas y se restringió el
	espacio de búsqueda de las correspondencias.

	Luego, para reducir el error propagado por cada alineación, se propuso una
	corrección de bucle.
	Se ajustó la última captura para que correspondiese con la primera, y
	esta transformación se agregó a las otras alineaciones con un peso
	proporcional a su posición en el bucle.


	%informes de prueba
	%gráficos

	%casos de uso/funcionalidades resueltas

			\Imagen{uml/align.pdf}
			\caption{\label{fig:align_class}Diagrama de clases del módulo de registración}
		\end{figure}

		\begin{itemize}
			\item {\bfseries Registración:} se encarga de obtener la \emph{Transformación} que
				permita alinear dos \emph{Nubes} entre sí.  Para esto establece
				correspondencias entre los \emph{Anclaje}.
			\item {\bfseries Anclaje:} a partir de puntos salientes de \emph{Nube} calcula
				\emph{descriptores}  que permitan asociarlos y un
				\emph{marco\_de\_referencia} para obtener una estimación de la
				\emph{Transformación}.
			\item {\bfseries Nube:} representa una vista del objeto que se desea alinear.
				Es una colección de \emph{Puntos} sin organización. Clase
				provista por PCL.
			\item {\bfseries Punto:} contiene las coordenadas $xyz$ obtenidas por el
				dispositivo de captura. El algoritmo estimará las normales.
			\item {\bfseries Transformación:} representa una transformación rígida
				(rotación y translación) que será aplicada a una \emph{Nube}
				para alinearla.
		\end{itemize}

	\section{Keypoints}
	%\lstset{title=Iss keypoints,caption=Determinación de los keypoints mediante ISS,label{alg:iss}}
	%\lstinputlisting{iss.py}

	\begin{algorithm}
		\begin{algorithmic}[1]
			\Function{ISS keypoints}{nube}
				\State keypoints $\gets\emptyset$
				\ForAll{$p \in \mbox{nube.puntos}$}
					\State vecinos $\gets$ obtener puntos cercanos(nube, p, $r_1$)
					\State m $\gets$ matriz de covarianza(vecinos)
					\State $\lambda$ = eigenvalores(m)
					\If{$\lambda_1/\lambda_2 > \mbox{umbral}_1$ and $\lambda_2/\lambda_3 > \mbox{umbral}_2$}
						\State keypoints.insert(p)
					\EndIf
				\EndFor
				\State\Return Non-Max Suppression(keypoints, $r_2$)
			\EndFunction
		\end{algorithmic}
		\caption{\label{alg:iss}Determinación de los keypoints mediante ISS}
	\end{algorithm}

		ISS:
		%descripción
		Por cada punto $p$ de la nube se calcula la matriz de covarianza con
		todos los puntos que caen dentro de una esfera de radio $r$ centrada en
		$p$.
		Un punto es detectado como keypoint si los eigenvalores de esta matriz
		de covarianza son suficientemente disímiles y además presenta la mayor disimiluted entre sus vecinos.

		La clase pcl::ISSKeypoint3D permite identificar estos keypoints, permitiendo definir el radio de la esfera y el nivel de disimilitud.
		%ver bien el problema
		Sin embargo, no pudieron encontrarse los parámetros adecuados.
		Se obtuvieron muy pocos keypoints, imposibilitando la detección de outliers,
		y además estos no resultaban representativos ya que sus correspondientes en la otra nube se encontraban demasiado alejados.

		Multiscale persistence:
		%descripción
		Por cada punto de la nube se calcula su descriptor para distintos tamaños de vecindad (escala).
		A partir de todos los descriptores en todas las escalas se estima una distribución gaussiana que los aproxime.
		Los keypoints quedan definidos como aquellos puntos cuyos descriptores se encuentran alejados de la media.

		Se utilizó Fast Point Feature Histograms (FPFH) como descriptor debido a su bajo costo computacional.

		all:
		Se realiza un submuestreo de los puntos de la nube simplemente para
		reducir el costo computacional.
		Nuevamente, y por las mismas razones, el descriptor elegido fue FPFH.
		El algoritmo de rechazo de correspondencias se encargará de eliminar outliers y asegurar la convergencia.

	\section{Alineación}
	En el caso de los keypoints de multiscale persistence se utilizó el
	algoritmo de sample consensus initial alignment (SAC-IA) para realizar la
	primera alineación.

	Este algoritmo selecciona de forma aleatoria keypoints de la nube, busca
	puntos cercanos en la otra nube, seleccionando nuevamente de forma
	aleatoria, y calcula la transformación que los alinee, obteniéndose una
	medida del error de la transformación.

	Este proceso se repite varias veces, quedándose con la transformación que
	produjo el menor error de alineación.

	Mediante este método, se obtuvieron buenos resultados en la mayoría de las
	capturas de `happy' %enf
	donde los ángulos eran cercanos a 25. %grados 
	Sin embargo, algunas alineaciones presentaban problemas de deslizamiento.
	En el caso de `bunny', cuyos ángulos eran cercanos a 45, %grados
	los resultados no fueron satisfactorios, presentándose confusión en las
	correspondencias del cuerpo e ignorando completamente las orejas.


	%El que tengo ahora
	Se utilizan todos los puntos de la nube.

	En cada punto se calcula un descriptor FPFH que se corresponderá con el más
	cercano utilizando una distancia de chi cuadrado.

	Luego se procede al rechazo de correspondencias erróneas.
	Por cada punto se calcula el marco de referencia provisto por el algoritmo de ISS. %citar
	Este marco de referencia nos permite calcular la transformación de
	alineación por cada correspondencia.
	Entonces, utilizando las suposiciones de ubicación de la cámara en la
	obtención de las capturas, se descartan aquellas correspondencias que
	requieren un movimiento en $y$ excesivo o una rotación por un eje no
	vertical. 

	Cada correspondencia entonces define un ángulo de giro $\theta$ sobre el
	eje $y$ y una translación en el plano $xz$.
	Se observará entonces una agrupación de los parámetros de estas
	transformaciones, mediante el algoritmo de k-means se buscará el centroide
	del clúster más grande.

	Las pruebas sobre `happy' y `bunny' dieron una distancia de aproximadamente
	5 %grados
	comparados al \emph{ground truth}.


	\section{Refinamiento}
	Una vez obtenida la alineación inicial, se procedió a realizar una segunda
	alineación mediante ICP.
	Se consideraron sólo las áreas solapadas y se restringió el
	espacio de búsqueda de las correspondencias.

	Luego, para reducir el error propagado por cada alineación, se propuso una
	corrección de bucle.
	Se ajustó la última captura para que correspondiese con la primera, y
	esta transformación se agregó a las otras alineaciones con un peso
	proporcional a su posición en el bucle.


	%diagramas de clase

	%informes de prueba

	%casos de uso/funcionalidades resueltas
