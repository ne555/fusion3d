\chapter{Módulo de registración}
	%¿Qué es la registración?
	Dado un conjunto de nubes de puntos correspondientes a distintas vistas de
	un objeto, la registración consiste en calcular las transformaciones de
	rotación y translación que lleven a cada vista a un sistema global de forma
	que las zonas comunes encajen perfectamente.

	Los métodos de registración seguirán los siguientes pasos:
	\begin{enumerate}
		\item Selección de puntos de la entrada (\emph{keypoints}).
		\item Cálculo de descriptores y determinación de correspondencias.
		\item Rechazo de correspondencias.
		\item Alineación.
	\end{enumerate}
	Variaciones en estos pasos permiten la implementación de diversos métodos.

		%pseudocódigo
		%ref{FPFH}
	Como medida de reducción de ruido y para evitar considerar puntos outliers
	durante la alineación, se realizó un preproceso de las nubes de entrada.
	%ver que suavice y que se llame así
	Primeramente se ajustaron los puntos a una superficie estimada mediante el
	método de mínimos cuadrados móviles y luego se descartaron puntos de poca
	confianza, como ser aquellos cerca de los bordes o cuyas normales sean
	ortogonales a la cámara.

	\section{Keypoints}
		ISS:
		%descripción
		Por cada punto $p$ de la nube se calcula la matriz de covarianza con
		todos los puntos que caen dentro de una esfera de radio $r$ centrada en
		$p$.
		Un punto es detectado como keypoint si los eigenvalores de esta matriz
		de covarianza son suficientemente disímiles y además presenta la mayor disimiluted entre sus vecinos.

		La clase pcl::ISSKeypoint3D permite identificar estos keypoints, permitiendo definir el radio de la esfera y el nivel de disimilitud.
		%ver bien el problema
		Sin embargo, no pudieron encontrarse los parámetros adecuados.
		Se obtuvieron muy pocos keypoints, imposibilitando la detección de outliers,
		y además estos no resultaban representativos ya que sus correspondientes en la otra nube se encontraban demasiado alejados.

		Multiscale persistence:
		%descripción
		Por cada punto de la nube se calcula su descriptor para distintos tamaños de vecindad (escala).
		A partir de todos los descriptores en todas las escalas se estima una distribución gaussiana que los aproxime.
		Los keypoints quedan definidos como aquellos puntos cuyos descriptores se encuentran alejados de la media.

		Se utilizó Fast Point Feature Histograms (FPFH) como descriptor debido a su bajo costo computacional.

		all:
		Se realiza un submuestreo de los puntos de la nube simplemente para
		reducir el costo computacional.
		Nuevamente, y por las mismas razones, el descriptor elegido fue FPFH.
		El algoritmo de rechazo de correspondencias se encargará de eliminar outliers y asegurar la convergencia.

	\section{Alineación}
	En el caso de los keypoints de multiscale persistence se utilizó el
	algoritmo de sample consensus initial alignment (SAC-IA) para realizar la
	primera alineación.

	Este algoritmo selecciona de forma aleatoria keypoints de la nube, busca
	puntos cercanos en la otra nube, seleccionando nuevamente de forma
	aleatoria, y calcula la transformación que los alinee, obteniéndose una
	medida del error de la transformación.

	Este proceso se repite varias veces, quedándose con la transformación que
	produjo el menor error de alineación.

	Mediante este método, se obtuvieron buenos resultados en la mayoría de las
	capturas de `happy' %enf
	donde los ángulos eran cercanos a 25. %grados 
	Sin embargo, algunas alineaciones presentaban problemas de deslizamiento.
	En el caso de `bunny', cuyos ángulos eran cercanos a 45, %grados
	los resultados no fueron satisfactorios, presentándose confusión en las
	correspondencias del cuerpo e ignorando completamente las orejas.


	%El que tengo ahora
	Se utilizan todos los puntos de la nube.

	En cada punto se calcula un descriptor FPFH que se corresponderá con el más
	cercano utilizando una distancia de chi cuadrado.

	Luego se procede al rechazo de correspondencias erróneas.
	Por cada punto se calcula el marco de referencia provisto por el algoritmo de ISS. %citar
	Este marco de referencia nos permite calcular la transformación de
	alineación por cada correspondencia.
	Entonces, utilizando las suposiciones de ubicación de la cámara en la
	obtención de las capturas, se descartan aquellas correspondencias que
	requieren un movimiento en $y$ excesivo o una rotación por un eje no
	vertical. 

	Cada correspondencia entonces define un ángulo de giro $\theta$ sobre el
	eje $y$ y una translación en el plano $xz$.
	Se observará entonces una agrupación de los parámetros de estas
	transformaciones, mediante el algoritmo de k-means se buscará el centroide
	del clúster más grande.

	Las pruebas sobre `happy' y `bunny' dieron una distancia de aproximadamente
	5 %grados
	comparados al \emph{ground truth}.


	\section{Refinamiento}
	Una vez obtenida la alineación inicial, se procedió a realizar una segunda
	alineación mediante ICP.
	Se consideraron sólo las áreas solapadas y se restringió el
	espacio de búsqueda de las correspondencias.

	Luego, para reducir el error propagado por cada alineación, se propuso una
	corrección de bucle.
	Se ajustó la última captura para que correspondiese con la primera, y
	esta transformación se agregó a las otras alineaciones con un peso
	proporcional a su posición en el bucle.


	%diagramas de clase

	%informes de prueba

	%casos de uso/funcionalidades resueltas
