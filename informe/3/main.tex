\documentclass{pfc}
\usepackage{algorithmicx}
\usepackage{algorithm}
\usepackage[noend]{algpseudocode}

\makeatletter
 \renewcommand{\ALG@name}{Algoritmo}
\makeatother
\renewcommand{\algorithmicfunction}{}

\title{Integración y pruebas}

\begin{document}
	\maketitle
	\section{Introducción}

	En esta etapa se integraron los módulos de registración, fusión y relleno de huecos.

	Se realizaron pruebas de rendimiento utilizando los modelos de la base de datos Stanford.

	Se resolvieron todas las funcionalidades propuestas.

	\section{Integración}
	\subsection{Formato de nube utilizado en cada módulo}
		Para la registración fue conveniente, en un primer momento, mantener
		separadas las normales de las posiciones, por lo que al probar diversas
		alternativas se siguió manteniendo esta forma.
		Sin embargo, el ajuste por ICP, la fusión y el relleno de huecos
		requerían que esta información estuviera junta para cada punto.

		Para el momento en que se comenzó con la fusión, modificar la
		registración era prohibitivamente costoso, ya que no se podía destinar nada de tiempo a la tarea.
		Además, ya se contaban con los resultados de esa etapa (las transformaciones de las vistas).
		Por eso se idearon funciones de conversión entre esos formatos en lugar de modificar la registración.

		Se planea una posterior modificación de la registración de forma que el
		formato utilizado transparente para el usuario.

	
	\subsection{Pasaje de registración de a pares a toda la lista}
		El algoritmo de alineación inicial funciona mediante pares de capturas,
		retornando entonces una transformación parcial entre las vistas.
		El ajuste por ICP también trabaja de a pares, por lo que no requiere aplicar la transformación.

		Si bien la corrección por bucle requiere de las transformaciones
		totales, el algoritmo ignora las nubes de puntos, por lo que no es
		necesario aplicar la transformación.

		Por esto, no es necesario aplicar la transformación hasta que el módulo de fusión lo requiera.
		Las capturas pueden mantenerse en su marco original, 
		donde $z$ es la distancia a la cámara (considerando perspectiva),
		si esto representa una ventaja para el usuario

	\subsection{Relleno de huecos}
		El resultado del módulo de fusión es una nube de puntos con su correspondiente triangulación.
		Sin embargo, el rellenado de huecos mediante \emph{advancing front}
		descarta todos aquellos puntos que no pertenezcan
		a la mayor componente conectada en la triangulación.

		Esto se realiza para simplificar la implementación del método al no
		tener que considerar posibles islas.
		Una versión futura podría evitar esta pérdida de información.

		En el caso del método de Poisson, la triangulación es completamente ignorada.



	\section{Funcionalidades}
		Respondiendo a los requerimientos identificados previamente,
		se implementaron las siguientes funcionalidades

		\begin{itemize}

			\item {Outliers}
	%{Se debe disponer de funciones para la detección y eliminación de puntos considerados outliers.}
		El ruido y los puntos extremos se consideran en varias etapas de la reconstrucción.
		Durante el preproceso, una triangulación Delaunay identifica y elimina puntos extremos.
		Durante la fusión, los puntos sin confirmación y poca confianza se consideran provenientes de ruido.
		Luego, al utilizar únicamente la componente de mayor tamaño, se eliminan pequeños grupos de puntos.

	{Registración Inicial}
	{Se debe disponer de funciones que dadas dos mallas calculen una
	transformación que las acerque lo suficiente como para poder
	utilizar luego ICP.}
		Se proveen dos métodos para la alineación inicial en el módulo de registración.

	{Área solapada}
	{Se debe disponer de funciones que establezcan los puntos en común (o una buena
	aproximación) entre dos mallas ya alineadas burdamente.}
		Los puntos en común se definieron como aquellos que están a menos de un umbral de distancia
		al punto más cercano en la otra nube.

	{Métricas}
	{Se debe disponer de funciones para evaluar la calidad de una registración.}
		Cómo métricas se utilizaron la distancia entre las nubes considerando sólo los puntos en común,
		y la razón de tamaños entre la nube de entrada y los puntos en común.

	{Corrección de bucle}
	{Se debe disponer de funciones para corregir el error propagado durante la registración
	una vez que se haya realizado una vuelta con las capturas.}
		El error de alineación del bucle se define como la transformación total
		calculada por el par primera-última al completar la vuelta. Debido a que `primera`
		definió el marco de referencia, si no se tuviera error se obtendría la identidad.
		Calculando la inversa de esta transformación se obtiene la corrección que será distribuida
		a las otras registraciones.

	{Combinación de nubes}
	{Se debe disponer de funciones para ajustar los puntos y sus normales,
	según la información provista por cada malla.}
		El módulo de fusión se encarga de esto, utilizando una representación de surfel con visibilidad, confianza y vecindad.

	{Triangulación}
	{Se debe disponer de funciones para obtener una triangulación dada una nube de puntos tridimensional.}
		La triangulación se realiza mediante el método de Greedy Projection Triangulation provisto por PCL.

	{Relleno}
	{Se debe disponer de funciones para lograr que una triangulación sea cerrada. Se
	estimará una superficie en las zonas donde se carezca de
	información.}
		El módulo de relleno de huecos da respuesta mediante los algoritmos de advancing front y
		la llamada a PCL::Poisson.
		\end{itemize}







	%casos de uso y requerimientos, cómo se los solucionó
	\CasoUso{Preprocesar nube}
		Función para reducir el ruido mediante mínimos cuadrados móviles.

	\CasoUso{Alinear dos nubes de puntos}
		Una registración inicial, seguida por una corrección mediante ICP.

	\CasoUso{Extraer superficie}
		La superficie es una malla triangulada. La triangulación se obtiene por Greedy Projection Triangulation, o luego de la reconstrucción de Poisson (que la provee).


	\CasoUso{Identificar huecos}
		La malla se almacena mediante una estructura de media arista. Los
		huecos se identifican como aquellas aristas que inciden en sólo una
		cara.
		Se hizo uso de PCL::getBoundBoundary transformando su resultado para
		obtener los puntos del contorno del hueco y ordenarlos según tamaño
		(cantidad de puntos).



	\Requerimiento{Tiempo de ejecución}
	{No se espera una ejecución a tiempo real de los algoritmos implementados.}
		El grueso del proceso se encuentra en la registración.
		Los tiempos de ejecución varían según la cantidad de puntos de la nube.
		%¿lineal? no creo,  ver si se puede estimar el orden

	\Requerimiento{Sistemas operativos}
	{El producto desarrollado estará destinado a utilizarse en los sistemas
	operativos Windows y Linux.}
		Esto se logró al utilizar bibliotecas multiplataforma como PCL,
		o aquellas que se restringen a las bibliotecas estándar de C++ como Delaunator (triangulación Delaunay) y DKM (k-means).

	\bibliographystyle{alpha}
	\bibliography{biblio}
\end{document}

