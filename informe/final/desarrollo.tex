\chapter{\label{part:desarrollo}Desarrollo}
El proceso de reconstrucción es lineal:
se elimina el ruido de las capturas parciales del objeto,
se las alinea en un marco de referencia global,
se las combina para obtener una sola malla total
y se rellenan los huecos de esta malla para obtener una superficie cerrada
que represente al objeto (figura~\ref{fig:proceso_de_reconstrucción}).

\begin{figure}[h]
	\centering
	\begin{scaletikzpicturetowidth}{\linewidth}
	\begin{tikzpicture}[node distance=2em and 2em]
	\tikzstyle{modulo} = [align=center,rectangle, minimum width=1em, minimum height=1em, text centered, draw=black]
	\tikzstyle{terminal} = [align=center,rectangle, minimum width=1em, minimum height=1em, text centered, draw=white]
	\tikzstyle{arrow} = [thick,->,>=stealth]

	\node (captura) [terminal] {Capturas};
	\node (preproceso) [modulo, right= of captura] {Preproceso};
	\node (registracion) [modulo, right=of preproceso] {Registración};
	\node (fusion) [modulo, below=of registracion] {Fusión};
	\node (relleno) [modulo, right=of fusion] {Rellenado\\de huecos};
	\node (superficie) [terminal, right=of relleno] {Superficie\\cerrada};

	\draw [arrow] (captura) -- (preproceso);
	\draw [arrow] (preproceso) -- (registracion);
	\draw [arrow] (registracion) -- (fusion);
	\draw [arrow] (fusion) -- (relleno);
	\draw [arrow] (relleno) -- (superficie);
\end{tikzpicture}

	\end{scaletikzpicturetowidth}
	\caption[Proceso de reconstrucción tridimensional]{\label{fig:proceso_de_reconstrucción}Proceso de reconstrucción tridimensional.}
\end{figure}

A continuación se desarrollan los detalles de implementación de cada uno de estos procesos.


\section{Módulo de preproceso}
%Escáners de luz estructurada
%Se proyecta en patrón de luz específico hacia el objeto.
%Este patrón es observado por el sensor.
%Se determinan las posiciones de los puntos mediante la intersección entre la
%dirección de la luz y la del sensor.
Este módulo se encargará de realizar una reducción de ruido a las nubes de entrada
antes de ser procesadas por las siguientes etapas,
por lo cual es necesario considerar
qué características posee el ruido y cómo se produce.
En los escáneres de luz estructurada dos fuentes de error son particularmente relevantes:
\begin{itemize}
	\item \emph{Ángulo de incidencia excesivo:\/}
		La luz proyectada impacta en una porción de la superficie del objeto
		que es casi paralela a su dirección.
		El sensor entonces capta una versión estirada y con menor intensidad del patrón, lo que 
		agrega incertidumbre en la posición de los puntos.
	\item \emph{Reflejo parcial:\/}
		Cuando, dada una línea del patrón, solamente una porción de esta incide en el objeto (figura~\ref{fig:error_adquisicion}).
		Se obtiene una posición incorrecta, ya que el método de triangulación
		supone que todo el ancho de la línea impactó en el objeto.
		Esto resulta en bordes distorsionados y en posiciones más alejadas que las reales.\cite{Turk:1994:ZPM:192161.192241}
\end{itemize}
Además, se cuentan con pequeñas variaciones en los puntos debidas a la sensibilidad del sensor.

\begin{figure}
	\Imagen{diagram/error_adquisición_borde}
	\caption{\label{fig:error_adquisicion}Error en la posición estimada debido a un reflejo parcial del patrón de luz.}
\end{figure}

Teniendo en cuenta estas consideraciones, se desarrolló el siguiente algoritmo
para el preproceso de las nubes de entrada:
\begin{itemize}
	\item Para independizarse de la escala de las capturas,
todos los parámetros que impliquen una distancia o vecindad
se establecen en relación a una medida de resolución de las nubes,
la cual se definió como el promedio de las distancias entre cada punto
contra su par más cercano.
Si bien pueden presentarse zonas de mayor densidad de puntos que otras,
surgen problemas para definir la localidad de estos agrupamientos,
por lo que se optó por una medida global.

\item Para reducir las pequeñas perturbaciones en las posiciones de los puntos,
se los proyectó a una superficie estimada mediante el método de mínimos cuadrados móviles.
Para construir esta superficie se utilizó la clase
\texttt{pcl::Moving\-Least\-Squares}, definiendo una vecindad de seis veces la
resolución de la nube con el fin de obtener aproximadamente 100 puntos para
realizar la estimación.
Los resultados de este proceso se evaluaron visualmente, pudiendo observarse la pérdida de detalles
al aumentar en exceso la vecindad (figura~\ref{fig:mls}). \TODO{ver imágenes}
\begin{figure}
	\caption{\label{fig:mls}\TODO{gráfico de curvatura antes y después de mls, distintas vecindades}}
\end{figure}

\item En cuanto a los errores debidos al método de adquisición, se decidió descartar estos puntos:
	\begin{itemize}
		\item Se eliminaron los puntos cuyas normales se encontraban a más de $80^{\circ}$ del eje $z$.
		\item Debido a que a partir de una triangulación es trivial obtener los puntos que pertenecen al borde
se proyectó la nube en el plano $z=0$ para realizar una triangulación Delaunay y así
trasladar estas conexiones a los puntos en el espacio.
		\item Se procedió a eliminar aquellas aristas cuya longitud superaba un umbral (establecido en tres veces la resolución de la nube).
		\item Se eliminaron los puntos que resultaron aislados y aquellos que delimitaban bordes.
	\end{itemize}
\end{itemize}

Si bien al finalizar esta etapa se obtiene una triangulación de la nube de entrada,
los algoritmos implementados en PCL no están diseñados para considerar las conectividades y trabajan únicamente con las vecindades.

\section{Registración}
Una vez finalizada la etapa de adquisición, se dispone de una nube de puntos
que representa la porción observada del objeto.
Para lograr una reconstrucción total es necesario combinar múltiples capturas
variando la posición relativa cámara-objeto.
Contar con un dispositivo que nos permita definir con precisión
tanto la posición como orientación de la cámara es altamente costoso,
por esta razón, en la etapa de registración se deben estimar las transformaciones que ubiquen cada
captura en un sistema de referencia global, de forma que las zonas comunes encajen perfectamente.


Esta operación se realiza usualmente entre dos capturas,
buscando las transformaciones de translación y rotación que ubiquen
la captura de partida en el marco de referencia de la captura objetivo.
De esta manera, el problema general cuenta con seis grados de libertad.
A continuación se presenta el caso en que
la transformación de alineación es pequeña,
más adelante se discute el caso de incrementos mayores,
y finalmente se extiende el proceso para abarcar $n$ capturas.

%esto va en $N$ capturas
%Para extender este proceso a $n$ capturas, puede alinearse $C_0$ con $C_1$,
%$C_1$ con $C_2$, $C_2$ con $C_3$, etc.,
%sin embargo, el error de registración se propagará en cada paso.


\subsection{Perturbaciones pequeñas: algoritmo iterativo del punto más cercano (ICP)}
Antes de describir este algoritmo, se planteará una versión simplificada del problema
que permitirá establecer ciertas definiciones.

Se supone que se dispone de una nube $A$ a la cual se le aplica
una transformación de translación y rotación arbitraria,
y luego se perturba levemente
las posiciones de los puntos transformados,
obteniéndose la nube $B$.
Se observa que ambas nubes poseen la misma cardinalidad y se pueden establecer las relaciones de origen a destino
$a_j \to b_j$ para cada punto $a_j \in A$.
Entonces, se puede definir el error de alineación como:
\[
	\text{Error} = \frac{1}{N} \sum_{j=1}^n || b_j - T \left(a_j\right) ||
\]
y buscar la transformación $T$ que minimice este error.

Existe una solución cerrada para este problema, obteniéndose la rotación al
calcular los eigenvectores de una matriz simétrica de $4\times4$,
cuyos detalles pueden consultarse en \cite{Horn87closed-formsolution}.

Sin embargo, estas suposiciones son demasiado fuertes.
Al trabajar con dos capturas realizadas desde distintas vistas, el solapamiento no será total,
ya que habrá puntos que serán observables en tan sólo una de las vistas.
Por lo tanto, primero es necesario determinar cuáles son los puntos comunes a ambas capturas,
y, además, definir las relaciones de origen-destino entre esos puntos comunes.
El algoritmo iterativo del punto más cercano (ICP) supone que las nubes se encuentran
lo suficientemente cerca como para establecer estas correspondencias mediante las coordenadas de los puntos.
El algoritmo se describe de la siguiente manera:
\begin{enumerate}
	\item Obtener las correspondencias:
		Para cada punto $a_j \in A$ buscar el punto más cercano en la nube $B$.
		Si la distancia entre esos puntos supera un umbral $d_{\text{max}}$,
		se considera que $a_j$ no pertenece a la zona común y no es considerado en esta iteración.
	\item Calcular la transformación que alinee los pares de puntos de la zona común.
	\item Aplicar la transformación.
	\item Repetir el proceso hasta que el error de alineación esté por debajo de un umbral $\tau$.
\end{enumerate}%\cite{conf/rss/SegalHT09}
Si bien el algoritmo converge a un mínimo local, puede que este no sea el mínimo global buscado.
Debe cumplirse la suposición de cercanía para obtener una correcta registración.\cite{regBesl92}

%generalized conf/rss/SegalHT09
En \cite{chen-medoni} se presenta una variante del algoritmo de ICP, llamada ICP {punto-a-plano},
que resulta más robusta y precisa al trabajar con nubes de puntos en 2.5D.
Una vez establecida una correspondencia $a_j \to b_j$, si bien no se trata del mismo punto,
se realiza la suposición de que pertenecen al mismo plano.
Por esta razón, podemos decir que conocemos con mucha confianza la posición del punto
a lo largo de su normal, pero no su ubicación en el plano.
Para reflejar esto, se cambia la métrica del error, proyectando el punto transformado sobre la normal
del punto destino (figura~\ref{fig:point_to_plane}).
\[
	\text{Error} = \frac{1}{N} \sum_{j=1}^n || n_j \cdot \left( b_j - T \left(a_j\right) \right) ||
\]

\begin{figure}
	\centering
	\input{diagram/icp_point_to_plane.pdf_tex}
	\caption[Medidas de error punto-a-plano entre dos superficies]{\label{fig:point_to_plane}Medidas de error punto-a-plano entre dos superficies
	(Imagen cortesía de \cite{icp_point_to_plane}).}
\end{figure}


\subsection{Distancias mayores: búsqueda de correspondencias mediante vectores de características}
Cuando la transformación de alineación requerida para alinear las dos nubes de puntos
no es lo suficientemente pequeña, no se obtiene una buena aproximación al establecer las correspondencias
utilizando únicamente las posiciones de los puntos.
Una posibilidad es emplear métodos semiautomáticos, donde el usuario provee de una alineación inicial
de las nubes o determina las correspondencias seleccionando puntos.
Sin embargo, este es un método lento y engorroso,
por lo que surge la necesidad de plantear un algoritmo completamente automático.

Para hallar las correspondencias de manera automática, se calcula un vector de características o \emph{descriptor}
sobre una vecindad de cada punto. Al comparar los descriptores con alguna medida de distancia,
si la distancia es pequeña (cercana a $0$), entonces las vecindades son similares
y podría plantearse una correspondencia entre los puntos.
Un buen descriptor deberá poseer las siguientes características:
\begin{itemize}
	\item Ser discriminante, es decir, la distancia entre los descriptores debe reflejar
		la distancia entre las superficies de las vecindades.
		Esto implica que superficies similares localmente producen descriptores parecidos, y
		en superficies diferentes la distancia entre descriptores será grande (figura~\ref{fig:descriptor_matriz_confusion}).
	\item Ser invariante respecto a translaciones y rotaciones.
	\item Ser robusto respecto al ruido de muestreo.
	\item Ser robusto respecto a la densidad de muestreo.
	\item Ser robusto respecto a la ausencia de muestras debido a oclusiones.
\end{itemize}

\begin{figure}
	\centering
	\input{diagram/fpfh_confusion.pdf_tex}
	\caption[Matriz de confusión del descriptor FPFH]{\label{fig:descriptor_matriz_confusion}Matriz de confusión del descriptor FPFH para distintas clases de superficies utilizando la distancia $\chi^2$\RefImagen{RusuDoctoralDissertation}.}
\end{figure}

Además de la elección del descriptor, un parámetro importante a determinar es
el tamaño de la vecindad que éste representará.
En general, la vecindad de un punto $p$ incluye a todos aquellos puntos $q$
que se encuentran dentro de una esfera de radio $r$ centrada en $p$:
\[ |p - q| \leq r \]
Si el valor de $r$ es demasiado pequeño, el descriptor perderá su poder discriminante
o incluso no podrá calcularse al no contar con la cantidad de vecinos mínimos necesarios.
% FIXME:
%sobrecarga significado «superficie»
%el riesgo al incrementar la vecindad es que sea «otra cara» de la superficie
%no sé, hablar de continuidad o de localidad
En cambio, si el valor de $r$ es demasiado grande, se corre el riesgo de considerar
dentro de la vecindad puntos pertenecientes a otra superficie, distorsionando el valor del descriptor.
La determinación de un valor adecuado de $r$ limita la posibilidad de obtener un método completamente
automático para el cálculo de los descriptores\cite{RusuDoctoralDissertation}.
 %This issue is of extreme importance and constitutes a limiting factor in the automatic estimation (i.e., without user given thresholds) of a point feature representation.


A continuación se describe el proceso de construcción de dos descriptores
que han mostrado buenos resultados en problemas de registración\cite{Rusu:2009:FPF:1703435.1703733}.

\subsubsection{Estimación de normales}
Para poder describir la geometría de una superficie,
primero debe estimarse su orientación en el sistema de coordenadas, es decir, estimar su normal.
Una opción es aproximar la normal mediante la estimación de un plano tangente a la superficie,
lo que transforma el problema en ajustar un plano a la vecindad de $p$
mediante el método de mínimos cuadrados.
Representando el plano mediante un punto $x_j$ y un vector normal $n_j$,
la distancia de un punto $q$ de la vecindad a este plano queda definida como
$d = (q_j - x_j) n_j$.
Para minimizar la sumatoria del cuadrado de todas las distancias, se tiene que:
\[x_j = \frac{1}{N} \sum_{k=1}^{N} q^{(k)}_j \]
para los $N$ puntos de la vecindad de $p$.
Y para obtener $n_j$ se analizan los eigenvalores y eigenvectores de la matriz de covarianza de la vecindad:
\[ C_{jk} = \sum_{r=1}{N} (q^{(r)}_j - x_j) (q^{(r)}_k - x_k) \]

Se observa que $C_{jk}$ es una matriz simétrica y por lo tanto todos sus eigenvalores son reales
$\lambda_j \in \Real$.
El eigenvector que corresponde al eigenvalor de menor magnitud corresponde a la normal del plano tangente,
es decir a $+n_j$ o a $-n_j$\cite{10.1109/34.334391}.
Si bien no puede determinarse matemáticamente el signo de $n_j$,
puede resolverse la ambigüedad al considerar que la nube de puntos que se obtiene del proceso de adquisición
es 2.5D y se conoce la ubicación de la cámara $v_j$ en un sistema de referencia local.
Por lo tanto, debe cumplirse que:
\[n_j v_j > 1\]
invirtiendo el signo de $n_j$ en caso de ser necesario.

\subsubsection{Histograma de características del punto (PFH)}
Dado que las normales no son invariantes a rotaciones y no capturan mucho
detalle de la superficie, no presentan un gran poder discriminante.
Por esta razón, no resulta adecuado utilizarlas como único descriptor.
Sin embargo, al representar las relaciones entre todos los puntos de la vecindad y sus normales,
pueden capturarse con más detalle las variaciones de la superficie,
aumentando la representatividad del descriptor.
Esta es la idea principal del histograma de características del punto (PFH) presentado en \cite{RusuDoctoralDissertation}.

El vector de características de PFH es un histograma de los valores de tres ángulos
y una distancia calculados para cada par de puntos y sus normales en la vecindad de $p$.
Para asegurar la replicabilidad del descriptor, se define un marco de referencia de Darboux
%FIXME: dónde poner la referencia
(figura~\ref{fig:pfh_marco_referencia})
determinado unívocamente de la siguiente manera:
\begin{itemize}
	\item Dado el par de puntos $p_j$ y $p_k$, se determina el origen $p_s$ del marco de referencia como:
		\[
			\vec{n}_j \cdot (p_k - p_j) > \vec{n}_k \cdot (p_j - p_k)
			\begin{cases}
				\text{Verdadero: }& p_s = p_j \quad p_t = p_k\\
				\text{Falso: }& p_s = p_k \quad p_t = p_j\\
			\end{cases}
		\]
	\item A partir de este origen, se determinan los tres vectores que definen el marco de referencia:
\[
	\begin{cases}
		u =& n_s \\
		v =& u \times \displaystyle\frac{p_t - p_s}{|p_t - p_s|} \\
		w =& u \times v \\
	\end{cases}
\]
\end{itemize}
Y entonces se procede a calcular las siguientes relaciones para estos puntos:
	\begin{align*}
		d        &= |p_t - p_s| \\
		\alpha   &= v \cdot n_t \\
		\phi     &= u \cdot \frac{p_t - p_s}{d}\\
		\theta   &= \atan(w \cdot n_t, u \cdot n_t)
	\end{align*}
Sin embargo, la distancia entre puntos $d$ no es de importancia para escaneos 2.5D,
ya que la distancia entre vecinos se incrementa con la distancia a la cámara,
y suele ser beneficioso omitirla en estos casos \cite{RusuDoctoralDissertation}.

%propio
Para entender el poder discriminante del descriptor PFH, es preciso comprender el significado
de los ángulos calculados entre los pares de puntos (ver figura~\ref{fig:pfh_angulos}).
Considerando el plano tangente a $p_s$, el ángulo $\phi$ mide la separación de la posición de $p_t$ respecto a este plano.
Luego, considerando el plano $wu$, donde se encuentran $p_s$, $p_t$ y $n_s$, y cuya normal es $v$:
\begin{itemize}
	\item el ángulo $\alpha$ mide el desvío de la normal $n_t$ respecto a este plano,
	\item el ángulo $\theta$ mide la distancia entre las proyecciones de las normales sobre este plano.
\end{itemize}
De esta manera, se logra describir con buen detalle las características de la superficie en la vecindad de cada punto.



\begin{figure}
	\centering
	\input{diagram/marco_ref_fpfh.pdf_tex}
	\caption[Representación gráfica del marco de Darboux]{\label{fig:pfh_marco_referencia}Representación gráfica del marco de Darboux y las características calculadas por el descriptor PFH \RefImagen{RusuDoctoralDissertation}.}
\end{figure}

\begin{figure}
	\centering
	\input{diagram/fpfh_meaning.pdf_tex}
	\caption[Representación gráfica de los ángulos $\phi$ y $\theta$ calculados por el descriptor PFH]{\label{fig:pfh_angulos}Representación gráfica de los ángulos $\phi$ y $\theta$ calculados por el descriptor PFH.}
\end{figure}

\subsubsection{PFH rápido (FPFH)}
Debido a que para obtener el descriptor PFH de un punto es necesario calcular
las relaciones entre todos los pares de puntos de su vecindad,
el orden de ejecución para una nube con $n$ puntos es $\bigO(nk^2)$,
donde $k$ es la cantidad de puntos en la vecindad,
y por esta razón el cálculo de este descriptor puede convertirse
en uno de los cuellos de botella del proceso.

Para solventar este problema, en \cite{Rusu:2009:FPF:1703435.1703733} se propone
una simplificación del descriptor PFH, llamada PFH rápido o simplemente FPFH,
que nos permite reducir el tiempo de ejecución a $\bigO(nk)$
manteniendo el poder descriptivo de PFH.

El cálculo de este nuevo descriptor se realiza de la siguiente manera:
\begin{enumerate}
	\item Calcular por cada punto $p$ y sus vecinos las tuplas $\langle \alpha, \phi, \theta \rangle$.
		Llamamos a estas características PFH simplificado (SPFH).
	\item Realizar un promedio ponderado de los valores del SPFH de $p$ y sus vecinos:
		\[
			FPFH(p) = SPFH(p) + \frac{1}{N} \sum_{j=1}^{N} \frac{1}{w_j} SPFH(p_j)
		\]
		donde $w_j$ es una medida de distancia entre $p$ y su vecino.
\end{enumerate}
%FIXME: fpfh para un punto toma sus vecinos, pfh coma todos los pares
% no quedó bien explicado
Las principales diferencias respecto a PFH son:
\begin{itemize}
	\item FPFH no interconecta todos los vecinos de $p$.
	\item FPFH puede incluir puntos fuera del radio $r$ de vecindad, pero a no más de $2r$.
\end{itemize}


\subsubsection{Puntos salientes (\emph{keypoints})}
Al describir las características de un buen descriptor, mencionamos la capacidad discriminante del mismo.
Se presenta una gran diferencia entre descriptores de superficies diferentes,
pero la distancia es cercana a $0$ en superficies parecidas,
y de esta forma se pueden establecer las correspondencias entre los puntos de dos nubes
al comparar sus descriptores.

Al analizar una zona homogénea en una nube, como, por ejemplo, un plano,
todos los puntos presentarán descriptores parecidos.
Surgen problemas al intentar establecer una correspondencia entre estos puntos
y los presentes en la otra nube, ya que todos son candidatos igualmente válidos.

Para solucionar este problema, es necesario identificar puntos salientes (\emph{keypoints})
en la nube, para entonces establecer las correspondencias entre los descriptores de estos keypoints.
Si bien un punto cualquiera será considerado como saliente o no dependiendo del detector utilizado,
un buen detector presentará las siguientes características:
\begin{itemize}
	\item Dispersión: un subconjunto pequeño de los puntos de la nube serán considerados como keypoints.
	\item Repetibilidad: un keypoint deberá ser detectado en la misma ubicación
		sobre el objeto sin importar la posición de la cámara al realizar la captura.
	\item Distinguibilidad: la superficie en la vecindad del keypoint
		presentará características únicas que serán capturadas por un descriptor.
\end{itemize}

Como ejemplos de detectores se puede mencionar el detector de esquinas Harris,
utilizado usualmente en imágenes 2D, adaptado al espacio tridimensional
traduciendo cambios en la intensidad de píxeles por ángulos entre las normales de puntos vecinos;
y el detector ISS, que analiza los eigenvalores de la matriz de covarianza en la vecindad de un punto.


\subsection{Registración de múltiples capturas}
Previamente se describieron métodos de registración entre dos capturas,
sin embargo, un objeto difícilmente pueda ser observable completamente desde tan sólo dos posiciones.
Por lo tanto, para lograr la reconstrucción tridimensional se tendrá una lista de capturas
${A_1, A_2, \ldots, A_n}$ correspondientes a distintas posiciones relativas cámara-objeto.
Surge entonces el problema de cómo obtener la lista de transformaciones ${T_1, T_2, \ldots, T_n}$
que las alinee a todas correctamente.

Una opción es utilizar una captura $B$ que presente zonas comunes con todas las capturas $A_j$
y entonces calcular la registración $A_j \to B$.
Esta captura $B$ puede obtenerse mediante un escaneo cilíndrico: el objeto se
ubica sobre una superficie que gira lentamente a una velocidad controlada
mientras que se proyecta una línea vertical sobre el objeto y se miden las
distancias de los puntos que caen sobre esta línea.

Una alternativa es realizar la registración entre pares sucesivos
$A_2 \to A_1$,
$A_3 \to A_2$,
\ldots,
$A_{n} \to A_{n-1}$.
Sin embargo, los errores producidos se propagarán con cada
captura incorporada, siendo especialmente apreciables al completar una
vuelta alrededor del objeto (figura~\ref{fig:error_bucle}).
Para corregir este error, se ajustará la alineación de la captura que cierre el bucle,
propagando luego esta corrección a las demás.


	\begin{figure}
		\Imagen{diagram/error_bucle_inhand}
		\caption[Visualización del error de bucle]{\label{fig:error_bucle}Visualización del error de bucle. Errores de tan sólo $1^{\circ}$
		en cada registración producen una discrepancia considerable
		donde el modelo debería cerrarse (círculo rojo) \RefImagen{5457479}.}
	\end{figure}

\section{Módulo de fusión}
	%¿citas?
	%en el de loop correction (surfel)
	El módulo de fusión tiene como objetivo obtener un modelo que
	describa la geometría del objeto escaneado.
	Para esto, el modelo se inicializa a partir de una captura cualquiera
	y se procesan una a una las capturas correspondientes a las otras vistas.
	En cada nueva vista se agrega información del objeto en zonas que no eran antes visibles
	y además se confirma o refuta la información ya presente en el modelo.
	Al combinar esta información, el módulo de fusión obtiene finalmente una malla
	que represente a la totalidad del objeto.

	\subsection{Método}
	Se utiliza una representación de \emph{surfel} para cada punto similar a la propuesta en \cite{5457479} %ref in-hand scanning
	debido a la facilidad de implementación de las funciones de actualización de los puntos.
	Como es necesario integrar la información de posición y normal de cada punto,
	se idearon funciones de conversión del formato de nube usado en el módulo de registración.

	Cada surfel tiene asociado un valor de confianza y las vistas en las que
	fue observado.  El valor de confianza nos indica la probabilidad de que sus
	valores de posición y normal no sean producto de un error de muestreo.
	Este valor se inicializa según el ángulo de su normal respecto a la línea
	de la cámara (figura~\ref{fig:confianza_surfel}).
	%y a su distancia al centro de la captura %terminé usando sólo las normales

	\begin{figure}
		\Imagen{img/bunny_confianza}
		\caption[Visualización de los valores de confianza de los súrfeles]{\label{fig:confianza_surfel}Visualización de los valores de confianza de los súrfeles para la captura \texttt{bun000}.}
	\end{figure}

	El algoritmo~\ref{alg:surfel} describe el agregado de una nueva vista.
	Por cada punto de la vista se busca qué surfel lo contiene y se actualiza
	su posición y normal ponderando según el nivel de confianza.
	\begin{eqnarray*}
		\hat{P}_k = \frac{\sum_{j} \alpha_j P^{(j)}_k}{\sum_{j} \alpha_j} \\
		\hat{n}_k = \frac{\sum_{j} \alpha_j n^{(j)}_k}{\sum_{j} \alpha_j}
	\end{eqnarray*}
	En caso de que haya caído fuera del dominio de la reconstrucción actual, se
	considera que es un nuevo punto y se lo agrega.
	Si la distancia de proximidad elegida es demasiado pequeña,
	nunca se realizará la actualización ya que todos los puntos serán considerados como nuevos surfels.
	En cambio, si la distancia es muy grande, se considerarán puntos
	que deberían haber sido descartados como atípicos.

	\begin{algorithm}
		\begin{algorithmic}[1]
			\Function{Agregar vista}{vista, reconstrucción}
				\ForAll{$p \in \mbox{vista.puntos}$}
					\State surfel $\gets$ reconstrucción.buscar(p)
					\If {surfel = $\emptyset$}
						\State reconstrucción.agregar(p)
					\Else
						\State surfel.actualizar(p)
					\EndIf
				\EndFor
			\EndFunction
		\end{algorithmic}
		\caption[Actualización de la reconstrucción al agregar una nueva vista]{\label{alg:surfel}Actualización de la reconstrucción al agregar una nueva vista.}
	\end{algorithm}

	Al terminar de procesar todas las vistas, habrá surfels que hayan sido
	observados desde sólo una posición y tengan un nivel de confianza bajo.
	Estos son considerados como ruido y eliminados de la
	reconstrucción.

	Finalmente, se procede a la obtención de una triangulación de los surfels.
	Para esto se empleó el algoritmo de \emph{Greedy Projection Triangulation} que provee PCL,
	el cual realiza una proyección de la vecindad de un punto en el plano definido por su normal
	y procede a conectar los puntos de modo que los triángulos resultantes
	respeten umbrales de longitud y ángulos definidos por el usuario (figura~\ref{fig:surface}).

	\begin{figure}
		\Imagen{img/bun_fusion}

		%\Imagen{img/bun_fusion_conf}
		\caption[Superficie reconstruida]{\label{fig:surface}Superficie reconstruida. En verde se destacan los bordes de los huecos.}
	\end{figure}

\section{Módulo de rellenado de huecos}
	Al finalizar el algoritmo de fusión se obtuvo una malla triangular a partir
	de la información proveniente de cada vista.
	Sin embargo, esta malla no es cerrada ya que existen zonas que ninguna
	vista pudo capturar y por lo tanto carecen de puntos, produciendo huecos en la misma.

	El módulo de rellenado de huecos se encargará de estimar, de forma automática, la superficie del
	objeto en estas zonas para así obtener finalmente una malla cerrada.

	Se plantearon dos métodos:
	\begin{itemize}
		\item \emph{Advancing front}, que trabaja localmente con los puntos que forman el contorno de cada hueco.
		\item Reconstrucción de Poisson, que trabaja con todos los puntos de la nube a la vez. 
	\end{itemize}


	\subsection{Advancing front}
		En este método, cada hueco se rellenará de forma independiente a los otros,
		utilizando únicamente los puntos que conforman cada contorno para estimar
		las posiciones de los nuevos puntos.
		Tomando en cuenta esas consideraciones, se diseñó el diagrama de clases presentado en la figura~\ref{fig:filling_class},
		cuyas clases principales se describen a continuación:
		\begin{itemize}
			\item {\bfseries Relleno de huecos:} La clase se encarga de estimar
				nuevos puntos en zonas donde se carece de información (huecos)
				y triangularlos para que la \emph{Malla} sea cerrada.
			\item {\bfseries Borde:} Es una colección de puntos ordenados
				que representa un borde de un hueco en la \emph{Malla}.
		\end{itemize}

		\begin{figure}
			\Imagen{uml/hole_filling}
			\caption{\label{fig:filling_class}Diagrama de clases del módulo de registración}
		\end{figure}

		Es posible que dentro de un hueco se observen puntos que no lograron
		conectarse al resto de la malla.  Si bien estas islas nos brindan
		información de la superficie, su presencia dificulta la identificación
		del contorno de cada hueco.  Por esta razón, se eliminaron todas las
		islas al trabajar únicamente con la componente conectada que contenía
		la mayor cantidad de puntos.
		De esta forma, una arista que defina sólo un triángulo formará parte de un hueco,
		y podrá obtenerse el contorno del mismo recorriendo el grafo de conectividades. 

		\TODO{gráficos}
		El rellenado se implementó mediante una variante del método de
		\emph{advancing front}\cite{advance_front}, descripta en el
		algoritmo~\ref{alg:adv_front}.

		Los nuevos puntos insertados son elegidos de forma que los triángulos resultantes sean
		aproximadamente equiláteros, como se detalla en el algoritmo~\ref{alg:new_point}.
		Estos puntos son luego proyectados en un plano de soporte definido
		mediante las normales de los puntos del ángulo candidato.

		En caso de que el nuevo punto cayese cerca de otro ya existente, se utilizará aquel.
		Esto implica dividir en dos el hueco, y cada nueva porción se rellenará de forma independiente.
		\TODO{gráfico}

		\begin{algorithm}
			\begin{algorithmic}[1]
				\Function{Advancing front}{Malla, Contorno}
					\State AF $\gets$ Contorno
					\Repeat
					\State $\alpha = \widehat{PCN} =$ ángulo mínimo(Contorno)
					\If{$\alpha < 75^{\circ}$}
						\State Malla.agregar triángulo(P, C, N)
						\State AF.eliminar punto(C)
					\ElsIf{$\alpha < 135^{\circ}$}
						\State nuevo $\gets$ crear punto(P, C, N, $\alpha/2$)
						\State Malla.agregar punto(nuevo)
						\State Malla.agregar triángulo(nuevo, C, N)
						\State AF.insertar punto(nuevo)
					\ElsIf{$\alpha < 180^{\circ}$}
						\State nuevo $\gets$ crear punto(P, C, N, $\alpha/3$)
						\State Malla.agregar punto(nuevo)
						\State Malla.agregar triángulo(nuevo, C, N)
						\State AF.insertar punto(nuevo)
					\EndIf
					\Until $\mbox{AF} \neq \emptyset$
				\EndFunction
			\end{algorithmic}
			\caption{\label{alg:adv_front}Relleno de huecos mediante el método de \emph{advancing front}.
			Los umbrales fueron elegidos de forma de obtener triángulos con ángulos cercanos a $60^{\circ}$.}
		\end{algorithm}

		\begin{algorithm}
			\begin{algorithmic}[1]
				\Function{crear punto}{P, C, N, $\theta$}
					\State planoA $\gets$ plano(P, C, N)
					\State planoB $\gets \left\{
						\begin{tabular}{l}
							.punto $\gets$ promedio(P, C, N) \\
							.normal $\gets$ promedio(P.normal, C.normal, N.normal)
						\end{tabular}
						\right.$
					\State Q $\gets$ rotar(
						punto = N,
						origen = C,
						\Statex normal = planoA.normal,
						ángulo = $\theta$
						)
					\State \Return proyección(Q, planoB)
				\EndFunction
			\end{algorithmic}
			\caption{\label{alg:new_point}Creación del nuevo punto}
		\end{algorithm}

		Con este método se pueden rellenar agujeros pequeños, obteniéndose una malla bastante regular (figura~\ref{fig:fill_good}).
		Sin embargo, debido a la localidad con la que se generan los nuevos
		puntos, el frente puede diverger o pretender unirse a puntos que no
		forman parte del contorno del hueco, resultando una malla mal formada,
		con aristas que corresponden a más de dos caras (figura~\ref{fig:fill_bad}).
		Para evitar la divergencia es necesario definir una superficie de
		soporte considerando todo el contorno del hueco, de forma de asegurar
		que los nuevos puntos no excedan los límites del hueco.

	\begin{figure}
		\Imagen{img/fill_good}
		\caption{\label{fig:fill_good}Relleno de un hueco pequeño mediante \emph{advancing front}.}
	\end{figure}

	\begin{figure}
		\Imagen{img/fill_bad}
		\caption{\label{fig:fill_bad}Fallo en el algoritmo de \emph{advancing front}. Se intentó completar un triángulo con un punto que no pertenecía al borde.}
		\TODO{cambiar gráfico}
	\end{figure}

	\subsection{Reconstrucción de Poisson}
	%Poisson
	La clase pcl::Poisson provee algoritmos de reconstrucción basados en \cite{Kazhdan:2006:PSR:1281957.1281965}, %cite Poisson surface reconstruction
	siendo el principal parámetro la profundidad del octree utilizado,
	impactando directamente en la resolución de la malla resultante.


	Entonces, se convierte el problema en la resolución de una ecuación de Poisson de la forma:
	\[\Delta\chi \equiv \nabla \cdot\nabla\chi = \nabla \vec{n}\]

	Se realiza una discretización del dominio mediante un octree, ya que sólo
	interesa la solución de la función en la proximidad de la superficie a
	reconstruir.
	Luego se definen funciones de soporte local que aproximen una gaussiana.

	Una vez resuelto el problema en el dominio, se extrae la isosuperficie mediante una variante del método de marching cubes.

	La malla resultante presenta un hueco en la base (figura~\ref{fig:fill_poisson}).
	Ya que los puntos del contorno del hueco pertenecen a un mismo plano, es posible rellenarlo mediante el método de \emph{advancing front}.

	\begin{figure}
		\Imagen{img/fill_poisson}
		\caption{\label{fig:fill_poisson}Reconstrucción de la superficie mediante el método de \emph{Poisson}. Todos los huecos fueron rellenados a excepción de la base.}
	\end{figure}


