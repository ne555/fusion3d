\chapter{Desarrollo}
%Intro
El proceso de reconstrucción es lineal:
se elimina el ruido de las capturas parciales del objeto,
se las alinea en un marco de referencia global,
se las combina para obtener una sola malla total
y se rellenan los huecos de esta malla para obtener una superficie cerrada que represente al objeto.

\begin{figure}[h]
	\centering
	\begin{scaletikzpicturetowidth}{\linewidth}
	\begin{tikzpicture}[node distance=2em and 2em]
	\tikzstyle{modulo} = [align=center,rectangle, minimum width=1em, minimum height=1em, text centered, draw=black]
	\tikzstyle{terminal} = [align=center,rectangle, minimum width=1em, minimum height=1em, text centered, draw=white]
	\tikzstyle{arrow} = [thick,->,>=stealth]

	\node (captura) [terminal] {Capturas};
	\node (preproceso) [modulo, right= of captura] {Preproceso};
	\node (registracion) [modulo, right=of preproceso] {Registración};
	\node (fusion) [modulo, below=of registracion] {Fusión};
	\node (relleno) [modulo, right=of fusion] {Rellenado\\de huecos};
	\node (superficie) [terminal, right=of relleno] {Superficie\\cerrada};

	\draw [arrow] (captura) -- (preproceso);
	\draw [arrow] (preproceso) -- (registracion);
	\draw [arrow] (registracion) -- (fusion);
	\draw [arrow] (fusion) -- (relleno);
	\draw [arrow] (relleno) -- (superficie);
\end{tikzpicture}

	\end{scaletikzpicturetowidth}
	\caption[Proceso de reconstrucción tridimensional]{\label{fig:proceso_de_reconstrucción}Proceso de reconstrucción tridimensional.}
\end{figure}

A continuación se desarrollan los detalles de implementación de cada uno de estos procesos.


\section{Módulo de preproceso}
%Escáners de luz estructurada
%Se proyecta en patrón de luz específico hacia el objeto.
%Este patrón es observado por el sensor.
%Se determinan las posiciones de los puntos mediante la intersección entre la
%dirección de la luz y la del sensor.
Este módulo se encargará de realizar una reducción de ruido a las nubes de entrada
antes de ser procesadas por las siguientes etapas,
por lo cual es necesario considerar
qué características posee el ruido y cómo se produce.
En los escáneres de luz estructurada dos fuentes de error son particularmente relevantes:
\begin{itemize}
	\item \emph{Ángulo de incidencia excesivo:\/}
		La luz proyectada impacta en una porción de la superficie del objeto
		que es casi paralela a su dirección.
		El sensor entonces capta una versión estirada y con menor intensidad del patrón, lo que 
		agrega incertidumbre en la posición de los puntos.
	\item \emph{Reflejo parcial:\/}
		Cuando, dada una línea del patrón, solamente una porción de esta incide en el objeto (figura~\ref{fig:error_adquisicion}).
		Se obtiene una posición incorrecta, ya que el método de triangulación
		supone que todo el ancho de la línea impactó en el objeto.
		Esto resulta en bordes distorsionados y en posiciones más alejadas que las reales.\cite{Turk:1994:ZPM:192161.192241}
\end{itemize}
Además, se cuentan con pequeñas variaciones en los puntos debidas a la sensibilidad del sensor.

\begin{figure}
	\Imagen{diagram/error_adquisición_borde}
	\caption{\label{fig:error_adquisicion}Error en la posición estimada debido a un reflejo parcial del patrón de luz.}
\end{figure}

Teniendo en cuenta estas consideraciones, se desarrolló el siguiente algoritmo
para el preproceso de las nubes de entrada:
\begin{itemize}
	\item Para independizarse de la escala de las capturas,
todos los parámetros que impliquen una distancia o vecindad
se establecen en relación a una medida de resolución de las nubes,
la cual se definió como el promedio de las distancias entre cada punto
contra su par más cercano.
Si bien pueden presentarse zonas de mayor densidad de puntos que otras,
surgen problemas para definir la localidad de estos agrupamientos,
por lo que se optó por una medida global.

\item Para reducir las pequeñas perturbaciones en las posiciones de los puntos,
se los proyectó a una superficie estimada mediante el método de mínimos cuadrados móviles.
Para construir esta superficie se utilizó la clase
\texttt{pcl::Moving\-Least\-Squares}, definiendo una vecindad de seis veces la
resolución de la nube con el fin de obtener aproximadamente 100 puntos para
realizar la estimación.
Los resultados de este proceso se evaluaron visualmente, pudiendo observarse la pérdida de detalles
al aumentar en exceso la vecindad (figura~\ref{fig:mls}). \TODO{ver imágenes}
\begin{figure}
	\caption{\label{fig:mls}\TODO{gráfico de curvatura antes y después de mls, distintas vecindades}}
\end{figure}

\item En cuanto a los errores debidos al método de adquisición, se decidió descartar estos puntos:
	\begin{itemize}
		\item Se eliminaron los puntos cuyas normales se encontraban a más de $80^{\circ}$ del eje $z$.
		\item Debido a que a partir de una triangulación es trivial obtener los puntos que pertenecen al borde
se proyectó la nube en el plano $z=0$ para realizar una triangulación Delaunay y así
trasladar estas conexiones a los puntos en el espacio.
		\item Se procedió a eliminar aquellas aristas cuya longitud superaba un umbral (establecido en tres veces la resolución de la nube).
		\item Se eliminaron los puntos que resultaron aislados y aquellos que delimitaban bordes.
	\end{itemize}
\end{itemize}

Si bien al finalizar esta etapa se obtiene una triangulación de la nube de entrada,
los algoritmos implementados en PCL no están diseñados para considerar las conectividades y trabajan únicamente con las vecindades.

\input{registración}
\input{fusión}
\section{Módulo de rellenado de huecos}
	Al finalizar el algoritmo de fusión se obtuvo una malla triangular a partir
	de la información proveniente de cada vista.
	Sin embargo, esta malla no es cerrada ya que existen zonas que ninguna
	vista pudo capturar y por lo tanto carecen de puntos, produciendo huecos en la misma.

	El módulo de rellenado de huecos se encargará de estimar, de forma automática, la superficie del
	objeto en estas zonas para así obtener finalmente una malla cerrada.

	Se plantearon dos métodos:
	\begin{itemize}
		\item \emph{Advancing front}, que trabaja localmente con los puntos que forman el contorno de cada hueco.
		\item Reconstrucción de Poisson, que trabaja con todos los puntos de la nube a la vez. 
	\end{itemize}


	\subsection{Advancing front}
		En este método, cada hueco se rellenará de forma independiente a los otros,
		utilizando únicamente los puntos que conforman cada contorno para estimar
		las posiciones de los nuevos puntos.
		Tomando en cuenta esas consideraciones, se diseñó el diagrama de clases presentado en la figura~\ref{fig:filling_class},
		cuyas clases principales se describen a continuación:
		\begin{itemize}
			\item {\bfseries Relleno de huecos:} La clase se encarga de estimar
				nuevos puntos en zonas donde se carece de información (huecos)
				y triangularlos para que la \emph{Malla} sea cerrada.
			\item {\bfseries Borde:} Es una colección de puntos ordenados
				que representa un borde de un hueco en la \emph{Malla}.
		\end{itemize}

		\begin{figure}
			\Imagen{uml/hole_filling}
			\caption{\label{fig:filling_class}Diagrama de clases del módulo de registración}
		\end{figure}

		Es posible que dentro de un hueco se observen puntos que no lograron
		conectarse al resto de la malla.  Si bien estas islas nos brindan
		información de la superficie, su presencia dificulta la identificación
		del contorno de cada hueco.  Por esta razón, se eliminaron todas las
		islas al trabajar únicamente con la componente conectada que contenía
		la mayor cantidad de puntos.
		De esta forma, una arista que defina sólo un triángulo formará parte de un hueco,
		y podrá obtenerse el contorno del mismo recorriendo el grafo de conectividades. 

		\TODO{gráficos}
		El rellenado se implementó mediante una variante del método de
		\emph{advancing front}\cite{advance_front}, descripta en el
		algoritmo~\ref{alg:adv_front}.

		Los nuevos puntos insertados son elegidos de forma que los triángulos resultantes sean
		aproximadamente equiláteros, como se detalla en el algoritmo~\ref{alg:new_point}.
		Estos puntos son luego proyectados en un plano de soporte definido
		mediante las normales de los puntos del ángulo candidato.

		En caso de que el nuevo punto cayese cerca de otro ya existente, se utilizará aquel.
		Esto implica dividir en dos el hueco, y cada nueva porción se rellenará de forma independiente.
		\TODO{gráfico}

		\begin{algorithm}
			\begin{algorithmic}[1]
				\Function{Advancing front}{Malla, Contorno}
					\State AF $\gets$ Contorno
					\Repeat
					\State $\alpha = \widehat{PCN} =$ ángulo mínimo(Contorno)
					\If{$\alpha < 75^{\circ}$}
						\State Malla.agregar triángulo(P, C, N)
						\State AF.eliminar punto(C)
					\ElsIf{$\alpha < 135^{\circ}$}
						\State nuevo $\gets$ crear punto(P, C, N, $\alpha/2$)
						\State Malla.agregar punto(nuevo)
						\State Malla.agregar triángulo(nuevo, C, N)
						\State AF.insertar punto(nuevo)
					\ElsIf{$\alpha < 180^{\circ}$}
						\State nuevo $\gets$ crear punto(P, C, N, $\alpha/3$)
						\State Malla.agregar punto(nuevo)
						\State Malla.agregar triángulo(nuevo, C, N)
						\State AF.insertar punto(nuevo)
					\EndIf
					\Until $\mbox{AF} \neq \emptyset$
				\EndFunction
			\end{algorithmic}
			\caption{\label{alg:adv_front}Relleno de huecos mediante el método de \emph{advancing front}.
			Los umbrales fueron elegidos de forma de obtener triángulos con ángulos cercanos a $60^{\circ}$.}
		\end{algorithm}

		\begin{algorithm}
			\begin{algorithmic}[1]
				\Function{crear punto}{P, C, N, $\theta$}
					\State planoA $\gets$ plano(P, C, N)
					\State planoB $\gets \left\{
						\begin{tabular}{l}
							.punto $\gets$ promedio(P, C, N) \\
							.normal $\gets$ promedio(P.normal, C.normal, N.normal)
						\end{tabular}
						\right.$
					\State Q $\gets$ rotar(
						punto = N,
						origen = C,
						\Statex normal = planoA.normal,
						ángulo = $\theta$
						)
					\State \Return proyección(Q, planoB)
				\EndFunction
			\end{algorithmic}
			\caption{\label{alg:new_point}Creación del nuevo punto}
		\end{algorithm}

		Con este método se pueden rellenar agujeros pequeños, obteniéndose una malla bastante regular (figura~\ref{fig:fill_good}).
		Sin embargo, debido a la localidad con la que se generan los nuevos
		puntos, el frente puede diverger o pretender unirse a puntos que no
		forman parte del contorno del hueco, resultando una malla mal formada,
		con aristas que corresponden a más de dos caras (figura~\ref{fig:fill_bad}).
		Para evitar la divergencia es necesario definir una superficie de
		soporte considerando todo el contorno del hueco, de forma de asegurar
		que los nuevos puntos no excedan los límites del hueco.

	\begin{figure}
		\Imagen{img/fill_good}
		\caption{\label{fig:fill_good}Relleno de un hueco pequeño mediante \emph{advancing front}.}
	\end{figure}

	\begin{figure}
		\Imagen{img/fill_bad}
		\caption{\label{fig:fill_bad}Fallo en el algoritmo de \emph{advancing front}. Se intentó completar un triángulo con un punto que no pertenecía al borde.}
		\TODO{cambiar gráfico}
	\end{figure}

	\subsection{Reconstrucción de Poisson}
	%Poisson
	La clase pcl::Poisson provee algoritmos de reconstrucción basados en \cite{Kazhdan:2006:PSR:1281957.1281965}, %cite Poisson surface reconstruction
	siendo el principal parámetro la profundidad del octree utilizado,
	impactando directamente en la resolución de la malla resultante.


	Entonces, se convierte el problema en la resolución de una ecuación de Poisson de la forma:
	\[\Delta\chi \equiv \nabla \cdot\nabla\chi = \nabla \vec{n}\]

	Se realiza una discretización del dominio mediante un octree, ya que sólo
	interesa la solución de la función en la proximidad de la superficie a
	reconstruir.
	Luego se definen funciones de soporte local que aproximen una gaussiana.

	Una vez resuelto el problema en el dominio, se extrae la isosuperficie mediante una variante del método de marching cubes.

	La malla resultante presenta un hueco en la base (figura~\ref{fig:fill_poisson}).
	Ya que los puntos del contorno del hueco pertenecen a un mismo plano, es posible rellenarlo mediante el método de \emph{advancing front}.

	\begin{figure}
		\Imagen{img/fill_poisson}
		\caption{\label{fig:fill_poisson}Reconstrucción de la superficie mediante el método de \emph{Poisson}. Todos los huecos fueron rellenados a excepción de la base.}
	\end{figure}


