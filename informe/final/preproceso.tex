\section{Módulo de preproceso}
%Escáners de luz estructurada
%Se proyecta en patrón de luz específico hacia el objeto.
%Este patrón es observado por el sensor.
%Se determinan las posiciones de los puntos mediante la intersección entre la
%dirección de la luz y la del sensor.
El módulo de preproceso se encarga de realizar una reducción de ruido a las nubes de entrada
antes de ser procesadas por las siguientes etapas.
Para esto, es necesario considerar qué características posee el ruido y cómo se produce,
y por esta razón no podemos independizarnos completamente del dispositivo de captura.

Considerando las fuentes de error más importantes de los escáneres de luz estructurada, se desarrolló
el siguiente algoritmo para el preproceso de cada nube de entrada:
\begin{itemize}
	\item Para independizarse de la escala de las capturas,
		todos los parámetros que impliquen una distancia o vecindad
		se establecen en relación a una medida de resolución de las nubes,
		definida como el promedio de las distancias entre cada punto
		contra su par más cercano (algoritmo~\ref{alg:resolución_de_nube}).
		Si bien pueden presentarse zonas de mayor densidad de puntos que otras,
		surgen problemas para definir la localidad de estos agrupamientos,
		por lo que se optó por una medida global.

\begin{algorithm}
	\begin{algorithmic}[1]
		\Function{Resolución}{nube}
			\State distancia $\gets0$
			\ForAll{$p \in \mbox{nube.puntos}$}
				\State v $\gets$ cercano(p, nube.puntos)[1]
				\State distancia $\gets \text{distancia} + d(p, v)$
			\EndFor
			\State\Return distancia / nube.tamaño
		\EndFunction
	\end{algorithmic}
	\caption[Medida de resolución de la nube]{\label{alg:resolución_de_nube}Medida de resolución de la nube como promedio entre las distancias de los pares más cercanos.}
\end{algorithm}

	\item Para reducir las pequeñas perturbaciones en las posiciones de los puntos,
		se los proyecta a una superficie estimada mediante el método de mínimos cuadrados móviles.
		Para construir esta superficie se utiliza la clase
		\texttt{pcl::Moving\-Least\-Squares}, definiendo una vecindad de seis veces la
		resolución de la nube con el fin de obtener aproximadamente 100 puntos para
		realizar la estimación.
		Los resultados de este proceso se evaluaron visualmente, pudiendo observarse la pérdida de detalles
		al aumentar en exceso la vecindad (figura~\ref{fig:mls}).
		\begin{figure}
			\Imagen{img/mls_6_10}
			\caption[Visualización de la curvatura]{\label{fig:mls}Visualización de la curvatura después de aplicar
			mínimos cuadrados móviles en una vecindad de seis veces la resolución (izquierda)
			y diez veces la resolución (derecha).
			Los detalles del cuerpo y rostro son percibidos con menor intensidad
			al aumentar el tamaño de la vecindad.}
		\end{figure}

	\item Para evitar las distorsiones debidas al ángulo de incidencia excesivo,
		se estiman las normales en cada punto mediante la técnica de covarianza propuesta en \cite{10.1109/34.334391}.
		Luego se eliminan los puntos cuyas normales se encontraban a más de $80^{\circ}$ del eje $z$.

	\item Se eliminan aquellos puntos que pueden ser producto de distorsiones debidas al reflejo parcial.
		Estos puntos se identifican de la siguiente manera:
		\begin{itemize}
			\item Considerando que la captura se encuentra en 2.5D,
				se proyecta la nube al plano $z=0$ para realizar una triangulación Delaunay y así
				trasladar estas conexiones a los puntos en el espacio.
			\item Se eliminan aquellas aristas cuya longitud supera un umbral.
				Estas aristas corresponden a saltos excesivos de profundidad
				o a regiones en las que no se registraron puntos.
			\item Los puntos buscados son aquellos que delimitan bordes en la malla o se encuentran aislados.
		\end{itemize}
\end{itemize}

Si bien al finalizar esta etapa se obtiene una triangulación de la nube de entrada,
los algoritmos implementados en PCL no están diseñados para considerar las conectividades y trabajan únicamente con las vecindades.


