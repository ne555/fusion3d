\section{Módulo de preproceso}
%Escáners de luz estructurada
%Se proyecta en patrón de luz específico hacia el objeto.
%Este patrón es observado por el sensor.
%Se determinan las posiciones de los puntos mediante la intersección entre la
%dirección de la luz y la del sensor.
El módulo de preproceso se encarga de realizar una reducción de ruido a las nubes de entrada
antes de ser procesadas por las siguientes etapas.
Para esto, es necesario considerar qué características posee el ruido y cómo se produce,
y por esta razón no podemos independizarnos completamente del dispositivo de captura.

En los escáneres de luz estructurada,
como los utilizados en la base de datos Stanford\cite{Turk:1994:ZPM:192161.192241}
y la biblioteca desarrollada en \cite{Pancho},
dos fuentes de error son particularmente relevantes:
\begin{itemize}
	\item \emph{Ángulo de incidencia excesivo:}
		La luz proyectada impacta en una porción de la superficie del objeto
		que es casi paralela a su dirección.
		El sensor entonces capta una versión estirada y con menor intensidad del patrón, lo que 
		agrega incertidumbre en la posición de los puntos.
	\item \emph{Reflejo parcial:}
		Se produce cuando una línea del patrón no incide completamente en el objeto (figura~\ref{fig:error_adquisicion}).
		Debido a que el método de triangulación supone que todo el ancho de la línea impactó en el objeto,
		se estima una posición incorrecta del punto muestreado.
		Esto resulta en bordes distorsionados y en posiciones más alejadas que las reales.\cite{Turk:1994:ZPM:192161.192241}
\end{itemize}
Además, deben tenerse en cuenta la existencia de pequeñas variaciones en las posiciones de los puntos.

\begin{figure}
	\Imagen{diagram/error_adquisición_borde}
	\caption{\label{fig:error_adquisicion}Error en la posición estimada debido a un reflejo parcial del patrón de luz.}
		\TODO{imagen patrón estirado}
\end{figure}

En base a todo eso, se desarrolló el siguiente algoritmo
para el preproceso de cada nube de entrada:
\begin{itemize}
	\item Para independizarse de la escala de las capturas,
		todos los parámetros que impliquen una distancia o vecindad
		se establecen en relación a una medida de resolución de las nubes,
		la cual se definió como el promedio de las distancias entre cada punto
		contra su par más cercano.
		Si bien pueden presentarse zonas de mayor densidad de puntos que otras,
		surgen problemas para definir la localidad de estos agrupamientos,
		por lo que se optó por una medida global.

	\item Para reducir las pequeñas perturbaciones en las posiciones de los puntos,
		se los proyectó a una superficie estimada mediante el método de mínimos cuadrados móviles.
		Para construir esta superficie se utilizó la clase
		\texttt{pcl::Moving\-Least\-Squares}, definiendo una vecindad de seis veces la
		resolución de la nube con el fin de obtener aproximadamente 100 puntos para
		realizar la estimación.
		Los resultados de este proceso se evaluaron visualmente, pudiendo observarse la pérdida de detalles
		al aumentar en exceso la vecindad (figura~\ref{fig:mls}). \TODO{ver imágenes}
		\begin{figure}
			\caption{\label{fig:mls}\TODO{gráfico de curvatura antes y después de mls, distintas vecindades}}
		\end{figure}

	\item Para evitar las distorsiones debidas al ángulo de incidencia excesivo,
		primeramente se estimaron las normales en cada punto mediante la técnica de covarianza propuesta en \cite{10.1109/34.334391}.
		Luego se eliminaron los puntos cuyas normales se encontraban a más de $80^{\circ}$ del eje $z$.

	\item Se eliminaron aquellos puntos que pudieran ser producto de distorsiones debidas al reflejo parcial.
		Para identificar estos puntos, se procedió de la siguiente manera:
		\begin{itemize}
			\item \TODO{captura como imagen 2.5D z apunta al ojo}
				Se proyectó la nube al plano $z=0$ para realizar una triangulación Delaunay y así
				trasladar estas conexiones a los puntos en el espacio.
			\item Se eliminaron aquellas aristas cuya longitud superaba un umbral.
				Estas aristas corresponden a saltos excesivos de profundidad
				o a regiones en las que no se registraron puntos.
			\item Los puntos buscados son aquellos que delimitan bordes en la malla o se encuentran aislados.
		\end{itemize}
\end{itemize}

Si bien al finalizar esta etapa se obtiene una triangulación de la nube de entrada,
los algoritmos implementados en PCL no están diseñados para considerar las conectividades y trabajan únicamente con las vecindades.
