\section{Preproceso}
%Escáners de luz estructurada
%Se proyecta en patrón de luz específico hacia el objeto.
%Este patrón es observado por el sensor.
%Se determinan las posiciones de los puntos mediante la intersección entre la
%dirección de la luz y la del sensor.

Para poder plantear un método de reducción de ruido, es necesario considerar
qué características posee y cómo se produce.
%zippered
En los escáneres de luz estructurada dos fuentes de error son particularmente relevantes:
\begin{itemize}
	\item Grazing angle %traducir
		La luz proyectada impacta en una porción del objeto que es casi paralela a su dirección.
		El sensor entonces capta una versión estirada y con menor intensidad del patrón, lo que 
		agrega incertidumbre en la posición de los puntos.
	\item Reflejo parcial %nombre
		Cuando, dada una línea del patrón, solamente una porción de esta incide en el objeto.
		Se obtiene una posición incorrecta, ya que el método de triangulación
		supone que todo el ancho de la línea impactó en el objeto.
		Esto resulta en bordes distorsionados y en posiciones más alejadas que las reales.
%fin zippered
\end{itemize}
Además, se cuentan con pequeñas variaciones en los puntos debidas a la sensibilidad del sensor.

Teniendo en cuenta estas consideraciones, se desarrolló el siguiente algoritmo
para el preproceso de las nubes de entrada.

Primeramente, para independizarse de la escala de las capturas,
todos los parámetros que impliquen una distancia o vecindad
se establecen en relación a una medida de resolución de las nubes,
la cual se definió como el promedio de las distancias entre cada punto
contra su par más cercano.
Si bien pueden presentarse zonas de mayor densidad de puntos que otras,
surgen problemas para definir la localidad de estos agrupamientos,
por lo que se optó por una medida global.

Para reducir las pequeñas perturbaciones en las posiciones de los puntos,
se los proyectó a una superficie estimada mediante el método de mínimos cuadrados móviles.
Para construir esta superficie se utilizó la clase
\texttt{pcl::MovingLeastSquares}, definiendo una vecindad de seis veces la
resolución de la nube con el fin de obtener aproximadamente 100 puntos para
realizar la estimación.
Los resultados de este proceso se evaluaron visualmente.
%TODO: gráfico curvatura antes y después mls (triangulada con delaunay)

En cuanto a los errores debidos al método de adquisición, se decidió descartar estos puntos.
Debido a que a partir de una triangulación es trivial obtener los puntos que pertenecen al borde
se proyectó la nube en el plano $z=0$ para realizar una triangulación Delaunay y luego
trasladar estas conexiones a los puntos en el espacio.
Luego, se procedió a eliminar aquellas aristas cuya longitud superaba un umbral.
Finalmente, se eliminaron los puntos que resultaron aislados y aquellos que delimitaban bordes.
Además, se eliminaron los puntos cuyas normales se encontraban a más de $80^{\circ}$ del eje $z$.

Si bien al finalizar esta etapa se obtiene una triangulación de la nube de entrada,
los algoritmos implementados en PCL no están diseñados para considerar las conectividades y trabajan únicamente con las vecindades.
