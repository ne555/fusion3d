\chapter{Conclusiones y trabajos futuros}
\section{Del producto}
	En este proyecto se realizó el desarrollo de una biblioteca de software para
	realizar la reconstrucción tridimensional de un objeto a partir de capturas de
	vistas parciales.
	Para esto, se dividió el problema en tres módulos: registración, fusión y
	rellenado de huecos, y se implementaron diversos algoritmos.

	Al trabajar directamente con las nubes de puntos no se restringió el
	dispositivo de captura a un hardware en particular. Sin embargo, las
	restricciones impuestas de base giratoria y limitar la cantidad de capturas
	requeridas fueron planteadas considerando una integración futura con
	el trabajo realizado por \cite{Pancho}.

	Las registraciones se obtuvieron en tiempos razonables, sin requerir hardware especial.
	Si bien se observa un efecto de «inflado» debido a la propagación de los errores de registración,
	este se encuentra suficientemente acotado respecto al tamaño del objeto.

	Al realizar las capturas solamente sobre una base giratoria,
	Debido a que las capturas se realizaron solamente sobre una base giratoria,
	no se lograron resolver todas las oclusiones, lo que genera la aparición de huecos de
	tamaño considerable.
	Aún así, el resultado es una malla cerrada, y la
	superficie estimada en las zonas sin información se une suavemente al resto de
	la malla.

\section{Del proceso}
La elección de la metodología en cascada modificada fue incorrecta.
La investigación bibliográfica se alargó demasiado y se desperdiciaron recursos
al abordar temas que tuvieron que ser descartados luego (como el uso de la
información de textura).
Además, se produjo un desfasaje entre la adquisición de los conocimientos y la
implementación de los mismos.

Hubiese sido mejor utilizar directamente una metodología incremental en todo el
proceso, con más incrementos de menor tamaño, como ser agregar un primer módulo
de preproceso que contenga la reducción de ruido y la operatoria básica con las
nubes de puntos.

En cuanto al desarrollo, uno de los principales problemas fue la definición de
métricas para evaluar los algoritmos y establecer qué niveles de error eran
aceptables y cuáles requerían una corrección.
Esto se dificulta, además, al considerar que los resultados producidos en una etapa
serán la entrada de otra, de la cual se desconoce su sensibilidad.
Muchas evaluaciones fueron primeramente visuales, resultando en un proceso lento
que en ocasiones fallaba en detectar errores considerables.



%\subsection{Riesgos efectivizados}
%Ausencia de repositorio de mallas tridimensionales
%(copiar base de datos)
%
%Falla en los equipos de trabajo
%(copiar parte de pcl)
%Meshlab: no se logró instalarlo en el nuevo equipo, se cambió a CloudCompare



\section{Trabajos futuros}
En esta sección se describirán actividades que excedieron el alcance de este proyecto
y podrían ser abordadas en una etapa posterior.

\begin{itemize}
	\item Combinar escaneos cilíndricos del objeto en varias posiciones sobre la base giratoria,
		buscando de esta forma eliminar huecos y reducir la propagación del error de alineación.
		Esto requerirá eliminar la restricción del eje de giro en la
		registración, por lo que deberán evaluarse otros métodos para detectar
		correspondencias erróneas.
	\item Ajustar los métodos para trabajar con el volumen de puntos generados
		por \cite{Pancho}.  Las capturas de los modelos con los cuales se
		realizaron las pruebas contenían a lo sumo ochenta mil puntos. Las
		reconstrucciones obtenidas en \cite{Pancho} pueden llegar a los
		dos millones, es decir, 25 veces más.
		Es necesario entonces, analizar la escalabilidad de los métodos
		propuestos y optimizarlos o reemplazarlos según resulte conveniente.
		En particular, puede plantearse la ejecución sobre GPU.
	\item Implementar métricas de calidad del mallado que consideren las
		características de la impresión 3D.
		En este proyecto solamente se consideraron las condiciones de que la malla
		resultase cerrada y no presente intersecciones consigo misma.
	%\item Mejorar la detección de \emph{outliers} en el módulo de fusión.
	%\item Implementar métodos de suavizado de mallas.
	\item Modificar el método de \emph{advancing front} para que utilice una
		superficie de soporte para establecer la posición de los nuevos puntos,
		asegurando de esta forma la convergencia del método y la suavidad del
		parche generado.
	\item Modificar el método de \emph{advancing front} para que considere las islas.
		Esto requerirá detectar dentro de qué hueco se encuentra cada isla.
\end{itemize}
