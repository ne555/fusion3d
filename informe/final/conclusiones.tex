\chapter{Conclusiones y trabajos futuros}
%TODO: introducción

\section{Conclusiones del producto}
%explayarse más
	En este proyecto se realizó el desarrollo de una biblioteca de software para lograr
	la reconstrucción tridimensional de un objeto a partir de capturas de
	vistas parciales.
	Para esto, se dividió el problema en tres módulos: registración, fusión y
	rellenado de huecos, y se implementaron diversos algoritmos.

	A pesar de que las capturas no contenían información de textura,
	el algoritmo de registración fue exitoso en casi todos los casos sin requerir
	ajustes a su conjunto de parámetros. 
	Además, se cuenta con medidas de la calidad de la alineación,
	que permiten detectar fallas durante esta etapa sin requerir de una inspección visual.

	%Al trabajar directamente con las nubes de puntos no se restringió el
	%dispositivo de captura a un hardware en particular.  Sin embargo, las
	%restricciones impuestas de base giratoria y limitar la cantidad de capturas
	%requeridas fueron planteadas considerando una integración futura con
	%el trabajo realizado por \cite{Pancho}.

	Las reconstrucciones fueron obtenidas en tiempos razonables y sin requerir hardware especial.
	Si bien se observa un efecto de «inflación/deflación» debido a la propagación de los errores de registración,
	este se encuentra suficientemente acotado respecto al tamaño del objeto.

	Debido a que se consideró solamente una posición del objeto sobre la base giratoria,
	se presenta una gran cantidad de oclusiones,
	lo que genera la aparición de huecos de tamaño considerable
	en la superficie reconstruida luego de la fusión.
	Aún así, el resultado final es una malla cerrada (a excepción de la base), y la
	superficie estimada en las zonas sin información se une suavemente al resto.

\section{Conclusiones del proceso}
%redacción informal
Se tuvieron problemas al implementar la metodología seleccionada.
El tiempo invertido en la etapa de investigación bibliográfica fue demasiado extenso
y se desperdiciaron recursos al abordar el tratamiento de la información de textura,
que finalmente debió ser descartada al no contar con un repositorio propio.
Además, se produjo un desfasaje temporal entre la adquisición de los conocimientos y la
implementación de los mismos, requiriendo un nuevo análisis.
Estos inconvenientes se hubieran resuelto
al utilizar directamente una metodología incremental en todo el proceso,
con más incrementos de menor tamaño, como ser agregar un primer módulo
de preproceso que contenga la reducción de ruido y la operatoria básica con las
nubes de puntos.

En cuanto al desarrollo, uno de los principales problemas fue la definición de métricas
para evaluar los algoritmos y establecer los niveles de error aceptables en cada etapa.
%Esto se dificulta, además, al considerar que los resultados producidos en una etapa
%serán la entrada de otra, de la cual se desconoce su sensibilidad.
Muchas evaluaciones fueron primeramente visuales, resultando en un proceso lento
que en ocasiones fallaba en detectar errores considerables.

Durante el desarrollo se efectivizó otro de los riesgos identificados para el proyecto:
la falla en los equipos de trabajo requiriendo su reemplazo.
Gracias a las copias de respaldo periódicas, fue posible recuperar fácilmente el trabajo realizado hasta ese momento.
Sin embargo, debido a que el nuevo equipo de trabajo contaba con otro sistema operativo (Clear Linux),
se requirió de un largo proceso de configuración para instalar la biblioteca PCL a partir de sus archivos fuentes.
%Primeramente, se contaba con un sistema operativo Arch Linux, donde la
%instalación de la biblioteca PCL se realiza mediante un script
%\texttt{PKGBUILD} que resuelve las dependencias y configura los módulos.  Fue
%necesario compilar los fuentes, pero fuera del tiempo requerido, no se tuvieron
%mayores inconvenientes.  En el nuevo equipo, se contaba esta vez con un sistema
%Clear Linux.  Ahora la instalación resultó más problemática.  Se requerían
%demasiados recursos de memoria, por lo que el sistema operativo detenía el
%proceso.  Fue necesario un largo proceso de configuración y prueba para lograr
%la instalación exitosa de la biblioteca.

%\subsection{Riesgos efectivizados}
%Ausencia de repositorio de mallas tridimensionales
%(copiar base de datos)
%
%Falla en los equipos de trabajo
%(copiar parte de pcl)
%Meshlab: no se logró instalarlo en el nuevo equipo, se cambió a CloudCompare



\section{Trabajos futuros}
%\TODO{fusión, mejora de la confianza. Confianza según cercanía al borde, promedio de confianza}

En esta sección se describen actividades que excedieron el alcance de este proyecto
y podrían ser abordadas en una etapa posterior.

\begin{itemize}
	\item Ajustar los métodos de registración para combinar escaneos del objeto
		en varias posiciones sobre la base giratoria,
		buscando de esta forma eliminar huecos y reducir la propagación del error de alineación.
		Esto requerirá eliminar la restricción del eje de giro en la registración
		y ajustar el algoritmo de corrección de bucle.
	\item Ajustar los métodos para trabajar con el volumen de puntos generados
		por \cite{Pancho}.
		Es necesario analizar la robustez del algoritmo al submuestreo de la entrada,
		realizar una selección de keypoints de las nubes de entrada
		y utilizar un método más eficiente para la búsqueda de correspondencias.
	\item Paralelizar los algoritmos utilizados en el módulo de registración,
		en particular la obtención de descriptores y la determinación de correspondencias.
	\item Utilizar la métrica de fitness para el ajuste automático de los parámetros utilizados en la registración.
	\item Implementar métricas de calidad del mallado que consideren las
		características de la impresión 3D.
		En este proyecto solamente se consideraron las condiciones de que la malla
		resultase cerrada y no presente intersecciones consigo misma.
	%\item Mejorar la detección de \emph{outliers} en el módulo de fusión.
	%\item Implementar métodos de suavizado de mallas.
	\item Modificar el método de \emph{advancing front} para que utilice una
		superficie de soporte para establecer la posición de los nuevos puntos,
		asegurando de esta forma la convergencia del método y la suavidad del
		parche generado.
	\item Modificar el método de \emph{advancing front} para que considere las islas.
		Esto requerirá detectar dentro de qué hueco se encuentra cada isla.
\end{itemize}
