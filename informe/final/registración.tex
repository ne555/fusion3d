\section{Módulo de registración}
	%¿Qué es la registración?
	%mover al marco teórico
	El módulo de registración se encarga de obtener las transformaciones de rotación y translación
	que lleven cada vista a un sistema global de forma
	que las zonas comunes encajen perfectamente.

	%para evitar que se corte en el quotation
	%\clearpage
	Debido a que
	«el algoritmo de \emph{Iterative Closest Point} (ICP) se ha convertido en el
	método dominante para realizar la registración de modelos tridimensionales
	utilizando únicamente la información de geometría de los mismos\cite{Rusinkiewicz02real-time3d}»
	y a que a pesar de que existe garantía de convergencia del método
	hacia un mínimo local este puede no ser el mínimo global buscado,
	es necesario proveer una alineación inicial
	lo suficientemente cercana para obtener una registración correcta\cite{regBesl92}.

	Por esto, y considerando que la distancia entre capturas sucesivas es considerable,
	se desarrollaron algoritmos para obtener una alineación inicial
	con cada par de capturas para luego refinar la registración mediante ICP.
	Estos algoritmos constan de los siguientes pasos:
	\begin{enumerate}
		\item Selección de puntos de la entrada.
		\item Cálculo de descriptores y determinación de correspondencias.
		\item Filtrado de correspondencias.
		\item Estimación de la transformación de alineación.
	\end{enumerate}
	Variaciones en la ejecución de estos pasos nos permiten implementar diversos métodos.


	Utilizando la primera captura para definir el sistema de coordenadas
	global, cada nueva vista se alinea con la anterior hasta completar una
	vuelta sobre el objeto, propagando el error de registración.
	Por esta razón, se incorporó un algoritmo de corrección de bucle al proceso de registración
	(figura~\ref{fig:flow_registracion}).


	\begin{figure}
		\Imagen{uml/registration_flow.pdf}
		\caption[Diagrama de flujo de la registración]{\label{fig:flow_registracion}Diagrama de flujo de la registración.}
	\end{figure}


	A continuación se describen dos métodos para calcular la alineación inicial entre un par de capturas. 

	\subsection{Alineación mediante \emph{sample consensus}}
		\subsubsection{Selección de keypoints}
			En base a los resultados obtenidos por \cite{ISS},
			se consideró utilizar el algoritmo de detección de keypoints basado en \emph{Intrinsic Shape Signatures} (algoritmo~\ref{alg:iss}),
			el cual se halla implementado en PCL en la clase \texttt{ISSKeypoint3D}, permitiendo
			definir el radio de la vecindad y el nivel de disimilitud de los eigenvalores.


			%ver bien el problema
			Sin embargo, no pudieron encontrarse los parámetros adecuados.
			Al observar las zonas comunes en los casos de estudio
			utilizados como referencia durante el desarrollo,
			eran pocos los keypoints que
			realmente se encontraban lo suficientemente cerca como para generar
			una correspondencia válida (figura~\ref{fig:iss_key}).



			\begin{algorithm}
				\begin{algorithmic}[1]
					\Function{ISS Keypoints}{nube, $r_1$, $\mbox{umbral}_1$, $\mbox{umbral}_2$, $r_2$}
						\State keypoints $\gets\emptyset$
						\ForAll{$p \in \mbox{nube.puntos}$}
							\State vecinos $\gets$ obtener puntos cercanos(nube, p, $r_1$)
							\State m $\gets$ matriz de covarianza(vecinos)
							\State $\lambda\gets\text{eigenvalores}(m)$
							\If{$\lambda_1/\lambda_2 > \text{umbral}_1$ and $\lambda_2/\lambda_3 > \text{umbral}_2$}
								\State keypoints.insert(p)
							\EndIf
						\EndFor
						\State\Return Non-Max Suppression(keypoints, $r_2$)
					\EndFunction
				\end{algorithmic}
				\caption[Determinación de los keypoints mediante ISS]{\label{alg:iss}Determinación de los keypoints mediante ISS}
			\end{algorithm}


			%A partir de acá ref Rusu FPFH
			Por esta razón, se cambió el método de selección de keypoints,
			eligiendo un análisis de persistencia multiescala:
			\begin{enumerate}
				\item Por cada punto de la nube se calcula su descriptor para distintos tamaños de vecindad (escala).
				\item A partir de todos los descriptores en todas las escalas se estima una distribución gaussiana que los aproxime.
				\item Los keypoints quedan definidos como aquellos puntos cuyos descriptores se encuentran alejados de la media\cite{Rusu:2009:FPF:1703435.1703733}.
			\end{enumerate}
			El algoritmo se encuentra implementado en PCL en la clase
			\texttt{Multiscale\-Feature\-Persistence} permitiendo ajustar las
			escalas a utilizar, el umbral para ser considerado saliente y el descriptor a utilizar.
			Debido a que es necesario calcular un descriptor para cada punto, y
			además en diferentes escalas, se eligió utilizar el método
			\emph{Fast Point Feature Histograms} (FPFH) para su construcción,
			el cual es lineal en la cantidad de puntos de la vecindad.
			%Este método calcula un histograma de los ángulos entre las normales del punto y sus vecinos\cite{Rusu:2009:FPF:1703435.1703733}.

			Con este algoritmo, los keypoints se agrupan formando líneas en zonas de cambio brusco de curvatura (figura~\ref{fig:multiscale_key}).


			\begin{figure}
				\centering
				\begin{subfigure}{\linewidth}
					\Imagen{img/iss_happy}
					\caption[Keypoints ISS]{\label{fig:iss_key}Keypoints ISS}
				\end{subfigure}

				\begin{subfigure}{\linewidth}
					\Imagen{img/multiscale_happy}
					\caption[Keypoints FPFH de persistencia multiescala]{\label{fig:multiscale_key}Keypoints FPFH de persistencia multiescala.}
				\end{subfigure}
				\caption[Visualización de los keypoints]{\label{fig:keypoints}Visualización de los keypoints calculados en las vistas
					\texttt{happy\_0} (verde) y \texttt{happy\_24} (rojo).
					A la derecha se seleccionaron aquellos cuyo par
					más cercano se encontraban a menos de 4 veces la resolución
					de la nube.}
			\end{figure}



		\subsubsection{Estimación de la transformación}
			Se utilizó el algoritmo de \emph{sample consensus initial alignment
			(SAC-IA)} para obtener una estimación inicial de la transformación de alineación.
			Este algoritmo consiste en:
			\begin{enumerate}
				\item Seleccionar al azar \emph{m} puntos de la nube A.
				\item Por cada punto, buscar aquellos con descriptores similares en B y seleccionar uno al azar.
				\item Calcular la transformación definida por estos puntos y sus correspondencias.
					Es decir, calcular la transformación que minimice
					\[ \sum |b_j - T(a_j)| \]
					para los puntos muestreados.
				\item Calcular una medida del error de transformación.
					Es decir, calcular el error
					\[ \sum |b_j - T(a_j)| \]
					para todos los puntos de la nube.
				\item Repetir varias veces y devolver aquella transformación que produjo el menor error.\cite{Rusu:2009:FPF:1703435.1703733}
			\end{enumerate}


	%El que tengo ahora
	\subsection{Alineación mediante búsqueda de clúster}
		En este método alternativo de alineación no se seleccionan keypoints,
		sino que se realiza un submuestreo uniforme de los puntos de la nube
		para reducir el costo computacional.
		Además del descriptor, en cada punto se estableció un marco de referencia
		que permite estimar una transformación de alineación
		considerando solamente los dos puntos que conforman una correspondencia\cite{ISS}.
		Entonces, se procede a un proceso de votación para determinar la alineación final.

		Para determinar el marco de referencia se utiliza la matriz de covarianza de la vecindad del punto:
		\begin{itemize}
			\item Se computan los eigenvalores ${\lambda_1, \lambda_2, \lambda_3}$ en orden decreciente y sus eigenvectores correspondientes
				$e^{1}, e^{2}, e^{3}$.
			\item Debido a que $e^{3}$ representa la normal del punto, se ajusta su sentido para que coincida con el del eje $z$.
			\item Los otros ejes se definen mediante $e^{1}$ y $e^{1} \times e^{3}$.
		\end{itemize}
		Se presenta una ambigüedad en el marco de referencia según el sentido que se le asigne a $e^{1}$ (figura~\ref{fig:marco_referencia_iss}),
		la cual se resuelve definiendo un eje de giro para la alineación.

		\begin{figure}
			\Imagen{diagram/marco_referencia_iss}
			\caption[Marcos de referencia]{\label{fig:marco_referencia_iss}Marcos de referencia. Se observa una ambigüedad equivalente a un giro de $180^{\circ}$ sobre el eje $e^{3}$ \RefImagen{ISS}.}
		\end{figure}

		Para establecer las correspondencias se eligió nuevamente el descriptor FPFH,
		comparando los histogramas entre cada par de puntos mediante la distancia $\chi^2$:
		\[ \chi^2 = \sum_{j=1}^{N} \frac{\left(a_j - b_j\right)^2}{a_j + b_j} \]
		Para identificar y eliminar las correspondencias erróneas se utilizan
		los marcos de referencia y las suposiciones de ubicación de la cámara
		en la obtención de las capturas.  Es decir, se descartan aquellas correspondencias que
		requieren un movimiento en $y$ excesivo o una rotación sobre un eje no
		vertical. 

		Cada correspondencia define un ángulo de giro $\theta$ sobre
		el eje $y$ y una translación en el plano $xz$, observándose una agrupación
		de los parámetros de estas transformaciones (figura~\ref{fig:cluster}).
		Mediante el algoritmo de k-means, se busca el centroide del clúster más grande
		para estimar la transformación final.

		\begin{figure}
			\centering
			\begin{subfigure}{\linewidth}
				\centering
				\begin{tikzpicture}
					% Title: gl2ps_renderer figure
% Creator: GL2PS 1.4.0, (C) 1999-2017 C. Geuzaine
% For: Octave
% CreationDate: Thu Mar 12 11:09:46 2020
%\begin{pgfpicture}
\color[rgb]{1.000000,1.000000,1.000000}
\pgfpathrectanglecorners{\pgfpoint{0pt}{0pt}}{\pgfpoint{350pt}{200pt}}
\pgfusepath{fill}
%\begin{pgfscope}
\pgfpathrectangle{\pgfpoint{0pt}{0pt}}{\pgfpoint{350pt}{200pt}}
\pgfusepath{fill}
\pgfpathrectangle{\pgfpoint{0pt}{0pt}}{\pgfpoint{350pt}{200pt}}
\pgfusepath{clip}
\pgfpathmoveto{\pgfpoint{45.500000pt}{185.000000pt}}
\pgflineto{\pgfpoint{316.749969pt}{32.500008pt}}
\pgflineto{\pgfpoint{45.500000pt}{32.500008pt}}
\pgfpathclose
\pgfusepath{fill,stroke}
\pgfpathmoveto{\pgfpoint{45.500000pt}{185.000000pt}}
\pgflineto{\pgfpoint{316.749969pt}{185.000000pt}}
\pgflineto{\pgfpoint{316.749969pt}{32.500008pt}}
\pgfpathclose
\pgfusepath{fill,stroke}
\color[rgb]{0,0,0}
\pgfsetlinewidth{0.500000pt}
\pgfpathmoveto{\pgfpoint{45.500000pt}{35.203644pt}}
\pgflineto{\pgfpoint{45.500000pt}{32.500008pt}}
\pgfusepath{stroke}
\pgfpathmoveto{\pgfpoint{45.500000pt}{182.296371pt}}
\pgflineto{\pgfpoint{45.500000pt}{185.000000pt}}
\pgfusepath{stroke}
\pgfpathmoveto{\pgfpoint{113.312492pt}{35.203644pt}}
\pgflineto{\pgfpoint{113.312492pt}{32.500008pt}}
\pgfusepath{stroke}
\pgfpathmoveto{\pgfpoint{113.312492pt}{182.296371pt}}
\pgflineto{\pgfpoint{113.312492pt}{185.000000pt}}
\pgfusepath{stroke}
\pgfpathmoveto{\pgfpoint{181.124985pt}{35.203644pt}}
\pgflineto{\pgfpoint{181.124985pt}{32.500008pt}}
\pgfusepath{stroke}
\pgfpathmoveto{\pgfpoint{181.124985pt}{182.296371pt}}
\pgflineto{\pgfpoint{181.124985pt}{185.000000pt}}
\pgfusepath{stroke}
\pgfpathmoveto{\pgfpoint{248.937485pt}{35.203644pt}}
\pgflineto{\pgfpoint{248.937485pt}{32.500008pt}}
\pgfusepath{stroke}
\pgfpathmoveto{\pgfpoint{248.937485pt}{182.296371pt}}
\pgflineto{\pgfpoint{248.937485pt}{185.000000pt}}
\pgfusepath{stroke}
\pgfpathmoveto{\pgfpoint{316.749969pt}{35.203644pt}}
\pgflineto{\pgfpoint{316.749969pt}{32.500008pt}}
\pgfusepath{stroke}
\pgfpathmoveto{\pgfpoint{316.749969pt}{182.296371pt}}
\pgflineto{\pgfpoint{316.749969pt}{185.000000pt}}
\pgfusepath{stroke}
{
\pgftransformshift{\pgfpoint{45.500000pt}{27.024506pt}}
\pgfnode{rectangle}{north}{\fontsize{8}{0}\selectfont{{0}}}{}{\pgfusepath{discard}}}
{
\pgftransformshift{\pgfpoint{113.312500pt}{27.024506pt}}
\pgfnode{rectangle}{north}{\fontsize{8}{0}\selectfont{{50}}}{}{\pgfusepath{discard}}}
{
\pgftransformshift{\pgfpoint{181.125000pt}{27.024506pt}}
\pgfnode{rectangle}{north}{\fontsize{8}{0}\selectfont{{100}}}{}{\pgfusepath{discard}}}
{
\pgftransformshift{\pgfpoint{248.937500pt}{27.024506pt}}
\pgfnode{rectangle}{north}{\fontsize{8}{0}\selectfont{{150}}}{}{\pgfusepath{discard}}}
{
\pgftransformshift{\pgfpoint{316.750000pt}{27.024506pt}}
\pgfnode{rectangle}{north}{\fontsize{8}{0}\selectfont{{200}}}{}{\pgfusepath{discard}}}
\pgfpathmoveto{\pgfpoint{48.215004pt}{32.500008pt}}
\pgflineto{\pgfpoint{45.500000pt}{32.500008pt}}
\pgfusepath{stroke}
\pgfpathmoveto{\pgfpoint{314.034973pt}{32.500008pt}}
\pgflineto{\pgfpoint{316.749969pt}{32.500008pt}}
\pgfusepath{stroke}
\pgfpathmoveto{\pgfpoint{48.215004pt}{63.000004pt}}
\pgflineto{\pgfpoint{45.500000pt}{63.000004pt}}
\pgfusepath{stroke}
\pgfpathmoveto{\pgfpoint{314.034973pt}{63.000004pt}}
\pgflineto{\pgfpoint{316.749969pt}{63.000004pt}}
\pgfusepath{stroke}
\pgfpathmoveto{\pgfpoint{48.215004pt}{93.500008pt}}
\pgflineto{\pgfpoint{45.500000pt}{93.500008pt}}
\pgfusepath{stroke}
\pgfpathmoveto{\pgfpoint{314.034973pt}{93.500008pt}}
\pgflineto{\pgfpoint{316.749969pt}{93.500008pt}}
\pgfusepath{stroke}
\pgfpathmoveto{\pgfpoint{48.215004pt}{124.000008pt}}
\pgflineto{\pgfpoint{45.500000pt}{124.000008pt}}
\pgfusepath{stroke}
\pgfpathmoveto{\pgfpoint{314.034973pt}{124.000008pt}}
\pgflineto{\pgfpoint{316.749969pt}{124.000008pt}}
\pgfusepath{stroke}
\pgfpathmoveto{\pgfpoint{48.215004pt}{154.500000pt}}
\pgflineto{\pgfpoint{45.500000pt}{154.500000pt}}
\pgfusepath{stroke}
\pgfpathmoveto{\pgfpoint{314.034973pt}{154.500000pt}}
\pgflineto{\pgfpoint{316.749969pt}{154.500000pt}}
\pgfusepath{stroke}
\pgfpathmoveto{\pgfpoint{48.215004pt}{185.000000pt}}
\pgflineto{\pgfpoint{45.500000pt}{185.000000pt}}
\pgfusepath{stroke}
\pgfpathmoveto{\pgfpoint{314.034973pt}{185.000000pt}}
\pgflineto{\pgfpoint{316.749969pt}{185.000000pt}}
\pgfusepath{stroke}
{
\pgftransformshift{\pgfpoint{42.495377pt}{32.500008pt}}
\pgfnode{rectangle}{east}{\fontsize{8}{0}\selectfont{{0}}}{}{\pgfusepath{discard}}}
{
\pgftransformshift{\pgfpoint{42.495377pt}{63.000004pt}}
\pgfnode{rectangle}{east}{\fontsize{8}{0}\selectfont{{0.05}}}{}{\pgfusepath{discard}}}
{
\pgftransformshift{\pgfpoint{42.495377pt}{93.500008pt}}
\pgfnode{rectangle}{east}{\fontsize{8}{0}\selectfont{{0.1}}}{}{\pgfusepath{discard}}}
{
\pgftransformshift{\pgfpoint{42.495377pt}{124.000008pt}}
\pgfnode{rectangle}{east}{\fontsize{8}{0}\selectfont{{0.15}}}{}{\pgfusepath{discard}}}
{
\pgftransformshift{\pgfpoint{42.495377pt}{154.500000pt}}
\pgfnode{rectangle}{east}{\fontsize{8}{0}\selectfont{{0.2}}}{}{\pgfusepath{discard}}}
{
\pgftransformshift{\pgfpoint{42.495377pt}{185.000000pt}}
\pgfnode{rectangle}{east}{\fontsize{8}{0}\selectfont{{0.25}}}{}{\pgfusepath{discard}}}
\pgfsetrectcap
\pgfsetdash{{16pt}{0pt}}{0pt}
\pgfpathmoveto{\pgfpoint{316.749969pt}{32.500008pt}}
\pgflineto{\pgfpoint{45.500000pt}{32.500008pt}}
\pgfusepath{stroke}
\pgfpathmoveto{\pgfpoint{316.749969pt}{185.000000pt}}
\pgflineto{\pgfpoint{45.500000pt}{185.000000pt}}
\pgfusepath{stroke}
\pgfpathmoveto{\pgfpoint{45.500000pt}{185.000000pt}}
\pgflineto{\pgfpoint{45.500000pt}{32.500008pt}}
\pgfusepath{stroke}
\pgfpathmoveto{\pgfpoint{316.749969pt}{185.000000pt}}
\pgflineto{\pgfpoint{316.749969pt}{32.500008pt}}
\pgfusepath{stroke}
{
\pgftransformshift{\pgfpoint{16.495392pt}{108.750000pt}}
\pgftransformrotate{90.000000}{\pgfnode{rectangle}{south}{\fontsize{9}{0}\selectfont{{frecuencia}}}{}{\pgfusepath{discard}}}}
{
\pgftransformshift{\pgfpoint{181.125000pt}{14.024513pt}}
\pgfnode{rectangle}{north}{\fontsize{9}{0}\selectfont{{ángulo de giro (grados)}}}{}{\pgfusepath{discard}}}
\pgfsetlinewidth{0.01pt}
\color[rgb]{1.000000,1.000000,1.000000}
\pgfpathmoveto{\pgfpoint{66.169388pt}{32.500008pt}}
\pgflineto{\pgfpoint{66.169388pt}{33.535660pt}}
\pgflineto{\pgfpoint{68.900536pt}{32.500008pt}}
\pgfpathclose
\pgfusepath{fill,stroke}
\pgfpathmoveto{\pgfpoint{68.900536pt}{33.535660pt}}
\pgflineto{\pgfpoint{68.900536pt}{32.500008pt}}
\pgflineto{\pgfpoint{66.169388pt}{33.535660pt}}
\pgfpathclose
\pgfusepath{fill,stroke}
\pgfpathmoveto{\pgfpoint{71.631691pt}{32.500008pt}}
\pgflineto{\pgfpoint{71.631691pt}{35.606972pt}}
\pgflineto{\pgfpoint{74.362839pt}{32.500008pt}}
\pgfpathclose
\pgfusepath{fill,stroke}
\pgfpathmoveto{\pgfpoint{74.362839pt}{35.606972pt}}
\pgflineto{\pgfpoint{74.362839pt}{32.500008pt}}
\pgflineto{\pgfpoint{71.631691pt}{35.606972pt}}
\pgfpathclose
\pgfusepath{fill,stroke}
\pgfpathmoveto{\pgfpoint{74.362839pt}{32.500008pt}}
\pgflineto{\pgfpoint{74.362839pt}{37.678272pt}}
\pgflineto{\pgfpoint{77.093994pt}{32.500008pt}}
\pgfpathclose
\pgfusepath{fill,stroke}
\pgfpathmoveto{\pgfpoint{77.093994pt}{37.678272pt}}
\pgflineto{\pgfpoint{77.093994pt}{32.500008pt}}
\pgflineto{\pgfpoint{74.362839pt}{37.678272pt}}
\pgfpathclose
\pgfusepath{fill,stroke}
\pgfpathmoveto{\pgfpoint{77.093994pt}{32.500008pt}}
\pgflineto{\pgfpoint{77.093994pt}{42.856537pt}}
\pgflineto{\pgfpoint{79.825134pt}{32.500008pt}}
\pgfpathclose
\pgfusepath{fill,stroke}
\pgfpathmoveto{\pgfpoint{79.825134pt}{42.856537pt}}
\pgflineto{\pgfpoint{79.825134pt}{32.500008pt}}
\pgflineto{\pgfpoint{77.093994pt}{42.856537pt}}
\pgfpathclose
\pgfusepath{fill,stroke}
\pgfpathmoveto{\pgfpoint{79.825134pt}{32.500008pt}}
\pgflineto{\pgfpoint{79.825134pt}{46.999157pt}}
\pgflineto{\pgfpoint{82.556297pt}{32.500008pt}}
\pgfpathclose
\pgfusepath{fill,stroke}
\pgfpathmoveto{\pgfpoint{82.556297pt}{46.999157pt}}
\pgflineto{\pgfpoint{82.556297pt}{32.500008pt}}
\pgflineto{\pgfpoint{79.825134pt}{46.999157pt}}
\pgfpathclose
\pgfusepath{fill,stroke}
\pgfpathmoveto{\pgfpoint{82.556297pt}{32.500008pt}}
\pgflineto{\pgfpoint{82.556297pt}{71.854843pt}}
\pgflineto{\pgfpoint{85.287437pt}{32.500008pt}}
\pgfpathclose
\pgfusepath{fill,stroke}
\pgfpathmoveto{\pgfpoint{85.287437pt}{71.854843pt}}
\pgflineto{\pgfpoint{85.287437pt}{32.500008pt}}
\pgflineto{\pgfpoint{82.556297pt}{71.854843pt}}
\pgfpathclose
\pgfusepath{fill,stroke}
\pgfpathmoveto{\pgfpoint{85.287437pt}{32.500008pt}}
\pgflineto{\pgfpoint{85.287437pt}{94.639221pt}}
\pgflineto{\pgfpoint{88.018585pt}{32.500008pt}}
\pgfpathclose
\pgfusepath{fill,stroke}
\pgfpathmoveto{\pgfpoint{88.018585pt}{94.639221pt}}
\pgflineto{\pgfpoint{88.018585pt}{32.500008pt}}
\pgflineto{\pgfpoint{85.287437pt}{94.639221pt}}
\pgfpathclose
\pgfusepath{fill,stroke}
\pgfpathmoveto{\pgfpoint{88.018585pt}{32.500008pt}}
\pgflineto{\pgfpoint{88.018585pt}{131.922760pt}}
\pgflineto{\pgfpoint{90.749741pt}{32.500008pt}}
\pgfpathclose
\pgfusepath{fill,stroke}
\pgfpathmoveto{\pgfpoint{90.749741pt}{131.922760pt}}
\pgflineto{\pgfpoint{90.749741pt}{32.500008pt}}
\pgflineto{\pgfpoint{88.018585pt}{131.922760pt}}
\pgfpathclose
\pgfusepath{fill,stroke}
\pgfpathmoveto{\pgfpoint{90.749741pt}{32.500008pt}}
\pgflineto{\pgfpoint{90.749741pt}{181.634125pt}}
\pgflineto{\pgfpoint{93.480896pt}{32.500008pt}}
\pgfpathclose
\pgfusepath{fill,stroke}
\pgfpathmoveto{\pgfpoint{93.480896pt}{181.634125pt}}
\pgflineto{\pgfpoint{93.480896pt}{32.500008pt}}
\pgflineto{\pgfpoint{90.749741pt}{181.634125pt}}
\pgfpathclose
\pgfusepath{fill,stroke}
\pgfpathmoveto{\pgfpoint{93.480896pt}{32.500008pt}}
\pgflineto{\pgfpoint{93.480896pt}{135.029709pt}}
\pgflineto{\pgfpoint{96.212044pt}{32.500008pt}}
\pgfpathclose
\pgfusepath{fill,stroke}
\pgfpathmoveto{\pgfpoint{96.212044pt}{135.029709pt}}
\pgflineto{\pgfpoint{96.212044pt}{32.500008pt}}
\pgflineto{\pgfpoint{93.480896pt}{135.029709pt}}
\pgfpathclose
\pgfusepath{fill,stroke}
\pgfpathmoveto{\pgfpoint{96.212044pt}{32.500008pt}}
\pgflineto{\pgfpoint{96.212044pt}{92.567917pt}}
\pgflineto{\pgfpoint{98.943199pt}{32.500008pt}}
\pgfpathclose
\pgfusepath{fill,stroke}
\pgfpathmoveto{\pgfpoint{98.943199pt}{92.567917pt}}
\pgflineto{\pgfpoint{98.943199pt}{32.500008pt}}
\pgflineto{\pgfpoint{96.212044pt}{92.567917pt}}
\pgfpathclose
\pgfusepath{fill,stroke}
\pgfpathmoveto{\pgfpoint{98.943199pt}{32.500008pt}}
\pgflineto{\pgfpoint{98.943199pt}{67.712227pt}}
\pgflineto{\pgfpoint{101.674339pt}{32.500008pt}}
\pgfpathclose
\pgfusepath{fill,stroke}
\pgfpathmoveto{\pgfpoint{101.674339pt}{67.712227pt}}
\pgflineto{\pgfpoint{101.674339pt}{32.500008pt}}
\pgflineto{\pgfpoint{98.943199pt}{67.712227pt}}
\pgfpathclose
\pgfusepath{fill,stroke}
\pgfpathmoveto{\pgfpoint{101.674339pt}{32.500008pt}}
\pgflineto{\pgfpoint{101.674339pt}{44.927849pt}}
\pgflineto{\pgfpoint{104.405495pt}{32.500008pt}}
\pgfpathclose
\pgfusepath{fill,stroke}
\pgfpathmoveto{\pgfpoint{104.405495pt}{44.927849pt}}
\pgflineto{\pgfpoint{104.405495pt}{32.500008pt}}
\pgflineto{\pgfpoint{101.674339pt}{44.927849pt}}
\pgfpathclose
\pgfusepath{fill,stroke}
\pgfpathmoveto{\pgfpoint{104.405495pt}{32.500008pt}}
\pgflineto{\pgfpoint{104.405495pt}{37.678272pt}}
\pgflineto{\pgfpoint{107.136650pt}{32.500008pt}}
\pgfpathclose
\pgfusepath{fill,stroke}
\pgfpathmoveto{\pgfpoint{107.136650pt}{37.678272pt}}
\pgflineto{\pgfpoint{107.136650pt}{32.500008pt}}
\pgflineto{\pgfpoint{104.405495pt}{37.678272pt}}
\pgfpathclose
\pgfusepath{fill,stroke}
\pgfpathmoveto{\pgfpoint{107.136650pt}{32.500008pt}}
\pgflineto{\pgfpoint{107.136650pt}{34.571312pt}}
\pgflineto{\pgfpoint{109.867805pt}{32.500008pt}}
\pgfpathclose
\pgfusepath{fill,stroke}
\pgfpathmoveto{\pgfpoint{109.867805pt}{34.571312pt}}
\pgflineto{\pgfpoint{109.867805pt}{32.500008pt}}
\pgflineto{\pgfpoint{107.136650pt}{34.571312pt}}
\pgfpathclose
\pgfusepath{fill,stroke}
\pgfpathmoveto{\pgfpoint{112.598953pt}{32.500008pt}}
\pgflineto{\pgfpoint{112.598953pt}{33.535660pt}}
\pgflineto{\pgfpoint{115.330109pt}{32.500008pt}}
\pgfpathclose
\pgfusepath{fill,stroke}
\pgfpathmoveto{\pgfpoint{115.330109pt}{33.535660pt}}
\pgflineto{\pgfpoint{115.330109pt}{32.500008pt}}
\pgflineto{\pgfpoint{112.598953pt}{33.535660pt}}
\pgfpathclose
\pgfusepath{fill,stroke}
\pgfpathmoveto{\pgfpoint{123.523560pt}{32.500008pt}}
\pgflineto{\pgfpoint{123.523560pt}{33.535660pt}}
\pgflineto{\pgfpoint{126.254715pt}{32.500008pt}}
\pgfpathclose
\pgfusepath{fill,stroke}
\pgfpathmoveto{\pgfpoint{126.254715pt}{33.535660pt}}
\pgflineto{\pgfpoint{126.254715pt}{32.500008pt}}
\pgflineto{\pgfpoint{123.523560pt}{33.535660pt}}
\pgfpathclose
\pgfusepath{fill,stroke}
\pgfpathmoveto{\pgfpoint{128.985870pt}{32.500008pt}}
\pgflineto{\pgfpoint{128.985870pt}{34.571312pt}}
\pgflineto{\pgfpoint{131.717010pt}{32.500008pt}}
\pgfpathclose
\pgfusepath{fill,stroke}
\pgfpathmoveto{\pgfpoint{131.717010pt}{34.571312pt}}
\pgflineto{\pgfpoint{131.717010pt}{32.500008pt}}
\pgflineto{\pgfpoint{128.985870pt}{34.571312pt}}
\pgfpathclose
\pgfusepath{fill,stroke}
\pgfpathmoveto{\pgfpoint{150.835068pt}{32.500008pt}}
\pgflineto{\pgfpoint{150.835068pt}{33.535660pt}}
\pgflineto{\pgfpoint{153.566208pt}{32.500008pt}}
\pgfpathclose
\pgfusepath{fill,stroke}
\pgfpathmoveto{\pgfpoint{153.566208pt}{33.535660pt}}
\pgflineto{\pgfpoint{153.566208pt}{32.500008pt}}
\pgflineto{\pgfpoint{150.835068pt}{33.535660pt}}
\pgfpathclose
\pgfusepath{fill,stroke}
\pgfpathmoveto{\pgfpoint{164.490814pt}{32.500008pt}}
\pgflineto{\pgfpoint{164.490814pt}{34.571312pt}}
\pgflineto{\pgfpoint{167.221985pt}{32.500008pt}}
\pgfpathclose
\pgfusepath{fill,stroke}
\pgfpathmoveto{\pgfpoint{167.221985pt}{34.571312pt}}
\pgflineto{\pgfpoint{167.221985pt}{32.500008pt}}
\pgflineto{\pgfpoint{164.490814pt}{34.571312pt}}
\pgfpathclose
\pgfusepath{fill,stroke}
\pgfpathmoveto{\pgfpoint{251.887665pt}{32.500008pt}}
\pgflineto{\pgfpoint{251.887665pt}{33.535660pt}}
\pgflineto{\pgfpoint{254.618805pt}{32.500008pt}}
\pgfpathclose
\pgfusepath{fill,stroke}
\pgfpathmoveto{\pgfpoint{254.618805pt}{33.535660pt}}
\pgflineto{\pgfpoint{254.618805pt}{32.500008pt}}
\pgflineto{\pgfpoint{251.887665pt}{33.535660pt}}
\pgfpathclose
\pgfusepath{fill,stroke}
\color[rgb]{0,0,0}
\pgfsetlinewidth{0.500000pt}
\pgfsetbuttcap
\pgfsetdash{}{0pt}
\pgfpathmoveto{\pgfpoint{66.169388pt}{33.535660pt}}
\pgflineto{\pgfpoint{68.900536pt}{33.535660pt}}
\pgfusepath{stroke}
\pgfpathmoveto{\pgfpoint{66.169388pt}{32.500008pt}}
\pgflineto{\pgfpoint{66.169388pt}{33.535660pt}}
\pgfusepath{stroke}
\pgfpathmoveto{\pgfpoint{68.900536pt}{32.500008pt}}
\pgflineto{\pgfpoint{66.169388pt}{32.500008pt}}
\pgfusepath{stroke}
\pgfpathmoveto{\pgfpoint{68.900536pt}{33.535660pt}}
\pgflineto{\pgfpoint{68.900536pt}{32.500008pt}}
\pgfusepath{stroke}
\pgfpathmoveto{\pgfpoint{71.631691pt}{35.606972pt}}
\pgflineto{\pgfpoint{74.362839pt}{35.606972pt}}
\pgfusepath{stroke}
\pgfpathmoveto{\pgfpoint{71.631691pt}{32.500008pt}}
\pgflineto{\pgfpoint{71.631691pt}{35.606972pt}}
\pgfusepath{stroke}
\pgfpathmoveto{\pgfpoint{74.362839pt}{32.500008pt}}
\pgflineto{\pgfpoint{71.631691pt}{32.500008pt}}
\pgfusepath{stroke}
\pgfpathmoveto{\pgfpoint{74.362839pt}{35.606972pt}}
\pgflineto{\pgfpoint{74.362839pt}{32.500008pt}}
\pgfusepath{stroke}
\pgfpathmoveto{\pgfpoint{74.362839pt}{37.678272pt}}
\pgflineto{\pgfpoint{77.093994pt}{37.678272pt}}
\pgfusepath{stroke}
\pgfpathmoveto{\pgfpoint{74.362839pt}{32.500008pt}}
\pgflineto{\pgfpoint{74.362839pt}{37.678272pt}}
\pgfusepath{stroke}
\pgfpathmoveto{\pgfpoint{77.093994pt}{32.500008pt}}
\pgflineto{\pgfpoint{74.362839pt}{32.500008pt}}
\pgfusepath{stroke}
\pgfpathmoveto{\pgfpoint{77.093994pt}{37.678272pt}}
\pgflineto{\pgfpoint{77.093994pt}{32.500008pt}}
\pgfusepath{stroke}
\pgfpathmoveto{\pgfpoint{77.093994pt}{42.856537pt}}
\pgflineto{\pgfpoint{79.825134pt}{42.856537pt}}
\pgfusepath{stroke}
\pgfpathmoveto{\pgfpoint{77.093994pt}{32.500008pt}}
\pgflineto{\pgfpoint{77.093994pt}{42.856537pt}}
\pgfusepath{stroke}
\pgfpathmoveto{\pgfpoint{79.825134pt}{32.500008pt}}
\pgflineto{\pgfpoint{77.093994pt}{32.500008pt}}
\pgfusepath{stroke}
\pgfpathmoveto{\pgfpoint{79.825134pt}{42.856537pt}}
\pgflineto{\pgfpoint{79.825134pt}{32.500008pt}}
\pgfusepath{stroke}
\pgfpathmoveto{\pgfpoint{79.825134pt}{46.999157pt}}
\pgflineto{\pgfpoint{82.556297pt}{46.999157pt}}
\pgfusepath{stroke}
\pgfpathmoveto{\pgfpoint{79.825134pt}{32.500008pt}}
\pgflineto{\pgfpoint{79.825134pt}{46.999157pt}}
\pgfusepath{stroke}
\pgfpathmoveto{\pgfpoint{82.556297pt}{32.500008pt}}
\pgflineto{\pgfpoint{79.825134pt}{32.500008pt}}
\pgfusepath{stroke}
\pgfpathmoveto{\pgfpoint{82.556297pt}{46.999157pt}}
\pgflineto{\pgfpoint{82.556297pt}{32.500008pt}}
\pgfusepath{stroke}
\pgfpathmoveto{\pgfpoint{82.556297pt}{71.854843pt}}
\pgflineto{\pgfpoint{85.287437pt}{71.854843pt}}
\pgfusepath{stroke}
\pgfpathmoveto{\pgfpoint{82.556297pt}{32.500008pt}}
\pgflineto{\pgfpoint{82.556297pt}{71.854843pt}}
\pgfusepath{stroke}
\pgfpathmoveto{\pgfpoint{85.287437pt}{32.500008pt}}
\pgflineto{\pgfpoint{82.556297pt}{32.500008pt}}
\pgfusepath{stroke}
\pgfpathmoveto{\pgfpoint{85.287437pt}{71.854843pt}}
\pgflineto{\pgfpoint{85.287437pt}{32.500008pt}}
\pgfusepath{stroke}
\pgfpathmoveto{\pgfpoint{85.287437pt}{94.639221pt}}
\pgflineto{\pgfpoint{88.018585pt}{94.639221pt}}
\pgfusepath{stroke}
\pgfpathmoveto{\pgfpoint{85.287437pt}{32.500008pt}}
\pgflineto{\pgfpoint{85.287437pt}{94.639221pt}}
\pgfusepath{stroke}
\pgfpathmoveto{\pgfpoint{88.018585pt}{32.500008pt}}
\pgflineto{\pgfpoint{85.287437pt}{32.500008pt}}
\pgfusepath{stroke}
\pgfpathmoveto{\pgfpoint{88.018585pt}{94.639221pt}}
\pgflineto{\pgfpoint{88.018585pt}{32.500008pt}}
\pgfusepath{stroke}
\pgfpathmoveto{\pgfpoint{88.018585pt}{131.922760pt}}
\pgflineto{\pgfpoint{90.749741pt}{131.922760pt}}
\pgfusepath{stroke}
\pgfpathmoveto{\pgfpoint{88.018585pt}{32.500008pt}}
\pgflineto{\pgfpoint{88.018585pt}{131.922760pt}}
\pgfusepath{stroke}
\pgfpathmoveto{\pgfpoint{90.749741pt}{32.500008pt}}
\pgflineto{\pgfpoint{88.018585pt}{32.500008pt}}
\pgfusepath{stroke}
\pgfpathmoveto{\pgfpoint{90.749741pt}{131.922760pt}}
\pgflineto{\pgfpoint{90.749741pt}{32.500008pt}}
\pgfusepath{stroke}
\pgfpathmoveto{\pgfpoint{90.749741pt}{181.634125pt}}
\pgflineto{\pgfpoint{93.480896pt}{181.634125pt}}
\pgfusepath{stroke}
\pgfpathmoveto{\pgfpoint{90.749741pt}{32.500008pt}}
\pgflineto{\pgfpoint{90.749741pt}{181.634125pt}}
\pgfusepath{stroke}
\pgfpathmoveto{\pgfpoint{93.480896pt}{32.500008pt}}
\pgflineto{\pgfpoint{90.749741pt}{32.500008pt}}
\pgfusepath{stroke}
\pgfpathmoveto{\pgfpoint{93.480896pt}{181.634125pt}}
\pgflineto{\pgfpoint{93.480896pt}{32.500008pt}}
\pgfusepath{stroke}
\pgfpathmoveto{\pgfpoint{93.480896pt}{135.029709pt}}
\pgflineto{\pgfpoint{96.212044pt}{135.029709pt}}
\pgfusepath{stroke}
\pgfpathmoveto{\pgfpoint{93.480896pt}{32.500008pt}}
\pgflineto{\pgfpoint{93.480896pt}{135.029709pt}}
\pgfusepath{stroke}
\pgfpathmoveto{\pgfpoint{96.212044pt}{32.500008pt}}
\pgflineto{\pgfpoint{93.480896pt}{32.500008pt}}
\pgfusepath{stroke}
\pgfpathmoveto{\pgfpoint{96.212044pt}{135.029709pt}}
\pgflineto{\pgfpoint{96.212044pt}{32.500008pt}}
\pgfusepath{stroke}
\pgfpathmoveto{\pgfpoint{96.212044pt}{92.567917pt}}
\pgflineto{\pgfpoint{98.943199pt}{92.567917pt}}
\pgfusepath{stroke}
\pgfpathmoveto{\pgfpoint{96.212044pt}{32.500008pt}}
\pgflineto{\pgfpoint{96.212044pt}{92.567917pt}}
\pgfusepath{stroke}
\pgfpathmoveto{\pgfpoint{98.943199pt}{32.500008pt}}
\pgflineto{\pgfpoint{96.212044pt}{32.500008pt}}
\pgfusepath{stroke}
\pgfpathmoveto{\pgfpoint{98.943199pt}{92.567917pt}}
\pgflineto{\pgfpoint{98.943199pt}{32.500008pt}}
\pgfusepath{stroke}
\pgfpathmoveto{\pgfpoint{98.943199pt}{67.712227pt}}
\pgflineto{\pgfpoint{101.674339pt}{67.712227pt}}
\pgfusepath{stroke}
\pgfpathmoveto{\pgfpoint{98.943199pt}{32.500008pt}}
\pgflineto{\pgfpoint{98.943199pt}{67.712227pt}}
\pgfusepath{stroke}
\pgfpathmoveto{\pgfpoint{101.674339pt}{32.500008pt}}
\pgflineto{\pgfpoint{98.943199pt}{32.500008pt}}
\pgfusepath{stroke}
\pgfpathmoveto{\pgfpoint{101.674339pt}{67.712227pt}}
\pgflineto{\pgfpoint{101.674339pt}{32.500008pt}}
\pgfusepath{stroke}
\pgfpathmoveto{\pgfpoint{101.674339pt}{44.927849pt}}
\pgflineto{\pgfpoint{104.405495pt}{44.927849pt}}
\pgfusepath{stroke}
\pgfpathmoveto{\pgfpoint{101.674339pt}{32.500008pt}}
\pgflineto{\pgfpoint{101.674339pt}{44.927849pt}}
\pgfusepath{stroke}
\pgfpathmoveto{\pgfpoint{104.405495pt}{32.500008pt}}
\pgflineto{\pgfpoint{101.674339pt}{32.500008pt}}
\pgfusepath{stroke}
\pgfpathmoveto{\pgfpoint{104.405495pt}{44.927849pt}}
\pgflineto{\pgfpoint{104.405495pt}{32.500008pt}}
\pgfusepath{stroke}
\pgfpathmoveto{\pgfpoint{104.405495pt}{37.678272pt}}
\pgflineto{\pgfpoint{107.136650pt}{37.678272pt}}
\pgfusepath{stroke}
\pgfpathmoveto{\pgfpoint{104.405495pt}{32.500008pt}}
\pgflineto{\pgfpoint{104.405495pt}{37.678272pt}}
\pgfusepath{stroke}
\pgfpathmoveto{\pgfpoint{107.136650pt}{32.500008pt}}
\pgflineto{\pgfpoint{104.405495pt}{32.500008pt}}
\pgfusepath{stroke}
\pgfpathmoveto{\pgfpoint{107.136650pt}{37.678272pt}}
\pgflineto{\pgfpoint{107.136650pt}{32.500008pt}}
\pgfusepath{stroke}
\pgfpathmoveto{\pgfpoint{107.136650pt}{34.571312pt}}
\pgflineto{\pgfpoint{109.867805pt}{34.571312pt}}
\pgfusepath{stroke}
\pgfpathmoveto{\pgfpoint{107.136650pt}{32.500008pt}}
\pgflineto{\pgfpoint{107.136650pt}{34.571312pt}}
\pgfusepath{stroke}
\pgfpathmoveto{\pgfpoint{109.867805pt}{32.500008pt}}
\pgflineto{\pgfpoint{107.136650pt}{32.500008pt}}
\pgfusepath{stroke}
\pgfpathmoveto{\pgfpoint{109.867805pt}{34.571312pt}}
\pgflineto{\pgfpoint{109.867805pt}{32.500008pt}}
\pgfusepath{stroke}
\pgfpathmoveto{\pgfpoint{112.598953pt}{33.535660pt}}
\pgflineto{\pgfpoint{115.330109pt}{33.535660pt}}
\pgfusepath{stroke}
\pgfpathmoveto{\pgfpoint{112.598953pt}{32.500008pt}}
\pgflineto{\pgfpoint{112.598953pt}{33.535660pt}}
\pgfusepath{stroke}
\pgfpathmoveto{\pgfpoint{115.330109pt}{32.500008pt}}
\pgflineto{\pgfpoint{112.598953pt}{32.500008pt}}
\pgfusepath{stroke}
\pgfpathmoveto{\pgfpoint{115.330109pt}{33.535660pt}}
\pgflineto{\pgfpoint{115.330109pt}{32.500008pt}}
\pgfusepath{stroke}
\pgfpathmoveto{\pgfpoint{123.523560pt}{33.535660pt}}
\pgflineto{\pgfpoint{126.254715pt}{33.535660pt}}
\pgfusepath{stroke}
\pgfpathmoveto{\pgfpoint{123.523560pt}{32.500008pt}}
\pgflineto{\pgfpoint{123.523560pt}{33.535660pt}}
\pgfusepath{stroke}
\pgfpathmoveto{\pgfpoint{126.254715pt}{32.500008pt}}
\pgflineto{\pgfpoint{123.523560pt}{32.500008pt}}
\pgfusepath{stroke}
\pgfpathmoveto{\pgfpoint{126.254715pt}{33.535660pt}}
\pgflineto{\pgfpoint{126.254715pt}{32.500008pt}}
\pgfusepath{stroke}
\pgfpathmoveto{\pgfpoint{128.985870pt}{34.571312pt}}
\pgflineto{\pgfpoint{131.717010pt}{34.571312pt}}
\pgfusepath{stroke}
\pgfpathmoveto{\pgfpoint{128.985870pt}{32.500008pt}}
\pgflineto{\pgfpoint{128.985870pt}{34.571312pt}}
\pgfusepath{stroke}
\pgfpathmoveto{\pgfpoint{131.717010pt}{32.500008pt}}
\pgflineto{\pgfpoint{128.985870pt}{32.500008pt}}
\pgfusepath{stroke}
\pgfpathmoveto{\pgfpoint{131.717010pt}{34.571312pt}}
\pgflineto{\pgfpoint{131.717010pt}{32.500008pt}}
\pgfusepath{stroke}
\pgfpathmoveto{\pgfpoint{150.835068pt}{33.535660pt}}
\pgflineto{\pgfpoint{153.566208pt}{33.535660pt}}
\pgfusepath{stroke}
\pgfpathmoveto{\pgfpoint{150.835068pt}{32.500008pt}}
\pgflineto{\pgfpoint{150.835068pt}{33.535660pt}}
\pgfusepath{stroke}
\pgfpathmoveto{\pgfpoint{153.566208pt}{32.500008pt}}
\pgflineto{\pgfpoint{150.835068pt}{32.500008pt}}
\pgfusepath{stroke}
\pgfpathmoveto{\pgfpoint{153.566208pt}{33.535660pt}}
\pgflineto{\pgfpoint{153.566208pt}{32.500008pt}}
\pgfusepath{stroke}
\pgfpathmoveto{\pgfpoint{164.490814pt}{34.571312pt}}
\pgflineto{\pgfpoint{167.221985pt}{34.571312pt}}
\pgfusepath{stroke}
\pgfpathmoveto{\pgfpoint{164.490814pt}{32.500008pt}}
\pgflineto{\pgfpoint{164.490814pt}{34.571312pt}}
\pgfusepath{stroke}
\pgfpathmoveto{\pgfpoint{167.221985pt}{32.500008pt}}
\pgflineto{\pgfpoint{164.490814pt}{32.500008pt}}
\pgfusepath{stroke}
\pgfpathmoveto{\pgfpoint{167.221985pt}{34.571312pt}}
\pgflineto{\pgfpoint{167.221985pt}{32.500008pt}}
\pgfusepath{stroke}
\pgfpathmoveto{\pgfpoint{251.887665pt}{33.535660pt}}
\pgflineto{\pgfpoint{254.618805pt}{33.535660pt}}
\pgfusepath{stroke}
\pgfpathmoveto{\pgfpoint{251.887665pt}{32.500008pt}}
\pgflineto{\pgfpoint{251.887665pt}{33.535660pt}}
\pgfusepath{stroke}
\pgfpathmoveto{\pgfpoint{254.618805pt}{32.500008pt}}
\pgflineto{\pgfpoint{251.887665pt}{32.500008pt}}
\pgfusepath{stroke}
\pgfpathmoveto{\pgfpoint{254.618805pt}{33.535660pt}}
\pgflineto{\pgfpoint{254.618805pt}{32.500008pt}}
\pgfusepath{stroke}
\pgfsetroundjoin
\pgfpathmoveto{\pgfpoint{316.749969pt}{32.500008pt}}
\pgflineto{\pgfpoint{45.500000pt}{32.500008pt}}
\pgfusepath{stroke}
%\end{pgfscope}
%\end{pgfpicture}

				\end{tikzpicture}
				%\caption{Ángulos sobre el eje $y$.}
			\end{subfigure}
			\begin{subfigure}{\linewidth}
				% Title: gl2ps_renderer figure
% Creator: GL2PS 1.4.0, (C) 1999-2017 C. Geuzaine
% For: Octave
% CreationDate: Wed Feb 26 12:22:37 2020
\begin{pgfpicture}
\color[rgb]{1.000000,1.000000,1.000000}
\pgfpathrectanglecorners{\pgfpoint{0.0pt}{0.0pt}}{\pgfpoint{350.0pt}{262.5pt}}
\pgfusepath{fill}
\begin{pgfscope}
\pgfpathrectangle{\pgfpoint{0.0pt}{0.0pt}}{\pgfpoint{350.0pt}{262.5pt}}
\pgfusepath{fill}
\pgfpathrectangle{\pgfpoint{0.0pt}{0.0pt}}{\pgfpoint{350.0pt}{262.5pt}}
\pgfusepath{clip}
\pgfpathmoveto{\pgfpoint{74.15640373263889pt}{242.81244835069444pt}}
\pgflineto{\pgfpoint{288.09378402777776pt}{28.874974479166667pt}}
\pgflineto{\pgfpoint{74.15640373263889pt}{28.874974479166667pt}}
\pgfpathclose
\pgfusepath{fill,stroke}
\pgfpathmoveto{\pgfpoint{74.15640373263889pt}{242.81244835069444pt}}
\pgflineto{\pgfpoint{288.09378402777776pt}{242.81244835069444pt}}
\pgflineto{\pgfpoint{288.09378402777776pt}{28.874974479166667pt}}
\pgfpathclose
\pgfusepath{fill,stroke}
\color[rgb]{0,0,0}
\pgfsetlinewidth{0.3038194444444444pt}
\pgfpathmoveto{\pgfpoint{114.39480755208332pt}{31.58808940972222pt}}
\pgflineto{\pgfpoint{114.39480755208332pt}{28.874974479166667pt}}
\pgfusepath{stroke}
\pgfpathmoveto{\pgfpoint{167.42055434027776pt}{31.58808940972222pt}}
\pgflineto{\pgfpoint{167.42055434027776pt}{28.874974479166667pt}}
\pgfusepath{stroke}
\pgfpathmoveto{\pgfpoint{220.44630052083332pt}{31.58808940972222pt}}
\pgflineto{\pgfpoint{220.44630052083332pt}{28.874974479166667pt}}
\pgfusepath{stroke}
\pgfpathmoveto{\pgfpoint{273.47195434027776pt}{31.58808940972222pt}}
\pgflineto{\pgfpoint{273.47195434027776pt}{28.874974479166667pt}}
\pgfusepath{stroke}
{
\pgftransformshift{\pgfpoint{114.3947984375pt}{24.316635243055554pt}}
\pgfnode{rectangle}{north}{\fontsize{8}{0}\selectfont{{-0.054}}}{}{\pgfusepath{discard}}}
{
\pgftransformshift{\pgfpoint{167.42051727430555pt}{24.316635243055554pt}}
\pgfnode{rectangle}{north}{\fontsize{8}{0}\selectfont{{-0.052}}}{}{\pgfusepath{discard}}}
{
\pgftransformshift{\pgfpoint{220.44624461805554pt}{24.316635243055554pt}}
\pgfnode{rectangle}{north}{\fontsize{8}{0}\selectfont{{-0.05}}}{}{\pgfusepath{discard}}}
{
\pgftransformshift{\pgfpoint{273.4718613715278pt}{24.316635243055554pt}}
\pgfnode{rectangle}{north}{\fontsize{8}{0}\selectfont{{-0.048}}}{}{\pgfusepath{discard}}}
\pgfpathmoveto{\pgfpoint{76.86941597222221pt}{28.874974479166667pt}}
\pgflineto{\pgfpoint{74.15640373263889pt}{28.874974479166667pt}}
\pgfusepath{stroke}
\pgfpathmoveto{\pgfpoint{76.86941597222221pt}{71.66247022569443pt}}
\pgflineto{\pgfpoint{74.15640373263889pt}{71.66247022569443pt}}
\pgfusepath{stroke}
\pgfpathmoveto{\pgfpoint{76.86941597222221pt}{114.44996597222222pt}}
\pgflineto{\pgfpoint{74.15640373263889pt}{114.44996597222222pt}}
\pgfusepath{stroke}
\pgfpathmoveto{\pgfpoint{76.86941597222221pt}{157.2374659722222pt}}
\pgflineto{\pgfpoint{74.15640373263889pt}{157.2374659722222pt}}
\pgfusepath{stroke}
\pgfpathmoveto{\pgfpoint{76.86941597222221pt}{200.02499392361108pt}}
\pgflineto{\pgfpoint{74.15640373263889pt}{200.02499392361108pt}}
\pgfusepath{stroke}
\pgfpathmoveto{\pgfpoint{76.86941597222221pt}{242.81244835069444pt}}
\pgflineto{\pgfpoint{74.15640373263889pt}{242.81244835069444pt}}
\pgfusepath{stroke}
{
\pgftransformshift{\pgfpoint{71.1173743923611pt}{28.87493741319444pt}}
\pgfnode{rectangle}{east}{\fontsize{8}{0}\selectfont{{-0.013}}}{}{\pgfusepath{discard}}}
{
\pgftransformshift{\pgfpoint{71.1173743923611pt}{71.66245199652776pt}}
\pgfnode{rectangle}{east}{\fontsize{8}{0}\selectfont{{-0.012}}}{}{\pgfusepath{discard}}}
{
\pgftransformshift{\pgfpoint{71.1173743923611pt}{114.44996597222222pt}}
\pgfnode{rectangle}{east}{\fontsize{8}{0}\selectfont{{-0.011}}}{}{\pgfusepath{discard}}}
{
\pgftransformshift{\pgfpoint{71.1173743923611pt}{157.2374659722222pt}}
\pgfnode{rectangle}{east}{\fontsize{8}{0}\selectfont{{-0.01}}}{}{\pgfusepath{discard}}}
{
\pgftransformshift{\pgfpoint{71.1173743923611pt}{200.02499392361108pt}}
\pgfnode{rectangle}{east}{\fontsize{8}{0}\selectfont{{-0.009}}}{}{\pgfusepath{discard}}}
{
\pgftransformshift{\pgfpoint{71.1173743923611pt}{242.81248541666668pt}}
\pgfnode{rectangle}{east}{\fontsize{8}{0}\selectfont{{-0.008}}}{}{\pgfusepath{discard}}}
\pgfsetrectcap
\pgfsetdash{{9.722222222222221pt}{0.0pt}}{0.0pt}
\pgfpathmoveto{\pgfpoint{288.09378402777776pt}{28.874974479166667pt}}
\pgflineto{\pgfpoint{74.15640373263889pt}{28.874974479166667pt}}
\pgfusepath{stroke}
\pgfpathmoveto{\pgfpoint{74.15640373263889pt}{242.81244835069444pt}}
\pgflineto{\pgfpoint{74.15640373263889pt}{28.874974479166667pt}}
\pgfusepath{stroke}
{
\pgftransformshift{\pgfpoint{181.1251032986111pt}{16.417329687499997pt}}
\pgfnode{rectangle}{north}{\fontsize{9}{0}\selectfont{{x}}}{}{\pgfusepath{discard}}}
{
\pgftransformshift{\pgfpoint{49.85003211805555pt}{135.8437487847222pt}}
\pgftransformrotate{90.000000}{\pgfnode{rectangle}{south}{\fontsize{9}{0}\selectfont{{z}}}{}{\pgfusepath{discard}}}}
\pgfsetlinewidth{0.006076388888888888pt}
\pgfsetroundcap
\pgfsetroundjoin
\color[rgb]{0.000000,0.000000,0.000000}
\pgfpathmoveto{\pgfpoint{124.19543281249999pt}{153.84444809027778pt}}
\pgflineto{\pgfpoint{122.93583567708332pt}{152.1107540798611pt}}
\pgflineto{\pgfpoint{123.84729401041666pt}{152.77296684027777pt}}
\pgfpathclose
\pgfusepath{fill,stroke}
\pgfpathmoveto{\pgfpoint{124.19543281249999pt}{153.84444809027778pt}}
\pgflineto{\pgfpoint{121.80921484375pt}{152.1107540798611pt}}
\pgflineto{\pgfpoint{122.93583567708332pt}{152.1107540798611pt}}
\pgfpathclose
\pgfusepath{fill,stroke}
\pgfpathmoveto{\pgfpoint{124.19543281249999pt}{153.84444809027778pt}}
\pgflineto{\pgfpoint{120.89775651041666pt}{152.77296684027777pt}}
\pgflineto{\pgfpoint{121.80921484375pt}{152.1107540798611pt}}
\pgfpathclose
\pgfusepath{fill,stroke}
\pgfpathmoveto{\pgfpoint{124.19543281249999pt}{153.84444809027778pt}}
\pgflineto{\pgfpoint{120.54960859374998pt}{153.84444809027778pt}}
\pgflineto{\pgfpoint{120.89775651041666pt}{152.77296684027777pt}}
\pgfpathclose
\pgfusepath{fill,stroke}
\pgfpathmoveto{\pgfpoint{124.19543281249999pt}{153.84444809027778pt}}
\pgflineto{\pgfpoint{120.89775651041666pt}{154.9159287326389pt}}
\pgflineto{\pgfpoint{120.54960859374998pt}{153.84444809027778pt}}
\pgfpathclose
\pgfusepath{fill,stroke}
\pgfpathmoveto{\pgfpoint{124.19543281249999pt}{153.84444809027778pt}}
\pgflineto{\pgfpoint{121.80921484375pt}{155.57814149305554pt}}
\pgflineto{\pgfpoint{120.89775651041666pt}{154.9159287326389pt}}
\pgfpathclose
\pgfusepath{fill,stroke}
\pgfpathmoveto{\pgfpoint{124.19543281249999pt}{153.84444809027778pt}}
\pgflineto{\pgfpoint{122.93583567708332pt}{155.57814149305554pt}}
\pgflineto{\pgfpoint{121.80921484375pt}{155.57814149305554pt}}
\pgfpathclose
\pgfusepath{fill,stroke}
\pgfpathmoveto{\pgfpoint{124.19543281249999pt}{153.84444809027778pt}}
\pgflineto{\pgfpoint{123.84729401041666pt}{154.9159287326389pt}}
\pgflineto{\pgfpoint{122.93583567708332pt}{155.57814149305554pt}}
\pgfpathclose
\pgfusepath{fill,stroke}
\pgfpathmoveto{\pgfpoint{137.64541388888887pt}{156.77111163194442pt}}
\pgflineto{\pgfpoint{136.38580763888888pt}{155.03742673611112pt}}
\pgflineto{\pgfpoint{137.2972659722222pt}{155.69964010416666pt}}
\pgfpathclose
\pgfusepath{fill,stroke}
\pgfpathmoveto{\pgfpoint{137.64541388888887pt}{156.77111163194442pt}}
\pgflineto{\pgfpoint{135.25917769097222pt}{155.03742673611112pt}}
\pgflineto{\pgfpoint{136.38580763888888pt}{155.03742673611112pt}}
\pgfpathclose
\pgfusepath{fill,stroke}
\pgfpathmoveto{\pgfpoint{137.64541388888887pt}{156.77111163194442pt}}
\pgflineto{\pgfpoint{134.34771935763888pt}{155.69964010416666pt}}
\pgflineto{\pgfpoint{135.25917769097222pt}{155.03742673611112pt}}
\pgfpathclose
\pgfusepath{fill,stroke}
\pgfpathmoveto{\pgfpoint{137.64541388888887pt}{156.77111163194442pt}}
\pgflineto{\pgfpoint{133.99958055555555pt}{156.77111163194442pt}}
\pgflineto{\pgfpoint{134.34771935763888pt}{155.69964010416666pt}}
\pgfpathclose
\pgfusepath{fill,stroke}
\pgfpathmoveto{\pgfpoint{137.64541388888887pt}{156.77111163194442pt}}
\pgflineto{\pgfpoint{134.34771935763888pt}{157.84260138888888pt}}
\pgflineto{\pgfpoint{133.99958055555555pt}{156.77111163194442pt}}
\pgfpathclose
\pgfusepath{fill,stroke}
\pgfpathmoveto{\pgfpoint{137.64541388888887pt}{156.77111163194442pt}}
\pgflineto{\pgfpoint{135.25917769097222pt}{158.50481475694443pt}}
\pgflineto{\pgfpoint{134.34771935763888pt}{157.84260138888888pt}}
\pgfpathclose
\pgfusepath{fill,stroke}
\pgfpathmoveto{\pgfpoint{137.64541388888887pt}{156.77111163194442pt}}
\pgflineto{\pgfpoint{136.38580763888888pt}{158.50481475694443pt}}
\pgflineto{\pgfpoint{135.25917769097222pt}{158.50481475694443pt}}
\pgfpathclose
\pgfusepath{fill,stroke}
\pgfpathmoveto{\pgfpoint{137.64541388888887pt}{156.77111163194442pt}}
\pgflineto{\pgfpoint{137.2972659722222pt}{157.84260138888888pt}}
\pgflineto{\pgfpoint{136.38580763888888pt}{158.50481475694443pt}}
\pgfpathclose
\pgfusepath{fill,stroke}
\pgfpathmoveto{\pgfpoint{134.43470772569444pt}{158.0448947048611pt}}
\pgflineto{\pgfpoint{133.1750923611111pt}{156.31120980902776pt}}
\pgflineto{\pgfpoint{134.08655069444444pt}{156.97342317708333pt}}
\pgfpathclose
\pgfusepath{fill,stroke}
\pgfpathmoveto{\pgfpoint{134.43470772569444pt}{158.0448947048611pt}}
\pgflineto{\pgfpoint{132.04847152777776pt}{156.31120980902776pt}}
\pgflineto{\pgfpoint{133.1750923611111pt}{156.31120980902776pt}}
\pgfpathclose
\pgfusepath{fill,stroke}
\pgfpathmoveto{\pgfpoint{134.43470772569444pt}{158.0448947048611pt}}
\pgflineto{\pgfpoint{131.13701319444442pt}{156.97342317708333pt}}
\pgflineto{\pgfpoint{132.04847152777776pt}{156.31120980902776pt}}
\pgfpathclose
\pgfusepath{fill,stroke}
\pgfpathmoveto{\pgfpoint{134.43470772569444pt}{158.0448947048611pt}}
\pgflineto{\pgfpoint{130.7888743923611pt}{158.0448947048611pt}}
\pgflineto{\pgfpoint{131.13701319444442pt}{156.97342317708333pt}}
\pgfpathclose
\pgfusepath{fill,stroke}
\pgfpathmoveto{\pgfpoint{134.43470772569444pt}{158.0448947048611pt}}
\pgflineto{\pgfpoint{131.13701319444442pt}{159.11638446180555pt}}
\pgflineto{\pgfpoint{130.7888743923611pt}{158.0448947048611pt}}
\pgfpathclose
\pgfusepath{fill,stroke}
\pgfpathmoveto{\pgfpoint{134.43470772569444pt}{158.0448947048611pt}}
\pgflineto{\pgfpoint{132.04847152777776pt}{159.7785978298611pt}}
\pgflineto{\pgfpoint{131.13701319444442pt}{159.11638446180555pt}}
\pgfpathclose
\pgfusepath{fill,stroke}
\pgfpathmoveto{\pgfpoint{134.43470772569444pt}{158.0448947048611pt}}
\pgflineto{\pgfpoint{133.1750923611111pt}{159.7785978298611pt}}
\pgflineto{\pgfpoint{132.04847152777776pt}{159.7785978298611pt}}
\pgfpathclose
\pgfusepath{fill,stroke}
\pgfpathmoveto{\pgfpoint{134.43470772569444pt}{158.0448947048611pt}}
\pgflineto{\pgfpoint{134.08655069444444pt}{159.11638446180555pt}}
\pgflineto{\pgfpoint{133.1750923611111pt}{159.7785978298611pt}}
\pgfpathclose
\pgfusepath{fill,stroke}
\pgfpathmoveto{\pgfpoint{144.00320338541664pt}{153.2839376736111pt}}
\pgflineto{\pgfpoint{142.74359652777775pt}{151.55023454861112pt}}
\pgflineto{\pgfpoint{143.6550548611111pt}{152.21244730902777pt}}
\pgfpathclose
\pgfusepath{fill,stroke}
\pgfpathmoveto{\pgfpoint{144.00320338541664pt}{153.2839376736111pt}}
\pgflineto{\pgfpoint{141.6169671875pt}{151.55023454861112pt}}
\pgflineto{\pgfpoint{142.74359652777775pt}{151.55023454861112pt}}
\pgfpathclose
\pgfusepath{fill,stroke}
\pgfpathmoveto{\pgfpoint{144.00320338541664pt}{153.2839376736111pt}}
\pgflineto{\pgfpoint{140.70550885416665pt}{152.21244730902777pt}}
\pgflineto{\pgfpoint{141.6169671875pt}{151.55023454861112pt}}
\pgfpathclose
\pgfusepath{fill,stroke}
\pgfpathmoveto{\pgfpoint{144.00320338541664pt}{153.2839376736111pt}}
\pgflineto{\pgfpoint{140.35737005208333pt}{153.2839376736111pt}}
\pgflineto{\pgfpoint{140.70550885416665pt}{152.21244730902777pt}}
\pgfpathclose
\pgfusepath{fill,stroke}
\pgfpathmoveto{\pgfpoint{144.00320338541664pt}{153.2839376736111pt}}
\pgflineto{\pgfpoint{140.70550885416665pt}{154.3554189236111pt}}
\pgflineto{\pgfpoint{140.35737005208333pt}{153.2839376736111pt}}
\pgfpathclose
\pgfusepath{fill,stroke}
\pgfpathmoveto{\pgfpoint{144.00320338541664pt}{153.2839376736111pt}}
\pgflineto{\pgfpoint{141.6169671875pt}{155.01763168402778pt}}
\pgflineto{\pgfpoint{140.70550885416665pt}{154.3554189236111pt}}
\pgfpathclose
\pgfusepath{fill,stroke}
\pgfpathmoveto{\pgfpoint{144.00320338541664pt}{153.2839376736111pt}}
\pgflineto{\pgfpoint{142.74359652777775pt}{155.01763168402778pt}}
\pgflineto{\pgfpoint{141.6169671875pt}{155.01763168402778pt}}
\pgfpathclose
\pgfusepath{fill,stroke}
\pgfpathmoveto{\pgfpoint{144.00320338541664pt}{153.2839376736111pt}}
\pgflineto{\pgfpoint{143.6550548611111pt}{154.3554189236111pt}}
\pgflineto{\pgfpoint{142.74359652777775pt}{155.01763168402778pt}}
\pgfpathclose
\pgfusepath{fill,stroke}
\pgfpathmoveto{\pgfpoint{138.7218279513889pt}{155.1237940972222pt}}
\pgflineto{\pgfpoint{137.46222109375pt}{153.39010008680555pt}}
\pgflineto{\pgfpoint{138.37367942708332pt}{154.05230373263888pt}}
\pgfpathclose
\pgfusepath{fill,stroke}
\pgfpathmoveto{\pgfpoint{138.7218279513889pt}{155.1237940972222pt}}
\pgflineto{\pgfpoint{136.33560086805556pt}{153.39010008680555pt}}
\pgflineto{\pgfpoint{137.46222109375pt}{153.39010008680555pt}}
\pgfpathclose
\pgfusepath{fill,stroke}
\pgfpathmoveto{\pgfpoint{138.7218279513889pt}{155.1237940972222pt}}
\pgflineto{\pgfpoint{135.42414253472222pt}{154.05230373263888pt}}
\pgflineto{\pgfpoint{136.33560086805556pt}{153.39010008680555pt}}
\pgfpathclose
\pgfusepath{fill,stroke}
\pgfpathmoveto{\pgfpoint{138.7218279513889pt}{155.1237940972222pt}}
\pgflineto{\pgfpoint{135.07599461805555pt}{155.1237940972222pt}}
\pgflineto{\pgfpoint{135.42414253472222pt}{154.05230373263888pt}}
\pgfpathclose
\pgfusepath{fill,stroke}
\pgfpathmoveto{\pgfpoint{138.7218279513889pt}{155.1237940972222pt}}
\pgflineto{\pgfpoint{135.42414253472222pt}{156.1952747395833pt}}
\pgflineto{\pgfpoint{135.07599461805555pt}{155.1237940972222pt}}
\pgfpathclose
\pgfusepath{fill,stroke}
\pgfpathmoveto{\pgfpoint{138.7218279513889pt}{155.1237940972222pt}}
\pgflineto{\pgfpoint{136.33560086805556pt}{156.85748810763886pt}}
\pgflineto{\pgfpoint{135.42414253472222pt}{156.1952747395833pt}}
\pgfpathclose
\pgfusepath{fill,stroke}
\pgfpathmoveto{\pgfpoint{138.7218279513889pt}{155.1237940972222pt}}
\pgflineto{\pgfpoint{137.46222109375pt}{156.85748810763886pt}}
\pgflineto{\pgfpoint{136.33560086805556pt}{156.85748810763886pt}}
\pgfpathclose
\pgfusepath{fill,stroke}
\pgfpathmoveto{\pgfpoint{138.7218279513889pt}{155.1237940972222pt}}
\pgflineto{\pgfpoint{138.37367942708332pt}{156.1952747395833pt}}
\pgflineto{\pgfpoint{137.46222109375pt}{156.85748810763886pt}}
\pgfpathclose
\pgfusepath{fill,stroke}
\pgfpathmoveto{\pgfpoint{121.17032994791666pt}{151.88049722222223pt}}
\pgflineto{\pgfpoint{119.91071458333332pt}{150.14680381944441pt}}
\pgflineto{\pgfpoint{120.82218203124998pt}{150.8090165798611pt}}
\pgfpathclose
\pgfusepath{fill,stroke}
\pgfpathmoveto{\pgfpoint{121.17032994791666pt}{151.88049722222223pt}}
\pgflineto{\pgfpoint{118.78409374999998pt}{150.14680381944441pt}}
\pgflineto{\pgfpoint{119.91071458333332pt}{150.14680381944441pt}}
\pgfpathclose
\pgfusepath{fill,stroke}
\pgfpathmoveto{\pgfpoint{121.17032994791666pt}{151.88049722222223pt}}
\pgflineto{\pgfpoint{117.87263541666665pt}{150.8090165798611pt}}
\pgflineto{\pgfpoint{118.78409374999998pt}{150.14680381944441pt}}
\pgfpathclose
\pgfusepath{fill,stroke}
\pgfpathmoveto{\pgfpoint{121.17032994791666pt}{151.88049722222223pt}}
\pgflineto{\pgfpoint{117.52449661458333pt}{151.88049722222223pt}}
\pgflineto{\pgfpoint{117.87263541666665pt}{150.8090165798611pt}}
\pgfpathclose
\pgfusepath{fill,stroke}
\pgfpathmoveto{\pgfpoint{121.17032994791666pt}{151.88049722222223pt}}
\pgflineto{\pgfpoint{117.87263541666665pt}{152.95198758680556pt}}
\pgflineto{\pgfpoint{117.52449661458333pt}{151.88049722222223pt}}
\pgfpathclose
\pgfusepath{fill,stroke}
\pgfpathmoveto{\pgfpoint{121.17032994791666pt}{151.88049722222223pt}}
\pgflineto{\pgfpoint{118.78409374999998pt}{153.6142003472222pt}}
\pgflineto{\pgfpoint{117.87263541666665pt}{152.95198758680556pt}}
\pgfpathclose
\pgfusepath{fill,stroke}
\pgfpathmoveto{\pgfpoint{121.17032994791666pt}{151.88049722222223pt}}
\pgflineto{\pgfpoint{119.91071458333332pt}{153.6142003472222pt}}
\pgflineto{\pgfpoint{118.78409374999998pt}{153.6142003472222pt}}
\pgfpathclose
\pgfusepath{fill,stroke}
\pgfpathmoveto{\pgfpoint{121.17032994791666pt}{151.88049722222223pt}}
\pgflineto{\pgfpoint{120.82218203124998pt}{152.95198758680556pt}}
\pgflineto{\pgfpoint{119.91071458333332pt}{153.6142003472222pt}}
\pgfpathclose
\pgfusepath{fill,stroke}
\pgfpathmoveto{\pgfpoint{151.00788828125pt}{150.96057387152777pt}}
\pgflineto{\pgfpoint{149.7482814236111pt}{149.22687986111112pt}}
\pgflineto{\pgfpoint{150.65973975694442pt}{149.88909322916666pt}}
\pgfpathclose
\pgfusepath{fill,stroke}
\pgfpathmoveto{\pgfpoint{151.00788828125pt}{150.96057387152777pt}}
\pgflineto{\pgfpoint{148.62165208333332pt}{149.22687986111112pt}}
\pgflineto{\pgfpoint{149.7482814236111pt}{149.22687986111112pt}}
\pgfpathclose
\pgfusepath{fill,stroke}
\pgfpathmoveto{\pgfpoint{151.00788828125pt}{150.96057387152777pt}}
\pgflineto{\pgfpoint{147.71019374999997pt}{149.88909322916666pt}}
\pgflineto{\pgfpoint{148.62165208333332pt}{149.22687986111112pt}}
\pgfpathclose
\pgfusepath{fill,stroke}
\pgfpathmoveto{\pgfpoint{151.00788828125pt}{150.96057387152777pt}}
\pgflineto{\pgfpoint{147.36205494791665pt}{150.96057387152777pt}}
\pgflineto{\pgfpoint{147.71019374999997pt}{149.88909322916666pt}}
\pgfpathclose
\pgfusepath{fill,stroke}
\pgfpathmoveto{\pgfpoint{151.00788828125pt}{150.96057387152777pt}}
\pgflineto{\pgfpoint{147.71019374999997pt}{152.03205451388888pt}}
\pgflineto{\pgfpoint{147.36205494791665pt}{150.96057387152777pt}}
\pgfpathclose
\pgfusepath{fill,stroke}
\pgfpathmoveto{\pgfpoint{151.00788828125pt}{150.96057387152777pt}}
\pgflineto{\pgfpoint{148.62165208333332pt}{152.69426788194443pt}}
\pgflineto{\pgfpoint{147.71019374999997pt}{152.03205451388888pt}}
\pgfpathclose
\pgfusepath{fill,stroke}
\pgfpathmoveto{\pgfpoint{151.00788828125pt}{150.96057387152777pt}}
\pgflineto{\pgfpoint{149.7482814236111pt}{152.69426788194443pt}}
\pgflineto{\pgfpoint{148.62165208333332pt}{152.69426788194443pt}}
\pgfpathclose
\pgfusepath{fill,stroke}
\pgfpathmoveto{\pgfpoint{151.00788828125pt}{150.96057387152777pt}}
\pgflineto{\pgfpoint{150.65973975694442pt}{152.03205451388888pt}}
\pgflineto{\pgfpoint{149.7482814236111pt}{152.69426788194443pt}}
\pgfpathclose
\pgfusepath{fill,stroke}
\pgfpathmoveto{\pgfpoint{139.46154383680556pt}{150.62683428819443pt}}
\pgflineto{\pgfpoint{138.20193758680554pt}{148.89314027777777pt}}
\pgflineto{\pgfpoint{139.11339592013888pt}{149.5553439236111pt}}
\pgfpathclose
\pgfusepath{fill,stroke}
\pgfpathmoveto{\pgfpoint{139.46154383680556pt}{150.62683428819443pt}}
\pgflineto{\pgfpoint{137.0753076388889pt}{148.89314027777777pt}}
\pgflineto{\pgfpoint{138.20193758680554pt}{148.89314027777777pt}}
\pgfpathclose
\pgfusepath{fill,stroke}
\pgfpathmoveto{\pgfpoint{139.46154383680556pt}{150.62683428819443pt}}
\pgflineto{\pgfpoint{136.16384930555554pt}{149.5553439236111pt}}
\pgflineto{\pgfpoint{137.0753076388889pt}{148.89314027777777pt}}
\pgfpathclose
\pgfusepath{fill,stroke}
\pgfpathmoveto{\pgfpoint{139.46154383680556pt}{150.62683428819443pt}}
\pgflineto{\pgfpoint{135.81571050347222pt}{150.62683428819443pt}}
\pgflineto{\pgfpoint{136.16384930555554pt}{149.5553439236111pt}}
\pgfpathclose
\pgfusepath{fill,stroke}
\pgfpathmoveto{\pgfpoint{139.46154383680556pt}{150.62683428819443pt}}
\pgflineto{\pgfpoint{136.16384930555554pt}{151.69831493055554pt}}
\pgflineto{\pgfpoint{135.81571050347222pt}{150.62683428819443pt}}
\pgfpathclose
\pgfusepath{fill,stroke}
\pgfpathmoveto{\pgfpoint{139.46154383680556pt}{150.62683428819443pt}}
\pgflineto{\pgfpoint{137.0753076388889pt}{152.3605282986111pt}}
\pgflineto{\pgfpoint{136.16384930555554pt}{151.69831493055554pt}}
\pgfpathclose
\pgfusepath{fill,stroke}
\pgfpathmoveto{\pgfpoint{139.46154383680556pt}{150.62683428819443pt}}
\pgflineto{\pgfpoint{138.20193758680554pt}{152.3605282986111pt}}
\pgflineto{\pgfpoint{137.0753076388889pt}{152.3605282986111pt}}
\pgfpathclose
\pgfusepath{fill,stroke}
\pgfpathmoveto{\pgfpoint{139.46154383680556pt}{150.62683428819443pt}}
\pgflineto{\pgfpoint{139.11339592013888pt}{151.69831493055554pt}}
\pgflineto{\pgfpoint{138.20193758680554pt}{152.3605282986111pt}}
\pgfpathclose
\pgfusepath{fill,stroke}
\pgfpathmoveto{\pgfpoint{142.90290546875pt}{140.8627220486111pt}}
\pgflineto{\pgfpoint{141.64329921874997pt}{139.12902803819443pt}}
\pgflineto{\pgfpoint{142.55475755208332pt}{139.7912407986111pt}}
\pgfpathclose
\pgfusepath{fill,stroke}
\pgfpathmoveto{\pgfpoint{142.90290546875pt}{140.8627220486111pt}}
\pgflineto{\pgfpoint{140.51667838541664pt}{139.12902803819443pt}}
\pgflineto{\pgfpoint{141.64329921874997pt}{139.12902803819443pt}}
\pgfpathclose
\pgfusepath{fill,stroke}
\pgfpathmoveto{\pgfpoint{142.90290546875pt}{140.8627220486111pt}}
\pgflineto{\pgfpoint{139.60522005208333pt}{139.7912407986111pt}}
\pgflineto{\pgfpoint{140.51667838541664pt}{139.12902803819443pt}}
\pgfpathclose
\pgfusepath{fill,stroke}
\pgfpathmoveto{\pgfpoint{142.90290546875pt}{140.8627220486111pt}}
\pgflineto{\pgfpoint{139.25707213541665pt}{140.8627220486111pt}}
\pgflineto{\pgfpoint{139.60522005208333pt}{139.7912407986111pt}}
\pgfpathclose
\pgfusepath{fill,stroke}
\pgfpathmoveto{\pgfpoint{142.90290546875pt}{140.8627220486111pt}}
\pgflineto{\pgfpoint{139.60522005208333pt}{141.93421180555555pt}}
\pgflineto{\pgfpoint{139.25707213541665pt}{140.8627220486111pt}}
\pgfpathclose
\pgfusepath{fill,stroke}
\pgfpathmoveto{\pgfpoint{142.90290546875pt}{140.8627220486111pt}}
\pgflineto{\pgfpoint{140.51667838541664pt}{142.5964251736111pt}}
\pgflineto{\pgfpoint{139.60522005208333pt}{141.93421180555555pt}}
\pgfpathclose
\pgfusepath{fill,stroke}
\pgfpathmoveto{\pgfpoint{142.90290546875pt}{140.8627220486111pt}}
\pgflineto{\pgfpoint{141.64329921874997pt}{142.5964251736111pt}}
\pgflineto{\pgfpoint{140.51667838541664pt}{142.5964251736111pt}}
\pgfpathclose
\pgfusepath{fill,stroke}
\pgfpathmoveto{\pgfpoint{142.90290546875pt}{140.8627220486111pt}}
\pgflineto{\pgfpoint{142.55475755208332pt}{141.93421180555555pt}}
\pgflineto{\pgfpoint{141.64329921874997pt}{142.5964251736111pt}}
\pgfpathclose
\pgfusepath{fill,stroke}
\pgfpathmoveto{\pgfpoint{138.9206717013889pt}{149.02229696180555pt}}
\pgflineto{\pgfpoint{137.66106484374998pt}{147.28860295138887pt}}
\pgflineto{\pgfpoint{138.57252317708333pt}{147.9508163194444pt}}
\pgfpathclose
\pgfusepath{fill,stroke}
\pgfpathmoveto{\pgfpoint{138.9206717013889pt}{149.02229696180555pt}}
\pgflineto{\pgfpoint{136.53444461805555pt}{147.28860295138887pt}}
\pgflineto{\pgfpoint{137.66106484374998pt}{147.28860295138887pt}}
\pgfpathclose
\pgfusepath{fill,stroke}
\pgfpathmoveto{\pgfpoint{138.9206717013889pt}{149.02229696180555pt}}
\pgflineto{\pgfpoint{135.6229862847222pt}{147.9508163194444pt}}
\pgflineto{\pgfpoint{136.53444461805555pt}{147.28860295138887pt}}
\pgfpathclose
\pgfusepath{fill,stroke}
\pgfpathmoveto{\pgfpoint{138.9206717013889pt}{149.02229696180555pt}}
\pgflineto{\pgfpoint{135.27483836805555pt}{149.02229696180555pt}}
\pgflineto{\pgfpoint{135.6229862847222pt}{147.9508163194444pt}}
\pgfpathclose
\pgfusepath{fill,stroke}
\pgfpathmoveto{\pgfpoint{138.9206717013889pt}{149.02229696180555pt}}
\pgflineto{\pgfpoint{135.6229862847222pt}{150.09377821180553pt}}
\pgflineto{\pgfpoint{135.27483836805555pt}{149.02229696180555pt}}
\pgfpathclose
\pgfusepath{fill,stroke}
\pgfpathmoveto{\pgfpoint{138.9206717013889pt}{149.02229696180555pt}}
\pgflineto{\pgfpoint{136.53444461805555pt}{150.7559909722222pt}}
\pgflineto{\pgfpoint{135.6229862847222pt}{150.09377821180553pt}}
\pgfpathclose
\pgfusepath{fill,stroke}
\pgfpathmoveto{\pgfpoint{138.9206717013889pt}{149.02229696180555pt}}
\pgflineto{\pgfpoint{137.66106484374998pt}{150.7559909722222pt}}
\pgflineto{\pgfpoint{136.53444461805555pt}{150.7559909722222pt}}
\pgfpathclose
\pgfusepath{fill,stroke}
\pgfpathmoveto{\pgfpoint{138.9206717013889pt}{149.02229696180555pt}}
\pgflineto{\pgfpoint{138.57252317708333pt}{150.09377821180553pt}}
\pgflineto{\pgfpoint{137.66106484374998pt}{150.7559909722222pt}}
\pgfpathclose
\pgfusepath{fill,stroke}
\pgfpathmoveto{\pgfpoint{132.2659521701389pt}{151.44407213541666pt}}
\pgflineto{\pgfpoint{131.00635503472222pt}{149.71037812499998pt}}
\pgflineto{\pgfpoint{131.91781336805553pt}{150.37259149305555pt}}
\pgfpathclose
\pgfusepath{fill,stroke}
\pgfpathmoveto{\pgfpoint{132.2659521701389pt}{151.44407213541666pt}}
\pgflineto{\pgfpoint{129.87973420138889pt}{149.71037812499998pt}}
\pgflineto{\pgfpoint{131.00635503472222pt}{149.71037812499998pt}}
\pgfpathclose
\pgfusepath{fill,stroke}
\pgfpathmoveto{\pgfpoint{132.2659521701389pt}{151.44407213541666pt}}
\pgflineto{\pgfpoint{128.96827586805554pt}{150.37259149305555pt}}
\pgflineto{\pgfpoint{129.87973420138889pt}{149.71037812499998pt}}
\pgfpathclose
\pgfusepath{fill,stroke}
\pgfpathmoveto{\pgfpoint{132.2659521701389pt}{151.44407213541666pt}}
\pgflineto{\pgfpoint{128.6201279513889pt}{151.44407213541666pt}}
\pgflineto{\pgfpoint{128.96827586805554pt}{150.37259149305555pt}}
\pgfpathclose
\pgfusepath{fill,stroke}
\pgfpathmoveto{\pgfpoint{132.2659521701389pt}{151.44407213541666pt}}
\pgflineto{\pgfpoint{128.96827586805554pt}{152.5155625pt}}
\pgflineto{\pgfpoint{128.6201279513889pt}{151.44407213541666pt}}
\pgfpathclose
\pgfusepath{fill,stroke}
\pgfpathmoveto{\pgfpoint{132.2659521701389pt}{151.44407213541666pt}}
\pgflineto{\pgfpoint{129.87973420138889pt}{153.17777526041664pt}}
\pgflineto{\pgfpoint{128.96827586805554pt}{152.5155625pt}}
\pgfpathclose
\pgfusepath{fill,stroke}
\pgfpathmoveto{\pgfpoint{132.2659521701389pt}{151.44407213541666pt}}
\pgflineto{\pgfpoint{131.00635503472222pt}{153.17777526041664pt}}
\pgflineto{\pgfpoint{129.87973420138889pt}{153.17777526041664pt}}
\pgfpathclose
\pgfusepath{fill,stroke}
\pgfpathmoveto{\pgfpoint{132.2659521701389pt}{151.44407213541666pt}}
\pgflineto{\pgfpoint{131.91781336805553pt}{152.5155625pt}}
\pgflineto{\pgfpoint{131.00635503472222pt}{153.17777526041664pt}}
\pgfpathclose
\pgfusepath{fill,stroke}
\pgfpathmoveto{\pgfpoint{176.09168402777777pt}{107.72808107638888pt}}
\pgflineto{\pgfpoint{174.83208689236108pt}{105.9943876736111pt}}
\pgflineto{\pgfpoint{175.74354522569442pt}{106.65660043402777pt}}
\pgfpathclose
\pgfusepath{fill,stroke}
\pgfpathmoveto{\pgfpoint{176.09168402777777pt}{107.72808107638888pt}}
\pgflineto{\pgfpoint{173.70544782986107pt}{105.9943876736111pt}}
\pgflineto{\pgfpoint{174.83208689236108pt}{105.9943876736111pt}}
\pgfpathclose
\pgfusepath{fill,stroke}
\pgfpathmoveto{\pgfpoint{176.09168402777777pt}{107.72808107638888pt}}
\pgflineto{\pgfpoint{172.79398949652776pt}{106.65660043402777pt}}
\pgflineto{\pgfpoint{173.70544782986107pt}{105.9943876736111pt}}
\pgfpathclose
\pgfusepath{fill,stroke}
\pgfpathmoveto{\pgfpoint{176.09168402777777pt}{107.72808107638888pt}}
\pgflineto{\pgfpoint{172.44585069444443pt}{107.72808107638888pt}}
\pgflineto{\pgfpoint{172.79398949652776pt}{106.65660043402777pt}}
\pgfpathclose
\pgfusepath{fill,stroke}
\pgfpathmoveto{\pgfpoint{176.09168402777777pt}{107.72808107638888pt}}
\pgflineto{\pgfpoint{172.79398949652776pt}{108.79957144097222pt}}
\pgflineto{\pgfpoint{172.44585069444443pt}{107.72808107638888pt}}
\pgfpathclose
\pgfusepath{fill,stroke}
\pgfpathmoveto{\pgfpoint{176.09168402777777pt}{107.72808107638888pt}}
\pgflineto{\pgfpoint{173.70544782986107pt}{109.46178420138888pt}}
\pgflineto{\pgfpoint{172.79398949652776pt}{108.79957144097222pt}}
\pgfpathclose
\pgfusepath{fill,stroke}
\pgfpathmoveto{\pgfpoint{176.09168402777777pt}{107.72808107638888pt}}
\pgflineto{\pgfpoint{174.83208689236108pt}{109.46178420138888pt}}
\pgflineto{\pgfpoint{173.70544782986107pt}{109.46178420138888pt}}
\pgfpathclose
\pgfusepath{fill,stroke}
\pgfpathmoveto{\pgfpoint{176.09168402777777pt}{107.72808107638888pt}}
\pgflineto{\pgfpoint{175.74354522569442pt}{108.79957144097222pt}}
\pgflineto{\pgfpoint{174.83208689236108pt}{109.46178420138888pt}}
\pgfpathclose
\pgfusepath{fill,stroke}
\pgfpathmoveto{\pgfpoint{146.6650815972222pt}{149.70261796875pt}}
\pgflineto{\pgfpoint{145.40547534722222pt}{147.9689239583333pt}}
\pgflineto{\pgfpoint{146.31693368055556pt}{148.63113671874999pt}}
\pgfpathclose
\pgfusepath{fill,stroke}
\pgfpathmoveto{\pgfpoint{146.6650815972222pt}{149.70261796875pt}}
\pgflineto{\pgfpoint{144.27885451388886pt}{147.9689239583333pt}}
\pgflineto{\pgfpoint{145.40547534722222pt}{147.9689239583333pt}}
\pgfpathclose
\pgfusepath{fill,stroke}
\pgfpathmoveto{\pgfpoint{146.6650815972222pt}{149.70261796875pt}}
\pgflineto{\pgfpoint{143.36739618055555pt}{148.63113671874999pt}}
\pgflineto{\pgfpoint{144.27885451388886pt}{147.9689239583333pt}}
\pgfpathclose
\pgfusepath{fill,stroke}
\pgfpathmoveto{\pgfpoint{146.6650815972222pt}{149.70261796875pt}}
\pgflineto{\pgfpoint{143.01924826388887pt}{149.70261796875pt}}
\pgflineto{\pgfpoint{143.36739618055555pt}{148.63113671874999pt}}
\pgfpathclose
\pgfusepath{fill,stroke}
\pgfpathmoveto{\pgfpoint{146.6650815972222pt}{149.70261796875pt}}
\pgflineto{\pgfpoint{143.36739618055555pt}{150.77410833333332pt}}
\pgflineto{\pgfpoint{143.01924826388887pt}{149.70261796875pt}}
\pgfpathclose
\pgfusepath{fill,stroke}
\pgfpathmoveto{\pgfpoint{146.6650815972222pt}{149.70261796875pt}}
\pgflineto{\pgfpoint{144.27885451388886pt}{151.43632109374997pt}}
\pgflineto{\pgfpoint{143.36739618055555pt}{150.77410833333332pt}}
\pgfpathclose
\pgfusepath{fill,stroke}
\pgfpathmoveto{\pgfpoint{146.6650815972222pt}{149.70261796875pt}}
\pgflineto{\pgfpoint{145.40547534722222pt}{151.43632109374997pt}}
\pgflineto{\pgfpoint{144.27885451388886pt}{151.43632109374997pt}}
\pgfpathclose
\pgfusepath{fill,stroke}
\pgfpathmoveto{\pgfpoint{146.6650815972222pt}{149.70261796875pt}}
\pgflineto{\pgfpoint{146.31693368055556pt}{150.77410833333332pt}}
\pgflineto{\pgfpoint{145.40547534722222pt}{151.43632109374997pt}}
\pgfpathclose
\pgfusepath{fill,stroke}
\pgfpathmoveto{\pgfpoint{139.5967550347222pt}{147.33647395833333pt}}
\pgflineto{\pgfpoint{138.33714878472222pt}{145.60277994791664pt}}
\pgflineto{\pgfpoint{139.24860711805553pt}{146.26499270833332pt}}
\pgfpathclose
\pgfusepath{fill,stroke}
\pgfpathmoveto{\pgfpoint{139.5967550347222pt}{147.33647395833333pt}}
\pgflineto{\pgfpoint{137.21052795138888pt}{145.60277994791664pt}}
\pgflineto{\pgfpoint{138.33714878472222pt}{145.60277994791664pt}}
\pgfpathclose
\pgfusepath{fill,stroke}
\pgfpathmoveto{\pgfpoint{139.5967550347222pt}{147.33647395833333pt}}
\pgflineto{\pgfpoint{136.29906961805554pt}{146.26499270833332pt}}
\pgflineto{\pgfpoint{137.21052795138888pt}{145.60277994791664pt}}
\pgfpathclose
\pgfusepath{fill,stroke}
\pgfpathmoveto{\pgfpoint{139.5967550347222pt}{147.33647395833333pt}}
\pgflineto{\pgfpoint{135.9509217013889pt}{147.33647395833333pt}}
\pgflineto{\pgfpoint{136.29906961805554pt}{146.26499270833332pt}}
\pgfpathclose
\pgfusepath{fill,stroke}
\pgfpathmoveto{\pgfpoint{139.5967550347222pt}{147.33647395833333pt}}
\pgflineto{\pgfpoint{136.29906961805554pt}{148.40795460069444pt}}
\pgflineto{\pgfpoint{135.9509217013889pt}{147.33647395833333pt}}
\pgfpathclose
\pgfusepath{fill,stroke}
\pgfpathmoveto{\pgfpoint{139.5967550347222pt}{147.33647395833333pt}}
\pgflineto{\pgfpoint{137.21052795138888pt}{149.07016736111112pt}}
\pgflineto{\pgfpoint{136.29906961805554pt}{148.40795460069444pt}}
\pgfpathclose
\pgfusepath{fill,stroke}
\pgfpathmoveto{\pgfpoint{139.5967550347222pt}{147.33647395833333pt}}
\pgflineto{\pgfpoint{138.33714878472222pt}{149.07016736111112pt}}
\pgflineto{\pgfpoint{137.21052795138888pt}{149.07016736111112pt}}
\pgfpathclose
\pgfusepath{fill,stroke}
\pgfpathmoveto{\pgfpoint{139.5967550347222pt}{147.33647395833333pt}}
\pgflineto{\pgfpoint{139.24860711805553pt}{148.40795460069444pt}}
\pgflineto{\pgfpoint{138.33714878472222pt}{149.07016736111112pt}}
\pgfpathclose
\pgfusepath{fill,stroke}
\pgfpathmoveto{\pgfpoint{132.98709678819444pt}{150.29736883680556pt}}
\pgflineto{\pgfpoint{131.72749965277777pt}{148.56367482638888pt}}
\pgflineto{\pgfpoint{132.6389579861111pt}{149.22588819444445pt}}
\pgfpathclose
\pgfusepath{fill,stroke}
\pgfpathmoveto{\pgfpoint{132.98709678819444pt}{150.29736883680556pt}}
\pgflineto{\pgfpoint{130.60087881944443pt}{148.56367482638888pt}}
\pgflineto{\pgfpoint{131.72749965277777pt}{148.56367482638888pt}}
\pgfpathclose
\pgfusepath{fill,stroke}
\pgfpathmoveto{\pgfpoint{132.98709678819444pt}{150.29736883680556pt}}
\pgflineto{\pgfpoint{129.68942048611112pt}{149.22588819444445pt}}
\pgflineto{\pgfpoint{130.60087881944443pt}{148.56367482638888pt}}
\pgfpathclose
\pgfusepath{fill,stroke}
\pgfpathmoveto{\pgfpoint{132.98709678819444pt}{150.29736883680556pt}}
\pgflineto{\pgfpoint{129.34127256944444pt}{150.29736883680556pt}}
\pgflineto{\pgfpoint{129.68942048611112pt}{149.22588819444445pt}}
\pgfpathclose
\pgfusepath{fill,stroke}
\pgfpathmoveto{\pgfpoint{132.98709678819444pt}{150.29736883680556pt}}
\pgflineto{\pgfpoint{129.68942048611112pt}{151.3688592013889pt}}
\pgflineto{\pgfpoint{129.34127256944444pt}{150.29736883680556pt}}
\pgfpathclose
\pgfusepath{fill,stroke}
\pgfpathmoveto{\pgfpoint{132.98709678819444pt}{150.29736883680556pt}}
\pgflineto{\pgfpoint{130.60087881944443pt}{152.03107196180554pt}}
\pgflineto{\pgfpoint{129.68942048611112pt}{151.3688592013889pt}}
\pgfpathclose
\pgfusepath{fill,stroke}
\pgfpathmoveto{\pgfpoint{132.98709678819444pt}{150.29736883680556pt}}
\pgflineto{\pgfpoint{131.72749965277777pt}{152.03107196180554pt}}
\pgflineto{\pgfpoint{130.60087881944443pt}{152.03107196180554pt}}
\pgfpathclose
\pgfusepath{fill,stroke}
\pgfpathmoveto{\pgfpoint{132.98709678819444pt}{150.29736883680556pt}}
\pgflineto{\pgfpoint{132.6389579861111pt}{151.3688592013889pt}}
\pgflineto{\pgfpoint{131.72749965277777pt}{152.03107196180554pt}}
\pgfpathclose
\pgfusepath{fill,stroke}
\pgfpathmoveto{\pgfpoint{132.74053090277778pt}{150.2160734375pt}}
\pgflineto{\pgfpoint{131.48092404513886pt}{148.48237942708332pt}}
\pgflineto{\pgfpoint{132.3923823784722pt}{149.1445927951389pt}}
\pgfpathclose
\pgfusepath{fill,stroke}
\pgfpathmoveto{\pgfpoint{132.74053090277778pt}{150.2160734375pt}}
\pgflineto{\pgfpoint{130.35430381944443pt}{148.48237942708332pt}}
\pgflineto{\pgfpoint{131.48092404513886pt}{148.48237942708332pt}}
\pgfpathclose
\pgfusepath{fill,stroke}
\pgfpathmoveto{\pgfpoint{132.74053090277778pt}{150.2160734375pt}}
\pgflineto{\pgfpoint{129.44284548611108pt}{149.1445927951389pt}}
\pgflineto{\pgfpoint{130.35430381944443pt}{148.48237942708332pt}}
\pgfpathclose
\pgfusepath{fill,stroke}
\pgfpathmoveto{\pgfpoint{132.74053090277778pt}{150.2160734375pt}}
\pgflineto{\pgfpoint{129.09469756944443pt}{150.2160734375pt}}
\pgflineto{\pgfpoint{129.44284548611108pt}{149.1445927951389pt}}
\pgfpathclose
\pgfusepath{fill,stroke}
\pgfpathmoveto{\pgfpoint{132.74053090277778pt}{150.2160734375pt}}
\pgflineto{\pgfpoint{129.44284548611108pt}{151.2875540798611pt}}
\pgflineto{\pgfpoint{129.09469756944443pt}{150.2160734375pt}}
\pgfpathclose
\pgfusepath{fill,stroke}
\pgfpathmoveto{\pgfpoint{132.74053090277778pt}{150.2160734375pt}}
\pgflineto{\pgfpoint{130.35430381944443pt}{151.94976744791666pt}}
\pgflineto{\pgfpoint{129.44284548611108pt}{151.2875540798611pt}}
\pgfpathclose
\pgfusepath{fill,stroke}
\pgfpathmoveto{\pgfpoint{132.74053090277778pt}{150.2160734375pt}}
\pgflineto{\pgfpoint{131.48092404513886pt}{151.94976744791666pt}}
\pgflineto{\pgfpoint{130.35430381944443pt}{151.94976744791666pt}}
\pgfpathclose
\pgfusepath{fill,stroke}
\pgfpathmoveto{\pgfpoint{132.74053090277778pt}{150.2160734375pt}}
\pgflineto{\pgfpoint{132.3923823784722pt}{151.2875540798611pt}}
\pgflineto{\pgfpoint{131.48092404513886pt}{151.94976744791666pt}}
\pgfpathclose
\pgfusepath{fill,stroke}
\pgfpathmoveto{\pgfpoint{151.9968126736111pt}{143.18180199652775pt}}
\pgflineto{\pgfpoint{150.7372064236111pt}{141.44809887152775pt}}
\pgflineto{\pgfpoint{151.6486647569444pt}{142.11031223958332pt}}
\pgfpathclose
\pgfusepath{fill,stroke}
\pgfpathmoveto{\pgfpoint{151.9968126736111pt}{143.18180199652775pt}}
\pgflineto{\pgfpoint{149.61058559027776pt}{141.44809887152775pt}}
\pgflineto{\pgfpoint{150.7372064236111pt}{141.44809887152775pt}}
\pgfpathclose
\pgfusepath{fill,stroke}
\pgfpathmoveto{\pgfpoint{151.9968126736111pt}{143.18180199652775pt}}
\pgflineto{\pgfpoint{148.69912725694442pt}{142.11031223958332pt}}
\pgflineto{\pgfpoint{149.61058559027776pt}{141.44809887152775pt}}
\pgfpathclose
\pgfusepath{fill,stroke}
\pgfpathmoveto{\pgfpoint{151.9968126736111pt}{143.18180199652775pt}}
\pgflineto{\pgfpoint{148.35097934027777pt}{143.18180199652775pt}}
\pgflineto{\pgfpoint{148.69912725694442pt}{142.11031223958332pt}}
\pgfpathclose
\pgfusepath{fill,stroke}
\pgfpathmoveto{\pgfpoint{151.9968126736111pt}{143.18180199652775pt}}
\pgflineto{\pgfpoint{148.69912725694442pt}{144.25328324652776pt}}
\pgflineto{\pgfpoint{148.35097934027777pt}{143.18180199652775pt}}
\pgfpathclose
\pgfusepath{fill,stroke}
\pgfpathmoveto{\pgfpoint{151.9968126736111pt}{143.18180199652775pt}}
\pgflineto{\pgfpoint{149.61058559027776pt}{144.91549600694444pt}}
\pgflineto{\pgfpoint{148.69912725694442pt}{144.25328324652776pt}}
\pgfpathclose
\pgfusepath{fill,stroke}
\pgfpathmoveto{\pgfpoint{151.9968126736111pt}{143.18180199652775pt}}
\pgflineto{\pgfpoint{150.7372064236111pt}{144.91549600694444pt}}
\pgflineto{\pgfpoint{149.61058559027776pt}{144.91549600694444pt}}
\pgfpathclose
\pgfusepath{fill,stroke}
\pgfpathmoveto{\pgfpoint{151.9968126736111pt}{143.18180199652775pt}}
\pgflineto{\pgfpoint{151.6486647569444pt}{144.25328324652776pt}}
\pgflineto{\pgfpoint{150.7372064236111pt}{144.91549600694444pt}}
\pgfpathclose
\pgfusepath{fill,stroke}
\pgfpathmoveto{\pgfpoint{153.22700859374999pt}{136.84070251736108pt}}
\pgflineto{\pgfpoint{151.96741145833332pt}{135.1069993923611pt}}
\pgflineto{\pgfpoint{152.87886979166666pt}{135.76921215277775pt}}
\pgfpathclose
\pgfusepath{fill,stroke}
\pgfpathmoveto{\pgfpoint{153.22700859374999pt}{136.84070251736108pt}}
\pgflineto{\pgfpoint{150.84078151041666pt}{135.1069993923611pt}}
\pgflineto{\pgfpoint{151.96741145833332pt}{135.1069993923611pt}}
\pgfpathclose
\pgfusepath{fill,stroke}
\pgfpathmoveto{\pgfpoint{153.22700859374999pt}{136.84070251736108pt}}
\pgflineto{\pgfpoint{149.92932317708332pt}{135.76921215277775pt}}
\pgflineto{\pgfpoint{150.84078151041666pt}{135.1069993923611pt}}
\pgfpathclose
\pgfusepath{fill,stroke}
\pgfpathmoveto{\pgfpoint{153.22700859374999pt}{136.84070251736108pt}}
\pgflineto{\pgfpoint{149.58117526041664pt}{136.84070251736108pt}}
\pgflineto{\pgfpoint{149.92932317708332pt}{135.76921215277775pt}}
\pgfpathclose
\pgfusepath{fill,stroke}
\pgfpathmoveto{\pgfpoint{153.22700859374999pt}{136.84070251736108pt}}
\pgflineto{\pgfpoint{149.92932317708332pt}{137.9121831597222pt}}
\pgflineto{\pgfpoint{149.58117526041664pt}{136.84070251736108pt}}
\pgfpathclose
\pgfusepath{fill,stroke}
\pgfpathmoveto{\pgfpoint{153.22700859374999pt}{136.84070251736108pt}}
\pgflineto{\pgfpoint{150.84078151041666pt}{138.57439652777776pt}}
\pgflineto{\pgfpoint{149.92932317708332pt}{137.9121831597222pt}}
\pgfpathclose
\pgfusepath{fill,stroke}
\pgfpathmoveto{\pgfpoint{153.22700859374999pt}{136.84070251736108pt}}
\pgflineto{\pgfpoint{151.96741145833332pt}{138.57439652777776pt}}
\pgflineto{\pgfpoint{150.84078151041666pt}{138.57439652777776pt}}
\pgfpathclose
\pgfusepath{fill,stroke}
\pgfpathmoveto{\pgfpoint{153.22700859374999pt}{136.84070251736108pt}}
\pgflineto{\pgfpoint{152.87886979166666pt}{137.9121831597222pt}}
\pgflineto{\pgfpoint{151.96741145833332pt}{138.57439652777776pt}}
\pgfpathclose
\pgfusepath{fill,stroke}
\pgfpathmoveto{\pgfpoint{111.53819322916667pt}{152.60789140625pt}}
\pgflineto{\pgfpoint{110.27859609375pt}{150.87418828124999pt}}
\pgflineto{\pgfpoint{111.19005442708333pt}{151.53640104166664pt}}
\pgfpathclose
\pgfusepath{fill,stroke}
\pgfpathmoveto{\pgfpoint{111.53819322916667pt}{152.60789140625pt}}
\pgflineto{\pgfpoint{109.15197526041666pt}{150.87418828124999pt}}
\pgflineto{\pgfpoint{110.27859609375pt}{150.87418828124999pt}}
\pgfpathclose
\pgfusepath{fill,stroke}
\pgfpathmoveto{\pgfpoint{111.53819322916667pt}{152.60789140625pt}}
\pgflineto{\pgfpoint{108.24051692708333pt}{151.53640104166664pt}}
\pgflineto{\pgfpoint{109.15197526041666pt}{150.87418828124999pt}}
\pgfpathclose
\pgfusepath{fill,stroke}
\pgfpathmoveto{\pgfpoint{111.53819322916667pt}{152.60789140625pt}}
\pgflineto{\pgfpoint{107.89236901041666pt}{152.60789140625pt}}
\pgflineto{\pgfpoint{108.24051692708333pt}{151.53640104166664pt}}
\pgfpathclose
\pgfusepath{fill,stroke}
\pgfpathmoveto{\pgfpoint{111.53819322916667pt}{152.60789140625pt}}
\pgflineto{\pgfpoint{108.24051692708333pt}{153.6793720486111pt}}
\pgflineto{\pgfpoint{107.89236901041666pt}{152.60789140625pt}}
\pgfpathclose
\pgfusepath{fill,stroke}
\pgfpathmoveto{\pgfpoint{111.53819322916667pt}{152.60789140625pt}}
\pgflineto{\pgfpoint{109.15197526041666pt}{154.34158480902775pt}}
\pgflineto{\pgfpoint{108.24051692708333pt}{153.6793720486111pt}}
\pgfpathclose
\pgfusepath{fill,stroke}
\pgfpathmoveto{\pgfpoint{111.53819322916667pt}{152.60789140625pt}}
\pgflineto{\pgfpoint{110.27859609375pt}{154.34158480902775pt}}
\pgflineto{\pgfpoint{109.15197526041666pt}{154.34158480902775pt}}
\pgfpathclose
\pgfusepath{fill,stroke}
\pgfpathmoveto{\pgfpoint{111.53819322916667pt}{152.60789140625pt}}
\pgflineto{\pgfpoint{111.19005442708333pt}{153.6793720486111pt}}
\pgflineto{\pgfpoint{110.27859609375pt}{154.34158480902775pt}}
\pgfpathclose
\pgfusepath{fill,stroke}
\pgfpathmoveto{\pgfpoint{164.66465008680555pt}{110.6419125pt}}
\pgflineto{\pgfpoint{163.40503411458332pt}{108.9082190972222pt}}
\pgflineto{\pgfpoint{164.31649244791666pt}{109.57043185763888pt}}
\pgfpathclose
\pgfusepath{fill,stroke}
\pgfpathmoveto{\pgfpoint{164.66465008680555pt}{110.6419125pt}}
\pgflineto{\pgfpoint{162.27841388888885pt}{108.9082190972222pt}}
\pgflineto{\pgfpoint{163.40503411458332pt}{108.9082190972222pt}}
\pgfpathclose
\pgfusepath{fill,stroke}
\pgfpathmoveto{\pgfpoint{164.66465008680555pt}{110.6419125pt}}
\pgflineto{\pgfpoint{161.36695555555553pt}{109.57043185763888pt}}
\pgflineto{\pgfpoint{162.27841388888885pt}{108.9082190972222pt}}
\pgfpathclose
\pgfusepath{fill,stroke}
\pgfpathmoveto{\pgfpoint{164.66465008680555pt}{110.6419125pt}}
\pgflineto{\pgfpoint{161.0188167534722pt}{110.6419125pt}}
\pgflineto{\pgfpoint{161.36695555555553pt}{109.57043185763888pt}}
\pgfpathclose
\pgfusepath{fill,stroke}
\pgfpathmoveto{\pgfpoint{164.66465008680555pt}{110.6419125pt}}
\pgflineto{\pgfpoint{161.36695555555553pt}{111.71339375pt}}
\pgflineto{\pgfpoint{161.0188167534722pt}{110.6419125pt}}
\pgfpathclose
\pgfusepath{fill,stroke}
\pgfpathmoveto{\pgfpoint{164.66465008680555pt}{110.6419125pt}}
\pgflineto{\pgfpoint{162.27841388888885pt}{112.37560651041666pt}}
\pgflineto{\pgfpoint{161.36695555555553pt}{111.71339375pt}}
\pgfpathclose
\pgfusepath{fill,stroke}
\pgfpathmoveto{\pgfpoint{164.66465008680555pt}{110.6419125pt}}
\pgflineto{\pgfpoint{163.40503411458332pt}{112.37560651041666pt}}
\pgflineto{\pgfpoint{162.27841388888885pt}{112.37560651041666pt}}
\pgfpathclose
\pgfusepath{fill,stroke}
\pgfpathmoveto{\pgfpoint{164.66465008680555pt}{110.6419125pt}}
\pgflineto{\pgfpoint{164.31649244791666pt}{111.71339375pt}}
\pgflineto{\pgfpoint{163.40503411458332pt}{112.37560651041666pt}}
\pgfpathclose
\pgfusepath{fill,stroke}
\pgfpathmoveto{\pgfpoint{262.7436638020833pt}{57.054844704861104pt}}
\pgflineto{\pgfpoint{261.4840478298611pt}{55.32115130208333pt}}
\pgflineto{\pgfpoint{262.39550616319445pt}{55.983359201388886pt}}
\pgfpathclose
\pgfusepath{fill,stroke}
\pgfpathmoveto{\pgfpoint{262.7436638020833pt}{57.054844704861104pt}}
\pgflineto{\pgfpoint{260.3574276041666pt}{55.32115130208333pt}}
\pgflineto{\pgfpoint{261.4840478298611pt}{55.32115130208333pt}}
\pgfpathclose
\pgfusepath{fill,stroke}
\pgfpathmoveto{\pgfpoint{262.7436638020833pt}{57.054844704861104pt}}
\pgflineto{\pgfpoint{259.4459692708333pt}{55.983359201388886pt}}
\pgflineto{\pgfpoint{260.3574276041666pt}{55.32115130208333pt}}
\pgfpathclose
\pgfusepath{fill,stroke}
\pgfpathmoveto{\pgfpoint{262.7436638020833pt}{57.054844704861104pt}}
\pgflineto{\pgfpoint{259.09783046875pt}{57.054844704861104pt}}
\pgflineto{\pgfpoint{259.4459692708333pt}{55.983359201388886pt}}
\pgfpathclose
\pgfusepath{fill,stroke}
\pgfpathmoveto{\pgfpoint{262.7436638020833pt}{57.054844704861104pt}}
\pgflineto{\pgfpoint{259.4459692708333pt}{58.126330208333336pt}}
\pgflineto{\pgfpoint{259.09783046875pt}{57.054844704861104pt}}
\pgfpathclose
\pgfusepath{fill,stroke}
\pgfpathmoveto{\pgfpoint{262.7436638020833pt}{57.054844704861104pt}}
\pgflineto{\pgfpoint{260.3574276041666pt}{58.78854357638888pt}}
\pgflineto{\pgfpoint{259.4459692708333pt}{58.126330208333336pt}}
\pgfpathclose
\pgfusepath{fill,stroke}
\pgfpathmoveto{\pgfpoint{262.7436638020833pt}{57.054844704861104pt}}
\pgflineto{\pgfpoint{261.4840478298611pt}{58.78854357638888pt}}
\pgflineto{\pgfpoint{260.3574276041666pt}{58.78854357638888pt}}
\pgfpathclose
\pgfusepath{fill,stroke}
\pgfpathmoveto{\pgfpoint{262.7436638020833pt}{57.054844704861104pt}}
\pgflineto{\pgfpoint{262.39550616319445pt}{58.126330208333336pt}}
\pgflineto{\pgfpoint{261.4840478298611pt}{58.78854357638888pt}}
\pgfpathclose
\pgfusepath{fill,stroke}
\pgfpathmoveto{\pgfpoint{239.02525633680554pt}{87.04888828124999pt}}
\pgflineto{\pgfpoint{237.7656409722222pt}{85.31518515624998pt}}
\pgflineto{\pgfpoint{238.67709930555554pt}{85.97739791666666pt}}
\pgfpathclose
\pgfusepath{fill,stroke}
\pgfpathmoveto{\pgfpoint{239.02525633680554pt}{87.04888828124999pt}}
\pgflineto{\pgfpoint{236.6390201388889pt}{85.31518515624998pt}}
\pgflineto{\pgfpoint{237.7656409722222pt}{85.31518515624998pt}}
\pgfpathclose
\pgfusepath{fill,stroke}
\pgfpathmoveto{\pgfpoint{239.02525633680554pt}{87.04888828124999pt}}
\pgflineto{\pgfpoint{235.72756180555555pt}{85.97739791666666pt}}
\pgflineto{\pgfpoint{236.6390201388889pt}{85.31518515624998pt}}
\pgfpathclose
\pgfusepath{fill,stroke}
\pgfpathmoveto{\pgfpoint{239.02525633680554pt}{87.04888828124999pt}}
\pgflineto{\pgfpoint{235.3794230034722pt}{87.04888828124999pt}}
\pgflineto{\pgfpoint{235.72756180555555pt}{85.97739791666666pt}}
\pgfpathclose
\pgfusepath{fill,stroke}
\pgfpathmoveto{\pgfpoint{239.02525633680554pt}{87.04888828124999pt}}
\pgflineto{\pgfpoint{235.72756180555555pt}{88.12036892361111pt}}
\pgflineto{\pgfpoint{235.3794230034722pt}{87.04888828124999pt}}
\pgfpathclose
\pgfusepath{fill,stroke}
\pgfpathmoveto{\pgfpoint{239.02525633680554pt}{87.04888828124999pt}}
\pgflineto{\pgfpoint{236.6390201388889pt}{88.78257256944444pt}}
\pgflineto{\pgfpoint{235.72756180555555pt}{88.12036892361111pt}}
\pgfpathclose
\pgfusepath{fill,stroke}
\pgfpathmoveto{\pgfpoint{239.02525633680554pt}{87.04888828124999pt}}
\pgflineto{\pgfpoint{237.7656409722222pt}{88.78257256944444pt}}
\pgflineto{\pgfpoint{236.6390201388889pt}{88.78257256944444pt}}
\pgfpathclose
\pgfusepath{fill,stroke}
\pgfpathmoveto{\pgfpoint{239.02525633680554pt}{87.04888828124999pt}}
\pgflineto{\pgfpoint{238.67709930555554pt}{88.12036892361111pt}}
\pgflineto{\pgfpoint{237.7656409722222pt}{88.78257256944444pt}}
\pgfpathclose
\pgfusepath{fill,stroke}
\pgfpathmoveto{\pgfpoint{239.98236901041665pt}{91.70416475694444pt}}
\pgflineto{\pgfpoint{238.7227536458333pt}{89.97047074652777pt}}
\pgflineto{\pgfpoint{239.63421197916665pt}{90.6326743923611pt}}
\pgfpathclose
\pgfusepath{fill,stroke}
\pgfpathmoveto{\pgfpoint{239.98236901041665pt}{91.70416475694444pt}}
\pgflineto{\pgfpoint{237.5961328125pt}{89.97047074652777pt}}
\pgflineto{\pgfpoint{238.7227536458333pt}{89.97047074652777pt}}
\pgfpathclose
\pgfusepath{fill,stroke}
\pgfpathmoveto{\pgfpoint{239.98236901041665pt}{91.70416475694444pt}}
\pgflineto{\pgfpoint{236.68467447916666pt}{90.6326743923611pt}}
\pgflineto{\pgfpoint{237.5961328125pt}{89.97047074652777pt}}
\pgfpathclose
\pgfusepath{fill,stroke}
\pgfpathmoveto{\pgfpoint{239.98236901041665pt}{91.70416475694444pt}}
\pgflineto{\pgfpoint{236.3365356770833pt}{91.70416475694444pt}}
\pgflineto{\pgfpoint{236.68467447916666pt}{90.6326743923611pt}}
\pgfpathclose
\pgfusepath{fill,stroke}
\pgfpathmoveto{\pgfpoint{239.98236901041665pt}{91.70416475694444pt}}
\pgflineto{\pgfpoint{236.68467447916666pt}{92.77565451388888pt}}
\pgflineto{\pgfpoint{236.3365356770833pt}{91.70416475694444pt}}
\pgfpathclose
\pgfusepath{fill,stroke}
\pgfpathmoveto{\pgfpoint{239.98236901041665pt}{91.70416475694444pt}}
\pgflineto{\pgfpoint{237.5961328125pt}{93.43785815972221pt}}
\pgflineto{\pgfpoint{236.68467447916666pt}{92.77565451388888pt}}
\pgfpathclose
\pgfusepath{fill,stroke}
\pgfpathmoveto{\pgfpoint{239.98236901041665pt}{91.70416475694444pt}}
\pgflineto{\pgfpoint{238.7227536458333pt}{93.43785815972221pt}}
\pgflineto{\pgfpoint{237.5961328125pt}{93.43785815972221pt}}
\pgfpathclose
\pgfusepath{fill,stroke}
\pgfpathmoveto{\pgfpoint{239.98236901041665pt}{91.70416475694444pt}}
\pgflineto{\pgfpoint{239.63421197916665pt}{92.77565451388888pt}}
\pgflineto{\pgfpoint{238.7227536458333pt}{93.43785815972221pt}}
\pgfpathclose
\pgfusepath{fill,stroke}
\pgfpathmoveto{\pgfpoint{143.45703628472222pt}{148.79980086805554pt}}
\pgflineto{\pgfpoint{142.1974300347222pt}{147.06610685763889pt}}
\pgflineto{\pgfpoint{143.1088792534722pt}{147.72832022569443pt}}
\pgfpathclose
\pgfusepath{fill,stroke}
\pgfpathmoveto{\pgfpoint{143.45703628472222pt}{148.79980086805554pt}}
\pgflineto{\pgfpoint{141.07080008680555pt}{147.06610685763889pt}}
\pgflineto{\pgfpoint{142.1974300347222pt}{147.06610685763889pt}}
\pgfpathclose
\pgfusepath{fill,stroke}
\pgfpathmoveto{\pgfpoint{143.45703628472222pt}{148.79980086805554pt}}
\pgflineto{\pgfpoint{140.1593417534722pt}{147.72832022569443pt}}
\pgflineto{\pgfpoint{141.07080008680555pt}{147.06610685763889pt}}
\pgfpathclose
\pgfusepath{fill,stroke}
\pgfpathmoveto{\pgfpoint{143.45703628472222pt}{148.79980086805554pt}}
\pgflineto{\pgfpoint{139.81119383680556pt}{148.79980086805554pt}}
\pgflineto{\pgfpoint{140.1593417534722pt}{147.72832022569443pt}}
\pgfpathclose
\pgfusepath{fill,stroke}
\pgfpathmoveto{\pgfpoint{143.45703628472222pt}{148.79980086805554pt}}
\pgflineto{\pgfpoint{140.1593417534722pt}{149.87128151041665pt}}
\pgflineto{\pgfpoint{139.81119383680556pt}{148.79980086805554pt}}
\pgfpathclose
\pgfusepath{fill,stroke}
\pgfpathmoveto{\pgfpoint{143.45703628472222pt}{148.79980086805554pt}}
\pgflineto{\pgfpoint{141.07080008680555pt}{150.5334948784722pt}}
\pgflineto{\pgfpoint{140.1593417534722pt}{149.87128151041665pt}}
\pgfpathclose
\pgfusepath{fill,stroke}
\pgfpathmoveto{\pgfpoint{143.45703628472222pt}{148.79980086805554pt}}
\pgflineto{\pgfpoint{142.1974300347222pt}{150.5334948784722pt}}
\pgflineto{\pgfpoint{141.07080008680555pt}{150.5334948784722pt}}
\pgfpathclose
\pgfusepath{fill,stroke}
\pgfpathmoveto{\pgfpoint{143.45703628472222pt}{148.79980086805554pt}}
\pgflineto{\pgfpoint{143.1088792534722pt}{149.87128151041665pt}}
\pgflineto{\pgfpoint{142.1974300347222pt}{150.5334948784722pt}}
\pgfpathclose
\pgfusepath{fill,stroke}
\pgfpathmoveto{\pgfpoint{289.91666362847224pt}{209.36250963541667pt}}
\pgflineto{\pgfpoint{288.65706649305554pt}{207.6288247395833pt}}
\pgflineto{\pgfpoint{289.56852482638885pt}{208.29103810763885pt}}
\pgfpathclose
\pgfusepath{fill,stroke}
\pgfpathmoveto{\pgfpoint{289.91666362847224pt}{209.36250963541667pt}}
\pgflineto{\pgfpoint{287.5304274305555pt}{207.6288247395833pt}}
\pgflineto{\pgfpoint{288.65706649305554pt}{207.6288247395833pt}}
\pgfpathclose
\pgfusepath{fill,stroke}
\pgfpathmoveto{\pgfpoint{289.91666362847224pt}{209.36250963541667pt}}
\pgflineto{\pgfpoint{286.6189690972222pt}{208.29103810763885pt}}
\pgflineto{\pgfpoint{287.5304274305555pt}{207.6288247395833pt}}
\pgfpathclose
\pgfusepath{fill,stroke}
\pgfpathmoveto{\pgfpoint{289.91666362847224pt}{209.36250963541667pt}}
\pgflineto{\pgfpoint{286.27083029513886pt}{209.36250963541667pt}}
\pgflineto{\pgfpoint{286.6189690972222pt}{208.29103810763885pt}}
\pgfpathclose
\pgfusepath{fill,stroke}
\pgfpathmoveto{\pgfpoint{289.91666362847224pt}{209.36250963541667pt}}
\pgflineto{\pgfpoint{286.6189690972222pt}{210.43399999999997pt}}
\pgflineto{\pgfpoint{286.27083029513886pt}{209.36250963541667pt}}
\pgfpathclose
\pgfusepath{fill,stroke}
\pgfpathmoveto{\pgfpoint{289.91666362847224pt}{209.36250963541667pt}}
\pgflineto{\pgfpoint{287.5304274305555pt}{211.09621276041665pt}}
\pgflineto{\pgfpoint{286.6189690972222pt}{210.43399999999997pt}}
\pgfpathclose
\pgfusepath{fill,stroke}
\pgfpathmoveto{\pgfpoint{289.91666362847224pt}{209.36250963541667pt}}
\pgflineto{\pgfpoint{288.65706649305554pt}{211.09621276041665pt}}
\pgflineto{\pgfpoint{287.5304274305555pt}{211.09621276041665pt}}
\pgfpathclose
\pgfusepath{fill,stroke}
\pgfpathmoveto{\pgfpoint{289.91666362847224pt}{209.36250963541667pt}}
\pgflineto{\pgfpoint{289.56852482638885pt}{210.43399999999997pt}}
\pgflineto{\pgfpoint{288.65706649305554pt}{211.09621276041665pt}}
\pgfpathclose
\pgfusepath{fill,stroke}
\pgfpathmoveto{\pgfpoint{156.46423211805555pt}{127.14506267361111pt}}
\pgflineto{\pgfpoint{155.20462586805556pt}{125.4113595486111pt}}
\pgflineto{\pgfpoint{156.1160933159722pt}{126.07357230902777pt}}
\pgfpathclose
\pgfusepath{fill,stroke}
\pgfpathmoveto{\pgfpoint{156.46423211805555pt}{127.14506267361111pt}}
\pgflineto{\pgfpoint{154.0780056423611pt}{125.4113595486111pt}}
\pgflineto{\pgfpoint{155.20462586805556pt}{125.4113595486111pt}}
\pgfpathclose
\pgfusepath{fill,stroke}
\pgfpathmoveto{\pgfpoint{156.46423211805555pt}{127.14506267361111pt}}
\pgflineto{\pgfpoint{153.16654730902778pt}{126.07357230902777pt}}
\pgflineto{\pgfpoint{154.0780056423611pt}{125.4113595486111pt}}
\pgfpathclose
\pgfusepath{fill,stroke}
\pgfpathmoveto{\pgfpoint{156.46423211805555pt}{127.14506267361111pt}}
\pgflineto{\pgfpoint{152.8183987847222pt}{127.14506267361111pt}}
\pgflineto{\pgfpoint{153.16654730902778pt}{126.07357230902777pt}}
\pgfpathclose
\pgfusepath{fill,stroke}
\pgfpathmoveto{\pgfpoint{156.46423211805555pt}{127.14506267361111pt}}
\pgflineto{\pgfpoint{153.16654730902778pt}{128.21654331597222pt}}
\pgflineto{\pgfpoint{152.8183987847222pt}{127.14506267361111pt}}
\pgfpathclose
\pgfusepath{fill,stroke}
\pgfpathmoveto{\pgfpoint{156.46423211805555pt}{127.14506267361111pt}}
\pgflineto{\pgfpoint{154.0780056423611pt}{128.87875607638887pt}}
\pgflineto{\pgfpoint{153.16654730902778pt}{128.21654331597222pt}}
\pgfpathclose
\pgfusepath{fill,stroke}
\pgfpathmoveto{\pgfpoint{156.46423211805555pt}{127.14506267361111pt}}
\pgflineto{\pgfpoint{155.20462586805556pt}{128.87875607638887pt}}
\pgflineto{\pgfpoint{154.0780056423611pt}{128.87875607638887pt}}
\pgfpathclose
\pgfusepath{fill,stroke}
\pgfpathmoveto{\pgfpoint{156.46423211805555pt}{127.14506267361111pt}}
\pgflineto{\pgfpoint{156.1160933159722pt}{128.21654331597222pt}}
\pgflineto{\pgfpoint{155.20462586805556pt}{128.87875607638887pt}}
\pgfpathclose
\pgfusepath{fill,stroke}
\pgfpathmoveto{\pgfpoint{213.94674522569443pt}{104.33931909722222pt}}
\pgflineto{\pgfpoint{212.68714809027773pt}{102.60561597222222pt}}
\pgflineto{\pgfpoint{213.59860642361107pt}{103.26782873263889pt}}
\pgfpathclose
\pgfusepath{fill,stroke}
\pgfpathmoveto{\pgfpoint{213.94674522569443pt}{104.33931909722222pt}}
\pgflineto{\pgfpoint{211.56050902777778pt}{102.60561597222222pt}}
\pgflineto{\pgfpoint{212.68714809027773pt}{102.60561597222222pt}}
\pgfpathclose
\pgfusepath{fill,stroke}
\pgfpathmoveto{\pgfpoint{213.94674522569443pt}{104.33931909722222pt}}
\pgflineto{\pgfpoint{210.64905069444444pt}{103.26782873263889pt}}
\pgflineto{\pgfpoint{211.56050902777778pt}{102.60561597222222pt}}
\pgfpathclose
\pgfusepath{fill,stroke}
\pgfpathmoveto{\pgfpoint{213.94674522569443pt}{104.33931909722222pt}}
\pgflineto{\pgfpoint{210.30091189236109pt}{104.33931909722222pt}}
\pgflineto{\pgfpoint{210.64905069444444pt}{103.26782873263889pt}}
\pgfpathclose
\pgfusepath{fill,stroke}
\pgfpathmoveto{\pgfpoint{213.94674522569443pt}{104.33931909722222pt}}
\pgflineto{\pgfpoint{210.64905069444444pt}{105.41079973958333pt}}
\pgflineto{\pgfpoint{210.30091189236109pt}{104.33931909722222pt}}
\pgfpathclose
\pgfusepath{fill,stroke}
\pgfpathmoveto{\pgfpoint{213.94674522569443pt}{104.33931909722222pt}}
\pgflineto{\pgfpoint{211.56050902777778pt}{106.0730125pt}}
\pgflineto{\pgfpoint{210.64905069444444pt}{105.41079973958333pt}}
\pgfpathclose
\pgfusepath{fill,stroke}
\pgfpathmoveto{\pgfpoint{213.94674522569443pt}{104.33931909722222pt}}
\pgflineto{\pgfpoint{212.68714809027773pt}{106.0730125pt}}
\pgflineto{\pgfpoint{211.56050902777778pt}{106.0730125pt}}
\pgfpathclose
\pgfusepath{fill,stroke}
\pgfpathmoveto{\pgfpoint{213.94674522569443pt}{104.33931909722222pt}}
\pgflineto{\pgfpoint{213.59860642361107pt}{105.41079973958333pt}}
\pgflineto{\pgfpoint{212.68714809027773pt}{106.0730125pt}}
\pgfpathclose
\pgfusepath{fill,stroke}
\pgfpathmoveto{\pgfpoint{215.08148776041665pt}{109.88457083333331pt}}
\pgflineto{\pgfpoint{213.82189062499998pt}{108.15087682291666pt}}
\pgflineto{\pgfpoint{214.73334895833332pt}{108.81308958333332pt}}
\pgfpathclose
\pgfusepath{fill,stroke}
\pgfpathmoveto{\pgfpoint{215.08148776041665pt}{109.88457083333331pt}}
\pgflineto{\pgfpoint{212.69527039930554pt}{108.15087682291666pt}}
\pgflineto{\pgfpoint{213.82189062499998pt}{108.15087682291666pt}}
\pgfpathclose
\pgfusepath{fill,stroke}
\pgfpathmoveto{\pgfpoint{215.08148776041665pt}{109.88457083333331pt}}
\pgflineto{\pgfpoint{211.7838120659722pt}{108.81308958333332pt}}
\pgflineto{\pgfpoint{212.69527039930554pt}{108.15087682291666pt}}
\pgfpathclose
\pgfusepath{fill,stroke}
\pgfpathmoveto{\pgfpoint{215.08148776041665pt}{109.88457083333331pt}}
\pgflineto{\pgfpoint{211.4356544270833pt}{109.88457083333331pt}}
\pgflineto{\pgfpoint{211.7838120659722pt}{108.81308958333332pt}}
\pgfpathclose
\pgfusepath{fill,stroke}
\pgfpathmoveto{\pgfpoint{215.08148776041665pt}{109.88457083333331pt}}
\pgflineto{\pgfpoint{211.7838120659722pt}{110.95605147569444pt}}
\pgflineto{\pgfpoint{211.4356544270833pt}{109.88457083333331pt}}
\pgfpathclose
\pgfusepath{fill,stroke}
\pgfpathmoveto{\pgfpoint{215.08148776041665pt}{109.88457083333331pt}}
\pgflineto{\pgfpoint{212.69527039930554pt}{111.61826423611109pt}}
\pgflineto{\pgfpoint{211.7838120659722pt}{110.95605147569444pt}}
\pgfpathclose
\pgfusepath{fill,stroke}
\pgfpathmoveto{\pgfpoint{215.08148776041665pt}{109.88457083333331pt}}
\pgflineto{\pgfpoint{213.82189062499998pt}{111.61826423611109pt}}
\pgflineto{\pgfpoint{212.69527039930554pt}{111.61826423611109pt}}
\pgfpathclose
\pgfusepath{fill,stroke}
\pgfpathmoveto{\pgfpoint{215.08148776041665pt}{109.88457083333331pt}}
\pgflineto{\pgfpoint{214.73334895833332pt}{110.95605147569444pt}}
\pgflineto{\pgfpoint{213.82189062499998pt}{111.61826423611109pt}}
\pgfpathclose
\pgfusepath{fill,stroke}
\pgfpathmoveto{\pgfpoint{140.64666024305555pt}{143.32300269097223pt}}
\pgflineto{\pgfpoint{139.38705399305556pt}{141.58929956597223pt}}
\pgflineto{\pgfpoint{140.29851232638887pt}{142.25151293402777pt}}
\pgfpathclose
\pgfusepath{fill,stroke}
\pgfpathmoveto{\pgfpoint{140.64666024305555pt}{143.32300269097223pt}}
\pgflineto{\pgfpoint{138.26042404513888pt}{141.58929956597223pt}}
\pgflineto{\pgfpoint{139.38705399305556pt}{141.58929956597223pt}}
\pgfpathclose
\pgfusepath{fill,stroke}
\pgfpathmoveto{\pgfpoint{140.64666024305555pt}{143.32300269097223pt}}
\pgflineto{\pgfpoint{137.34897482638888pt}{142.25151293402777pt}}
\pgflineto{\pgfpoint{138.26042404513888pt}{141.58929956597223pt}}
\pgfpathclose
\pgfusepath{fill,stroke}
\pgfpathmoveto{\pgfpoint{140.64666024305555pt}{143.32300269097223pt}}
\pgflineto{\pgfpoint{137.0008177951389pt}{143.32300269097223pt}}
\pgflineto{\pgfpoint{137.34897482638888pt}{142.25151293402777pt}}
\pgfpathclose
\pgfusepath{fill,stroke}
\pgfpathmoveto{\pgfpoint{140.64666024305555pt}{143.32300269097223pt}}
\pgflineto{\pgfpoint{137.34897482638888pt}{144.3944839409722pt}}
\pgflineto{\pgfpoint{137.0008177951389pt}{143.32300269097223pt}}
\pgfpathclose
\pgfusepath{fill,stroke}
\pgfpathmoveto{\pgfpoint{140.64666024305555pt}{143.32300269097223pt}}
\pgflineto{\pgfpoint{138.26042404513888pt}{145.0566967013889pt}}
\pgflineto{\pgfpoint{137.34897482638888pt}{144.3944839409722pt}}
\pgfpathclose
\pgfusepath{fill,stroke}
\pgfpathmoveto{\pgfpoint{140.64666024305555pt}{143.32300269097223pt}}
\pgflineto{\pgfpoint{139.38705399305556pt}{145.0566967013889pt}}
\pgflineto{\pgfpoint{138.26042404513888pt}{145.0566967013889pt}}
\pgfpathclose
\pgfusepath{fill,stroke}
\pgfpathmoveto{\pgfpoint{140.64666024305555pt}{143.32300269097223pt}}
\pgflineto{\pgfpoint{140.29851232638887pt}{144.3944839409722pt}}
\pgflineto{\pgfpoint{139.38705399305556pt}{145.0566967013889pt}}
\pgfpathclose
\pgfusepath{fill,stroke}
\pgfpathmoveto{\pgfpoint{157.36037369791666pt}{140.08398967013886pt}}
\pgflineto{\pgfpoint{156.10075772569442pt}{138.3502962673611pt}}
\pgflineto{\pgfpoint{157.01221605902776pt}{139.01250902777775pt}}
\pgfpathclose
\pgfusepath{fill,stroke}
\pgfpathmoveto{\pgfpoint{157.36037369791666pt}{140.08398967013886pt}}
\pgflineto{\pgfpoint{154.97413749999998pt}{138.3502962673611pt}}
\pgflineto{\pgfpoint{156.10075772569442pt}{138.3502962673611pt}}
\pgfpathclose
\pgfusepath{fill,stroke}
\pgfpathmoveto{\pgfpoint{157.36037369791666pt}{140.08398967013886pt}}
\pgflineto{\pgfpoint{154.06267916666664pt}{139.01250902777775pt}}
\pgflineto{\pgfpoint{154.97413749999998pt}{138.3502962673611pt}}
\pgfpathclose
\pgfusepath{fill,stroke}
\pgfpathmoveto{\pgfpoint{157.36037369791666pt}{140.08398967013886pt}}
\pgflineto{\pgfpoint{153.71453125pt}{140.08398967013886pt}}
\pgflineto{\pgfpoint{154.06267916666664pt}{139.01250902777775pt}}
\pgfpathclose
\pgfusepath{fill,stroke}
\pgfpathmoveto{\pgfpoint{157.36037369791666pt}{140.08398967013886pt}}
\pgflineto{\pgfpoint{154.06267916666664pt}{141.1554709201389pt}}
\pgflineto{\pgfpoint{153.71453125pt}{140.08398967013886pt}}
\pgfpathclose
\pgfusepath{fill,stroke}
\pgfpathmoveto{\pgfpoint{157.36037369791666pt}{140.08398967013886pt}}
\pgflineto{\pgfpoint{154.97413749999998pt}{141.81768368055555pt}}
\pgflineto{\pgfpoint{154.06267916666664pt}{141.1554709201389pt}}
\pgfpathclose
\pgfusepath{fill,stroke}
\pgfpathmoveto{\pgfpoint{157.36037369791666pt}{140.08398967013886pt}}
\pgflineto{\pgfpoint{156.10075772569442pt}{141.81768368055555pt}}
\pgflineto{\pgfpoint{154.97413749999998pt}{141.81768368055555pt}}
\pgfpathclose
\pgfusepath{fill,stroke}
\pgfpathmoveto{\pgfpoint{157.36037369791666pt}{140.08398967013886pt}}
\pgflineto{\pgfpoint{157.01221605902776pt}{141.1554709201389pt}}
\pgflineto{\pgfpoint{156.10075772569442pt}{141.81768368055555pt}}
\pgfpathclose
\pgfusepath{fill,stroke}
\pgfpathmoveto{\pgfpoint{119.78369010416665pt}{165.96958619791664pt}}
\pgflineto{\pgfpoint{118.52409296874998pt}{164.23588307291666pt}}
\pgflineto{\pgfpoint{119.43555130208333pt}{164.8980964409722pt}}
\pgfpathclose
\pgfusepath{fill,stroke}
\pgfpathmoveto{\pgfpoint{119.78369010416665pt}{165.96958619791664pt}}
\pgflineto{\pgfpoint{117.39747274305554pt}{164.23588307291666pt}}
\pgflineto{\pgfpoint{118.52409296874998pt}{164.23588307291666pt}}
\pgfpathclose
\pgfusepath{fill,stroke}
\pgfpathmoveto{\pgfpoint{119.78369010416665pt}{165.96958619791664pt}}
\pgflineto{\pgfpoint{116.48601440972222pt}{164.8980964409722pt}}
\pgflineto{\pgfpoint{117.39747274305554pt}{164.23588307291666pt}}
\pgfpathclose
\pgfusepath{fill,stroke}
\pgfpathmoveto{\pgfpoint{119.78369010416665pt}{165.96958619791664pt}}
\pgflineto{\pgfpoint{116.13786588541666pt}{165.96958619791664pt}}
\pgflineto{\pgfpoint{116.48601440972222pt}{164.8980964409722pt}}
\pgfpathclose
\pgfusepath{fill,stroke}
\pgfpathmoveto{\pgfpoint{119.78369010416665pt}{165.96958619791664pt}}
\pgflineto{\pgfpoint{116.48601440972222pt}{167.04105772569443pt}}
\pgflineto{\pgfpoint{116.13786588541666pt}{165.96958619791664pt}}
\pgfpathclose
\pgfusepath{fill,stroke}
\pgfpathmoveto{\pgfpoint{119.78369010416665pt}{165.96958619791664pt}}
\pgflineto{\pgfpoint{117.39747274305554pt}{167.70327109374998pt}}
\pgflineto{\pgfpoint{116.48601440972222pt}{167.04105772569443pt}}
\pgfpathclose
\pgfusepath{fill,stroke}
\pgfpathmoveto{\pgfpoint{119.78369010416665pt}{165.96958619791664pt}}
\pgflineto{\pgfpoint{118.52409296874998pt}{167.70327109374998pt}}
\pgflineto{\pgfpoint{117.39747274305554pt}{167.70327109374998pt}}
\pgfpathclose
\pgfusepath{fill,stroke}
\pgfpathmoveto{\pgfpoint{119.78369010416665pt}{165.96958619791664pt}}
\pgflineto{\pgfpoint{119.43555130208333pt}{167.04105772569443pt}}
\pgflineto{\pgfpoint{118.52409296874998pt}{167.70327109374998pt}}
\pgfpathclose
\pgfusepath{fill,stroke}
\pgfpathmoveto{\pgfpoint{233.3859603298611pt}{119.72570434027777pt}}
\pgflineto{\pgfpoint{232.12636319444442pt}{117.99200121527777pt}}
\pgflineto{\pgfpoint{233.03782152777774pt}{118.65421397569445pt}}
\pgfpathclose
\pgfusepath{fill,stroke}
\pgfpathmoveto{\pgfpoint{233.3859603298611pt}{119.72570434027777pt}}
\pgflineto{\pgfpoint{230.99972413194442pt}{117.99200121527777pt}}
\pgflineto{\pgfpoint{232.12636319444442pt}{117.99200121527777pt}}
\pgfpathclose
\pgfusepath{fill,stroke}
\pgfpathmoveto{\pgfpoint{233.3859603298611pt}{119.72570434027777pt}}
\pgflineto{\pgfpoint{230.08826579861108pt}{118.65421397569445pt}}
\pgflineto{\pgfpoint{230.99972413194442pt}{117.99200121527777pt}}
\pgfpathclose
\pgfusepath{fill,stroke}
\pgfpathmoveto{\pgfpoint{233.3859603298611pt}{119.72570434027777pt}}
\pgflineto{\pgfpoint{229.74012699652778pt}{119.72570434027777pt}}
\pgflineto{\pgfpoint{230.08826579861108pt}{118.65421397569445pt}}
\pgfpathclose
\pgfusepath{fill,stroke}
\pgfpathmoveto{\pgfpoint{233.3859603298611pt}{119.72570434027777pt}}
\pgflineto{\pgfpoint{230.08826579861108pt}{120.79718498263888pt}}
\pgflineto{\pgfpoint{229.74012699652778pt}{119.72570434027777pt}}
\pgfpathclose
\pgfusepath{fill,stroke}
\pgfpathmoveto{\pgfpoint{233.3859603298611pt}{119.72570434027777pt}}
\pgflineto{\pgfpoint{230.99972413194442pt}{121.45939774305553pt}}
\pgflineto{\pgfpoint{230.08826579861108pt}{120.79718498263888pt}}
\pgfpathclose
\pgfusepath{fill,stroke}
\pgfpathmoveto{\pgfpoint{233.3859603298611pt}{119.72570434027777pt}}
\pgflineto{\pgfpoint{232.12636319444442pt}{121.45939774305553pt}}
\pgflineto{\pgfpoint{230.99972413194442pt}{121.45939774305553pt}}
\pgfpathclose
\pgfusepath{fill,stroke}
\pgfpathmoveto{\pgfpoint{233.3859603298611pt}{119.72570434027777pt}}
\pgflineto{\pgfpoint{233.03782152777774pt}{120.79718498263888pt}}
\pgflineto{\pgfpoint{232.12636319444442pt}{121.45939774305553pt}}
\pgfpathclose
\pgfusepath{fill,stroke}
\pgfpathmoveto{\pgfpoint{137.22915512152775pt}{148.20933324652776pt}}
\pgflineto{\pgfpoint{135.96954887152776pt}{146.47563984374997pt}}
\pgflineto{\pgfpoint{136.8810072048611pt}{147.13785260416665pt}}
\pgfpathclose
\pgfusepath{fill,stroke}
\pgfpathmoveto{\pgfpoint{137.22915512152775pt}{148.20933324652776pt}}
\pgflineto{\pgfpoint{134.84292803819443pt}{146.47563984374997pt}}
\pgflineto{\pgfpoint{135.96954887152776pt}{146.47563984374997pt}}
\pgfpathclose
\pgfusepath{fill,stroke}
\pgfpathmoveto{\pgfpoint{137.22915512152775pt}{148.20933324652776pt}}
\pgflineto{\pgfpoint{133.93146970486112pt}{147.13785260416665pt}}
\pgflineto{\pgfpoint{134.84292803819443pt}{146.47563984374997pt}}
\pgfpathclose
\pgfusepath{fill,stroke}
\pgfpathmoveto{\pgfpoint{137.22915512152775pt}{148.20933324652776pt}}
\pgflineto{\pgfpoint{133.58332178819444pt}{148.20933324652776pt}}
\pgflineto{\pgfpoint{133.93146970486112pt}{147.13785260416665pt}}
\pgfpathclose
\pgfusepath{fill,stroke}
\pgfpathmoveto{\pgfpoint{137.22915512152775pt}{148.20933324652776pt}}
\pgflineto{\pgfpoint{133.93146970486112pt}{149.2808236111111pt}}
\pgflineto{\pgfpoint{133.58332178819444pt}{148.20933324652776pt}}
\pgfpathclose
\pgfusepath{fill,stroke}
\pgfpathmoveto{\pgfpoint{137.22915512152775pt}{148.20933324652776pt}}
\pgflineto{\pgfpoint{134.84292803819443pt}{149.94303637152777pt}}
\pgflineto{\pgfpoint{133.93146970486112pt}{149.2808236111111pt}}
\pgfpathclose
\pgfusepath{fill,stroke}
\pgfpathmoveto{\pgfpoint{137.22915512152775pt}{148.20933324652776pt}}
\pgflineto{\pgfpoint{135.96954887152776pt}{149.94303637152777pt}}
\pgflineto{\pgfpoint{134.84292803819443pt}{149.94303637152777pt}}
\pgfpathclose
\pgfusepath{fill,stroke}
\pgfpathmoveto{\pgfpoint{137.22915512152775pt}{148.20933324652776pt}}
\pgflineto{\pgfpoint{136.8810072048611pt}{149.2808236111111pt}}
\pgflineto{\pgfpoint{135.96954887152776pt}{149.94303637152777pt}}
\pgfpathclose
\pgfusepath{fill,stroke}
\pgfpathmoveto{\pgfpoint{122.58345920138888pt}{164.81902621527777pt}}
\pgflineto{\pgfpoint{121.32385295138889pt}{163.08532309027777pt}}
\pgflineto{\pgfpoint{122.23531128472221pt}{163.74753585069442pt}}
\pgfpathclose
\pgfusepath{fill,stroke}
\pgfpathmoveto{\pgfpoint{122.58345920138888pt}{164.81902621527777pt}}
\pgflineto{\pgfpoint{120.19723272569443pt}{163.08532309027777pt}}
\pgflineto{\pgfpoint{121.32385295138889pt}{163.08532309027777pt}}
\pgfpathclose
\pgfusepath{fill,stroke}
\pgfpathmoveto{\pgfpoint{122.58345920138888pt}{164.81902621527777pt}}
\pgflineto{\pgfpoint{119.2857743923611pt}{163.74753585069442pt}}
\pgflineto{\pgfpoint{120.19723272569443pt}{163.08532309027777pt}}
\pgfpathclose
\pgfusepath{fill,stroke}
\pgfpathmoveto{\pgfpoint{122.58345920138888pt}{164.81902621527777pt}}
\pgflineto{\pgfpoint{118.93763559027778pt}{164.81902621527777pt}}
\pgflineto{\pgfpoint{119.2857743923611pt}{163.74753585069442pt}}
\pgfpathclose
\pgfusepath{fill,stroke}
\pgfpathmoveto{\pgfpoint{122.58345920138888pt}{164.81902621527777pt}}
\pgflineto{\pgfpoint{119.2857743923611pt}{165.89049774305553pt}}
\pgflineto{\pgfpoint{118.93763559027778pt}{164.81902621527777pt}}
\pgfpathclose
\pgfusepath{fill,stroke}
\pgfpathmoveto{\pgfpoint{122.58345920138888pt}{164.81902621527777pt}}
\pgflineto{\pgfpoint{120.19723272569443pt}{166.5527105034722pt}}
\pgflineto{\pgfpoint{119.2857743923611pt}{165.89049774305553pt}}
\pgfpathclose
\pgfusepath{fill,stroke}
\pgfpathmoveto{\pgfpoint{122.58345920138888pt}{164.81902621527777pt}}
\pgflineto{\pgfpoint{121.32385295138889pt}{166.5527105034722pt}}
\pgflineto{\pgfpoint{120.19723272569443pt}{166.5527105034722pt}}
\pgfpathclose
\pgfusepath{fill,stroke}
\pgfpathmoveto{\pgfpoint{122.58345920138888pt}{164.81902621527777pt}}
\pgflineto{\pgfpoint{122.23531128472221pt}{165.89049774305553pt}}
\pgflineto{\pgfpoint{121.32385295138889pt}{166.5527105034722pt}}
\pgfpathclose
\pgfusepath{fill,stroke}
\pgfpathmoveto{\pgfpoint{117.91984774305554pt}{172.42022352430556pt}}
\pgflineto{\pgfpoint{116.66024149305555pt}{170.68652039930555pt}}
\pgflineto{\pgfpoint{117.57169982638888pt}{171.3487331597222pt}}
\pgfpathclose
\pgfusepath{fill,stroke}
\pgfpathmoveto{\pgfpoint{117.91984774305554pt}{172.42022352430556pt}}
\pgflineto{\pgfpoint{115.53361154513887pt}{170.68652039930555pt}}
\pgflineto{\pgfpoint{116.66024149305555pt}{170.68652039930555pt}}
\pgfpathclose
\pgfusepath{fill,stroke}
\pgfpathmoveto{\pgfpoint{117.91984774305554pt}{172.42022352430556pt}}
\pgflineto{\pgfpoint{114.62215321180554pt}{171.3487331597222pt}}
\pgflineto{\pgfpoint{115.53361154513887pt}{170.68652039930555pt}}
\pgfpathclose
\pgfusepath{fill,stroke}
\pgfpathmoveto{\pgfpoint{117.91984774305554pt}{172.42022352430556pt}}
\pgflineto{\pgfpoint{114.27401440972221pt}{172.42022352430556pt}}
\pgflineto{\pgfpoint{114.62215321180554pt}{171.3487331597222pt}}
\pgfpathclose
\pgfusepath{fill,stroke}
\pgfpathmoveto{\pgfpoint{117.91984774305554pt}{172.42022352430556pt}}
\pgflineto{\pgfpoint{114.62215321180554pt}{173.49169505208332pt}}
\pgflineto{\pgfpoint{114.27401440972221pt}{172.42022352430556pt}}
\pgfpathclose
\pgfusepath{fill,stroke}
\pgfpathmoveto{\pgfpoint{117.91984774305554pt}{172.42022352430556pt}}
\pgflineto{\pgfpoint{115.53361154513887pt}{174.1539078125pt}}
\pgflineto{\pgfpoint{114.62215321180554pt}{173.49169505208332pt}}
\pgfpathclose
\pgfusepath{fill,stroke}
\pgfpathmoveto{\pgfpoint{117.91984774305554pt}{172.42022352430556pt}}
\pgflineto{\pgfpoint{116.66024149305555pt}{174.1539078125pt}}
\pgflineto{\pgfpoint{115.53361154513887pt}{174.1539078125pt}}
\pgfpathclose
\pgfusepath{fill,stroke}
\pgfpathmoveto{\pgfpoint{117.91984774305554pt}{172.42022352430556pt}}
\pgflineto{\pgfpoint{117.57169982638888pt}{173.49169505208332pt}}
\pgflineto{\pgfpoint{116.66024149305555pt}{174.1539078125pt}}
\pgfpathclose
\pgfusepath{fill,stroke}
\pgfpathmoveto{\pgfpoint{130.41004618055553pt}{151.51253541666665pt}}
\pgflineto{\pgfpoint{129.15043993055554pt}{149.77884140625pt}}
\pgflineto{\pgfpoint{130.06189826388888pt}{150.44104505208333pt}}
\pgfpathclose
\pgfusepath{fill,stroke}
\pgfpathmoveto{\pgfpoint{130.41004618055553pt}{151.51253541666665pt}}
\pgflineto{\pgfpoint{128.0238190972222pt}{149.77884140625pt}}
\pgflineto{\pgfpoint{129.15043993055554pt}{149.77884140625pt}}
\pgfpathclose
\pgfusepath{fill,stroke}
\pgfpathmoveto{\pgfpoint{130.41004618055553pt}{151.51253541666665pt}}
\pgflineto{\pgfpoint{127.11235164930555pt}{150.44104505208333pt}}
\pgflineto{\pgfpoint{128.0238190972222pt}{149.77884140625pt}}
\pgfpathclose
\pgfusepath{fill,stroke}
\pgfpathmoveto{\pgfpoint{130.41004618055553pt}{151.51253541666665pt}}
\pgflineto{\pgfpoint{126.76421284722221pt}{151.51253541666665pt}}
\pgflineto{\pgfpoint{127.11235164930555pt}{150.44104505208333pt}}
\pgfpathclose
\pgfusepath{fill,stroke}
\pgfpathmoveto{\pgfpoint{130.41004618055553pt}{151.51253541666665pt}}
\pgflineto{\pgfpoint{127.11235164930555pt}{152.58401605902776pt}}
\pgflineto{\pgfpoint{126.76421284722221pt}{151.51253541666665pt}}
\pgfpathclose
\pgfusepath{fill,stroke}
\pgfpathmoveto{\pgfpoint{130.41004618055553pt}{151.51253541666665pt}}
\pgflineto{\pgfpoint{128.0238190972222pt}{153.24622942708334pt}}
\pgflineto{\pgfpoint{127.11235164930555pt}{152.58401605902776pt}}
\pgfpathclose
\pgfusepath{fill,stroke}
\pgfpathmoveto{\pgfpoint{130.41004618055553pt}{151.51253541666665pt}}
\pgflineto{\pgfpoint{129.15043993055554pt}{153.24622942708334pt}}
\pgflineto{\pgfpoint{128.0238190972222pt}{153.24622942708334pt}}
\pgfpathclose
\pgfusepath{fill,stroke}
\pgfpathmoveto{\pgfpoint{130.41004618055553pt}{151.51253541666665pt}}
\pgflineto{\pgfpoint{130.06189826388888pt}{152.58401605902776pt}}
\pgflineto{\pgfpoint{129.15043993055554pt}{153.24622942708334pt}}
\pgfpathclose
\pgfusepath{fill,stroke}
\pgfpathmoveto{\pgfpoint{134.6918714409722pt}{150.4941915798611pt}}
\pgflineto{\pgfpoint{133.43227430555555pt}{148.76049756944442pt}}
\pgflineto{\pgfpoint{134.34373263888887pt}{149.4227103298611pt}}
\pgfpathclose
\pgfusepath{fill,stroke}
\pgfpathmoveto{\pgfpoint{134.6918714409722pt}{150.4941915798611pt}}
\pgflineto{\pgfpoint{132.30564435763887pt}{148.76049756944442pt}}
\pgflineto{\pgfpoint{133.43227430555555pt}{148.76049756944442pt}}
\pgfpathclose
\pgfusepath{fill,stroke}
\pgfpathmoveto{\pgfpoint{134.6918714409722pt}{150.4941915798611pt}}
\pgflineto{\pgfpoint{131.39418602430555pt}{149.4227103298611pt}}
\pgflineto{\pgfpoint{132.30564435763887pt}{148.76049756944442pt}}
\pgfpathclose
\pgfusepath{fill,stroke}
\pgfpathmoveto{\pgfpoint{134.6918714409722pt}{150.4941915798611pt}}
\pgflineto{\pgfpoint{131.04603810763888pt}{150.4941915798611pt}}
\pgflineto{\pgfpoint{131.39418602430555pt}{149.4227103298611pt}}
\pgfpathclose
\pgfusepath{fill,stroke}
\pgfpathmoveto{\pgfpoint{134.6918714409722pt}{150.4941915798611pt}}
\pgflineto{\pgfpoint{131.39418602430555pt}{151.56568133680554pt}}
\pgflineto{\pgfpoint{131.04603810763888pt}{150.4941915798611pt}}
\pgfpathclose
\pgfusepath{fill,stroke}
\pgfpathmoveto{\pgfpoint{134.6918714409722pt}{150.4941915798611pt}}
\pgflineto{\pgfpoint{132.30564435763887pt}{152.2278947048611pt}}
\pgflineto{\pgfpoint{131.39418602430555pt}{151.56568133680554pt}}
\pgfpathclose
\pgfusepath{fill,stroke}
\pgfpathmoveto{\pgfpoint{134.6918714409722pt}{150.4941915798611pt}}
\pgflineto{\pgfpoint{133.43227430555555pt}{152.2278947048611pt}}
\pgflineto{\pgfpoint{132.30564435763887pt}{152.2278947048611pt}}
\pgfpathclose
\pgfusepath{fill,stroke}
\pgfpathmoveto{\pgfpoint{134.6918714409722pt}{150.4941915798611pt}}
\pgflineto{\pgfpoint{134.34373263888887pt}{151.56568133680554pt}}
\pgflineto{\pgfpoint{133.43227430555555pt}{152.2278947048611pt}}
\pgfpathclose
\pgfusepath{fill,stroke}
\pgfpathmoveto{\pgfpoint{125.9346800347222pt}{159.43720868055553pt}}
\pgflineto{\pgfpoint{124.67508289930554pt}{157.70350555555555pt}}
\pgflineto{\pgfpoint{125.58654123263888pt}{158.36573715277777pt}}
\pgfpathclose
\pgfusepath{fill,stroke}
\pgfpathmoveto{\pgfpoint{125.9346800347222pt}{159.43720868055553pt}}
\pgflineto{\pgfpoint{123.5484626736111pt}{157.70350555555555pt}}
\pgflineto{\pgfpoint{124.67508289930554pt}{157.70350555555555pt}}
\pgfpathclose
\pgfusepath{fill,stroke}
\pgfpathmoveto{\pgfpoint{125.9346800347222pt}{159.43720868055553pt}}
\pgflineto{\pgfpoint{122.63699522569443pt}{158.36573715277777pt}}
\pgflineto{\pgfpoint{123.5484626736111pt}{157.70350555555555pt}}
\pgfpathclose
\pgfusepath{fill,stroke}
\pgfpathmoveto{\pgfpoint{125.9346800347222pt}{159.43720868055553pt}}
\pgflineto{\pgfpoint{122.28884670138888pt}{159.43720868055553pt}}
\pgflineto{\pgfpoint{122.63699522569443pt}{158.36573715277777pt}}
\pgfpathclose
\pgfusepath{fill,stroke}
\pgfpathmoveto{\pgfpoint{125.9346800347222pt}{159.43720868055553pt}}
\pgflineto{\pgfpoint{122.63699522569443pt}{160.50869904513888pt}}
\pgflineto{\pgfpoint{122.28884670138888pt}{159.43720868055553pt}}
\pgfpathclose
\pgfusepath{fill,stroke}
\pgfpathmoveto{\pgfpoint{125.9346800347222pt}{159.43720868055553pt}}
\pgflineto{\pgfpoint{123.5484626736111pt}{161.17091180555553pt}}
\pgflineto{\pgfpoint{122.63699522569443pt}{160.50869904513888pt}}
\pgfpathclose
\pgfusepath{fill,stroke}
\pgfpathmoveto{\pgfpoint{125.9346800347222pt}{159.43720868055553pt}}
\pgflineto{\pgfpoint{124.67508289930554pt}{161.17091180555553pt}}
\pgflineto{\pgfpoint{123.5484626736111pt}{161.17091180555553pt}}
\pgfpathclose
\pgfusepath{fill,stroke}
\pgfpathmoveto{\pgfpoint{125.9346800347222pt}{159.43720868055553pt}}
\pgflineto{\pgfpoint{125.58654123263888pt}{160.50869904513888pt}}
\pgflineto{\pgfpoint{124.67508289930554pt}{161.17091180555553pt}}
\pgfpathclose
\pgfusepath{fill,stroke}
\pgfpathmoveto{\pgfpoint{131.40693489583333pt}{150.0021353298611pt}}
\pgflineto{\pgfpoint{130.14733776041666pt}{148.26844131944443pt}}
\pgflineto{\pgfpoint{131.05879609375pt}{148.93064496527776pt}}
\pgfpathclose
\pgfusepath{fill,stroke}
\pgfpathmoveto{\pgfpoint{131.40693489583333pt}{150.0021353298611pt}}
\pgflineto{\pgfpoint{129.02071753472222pt}{148.26844131944443pt}}
\pgflineto{\pgfpoint{130.14733776041666pt}{148.26844131944443pt}}
\pgfpathclose
\pgfusepath{fill,stroke}
\pgfpathmoveto{\pgfpoint{131.40693489583333pt}{150.0021353298611pt}}
\pgflineto{\pgfpoint{128.10925008680553pt}{148.93064496527776pt}}
\pgflineto{\pgfpoint{129.02071753472222pt}{148.26844131944443pt}}
\pgfpathclose
\pgfusepath{fill,stroke}
\pgfpathmoveto{\pgfpoint{131.40693489583333pt}{150.0021353298611pt}}
\pgflineto{\pgfpoint{127.7611015625pt}{150.0021353298611pt}}
\pgflineto{\pgfpoint{128.10925008680553pt}{148.93064496527776pt}}
\pgfpathclose
\pgfusepath{fill,stroke}
\pgfpathmoveto{\pgfpoint{131.40693489583333pt}{150.0021353298611pt}}
\pgflineto{\pgfpoint{128.10925008680553pt}{151.0736159722222pt}}
\pgflineto{\pgfpoint{127.7611015625pt}{150.0021353298611pt}}
\pgfpathclose
\pgfusepath{fill,stroke}
\pgfpathmoveto{\pgfpoint{131.40693489583333pt}{150.0021353298611pt}}
\pgflineto{\pgfpoint{129.02071753472222pt}{151.73582873263888pt}}
\pgflineto{\pgfpoint{128.10925008680553pt}{151.0736159722222pt}}
\pgfpathclose
\pgfusepath{fill,stroke}
\pgfpathmoveto{\pgfpoint{131.40693489583333pt}{150.0021353298611pt}}
\pgflineto{\pgfpoint{130.14733776041666pt}{151.73582873263888pt}}
\pgflineto{\pgfpoint{129.02071753472222pt}{151.73582873263888pt}}
\pgfpathclose
\pgfusepath{fill,stroke}
\pgfpathmoveto{\pgfpoint{131.40693489583333pt}{150.0021353298611pt}}
\pgflineto{\pgfpoint{131.05879609375pt}{151.0736159722222pt}}
\pgflineto{\pgfpoint{130.14733776041666pt}{151.73582873263888pt}}
\pgfpathclose
\pgfusepath{fill,stroke}
\pgfpathmoveto{\pgfpoint{143.17333697916666pt}{142.67263949652778pt}}
\pgflineto{\pgfpoint{141.91373072916664pt}{140.9389454861111pt}}
\pgflineto{\pgfpoint{142.8251890625pt}{141.60115824652775pt}}
\pgfpathclose
\pgfusepath{fill,stroke}
\pgfpathmoveto{\pgfpoint{143.17333697916666pt}{142.67263949652778pt}}
\pgflineto{\pgfpoint{140.78710078125pt}{140.9389454861111pt}}
\pgflineto{\pgfpoint{141.91373072916664pt}{140.9389454861111pt}}
\pgfpathclose
\pgfusepath{fill,stroke}
\pgfpathmoveto{\pgfpoint{143.17333697916666pt}{142.67263949652778pt}}
\pgflineto{\pgfpoint{139.87564244791665pt}{141.60115824652775pt}}
\pgflineto{\pgfpoint{140.78710078125pt}{140.9389454861111pt}}
\pgfpathclose
\pgfusepath{fill,stroke}
\pgfpathmoveto{\pgfpoint{143.17333697916666pt}{142.67263949652778pt}}
\pgflineto{\pgfpoint{139.52750364583332pt}{142.67263949652778pt}}
\pgflineto{\pgfpoint{139.87564244791665pt}{141.60115824652775pt}}
\pgfpathclose
\pgfusepath{fill,stroke}
\pgfpathmoveto{\pgfpoint{143.17333697916666pt}{142.67263949652778pt}}
\pgflineto{\pgfpoint{139.87564244791665pt}{143.7441201388889pt}}
\pgflineto{\pgfpoint{139.52750364583332pt}{142.67263949652778pt}}
\pgfpathclose
\pgfusepath{fill,stroke}
\pgfpathmoveto{\pgfpoint{143.17333697916666pt}{142.67263949652778pt}}
\pgflineto{\pgfpoint{140.78710078125pt}{144.40633350694443pt}}
\pgflineto{\pgfpoint{139.87564244791665pt}{143.7441201388889pt}}
\pgfpathclose
\pgfusepath{fill,stroke}
\pgfpathmoveto{\pgfpoint{143.17333697916666pt}{142.67263949652778pt}}
\pgflineto{\pgfpoint{141.91373072916664pt}{144.40633350694443pt}}
\pgflineto{\pgfpoint{140.78710078125pt}{144.40633350694443pt}}
\pgfpathclose
\pgfusepath{fill,stroke}
\pgfpathmoveto{\pgfpoint{143.17333697916666pt}{142.67263949652778pt}}
\pgflineto{\pgfpoint{142.8251890625pt}{143.7441201388889pt}}
\pgflineto{\pgfpoint{141.91373072916664pt}{144.40633350694443pt}}
\pgfpathclose
\pgfusepath{fill,stroke}
\pgfpathmoveto{\pgfpoint{126.85467812499999pt}{152.34260538194442pt}}
\pgflineto{\pgfpoint{125.59507187499999pt}{150.60891197916666pt}}
\pgflineto{\pgfpoint{126.50653020833332pt}{151.271115625pt}}
\pgfpathclose
\pgfusepath{fill,stroke}
\pgfpathmoveto{\pgfpoint{126.85467812499999pt}{152.34260538194442pt}}
\pgflineto{\pgfpoint{124.46845104166665pt}{150.60891197916666pt}}
\pgflineto{\pgfpoint{125.59507187499999pt}{150.60891197916666pt}}
\pgfpathclose
\pgfusepath{fill,stroke}
\pgfpathmoveto{\pgfpoint{126.85467812499999pt}{152.34260538194442pt}}
\pgflineto{\pgfpoint{123.55699270833333pt}{151.271115625pt}}
\pgflineto{\pgfpoint{124.46845104166665pt}{150.60891197916666pt}}
\pgfpathclose
\pgfusepath{fill,stroke}
\pgfpathmoveto{\pgfpoint{126.85467812499999pt}{152.34260538194442pt}}
\pgflineto{\pgfpoint{123.20885390624998pt}{152.34260538194442pt}}
\pgflineto{\pgfpoint{123.55699270833333pt}{151.271115625pt}}
\pgfpathclose
\pgfusepath{fill,stroke}
\pgfpathmoveto{\pgfpoint{126.85467812499999pt}{152.34260538194442pt}}
\pgflineto{\pgfpoint{123.55699270833333pt}{153.41409574652778pt}}
\pgflineto{\pgfpoint{123.20885390624998pt}{152.34260538194442pt}}
\pgfpathclose
\pgfusepath{fill,stroke}
\pgfpathmoveto{\pgfpoint{126.85467812499999pt}{152.34260538194442pt}}
\pgflineto{\pgfpoint{124.46845104166665pt}{154.07629939236108pt}}
\pgflineto{\pgfpoint{123.55699270833333pt}{153.41409574652778pt}}
\pgfpathclose
\pgfusepath{fill,stroke}
\pgfpathmoveto{\pgfpoint{126.85467812499999pt}{152.34260538194442pt}}
\pgflineto{\pgfpoint{125.59507187499999pt}{154.07629939236108pt}}
\pgflineto{\pgfpoint{124.46845104166665pt}{154.07629939236108pt}}
\pgfpathclose
\pgfusepath{fill,stroke}
\pgfpathmoveto{\pgfpoint{126.85467812499999pt}{152.34260538194442pt}}
\pgflineto{\pgfpoint{126.50653020833332pt}{153.41409574652778pt}}
\pgflineto{\pgfpoint{125.59507187499999pt}{154.07629939236108pt}}
\pgfpathclose
\pgfusepath{fill,stroke}
\pgfpathmoveto{\pgfpoint{133.04277413194444pt}{137.46111762152776pt}}
\pgflineto{\pgfpoint{131.78316727430555pt}{135.72742421875pt}}
\pgflineto{\pgfpoint{132.69462560763887pt}{136.38963697916665pt}}
\pgfpathclose
\pgfusepath{fill,stroke}
\pgfpathmoveto{\pgfpoint{133.04277413194444pt}{137.46111762152776pt}}
\pgflineto{\pgfpoint{130.65654704861112pt}{135.72742421875pt}}
\pgflineto{\pgfpoint{131.78316727430555pt}{135.72742421875pt}}
\pgfpathclose
\pgfusepath{fill,stroke}
\pgfpathmoveto{\pgfpoint{133.04277413194444pt}{137.46111762152776pt}}
\pgflineto{\pgfpoint{129.74508871527777pt}{136.38963697916665pt}}
\pgflineto{\pgfpoint{130.65654704861112pt}{135.72742421875pt}}
\pgfpathclose
\pgfusepath{fill,stroke}
\pgfpathmoveto{\pgfpoint{133.04277413194444pt}{137.46111762152776pt}}
\pgflineto{\pgfpoint{129.3969407986111pt}{137.46111762152776pt}}
\pgflineto{\pgfpoint{129.74508871527777pt}{136.38963697916665pt}}
\pgfpathclose
\pgfusepath{fill,stroke}
\pgfpathmoveto{\pgfpoint{133.04277413194444pt}{137.46111762152776pt}}
\pgflineto{\pgfpoint{129.74508871527777pt}{138.53259887152777pt}}
\pgflineto{\pgfpoint{129.3969407986111pt}{137.46111762152776pt}}
\pgfpathclose
\pgfusepath{fill,stroke}
\pgfpathmoveto{\pgfpoint{133.04277413194444pt}{137.46111762152776pt}}
\pgflineto{\pgfpoint{130.65654704861112pt}{139.19481163194445pt}}
\pgflineto{\pgfpoint{129.74508871527777pt}{138.53259887152777pt}}
\pgfpathclose
\pgfusepath{fill,stroke}
\pgfpathmoveto{\pgfpoint{133.04277413194444pt}{137.46111762152776pt}}
\pgflineto{\pgfpoint{131.78316727430555pt}{139.19481163194445pt}}
\pgflineto{\pgfpoint{130.65654704861112pt}{139.19481163194445pt}}
\pgfpathclose
\pgfusepath{fill,stroke}
\pgfpathmoveto{\pgfpoint{133.04277413194444pt}{137.46111762152776pt}}
\pgflineto{\pgfpoint{132.69462560763887pt}{138.53259887152777pt}}
\pgflineto{\pgfpoint{131.78316727430555pt}{139.19481163194445pt}}
\pgfpathclose
\pgfusepath{fill,stroke}
\pgfpathmoveto{\pgfpoint{119.87648263888889pt}{175.17915017361108pt}}
\pgflineto{\pgfpoint{118.61688550347222pt}{173.44544704861107pt}}
\pgflineto{\pgfpoint{119.52834383680555pt}{174.10766041666668pt}}
\pgfpathclose
\pgfusepath{fill,stroke}
\pgfpathmoveto{\pgfpoint{119.87648263888889pt}{175.17915017361108pt}}
\pgflineto{\pgfpoint{117.49026527777778pt}{173.44544704861107pt}}
\pgflineto{\pgfpoint{118.61688550347222pt}{173.44544704861107pt}}
\pgfpathclose
\pgfusepath{fill,stroke}
\pgfpathmoveto{\pgfpoint{119.87648263888889pt}{175.17915017361108pt}}
\pgflineto{\pgfpoint{116.57880694444444pt}{174.10766041666668pt}}
\pgflineto{\pgfpoint{117.49026527777778pt}{173.44544704861107pt}}
\pgfpathclose
\pgfusepath{fill,stroke}
\pgfpathmoveto{\pgfpoint{119.87648263888889pt}{175.17915017361108pt}}
\pgflineto{\pgfpoint{116.23065842013888pt}{175.17915017361108pt}}
\pgflineto{\pgfpoint{116.57880694444444pt}{174.10766041666668pt}}
\pgfpathclose
\pgfusepath{fill,stroke}
\pgfpathmoveto{\pgfpoint{119.87648263888889pt}{175.17915017361108pt}}
\pgflineto{\pgfpoint{116.57880694444444pt}{176.2506223090278pt}}
\pgflineto{\pgfpoint{116.23065842013888pt}{175.17915017361108pt}}
\pgfpathclose
\pgfusepath{fill,stroke}
\pgfpathmoveto{\pgfpoint{119.87648263888889pt}{175.17915017361108pt}}
\pgflineto{\pgfpoint{117.49026527777778pt}{176.91283506944444pt}}
\pgflineto{\pgfpoint{116.57880694444444pt}{176.2506223090278pt}}
\pgfpathclose
\pgfusepath{fill,stroke}
\pgfpathmoveto{\pgfpoint{119.87648263888889pt}{175.17915017361108pt}}
\pgflineto{\pgfpoint{118.61688550347222pt}{176.91283506944444pt}}
\pgflineto{\pgfpoint{117.49026527777778pt}{176.91283506944444pt}}
\pgfpathclose
\pgfusepath{fill,stroke}
\pgfpathmoveto{\pgfpoint{119.87648263888889pt}{175.17915017361108pt}}
\pgflineto{\pgfpoint{119.52834383680555pt}{176.2506223090278pt}}
\pgflineto{\pgfpoint{118.61688550347222pt}{176.91283506944444pt}}
\pgfpathclose
\pgfusepath{fill,stroke}
\pgfpathmoveto{\pgfpoint{145.00007387152777pt}{131.59923315972222pt}}
\pgflineto{\pgfpoint{143.74046701388886pt}{129.86553914930556pt}}
\pgflineto{\pgfpoint{144.6519253472222pt}{130.5277519097222pt}}
\pgfpathclose
\pgfusepath{fill,stroke}
\pgfpathmoveto{\pgfpoint{145.00007387152777pt}{131.59923315972222pt}}
\pgflineto{\pgfpoint{142.6138376736111pt}{129.86553914930556pt}}
\pgflineto{\pgfpoint{143.74046701388886pt}{129.86553914930556pt}}
\pgfpathclose
\pgfusepath{fill,stroke}
\pgfpathmoveto{\pgfpoint{145.00007387152777pt}{131.59923315972222pt}}
\pgflineto{\pgfpoint{141.70237934027776pt}{130.5277519097222pt}}
\pgflineto{\pgfpoint{142.6138376736111pt}{129.86553914930556pt}}
\pgfpathclose
\pgfusepath{fill,stroke}
\pgfpathmoveto{\pgfpoint{145.00007387152777pt}{131.59923315972222pt}}
\pgflineto{\pgfpoint{141.35424053819443pt}{131.59923315972222pt}}
\pgflineto{\pgfpoint{141.70237934027776pt}{130.5277519097222pt}}
\pgfpathclose
\pgfusepath{fill,stroke}
\pgfpathmoveto{\pgfpoint{145.00007387152777pt}{131.59923315972222pt}}
\pgflineto{\pgfpoint{141.70237934027776pt}{132.67071380208333pt}}
\pgflineto{\pgfpoint{141.35424053819443pt}{131.59923315972222pt}}
\pgfpathclose
\pgfusepath{fill,stroke}
\pgfpathmoveto{\pgfpoint{145.00007387152777pt}{131.59923315972222pt}}
\pgflineto{\pgfpoint{142.6138376736111pt}{133.33292656249998pt}}
\pgflineto{\pgfpoint{141.70237934027776pt}{132.67071380208333pt}}
\pgfpathclose
\pgfusepath{fill,stroke}
\pgfpathmoveto{\pgfpoint{145.00007387152777pt}{131.59923315972222pt}}
\pgflineto{\pgfpoint{143.74046701388886pt}{133.33292656249998pt}}
\pgflineto{\pgfpoint{142.6138376736111pt}{133.33292656249998pt}}
\pgfpathclose
\pgfusepath{fill,stroke}
\pgfpathmoveto{\pgfpoint{145.00007387152777pt}{131.59923315972222pt}}
\pgflineto{\pgfpoint{144.6519253472222pt}{132.67071380208333pt}}
\pgflineto{\pgfpoint{143.74046701388886pt}{133.33292656249998pt}}
\pgfpathclose
\pgfusepath{fill,stroke}
\pgfpathmoveto{\pgfpoint{136.12622430555552pt}{136.9947353298611pt}}
\pgflineto{\pgfpoint{134.86662717013888pt}{135.26104131944444pt}}
\pgflineto{\pgfpoint{135.77807638888888pt}{135.9232546875pt}}
\pgfpathclose
\pgfusepath{fill,stroke}
\pgfpathmoveto{\pgfpoint{136.12622430555552pt}{136.9947353298611pt}}
\pgflineto{\pgfpoint{133.73998810763888pt}{135.26104131944444pt}}
\pgflineto{\pgfpoint{134.86662717013888pt}{135.26104131944444pt}}
\pgfpathclose
\pgfusepath{fill,stroke}
\pgfpathmoveto{\pgfpoint{136.12622430555552pt}{136.9947353298611pt}}
\pgflineto{\pgfpoint{132.8285388888889pt}{135.9232546875pt}}
\pgflineto{\pgfpoint{133.73998810763888pt}{135.26104131944444pt}}
\pgfpathclose
\pgfusepath{fill,stroke}
\pgfpathmoveto{\pgfpoint{136.12622430555552pt}{136.9947353298611pt}}
\pgflineto{\pgfpoint{132.4803909722222pt}{136.9947353298611pt}}
\pgflineto{\pgfpoint{132.8285388888889pt}{135.9232546875pt}}
\pgfpathclose
\pgfusepath{fill,stroke}
\pgfpathmoveto{\pgfpoint{136.12622430555552pt}{136.9947353298611pt}}
\pgflineto{\pgfpoint{132.8285388888889pt}{138.0662165798611pt}}
\pgflineto{\pgfpoint{132.4803909722222pt}{136.9947353298611pt}}
\pgfpathclose
\pgfusepath{fill,stroke}
\pgfpathmoveto{\pgfpoint{136.12622430555552pt}{136.9947353298611pt}}
\pgflineto{\pgfpoint{133.73998810763888pt}{138.72842934027776pt}}
\pgflineto{\pgfpoint{132.8285388888889pt}{138.0662165798611pt}}
\pgfpathclose
\pgfusepath{fill,stroke}
\pgfpathmoveto{\pgfpoint{136.12622430555552pt}{136.9947353298611pt}}
\pgflineto{\pgfpoint{134.86662717013888pt}{138.72842934027776pt}}
\pgflineto{\pgfpoint{133.73998810763888pt}{138.72842934027776pt}}
\pgfpathclose
\pgfusepath{fill,stroke}
\pgfpathmoveto{\pgfpoint{136.12622430555552pt}{136.9947353298611pt}}
\pgflineto{\pgfpoint{135.77807638888888pt}{138.0662165798611pt}}
\pgflineto{\pgfpoint{134.86662717013888pt}{138.72842934027776pt}}
\pgfpathclose
\pgfusepath{fill,stroke}
\pgfpathmoveto{\pgfpoint{138.9127256076389pt}{125.45493697916666pt}}
\pgflineto{\pgfpoint{137.65311874999998pt}{123.72124357638887pt}}
\pgflineto{\pgfpoint{138.56457708333332pt}{124.38345633680555pt}}
\pgfpathclose
\pgfusepath{fill,stroke}
\pgfpathmoveto{\pgfpoint{138.9127256076389pt}{125.45493697916666pt}}
\pgflineto{\pgfpoint{136.52648940972222pt}{123.72124357638887pt}}
\pgflineto{\pgfpoint{137.65311874999998pt}{123.72124357638887pt}}
\pgfpathclose
\pgfusepath{fill,stroke}
\pgfpathmoveto{\pgfpoint{138.9127256076389pt}{125.45493697916666pt}}
\pgflineto{\pgfpoint{135.61503107638887pt}{124.38345633680555pt}}
\pgflineto{\pgfpoint{136.52648940972222pt}{123.72124357638887pt}}
\pgfpathclose
\pgfusepath{fill,stroke}
\pgfpathmoveto{\pgfpoint{138.9127256076389pt}{125.45493697916666pt}}
\pgflineto{\pgfpoint{135.26689227430555pt}{125.45493697916666pt}}
\pgflineto{\pgfpoint{135.61503107638887pt}{124.38345633680555pt}}
\pgfpathclose
\pgfusepath{fill,stroke}
\pgfpathmoveto{\pgfpoint{138.9127256076389pt}{125.45493697916666pt}}
\pgflineto{\pgfpoint{135.61503107638887pt}{126.52641822916667pt}}
\pgflineto{\pgfpoint{135.26689227430555pt}{125.45493697916666pt}}
\pgfpathclose
\pgfusepath{fill,stroke}
\pgfpathmoveto{\pgfpoint{138.9127256076389pt}{125.45493697916666pt}}
\pgflineto{\pgfpoint{136.52648940972222pt}{127.18863098958332pt}}
\pgflineto{\pgfpoint{135.61503107638887pt}{126.52641822916667pt}}
\pgfpathclose
\pgfusepath{fill,stroke}
\pgfpathmoveto{\pgfpoint{138.9127256076389pt}{125.45493697916666pt}}
\pgflineto{\pgfpoint{137.65311874999998pt}{127.18863098958332pt}}
\pgflineto{\pgfpoint{136.52648940972222pt}{127.18863098958332pt}}
\pgfpathclose
\pgfusepath{fill,stroke}
\pgfpathmoveto{\pgfpoint{138.9127256076389pt}{125.45493697916666pt}}
\pgflineto{\pgfpoint{138.56457708333332pt}{126.52641822916667pt}}
\pgflineto{\pgfpoint{137.65311874999998pt}{127.18863098958332pt}}
\pgfpathclose
\pgfusepath{fill,stroke}
\pgfpathmoveto{\pgfpoint{81.40105251736111pt}{150.13477803819444pt}}
\pgflineto{\pgfpoint{80.1414462673611pt}{148.40107491319444pt}}
\pgflineto{\pgfpoint{81.05290460069445pt}{149.06328767361111pt}}
\pgfpathclose
\pgfusepath{fill,stroke}
\pgfpathmoveto{\pgfpoint{81.40105251736111pt}{150.13477803819444pt}}
\pgflineto{\pgfpoint{79.01481631944444pt}{148.40107491319444pt}}
\pgflineto{\pgfpoint{80.1414462673611pt}{148.40107491319444pt}}
\pgfpathclose
\pgfusepath{fill,stroke}
\pgfpathmoveto{\pgfpoint{81.40105251736111pt}{150.13477803819444pt}}
\pgflineto{\pgfpoint{78.1033579861111pt}{149.06328767361111pt}}
\pgflineto{\pgfpoint{79.01481631944444pt}{148.40107491319444pt}}
\pgfpathclose
\pgfusepath{fill,stroke}
\pgfpathmoveto{\pgfpoint{81.40105251736111pt}{150.13477803819444pt}}
\pgflineto{\pgfpoint{77.75521006944444pt}{150.13477803819444pt}}
\pgflineto{\pgfpoint{78.1033579861111pt}{149.06328767361111pt}}
\pgfpathclose
\pgfusepath{fill,stroke}
\pgfpathmoveto{\pgfpoint{81.40105251736111pt}{150.13477803819444pt}}
\pgflineto{\pgfpoint{78.1033579861111pt}{151.20625868055555pt}}
\pgflineto{\pgfpoint{77.75521006944444pt}{150.13477803819444pt}}
\pgfpathclose
\pgfusepath{fill,stroke}
\pgfpathmoveto{\pgfpoint{81.40105251736111pt}{150.13477803819444pt}}
\pgflineto{\pgfpoint{79.01481631944444pt}{151.8684720486111pt}}
\pgflineto{\pgfpoint{78.1033579861111pt}{151.20625868055555pt}}
\pgfpathclose
\pgfusepath{fill,stroke}
\pgfpathmoveto{\pgfpoint{81.40105251736111pt}{150.13477803819444pt}}
\pgflineto{\pgfpoint{80.1414462673611pt}{151.8684720486111pt}}
\pgflineto{\pgfpoint{79.01481631944444pt}{151.8684720486111pt}}
\pgfpathclose
\pgfusepath{fill,stroke}
\pgfpathmoveto{\pgfpoint{81.40105251736111pt}{150.13477803819444pt}}
\pgflineto{\pgfpoint{81.05290460069445pt}{151.20625868055555pt}}
\pgflineto{\pgfpoint{80.1414462673611pt}{151.8684720486111pt}}
\pgfpathclose
\pgfusepath{fill,stroke}
\pgfpathmoveto{\pgfpoint{109.18915546874999pt}{142.56567013888886pt}}
\pgflineto{\pgfpoint{107.92955833333333pt}{140.83196701388889pt}}
\pgflineto{\pgfpoint{108.84101666666666pt}{141.49418038194443pt}}
\pgfpathclose
\pgfusepath{fill,stroke}
\pgfpathmoveto{\pgfpoint{109.18915546874999pt}{142.56567013888886pt}}
\pgflineto{\pgfpoint{106.8029375pt}{140.83196701388889pt}}
\pgflineto{\pgfpoint{107.92955833333333pt}{140.83196701388889pt}}
\pgfpathclose
\pgfusepath{fill,stroke}
\pgfpathmoveto{\pgfpoint{109.18915546874999pt}{142.56567013888886pt}}
\pgflineto{\pgfpoint{105.89147916666666pt}{141.49418038194443pt}}
\pgflineto{\pgfpoint{106.8029375pt}{140.83196701388889pt}}
\pgfpathclose
\pgfusepath{fill,stroke}
\pgfpathmoveto{\pgfpoint{109.18915546874999pt}{142.56567013888886pt}}
\pgflineto{\pgfpoint{105.54333124999998pt}{142.56567013888886pt}}
\pgflineto{\pgfpoint{105.89147916666666pt}{141.49418038194443pt}}
\pgfpathclose
\pgfusepath{fill,stroke}
\pgfpathmoveto{\pgfpoint{109.18915546874999pt}{142.56567013888886pt}}
\pgflineto{\pgfpoint{105.89147916666666pt}{143.6371513888889pt}}
\pgflineto{\pgfpoint{105.54333124999998pt}{142.56567013888886pt}}
\pgfpathclose
\pgfusepath{fill,stroke}
\pgfpathmoveto{\pgfpoint{109.18915546874999pt}{142.56567013888886pt}}
\pgflineto{\pgfpoint{106.8029375pt}{144.29936414930555pt}}
\pgflineto{\pgfpoint{105.89147916666666pt}{143.6371513888889pt}}
\pgfpathclose
\pgfusepath{fill,stroke}
\pgfpathmoveto{\pgfpoint{109.18915546874999pt}{142.56567013888886pt}}
\pgflineto{\pgfpoint{107.92955833333333pt}{144.29936414930555pt}}
\pgflineto{\pgfpoint{106.8029375pt}{144.29936414930555pt}}
\pgfpathclose
\pgfusepath{fill,stroke}
\pgfpathmoveto{\pgfpoint{109.18915546874999pt}{142.56567013888886pt}}
\pgflineto{\pgfpoint{108.84101666666666pt}{143.6371513888889pt}}
\pgflineto{\pgfpoint{107.92955833333333pt}{144.29936414930555pt}}
\pgfpathclose
\pgfusepath{fill,stroke}
\pgfpathmoveto{\pgfpoint{124.11059548611111pt}{132.72453663194443pt}}
\pgflineto{\pgfpoint{122.8509892361111pt}{130.99084322916664pt}}
\pgflineto{\pgfpoint{123.76244756944443pt}{131.65305598958332pt}}
\pgfpathclose
\pgfusepath{fill,stroke}
\pgfpathmoveto{\pgfpoint{124.11059548611111pt}{132.72453663194443pt}}
\pgflineto{\pgfpoint{121.72436840277777pt}{130.99084322916664pt}}
\pgflineto{\pgfpoint{122.8509892361111pt}{130.99084322916664pt}}
\pgfpathclose
\pgfusepath{fill,stroke}
\pgfpathmoveto{\pgfpoint{124.11059548611111pt}{132.72453663194443pt}}
\pgflineto{\pgfpoint{120.81291006944444pt}{131.65305598958332pt}}
\pgflineto{\pgfpoint{121.72436840277777pt}{130.99084322916664pt}}
\pgfpathclose
\pgfusepath{fill,stroke}
\pgfpathmoveto{\pgfpoint{124.11059548611111pt}{132.72453663194443pt}}
\pgflineto{\pgfpoint{120.46476215277778pt}{132.72453663194443pt}}
\pgflineto{\pgfpoint{120.81291006944444pt}{131.65305598958332pt}}
\pgfpathclose
\pgfusepath{fill,stroke}
\pgfpathmoveto{\pgfpoint{124.11059548611111pt}{132.72453663194443pt}}
\pgflineto{\pgfpoint{120.81291006944444pt}{133.79602699652776pt}}
\pgflineto{\pgfpoint{120.46476215277778pt}{132.72453663194443pt}}
\pgfpathclose
\pgfusepath{fill,stroke}
\pgfpathmoveto{\pgfpoint{124.11059548611111pt}{132.72453663194443pt}}
\pgflineto{\pgfpoint{121.72436840277777pt}{134.45823975694444pt}}
\pgflineto{\pgfpoint{120.81291006944444pt}{133.79602699652776pt}}
\pgfpathclose
\pgfusepath{fill,stroke}
\pgfpathmoveto{\pgfpoint{124.11059548611111pt}{132.72453663194443pt}}
\pgflineto{\pgfpoint{122.8509892361111pt}{134.45823975694444pt}}
\pgflineto{\pgfpoint{121.72436840277777pt}{134.45823975694444pt}}
\pgfpathclose
\pgfusepath{fill,stroke}
\pgfpathmoveto{\pgfpoint{124.11059548611111pt}{132.72453663194443pt}}
\pgflineto{\pgfpoint{123.76244756944443pt}{133.79602699652776pt}}
\pgflineto{\pgfpoint{122.8509892361111pt}{134.45823975694444pt}}
\pgfpathclose
\pgfusepath{fill,stroke}
\pgfpathmoveto{\pgfpoint{141.0841237847222pt}{127.95373342013887pt}}
\pgflineto{\pgfpoint{139.82451753472222pt}{126.22003940972222pt}}
\pgflineto{\pgfpoint{140.73597586805556pt}{126.88225217013888pt}}
\pgfpathclose
\pgfusepath{fill,stroke}
\pgfpathmoveto{\pgfpoint{141.0841237847222pt}{127.95373342013887pt}}
\pgflineto{\pgfpoint{138.69789730902778pt}{126.22003940972222pt}}
\pgflineto{\pgfpoint{139.82451753472222pt}{126.22003940972222pt}}
\pgfpathclose
\pgfusepath{fill,stroke}
\pgfpathmoveto{\pgfpoint{141.0841237847222pt}{127.95373342013887pt}}
\pgflineto{\pgfpoint{137.78643897569444pt}{126.88225217013888pt}}
\pgflineto{\pgfpoint{138.69789730902778pt}{126.22003940972222pt}}
\pgfpathclose
\pgfusepath{fill,stroke}
\pgfpathmoveto{\pgfpoint{141.0841237847222pt}{127.95373342013887pt}}
\pgflineto{\pgfpoint{137.43829045138887pt}{127.95373342013887pt}}
\pgflineto{\pgfpoint{137.78643897569444pt}{126.88225217013888pt}}
\pgfpathclose
\pgfusepath{fill,stroke}
\pgfpathmoveto{\pgfpoint{141.0841237847222pt}{127.95373342013887pt}}
\pgflineto{\pgfpoint{137.78643897569444pt}{129.0252140625pt}}
\pgflineto{\pgfpoint{137.43829045138887pt}{127.95373342013887pt}}
\pgfpathclose
\pgfusepath{fill,stroke}
\pgfpathmoveto{\pgfpoint{141.0841237847222pt}{127.95373342013887pt}}
\pgflineto{\pgfpoint{138.69789730902778pt}{129.68742743055554pt}}
\pgflineto{\pgfpoint{137.78643897569444pt}{129.0252140625pt}}
\pgfpathclose
\pgfusepath{fill,stroke}
\pgfpathmoveto{\pgfpoint{141.0841237847222pt}{127.95373342013887pt}}
\pgflineto{\pgfpoint{139.82451753472222pt}{129.68742743055554pt}}
\pgflineto{\pgfpoint{138.69789730902778pt}{129.68742743055554pt}}
\pgfpathclose
\pgfusepath{fill,stroke}
\pgfpathmoveto{\pgfpoint{141.0841237847222pt}{127.95373342013887pt}}
\pgflineto{\pgfpoint{140.73597586805556pt}{129.0252140625pt}}
\pgflineto{\pgfpoint{139.82451753472222pt}{129.68742743055554pt}}
\pgfpathclose
\pgfusepath{fill,stroke}
\pgfpathmoveto{\pgfpoint{131.2345993923611pt}{132.97699053819443pt}}
\pgflineto{\pgfpoint{129.9749931423611pt}{131.24329652777777pt}}
\pgflineto{\pgfpoint{130.88645147569443pt}{131.90550928819442pt}}
\pgfpathclose
\pgfusepath{fill,stroke}
\pgfpathmoveto{\pgfpoint{131.2345993923611pt}{132.97699053819443pt}}
\pgflineto{\pgfpoint{128.84837230902778pt}{131.24329652777777pt}}
\pgflineto{\pgfpoint{129.9749931423611pt}{131.24329652777777pt}}
\pgfpathclose
\pgfusepath{fill,stroke}
\pgfpathmoveto{\pgfpoint{131.2345993923611pt}{132.97699053819443pt}}
\pgflineto{\pgfpoint{127.93691397569444pt}{131.90550928819442pt}}
\pgflineto{\pgfpoint{128.84837230902778pt}{131.24329652777777pt}}
\pgfpathclose
\pgfusepath{fill,stroke}
\pgfpathmoveto{\pgfpoint{131.2345993923611pt}{132.97699053819443pt}}
\pgflineto{\pgfpoint{127.5887751736111pt}{132.97699053819443pt}}
\pgflineto{\pgfpoint{127.93691397569444pt}{131.90550928819442pt}}
\pgfpathclose
\pgfusepath{fill,stroke}
\pgfpathmoveto{\pgfpoint{131.2345993923611pt}{132.97699053819443pt}}
\pgflineto{\pgfpoint{127.93691397569444pt}{134.04847118055554pt}}
\pgflineto{\pgfpoint{127.5887751736111pt}{132.97699053819443pt}}
\pgfpathclose
\pgfusepath{fill,stroke}
\pgfpathmoveto{\pgfpoint{131.2345993923611pt}{132.97699053819443pt}}
\pgflineto{\pgfpoint{128.84837230902778pt}{134.71068394097222pt}}
\pgflineto{\pgfpoint{127.93691397569444pt}{134.04847118055554pt}}
\pgfpathclose
\pgfusepath{fill,stroke}
\pgfpathmoveto{\pgfpoint{131.2345993923611pt}{132.97699053819443pt}}
\pgflineto{\pgfpoint{129.9749931423611pt}{134.71068394097222pt}}
\pgflineto{\pgfpoint{128.84837230902778pt}{134.71068394097222pt}}
\pgfpathclose
\pgfusepath{fill,stroke}
\pgfpathmoveto{\pgfpoint{131.2345993923611pt}{132.97699053819443pt}}
\pgflineto{\pgfpoint{130.88645147569443pt}{134.04847118055554pt}}
\pgflineto{\pgfpoint{129.9749931423611pt}{134.71068394097222pt}}
\pgfpathclose
\pgfusepath{fill,stroke}
\pgfpathmoveto{\pgfpoint{83.60956649305554pt}{194.4596317708333pt}}
\pgflineto{\pgfpoint{82.34996024305555pt}{192.7259286458333pt}}
\pgflineto{\pgfpoint{83.26141857638888pt}{193.38814140624996pt}}
\pgfpathclose
\pgfusepath{fill,stroke}
\pgfpathmoveto{\pgfpoint{83.60956649305554pt}{194.4596317708333pt}}
\pgflineto{\pgfpoint{81.2233302951389pt}{192.7259286458333pt}}
\pgflineto{\pgfpoint{82.34996024305555pt}{192.7259286458333pt}}
\pgfpathclose
\pgfusepath{fill,stroke}
\pgfpathmoveto{\pgfpoint{83.60956649305554pt}{194.4596317708333pt}}
\pgflineto{\pgfpoint{80.31188107638889pt}{193.38814140624996pt}}
\pgflineto{\pgfpoint{81.2233302951389pt}{192.7259286458333pt}}
\pgfpathclose
\pgfusepath{fill,stroke}
\pgfpathmoveto{\pgfpoint{83.60956649305554pt}{194.4596317708333pt}}
\pgflineto{\pgfpoint{79.96372404513887pt}{194.4596317708333pt}}
\pgflineto{\pgfpoint{80.31188107638889pt}{193.38814140624996pt}}
\pgfpathclose
\pgfusepath{fill,stroke}
\pgfpathmoveto{\pgfpoint{83.60956649305554pt}{194.4596317708333pt}}
\pgflineto{\pgfpoint{80.31188107638889pt}{195.53110329861107pt}}
\pgflineto{\pgfpoint{79.96372404513887pt}{194.4596317708333pt}}
\pgfpathclose
\pgfusepath{fill,stroke}
\pgfpathmoveto{\pgfpoint{83.60956649305554pt}{194.4596317708333pt}}
\pgflineto{\pgfpoint{81.2233302951389pt}{196.19333489583332pt}}
\pgflineto{\pgfpoint{80.31188107638889pt}{195.53110329861107pt}}
\pgfpathclose
\pgfusepath{fill,stroke}
\pgfpathmoveto{\pgfpoint{83.60956649305554pt}{194.4596317708333pt}}
\pgflineto{\pgfpoint{82.34996024305555pt}{196.19333489583332pt}}
\pgflineto{\pgfpoint{81.2233302951389pt}{196.19333489583332pt}}
\pgfpathclose
\pgfusepath{fill,stroke}
\pgfpathmoveto{\pgfpoint{83.60956649305554pt}{194.4596317708333pt}}
\pgflineto{\pgfpoint{83.26141857638888pt}{195.53110329861107pt}}
\pgflineto{\pgfpoint{82.34996024305555pt}{196.19333489583332pt}}
\pgfpathclose
\pgfusepath{fill,stroke}
\pgfpathmoveto{\pgfpoint{129.5801986111111pt}{145.17142621527776pt}}
\pgflineto{\pgfpoint{128.32060147569445pt}{143.43773281249997pt}}
\pgflineto{\pgfpoint{129.23205980902776pt}{144.09994557291665pt}}
\pgfpathclose
\pgfusepath{fill,stroke}
\pgfpathmoveto{\pgfpoint{129.5801986111111pt}{145.17142621527776pt}}
\pgflineto{\pgfpoint{127.19397152777778pt}{143.43773281249997pt}}
\pgflineto{\pgfpoint{128.32060147569445pt}{143.43773281249997pt}}
\pgfpathclose
\pgfusepath{fill,stroke}
\pgfpathmoveto{\pgfpoint{129.5801986111111pt}{145.17142621527776pt}}
\pgflineto{\pgfpoint{126.28251319444443pt}{144.09994557291665pt}}
\pgflineto{\pgfpoint{127.19397152777778pt}{143.43773281249997pt}}
\pgfpathclose
\pgfusepath{fill,stroke}
\pgfpathmoveto{\pgfpoint{129.5801986111111pt}{145.17142621527776pt}}
\pgflineto{\pgfpoint{125.93436527777776pt}{145.17142621527776pt}}
\pgflineto{\pgfpoint{126.28251319444443pt}{144.09994557291665pt}}
\pgfpathclose
\pgfusepath{fill,stroke}
\pgfpathmoveto{\pgfpoint{129.5801986111111pt}{145.17142621527776pt}}
\pgflineto{\pgfpoint{126.28251319444443pt}{146.24290746527777pt}}
\pgflineto{\pgfpoint{125.93436527777776pt}{145.17142621527776pt}}
\pgfpathclose
\pgfusepath{fill,stroke}
\pgfpathmoveto{\pgfpoint{129.5801986111111pt}{145.17142621527776pt}}
\pgflineto{\pgfpoint{127.19397152777778pt}{146.90512022569445pt}}
\pgflineto{\pgfpoint{126.28251319444443pt}{146.24290746527777pt}}
\pgfpathclose
\pgfusepath{fill,stroke}
\pgfpathmoveto{\pgfpoint{129.5801986111111pt}{145.17142621527776pt}}
\pgflineto{\pgfpoint{128.32060147569445pt}{146.90512022569445pt}}
\pgflineto{\pgfpoint{127.19397152777778pt}{146.90512022569445pt}}
\pgfpathclose
\pgfusepath{fill,stroke}
\pgfpathmoveto{\pgfpoint{129.5801986111111pt}{145.17142621527776pt}}
\pgflineto{\pgfpoint{129.23205980902776pt}{146.24290746527777pt}}
\pgflineto{\pgfpoint{128.32060147569445pt}{146.90512022569445pt}}
\pgfpathclose
\pgfusepath{fill,stroke}
\pgfpathmoveto{\pgfpoint{121.59982005208332pt}{155.15375130208332pt}}
\pgflineto{\pgfpoint{120.34022291666666pt}{153.42005729166667pt}}
\pgflineto{\pgfpoint{121.25168124999999pt}{154.0822706597222pt}}
\pgfpathclose
\pgfusepath{fill,stroke}
\pgfpathmoveto{\pgfpoint{121.59982005208332pt}{155.15375130208332pt}}
\pgflineto{\pgfpoint{119.21360208333333pt}{153.42005729166667pt}}
\pgflineto{\pgfpoint{120.34022291666666pt}{153.42005729166667pt}}
\pgfpathclose
\pgfusepath{fill,stroke}
\pgfpathmoveto{\pgfpoint{121.59982005208332pt}{155.15375130208332pt}}
\pgflineto{\pgfpoint{118.30214375pt}{154.0822706597222pt}}
\pgflineto{\pgfpoint{119.21360208333333pt}{153.42005729166667pt}}
\pgfpathclose
\pgfusepath{fill,stroke}
\pgfpathmoveto{\pgfpoint{121.59982005208332pt}{155.15375130208332pt}}
\pgflineto{\pgfpoint{117.95399583333332pt}{155.15375130208332pt}}
\pgflineto{\pgfpoint{118.30214375pt}{154.0822706597222pt}}
\pgfpathclose
\pgfusepath{fill,stroke}
\pgfpathmoveto{\pgfpoint{121.59982005208332pt}{155.15375130208332pt}}
\pgflineto{\pgfpoint{118.30214375pt}{156.22524166666665pt}}
\pgflineto{\pgfpoint{117.95399583333332pt}{155.15375130208332pt}}
\pgfpathclose
\pgfusepath{fill,stroke}
\pgfpathmoveto{\pgfpoint{121.59982005208332pt}{155.15375130208332pt}}
\pgflineto{\pgfpoint{119.21360208333333pt}{156.88745442708333pt}}
\pgflineto{\pgfpoint{118.30214375pt}{156.22524166666665pt}}
\pgfpathclose
\pgfusepath{fill,stroke}
\pgfpathmoveto{\pgfpoint{121.59982005208332pt}{155.15375130208332pt}}
\pgflineto{\pgfpoint{120.34022291666666pt}{156.88745442708333pt}}
\pgflineto{\pgfpoint{119.21360208333333pt}{156.88745442708333pt}}
\pgfpathclose
\pgfusepath{fill,stroke}
\pgfpathmoveto{\pgfpoint{121.59982005208332pt}{155.15375130208332pt}}
\pgflineto{\pgfpoint{121.25168124999999pt}{156.22524166666665pt}}
\pgflineto{\pgfpoint{120.34022291666666pt}{156.88745442708333pt}}
\pgfpathclose
\pgfusepath{fill,stroke}
\pgfpathmoveto{\pgfpoint{141.45530303819444pt}{132.3822390625pt}}
\pgflineto{\pgfpoint{140.19570590277775pt}{130.6485456597222pt}}
\pgflineto{\pgfpoint{141.1071642361111pt}{131.31075842013888pt}}
\pgfpathclose
\pgfusepath{fill,stroke}
\pgfpathmoveto{\pgfpoint{141.45530303819444pt}{132.3822390625pt}}
\pgflineto{\pgfpoint{139.0690765625pt}{130.6485456597222pt}}
\pgflineto{\pgfpoint{140.19570590277775pt}{130.6485456597222pt}}
\pgfpathclose
\pgfusepath{fill,stroke}
\pgfpathmoveto{\pgfpoint{141.45530303819444pt}{132.3822390625pt}}
\pgflineto{\pgfpoint{138.15761822916664pt}{131.31075842013888pt}}
\pgflineto{\pgfpoint{139.0690765625pt}{130.6485456597222pt}}
\pgfpathclose
\pgfusepath{fill,stroke}
\pgfpathmoveto{\pgfpoint{141.45530303819444pt}{132.3822390625pt}}
\pgflineto{\pgfpoint{137.8094697048611pt}{132.3822390625pt}}
\pgflineto{\pgfpoint{138.15761822916664pt}{131.31075842013888pt}}
\pgfpathclose
\pgfusepath{fill,stroke}
\pgfpathmoveto{\pgfpoint{141.45530303819444pt}{132.3822390625pt}}
\pgflineto{\pgfpoint{138.15761822916664pt}{133.45372031249997pt}}
\pgflineto{\pgfpoint{137.8094697048611pt}{132.3822390625pt}}
\pgfpathclose
\pgfusepath{fill,stroke}
\pgfpathmoveto{\pgfpoint{141.45530303819444pt}{132.3822390625pt}}
\pgflineto{\pgfpoint{139.0690765625pt}{134.11593307291665pt}}
\pgflineto{\pgfpoint{138.15761822916664pt}{133.45372031249997pt}}
\pgfpathclose
\pgfusepath{fill,stroke}
\pgfpathmoveto{\pgfpoint{141.45530303819444pt}{132.3822390625pt}}
\pgflineto{\pgfpoint{140.19570590277775pt}{134.11593307291665pt}}
\pgflineto{\pgfpoint{139.0690765625pt}{134.11593307291665pt}}
\pgfpathclose
\pgfusepath{fill,stroke}
\pgfpathmoveto{\pgfpoint{141.45530303819444pt}{132.3822390625pt}}
\pgflineto{\pgfpoint{141.1071642361111pt}{133.45372031249997pt}}
\pgflineto{\pgfpoint{140.19570590277775pt}{134.11593307291665pt}}
\pgfpathclose
\pgfusepath{fill,stroke}
\pgfpathmoveto{\pgfpoint{108.19228498263887pt}{166.8522837673611pt}}
\pgflineto{\pgfpoint{106.93268784722221pt}{165.11859887152778pt}}
\pgflineto{\pgfpoint{107.84414618055554pt}{165.78079340277776pt}}
\pgfpathclose
\pgfusepath{fill,stroke}
\pgfpathmoveto{\pgfpoint{108.19228498263887pt}{166.8522837673611pt}}
\pgflineto{\pgfpoint{105.80606701388888pt}{165.11859887152778pt}}
\pgflineto{\pgfpoint{106.93268784722221pt}{165.11859887152778pt}}
\pgfpathclose
\pgfusepath{fill,stroke}
\pgfpathmoveto{\pgfpoint{108.19228498263887pt}{166.8522837673611pt}}
\pgflineto{\pgfpoint{104.89459956597221pt}{165.78079340277776pt}}
\pgflineto{\pgfpoint{105.80606701388888pt}{165.11859887152778pt}}
\pgfpathclose
\pgfusepath{fill,stroke}
\pgfpathmoveto{\pgfpoint{108.19228498263887pt}{166.8522837673611pt}}
\pgflineto{\pgfpoint{104.54645164930554pt}{166.8522837673611pt}}
\pgflineto{\pgfpoint{104.89459956597221pt}{165.78079340277776pt}}
\pgfpathclose
\pgfusepath{fill,stroke}
\pgfpathmoveto{\pgfpoint{108.19228498263887pt}{166.8522837673611pt}}
\pgflineto{\pgfpoint{104.89459956597221pt}{167.92377352430555pt}}
\pgflineto{\pgfpoint{104.54645164930554pt}{166.8522837673611pt}}
\pgfpathclose
\pgfusepath{fill,stroke}
\pgfpathmoveto{\pgfpoint{108.19228498263887pt}{166.8522837673611pt}}
\pgflineto{\pgfpoint{105.80606701388888pt}{168.58596805555555pt}}
\pgflineto{\pgfpoint{104.89459956597221pt}{167.92377352430555pt}}
\pgfpathclose
\pgfusepath{fill,stroke}
\pgfpathmoveto{\pgfpoint{108.19228498263887pt}{166.8522837673611pt}}
\pgflineto{\pgfpoint{106.93268784722221pt}{168.58596805555555pt}}
\pgflineto{\pgfpoint{105.80606701388888pt}{168.58596805555555pt}}
\pgfpathclose
\pgfusepath{fill,stroke}
\pgfpathmoveto{\pgfpoint{108.19228498263887pt}{166.8522837673611pt}}
\pgflineto{\pgfpoint{107.84414618055554pt}{167.92377352430555pt}}
\pgflineto{\pgfpoint{106.93268784722221pt}{168.58596805555555pt}}
\pgfpathclose
\pgfusepath{fill,stroke}
\pgfpathmoveto{\pgfpoint{139.13013151041665pt}{137.75634262152778pt}}
\pgflineto{\pgfpoint{137.87052526041666pt}{136.0226486111111pt}}
\pgflineto{\pgfpoint{138.78198359374997pt}{136.68486137152777pt}}
\pgfpathclose
\pgfusepath{fill,stroke}
\pgfpathmoveto{\pgfpoint{139.13013151041665pt}{137.75634262152778pt}}
\pgflineto{\pgfpoint{136.74390442708332pt}{136.0226486111111pt}}
\pgflineto{\pgfpoint{137.87052526041666pt}{136.0226486111111pt}}
\pgfpathclose
\pgfusepath{fill,stroke}
\pgfpathmoveto{\pgfpoint{139.13013151041665pt}{137.75634262152778pt}}
\pgflineto{\pgfpoint{135.83244609374998pt}{136.68486137152777pt}}
\pgflineto{\pgfpoint{136.74390442708332pt}{136.0226486111111pt}}
\pgfpathclose
\pgfusepath{fill,stroke}
\pgfpathmoveto{\pgfpoint{139.13013151041665pt}{137.75634262152778pt}}
\pgflineto{\pgfpoint{135.48429817708333pt}{137.75634262152778pt}}
\pgflineto{\pgfpoint{135.83244609374998pt}{136.68486137152777pt}}
\pgfpathclose
\pgfusepath{fill,stroke}
\pgfpathmoveto{\pgfpoint{139.13013151041665pt}{137.75634262152778pt}}
\pgflineto{\pgfpoint{135.83244609374998pt}{138.8278323784722pt}}
\pgflineto{\pgfpoint{135.48429817708333pt}{137.75634262152778pt}}
\pgfpathclose
\pgfusepath{fill,stroke}
\pgfpathmoveto{\pgfpoint{139.13013151041665pt}{137.75634262152778pt}}
\pgflineto{\pgfpoint{136.74390442708332pt}{139.49004513888886pt}}
\pgflineto{\pgfpoint{135.83244609374998pt}{138.8278323784722pt}}
\pgfpathclose
\pgfusepath{fill,stroke}
\pgfpathmoveto{\pgfpoint{139.13013151041665pt}{137.75634262152778pt}}
\pgflineto{\pgfpoint{137.87052526041666pt}{139.49004513888886pt}}
\pgflineto{\pgfpoint{136.74390442708332pt}{139.49004513888886pt}}
\pgfpathclose
\pgfusepath{fill,stroke}
\pgfpathmoveto{\pgfpoint{139.13013151041665pt}{137.75634262152778pt}}
\pgflineto{\pgfpoint{138.78198359374997pt}{138.8278323784722pt}}
\pgflineto{\pgfpoint{137.87052526041666pt}{139.49004513888886pt}}
\pgfpathclose
\pgfusepath{fill,stroke}
\pgfpathmoveto{\pgfpoint{110.71364852430555pt}{124.03013437499999pt}}
\pgflineto{\pgfpoint{109.45403315972221pt}{122.29643124999998pt}}
\pgflineto{\pgfpoint{110.36550060763888pt}{122.95864401041665pt}}
\pgfpathclose
\pgfusepath{fill,stroke}
\pgfpathmoveto{\pgfpoint{110.71364852430555pt}{124.03013437499999pt}}
\pgflineto{\pgfpoint{108.32741232638888pt}{122.29643124999998pt}}
\pgflineto{\pgfpoint{109.45403315972221pt}{122.29643124999998pt}}
\pgfpathclose
\pgfusepath{fill,stroke}
\pgfpathmoveto{\pgfpoint{110.71364852430555pt}{124.03013437499999pt}}
\pgflineto{\pgfpoint{107.41595399305555pt}{122.95864401041665pt}}
\pgflineto{\pgfpoint{108.32741232638888pt}{122.29643124999998pt}}
\pgfpathclose
\pgfusepath{fill,stroke}
\pgfpathmoveto{\pgfpoint{110.71364852430555pt}{124.03013437499999pt}}
\pgflineto{\pgfpoint{107.06781519097221pt}{124.03013437499999pt}}
\pgflineto{\pgfpoint{107.41595399305555pt}{122.95864401041665pt}}
\pgfpathclose
\pgfusepath{fill,stroke}
\pgfpathmoveto{\pgfpoint{110.71364852430555pt}{124.03013437499999pt}}
\pgflineto{\pgfpoint{107.41595399305555pt}{125.101615625pt}}
\pgflineto{\pgfpoint{107.06781519097221pt}{124.03013437499999pt}}
\pgfpathclose
\pgfusepath{fill,stroke}
\pgfpathmoveto{\pgfpoint{110.71364852430555pt}{124.03013437499999pt}}
\pgflineto{\pgfpoint{108.32741232638888pt}{125.76382838541666pt}}
\pgflineto{\pgfpoint{107.41595399305555pt}{125.101615625pt}}
\pgfpathclose
\pgfusepath{fill,stroke}
\pgfpathmoveto{\pgfpoint{110.71364852430555pt}{124.03013437499999pt}}
\pgflineto{\pgfpoint{109.45403315972221pt}{125.76382838541666pt}}
\pgflineto{\pgfpoint{108.32741232638888pt}{125.76382838541666pt}}
\pgfpathclose
\pgfusepath{fill,stroke}
\pgfpathmoveto{\pgfpoint{110.71364852430555pt}{124.03013437499999pt}}
\pgflineto{\pgfpoint{110.36550060763888pt}{125.101615625pt}}
\pgflineto{\pgfpoint{109.45403315972221pt}{125.76382838541666pt}}
\pgfpathclose
\pgfusepath{fill,stroke}
\pgfpathmoveto{\pgfpoint{109.48874696180555pt}{163.63593993055554pt}}
\pgflineto{\pgfpoint{108.22914982638889pt}{161.90223680555553pt}}
\pgflineto{\pgfpoint{109.14060815972222pt}{162.56444956597218pt}}
\pgfpathclose
\pgfusepath{fill,stroke}
\pgfpathmoveto{\pgfpoint{109.48874696180555pt}{163.63593993055554pt}}
\pgflineto{\pgfpoint{107.10252899305556pt}{161.90223680555553pt}}
\pgflineto{\pgfpoint{108.22914982638889pt}{161.90223680555553pt}}
\pgfpathclose
\pgfusepath{fill,stroke}
\pgfpathmoveto{\pgfpoint{109.48874696180555pt}{163.63593993055554pt}}
\pgflineto{\pgfpoint{106.19107065972221pt}{162.56444956597218pt}}
\pgflineto{\pgfpoint{107.10252899305556pt}{161.90223680555553pt}}
\pgfpathclose
\pgfusepath{fill,stroke}
\pgfpathmoveto{\pgfpoint{109.48874696180555pt}{163.63593993055554pt}}
\pgflineto{\pgfpoint{105.84292274305554pt}{163.63593993055554pt}}
\pgflineto{\pgfpoint{106.19107065972221pt}{162.56444956597218pt}}
\pgfpathclose
\pgfusepath{fill,stroke}
\pgfpathmoveto{\pgfpoint{109.48874696180555pt}{163.63593993055554pt}}
\pgflineto{\pgfpoint{106.19107065972221pt}{164.7074114583333pt}}
\pgflineto{\pgfpoint{105.84292274305554pt}{163.63593993055554pt}}
\pgfpathclose
\pgfusepath{fill,stroke}
\pgfpathmoveto{\pgfpoint{109.48874696180555pt}{163.63593993055554pt}}
\pgflineto{\pgfpoint{107.10252899305556pt}{165.3696248263889pt}}
\pgflineto{\pgfpoint{106.19107065972221pt}{164.7074114583333pt}}
\pgfpathclose
\pgfusepath{fill,stroke}
\pgfpathmoveto{\pgfpoint{109.48874696180555pt}{163.63593993055554pt}}
\pgflineto{\pgfpoint{108.22914982638889pt}{165.3696248263889pt}}
\pgflineto{\pgfpoint{107.10252899305556pt}{165.3696248263889pt}}
\pgfpathclose
\pgfusepath{fill,stroke}
\pgfpathmoveto{\pgfpoint{109.48874696180555pt}{163.63593993055554pt}}
\pgflineto{\pgfpoint{109.14060815972222pt}{164.7074114583333pt}}
\pgflineto{\pgfpoint{108.22914982638889pt}{165.3696248263889pt}}
\pgfpathclose
\pgfusepath{fill,stroke}
\pgfpathmoveto{\pgfpoint{131.94779791666667pt}{171.3051745659722pt}}
\pgflineto{\pgfpoint{130.68819166666665pt}{169.5714714409722pt}}
\pgflineto{\pgfpoint{131.59965pt}{170.23368420138888pt}}
\pgfpathclose
\pgfusepath{fill,stroke}
\pgfpathmoveto{\pgfpoint{131.94779791666667pt}{171.3051745659722pt}}
\pgflineto{\pgfpoint{129.56157144097222pt}{169.5714714409722pt}}
\pgflineto{\pgfpoint{130.68819166666665pt}{169.5714714409722pt}}
\pgfpathclose
\pgfusepath{fill,stroke}
\pgfpathmoveto{\pgfpoint{131.94779791666667pt}{171.3051745659722pt}}
\pgflineto{\pgfpoint{128.65010338541666pt}{170.23368420138888pt}}
\pgflineto{\pgfpoint{129.56157144097222pt}{169.5714714409722pt}}
\pgfpathclose
\pgfusepath{fill,stroke}
\pgfpathmoveto{\pgfpoint{131.94779791666667pt}{171.3051745659722pt}}
\pgflineto{\pgfpoint{128.30196458333333pt}{171.3051745659722pt}}
\pgflineto{\pgfpoint{128.65010338541666pt}{170.23368420138888pt}}
\pgfpathclose
\pgfusepath{fill,stroke}
\pgfpathmoveto{\pgfpoint{131.94779791666667pt}{171.3051745659722pt}}
\pgflineto{\pgfpoint{128.65010338541666pt}{172.37666432291667pt}}
\pgflineto{\pgfpoint{128.30196458333333pt}{171.3051745659722pt}}
\pgfpathclose
\pgfusepath{fill,stroke}
\pgfpathmoveto{\pgfpoint{131.94779791666667pt}{171.3051745659722pt}}
\pgflineto{\pgfpoint{129.56157144097222pt}{173.03885885416665pt}}
\pgflineto{\pgfpoint{128.65010338541666pt}{172.37666432291667pt}}
\pgfpathclose
\pgfusepath{fill,stroke}
\pgfpathmoveto{\pgfpoint{131.94779791666667pt}{171.3051745659722pt}}
\pgflineto{\pgfpoint{130.68819166666665pt}{173.03885885416665pt}}
\pgflineto{\pgfpoint{129.56157144097222pt}{173.03885885416665pt}}
\pgfpathclose
\pgfusepath{fill,stroke}
\pgfpathmoveto{\pgfpoint{131.94779791666667pt}{171.3051745659722pt}}
\pgflineto{\pgfpoint{131.59965pt}{172.37666432291667pt}}
\pgflineto{\pgfpoint{130.68819166666665pt}{173.03885885416665pt}}
\pgfpathclose
\pgfusepath{fill,stroke}
\pgfpathmoveto{\pgfpoint{106.51402708333332pt}{171.96323350694442pt}}
\pgflineto{\pgfpoint{105.25441171874999pt}{170.2295486111111pt}}
\pgflineto{\pgfpoint{106.16587916666666pt}{170.89176197916666pt}}
\pgfpathclose
\pgfusepath{fill,stroke}
\pgfpathmoveto{\pgfpoint{106.51402708333332pt}{171.96323350694442pt}}
\pgflineto{\pgfpoint{104.12779149305555pt}{170.2295486111111pt}}
\pgflineto{\pgfpoint{105.25441171874999pt}{170.2295486111111pt}}
\pgfpathclose
\pgfusepath{fill,stroke}
\pgfpathmoveto{\pgfpoint{106.51402708333332pt}{171.96323350694442pt}}
\pgflineto{\pgfpoint{103.21633315972223pt}{170.89176197916666pt}}
\pgflineto{\pgfpoint{104.12779149305555pt}{170.2295486111111pt}}
\pgfpathclose
\pgfusepath{fill,stroke}
\pgfpathmoveto{\pgfpoint{106.51402708333332pt}{171.96323350694442pt}}
\pgflineto{\pgfpoint{102.86819374999999pt}{171.96323350694442pt}}
\pgflineto{\pgfpoint{103.21633315972223pt}{170.89176197916666pt}}
\pgfpathclose
\pgfusepath{fill,stroke}
\pgfpathmoveto{\pgfpoint{106.51402708333332pt}{171.96323350694442pt}}
\pgflineto{\pgfpoint{103.21633315972223pt}{173.03472387152777pt}}
\pgflineto{\pgfpoint{102.86819374999999pt}{171.96323350694442pt}}
\pgfpathclose
\pgfusepath{fill,stroke}
\pgfpathmoveto{\pgfpoint{106.51402708333332pt}{171.96323350694442pt}}
\pgflineto{\pgfpoint{104.12779149305555pt}{173.69693663194442pt}}
\pgflineto{\pgfpoint{103.21633315972223pt}{173.03472387152777pt}}
\pgfpathclose
\pgfusepath{fill,stroke}
\pgfpathmoveto{\pgfpoint{106.51402708333332pt}{171.96323350694442pt}}
\pgflineto{\pgfpoint{105.25441171874999pt}{173.69693663194442pt}}
\pgflineto{\pgfpoint{104.12779149305555pt}{173.69693663194442pt}}
\pgfpathclose
\pgfusepath{fill,stroke}
\pgfpathmoveto{\pgfpoint{106.51402708333332pt}{171.96323350694442pt}}
\pgflineto{\pgfpoint{106.16587916666666pt}{173.03472387152777pt}}
\pgflineto{\pgfpoint{105.25441171874999pt}{173.69693663194442pt}}
\pgfpathclose
\pgfusepath{fill,stroke}
\pgfpathmoveto{\pgfpoint{126.38540286458331pt}{152.43674262152777pt}}
\pgflineto{\pgfpoint{125.12578689236109pt}{150.70304861111111pt}}
\pgflineto{\pgfpoint{126.03725434027777pt}{151.36526137152777pt}}
\pgfpathclose
\pgfusepath{fill,stroke}
\pgfpathmoveto{\pgfpoint{126.38540286458331pt}{152.43674262152777pt}}
\pgflineto{\pgfpoint{123.99916666666667pt}{150.70304861111111pt}}
\pgflineto{\pgfpoint{125.12578689236109pt}{150.70304861111111pt}}
\pgfpathclose
\pgfusepath{fill,stroke}
\pgfpathmoveto{\pgfpoint{126.38540286458331pt}{152.43674262152777pt}}
\pgflineto{\pgfpoint{123.08770833333334pt}{151.36526137152777pt}}
\pgflineto{\pgfpoint{123.99916666666667pt}{150.70304861111111pt}}
\pgfpathclose
\pgfusepath{fill,stroke}
\pgfpathmoveto{\pgfpoint{126.38540286458331pt}{152.43674262152777pt}}
\pgflineto{\pgfpoint{122.73956953124998pt}{152.43674262152777pt}}
\pgflineto{\pgfpoint{123.08770833333334pt}{151.36526137152777pt}}
\pgfpathclose
\pgfusepath{fill,stroke}
\pgfpathmoveto{\pgfpoint{126.38540286458331pt}{152.43674262152777pt}}
\pgflineto{\pgfpoint{123.08770833333334pt}{153.50822326388888pt}}
\pgflineto{\pgfpoint{122.73956953124998pt}{152.43674262152777pt}}
\pgfpathclose
\pgfusepath{fill,stroke}
\pgfpathmoveto{\pgfpoint{126.38540286458331pt}{152.43674262152777pt}}
\pgflineto{\pgfpoint{123.99916666666667pt}{154.17043663194443pt}}
\pgflineto{\pgfpoint{123.08770833333334pt}{153.50822326388888pt}}
\pgfpathclose
\pgfusepath{fill,stroke}
\pgfpathmoveto{\pgfpoint{126.38540286458331pt}{152.43674262152777pt}}
\pgflineto{\pgfpoint{125.12578689236109pt}{154.17043663194443pt}}
\pgflineto{\pgfpoint{123.99916666666667pt}{154.17043663194443pt}}
\pgfpathclose
\pgfusepath{fill,stroke}
\pgfpathmoveto{\pgfpoint{126.38540286458331pt}{152.43674262152777pt}}
\pgflineto{\pgfpoint{126.03725434027777pt}{153.50822326388888pt}}
\pgflineto{\pgfpoint{125.12578689236109pt}{154.17043663194443pt}}
\pgfpathclose
\pgfusepath{fill,stroke}
\pgfpathmoveto{\pgfpoint{141.46590998263886pt}{133.68725902777777pt}}
\pgflineto{\pgfpoint{140.20631284722222pt}{131.95356501736111pt}}
\pgflineto{\pgfpoint{141.11777118055554pt}{132.61577838541666pt}}
\pgfpathclose
\pgfusepath{fill,stroke}
\pgfpathmoveto{\pgfpoint{141.46590998263886pt}{133.68725902777777pt}}
\pgflineto{\pgfpoint{139.07968350694443pt}{131.95356501736111pt}}
\pgflineto{\pgfpoint{140.20631284722222pt}{131.95356501736111pt}}
\pgfpathclose
\pgfusepath{fill,stroke}
\pgfpathmoveto{\pgfpoint{141.46590998263886pt}{133.68725902777777pt}}
\pgflineto{\pgfpoint{138.1682251736111pt}{132.61577838541666pt}}
\pgflineto{\pgfpoint{139.07968350694443pt}{131.95356501736111pt}}
\pgfpathclose
\pgfusepath{fill,stroke}
\pgfpathmoveto{\pgfpoint{141.46590998263886pt}{133.68725902777777pt}}
\pgflineto{\pgfpoint{137.82007664930555pt}{133.68725902777777pt}}
\pgflineto{\pgfpoint{138.1682251736111pt}{132.61577838541666pt}}
\pgfpathclose
\pgfusepath{fill,stroke}
\pgfpathmoveto{\pgfpoint{141.46590998263886pt}{133.68725902777777pt}}
\pgflineto{\pgfpoint{138.1682251736111pt}{134.75874027777778pt}}
\pgflineto{\pgfpoint{137.82007664930555pt}{133.68725902777777pt}}
\pgfpathclose
\pgfusepath{fill,stroke}
\pgfpathmoveto{\pgfpoint{141.46590998263886pt}{133.68725902777777pt}}
\pgflineto{\pgfpoint{139.07968350694443pt}{135.42095303819443pt}}
\pgflineto{\pgfpoint{138.1682251736111pt}{134.75874027777778pt}}
\pgfpathclose
\pgfusepath{fill,stroke}
\pgfpathmoveto{\pgfpoint{141.46590998263886pt}{133.68725902777777pt}}
\pgflineto{\pgfpoint{140.20631284722222pt}{135.42095303819443pt}}
\pgflineto{\pgfpoint{139.07968350694443pt}{135.42095303819443pt}}
\pgfpathclose
\pgfusepath{fill,stroke}
\pgfpathmoveto{\pgfpoint{141.46590998263886pt}{133.68725902777777pt}}
\pgflineto{\pgfpoint{141.11777118055554pt}{134.75874027777778pt}}
\pgflineto{\pgfpoint{140.20631284722222pt}{135.42095303819443pt}}
\pgfpathclose
\pgfusepath{fill,stroke}
\pgfpathmoveto{\pgfpoint{146.5881903645833pt}{128.56559114583334pt}}
\pgflineto{\pgfpoint{145.32859322916664pt}{126.83189713541665pt}}
\pgflineto{\pgfpoint{146.24005156249999pt}{127.49410989583333pt}}
\pgfpathclose
\pgfusepath{fill,stroke}
\pgfpathmoveto{\pgfpoint{146.5881903645833pt}{128.56559114583334pt}}
\pgflineto{\pgfpoint{144.20196328125pt}{126.83189713541665pt}}
\pgflineto{\pgfpoint{145.32859322916664pt}{126.83189713541665pt}}
\pgfpathclose
\pgfusepath{fill,stroke}
\pgfpathmoveto{\pgfpoint{146.5881903645833pt}{128.56559114583334pt}}
\pgflineto{\pgfpoint{143.29050494791667pt}{127.49410989583333pt}}
\pgflineto{\pgfpoint{144.20196328125pt}{126.83189713541665pt}}
\pgfpathclose
\pgfusepath{fill,stroke}
\pgfpathmoveto{\pgfpoint{146.5881903645833pt}{128.56559114583334pt}}
\pgflineto{\pgfpoint{142.94235703125pt}{128.56559114583334pt}}
\pgflineto{\pgfpoint{143.29050494791667pt}{127.49410989583333pt}}
\pgfpathclose
\pgfusepath{fill,stroke}
\pgfpathmoveto{\pgfpoint{146.5881903645833pt}{128.56559114583334pt}}
\pgflineto{\pgfpoint{143.29050494791667pt}{129.63708090277777pt}}
\pgflineto{\pgfpoint{142.94235703125pt}{128.56559114583334pt}}
\pgfpathclose
\pgfusepath{fill,stroke}
\pgfpathmoveto{\pgfpoint{146.5881903645833pt}{128.56559114583334pt}}
\pgflineto{\pgfpoint{144.20196328125pt}{130.29929427083331pt}}
\pgflineto{\pgfpoint{143.29050494791667pt}{129.63708090277777pt}}
\pgfpathclose
\pgfusepath{fill,stroke}
\pgfpathmoveto{\pgfpoint{146.5881903645833pt}{128.56559114583334pt}}
\pgflineto{\pgfpoint{145.32859322916664pt}{130.29929427083331pt}}
\pgflineto{\pgfpoint{144.20196328125pt}{130.29929427083331pt}}
\pgfpathclose
\pgfusepath{fill,stroke}
\pgfpathmoveto{\pgfpoint{146.5881903645833pt}{128.56559114583334pt}}
\pgflineto{\pgfpoint{146.24005156249999pt}{129.63708090277777pt}}
\pgflineto{\pgfpoint{145.32859322916664pt}{130.29929427083331pt}}
\pgfpathclose
\pgfusepath{fill,stroke}
\pgfpathmoveto{\pgfpoint{144.31605355902778pt}{131.22269453125pt}}
\pgflineto{\pgfpoint{143.05644670138886pt}{129.48900052083332pt}}
\pgflineto{\pgfpoint{143.9679050347222pt}{130.15121328125pt}}
\pgfpathclose
\pgfusepath{fill,stroke}
\pgfpathmoveto{\pgfpoint{144.31605355902778pt}{131.22269453125pt}}
\pgflineto{\pgfpoint{141.9298173611111pt}{129.48900052083332pt}}
\pgflineto{\pgfpoint{143.05644670138886pt}{129.48900052083332pt}}
\pgfpathclose
\pgfusepath{fill,stroke}
\pgfpathmoveto{\pgfpoint{144.31605355902778pt}{131.22269453125pt}}
\pgflineto{\pgfpoint{141.01835902777776pt}{130.15121328125pt}}
\pgflineto{\pgfpoint{141.9298173611111pt}{129.48900052083332pt}}
\pgfpathclose
\pgfusepath{fill,stroke}
\pgfpathmoveto{\pgfpoint{144.31605355902778pt}{131.22269453125pt}}
\pgflineto{\pgfpoint{140.67022022569444pt}{131.22269453125pt}}
\pgflineto{\pgfpoint{141.01835902777776pt}{130.15121328125pt}}
\pgfpathclose
\pgfusepath{fill,stroke}
\pgfpathmoveto{\pgfpoint{144.31605355902778pt}{131.22269453125pt}}
\pgflineto{\pgfpoint{141.01835902777776pt}{132.29418489583333pt}}
\pgflineto{\pgfpoint{140.67022022569444pt}{131.22269453125pt}}
\pgfpathclose
\pgfusepath{fill,stroke}
\pgfpathmoveto{\pgfpoint{144.31605355902778pt}{131.22269453125pt}}
\pgflineto{\pgfpoint{141.9298173611111pt}{132.95639765625pt}}
\pgflineto{\pgfpoint{141.01835902777776pt}{132.29418489583333pt}}
\pgfpathclose
\pgfusepath{fill,stroke}
\pgfpathmoveto{\pgfpoint{144.31605355902778pt}{131.22269453125pt}}
\pgflineto{\pgfpoint{143.05644670138886pt}{132.95639765625pt}}
\pgflineto{\pgfpoint{141.9298173611111pt}{132.95639765625pt}}
\pgfpathclose
\pgfusepath{fill,stroke}
\pgfpathmoveto{\pgfpoint{144.31605355902778pt}{131.22269453125pt}}
\pgflineto{\pgfpoint{143.9679050347222pt}{132.29418489583333pt}}
\pgflineto{\pgfpoint{143.05644670138886pt}{132.95639765625pt}}
\pgfpathclose
\pgfusepath{fill,stroke}
\pgfpathmoveto{\pgfpoint{122.86448854166666pt}{125.96410920138888pt}}
\pgflineto{\pgfpoint{121.60488229166666pt}{124.23041519097221pt}}
\pgflineto{\pgfpoint{122.51634062499998pt}{124.89262795138887pt}}
\pgfpathclose
\pgfusepath{fill,stroke}
\pgfpathmoveto{\pgfpoint{122.86448854166666pt}{125.96410920138888pt}}
\pgflineto{\pgfpoint{120.47826206597222pt}{124.23041519097221pt}}
\pgflineto{\pgfpoint{121.60488229166666pt}{124.23041519097221pt}}
\pgfpathclose
\pgfusepath{fill,stroke}
\pgfpathmoveto{\pgfpoint{122.86448854166666pt}{125.96410920138888pt}}
\pgflineto{\pgfpoint{119.56680373263887pt}{124.89262795138887pt}}
\pgflineto{\pgfpoint{120.47826206597222pt}{124.23041519097221pt}}
\pgfpathclose
\pgfusepath{fill,stroke}
\pgfpathmoveto{\pgfpoint{122.86448854166666pt}{125.96410920138888pt}}
\pgflineto{\pgfpoint{119.21865520833333pt}{125.96410920138888pt}}
\pgflineto{\pgfpoint{119.56680373263887pt}{124.89262795138887pt}}
\pgfpathclose
\pgfusepath{fill,stroke}
\pgfpathmoveto{\pgfpoint{122.86448854166666pt}{125.96410920138888pt}}
\pgflineto{\pgfpoint{119.56680373263887pt}{127.0355995659722pt}}
\pgflineto{\pgfpoint{119.21865520833333pt}{125.96410920138888pt}}
\pgfpathclose
\pgfusepath{fill,stroke}
\pgfpathmoveto{\pgfpoint{122.86448854166666pt}{125.96410920138888pt}}
\pgflineto{\pgfpoint{120.47826206597222pt}{127.69781232638887pt}}
\pgflineto{\pgfpoint{119.56680373263887pt}{127.0355995659722pt}}
\pgfpathclose
\pgfusepath{fill,stroke}
\pgfpathmoveto{\pgfpoint{122.86448854166666pt}{125.96410920138888pt}}
\pgflineto{\pgfpoint{121.60488229166666pt}{127.69781232638887pt}}
\pgflineto{\pgfpoint{120.47826206597222pt}{127.69781232638887pt}}
\pgfpathclose
\pgfusepath{fill,stroke}
\pgfpathmoveto{\pgfpoint{122.86448854166666pt}{125.96410920138888pt}}
\pgflineto{\pgfpoint{122.51634062499998pt}{127.0355995659722pt}}
\pgflineto{\pgfpoint{121.60488229166666pt}{127.69781232638887pt}}
\pgfpathclose
\pgfusepath{fill,stroke}
\pgfpathmoveto{\pgfpoint{120.82829184027777pt}{158.00168793402779pt}}
\pgflineto{\pgfpoint{119.56869470486112pt}{156.26798480902778pt}}
\pgflineto{\pgfpoint{120.48015303819444pt}{156.93019756944443pt}}
\pgfpathclose
\pgfusepath{fill,stroke}
\pgfpathmoveto{\pgfpoint{120.82829184027777pt}{158.00168793402779pt}}
\pgflineto{\pgfpoint{118.44207447916666pt}{156.26798480902778pt}}
\pgflineto{\pgfpoint{119.56869470486112pt}{156.26798480902778pt}}
\pgfpathclose
\pgfusepath{fill,stroke}
\pgfpathmoveto{\pgfpoint{120.82829184027777pt}{158.00168793402779pt}}
\pgflineto{\pgfpoint{117.53061614583333pt}{156.93019756944443pt}}
\pgflineto{\pgfpoint{118.44207447916666pt}{156.26798480902778pt}}
\pgfpathclose
\pgfusepath{fill,stroke}
\pgfpathmoveto{\pgfpoint{120.82829184027777pt}{158.00168793402779pt}}
\pgflineto{\pgfpoint{117.18246822916666pt}{158.00168793402779pt}}
\pgflineto{\pgfpoint{117.53061614583333pt}{156.93019756944443pt}}
\pgfpathclose
\pgfusepath{fill,stroke}
\pgfpathmoveto{\pgfpoint{120.82829184027777pt}{158.00168793402779pt}}
\pgflineto{\pgfpoint{117.53061614583333pt}{159.0731776909722pt}}
\pgflineto{\pgfpoint{117.18246822916666pt}{158.00168793402779pt}}
\pgfpathclose
\pgfusepath{fill,stroke}
\pgfpathmoveto{\pgfpoint{120.82829184027777pt}{158.00168793402779pt}}
\pgflineto{\pgfpoint{118.44207447916666pt}{159.73539105902776pt}}
\pgflineto{\pgfpoint{117.53061614583333pt}{159.0731776909722pt}}
\pgfpathclose
\pgfusepath{fill,stroke}
\pgfpathmoveto{\pgfpoint{120.82829184027777pt}{158.00168793402779pt}}
\pgflineto{\pgfpoint{119.56869470486112pt}{159.73539105902776pt}}
\pgflineto{\pgfpoint{118.44207447916666pt}{159.73539105902776pt}}
\pgfpathclose
\pgfusepath{fill,stroke}
\pgfpathmoveto{\pgfpoint{120.82829184027777pt}{158.00168793402779pt}}
\pgflineto{\pgfpoint{120.48015303819444pt}{159.0731776909722pt}}
\pgflineto{\pgfpoint{119.56869470486112pt}{159.73539105902776pt}}
\pgfpathclose
\pgfusepath{fill,stroke}
\pgfpathmoveto{\pgfpoint{116.49081718749999pt}{158.33628975694444pt}}
\pgflineto{\pgfpoint{115.23120121527776pt}{156.60258663194443pt}}
\pgflineto{\pgfpoint{116.14266927083332pt}{157.2647993923611pt}}
\pgfpathclose
\pgfusepath{fill,stroke}
\pgfpathmoveto{\pgfpoint{116.49081718749999pt}{158.33628975694444pt}}
\pgflineto{\pgfpoint{114.10458098958331pt}{156.60258663194443pt}}
\pgflineto{\pgfpoint{115.23120121527776pt}{156.60258663194443pt}}
\pgfpathclose
\pgfusepath{fill,stroke}
\pgfpathmoveto{\pgfpoint{116.49081718749999pt}{158.33628975694444pt}}
\pgflineto{\pgfpoint{113.19312265624998pt}{157.2647993923611pt}}
\pgflineto{\pgfpoint{114.10458098958331pt}{156.60258663194443pt}}
\pgfpathclose
\pgfusepath{fill,stroke}
\pgfpathmoveto{\pgfpoint{116.49081718749999pt}{158.33628975694444pt}}
\pgflineto{\pgfpoint{112.84498385416666pt}{158.33628975694444pt}}
\pgflineto{\pgfpoint{113.19312265624998pt}{157.2647993923611pt}}
\pgfpathclose
\pgfusepath{fill,stroke}
\pgfpathmoveto{\pgfpoint{116.49081718749999pt}{158.33628975694444pt}}
\pgflineto{\pgfpoint{113.19312265624998pt}{159.4077612847222pt}}
\pgflineto{\pgfpoint{112.84498385416666pt}{158.33628975694444pt}}
\pgfpathclose
\pgfusepath{fill,stroke}
\pgfpathmoveto{\pgfpoint{116.49081718749999pt}{158.33628975694444pt}}
\pgflineto{\pgfpoint{114.10458098958331pt}{160.06997404513888pt}}
\pgflineto{\pgfpoint{113.19312265624998pt}{159.4077612847222pt}}
\pgfpathclose
\pgfusepath{fill,stroke}
\pgfpathmoveto{\pgfpoint{116.49081718749999pt}{158.33628975694444pt}}
\pgflineto{\pgfpoint{115.23120121527776pt}{160.06997404513888pt}}
\pgflineto{\pgfpoint{114.10458098958331pt}{160.06997404513888pt}}
\pgfpathclose
\pgfusepath{fill,stroke}
\pgfpathmoveto{\pgfpoint{116.49081718749999pt}{158.33628975694444pt}}
\pgflineto{\pgfpoint{116.14266927083332pt}{159.4077612847222pt}}
\pgflineto{\pgfpoint{115.23120121527776pt}{160.06997404513888pt}}
\pgfpathclose
\pgfusepath{fill,stroke}
\pgfpathmoveto{\pgfpoint{139.54638116319444pt}{133.24655008680554pt}}
\pgflineto{\pgfpoint{138.28678402777777pt}{131.51285668402778pt}}
\pgflineto{\pgfpoint{139.19824236111108pt}{132.17506944444443pt}}
\pgfpathclose
\pgfusepath{fill,stroke}
\pgfpathmoveto{\pgfpoint{139.54638116319444pt}{133.24655008680554pt}}
\pgflineto{\pgfpoint{137.16015407986112pt}{131.51285668402778pt}}
\pgflineto{\pgfpoint{138.28678402777777pt}{131.51285668402778pt}}
\pgfpathclose
\pgfusepath{fill,stroke}
\pgfpathmoveto{\pgfpoint{139.54638116319444pt}{133.24655008680554pt}}
\pgflineto{\pgfpoint{136.24869574652777pt}{132.17506944444443pt}}
\pgflineto{\pgfpoint{137.16015407986112pt}{131.51285668402778pt}}
\pgfpathclose
\pgfusepath{fill,stroke}
\pgfpathmoveto{\pgfpoint{139.54638116319444pt}{133.24655008680554pt}}
\pgflineto{\pgfpoint{135.9005478298611pt}{133.24655008680554pt}}
\pgflineto{\pgfpoint{136.24869574652777pt}{132.17506944444443pt}}
\pgfpathclose
\pgfusepath{fill,stroke}
\pgfpathmoveto{\pgfpoint{139.54638116319444pt}{133.24655008680554pt}}
\pgflineto{\pgfpoint{136.24869574652777pt}{134.31803133680555pt}}
\pgflineto{\pgfpoint{135.9005478298611pt}{133.24655008680554pt}}
\pgfpathclose
\pgfusepath{fill,stroke}
\pgfpathmoveto{\pgfpoint{139.54638116319444pt}{133.24655008680554pt}}
\pgflineto{\pgfpoint{137.16015407986112pt}{134.9802440972222pt}}
\pgflineto{\pgfpoint{136.24869574652777pt}{134.31803133680555pt}}
\pgfpathclose
\pgfusepath{fill,stroke}
\pgfpathmoveto{\pgfpoint{139.54638116319444pt}{133.24655008680554pt}}
\pgflineto{\pgfpoint{138.28678402777777pt}{134.9802440972222pt}}
\pgflineto{\pgfpoint{137.16015407986112pt}{134.9802440972222pt}}
\pgfpathclose
\pgfusepath{fill,stroke}
\pgfpathmoveto{\pgfpoint{139.54638116319444pt}{133.24655008680554pt}}
\pgflineto{\pgfpoint{139.19824236111108pt}{134.31803133680555pt}}
\pgflineto{\pgfpoint{138.28678402777777pt}{134.9802440972222pt}}
\pgfpathclose
\pgfusepath{fill,stroke}
\pgfpathmoveto{\pgfpoint{135.23273446180553pt}{122.99893958333332pt}}
\pgflineto{\pgfpoint{133.97312821180554pt}{121.26523645833333pt}}
\pgflineto{\pgfpoint{134.8845865451389pt}{121.92744921875pt}}
\pgfpathclose
\pgfusepath{fill,stroke}
\pgfpathmoveto{\pgfpoint{135.23273446180553pt}{122.99893958333332pt}}
\pgflineto{\pgfpoint{132.84649826388886pt}{121.26523645833333pt}}
\pgflineto{\pgfpoint{133.97312821180554pt}{121.26523645833333pt}}
\pgfpathclose
\pgfusepath{fill,stroke}
\pgfpathmoveto{\pgfpoint{135.23273446180553pt}{122.99893958333332pt}}
\pgflineto{\pgfpoint{131.93503993055555pt}{121.92744921875pt}}
\pgflineto{\pgfpoint{132.84649826388886pt}{121.26523645833333pt}}
\pgfpathclose
\pgfusepath{fill,stroke}
\pgfpathmoveto{\pgfpoint{135.23273446180553pt}{122.99893958333332pt}}
\pgflineto{\pgfpoint{131.58690112847222pt}{122.99893958333332pt}}
\pgflineto{\pgfpoint{131.93503993055555pt}{121.92744921875pt}}
\pgfpathclose
\pgfusepath{fill,stroke}
\pgfpathmoveto{\pgfpoint{135.23273446180553pt}{122.99893958333332pt}}
\pgflineto{\pgfpoint{131.93503993055555pt}{124.07042083333332pt}}
\pgflineto{\pgfpoint{131.58690112847222pt}{122.99893958333332pt}}
\pgfpathclose
\pgfusepath{fill,stroke}
\pgfpathmoveto{\pgfpoint{135.23273446180553pt}{122.99893958333332pt}}
\pgflineto{\pgfpoint{132.84649826388886pt}{124.73263359375pt}}
\pgflineto{\pgfpoint{131.93503993055555pt}{124.07042083333332pt}}
\pgfpathclose
\pgfusepath{fill,stroke}
\pgfpathmoveto{\pgfpoint{135.23273446180553pt}{122.99893958333332pt}}
\pgflineto{\pgfpoint{133.97312821180554pt}{124.73263359375pt}}
\pgflineto{\pgfpoint{132.84649826388886pt}{124.73263359375pt}}
\pgfpathclose
\pgfusepath{fill,stroke}
\pgfpathmoveto{\pgfpoint{135.23273446180553pt}{122.99893958333332pt}}
\pgflineto{\pgfpoint{134.8845865451389pt}{124.07042083333332pt}}
\pgflineto{\pgfpoint{133.97312821180554pt}{124.73263359375pt}}
\pgfpathclose
\pgfusepath{fill,stroke}
\pgfpathmoveto{\pgfpoint{119.68030946180555pt}{133.54178420138888pt}}
\pgflineto{\pgfpoint{118.42069348958333pt}{131.80808107638887pt}}
\pgflineto{\pgfpoint{119.33216093749999pt}{132.47029383680555pt}}
\pgfpathclose
\pgfusepath{fill,stroke}
\pgfpathmoveto{\pgfpoint{119.68030946180555pt}{133.54178420138888pt}}
\pgflineto{\pgfpoint{117.29407326388888pt}{131.80808107638887pt}}
\pgflineto{\pgfpoint{118.42069348958333pt}{131.80808107638887pt}}
\pgfpathclose
\pgfusepath{fill,stroke}
\pgfpathmoveto{\pgfpoint{119.68030946180555pt}{133.54178420138888pt}}
\pgflineto{\pgfpoint{116.38261493055555pt}{132.47029383680555pt}}
\pgflineto{\pgfpoint{117.29407326388888pt}{131.80808107638887pt}}
\pgfpathclose
\pgfusepath{fill,stroke}
\pgfpathmoveto{\pgfpoint{119.68030946180555pt}{133.54178420138888pt}}
\pgflineto{\pgfpoint{116.03447612847222pt}{133.54178420138888pt}}
\pgflineto{\pgfpoint{116.38261493055555pt}{132.47029383680555pt}}
\pgfpathclose
\pgfusepath{fill,stroke}
\pgfpathmoveto{\pgfpoint{119.68030946180555pt}{133.54178420138888pt}}
\pgflineto{\pgfpoint{116.38261493055555pt}{134.61326484375pt}}
\pgflineto{\pgfpoint{116.03447612847222pt}{133.54178420138888pt}}
\pgfpathclose
\pgfusepath{fill,stroke}
\pgfpathmoveto{\pgfpoint{119.68030946180555pt}{133.54178420138888pt}}
\pgflineto{\pgfpoint{117.29407326388888pt}{135.27547760416667pt}}
\pgflineto{\pgfpoint{116.38261493055555pt}{134.61326484375pt}}
\pgfpathclose
\pgfusepath{fill,stroke}
\pgfpathmoveto{\pgfpoint{119.68030946180555pt}{133.54178420138888pt}}
\pgflineto{\pgfpoint{118.42069348958333pt}{135.27547760416667pt}}
\pgflineto{\pgfpoint{117.29407326388888pt}{135.27547760416667pt}}
\pgfpathclose
\pgfusepath{fill,stroke}
\pgfpathmoveto{\pgfpoint{119.68030946180555pt}{133.54178420138888pt}}
\pgflineto{\pgfpoint{119.33216093749999pt}{134.61326484375pt}}
\pgflineto{\pgfpoint{118.42069348958333pt}{135.27547760416667pt}}
\pgfpathclose
\pgfusepath{fill,stroke}
\pgfpathmoveto{\pgfpoint{128.4613567708333pt}{139.40366927083332pt}}
\pgflineto{\pgfpoint{127.20175052083333pt}{137.6699661458333pt}}
\pgflineto{\pgfpoint{128.11320885416666pt}{138.3321880208333pt}}
\pgfpathclose
\pgfusepath{fill,stroke}
\pgfpathmoveto{\pgfpoint{128.4613567708333pt}{139.40366927083332pt}}
\pgflineto{\pgfpoint{126.07513029513888pt}{137.6699661458333pt}}
\pgflineto{\pgfpoint{127.20175052083333pt}{137.6699661458333pt}}
\pgfpathclose
\pgfusepath{fill,stroke}
\pgfpathmoveto{\pgfpoint{128.4613567708333pt}{139.40366927083332pt}}
\pgflineto{\pgfpoint{125.16367196180555pt}{138.3321880208333pt}}
\pgflineto{\pgfpoint{126.07513029513888pt}{137.6699661458333pt}}
\pgfpathclose
\pgfusepath{fill,stroke}
\pgfpathmoveto{\pgfpoint{128.4613567708333pt}{139.40366927083332pt}}
\pgflineto{\pgfpoint{124.81553315972222pt}{139.40366927083332pt}}
\pgflineto{\pgfpoint{125.16367196180555pt}{138.3321880208333pt}}
\pgfpathclose
\pgfusepath{fill,stroke}
\pgfpathmoveto{\pgfpoint{128.4613567708333pt}{139.40366927083332pt}}
\pgflineto{\pgfpoint{125.16367196180555pt}{140.47514991319443pt}}
\pgflineto{\pgfpoint{124.81553315972222pt}{139.40366927083332pt}}
\pgfpathclose
\pgfusepath{fill,stroke}
\pgfpathmoveto{\pgfpoint{128.4613567708333pt}{139.40366927083332pt}}
\pgflineto{\pgfpoint{126.07513029513888pt}{141.1373626736111pt}}
\pgflineto{\pgfpoint{125.16367196180555pt}{140.47514991319443pt}}
\pgfpathclose
\pgfusepath{fill,stroke}
\pgfpathmoveto{\pgfpoint{128.4613567708333pt}{139.40366927083332pt}}
\pgflineto{\pgfpoint{127.20175052083333pt}{141.1373626736111pt}}
\pgflineto{\pgfpoint{126.07513029513888pt}{141.1373626736111pt}}
\pgfpathclose
\pgfusepath{fill,stroke}
\pgfpathmoveto{\pgfpoint{128.4613567708333pt}{139.40366927083332pt}}
\pgflineto{\pgfpoint{128.11320885416666pt}{140.47514991319443pt}}
\pgflineto{\pgfpoint{127.20175052083333pt}{141.1373626736111pt}}
\pgfpathclose
\pgfusepath{fill,stroke}
\pgfpathmoveto{\pgfpoint{119.30381701388889pt}{159.01789956597221pt}}
\pgflineto{\pgfpoint{118.04421076388888pt}{157.2841964409722pt}}
\pgflineto{\pgfpoint{118.95566909722221pt}{157.9464092013889pt}}
\pgfpathclose
\pgfusepath{fill,stroke}
\pgfpathmoveto{\pgfpoint{119.30381701388889pt}{159.01789956597221pt}}
\pgflineto{\pgfpoint{116.91759053819443pt}{157.2841964409722pt}}
\pgflineto{\pgfpoint{118.04421076388888pt}{157.2841964409722pt}}
\pgfpathclose
\pgfusepath{fill,stroke}
\pgfpathmoveto{\pgfpoint{119.30381701388889pt}{159.01789956597221pt}}
\pgflineto{\pgfpoint{116.00612248263889pt}{157.9464092013889pt}}
\pgflineto{\pgfpoint{116.91759053819443pt}{157.2841964409722pt}}
\pgfpathclose
\pgfusepath{fill,stroke}
\pgfpathmoveto{\pgfpoint{119.30381701388889pt}{159.01789956597221pt}}
\pgflineto{\pgfpoint{115.65798368055556pt}{159.01789956597221pt}}
\pgflineto{\pgfpoint{116.00612248263889pt}{157.9464092013889pt}}
\pgfpathclose
\pgfusepath{fill,stroke}
\pgfpathmoveto{\pgfpoint{119.30381701388889pt}{159.01789956597221pt}}
\pgflineto{\pgfpoint{116.00612248263889pt}{160.08937109374997pt}}
\pgflineto{\pgfpoint{115.65798368055556pt}{159.01789956597221pt}}
\pgfpathclose
\pgfusepath{fill,stroke}
\pgfpathmoveto{\pgfpoint{119.30381701388889pt}{159.01789956597221pt}}
\pgflineto{\pgfpoint{116.91759053819443pt}{160.75158385416665pt}}
\pgflineto{\pgfpoint{116.00612248263889pt}{160.08937109374997pt}}
\pgfpathclose
\pgfusepath{fill,stroke}
\pgfpathmoveto{\pgfpoint{119.30381701388889pt}{159.01789956597221pt}}
\pgflineto{\pgfpoint{118.04421076388888pt}{160.75158385416665pt}}
\pgflineto{\pgfpoint{116.91759053819443pt}{160.75158385416665pt}}
\pgfpathclose
\pgfusepath{fill,stroke}
\pgfpathmoveto{\pgfpoint{119.30381701388889pt}{159.01789956597221pt}}
\pgflineto{\pgfpoint{118.95566909722221pt}{160.08937109374997pt}}
\pgflineto{\pgfpoint{118.04421076388888pt}{160.75158385416665pt}}
\pgfpathclose
\pgfusepath{fill,stroke}
\pgfpathmoveto{\pgfpoint{124.72305538194443pt}{124.6120251736111pt}}
\pgflineto{\pgfpoint{123.46344001736111pt}{122.87833177083331pt}}
\pgflineto{\pgfpoint{124.37490746527777pt}{123.54054453124999pt}}
\pgfpathclose
\pgfusepath{fill,stroke}
\pgfpathmoveto{\pgfpoint{124.72305538194443pt}{124.6120251736111pt}}
\pgflineto{\pgfpoint{122.33681918402776pt}{122.87833177083331pt}}
\pgflineto{\pgfpoint{123.46344001736111pt}{122.87833177083331pt}}
\pgfpathclose
\pgfusepath{fill,stroke}
\pgfpathmoveto{\pgfpoint{124.72305538194443pt}{124.6120251736111pt}}
\pgflineto{\pgfpoint{121.42536085069443pt}{123.54054453124999pt}}
\pgflineto{\pgfpoint{122.33681918402776pt}{122.87833177083331pt}}
\pgfpathclose
\pgfusepath{fill,stroke}
\pgfpathmoveto{\pgfpoint{124.72305538194443pt}{124.6120251736111pt}}
\pgflineto{\pgfpoint{121.07722204861109pt}{124.6120251736111pt}}
\pgflineto{\pgfpoint{121.42536085069443pt}{123.54054453124999pt}}
\pgfpathclose
\pgfusepath{fill,stroke}
\pgfpathmoveto{\pgfpoint{124.72305538194443pt}{124.6120251736111pt}}
\pgflineto{\pgfpoint{121.42536085069443pt}{125.68351553819443pt}}
\pgflineto{\pgfpoint{121.07722204861109pt}{124.6120251736111pt}}
\pgfpathclose
\pgfusepath{fill,stroke}
\pgfpathmoveto{\pgfpoint{124.72305538194443pt}{124.6120251736111pt}}
\pgflineto{\pgfpoint{122.33681918402776pt}{126.34572829861109pt}}
\pgflineto{\pgfpoint{121.42536085069443pt}{125.68351553819443pt}}
\pgfpathclose
\pgfusepath{fill,stroke}
\pgfpathmoveto{\pgfpoint{124.72305538194443pt}{124.6120251736111pt}}
\pgflineto{\pgfpoint{123.46344001736111pt}{126.34572829861109pt}}
\pgflineto{\pgfpoint{122.33681918402776pt}{126.34572829861109pt}}
\pgfpathclose
\pgfusepath{fill,stroke}
\pgfpathmoveto{\pgfpoint{124.72305538194443pt}{124.6120251736111pt}}
\pgflineto{\pgfpoint{124.37490746527777pt}{125.68351553819443pt}}
\pgflineto{\pgfpoint{123.46344001736111pt}{126.34572829861109pt}}
\pgfpathclose
\pgfusepath{fill,stroke}
\pgfpathmoveto{\pgfpoint{135.76564288194442pt}{141.00819687499998pt}}
\pgflineto{\pgfpoint{134.50603602430556pt}{139.27449375pt}}
\pgflineto{\pgfpoint{135.41749435763887pt}{139.93670651041666pt}}
\pgfpathclose
\pgfusepath{fill,stroke}
\pgfpathmoveto{\pgfpoint{135.76564288194442pt}{141.00819687499998pt}}
\pgflineto{\pgfpoint{133.37940668402777pt}{139.27449375pt}}
\pgflineto{\pgfpoint{134.50603602430556pt}{139.27449375pt}}
\pgfpathclose
\pgfusepath{fill,stroke}
\pgfpathmoveto{\pgfpoint{135.76564288194442pt}{141.00819687499998pt}}
\pgflineto{\pgfpoint{132.46794835069443pt}{139.93670651041666pt}}
\pgflineto{\pgfpoint{133.37940668402777pt}{139.27449375pt}}
\pgfpathclose
\pgfusepath{fill,stroke}
\pgfpathmoveto{\pgfpoint{135.76564288194442pt}{141.00819687499998pt}}
\pgflineto{\pgfpoint{132.1198095486111pt}{141.00819687499998pt}}
\pgflineto{\pgfpoint{132.46794835069443pt}{139.93670651041666pt}}
\pgfpathclose
\pgfusepath{fill,stroke}
\pgfpathmoveto{\pgfpoint{135.76564288194442pt}{141.00819687499998pt}}
\pgflineto{\pgfpoint{132.46794835069443pt}{142.079678125pt}}
\pgflineto{\pgfpoint{132.1198095486111pt}{141.00819687499998pt}}
\pgfpathclose
\pgfusepath{fill,stroke}
\pgfpathmoveto{\pgfpoint{135.76564288194442pt}{141.00819687499998pt}}
\pgflineto{\pgfpoint{133.37940668402777pt}{142.74189088541667pt}}
\pgflineto{\pgfpoint{132.46794835069443pt}{142.079678125pt}}
\pgfpathclose
\pgfusepath{fill,stroke}
\pgfpathmoveto{\pgfpoint{135.76564288194442pt}{141.00819687499998pt}}
\pgflineto{\pgfpoint{134.50603602430556pt}{142.74189088541667pt}}
\pgflineto{\pgfpoint{133.37940668402777pt}{142.74189088541667pt}}
\pgfpathclose
\pgfusepath{fill,stroke}
\pgfpathmoveto{\pgfpoint{135.76564288194442pt}{141.00819687499998pt}}
\pgflineto{\pgfpoint{135.41749435763887pt}{142.079678125pt}}
\pgflineto{\pgfpoint{134.50603602430556pt}{142.74189088541667pt}}
\pgfpathclose
\pgfusepath{fill,stroke}
\pgfpathmoveto{\pgfpoint{155.6131706597222pt}{140.55037256944445pt}}
\pgflineto{\pgfpoint{154.35357352430555pt}{138.81666944444444pt}}
\pgflineto{\pgfpoint{155.2650318576389pt}{139.47889131944444pt}}
\pgfpathclose
\pgfusepath{fill,stroke}
\pgfpathmoveto{\pgfpoint{155.6131706597222pt}{140.55037256944445pt}}
\pgflineto{\pgfpoint{153.22694357638886pt}{138.81666944444444pt}}
\pgflineto{\pgfpoint{154.35357352430555pt}{138.81666944444444pt}}
\pgfpathclose
\pgfusepath{fill,stroke}
\pgfpathmoveto{\pgfpoint{155.6131706597222pt}{140.55037256944445pt}}
\pgflineto{\pgfpoint{152.31548524305555pt}{139.47889131944444pt}}
\pgflineto{\pgfpoint{153.22694357638886pt}{138.81666944444444pt}}
\pgfpathclose
\pgfusepath{fill,stroke}
\pgfpathmoveto{\pgfpoint{155.6131706597222pt}{140.55037256944445pt}}
\pgflineto{\pgfpoint{151.96733732638887pt}{140.55037256944445pt}}
\pgflineto{\pgfpoint{152.31548524305555pt}{139.47889131944444pt}}
\pgfpathclose
\pgfusepath{fill,stroke}
\pgfpathmoveto{\pgfpoint{155.6131706597222pt}{140.55037256944445pt}}
\pgflineto{\pgfpoint{152.31548524305555pt}{141.62185321180556pt}}
\pgflineto{\pgfpoint{151.96733732638887pt}{140.55037256944445pt}}
\pgfpathclose
\pgfusepath{fill,stroke}
\pgfpathmoveto{\pgfpoint{155.6131706597222pt}{140.55037256944445pt}}
\pgflineto{\pgfpoint{153.22694357638886pt}{142.2840659722222pt}}
\pgflineto{\pgfpoint{152.31548524305555pt}{141.62185321180556pt}}
\pgfpathclose
\pgfusepath{fill,stroke}
\pgfpathmoveto{\pgfpoint{155.6131706597222pt}{140.55037256944445pt}}
\pgflineto{\pgfpoint{154.35357352430555pt}{142.2840659722222pt}}
\pgflineto{\pgfpoint{153.22694357638886pt}{142.2840659722222pt}}
\pgfpathclose
\pgfusepath{fill,stroke}
\pgfpathmoveto{\pgfpoint{155.6131706597222pt}{140.55037256944445pt}}
\pgflineto{\pgfpoint{155.2650318576389pt}{141.62185321180556pt}}
\pgflineto{\pgfpoint{154.35357352430555pt}{142.2840659722222pt}}
\pgfpathclose
\pgfusepath{fill,stroke}
\pgfpathmoveto{\pgfpoint{130.4020909722222pt}{128.53135920138888pt}}
\pgflineto{\pgfpoint{129.14249383680553pt}{126.79766519097221pt}}
\pgflineto{\pgfpoint{130.05395217013887pt}{127.45987855902776pt}}
\pgfpathclose
\pgfusepath{fill,stroke}
\pgfpathmoveto{\pgfpoint{130.4020909722222pt}{128.53135920138888pt}}
\pgflineto{\pgfpoint{128.0158730034722pt}{126.79766519097221pt}}
\pgflineto{\pgfpoint{129.14249383680553pt}{126.79766519097221pt}}
\pgfpathclose
\pgfusepath{fill,stroke}
\pgfpathmoveto{\pgfpoint{130.4020909722222pt}{128.53135920138888pt}}
\pgflineto{\pgfpoint{127.10441467013888pt}{127.45987855902776pt}}
\pgflineto{\pgfpoint{128.0158730034722pt}{126.79766519097221pt}}
\pgfpathclose
\pgfusepath{fill,stroke}
\pgfpathmoveto{\pgfpoint{130.4020909722222pt}{128.53135920138888pt}}
\pgflineto{\pgfpoint{126.75626675347222pt}{128.53135920138888pt}}
\pgflineto{\pgfpoint{127.10441467013888pt}{127.45987855902776pt}}
\pgfpathclose
\pgfusepath{fill,stroke}
\pgfpathmoveto{\pgfpoint{130.4020909722222pt}{128.53135920138888pt}}
\pgflineto{\pgfpoint{127.10441467013888pt}{129.6028404513889pt}}
\pgflineto{\pgfpoint{126.75626675347222pt}{128.53135920138888pt}}
\pgfpathclose
\pgfusepath{fill,stroke}
\pgfpathmoveto{\pgfpoint{130.4020909722222pt}{128.53135920138888pt}}
\pgflineto{\pgfpoint{128.0158730034722pt}{130.26505321180557pt}}
\pgflineto{\pgfpoint{127.10441467013888pt}{129.6028404513889pt}}
\pgfpathclose
\pgfusepath{fill,stroke}
\pgfpathmoveto{\pgfpoint{130.4020909722222pt}{128.53135920138888pt}}
\pgflineto{\pgfpoint{129.14249383680553pt}{130.26505321180557pt}}
\pgflineto{\pgfpoint{128.0158730034722pt}{130.26505321180557pt}}
\pgfpathclose
\pgfusepath{fill,stroke}
\pgfpathmoveto{\pgfpoint{130.4020909722222pt}{128.53135920138888pt}}
\pgflineto{\pgfpoint{130.05395217013887pt}{129.6028404513889pt}}
\pgflineto{\pgfpoint{129.14249383680553pt}{130.26505321180557pt}}
\pgfpathclose
\pgfusepath{fill,stroke}
\pgfpathmoveto{\pgfpoint{127.18343871527777pt}{137.5638037326389pt}}
\pgflineto{\pgfpoint{125.92383246527775pt}{135.8301097222222pt}}
\pgflineto{\pgfpoint{126.8352907986111pt}{136.49232248263888pt}}
\pgfpathclose
\pgfusepath{fill,stroke}
\pgfpathmoveto{\pgfpoint{127.18343871527777pt}{137.5638037326389pt}}
\pgflineto{\pgfpoint{124.79721163194442pt}{135.8301097222222pt}}
\pgflineto{\pgfpoint{125.92383246527775pt}{135.8301097222222pt}}
\pgfpathclose
\pgfusepath{fill,stroke}
\pgfpathmoveto{\pgfpoint{127.18343871527777pt}{137.5638037326389pt}}
\pgflineto{\pgfpoint{123.88575329861109pt}{136.49232248263888pt}}
\pgflineto{\pgfpoint{124.79721163194442pt}{135.8301097222222pt}}
\pgfpathclose
\pgfusepath{fill,stroke}
\pgfpathmoveto{\pgfpoint{127.18343871527777pt}{137.5638037326389pt}}
\pgflineto{\pgfpoint{123.53761449652777pt}{137.5638037326389pt}}
\pgflineto{\pgfpoint{123.88575329861109pt}{136.49232248263888pt}}
\pgfpathclose
\pgfusepath{fill,stroke}
\pgfpathmoveto{\pgfpoint{127.18343871527777pt}{137.5638037326389pt}}
\pgflineto{\pgfpoint{123.88575329861109pt}{138.635284375pt}}
\pgflineto{\pgfpoint{123.53761449652777pt}{137.5638037326389pt}}
\pgfpathclose
\pgfusepath{fill,stroke}
\pgfpathmoveto{\pgfpoint{127.18343871527777pt}{137.5638037326389pt}}
\pgflineto{\pgfpoint{124.79721163194442pt}{139.29750625pt}}
\pgflineto{\pgfpoint{123.88575329861109pt}{138.635284375pt}}
\pgfpathclose
\pgfusepath{fill,stroke}
\pgfpathmoveto{\pgfpoint{127.18343871527777pt}{137.5638037326389pt}}
\pgflineto{\pgfpoint{125.92383246527775pt}{139.29750625pt}}
\pgflineto{\pgfpoint{124.79721163194442pt}{139.29750625pt}}
\pgfpathclose
\pgfusepath{fill,stroke}
\pgfpathmoveto{\pgfpoint{127.18343871527777pt}{137.5638037326389pt}}
\pgflineto{\pgfpoint{126.8352907986111pt}{138.635284375pt}}
\pgflineto{\pgfpoint{125.92383246527775pt}{139.29750625pt}}
\pgfpathclose
\pgfusepath{fill,stroke}
\pgfpathmoveto{\pgfpoint{119.86059105902775pt}{115.63949565972221pt}}
\pgflineto{\pgfpoint{118.60098420138888pt}{113.90580225694443pt}}
\pgflineto{\pgfpoint{119.51245225694443pt}{114.5680150173611pt}}
\pgfpathclose
\pgfusepath{fill,stroke}
\pgfpathmoveto{\pgfpoint{119.86059105902775pt}{115.63949565972221pt}}
\pgflineto{\pgfpoint{117.47436397569443pt}{113.90580225694443pt}}
\pgflineto{\pgfpoint{118.60098420138888pt}{113.90580225694443pt}}
\pgfpathclose
\pgfusepath{fill,stroke}
\pgfpathmoveto{\pgfpoint{119.86059105902775pt}{115.63949565972221pt}}
\pgflineto{\pgfpoint{116.5629056423611pt}{114.5680150173611pt}}
\pgflineto{\pgfpoint{117.47436397569443pt}{113.90580225694443pt}}
\pgfpathclose
\pgfusepath{fill,stroke}
\pgfpathmoveto{\pgfpoint{119.86059105902775pt}{115.63949565972221pt}}
\pgflineto{\pgfpoint{116.21475772569443pt}{115.63949565972221pt}}
\pgflineto{\pgfpoint{116.5629056423611pt}{114.5680150173611pt}}
\pgfpathclose
\pgfusepath{fill,stroke}
\pgfpathmoveto{\pgfpoint{119.86059105902775pt}{115.63949565972221pt}}
\pgflineto{\pgfpoint{116.5629056423611pt}{116.71098602430554pt}}
\pgflineto{\pgfpoint{116.21475772569443pt}{115.63949565972221pt}}
\pgfpathclose
\pgfusepath{fill,stroke}
\pgfpathmoveto{\pgfpoint{119.86059105902775pt}{115.63949565972221pt}}
\pgflineto{\pgfpoint{117.47436397569443pt}{117.3731987847222pt}}
\pgflineto{\pgfpoint{116.5629056423611pt}{116.71098602430554pt}}
\pgfpathclose
\pgfusepath{fill,stroke}
\pgfpathmoveto{\pgfpoint{119.86059105902775pt}{115.63949565972221pt}}
\pgflineto{\pgfpoint{118.60098420138888pt}{117.3731987847222pt}}
\pgflineto{\pgfpoint{117.47436397569443pt}{117.3731987847222pt}}
\pgfpathclose
\pgfusepath{fill,stroke}
\pgfpathmoveto{\pgfpoint{119.86059105902775pt}{115.63949565972221pt}}
\pgflineto{\pgfpoint{119.51245225694443pt}{116.71098602430554pt}}
\pgflineto{\pgfpoint{118.60098420138888pt}{117.3731987847222pt}}
\pgfpathclose
\pgfusepath{fill,stroke}
\pgfpathmoveto{\pgfpoint{124.63289635416666pt}{140.55892994791665pt}}
\pgflineto{\pgfpoint{123.37329010416666pt}{138.8252365451389pt}}
\pgflineto{\pgfpoint{124.28475755208333pt}{139.48744930555554pt}}
\pgfpathclose
\pgfusepath{fill,stroke}
\pgfpathmoveto{\pgfpoint{124.63289635416666pt}{140.55892994791665pt}}
\pgflineto{\pgfpoint{122.24666927083332pt}{138.8252365451389pt}}
\pgflineto{\pgfpoint{123.37329010416666pt}{138.8252365451389pt}}
\pgfpathclose
\pgfusepath{fill,stroke}
\pgfpathmoveto{\pgfpoint{124.63289635416666pt}{140.55892994791665pt}}
\pgflineto{\pgfpoint{121.33521093749998pt}{139.48744930555554pt}}
\pgflineto{\pgfpoint{122.24666927083332pt}{138.8252365451389pt}}
\pgfpathclose
\pgfusepath{fill,stroke}
\pgfpathmoveto{\pgfpoint{124.63289635416666pt}{140.55892994791665pt}}
\pgflineto{\pgfpoint{120.98706302083333pt}{140.55892994791665pt}}
\pgflineto{\pgfpoint{121.33521093749998pt}{139.48744930555554pt}}
\pgfpathclose
\pgfusepath{fill,stroke}
\pgfpathmoveto{\pgfpoint{124.63289635416666pt}{140.55892994791665pt}}
\pgflineto{\pgfpoint{121.33521093749998pt}{141.63041119791666pt}}
\pgflineto{\pgfpoint{120.98706302083333pt}{140.55892994791665pt}}
\pgfpathclose
\pgfusepath{fill,stroke}
\pgfpathmoveto{\pgfpoint{124.63289635416666pt}{140.55892994791665pt}}
\pgflineto{\pgfpoint{122.24666927083332pt}{142.2926239583333pt}}
\pgflineto{\pgfpoint{121.33521093749998pt}{141.63041119791666pt}}
\pgfpathclose
\pgfusepath{fill,stroke}
\pgfpathmoveto{\pgfpoint{124.63289635416666pt}{140.55892994791665pt}}
\pgflineto{\pgfpoint{123.37329010416666pt}{142.2926239583333pt}}
\pgflineto{\pgfpoint{122.24666927083332pt}{142.2926239583333pt}}
\pgfpathclose
\pgfusepath{fill,stroke}
\pgfpathmoveto{\pgfpoint{124.63289635416666pt}{140.55892994791665pt}}
\pgflineto{\pgfpoint{124.28475755208333pt}{141.63041119791666pt}}
\pgflineto{\pgfpoint{123.37329010416666pt}{142.2926239583333pt}}
\pgfpathclose
\pgfusepath{fill,stroke}
\pgfpathmoveto{\pgfpoint{129.20901935763888pt}{124.71900364583333pt}}
\pgflineto{\pgfpoint{127.94941249999998pt}{122.98531024305555pt}}
\pgflineto{\pgfpoint{128.8608708333333pt}{123.64752300347222pt}}
\pgfpathclose
\pgfusepath{fill,stroke}
\pgfpathmoveto{\pgfpoint{129.20901935763888pt}{124.71900364583333pt}}
\pgflineto{\pgfpoint{126.82279227430554pt}{122.98531024305555pt}}
\pgflineto{\pgfpoint{127.94941249999998pt}{122.98531024305555pt}}
\pgfpathclose
\pgfusepath{fill,stroke}
\pgfpathmoveto{\pgfpoint{129.20901935763888pt}{124.71900364583333pt}}
\pgflineto{\pgfpoint{125.91133394097221pt}{123.64752300347222pt}}
\pgflineto{\pgfpoint{126.82279227430554pt}{122.98531024305555pt}}
\pgfpathclose
\pgfusepath{fill,stroke}
\pgfpathmoveto{\pgfpoint{129.20901935763888pt}{124.71900364583333pt}}
\pgflineto{\pgfpoint{125.56319513888887pt}{124.71900364583333pt}}
\pgflineto{\pgfpoint{125.91133394097221pt}{123.64752300347222pt}}
\pgfpathclose
\pgfusepath{fill,stroke}
\pgfpathmoveto{\pgfpoint{129.20901935763888pt}{124.71900364583333pt}}
\pgflineto{\pgfpoint{125.91133394097221pt}{125.79048489583332pt}}
\pgflineto{\pgfpoint{125.56319513888887pt}{124.71900364583333pt}}
\pgfpathclose
\pgfusepath{fill,stroke}
\pgfpathmoveto{\pgfpoint{129.20901935763888pt}{124.71900364583333pt}}
\pgflineto{\pgfpoint{126.82279227430554pt}{126.45269765624998pt}}
\pgflineto{\pgfpoint{125.91133394097221pt}{125.79048489583332pt}}
\pgfpathclose
\pgfusepath{fill,stroke}
\pgfpathmoveto{\pgfpoint{129.20901935763888pt}{124.71900364583333pt}}
\pgflineto{\pgfpoint{127.94941249999998pt}{126.45269765624998pt}}
\pgflineto{\pgfpoint{126.82279227430554pt}{126.45269765624998pt}}
\pgfpathclose
\pgfusepath{fill,stroke}
\pgfpathmoveto{\pgfpoint{129.20901935763888pt}{124.71900364583333pt}}
\pgflineto{\pgfpoint{128.8608708333333pt}{125.79048489583332pt}}
\pgflineto{\pgfpoint{127.94941249999998pt}{126.45269765624998pt}}
\pgfpathclose
\pgfusepath{fill,stroke}
\pgfpathmoveto{\pgfpoint{122.2387790798611pt}{135.49716736111108pt}}
\pgflineto{\pgfpoint{120.97918194444443pt}{133.76347335069443pt}}
\pgflineto{\pgfpoint{121.89064027777778pt}{134.42568671874997pt}}
\pgfpathclose
\pgfusepath{fill,stroke}
\pgfpathmoveto{\pgfpoint{122.2387790798611pt}{135.49716736111108pt}}
\pgflineto{\pgfpoint{119.85256171875pt}{133.76347335069443pt}}
\pgflineto{\pgfpoint{120.97918194444443pt}{133.76347335069443pt}}
\pgfpathclose
\pgfusepath{fill,stroke}
\pgfpathmoveto{\pgfpoint{122.2387790798611pt}{135.49716736111108pt}}
\pgflineto{\pgfpoint{118.94110338541667pt}{134.42568671874997pt}}
\pgflineto{\pgfpoint{119.85256171875pt}{133.76347335069443pt}}
\pgfpathclose
\pgfusepath{fill,stroke}
\pgfpathmoveto{\pgfpoint{122.2387790798611pt}{135.49716736111108pt}}
\pgflineto{\pgfpoint{118.5929548611111pt}{135.49716736111108pt}}
\pgflineto{\pgfpoint{118.94110338541667pt}{134.42568671874997pt}}
\pgfpathclose
\pgfusepath{fill,stroke}
\pgfpathmoveto{\pgfpoint{122.2387790798611pt}{135.49716736111108pt}}
\pgflineto{\pgfpoint{118.94110338541667pt}{136.56864800347222pt}}
\pgflineto{\pgfpoint{118.5929548611111pt}{135.49716736111108pt}}
\pgfpathclose
\pgfusepath{fill,stroke}
\pgfpathmoveto{\pgfpoint{122.2387790798611pt}{135.49716736111108pt}}
\pgflineto{\pgfpoint{119.85256171875pt}{137.23086137152777pt}}
\pgflineto{\pgfpoint{118.94110338541667pt}{136.56864800347222pt}}
\pgfpathclose
\pgfusepath{fill,stroke}
\pgfpathmoveto{\pgfpoint{122.2387790798611pt}{135.49716736111108pt}}
\pgflineto{\pgfpoint{120.97918194444443pt}{137.23086137152777pt}}
\pgflineto{\pgfpoint{119.85256171875pt}{137.23086137152777pt}}
\pgfpathclose
\pgfusepath{fill,stroke}
\pgfpathmoveto{\pgfpoint{122.2387790798611pt}{135.49716736111108pt}}
\pgflineto{\pgfpoint{121.89064027777778pt}{136.56864800347222pt}}
\pgflineto{\pgfpoint{120.97918194444443pt}{137.23086137152777pt}}
\pgfpathclose
\pgfusepath{fill,stroke}
\pgfpathmoveto{\pgfpoint{107.7919269097222pt}{138.35109348958332pt}}
\pgflineto{\pgfpoint{106.53232977430555pt}{136.61739036458331pt}}
\pgflineto{\pgfpoint{107.44378810763888pt}{137.279603125pt}}
\pgfpathclose
\pgfusepath{fill,stroke}
\pgfpathmoveto{\pgfpoint{107.7919269097222pt}{138.35109348958332pt}}
\pgflineto{\pgfpoint{105.4057095486111pt}{136.61739036458331pt}}
\pgflineto{\pgfpoint{106.53232977430555pt}{136.61739036458331pt}}
\pgfpathclose
\pgfusepath{fill,stroke}
\pgfpathmoveto{\pgfpoint{107.7919269097222pt}{138.35109348958332pt}}
\pgflineto{\pgfpoint{104.49425121527777pt}{137.279603125pt}}
\pgflineto{\pgfpoint{105.4057095486111pt}{136.61739036458331pt}}
\pgfpathclose
\pgfusepath{fill,stroke}
\pgfpathmoveto{\pgfpoint{107.7919269097222pt}{138.35109348958332pt}}
\pgflineto{\pgfpoint{104.14610329861111pt}{138.35109348958332pt}}
\pgflineto{\pgfpoint{104.49425121527777pt}{137.279603125pt}}
\pgfpathclose
\pgfusepath{fill,stroke}
\pgfpathmoveto{\pgfpoint{107.7919269097222pt}{138.35109348958332pt}}
\pgflineto{\pgfpoint{104.49425121527777pt}{139.42257413194443pt}}
\pgflineto{\pgfpoint{104.14610329861111pt}{138.35109348958332pt}}
\pgfpathclose
\pgfusepath{fill,stroke}
\pgfpathmoveto{\pgfpoint{107.7919269097222pt}{138.35109348958332pt}}
\pgflineto{\pgfpoint{105.4057095486111pt}{140.08478749999998pt}}
\pgflineto{\pgfpoint{104.49425121527777pt}{139.42257413194443pt}}
\pgfpathclose
\pgfusepath{fill,stroke}
\pgfpathmoveto{\pgfpoint{107.7919269097222pt}{138.35109348958332pt}}
\pgflineto{\pgfpoint{106.53232977430555pt}{140.08478749999998pt}}
\pgflineto{\pgfpoint{105.4057095486111pt}{140.08478749999998pt}}
\pgfpathclose
\pgfusepath{fill,stroke}
\pgfpathmoveto{\pgfpoint{107.7919269097222pt}{138.35109348958332pt}}
\pgflineto{\pgfpoint{107.44378810763888pt}{139.42257413194443pt}}
\pgflineto{\pgfpoint{106.53232977430555pt}{140.08478749999998pt}}
\pgfpathclose
\pgfusepath{fill,stroke}
\pgfpathmoveto{\pgfpoint{109.82811449652777pt}{144.67936996527777pt}}
\pgflineto{\pgfpoint{108.5685173611111pt}{142.94567656249998pt}}
\pgflineto{\pgfpoint{109.47997569444445pt}{143.60788932291666pt}}
\pgfpathclose
\pgfusepath{fill,stroke}
\pgfpathmoveto{\pgfpoint{109.82811449652777pt}{144.67936996527777pt}}
\pgflineto{\pgfpoint{107.44189713541667pt}{142.94567656249998pt}}
\pgflineto{\pgfpoint{108.5685173611111pt}{142.94567656249998pt}}
\pgfpathclose
\pgfusepath{fill,stroke}
\pgfpathmoveto{\pgfpoint{109.82811449652777pt}{144.67936996527777pt}}
\pgflineto{\pgfpoint{106.53043880208332pt}{143.60788932291666pt}}
\pgflineto{\pgfpoint{107.44189713541667pt}{142.94567656249998pt}}
\pgfpathclose
\pgfusepath{fill,stroke}
\pgfpathmoveto{\pgfpoint{109.82811449652777pt}{144.67936996527777pt}}
\pgflineto{\pgfpoint{106.18229027777777pt}{144.67936996527777pt}}
\pgflineto{\pgfpoint{106.53043880208332pt}{143.60788932291666pt}}
\pgfpathclose
\pgfusepath{fill,stroke}
\pgfpathmoveto{\pgfpoint{109.82811449652777pt}{144.67936996527777pt}}
\pgflineto{\pgfpoint{106.53043880208332pt}{145.75085121527778pt}}
\pgflineto{\pgfpoint{106.18229027777777pt}{144.67936996527777pt}}
\pgfpathclose
\pgfusepath{fill,stroke}
\pgfpathmoveto{\pgfpoint{109.82811449652777pt}{144.67936996527777pt}}
\pgflineto{\pgfpoint{107.44189713541667pt}{146.41306397569443pt}}
\pgflineto{\pgfpoint{106.53043880208332pt}{145.75085121527778pt}}
\pgfpathclose
\pgfusepath{fill,stroke}
\pgfpathmoveto{\pgfpoint{109.82811449652777pt}{144.67936996527777pt}}
\pgflineto{\pgfpoint{108.5685173611111pt}{146.41306397569443pt}}
\pgflineto{\pgfpoint{107.44189713541667pt}{146.41306397569443pt}}
\pgfpathclose
\pgfusepath{fill,stroke}
\pgfpathmoveto{\pgfpoint{109.82811449652777pt}{144.67936996527777pt}}
\pgflineto{\pgfpoint{109.47997569444445pt}{145.75085121527778pt}}
\pgflineto{\pgfpoint{108.5685173611111pt}{146.41306397569443pt}}
\pgfpathclose
\pgfusepath{fill,stroke}
\pgfpathmoveto{\pgfpoint{126.70090477430554pt}{137.82053116319443pt}}
\pgflineto{\pgfpoint{125.44130763888887pt}{136.08682803819443pt}}
\pgflineto{\pgfpoint{126.35276597222222pt}{136.7490407986111pt}}
\pgfpathclose
\pgfusepath{fill,stroke}
\pgfpathmoveto{\pgfpoint{126.70090477430554pt}{137.82053116319443pt}}
\pgflineto{\pgfpoint{124.31468680555554pt}{136.08682803819443pt}}
\pgflineto{\pgfpoint{125.44130763888887pt}{136.08682803819443pt}}
\pgfpathclose
\pgfusepath{fill,stroke}
\pgfpathmoveto{\pgfpoint{126.70090477430554pt}{137.82053116319443pt}}
\pgflineto{\pgfpoint{123.40322847222221pt}{136.7490407986111pt}}
\pgflineto{\pgfpoint{124.31468680555554pt}{136.08682803819443pt}}
\pgfpathclose
\pgfusepath{fill,stroke}
\pgfpathmoveto{\pgfpoint{126.70090477430554pt}{137.82053116319443pt}}
\pgflineto{\pgfpoint{123.05508055555555pt}{137.82053116319443pt}}
\pgflineto{\pgfpoint{123.40322847222221pt}{136.7490407986111pt}}
\pgfpathclose
\pgfusepath{fill,stroke}
\pgfpathmoveto{\pgfpoint{126.70090477430554pt}{137.82053116319443pt}}
\pgflineto{\pgfpoint{123.40322847222221pt}{138.89201241319444pt}}
\pgflineto{\pgfpoint{123.05508055555555pt}{137.82053116319443pt}}
\pgfpathclose
\pgfusepath{fill,stroke}
\pgfpathmoveto{\pgfpoint{126.70090477430554pt}{137.82053116319443pt}}
\pgflineto{\pgfpoint{124.31468680555554pt}{139.5542251736111pt}}
\pgflineto{\pgfpoint{123.40322847222221pt}{138.89201241319444pt}}
\pgfpathclose
\pgfusepath{fill,stroke}
\pgfpathmoveto{\pgfpoint{126.70090477430554pt}{137.82053116319443pt}}
\pgflineto{\pgfpoint{125.44130763888887pt}{139.5542251736111pt}}
\pgflineto{\pgfpoint{124.31468680555554pt}{139.5542251736111pt}}
\pgfpathclose
\pgfusepath{fill,stroke}
\pgfpathmoveto{\pgfpoint{126.70090477430554pt}{137.82053116319443pt}}
\pgflineto{\pgfpoint{126.35276597222222pt}{138.89201241319444pt}}
\pgflineto{\pgfpoint{125.44130763888887pt}{139.5542251736111pt}}
\pgfpathclose
\pgfusepath{fill,stroke}
\pgfpathmoveto{\pgfpoint{130.45512569444443pt}{123.47816371527777pt}}
\pgflineto{\pgfpoint{129.19551944444444pt}{121.7444697048611pt}}
\pgflineto{\pgfpoint{130.1069868923611pt}{122.40668246527777pt}}
\pgfpathclose
\pgfusepath{fill,stroke}
\pgfpathmoveto{\pgfpoint{130.45512569444443pt}{123.47816371527777pt}}
\pgflineto{\pgfpoint{128.0688986111111pt}{121.7444697048611pt}}
\pgflineto{\pgfpoint{129.19551944444444pt}{121.7444697048611pt}}
\pgfpathclose
\pgfusepath{fill,stroke}
\pgfpathmoveto{\pgfpoint{130.45512569444443pt}{123.47816371527777pt}}
\pgflineto{\pgfpoint{127.15744027777777pt}{122.40668246527777pt}}
\pgflineto{\pgfpoint{128.0688986111111pt}{121.7444697048611pt}}
\pgfpathclose
\pgfusepath{fill,stroke}
\pgfpathmoveto{\pgfpoint{130.45512569444443pt}{123.47816371527777pt}}
\pgflineto{\pgfpoint{126.8092923611111pt}{123.47816371527777pt}}
\pgflineto{\pgfpoint{127.15744027777777pt}{122.40668246527777pt}}
\pgfpathclose
\pgfusepath{fill,stroke}
\pgfpathmoveto{\pgfpoint{130.45512569444443pt}{123.47816371527777pt}}
\pgflineto{\pgfpoint{127.15744027777777pt}{124.54964435763888pt}}
\pgflineto{\pgfpoint{126.8092923611111pt}{123.47816371527777pt}}
\pgfpathclose
\pgfusepath{fill,stroke}
\pgfpathmoveto{\pgfpoint{130.45512569444443pt}{123.47816371527777pt}}
\pgflineto{\pgfpoint{128.0688986111111pt}{125.21185772569443pt}}
\pgflineto{\pgfpoint{127.15744027777777pt}{124.54964435763888pt}}
\pgfpathclose
\pgfusepath{fill,stroke}
\pgfpathmoveto{\pgfpoint{130.45512569444443pt}{123.47816371527777pt}}
\pgflineto{\pgfpoint{129.19551944444444pt}{125.21185772569443pt}}
\pgflineto{\pgfpoint{128.0688986111111pt}{125.21185772569443pt}}
\pgfpathclose
\pgfusepath{fill,stroke}
\pgfpathmoveto{\pgfpoint{130.45512569444443pt}{123.47816371527777pt}}
\pgflineto{\pgfpoint{130.1069868923611pt}{124.54964435763888pt}}
\pgflineto{\pgfpoint{129.19551944444444pt}{125.21185772569443pt}}
\pgfpathclose
\pgfusepath{fill,stroke}
\pgfpathmoveto{\pgfpoint{132.93672291666667pt}{129.07903749999997pt}}
\pgflineto{\pgfpoint{131.67711605902775pt}{127.34534348958333pt}}
\pgflineto{\pgfpoint{132.5885743923611pt}{128.00755625pt}}
\pgfpathclose
\pgfusepath{fill,stroke}
\pgfpathmoveto{\pgfpoint{132.93672291666667pt}{129.07903749999997pt}}
\pgflineto{\pgfpoint{130.55049583333331pt}{127.34534348958333pt}}
\pgflineto{\pgfpoint{131.67711605902775pt}{127.34534348958333pt}}
\pgfpathclose
\pgfusepath{fill,stroke}
\pgfpathmoveto{\pgfpoint{132.93672291666667pt}{129.07903749999997pt}}
\pgflineto{\pgfpoint{129.6390375pt}{128.00755625pt}}
\pgflineto{\pgfpoint{130.55049583333331pt}{127.34534348958333pt}}
\pgfpathclose
\pgfusepath{fill,stroke}
\pgfpathmoveto{\pgfpoint{132.93672291666667pt}{129.07903749999997pt}}
\pgflineto{\pgfpoint{129.29088958333332pt}{129.07903749999997pt}}
\pgflineto{\pgfpoint{129.6390375pt}{128.00755625pt}}
\pgfpathclose
\pgfusepath{fill,stroke}
\pgfpathmoveto{\pgfpoint{132.93672291666667pt}{129.07903749999997pt}}
\pgflineto{\pgfpoint{129.6390375pt}{130.15052725694443pt}}
\pgflineto{\pgfpoint{129.29088958333332pt}{129.07903749999997pt}}
\pgfpathclose
\pgfusepath{fill,stroke}
\pgfpathmoveto{\pgfpoint{132.93672291666667pt}{129.07903749999997pt}}
\pgflineto{\pgfpoint{130.55049583333331pt}{130.81274062499998pt}}
\pgflineto{\pgfpoint{129.6390375pt}{130.15052725694443pt}}
\pgfpathclose
\pgfusepath{fill,stroke}
\pgfpathmoveto{\pgfpoint{132.93672291666667pt}{129.07903749999997pt}}
\pgflineto{\pgfpoint{131.67711605902775pt}{130.81274062499998pt}}
\pgflineto{\pgfpoint{130.55049583333331pt}{130.81274062499998pt}}
\pgfpathclose
\pgfusepath{fill,stroke}
\pgfpathmoveto{\pgfpoint{132.93672291666667pt}{129.07903749999997pt}}
\pgflineto{\pgfpoint{132.5885743923611pt}{130.15052725694443pt}}
\pgflineto{\pgfpoint{131.67711605902775pt}{130.81274062499998pt}}
\pgfpathclose
\pgfusepath{fill,stroke}
\pgfpathmoveto{\pgfpoint{145.7079689236111pt}{108.6907943576389pt}}
\pgflineto{\pgfpoint{144.4483626736111pt}{106.95710034722221pt}}
\pgflineto{\pgfpoint{145.35982100694443pt}{107.61931310763887pt}}
\pgfpathclose
\pgfusepath{fill,stroke}
\pgfpathmoveto{\pgfpoint{145.7079689236111pt}{108.6907943576389pt}}
\pgflineto{\pgfpoint{143.32174184027775pt}{106.95710034722221pt}}
\pgflineto{\pgfpoint{144.4483626736111pt}{106.95710034722221pt}}
\pgfpathclose
\pgfusepath{fill,stroke}
\pgfpathmoveto{\pgfpoint{145.7079689236111pt}{108.6907943576389pt}}
\pgflineto{\pgfpoint{142.41028350694444pt}{107.61931310763887pt}}
\pgflineto{\pgfpoint{143.32174184027775pt}{106.95710034722221pt}}
\pgfpathclose
\pgfusepath{fill,stroke}
\pgfpathmoveto{\pgfpoint{145.7079689236111pt}{108.6907943576389pt}}
\pgflineto{\pgfpoint{142.06213559027776pt}{108.6907943576389pt}}
\pgflineto{\pgfpoint{142.41028350694444pt}{107.61931310763887pt}}
\pgfpathclose
\pgfusepath{fill,stroke}
\pgfpathmoveto{\pgfpoint{145.7079689236111pt}{108.6907943576389pt}}
\pgflineto{\pgfpoint{142.41028350694444pt}{109.76228472222222pt}}
\pgflineto{\pgfpoint{142.06213559027776pt}{108.6907943576389pt}}
\pgfpathclose
\pgfusepath{fill,stroke}
\pgfpathmoveto{\pgfpoint{145.7079689236111pt}{108.6907943576389pt}}
\pgflineto{\pgfpoint{143.32174184027775pt}{110.42449748263887pt}}
\pgflineto{\pgfpoint{142.41028350694444pt}{109.76228472222222pt}}
\pgfpathclose
\pgfusepath{fill,stroke}
\pgfpathmoveto{\pgfpoint{145.7079689236111pt}{108.6907943576389pt}}
\pgflineto{\pgfpoint{144.4483626736111pt}{110.42449748263887pt}}
\pgflineto{\pgfpoint{143.32174184027775pt}{110.42449748263887pt}}
\pgfpathclose
\pgfusepath{fill,stroke}
\pgfpathmoveto{\pgfpoint{145.7079689236111pt}{108.6907943576389pt}}
\pgflineto{\pgfpoint{145.35982100694443pt}{109.76228472222222pt}}
\pgflineto{\pgfpoint{144.4483626736111pt}{110.42449748263887pt}}
\pgfpathclose
\pgfusepath{fill,stroke}
\pgfpathmoveto{\pgfpoint{94.53020998263888pt}{170.87173480902777pt}}
\pgflineto{\pgfpoint{93.27060373263889pt}{169.13804991319444pt}}
\pgflineto{\pgfpoint{94.18206206597222pt}{169.80024444444442pt}}
\pgfpathclose
\pgfusepath{fill,stroke}
\pgfpathmoveto{\pgfpoint{94.53020998263888pt}{170.87173480902777pt}}
\pgflineto{\pgfpoint{92.14398350694444pt}{169.13804991319444pt}}
\pgflineto{\pgfpoint{93.27060373263889pt}{169.13804991319444pt}}
\pgfpathclose
\pgfusepath{fill,stroke}
\pgfpathmoveto{\pgfpoint{94.53020998263888pt}{170.87173480902777pt}}
\pgflineto{\pgfpoint{91.23252517361111pt}{169.80024444444442pt}}
\pgflineto{\pgfpoint{92.14398350694444pt}{169.13804991319444pt}}
\pgfpathclose
\pgfusepath{fill,stroke}
\pgfpathmoveto{\pgfpoint{94.53020998263888pt}{170.87173480902777pt}}
\pgflineto{\pgfpoint{90.88437664930555pt}{170.87173480902777pt}}
\pgflineto{\pgfpoint{91.23252517361111pt}{169.80024444444442pt}}
\pgfpathclose
\pgfusepath{fill,stroke}
\pgfpathmoveto{\pgfpoint{94.53020998263888pt}{170.87173480902777pt}}
\pgflineto{\pgfpoint{91.23252517361111pt}{171.9432245659722pt}}
\pgflineto{\pgfpoint{90.88437664930555pt}{170.87173480902777pt}}
\pgfpathclose
\pgfusepath{fill,stroke}
\pgfpathmoveto{\pgfpoint{94.53020998263888pt}{170.87173480902777pt}}
\pgflineto{\pgfpoint{92.14398350694444pt}{172.6054190972222pt}}
\pgflineto{\pgfpoint{91.23252517361111pt}{171.9432245659722pt}}
\pgfpathclose
\pgfusepath{fill,stroke}
\pgfpathmoveto{\pgfpoint{94.53020998263888pt}{170.87173480902777pt}}
\pgflineto{\pgfpoint{93.27060373263889pt}{172.6054190972222pt}}
\pgflineto{\pgfpoint{92.14398350694444pt}{172.6054190972222pt}}
\pgfpathclose
\pgfusepath{fill,stroke}
\pgfpathmoveto{\pgfpoint{94.53020998263888pt}{170.87173480902777pt}}
\pgflineto{\pgfpoint{94.18206206597222pt}{171.9432245659722pt}}
\pgflineto{\pgfpoint{93.27060373263889pt}{172.6054190972222pt}}
\pgfpathclose
\pgfusepath{fill,stroke}
\pgfpathmoveto{\pgfpoint{178.31612664930555pt}{124.29112743055555pt}}
\pgflineto{\pgfpoint{177.05652951388888pt}{122.55742430555554pt}}
\pgflineto{\pgfpoint{177.9679878472222pt}{123.21964618055554pt}}
\pgfpathclose
\pgfusepath{fill,stroke}
\pgfpathmoveto{\pgfpoint{178.31612664930555pt}{124.29112743055555pt}}
\pgflineto{\pgfpoint{175.92990928819444pt}{122.55742430555554pt}}
\pgflineto{\pgfpoint{177.05652951388888pt}{122.55742430555554pt}}
\pgfpathclose
\pgfusepath{fill,stroke}
\pgfpathmoveto{\pgfpoint{178.31612664930555pt}{124.29112743055555pt}}
\pgflineto{\pgfpoint{175.0184509548611pt}{123.21964618055554pt}}
\pgflineto{\pgfpoint{175.92990928819444pt}{122.55742430555554pt}}
\pgfpathclose
\pgfusepath{fill,stroke}
\pgfpathmoveto{\pgfpoint{178.31612664930555pt}{124.29112743055555pt}}
\pgflineto{\pgfpoint{174.67029331597223pt}{124.29112743055555pt}}
\pgflineto{\pgfpoint{175.0184509548611pt}{123.21964618055554pt}}
\pgfpathclose
\pgfusepath{fill,stroke}
\pgfpathmoveto{\pgfpoint{178.31612664930555pt}{124.29112743055555pt}}
\pgflineto{\pgfpoint{175.0184509548611pt}{125.36260807291666pt}}
\pgflineto{\pgfpoint{174.67029331597223pt}{124.29112743055555pt}}
\pgfpathclose
\pgfusepath{fill,stroke}
\pgfpathmoveto{\pgfpoint{178.31612664930555pt}{124.29112743055555pt}}
\pgflineto{\pgfpoint{175.92990928819444pt}{126.02482083333332pt}}
\pgflineto{\pgfpoint{175.0184509548611pt}{125.36260807291666pt}}
\pgfpathclose
\pgfusepath{fill,stroke}
\pgfpathmoveto{\pgfpoint{178.31612664930555pt}{124.29112743055555pt}}
\pgflineto{\pgfpoint{177.05652951388888pt}{126.02482083333332pt}}
\pgflineto{\pgfpoint{175.92990928819444pt}{126.02482083333332pt}}
\pgfpathclose
\pgfusepath{fill,stroke}
\pgfpathmoveto{\pgfpoint{178.31612664930555pt}{124.29112743055555pt}}
\pgflineto{\pgfpoint{177.9679878472222pt}{125.36260807291666pt}}
\pgflineto{\pgfpoint{177.05652951388888pt}{126.02482083333332pt}}
\pgfpathclose
\pgfusepath{fill,stroke}
\pgfpathmoveto{\pgfpoint{99.46956614583333pt}{163.0086822048611pt}}
\pgflineto{\pgfpoint{98.20995078124999pt}{161.27497907986108pt}}
\pgflineto{\pgfpoint{99.12141822916665pt}{161.93719184027776pt}}
\pgfpathclose
\pgfusepath{fill,stroke}
\pgfpathmoveto{\pgfpoint{99.46956614583333pt}{163.0086822048611pt}}
\pgflineto{\pgfpoint{97.08333967013887pt}{161.27497907986108pt}}
\pgflineto{\pgfpoint{98.20995078124999pt}{161.27497907986108pt}}
\pgfpathclose
\pgfusepath{fill,stroke}
\pgfpathmoveto{\pgfpoint{99.46956614583333pt}{163.0086822048611pt}}
\pgflineto{\pgfpoint{96.17187161458332pt}{161.93719184027776pt}}
\pgflineto{\pgfpoint{97.08333967013887pt}{161.27497907986108pt}}
\pgfpathclose
\pgfusepath{fill,stroke}
\pgfpathmoveto{\pgfpoint{99.46956614583333pt}{163.0086822048611pt}}
\pgflineto{\pgfpoint{95.8237328125pt}{163.0086822048611pt}}
\pgflineto{\pgfpoint{96.17187161458332pt}{161.93719184027776pt}}
\pgfpathclose
\pgfusepath{fill,stroke}
\pgfpathmoveto{\pgfpoint{99.46956614583333pt}{163.0086822048611pt}}
\pgflineto{\pgfpoint{96.17187161458332pt}{164.08017196180555pt}}
\pgflineto{\pgfpoint{95.8237328125pt}{163.0086822048611pt}}
\pgfpathclose
\pgfusepath{fill,stroke}
\pgfpathmoveto{\pgfpoint{99.46956614583333pt}{163.0086822048611pt}}
\pgflineto{\pgfpoint{97.08333967013887pt}{164.74238472222223pt}}
\pgflineto{\pgfpoint{96.17187161458332pt}{164.08017196180555pt}}
\pgfpathclose
\pgfusepath{fill,stroke}
\pgfpathmoveto{\pgfpoint{99.46956614583333pt}{163.0086822048611pt}}
\pgflineto{\pgfpoint{98.20995078124999pt}{164.74238472222223pt}}
\pgflineto{\pgfpoint{97.08333967013887pt}{164.74238472222223pt}}
\pgfpathclose
\pgfusepath{fill,stroke}
\pgfpathmoveto{\pgfpoint{99.46956614583333pt}{163.0086822048611pt}}
\pgflineto{\pgfpoint{99.12141822916665pt}{164.08017196180555pt}}
\pgflineto{\pgfpoint{98.20995078124999pt}{164.74238472222223pt}}
\pgfpathclose
\pgfusepath{fill,stroke}
\pgfpathmoveto{\pgfpoint{110.58108967013888pt}{116.67066310763887pt}}
\pgflineto{\pgfpoint{109.3214828125pt}{114.93696909722222pt}}
\pgflineto{\pgfpoint{110.23294114583332pt}{115.59918185763888pt}}
\pgfpathclose
\pgfusepath{fill,stroke}
\pgfpathmoveto{\pgfpoint{110.58108967013888pt}{116.67066310763887pt}}
\pgflineto{\pgfpoint{108.19486258680554pt}{114.93696909722222pt}}
\pgflineto{\pgfpoint{109.3214828125pt}{114.93696909722222pt}}
\pgfpathclose
\pgfusepath{fill,stroke}
\pgfpathmoveto{\pgfpoint{110.58108967013888pt}{116.67066310763887pt}}
\pgflineto{\pgfpoint{107.28340425347221pt}{115.59918185763888pt}}
\pgflineto{\pgfpoint{108.19486258680554pt}{114.93696909722222pt}}
\pgfpathclose
\pgfusepath{fill,stroke}
\pgfpathmoveto{\pgfpoint{110.58108967013888pt}{116.67066310763887pt}}
\pgflineto{\pgfpoint{106.93526545138887pt}{116.67066310763887pt}}
\pgflineto{\pgfpoint{107.28340425347221pt}{115.59918185763888pt}}
\pgfpathclose
\pgfusepath{fill,stroke}
\pgfpathmoveto{\pgfpoint{110.58108967013888pt}{116.67066310763887pt}}
\pgflineto{\pgfpoint{107.28340425347221pt}{117.74214374999998pt}}
\pgflineto{\pgfpoint{106.93526545138887pt}{116.67066310763887pt}}
\pgfpathclose
\pgfusepath{fill,stroke}
\pgfpathmoveto{\pgfpoint{110.58108967013888pt}{116.67066310763887pt}}
\pgflineto{\pgfpoint{108.19486258680554pt}{118.404365625pt}}
\pgflineto{\pgfpoint{107.28340425347221pt}{117.74214374999998pt}}
\pgfpathclose
\pgfusepath{fill,stroke}
\pgfpathmoveto{\pgfpoint{110.58108967013888pt}{116.67066310763887pt}}
\pgflineto{\pgfpoint{109.3214828125pt}{118.404365625pt}}
\pgflineto{\pgfpoint{108.19486258680554pt}{118.404365625pt}}
\pgfpathclose
\pgfusepath{fill,stroke}
\pgfpathmoveto{\pgfpoint{110.58108967013888pt}{116.67066310763887pt}}
\pgflineto{\pgfpoint{110.23294114583332pt}{117.74214374999998pt}}
\pgflineto{\pgfpoint{109.3214828125pt}{118.404365625pt}}
\pgfpathclose
\pgfusepath{fill,stroke}
\pgfpathmoveto{\pgfpoint{127.05087925347222pt}{120.87668177083333pt}}
\pgflineto{\pgfpoint{125.79126328125pt}{119.14298776041666pt}}
\pgflineto{\pgfpoint{126.70273133680554pt}{119.80520112847222pt}}
\pgfpathclose
\pgfusepath{fill,stroke}
\pgfpathmoveto{\pgfpoint{127.05087925347222pt}{120.87668177083333pt}}
\pgflineto{\pgfpoint{124.66464305555554pt}{119.14298776041666pt}}
\pgflineto{\pgfpoint{125.79126328125pt}{119.14298776041666pt}}
\pgfpathclose
\pgfusepath{fill,stroke}
\pgfpathmoveto{\pgfpoint{127.05087925347222pt}{120.87668177083333pt}}
\pgflineto{\pgfpoint{123.75318472222222pt}{119.80520112847222pt}}
\pgflineto{\pgfpoint{124.66464305555554pt}{119.14298776041666pt}}
\pgfpathclose
\pgfusepath{fill,stroke}
\pgfpathmoveto{\pgfpoint{127.05087925347222pt}{120.87668177083333pt}}
\pgflineto{\pgfpoint{123.40504592013889pt}{120.87668177083333pt}}
\pgflineto{\pgfpoint{123.75318472222222pt}{119.80520112847222pt}}
\pgfpathclose
\pgfusepath{fill,stroke}
\pgfpathmoveto{\pgfpoint{127.05087925347222pt}{120.87668177083333pt}}
\pgflineto{\pgfpoint{123.75318472222222pt}{121.94816241319444pt}}
\pgflineto{\pgfpoint{123.40504592013889pt}{120.87668177083333pt}}
\pgfpathclose
\pgfusepath{fill,stroke}
\pgfpathmoveto{\pgfpoint{127.05087925347222pt}{120.87668177083333pt}}
\pgflineto{\pgfpoint{124.66464305555554pt}{122.61037578124999pt}}
\pgflineto{\pgfpoint{123.75318472222222pt}{121.94816241319444pt}}
\pgfpathclose
\pgfusepath{fill,stroke}
\pgfpathmoveto{\pgfpoint{127.05087925347222pt}{120.87668177083333pt}}
\pgflineto{\pgfpoint{125.79126328125pt}{122.61037578124999pt}}
\pgflineto{\pgfpoint{124.66464305555554pt}{122.61037578124999pt}}
\pgfpathclose
\pgfusepath{fill,stroke}
\pgfpathmoveto{\pgfpoint{127.05087925347222pt}{120.87668177083333pt}}
\pgflineto{\pgfpoint{126.70273133680554pt}{121.94816241319444pt}}
\pgflineto{\pgfpoint{125.79126328125pt}{122.61037578124999pt}}
\pgfpathclose
\pgfusepath{fill,stroke}
\pgfpathmoveto{\pgfpoint{138.60516918402777pt}{118.93839574652777pt}}
\pgflineto{\pgfpoint{137.34556293402775pt}{117.20470173611109pt}}
\pgflineto{\pgfpoint{138.2570212673611pt}{117.86691510416667pt}}
\pgfpathclose
\pgfusepath{fill,stroke}
\pgfpathmoveto{\pgfpoint{138.60516918402777pt}{118.93839574652777pt}}
\pgflineto{\pgfpoint{136.21894270833332pt}{117.20470173611109pt}}
\pgflineto{\pgfpoint{137.34556293402775pt}{117.20470173611109pt}}
\pgfpathclose
\pgfusepath{fill,stroke}
\pgfpathmoveto{\pgfpoint{138.60516918402777pt}{118.93839574652777pt}}
\pgflineto{\pgfpoint{135.30748437499997pt}{117.86691510416667pt}}
\pgflineto{\pgfpoint{136.21894270833332pt}{117.20470173611109pt}}
\pgfpathclose
\pgfusepath{fill,stroke}
\pgfpathmoveto{\pgfpoint{138.60516918402777pt}{118.93839574652777pt}}
\pgflineto{\pgfpoint{134.95933585069443pt}{118.93839574652777pt}}
\pgflineto{\pgfpoint{135.30748437499997pt}{117.86691510416667pt}}
\pgfpathclose
\pgfusepath{fill,stroke}
\pgfpathmoveto{\pgfpoint{138.60516918402777pt}{118.93839574652777pt}}
\pgflineto{\pgfpoint{135.30748437499997pt}{120.00987638888888pt}}
\pgflineto{\pgfpoint{134.95933585069443pt}{118.93839574652777pt}}
\pgfpathclose
\pgfusepath{fill,stroke}
\pgfpathmoveto{\pgfpoint{138.60516918402777pt}{118.93839574652777pt}}
\pgflineto{\pgfpoint{136.21894270833332pt}{120.67209887152778pt}}
\pgflineto{\pgfpoint{135.30748437499997pt}{120.00987638888888pt}}
\pgfpathclose
\pgfusepath{fill,stroke}
\pgfpathmoveto{\pgfpoint{138.60516918402777pt}{118.93839574652777pt}}
\pgflineto{\pgfpoint{137.34556293402775pt}{120.67209887152778pt}}
\pgflineto{\pgfpoint{136.21894270833332pt}{120.67209887152778pt}}
\pgfpathclose
\pgfusepath{fill,stroke}
\pgfpathmoveto{\pgfpoint{138.60516918402777pt}{118.93839574652777pt}}
\pgflineto{\pgfpoint{138.2570212673611pt}{120.00987638888888pt}}
\pgflineto{\pgfpoint{137.34556293402775pt}{120.67209887152778pt}}
\pgfpathclose
\pgfusepath{fill,stroke}
\pgfpathmoveto{\pgfpoint{134.77936050347222pt}{118.61750711805554pt}}
\pgflineto{\pgfpoint{133.51976336805555pt}{116.88381310763887pt}}
\pgflineto{\pgfpoint{134.43122170138886pt}{117.54602586805554pt}}
\pgfpathclose
\pgfusepath{fill,stroke}
\pgfpathmoveto{\pgfpoint{134.77936050347222pt}{118.61750711805554pt}}
\pgflineto{\pgfpoint{132.39314314236108pt}{116.88381310763887pt}}
\pgflineto{\pgfpoint{133.51976336805555pt}{116.88381310763887pt}}
\pgfpathclose
\pgfusepath{fill,stroke}
\pgfpathmoveto{\pgfpoint{134.77936050347222pt}{118.61750711805554pt}}
\pgflineto{\pgfpoint{131.48168480902777pt}{117.54602586805554pt}}
\pgflineto{\pgfpoint{132.39314314236108pt}{116.88381310763887pt}}
\pgfpathclose
\pgfusepath{fill,stroke}
\pgfpathmoveto{\pgfpoint{134.77936050347222pt}{118.61750711805554pt}}
\pgflineto{\pgfpoint{131.13353628472223pt}{118.61750711805554pt}}
\pgflineto{\pgfpoint{131.48168480902777pt}{117.54602586805554pt}}
\pgfpathclose
\pgfusepath{fill,stroke}
\pgfpathmoveto{\pgfpoint{134.77936050347222pt}{118.61750711805554pt}}
\pgflineto{\pgfpoint{131.48168480902777pt}{119.68898776041665pt}}
\pgflineto{\pgfpoint{131.13353628472223pt}{118.61750711805554pt}}
\pgfpathclose
\pgfusepath{fill,stroke}
\pgfpathmoveto{\pgfpoint{134.77936050347222pt}{118.61750711805554pt}}
\pgflineto{\pgfpoint{132.39314314236108pt}{120.35120052083333pt}}
\pgflineto{\pgfpoint{131.48168480902777pt}{119.68898776041665pt}}
\pgfpathclose
\pgfusepath{fill,stroke}
\pgfpathmoveto{\pgfpoint{134.77936050347222pt}{118.61750711805554pt}}
\pgflineto{\pgfpoint{133.51976336805555pt}{120.35120052083333pt}}
\pgflineto{\pgfpoint{132.39314314236108pt}{120.35120052083333pt}}
\pgfpathclose
\pgfusepath{fill,stroke}
\pgfpathmoveto{\pgfpoint{134.77936050347222pt}{118.61750711805554pt}}
\pgflineto{\pgfpoint{134.43122170138886pt}{119.68898776041665pt}}
\pgflineto{\pgfpoint{133.51976336805555pt}{120.35120052083333pt}}
\pgfpathclose
\pgfusepath{fill,stroke}
\pgfpathmoveto{\pgfpoint{135.20358420138888pt}{124.14991701388888pt}}
\pgflineto{\pgfpoint{133.94396822916667pt}{122.4162236111111pt}}
\pgflineto{\pgfpoint{134.85542656249999pt}{123.07843637152777pt}}
\pgfpathclose
\pgfusepath{fill,stroke}
\pgfpathmoveto{\pgfpoint{135.20358420138888pt}{124.14991701388888pt}}
\pgflineto{\pgfpoint{132.8173480034722pt}{122.4162236111111pt}}
\pgflineto{\pgfpoint{133.94396822916667pt}{122.4162236111111pt}}
\pgfpathclose
\pgfusepath{fill,stroke}
\pgfpathmoveto{\pgfpoint{135.20358420138888pt}{124.14991701388888pt}}
\pgflineto{\pgfpoint{131.9058896701389pt}{123.07843637152777pt}}
\pgflineto{\pgfpoint{132.8173480034722pt}{122.4162236111111pt}}
\pgfpathclose
\pgfusepath{fill,stroke}
\pgfpathmoveto{\pgfpoint{135.20358420138888pt}{124.14991701388888pt}}
\pgflineto{\pgfpoint{131.55774114583332pt}{124.14991701388888pt}}
\pgflineto{\pgfpoint{131.9058896701389pt}{123.07843637152777pt}}
\pgfpathclose
\pgfusepath{fill,stroke}
\pgfpathmoveto{\pgfpoint{135.20358420138888pt}{124.14991701388888pt}}
\pgflineto{\pgfpoint{131.9058896701389pt}{125.22139826388887pt}}
\pgflineto{\pgfpoint{131.55774114583332pt}{124.14991701388888pt}}
\pgfpathclose
\pgfusepath{fill,stroke}
\pgfpathmoveto{\pgfpoint{135.20358420138888pt}{124.14991701388888pt}}
\pgflineto{\pgfpoint{132.8173480034722pt}{125.88361102430554pt}}
\pgflineto{\pgfpoint{131.9058896701389pt}{125.22139826388887pt}}
\pgfpathclose
\pgfusepath{fill,stroke}
\pgfpathmoveto{\pgfpoint{135.20358420138888pt}{124.14991701388888pt}}
\pgflineto{\pgfpoint{133.94396822916667pt}{125.88361102430554pt}}
\pgflineto{\pgfpoint{132.8173480034722pt}{125.88361102430554pt}}
\pgfpathclose
\pgfusepath{fill,stroke}
\pgfpathmoveto{\pgfpoint{135.20358420138888pt}{124.14991701388888pt}}
\pgflineto{\pgfpoint{134.85542656249999pt}{125.22139826388887pt}}
\pgflineto{\pgfpoint{133.94396822916667pt}{125.88361102430554pt}}
\pgfpathclose
\pgfusepath{fill,stroke}
\pgfpathmoveto{\pgfpoint{128.77951041666665pt}{118.52336987847221pt}}
\pgflineto{\pgfpoint{127.51990416666665pt}{116.78966675347222pt}}
\pgflineto{\pgfpoint{128.43136249999998pt}{117.45187951388888pt}}
\pgfpathclose
\pgfusepath{fill,stroke}
\pgfpathmoveto{\pgfpoint{128.77951041666665pt}{118.52336987847221pt}}
\pgflineto{\pgfpoint{126.39328394097222pt}{116.78966675347222pt}}
\pgflineto{\pgfpoint{127.51990416666665pt}{116.78966675347222pt}}
\pgfpathclose
\pgfusepath{fill,stroke}
\pgfpathmoveto{\pgfpoint{128.77951041666665pt}{118.52336987847221pt}}
\pgflineto{\pgfpoint{125.48182560763888pt}{117.45187951388888pt}}
\pgflineto{\pgfpoint{126.39328394097222pt}{116.78966675347222pt}}
\pgfpathclose
\pgfusepath{fill,stroke}
\pgfpathmoveto{\pgfpoint{128.77951041666665pt}{118.52336987847221pt}}
\pgflineto{\pgfpoint{125.13367708333332pt}{118.52336987847221pt}}
\pgflineto{\pgfpoint{125.48182560763888pt}{117.45187951388888pt}}
\pgfpathclose
\pgfusepath{fill,stroke}
\pgfpathmoveto{\pgfpoint{128.77951041666665pt}{118.52336987847221pt}}
\pgflineto{\pgfpoint{125.48182560763888pt}{119.59485052083332pt}}
\pgflineto{\pgfpoint{125.13367708333332pt}{118.52336987847221pt}}
\pgfpathclose
\pgfusepath{fill,stroke}
\pgfpathmoveto{\pgfpoint{128.77951041666665pt}{118.52336987847221pt}}
\pgflineto{\pgfpoint{126.39328394097222pt}{120.25706388888888pt}}
\pgflineto{\pgfpoint{125.48182560763888pt}{119.59485052083332pt}}
\pgfpathclose
\pgfusepath{fill,stroke}
\pgfpathmoveto{\pgfpoint{128.77951041666665pt}{118.52336987847221pt}}
\pgflineto{\pgfpoint{127.51990416666665pt}{120.25706388888888pt}}
\pgflineto{\pgfpoint{126.39328394097222pt}{120.25706388888888pt}}
\pgfpathclose
\pgfusepath{fill,stroke}
\pgfpathmoveto{\pgfpoint{128.77951041666665pt}{118.52336987847221pt}}
\pgflineto{\pgfpoint{128.43136249999998pt}{119.59485052083332pt}}
\pgflineto{\pgfpoint{127.51990416666665pt}{120.25706388888888pt}}
\pgfpathclose
\pgfusepath{fill,stroke}
\pgfpathmoveto{\pgfpoint{106.17730217013889pt}{121.5441609375pt}}
\pgflineto{\pgfpoint{104.91769592013887pt}{119.81046753472222pt}}
\pgflineto{\pgfpoint{105.82915425347221pt}{120.47268029513887pt}}
\pgfpathclose
\pgfusepath{fill,stroke}
\pgfpathmoveto{\pgfpoint{106.17730217013889pt}{121.5441609375pt}}
\pgflineto{\pgfpoint{103.79107508680553pt}{119.81046753472222pt}}
\pgflineto{\pgfpoint{104.91769592013887pt}{119.81046753472222pt}}
\pgfpathclose
\pgfusepath{fill,stroke}
\pgfpathmoveto{\pgfpoint{106.17730217013889pt}{121.5441609375pt}}
\pgflineto{\pgfpoint{102.8796167534722pt}{120.47268029513887pt}}
\pgflineto{\pgfpoint{103.79107508680553pt}{119.81046753472222pt}}
\pgfpathclose
\pgfusepath{fill,stroke}
\pgfpathmoveto{\pgfpoint{106.17730217013889pt}{121.5441609375pt}}
\pgflineto{\pgfpoint{102.53147795138888pt}{121.5441609375pt}}
\pgflineto{\pgfpoint{102.8796167534722pt}{120.47268029513887pt}}
\pgfpathclose
\pgfusepath{fill,stroke}
\pgfpathmoveto{\pgfpoint{106.17730217013889pt}{121.5441609375pt}}
\pgflineto{\pgfpoint{102.8796167534722pt}{122.61565130208334pt}}
\pgflineto{\pgfpoint{102.53147795138888pt}{121.5441609375pt}}
\pgfpathclose
\pgfusepath{fill,stroke}
\pgfpathmoveto{\pgfpoint{106.17730217013889pt}{121.5441609375pt}}
\pgflineto{\pgfpoint{103.79107508680553pt}{123.2778640625pt}}
\pgflineto{\pgfpoint{102.8796167534722pt}{122.61565130208334pt}}
\pgfpathclose
\pgfusepath{fill,stroke}
\pgfpathmoveto{\pgfpoint{106.17730217013889pt}{121.5441609375pt}}
\pgflineto{\pgfpoint{104.91769592013887pt}{123.2778640625pt}}
\pgflineto{\pgfpoint{103.79107508680553pt}{123.2778640625pt}}
\pgfpathclose
\pgfusepath{fill,stroke}
\pgfpathmoveto{\pgfpoint{106.17730217013889pt}{121.5441609375pt}}
\pgflineto{\pgfpoint{105.82915425347221pt}{122.61565130208334pt}}
\pgflineto{\pgfpoint{104.91769592013887pt}{123.2778640625pt}}
\pgfpathclose
\pgfusepath{fill,stroke}
\pgfpathmoveto{\pgfpoint{87.25243272569445pt}{161.79865199652778pt}}
\pgflineto{\pgfpoint{85.99282647569443pt}{160.06494887152778pt}}
\pgflineto{\pgfpoint{86.90428480902777pt}{160.72716163194443pt}}
\pgfpathclose
\pgfusepath{fill,stroke}
\pgfpathmoveto{\pgfpoint{87.25243272569445pt}{161.79865199652778pt}}
\pgflineto{\pgfpoint{84.86619652777777pt}{160.06494887152778pt}}
\pgflineto{\pgfpoint{85.99282647569443pt}{160.06494887152778pt}}
\pgfpathclose
\pgfusepath{fill,stroke}
\pgfpathmoveto{\pgfpoint{87.25243272569445pt}{161.79865199652778pt}}
\pgflineto{\pgfpoint{83.95474730902777pt}{160.72716163194443pt}}
\pgflineto{\pgfpoint{84.86619652777777pt}{160.06494887152778pt}}
\pgfpathclose
\pgfusepath{fill,stroke}
\pgfpathmoveto{\pgfpoint{87.25243272569445pt}{161.79865199652778pt}}
\pgflineto{\pgfpoint{83.60659027777777pt}{161.79865199652778pt}}
\pgflineto{\pgfpoint{83.95474730902777pt}{160.72716163194443pt}}
\pgfpathclose
\pgfusepath{fill,stroke}
\pgfpathmoveto{\pgfpoint{87.25243272569445pt}{161.79865199652778pt}}
\pgflineto{\pgfpoint{83.95474730902777pt}{162.87014236111108pt}}
\pgflineto{\pgfpoint{83.60659027777777pt}{161.79865199652778pt}}
\pgfpathclose
\pgfusepath{fill,stroke}
\pgfpathmoveto{\pgfpoint{87.25243272569445pt}{161.79865199652778pt}}
\pgflineto{\pgfpoint{84.86619652777777pt}{163.5323362847222pt}}
\pgflineto{\pgfpoint{83.95474730902777pt}{162.87014236111108pt}}
\pgfpathclose
\pgfusepath{fill,stroke}
\pgfpathmoveto{\pgfpoint{87.25243272569445pt}{161.79865199652778pt}}
\pgflineto{\pgfpoint{85.99282647569443pt}{163.5323362847222pt}}
\pgflineto{\pgfpoint{84.86619652777777pt}{163.5323362847222pt}}
\pgfpathclose
\pgfusepath{fill,stroke}
\pgfpathmoveto{\pgfpoint{87.25243272569445pt}{161.79865199652778pt}}
\pgflineto{\pgfpoint{86.90428480902777pt}{162.87014236111108pt}}
\pgflineto{\pgfpoint{85.99282647569443pt}{163.5323362847222pt}}
\pgfpathclose
\pgfusepath{fill,stroke}
\pgfpathmoveto{\pgfpoint{110.31065755208333pt}{119.28499539930554pt}}
\pgflineto{\pgfpoint{109.05106041666666pt}{117.55129227430554pt}}
\pgflineto{\pgfpoint{109.96251875pt}{118.21350503472222pt}}
\pgfpathclose
\pgfusepath{fill,stroke}
\pgfpathmoveto{\pgfpoint{110.31065755208333pt}{119.28499539930554pt}}
\pgflineto{\pgfpoint{107.92444019097222pt}{117.55129227430554pt}}
\pgflineto{\pgfpoint{109.05106041666666pt}{117.55129227430554pt}}
\pgfpathclose
\pgfusepath{fill,stroke}
\pgfpathmoveto{\pgfpoint{110.31065755208333pt}{119.28499539930554pt}}
\pgflineto{\pgfpoint{107.01297274305554pt}{118.21350503472222pt}}
\pgflineto{\pgfpoint{107.92444019097222pt}{117.55129227430554pt}}
\pgfpathclose
\pgfusepath{fill,stroke}
\pgfpathmoveto{\pgfpoint{110.31065755208333pt}{119.28499539930554pt}}
\pgflineto{\pgfpoint{106.66482421875pt}{119.28499539930554pt}}
\pgflineto{\pgfpoint{107.01297274305554pt}{118.21350503472222pt}}
\pgfpathclose
\pgfusepath{fill,stroke}
\pgfpathmoveto{\pgfpoint{110.31065755208333pt}{119.28499539930554pt}}
\pgflineto{\pgfpoint{107.01297274305554pt}{120.35647604166665pt}}
\pgflineto{\pgfpoint{106.66482421875pt}{119.28499539930554pt}}
\pgfpathclose
\pgfusepath{fill,stroke}
\pgfpathmoveto{\pgfpoint{110.31065755208333pt}{119.28499539930554pt}}
\pgflineto{\pgfpoint{107.92444019097222pt}{121.01868940972221pt}}
\pgflineto{\pgfpoint{107.01297274305554pt}{120.35647604166665pt}}
\pgfpathclose
\pgfusepath{fill,stroke}
\pgfpathmoveto{\pgfpoint{110.31065755208333pt}{119.28499539930554pt}}
\pgflineto{\pgfpoint{109.05106041666666pt}{121.01868940972221pt}}
\pgflineto{\pgfpoint{107.92444019097222pt}{121.01868940972221pt}}
\pgfpathclose
\pgfusepath{fill,stroke}
\pgfpathmoveto{\pgfpoint{110.31065755208333pt}{119.28499539930554pt}}
\pgflineto{\pgfpoint{109.96251875pt}{120.35647604166665pt}}
\pgflineto{\pgfpoint{109.05106041666666pt}{121.01868940972221pt}}
\pgfpathclose
\pgfusepath{fill,stroke}
\pgfpathmoveto{\pgfpoint{125.8763509548611pt}{120.39318350694442pt}}
\pgflineto{\pgfpoint{124.6167447048611pt}{118.65948038194445pt}}
\pgflineto{\pgfpoint{125.52820303819445pt}{119.32170286458332pt}}
\pgfpathclose
\pgfusepath{fill,stroke}
\pgfpathmoveto{\pgfpoint{125.8763509548611pt}{120.39318350694442pt}}
\pgflineto{\pgfpoint{123.49012447916667pt}{118.65948038194445pt}}
\pgflineto{\pgfpoint{124.6167447048611pt}{118.65948038194445pt}}
\pgfpathclose
\pgfusepath{fill,stroke}
\pgfpathmoveto{\pgfpoint{125.8763509548611pt}{120.39318350694442pt}}
\pgflineto{\pgfpoint{122.57866614583332pt}{119.32170286458332pt}}
\pgflineto{\pgfpoint{123.49012447916667pt}{118.65948038194445pt}}
\pgfpathclose
\pgfusepath{fill,stroke}
\pgfpathmoveto{\pgfpoint{125.8763509548611pt}{120.39318350694442pt}}
\pgflineto{\pgfpoint{122.23052734374998pt}{120.39318350694442pt}}
\pgflineto{\pgfpoint{122.57866614583332pt}{119.32170286458332pt}}
\pgfpathclose
\pgfusepath{fill,stroke}
\pgfpathmoveto{\pgfpoint{125.8763509548611pt}{120.39318350694442pt}}
\pgflineto{\pgfpoint{122.57866614583332pt}{121.46466414930553pt}}
\pgflineto{\pgfpoint{122.23052734374998pt}{120.39318350694442pt}}
\pgfpathclose
\pgfusepath{fill,stroke}
\pgfpathmoveto{\pgfpoint{125.8763509548611pt}{120.39318350694442pt}}
\pgflineto{\pgfpoint{123.49012447916667pt}{122.12687751736111pt}}
\pgflineto{\pgfpoint{122.57866614583332pt}{121.46466414930553pt}}
\pgfpathclose
\pgfusepath{fill,stroke}
\pgfpathmoveto{\pgfpoint{125.8763509548611pt}{120.39318350694442pt}}
\pgflineto{\pgfpoint{124.6167447048611pt}{122.12687751736111pt}}
\pgflineto{\pgfpoint{123.49012447916667pt}{122.12687751736111pt}}
\pgfpathclose
\pgfusepath{fill,stroke}
\pgfpathmoveto{\pgfpoint{125.8763509548611pt}{120.39318350694442pt}}
\pgflineto{\pgfpoint{125.52820303819445pt}{121.46466414930553pt}}
\pgflineto{\pgfpoint{124.6167447048611pt}{122.12687751736111pt}}
\pgfpathclose
\pgfusepath{fill,stroke}
\pgfpathmoveto{\pgfpoint{85.587425pt}{112.92248697916666pt}}
\pgflineto{\pgfpoint{84.32782725694443pt}{111.18879296875pt}}
\pgflineto{\pgfpoint{85.23927647569444pt}{111.85100633680555pt}}
\pgfpathclose
\pgfusepath{fill,stroke}
\pgfpathmoveto{\pgfpoint{85.587425pt}{112.92248697916666pt}}
\pgflineto{\pgfpoint{83.20118880208334pt}{111.18879296875pt}}
\pgflineto{\pgfpoint{84.32782725694443pt}{111.18879296875pt}}
\pgfpathclose
\pgfusepath{fill,stroke}
\pgfpathmoveto{\pgfpoint{85.587425pt}{112.92248697916666pt}}
\pgflineto{\pgfpoint{82.28973958333333pt}{111.85100633680555pt}}
\pgflineto{\pgfpoint{83.20118880208334pt}{111.18879296875pt}}
\pgfpathclose
\pgfusepath{fill,stroke}
\pgfpathmoveto{\pgfpoint{85.587425pt}{112.92248697916666pt}}
\pgflineto{\pgfpoint{81.94159166666667pt}{112.92248697916666pt}}
\pgflineto{\pgfpoint{82.28973958333333pt}{111.85100633680555pt}}
\pgfpathclose
\pgfusepath{fill,stroke}
\pgfpathmoveto{\pgfpoint{85.587425pt}{112.92248697916666pt}}
\pgflineto{\pgfpoint{82.28973958333333pt}{113.99396762152777pt}}
\pgflineto{\pgfpoint{81.94159166666667pt}{112.92248697916666pt}}
\pgfpathclose
\pgfusepath{fill,stroke}
\pgfpathmoveto{\pgfpoint{85.587425pt}{112.92248697916666pt}}
\pgflineto{\pgfpoint{83.20118880208334pt}{114.65618098958332pt}}
\pgflineto{\pgfpoint{82.28973958333333pt}{113.99396762152777pt}}
\pgfpathclose
\pgfusepath{fill,stroke}
\pgfpathmoveto{\pgfpoint{85.587425pt}{112.92248697916666pt}}
\pgflineto{\pgfpoint{84.32782725694443pt}{114.65618098958332pt}}
\pgflineto{\pgfpoint{83.20118880208334pt}{114.65618098958332pt}}
\pgfpathclose
\pgfusepath{fill,stroke}
\pgfpathmoveto{\pgfpoint{85.587425pt}{112.92248697916666pt}}
\pgflineto{\pgfpoint{85.23927647569444pt}{113.99396762152777pt}}
\pgflineto{\pgfpoint{84.32782725694443pt}{114.65618098958332pt}}
\pgfpathclose
\pgfusepath{fill,stroke}
\pgfpathmoveto{\pgfpoint{112.41579027777776pt}{118.49768680555555pt}}
\pgflineto{\pgfpoint{111.15617491319443pt}{116.76399340277776pt}}
\pgflineto{\pgfpoint{112.0676423611111pt}{117.42620616319444pt}}
\pgfpathclose
\pgfusepath{fill,stroke}
\pgfpathmoveto{\pgfpoint{112.41579027777776pt}{118.49768680555555pt}}
\pgflineto{\pgfpoint{110.02955407986111pt}{116.76399340277776pt}}
\pgflineto{\pgfpoint{111.15617491319443pt}{116.76399340277776pt}}
\pgfpathclose
\pgfusepath{fill,stroke}
\pgfpathmoveto{\pgfpoint{112.41579027777776pt}{118.49768680555555pt}}
\pgflineto{\pgfpoint{109.11809574652777pt}{117.42620616319444pt}}
\pgflineto{\pgfpoint{110.02955407986111pt}{116.76399340277776pt}}
\pgfpathclose
\pgfusepath{fill,stroke}
\pgfpathmoveto{\pgfpoint{112.41579027777776pt}{118.49768680555555pt}}
\pgflineto{\pgfpoint{108.76995694444443pt}{118.49768680555555pt}}
\pgflineto{\pgfpoint{109.11809574652777pt}{117.42620616319444pt}}
\pgfpathclose
\pgfusepath{fill,stroke}
\pgfpathmoveto{\pgfpoint{112.41579027777776pt}{118.49768680555555pt}}
\pgflineto{\pgfpoint{109.11809574652777pt}{119.56917717013887pt}}
\pgflineto{\pgfpoint{108.76995694444443pt}{118.49768680555555pt}}
\pgfpathclose
\pgfusepath{fill,stroke}
\pgfpathmoveto{\pgfpoint{112.41579027777776pt}{118.49768680555555pt}}
\pgflineto{\pgfpoint{110.02955407986111pt}{120.23138993055554pt}}
\pgflineto{\pgfpoint{109.11809574652777pt}{119.56917717013887pt}}
\pgfpathclose
\pgfusepath{fill,stroke}
\pgfpathmoveto{\pgfpoint{112.41579027777776pt}{118.49768680555555pt}}
\pgflineto{\pgfpoint{111.15617491319443pt}{120.23138993055554pt}}
\pgflineto{\pgfpoint{110.02955407986111pt}{120.23138993055554pt}}
\pgfpathclose
\pgfusepath{fill,stroke}
\pgfpathmoveto{\pgfpoint{112.41579027777776pt}{118.49768680555555pt}}
\pgflineto{\pgfpoint{112.0676423611111pt}{119.56917717013887pt}}
\pgflineto{\pgfpoint{111.15617491319443pt}{120.23138993055554pt}}
\pgfpathclose
\pgfusepath{fill,stroke}
\pgfpathmoveto{\pgfpoint{75.97918125pt}{120.47021388888889pt}}
\pgflineto{\pgfpoint{74.71957499999999pt}{118.73651076388889pt}}
\pgflineto{\pgfpoint{75.63103333333333pt}{119.39872352430554pt}}
\pgfpathclose
\pgfusepath{fill,stroke}
\pgfpathmoveto{\pgfpoint{75.97918125pt}{120.47021388888889pt}}
\pgflineto{\pgfpoint{73.59294505208332pt}{118.73651076388889pt}}
\pgflineto{\pgfpoint{74.71957499999999pt}{118.73651076388889pt}}
\pgfpathclose
\pgfusepath{fill,stroke}
\pgfpathmoveto{\pgfpoint{75.97918125pt}{120.47021388888889pt}}
\pgflineto{\pgfpoint{72.68148671875pt}{119.39872352430554pt}}
\pgflineto{\pgfpoint{73.59294505208332pt}{118.73651076388889pt}}
\pgfpathclose
\pgfusepath{fill,stroke}
\pgfpathmoveto{\pgfpoint{75.97918125pt}{120.47021388888889pt}}
\pgflineto{\pgfpoint{72.33333880208333pt}{120.47021388888889pt}}
\pgflineto{\pgfpoint{72.68148671875pt}{119.39872352430554pt}}
\pgfpathclose
\pgfusepath{fill,stroke}
\pgfpathmoveto{\pgfpoint{75.97918125pt}{120.47021388888889pt}}
\pgflineto{\pgfpoint{72.68148671875pt}{121.54169453125pt}}
\pgflineto{\pgfpoint{72.33333880208333pt}{120.47021388888889pt}}
\pgfpathclose
\pgfusepath{fill,stroke}
\pgfpathmoveto{\pgfpoint{75.97918125pt}{120.47021388888889pt}}
\pgflineto{\pgfpoint{73.59294505208332pt}{122.20390789930555pt}}
\pgflineto{\pgfpoint{72.68148671875pt}{121.54169453125pt}}
\pgfpathclose
\pgfusepath{fill,stroke}
\pgfpathmoveto{\pgfpoint{75.97918125pt}{120.47021388888889pt}}
\pgflineto{\pgfpoint{74.71957499999999pt}{122.20390789930555pt}}
\pgflineto{\pgfpoint{73.59294505208332pt}{122.20390789930555pt}}
\pgfpathclose
\pgfusepath{fill,stroke}
\pgfpathmoveto{\pgfpoint{75.97918125pt}{120.47021388888889pt}}
\pgflineto{\pgfpoint{75.63103333333333pt}{121.54169453125pt}}
\pgflineto{\pgfpoint{74.71957499999999pt}{122.20390789930555pt}}
\pgfpathclose
\pgfusepath{fill,stroke}
\pgfpathmoveto{\pgfpoint{151.6548025173611pt}{100.57829201388888pt}}
\pgflineto{\pgfpoint{150.39520538194444pt}{98.84459861111111pt}}
\pgflineto{\pgfpoint{151.30666371527778pt}{99.50681137152777pt}}
\pgfpathclose
\pgfusepath{fill,stroke}
\pgfpathmoveto{\pgfpoint{151.6548025173611pt}{100.57829201388888pt}}
\pgflineto{\pgfpoint{149.26857543402775pt}{98.84459861111111pt}}
\pgflineto{\pgfpoint{150.39520538194444pt}{98.84459861111111pt}}
\pgfpathclose
\pgfusepath{fill,stroke}
\pgfpathmoveto{\pgfpoint{151.6548025173611pt}{100.57829201388888pt}}
\pgflineto{\pgfpoint{148.35711710069444pt}{99.50681137152777pt}}
\pgflineto{\pgfpoint{149.26857543402775pt}{98.84459861111111pt}}
\pgfpathclose
\pgfusepath{fill,stroke}
\pgfpathmoveto{\pgfpoint{151.6548025173611pt}{100.57829201388888pt}}
\pgflineto{\pgfpoint{148.00896918402776pt}{100.57829201388888pt}}
\pgflineto{\pgfpoint{148.35711710069444pt}{99.50681137152777pt}}
\pgfpathclose
\pgfusepath{fill,stroke}
\pgfpathmoveto{\pgfpoint{151.6548025173611pt}{100.57829201388888pt}}
\pgflineto{\pgfpoint{148.35711710069444pt}{101.64978237847221pt}}
\pgflineto{\pgfpoint{148.00896918402776pt}{100.57829201388888pt}}
\pgfpathclose
\pgfusepath{fill,stroke}
\pgfpathmoveto{\pgfpoint{151.6548025173611pt}{100.57829201388888pt}}
\pgflineto{\pgfpoint{149.26857543402775pt}{102.31199513888889pt}}
\pgflineto{\pgfpoint{148.35711710069444pt}{101.64978237847221pt}}
\pgfpathclose
\pgfusepath{fill,stroke}
\pgfpathmoveto{\pgfpoint{151.6548025173611pt}{100.57829201388888pt}}
\pgflineto{\pgfpoint{150.39520538194444pt}{102.31199513888889pt}}
\pgflineto{\pgfpoint{149.26857543402775pt}{102.31199513888889pt}}
\pgfpathclose
\pgfusepath{fill,stroke}
\pgfpathmoveto{\pgfpoint{151.6548025173611pt}{100.57829201388888pt}}
\pgflineto{\pgfpoint{151.30666371527778pt}{101.64978237847221pt}}
\pgflineto{\pgfpoint{150.39520538194444pt}{102.31199513888889pt}}
\pgfpathclose
\pgfusepath{fill,stroke}
\pgfpathmoveto{\pgfpoint{123.42922690972222pt}{130.7391962673611pt}}
\pgflineto{\pgfpoint{122.1696109375pt}{129.00550225694442pt}}
\pgflineto{\pgfpoint{123.08107899305554pt}{129.6677150173611pt}}
\pgfpathclose
\pgfusepath{fill,stroke}
\pgfpathmoveto{\pgfpoint{123.42922690972222pt}{130.7391962673611pt}}
\pgflineto{\pgfpoint{121.04299071180554pt}{129.00550225694442pt}}
\pgflineto{\pgfpoint{122.1696109375pt}{129.00550225694442pt}}
\pgfpathclose
\pgfusepath{fill,stroke}
\pgfpathmoveto{\pgfpoint{123.42922690972222pt}{130.7391962673611pt}}
\pgflineto{\pgfpoint{120.13153237847222pt}{129.6677150173611pt}}
\pgflineto{\pgfpoint{121.04299071180554pt}{129.00550225694442pt}}
\pgfpathclose
\pgfusepath{fill,stroke}
\pgfpathmoveto{\pgfpoint{123.42922690972222pt}{130.7391962673611pt}}
\pgflineto{\pgfpoint{119.78339357638889pt}{130.7391962673611pt}}
\pgflineto{\pgfpoint{120.13153237847222pt}{129.6677150173611pt}}
\pgfpathclose
\pgfusepath{fill,stroke}
\pgfpathmoveto{\pgfpoint{123.42922690972222pt}{130.7391962673611pt}}
\pgflineto{\pgfpoint{120.13153237847222pt}{131.81067690972222pt}}
\pgflineto{\pgfpoint{119.78339357638889pt}{130.7391962673611pt}}
\pgfpathclose
\pgfusepath{fill,stroke}
\pgfpathmoveto{\pgfpoint{123.42922690972222pt}{130.7391962673611pt}}
\pgflineto{\pgfpoint{121.04299071180554pt}{132.4728993923611pt}}
\pgflineto{\pgfpoint{120.13153237847222pt}{131.81067690972222pt}}
\pgfpathclose
\pgfusepath{fill,stroke}
\pgfpathmoveto{\pgfpoint{123.42922690972222pt}{130.7391962673611pt}}
\pgflineto{\pgfpoint{122.1696109375pt}{132.4728993923611pt}}
\pgflineto{\pgfpoint{121.04299071180554pt}{132.4728993923611pt}}
\pgfpathclose
\pgfusepath{fill,stroke}
\pgfpathmoveto{\pgfpoint{123.42922690972222pt}{130.7391962673611pt}}
\pgflineto{\pgfpoint{123.08107899305554pt}{131.81067690972222pt}}
\pgflineto{\pgfpoint{122.1696109375pt}{132.4728993923611pt}}
\pgfpathclose
\pgfusepath{fill,stroke}
\pgfpathmoveto{\pgfpoint{97.80189696180555pt}{116.74769348958331pt}}
\pgflineto{\pgfpoint{96.54229071180553pt}{115.01399947916666pt}}
\pgflineto{\pgfpoint{97.45374904513888pt}{115.67621223958332pt}}
\pgfpathclose
\pgfusepath{fill,stroke}
\pgfpathmoveto{\pgfpoint{97.80189696180555pt}{116.74769348958331pt}}
\pgflineto{\pgfpoint{95.4156704861111pt}{115.01399947916666pt}}
\pgflineto{\pgfpoint{96.54229071180553pt}{115.01399947916666pt}}
\pgfpathclose
\pgfusepath{fill,stroke}
\pgfpathmoveto{\pgfpoint{97.80189696180555pt}{116.74769348958331pt}}
\pgflineto{\pgfpoint{94.50420303819443pt}{115.67621223958332pt}}
\pgflineto{\pgfpoint{95.4156704861111pt}{115.01399947916666pt}}
\pgfpathclose
\pgfusepath{fill,stroke}
\pgfpathmoveto{\pgfpoint{97.80189696180555pt}{116.74769348958331pt}}
\pgflineto{\pgfpoint{94.15606362847221pt}{116.74769348958331pt}}
\pgflineto{\pgfpoint{94.50420303819443pt}{115.67621223958332pt}}
\pgfpathclose
\pgfusepath{fill,stroke}
\pgfpathmoveto{\pgfpoint{97.80189696180555pt}{116.74769348958331pt}}
\pgflineto{\pgfpoint{94.50420303819443pt}{117.81917413194442pt}}
\pgflineto{\pgfpoint{94.15606362847221pt}{116.74769348958331pt}}
\pgfpathclose
\pgfusepath{fill,stroke}
\pgfpathmoveto{\pgfpoint{97.80189696180555pt}{116.74769348958331pt}}
\pgflineto{\pgfpoint{95.4156704861111pt}{118.4813868923611pt}}
\pgflineto{\pgfpoint{94.50420303819443pt}{117.81917413194442pt}}
\pgfpathclose
\pgfusepath{fill,stroke}
\pgfpathmoveto{\pgfpoint{97.80189696180555pt}{116.74769348958331pt}}
\pgflineto{\pgfpoint{96.54229071180553pt}{118.4813868923611pt}}
\pgflineto{\pgfpoint{95.4156704861111pt}{118.4813868923611pt}}
\pgfpathclose
\pgfusepath{fill,stroke}
\pgfpathmoveto{\pgfpoint{97.80189696180555pt}{116.74769348958331pt}}
\pgflineto{\pgfpoint{97.45374904513888pt}{117.81917413194442pt}}
\pgflineto{\pgfpoint{96.54229071180553pt}{118.4813868923611pt}}
\pgfpathclose
\pgfusepath{fill,stroke}
\pgfpathmoveto{\pgfpoint{120.43061406249998pt}{113.32468984374998pt}}
\pgflineto{\pgfpoint{119.1710072048611pt}{111.59099583333332pt}}
\pgflineto{\pgfpoint{120.08246553819444pt}{112.25320920138887pt}}
\pgfpathclose
\pgfusepath{fill,stroke}
\pgfpathmoveto{\pgfpoint{120.43061406249998pt}{113.32468984374998pt}}
\pgflineto{\pgfpoint{118.04438697916666pt}{111.59099583333332pt}}
\pgflineto{\pgfpoint{119.1710072048611pt}{111.59099583333332pt}}
\pgfpathclose
\pgfusepath{fill,stroke}
\pgfpathmoveto{\pgfpoint{120.43061406249998pt}{113.32468984374998pt}}
\pgflineto{\pgfpoint{117.13292864583333pt}{112.25320920138887pt}}
\pgflineto{\pgfpoint{118.04438697916666pt}{111.59099583333332pt}}
\pgfpathclose
\pgfusepath{fill,stroke}
\pgfpathmoveto{\pgfpoint{120.43061406249998pt}{113.32468984374998pt}}
\pgflineto{\pgfpoint{116.78478984374999pt}{113.32468984374998pt}}
\pgflineto{\pgfpoint{117.13292864583333pt}{112.25320920138887pt}}
\pgfpathclose
\pgfusepath{fill,stroke}
\pgfpathmoveto{\pgfpoint{120.43061406249998pt}{113.32468984374998pt}}
\pgflineto{\pgfpoint{117.13292864583333pt}{114.39617048611109pt}}
\pgflineto{\pgfpoint{116.78478984374999pt}{113.32468984374998pt}}
\pgfpathclose
\pgfusepath{fill,stroke}
\pgfpathmoveto{\pgfpoint{120.43061406249998pt}{113.32468984374998pt}}
\pgflineto{\pgfpoint{118.04438697916666pt}{115.05838385416666pt}}
\pgflineto{\pgfpoint{117.13292864583333pt}{114.39617048611109pt}}
\pgfpathclose
\pgfusepath{fill,stroke}
\pgfpathmoveto{\pgfpoint{120.43061406249998pt}{113.32468984374998pt}}
\pgflineto{\pgfpoint{119.1710072048611pt}{115.05838385416666pt}}
\pgflineto{\pgfpoint{118.04438697916666pt}{115.05838385416666pt}}
\pgfpathclose
\pgfusepath{fill,stroke}
\pgfpathmoveto{\pgfpoint{120.43061406249998pt}{113.32468984374998pt}}
\pgflineto{\pgfpoint{120.08246553819444pt}{114.39617048611109pt}}
\pgflineto{\pgfpoint{119.1710072048611pt}{115.05838385416666pt}}
\pgfpathclose
\pgfusepath{fill,stroke}
\pgfpathmoveto{\pgfpoint{102.60072022569445pt}{120.4359728298611pt}}
\pgflineto{\pgfpoint{101.34112309027778pt}{118.70227942708331pt}}
\pgflineto{\pgfpoint{102.2525814236111pt}{119.36449218749999pt}}
\pgfpathclose
\pgfusepath{fill,stroke}
\pgfpathmoveto{\pgfpoint{102.60072022569445pt}{120.4359728298611pt}}
\pgflineto{\pgfpoint{100.21450225694444pt}{118.70227942708331pt}}
\pgflineto{\pgfpoint{101.34112309027778pt}{118.70227942708331pt}}
\pgfpathclose
\pgfusepath{fill,stroke}
\pgfpathmoveto{\pgfpoint{102.60072022569445pt}{120.4359728298611pt}}
\pgflineto{\pgfpoint{99.30304392361111pt}{119.36449218749999pt}}
\pgflineto{\pgfpoint{100.21450225694444pt}{118.70227942708331pt}}
\pgfpathclose
\pgfusepath{fill,stroke}
\pgfpathmoveto{\pgfpoint{102.60072022569445pt}{120.4359728298611pt}}
\pgflineto{\pgfpoint{98.95489600694444pt}{120.4359728298611pt}}
\pgflineto{\pgfpoint{99.30304392361111pt}{119.36449218749999pt}}
\pgfpathclose
\pgfusepath{fill,stroke}
\pgfpathmoveto{\pgfpoint{102.60072022569445pt}{120.4359728298611pt}}
\pgflineto{\pgfpoint{99.30304392361111pt}{121.5074540798611pt}}
\pgflineto{\pgfpoint{98.95489600694444pt}{120.4359728298611pt}}
\pgfpathclose
\pgfusepath{fill,stroke}
\pgfpathmoveto{\pgfpoint{102.60072022569445pt}{120.4359728298611pt}}
\pgflineto{\pgfpoint{100.21450225694444pt}{122.16966684027777pt}}
\pgflineto{\pgfpoint{99.30304392361111pt}{121.5074540798611pt}}
\pgfpathclose
\pgfusepath{fill,stroke}
\pgfpathmoveto{\pgfpoint{102.60072022569445pt}{120.4359728298611pt}}
\pgflineto{\pgfpoint{101.34112309027778pt}{122.16966684027777pt}}
\pgflineto{\pgfpoint{100.21450225694444pt}{122.16966684027777pt}}
\pgfpathclose
\pgfusepath{fill,stroke}
\pgfpathmoveto{\pgfpoint{102.60072022569445pt}{120.4359728298611pt}}
\pgflineto{\pgfpoint{102.2525814236111pt}{121.5074540798611pt}}
\pgflineto{\pgfpoint{101.34112309027778pt}{122.16966684027777pt}}
\pgfpathclose
\pgfusepath{fill,stroke}
\pgfpathmoveto{\pgfpoint{130.2430232638889pt}{108.4212433159722pt}}
\pgflineto{\pgfpoint{128.98342612847222pt}{106.68754019097221pt}}
\pgflineto{\pgfpoint{129.89488446180556pt}{107.34975355902777pt}}
\pgfpathclose
\pgfusepath{fill,stroke}
\pgfpathmoveto{\pgfpoint{130.2430232638889pt}{108.4212433159722pt}}
\pgflineto{\pgfpoint{127.85680529513888pt}{106.68754019097221pt}}
\pgflineto{\pgfpoint{128.98342612847222pt}{106.68754019097221pt}}
\pgfpathclose
\pgfusepath{fill,stroke}
\pgfpathmoveto{\pgfpoint{130.2430232638889pt}{108.4212433159722pt}}
\pgflineto{\pgfpoint{126.9453378472222pt}{107.34975355902777pt}}
\pgflineto{\pgfpoint{127.85680529513888pt}{106.68754019097221pt}}
\pgfpathclose
\pgfusepath{fill,stroke}
\pgfpathmoveto{\pgfpoint{130.2430232638889pt}{108.4212433159722pt}}
\pgflineto{\pgfpoint{126.59718993055554pt}{108.4212433159722pt}}
\pgflineto{\pgfpoint{126.9453378472222pt}{107.34975355902777pt}}
\pgfpathclose
\pgfusepath{fill,stroke}
\pgfpathmoveto{\pgfpoint{130.2430232638889pt}{108.4212433159722pt}}
\pgflineto{\pgfpoint{126.9453378472222pt}{109.49272456597221pt}}
\pgflineto{\pgfpoint{126.59718993055554pt}{108.4212433159722pt}}
\pgfpathclose
\pgfusepath{fill,stroke}
\pgfpathmoveto{\pgfpoint{130.2430232638889pt}{108.4212433159722pt}}
\pgflineto{\pgfpoint{127.85680529513888pt}{110.15493732638889pt}}
\pgflineto{\pgfpoint{126.9453378472222pt}{109.49272456597221pt}}
\pgfpathclose
\pgfusepath{fill,stroke}
\pgfpathmoveto{\pgfpoint{130.2430232638889pt}{108.4212433159722pt}}
\pgflineto{\pgfpoint{128.98342612847222pt}{110.15493732638889pt}}
\pgflineto{\pgfpoint{127.85680529513888pt}{110.15493732638889pt}}
\pgfpathclose
\pgfusepath{fill,stroke}
\pgfpathmoveto{\pgfpoint{130.2430232638889pt}{108.4212433159722pt}}
\pgflineto{\pgfpoint{129.89488446180556pt}{109.49272456597221pt}}
\pgflineto{\pgfpoint{128.98342612847222pt}{110.15493732638889pt}}
\pgfpathclose
\pgfusepath{fill,stroke}
\pgfpathmoveto{\pgfpoint{181.53744036458332pt}{92.81235286458333pt}}
\pgflineto{\pgfpoint{180.27782439236108pt}{91.07865885416666pt}}
\pgflineto{\pgfpoint{181.18928272569443pt}{91.74087161458333pt}}
\pgfpathclose
\pgfusepath{fill,stroke}
\pgfpathmoveto{\pgfpoint{181.53744036458332pt}{92.81235286458333pt}}
\pgflineto{\pgfpoint{179.15120416666667pt}{91.07865885416666pt}}
\pgflineto{\pgfpoint{180.27782439236108pt}{91.07865885416666pt}}
\pgfpathclose
\pgfusepath{fill,stroke}
\pgfpathmoveto{\pgfpoint{181.53744036458332pt}{92.81235286458333pt}}
\pgflineto{\pgfpoint{178.23974583333333pt}{91.74087161458333pt}}
\pgflineto{\pgfpoint{179.15120416666667pt}{91.07865885416666pt}}
\pgfpathclose
\pgfusepath{fill,stroke}
\pgfpathmoveto{\pgfpoint{181.53744036458332pt}{92.81235286458333pt}}
\pgflineto{\pgfpoint{177.89160703124998pt}{92.81235286458333pt}}
\pgflineto{\pgfpoint{178.23974583333333pt}{91.74087161458333pt}}
\pgfpathclose
\pgfusepath{fill,stroke}
\pgfpathmoveto{\pgfpoint{181.53744036458332pt}{92.81235286458333pt}}
\pgflineto{\pgfpoint{178.23974583333333pt}{93.88384262152776pt}}
\pgflineto{\pgfpoint{177.89160703124998pt}{92.81235286458333pt}}
\pgfpathclose
\pgfusepath{fill,stroke}
\pgfpathmoveto{\pgfpoint{181.53744036458332pt}{92.81235286458333pt}}
\pgflineto{\pgfpoint{179.15120416666667pt}{94.54605598958334pt}}
\pgflineto{\pgfpoint{178.23974583333333pt}{93.88384262152776pt}}
\pgfpathclose
\pgfusepath{fill,stroke}
\pgfpathmoveto{\pgfpoint{181.53744036458332pt}{92.81235286458333pt}}
\pgflineto{\pgfpoint{180.27782439236108pt}{94.54605598958334pt}}
\pgflineto{\pgfpoint{179.15120416666667pt}{94.54605598958334pt}}
\pgfpathclose
\pgfusepath{fill,stroke}
\pgfpathmoveto{\pgfpoint{181.53744036458332pt}{92.81235286458333pt}}
\pgflineto{\pgfpoint{181.18928272569443pt}{93.88384262152776pt}}
\pgflineto{\pgfpoint{180.27782439236108pt}{94.54605598958334pt}}
\pgfpathclose
\pgfusepath{fill,stroke}
\pgfpathmoveto{\pgfpoint{134.02110069444444pt}{127.15788567708331pt}}
\pgflineto{\pgfpoint{132.76149444444445pt}{125.42419166666666pt}}
\pgflineto{\pgfpoint{133.67295277777777pt}{126.08640442708332pt}}
\pgfpathclose
\pgfusepath{fill,stroke}
\pgfpathmoveto{\pgfpoint{134.02110069444444pt}{127.15788567708331pt}}
\pgflineto{\pgfpoint{131.6348736111111pt}{125.42419166666666pt}}
\pgflineto{\pgfpoint{132.76149444444445pt}{125.42419166666666pt}}
\pgfpathclose
\pgfusepath{fill,stroke}
\pgfpathmoveto{\pgfpoint{134.02110069444444pt}{127.15788567708331pt}}
\pgflineto{\pgfpoint{130.72341527777775pt}{126.08640442708332pt}}
\pgflineto{\pgfpoint{131.6348736111111pt}{125.42419166666666pt}}
\pgfpathclose
\pgfusepath{fill,stroke}
\pgfpathmoveto{\pgfpoint{134.02110069444444pt}{127.15788567708331pt}}
\pgflineto{\pgfpoint{130.3752673611111pt}{127.15788567708331pt}}
\pgflineto{\pgfpoint{130.72341527777775pt}{126.08640442708332pt}}
\pgfpathclose
\pgfusepath{fill,stroke}
\pgfpathmoveto{\pgfpoint{134.02110069444444pt}{127.15788567708331pt}}
\pgflineto{\pgfpoint{130.72341527777775pt}{128.2293754340278pt}}
\pgflineto{\pgfpoint{130.3752673611111pt}{127.15788567708331pt}}
\pgfpathclose
\pgfusepath{fill,stroke}
\pgfpathmoveto{\pgfpoint{134.02110069444444pt}{127.15788567708331pt}}
\pgflineto{\pgfpoint{131.6348736111111pt}{128.89158880208333pt}}
\pgflineto{\pgfpoint{130.72341527777775pt}{128.2293754340278pt}}
\pgfpathclose
\pgfusepath{fill,stroke}
\pgfpathmoveto{\pgfpoint{134.02110069444444pt}{127.15788567708331pt}}
\pgflineto{\pgfpoint{132.76149444444445pt}{128.89158880208333pt}}
\pgflineto{\pgfpoint{131.6348736111111pt}{128.89158880208333pt}}
\pgfpathclose
\pgfusepath{fill,stroke}
\pgfpathmoveto{\pgfpoint{134.02110069444444pt}{127.15788567708331pt}}
\pgflineto{\pgfpoint{133.67295277777777pt}{128.2293754340278pt}}
\pgflineto{\pgfpoint{132.76149444444445pt}{128.89158880208333pt}}
\pgfpathclose
\pgfusepath{fill,stroke}
\pgfpathmoveto{\pgfpoint{132.06710842013888pt}{124.78318368055555pt}}
\pgflineto{\pgfpoint{130.80749244791664pt}{123.04948967013887pt}}
\pgflineto{\pgfpoint{131.7189598958333pt}{123.71170243055555pt}}
\pgfpathclose
\pgfusepath{fill,stroke}
\pgfpathmoveto{\pgfpoint{132.06710842013888pt}{124.78318368055555pt}}
\pgflineto{\pgfpoint{129.6808722222222pt}{123.04948967013887pt}}
\pgflineto{\pgfpoint{130.80749244791664pt}{123.04948967013887pt}}
\pgfpathclose
\pgfusepath{fill,stroke}
\pgfpathmoveto{\pgfpoint{132.06710842013888pt}{124.78318368055555pt}}
\pgflineto{\pgfpoint{128.7694138888889pt}{123.71170243055555pt}}
\pgflineto{\pgfpoint{129.6808722222222pt}{123.04948967013887pt}}
\pgfpathclose
\pgfusepath{fill,stroke}
\pgfpathmoveto{\pgfpoint{132.06710842013888pt}{124.78318368055555pt}}
\pgflineto{\pgfpoint{128.42127508680554pt}{124.78318368055555pt}}
\pgflineto{\pgfpoint{128.7694138888889pt}{123.71170243055555pt}}
\pgfpathclose
\pgfusepath{fill,stroke}
\pgfpathmoveto{\pgfpoint{132.06710842013888pt}{124.78318368055555pt}}
\pgflineto{\pgfpoint{128.7694138888889pt}{125.85466432291666pt}}
\pgflineto{\pgfpoint{128.42127508680554pt}{124.78318368055555pt}}
\pgfpathclose
\pgfusepath{fill,stroke}
\pgfpathmoveto{\pgfpoint{132.06710842013888pt}{124.78318368055555pt}}
\pgflineto{\pgfpoint{129.6808722222222pt}{126.51687708333333pt}}
\pgflineto{\pgfpoint{128.7694138888889pt}{125.85466432291666pt}}
\pgfpathclose
\pgfusepath{fill,stroke}
\pgfpathmoveto{\pgfpoint{132.06710842013888pt}{124.78318368055555pt}}
\pgflineto{\pgfpoint{130.80749244791664pt}{126.51687708333333pt}}
\pgflineto{\pgfpoint{129.6808722222222pt}{126.51687708333333pt}}
\pgfpathclose
\pgfusepath{fill,stroke}
\pgfpathmoveto{\pgfpoint{132.06710842013888pt}{124.78318368055555pt}}
\pgflineto{\pgfpoint{131.7189598958333pt}{125.85466432291666pt}}
\pgflineto{\pgfpoint{130.80749244791664pt}{126.51687708333333pt}}
\pgfpathclose
\pgfusepath{fill,stroke}
\pgfpathmoveto{\pgfpoint{111.72644826388888pt}{121.55699305555555pt}}
\pgflineto{\pgfpoint{110.46684201388888pt}{119.82329965277778pt}}
\pgflineto{\pgfpoint{111.37830034722222pt}{120.48551241319443pt}}
\pgfpathclose
\pgfusepath{fill,stroke}
\pgfpathmoveto{\pgfpoint{111.72644826388888pt}{121.55699305555555pt}}
\pgflineto{\pgfpoint{109.34021206597221pt}{119.82329965277778pt}}
\pgflineto{\pgfpoint{110.46684201388888pt}{119.82329965277778pt}}
\pgfpathclose
\pgfusepath{fill,stroke}
\pgfpathmoveto{\pgfpoint{111.72644826388888pt}{121.55699305555555pt}}
\pgflineto{\pgfpoint{108.42875373263888pt}{120.48551241319443pt}}
\pgflineto{\pgfpoint{109.34021206597221pt}{119.82329965277778pt}}
\pgfpathclose
\pgfusepath{fill,stroke}
\pgfpathmoveto{\pgfpoint{111.72644826388888pt}{121.55699305555555pt}}
\pgflineto{\pgfpoint{108.08061493055554pt}{121.55699305555555pt}}
\pgflineto{\pgfpoint{108.42875373263888pt}{120.48551241319443pt}}
\pgfpathclose
\pgfusepath{fill,stroke}
\pgfpathmoveto{\pgfpoint{111.72644826388888pt}{121.55699305555555pt}}
\pgflineto{\pgfpoint{108.42875373263888pt}{122.62847430555554pt}}
\pgflineto{\pgfpoint{108.08061493055554pt}{121.55699305555555pt}}
\pgfpathclose
\pgfusepath{fill,stroke}
\pgfpathmoveto{\pgfpoint{111.72644826388888pt}{121.55699305555555pt}}
\pgflineto{\pgfpoint{109.34021206597221pt}{123.29068706597221pt}}
\pgflineto{\pgfpoint{108.42875373263888pt}{122.62847430555554pt}}
\pgfpathclose
\pgfusepath{fill,stroke}
\pgfpathmoveto{\pgfpoint{111.72644826388888pt}{121.55699305555555pt}}
\pgflineto{\pgfpoint{110.46684201388888pt}{123.29068706597221pt}}
\pgflineto{\pgfpoint{109.34021206597221pt}{123.29068706597221pt}}
\pgfpathclose
\pgfusepath{fill,stroke}
\pgfpathmoveto{\pgfpoint{111.72644826388888pt}{121.55699305555555pt}}
\pgflineto{\pgfpoint{111.37830034722222pt}{122.62847430555554pt}}
\pgflineto{\pgfpoint{110.46684201388888pt}{123.29068706597221pt}}
\pgfpathclose
\pgfusepath{fill,stroke}
\pgfpathmoveto{\pgfpoint{113.24298671874999pt}{115.33569505208331pt}}
\pgflineto{\pgfpoint{111.98337074652777pt}{113.60200104166665pt}}
\pgflineto{\pgfpoint{112.89483819444445pt}{114.26421380208332pt}}
\pgfpathclose
\pgfusepath{fill,stroke}
\pgfpathmoveto{\pgfpoint{113.24298671874999pt}{115.33569505208331pt}}
\pgflineto{\pgfpoint{110.85675052083332pt}{113.60200104166665pt}}
\pgflineto{\pgfpoint{111.98337074652777pt}{113.60200104166665pt}}
\pgfpathclose
\pgfusepath{fill,stroke}
\pgfpathmoveto{\pgfpoint{113.24298671874999pt}{115.33569505208331pt}}
\pgflineto{\pgfpoint{109.94529218749999pt}{114.26421380208332pt}}
\pgflineto{\pgfpoint{110.85675052083332pt}{113.60200104166665pt}}
\pgfpathclose
\pgfusepath{fill,stroke}
\pgfpathmoveto{\pgfpoint{113.24298671874999pt}{115.33569505208331pt}}
\pgflineto{\pgfpoint{109.59715338541666pt}{115.33569505208331pt}}
\pgflineto{\pgfpoint{109.94529218749999pt}{114.26421380208332pt}}
\pgfpathclose
\pgfusepath{fill,stroke}
\pgfpathmoveto{\pgfpoint{113.24298671874999pt}{115.33569505208331pt}}
\pgflineto{\pgfpoint{109.94529218749999pt}{116.40718541666666pt}}
\pgflineto{\pgfpoint{109.59715338541666pt}{115.33569505208331pt}}
\pgfpathclose
\pgfusepath{fill,stroke}
\pgfpathmoveto{\pgfpoint{113.24298671874999pt}{115.33569505208331pt}}
\pgflineto{\pgfpoint{110.85675052083332pt}{117.06939817708331pt}}
\pgflineto{\pgfpoint{109.94529218749999pt}{116.40718541666666pt}}
\pgfpathclose
\pgfusepath{fill,stroke}
\pgfpathmoveto{\pgfpoint{113.24298671874999pt}{115.33569505208331pt}}
\pgflineto{\pgfpoint{111.98337074652777pt}{117.06939817708331pt}}
\pgflineto{\pgfpoint{110.85675052083332pt}{117.06939817708331pt}}
\pgfpathclose
\pgfusepath{fill,stroke}
\pgfpathmoveto{\pgfpoint{113.24298671874999pt}{115.33569505208331pt}}
\pgflineto{\pgfpoint{112.89483819444445pt}{116.40718541666666pt}}
\pgflineto{\pgfpoint{111.98337074652777pt}{117.06939817708331pt}}
\pgfpathclose
\pgfusepath{fill,stroke}
\pgfpathmoveto{\pgfpoint{125.18171458333333pt}{136.42138368055555pt}}
\pgflineto{\pgfpoint{123.92210833333333pt}{134.68768055555554pt}}
\pgflineto{\pgfpoint{124.83356666666666pt}{135.34989331597222pt}}
\pgfpathclose
\pgfusepath{fill,stroke}
\pgfpathmoveto{\pgfpoint{125.18171458333333pt}{136.42138368055555pt}}
\pgflineto{\pgfpoint{122.7954875pt}{134.68768055555554pt}}
\pgflineto{\pgfpoint{123.92210833333333pt}{134.68768055555554pt}}
\pgfpathclose
\pgfusepath{fill,stroke}
\pgfpathmoveto{\pgfpoint{125.18171458333333pt}{136.42138368055555pt}}
\pgflineto{\pgfpoint{121.88402916666665pt}{135.34989331597222pt}}
\pgflineto{\pgfpoint{122.7954875pt}{134.68768055555554pt}}
\pgfpathclose
\pgfusepath{fill,stroke}
\pgfpathmoveto{\pgfpoint{125.18171458333333pt}{136.42138368055555pt}}
\pgflineto{\pgfpoint{121.53588125pt}{136.42138368055555pt}}
\pgflineto{\pgfpoint{121.88402916666665pt}{135.34989331597222pt}}
\pgfpathclose
\pgfusepath{fill,stroke}
\pgfpathmoveto{\pgfpoint{125.18171458333333pt}{136.42138368055555pt}}
\pgflineto{\pgfpoint{121.88402916666665pt}{137.49286493055553pt}}
\pgflineto{\pgfpoint{121.53588125pt}{136.42138368055555pt}}
\pgfpathclose
\pgfusepath{fill,stroke}
\pgfpathmoveto{\pgfpoint{125.18171458333333pt}{136.42138368055555pt}}
\pgflineto{\pgfpoint{122.7954875pt}{138.1550776909722pt}}
\pgflineto{\pgfpoint{121.88402916666665pt}{137.49286493055553pt}}
\pgfpathclose
\pgfusepath{fill,stroke}
\pgfpathmoveto{\pgfpoint{125.18171458333333pt}{136.42138368055555pt}}
\pgflineto{\pgfpoint{123.92210833333333pt}{138.1550776909722pt}}
\pgflineto{\pgfpoint{122.7954875pt}{138.1550776909722pt}}
\pgfpathclose
\pgfusepath{fill,stroke}
\pgfpathmoveto{\pgfpoint{125.18171458333333pt}{136.42138368055555pt}}
\pgflineto{\pgfpoint{124.83356666666666pt}{137.49286493055553pt}}
\pgflineto{\pgfpoint{123.92210833333333pt}{138.1550776909722pt}}
\pgfpathclose
\pgfusepath{fill,stroke}
\pgfpathmoveto{\pgfpoint{111.14846093749999pt}{142.29610086805556pt}}
\pgflineto{\pgfpoint{109.88886380208332pt}{140.56239774305556pt}}
\pgflineto{\pgfpoint{110.80032213541666pt}{141.2246105034722pt}}
\pgfpathclose
\pgfusepath{fill,stroke}
\pgfpathmoveto{\pgfpoint{111.14846093749999pt}{142.29610086805556pt}}
\pgflineto{\pgfpoint{108.76224296875pt}{140.56239774305556pt}}
\pgflineto{\pgfpoint{109.88886380208332pt}{140.56239774305556pt}}
\pgfpathclose
\pgfusepath{fill,stroke}
\pgfpathmoveto{\pgfpoint{111.14846093749999pt}{142.29610086805556pt}}
\pgflineto{\pgfpoint{107.85077552083332pt}{141.2246105034722pt}}
\pgflineto{\pgfpoint{108.76224296875pt}{140.56239774305556pt}}
\pgfpathclose
\pgfusepath{fill,stroke}
\pgfpathmoveto{\pgfpoint{111.14846093749999pt}{142.29610086805556pt}}
\pgflineto{\pgfpoint{107.50262760416666pt}{142.29610086805556pt}}
\pgflineto{\pgfpoint{107.85077552083332pt}{141.2246105034722pt}}
\pgfpathclose
\pgfusepath{fill,stroke}
\pgfpathmoveto{\pgfpoint{111.14846093749999pt}{142.29610086805556pt}}
\pgflineto{\pgfpoint{107.85077552083332pt}{143.36758211805554pt}}
\pgflineto{\pgfpoint{107.50262760416666pt}{142.29610086805556pt}}
\pgfpathclose
\pgfusepath{fill,stroke}
\pgfpathmoveto{\pgfpoint{111.14846093749999pt}{142.29610086805556pt}}
\pgflineto{\pgfpoint{108.76224296875pt}{144.02979487847222pt}}
\pgflineto{\pgfpoint{107.85077552083332pt}{143.36758211805554pt}}
\pgfpathclose
\pgfusepath{fill,stroke}
\pgfpathmoveto{\pgfpoint{111.14846093749999pt}{142.29610086805556pt}}
\pgflineto{\pgfpoint{109.88886380208332pt}{144.02979487847222pt}}
\pgflineto{\pgfpoint{108.76224296875pt}{144.02979487847222pt}}
\pgfpathclose
\pgfusepath{fill,stroke}
\pgfpathmoveto{\pgfpoint{111.14846093749999pt}{142.29610086805556pt}}
\pgflineto{\pgfpoint{110.80032213541666pt}{143.36758211805554pt}}
\pgflineto{\pgfpoint{109.88886380208332pt}{144.02979487847222pt}}
\pgfpathclose
\pgfusepath{fill,stroke}
\pgfpathmoveto{\pgfpoint{149.85723472222222pt}{111.20670616319444pt}}
\pgflineto{\pgfpoint{148.5976284722222pt}{109.47301276041667pt}}
\pgflineto{\pgfpoint{149.50908680555554pt}{110.13522552083333pt}}
\pgfpathclose
\pgfusepath{fill,stroke}
\pgfpathmoveto{\pgfpoint{149.85723472222222pt}{111.20670616319444pt}}
\pgflineto{\pgfpoint{147.47099852430554pt}{109.47301276041667pt}}
\pgflineto{\pgfpoint{148.5976284722222pt}{109.47301276041667pt}}
\pgfpathclose
\pgfusepath{fill,stroke}
\pgfpathmoveto{\pgfpoint{149.85723472222222pt}{111.20670616319444pt}}
\pgflineto{\pgfpoint{146.55954019097223pt}{110.13522552083333pt}}
\pgflineto{\pgfpoint{147.47099852430554pt}{109.47301276041667pt}}
\pgfpathclose
\pgfusepath{fill,stroke}
\pgfpathmoveto{\pgfpoint{149.85723472222222pt}{111.20670616319444pt}}
\pgflineto{\pgfpoint{146.21140138888887pt}{111.20670616319444pt}}
\pgflineto{\pgfpoint{146.55954019097223pt}{110.13522552083333pt}}
\pgfpathclose
\pgfusepath{fill,stroke}
\pgfpathmoveto{\pgfpoint{149.85723472222222pt}{111.20670616319444pt}}
\pgflineto{\pgfpoint{146.55954019097223pt}{112.27819652777777pt}}
\pgflineto{\pgfpoint{146.21140138888887pt}{111.20670616319444pt}}
\pgfpathclose
\pgfusepath{fill,stroke}
\pgfpathmoveto{\pgfpoint{149.85723472222222pt}{111.20670616319444pt}}
\pgflineto{\pgfpoint{147.47099852430554pt}{112.94040928819445pt}}
\pgflineto{\pgfpoint{146.55954019097223pt}{112.27819652777777pt}}
\pgfpathclose
\pgfusepath{fill,stroke}
\pgfpathmoveto{\pgfpoint{149.85723472222222pt}{111.20670616319444pt}}
\pgflineto{\pgfpoint{148.5976284722222pt}{112.94040928819445pt}}
\pgflineto{\pgfpoint{147.47099852430554pt}{112.94040928819445pt}}
\pgfpathclose
\pgfusepath{fill,stroke}
\pgfpathmoveto{\pgfpoint{149.85723472222222pt}{111.20670616319444pt}}
\pgflineto{\pgfpoint{149.50908680555554pt}{112.27819652777777pt}}
\pgflineto{\pgfpoint{148.5976284722222pt}{112.94040928819445pt}}
\pgfpathclose
\pgfusepath{fill,stroke}
\pgfpathmoveto{\pgfpoint{105.80347118055555pt}{139.40366927083332pt}}
\pgflineto{\pgfpoint{104.54387404513889pt}{137.6699661458333pt}}
\pgflineto{\pgfpoint{105.45533237847222pt}{138.3321880208333pt}}
\pgfpathclose
\pgfusepath{fill,stroke}
\pgfpathmoveto{\pgfpoint{105.80347118055555pt}{139.40366927083332pt}}
\pgflineto{\pgfpoint{103.41725321180556pt}{137.6699661458333pt}}
\pgflineto{\pgfpoint{104.54387404513889pt}{137.6699661458333pt}}
\pgfpathclose
\pgfusepath{fill,stroke}
\pgfpathmoveto{\pgfpoint{105.80347118055555pt}{139.40366927083332pt}}
\pgflineto{\pgfpoint{102.50578576388888pt}{138.3321880208333pt}}
\pgflineto{\pgfpoint{103.41725321180556pt}{137.6699661458333pt}}
\pgfpathclose
\pgfusepath{fill,stroke}
\pgfpathmoveto{\pgfpoint{105.80347118055555pt}{139.40366927083332pt}}
\pgflineto{\pgfpoint{102.15763784722222pt}{139.40366927083332pt}}
\pgflineto{\pgfpoint{102.50578576388888pt}{138.3321880208333pt}}
\pgfpathclose
\pgfusepath{fill,stroke}
\pgfpathmoveto{\pgfpoint{105.80347118055555pt}{139.40366927083332pt}}
\pgflineto{\pgfpoint{102.50578576388888pt}{140.47514991319443pt}}
\pgflineto{\pgfpoint{102.15763784722222pt}{139.40366927083332pt}}
\pgfpathclose
\pgfusepath{fill,stroke}
\pgfpathmoveto{\pgfpoint{105.80347118055555pt}{139.40366927083332pt}}
\pgflineto{\pgfpoint{103.41725321180556pt}{141.1373626736111pt}}
\pgflineto{\pgfpoint{102.50578576388888pt}{140.47514991319443pt}}
\pgfpathclose
\pgfusepath{fill,stroke}
\pgfpathmoveto{\pgfpoint{105.80347118055555pt}{139.40366927083332pt}}
\pgflineto{\pgfpoint{104.54387404513889pt}{141.1373626736111pt}}
\pgflineto{\pgfpoint{103.41725321180556pt}{141.1373626736111pt}}
\pgfpathclose
\pgfusepath{fill,stroke}
\pgfpathmoveto{\pgfpoint{105.80347118055555pt}{139.40366927083332pt}}
\pgflineto{\pgfpoint{105.45533237847222pt}{140.47514991319443pt}}
\pgflineto{\pgfpoint{104.54387404513889pt}{141.1373626736111pt}}
\pgfpathclose
\pgfusepath{fill,stroke}
\pgfpathmoveto{\pgfpoint{131.47056744791666pt}{123.73488203124998pt}}
\pgflineto{\pgfpoint{130.21096119791667pt}{122.0011886284722pt}}
\pgflineto{\pgfpoint{131.12241953125pt}{122.66340138888887pt}}
\pgfpathclose
\pgfusepath{fill,stroke}
\pgfpathmoveto{\pgfpoint{131.47056744791666pt}{123.73488203124998pt}}
\pgflineto{\pgfpoint{129.0843409722222pt}{122.0011886284722pt}}
\pgflineto{\pgfpoint{130.21096119791667pt}{122.0011886284722pt}}
\pgfpathclose
\pgfusepath{fill,stroke}
\pgfpathmoveto{\pgfpoint{131.47056744791666pt}{123.73488203124998pt}}
\pgflineto{\pgfpoint{128.1728826388889pt}{122.66340138888887pt}}
\pgflineto{\pgfpoint{129.0843409722222pt}{122.0011886284722pt}}
\pgfpathclose
\pgfusepath{fill,stroke}
\pgfpathmoveto{\pgfpoint{131.47056744791666pt}{123.73488203124998pt}}
\pgflineto{\pgfpoint{127.82473411458332pt}{123.73488203124998pt}}
\pgflineto{\pgfpoint{128.1728826388889pt}{122.66340138888887pt}}
\pgfpathclose
\pgfusepath{fill,stroke}
\pgfpathmoveto{\pgfpoint{131.47056744791666pt}{123.73488203124998pt}}
\pgflineto{\pgfpoint{128.1728826388889pt}{124.80637239583334pt}}
\pgflineto{\pgfpoint{127.82473411458332pt}{123.73488203124998pt}}
\pgfpathclose
\pgfusepath{fill,stroke}
\pgfpathmoveto{\pgfpoint{131.47056744791666pt}{123.73488203124998pt}}
\pgflineto{\pgfpoint{129.0843409722222pt}{125.46858515624999pt}}
\pgflineto{\pgfpoint{128.1728826388889pt}{124.80637239583334pt}}
\pgfpathclose
\pgfusepath{fill,stroke}
\pgfpathmoveto{\pgfpoint{131.47056744791666pt}{123.73488203124998pt}}
\pgflineto{\pgfpoint{130.21096119791667pt}{125.46858515624999pt}}
\pgflineto{\pgfpoint{129.0843409722222pt}{125.46858515624999pt}}
\pgfpathclose
\pgfusepath{fill,stroke}
\pgfpathmoveto{\pgfpoint{131.47056744791666pt}{123.73488203124998pt}}
\pgflineto{\pgfpoint{131.12241953125pt}{124.80637239583334pt}}
\pgflineto{\pgfpoint{130.21096119791667pt}{125.46858515624999pt}}
\pgfpathclose
\pgfusepath{fill,stroke}
\pgfpathmoveto{\pgfpoint{116.00297005208333pt}{123.10590894097221pt}}
\pgflineto{\pgfpoint{114.74337291666666pt}{121.37221493055554pt}}
\pgflineto{\pgfpoint{115.65483125pt}{122.03442769097221pt}}
\pgfpathclose
\pgfusepath{fill,stroke}
\pgfpathmoveto{\pgfpoint{116.00297005208333pt}{123.10590894097221pt}}
\pgflineto{\pgfpoint{113.61675269097222pt}{121.37221493055554pt}}
\pgflineto{\pgfpoint{114.74337291666666pt}{121.37221493055554pt}}
\pgfpathclose
\pgfusepath{fill,stroke}
\pgfpathmoveto{\pgfpoint{116.00297005208333pt}{123.10590894097221pt}}
\pgflineto{\pgfpoint{112.70528524305554pt}{122.03442769097221pt}}
\pgflineto{\pgfpoint{113.61675269097222pt}{121.37221493055554pt}}
\pgfpathclose
\pgfusepath{fill,stroke}
\pgfpathmoveto{\pgfpoint{116.00297005208333pt}{123.10590894097221pt}}
\pgflineto{\pgfpoint{112.35713671875pt}{123.10590894097221pt}}
\pgflineto{\pgfpoint{112.70528524305554pt}{122.03442769097221pt}}
\pgfpathclose
\pgfusepath{fill,stroke}
\pgfpathmoveto{\pgfpoint{116.00297005208333pt}{123.10590894097221pt}}
\pgflineto{\pgfpoint{112.70528524305554pt}{124.17738958333332pt}}
\pgflineto{\pgfpoint{112.35713671875pt}{123.10590894097221pt}}
\pgfpathclose
\pgfusepath{fill,stroke}
\pgfpathmoveto{\pgfpoint{116.00297005208333pt}{123.10590894097221pt}}
\pgflineto{\pgfpoint{113.61675269097222pt}{124.83960234375pt}}
\pgflineto{\pgfpoint{112.70528524305554pt}{124.17738958333332pt}}
\pgfpathclose
\pgfusepath{fill,stroke}
\pgfpathmoveto{\pgfpoint{116.00297005208333pt}{123.10590894097221pt}}
\pgflineto{\pgfpoint{114.74337291666666pt}{124.83960234375pt}}
\pgflineto{\pgfpoint{113.61675269097222pt}{124.83960234375pt}}
\pgfpathclose
\pgfusepath{fill,stroke}
\pgfpathmoveto{\pgfpoint{116.00297005208333pt}{123.10590894097221pt}}
\pgflineto{\pgfpoint{115.65483125pt}{124.17738958333332pt}}
\pgflineto{\pgfpoint{114.74337291666666pt}{124.83960234375pt}}
\pgfpathclose
\pgfusepath{fill,stroke}
\pgfpathmoveto{\pgfpoint{112.45289635416665pt}{120.35038446180553pt}}
\pgflineto{\pgfpoint{111.19329921874998pt}{118.61669105902777pt}}
\pgflineto{\pgfpoint{112.10475755208333pt}{119.27890381944442pt}}
\pgfpathclose
\pgfusepath{fill,stroke}
\pgfpathmoveto{\pgfpoint{112.45289635416665pt}{120.35038446180553pt}}
\pgflineto{\pgfpoint{110.06667899305555pt}{118.61669105902777pt}}
\pgflineto{\pgfpoint{111.19329921874998pt}{118.61669105902777pt}}
\pgfpathclose
\pgfusepath{fill,stroke}
\pgfpathmoveto{\pgfpoint{112.45289635416665pt}{120.35038446180553pt}}
\pgflineto{\pgfpoint{109.1552109375pt}{119.27890381944442pt}}
\pgflineto{\pgfpoint{110.06667899305555pt}{118.61669105902777pt}}
\pgfpathclose
\pgfusepath{fill,stroke}
\pgfpathmoveto{\pgfpoint{112.45289635416665pt}{120.35038446180553pt}}
\pgflineto{\pgfpoint{108.80706302083333pt}{120.35038446180553pt}}
\pgflineto{\pgfpoint{109.1552109375pt}{119.27890381944442pt}}
\pgfpathclose
\pgfusepath{fill,stroke}
\pgfpathmoveto{\pgfpoint{112.45289635416665pt}{120.35038446180553pt}}
\pgflineto{\pgfpoint{109.1552109375pt}{121.42187482638889pt}}
\pgflineto{\pgfpoint{108.80706302083333pt}{120.35038446180553pt}}
\pgfpathclose
\pgfusepath{fill,stroke}
\pgfpathmoveto{\pgfpoint{112.45289635416665pt}{120.35038446180553pt}}
\pgflineto{\pgfpoint{110.06667899305555pt}{122.08408758680554pt}}
\pgflineto{\pgfpoint{109.1552109375pt}{121.42187482638889pt}}
\pgfpathclose
\pgfusepath{fill,stroke}
\pgfpathmoveto{\pgfpoint{112.45289635416665pt}{120.35038446180553pt}}
\pgflineto{\pgfpoint{111.19329921874998pt}{122.08408758680554pt}}
\pgflineto{\pgfpoint{110.06667899305555pt}{122.08408758680554pt}}
\pgfpathclose
\pgfusepath{fill,stroke}
\pgfpathmoveto{\pgfpoint{112.45289635416665pt}{120.35038446180553pt}}
\pgflineto{\pgfpoint{112.10475755208333pt}{121.42187482638889pt}}
\pgflineto{\pgfpoint{111.19329921874998pt}{122.08408758680554pt}}
\pgfpathclose
\pgfusepath{fill,stroke}
\pgfpathmoveto{\pgfpoint{131.4785323784722pt}{134.86391953125pt}}
\pgflineto{\pgfpoint{130.21891640625pt}{133.13021640624999pt}}
\pgflineto{\pgfpoint{131.13037473958332pt}{133.79242977430553pt}}
\pgfpathclose
\pgfusepath{fill,stroke}
\pgfpathmoveto{\pgfpoint{131.4785323784722pt}{134.86391953125pt}}
\pgflineto{\pgfpoint{129.09229618055554pt}{133.13021640624999pt}}
\pgflineto{\pgfpoint{130.21891640625pt}{133.13021640624999pt}}
\pgfpathclose
\pgfusepath{fill,stroke}
\pgfpathmoveto{\pgfpoint{131.4785323784722pt}{134.86391953125pt}}
\pgflineto{\pgfpoint{128.18083784722222pt}{133.79242977430553pt}}
\pgflineto{\pgfpoint{129.09229618055554pt}{133.13021640624999pt}}
\pgfpathclose
\pgfusepath{fill,stroke}
\pgfpathmoveto{\pgfpoint{131.4785323784722pt}{134.86391953125pt}}
\pgflineto{\pgfpoint{127.83269904513888pt}{134.86391953125pt}}
\pgflineto{\pgfpoint{128.18083784722222pt}{133.79242977430553pt}}
\pgfpathclose
\pgfusepath{fill,stroke}
\pgfpathmoveto{\pgfpoint{131.4785323784722pt}{134.86391953125pt}}
\pgflineto{\pgfpoint{128.18083784722222pt}{135.93540078125pt}}
\pgflineto{\pgfpoint{127.83269904513888pt}{134.86391953125pt}}
\pgfpathclose
\pgfusepath{fill,stroke}
\pgfpathmoveto{\pgfpoint{131.4785323784722pt}{134.86391953125pt}}
\pgflineto{\pgfpoint{129.09229618055554pt}{136.59761354166665pt}}
\pgflineto{\pgfpoint{128.18083784722222pt}{135.93540078125pt}}
\pgfpathclose
\pgfusepath{fill,stroke}
\pgfpathmoveto{\pgfpoint{131.4785323784722pt}{134.86391953125pt}}
\pgflineto{\pgfpoint{130.21891640625pt}{136.59761354166665pt}}
\pgflineto{\pgfpoint{129.09229618055554pt}{136.59761354166665pt}}
\pgfpathclose
\pgfusepath{fill,stroke}
\pgfpathmoveto{\pgfpoint{131.4785323784722pt}{134.86391953125pt}}
\pgflineto{\pgfpoint{131.13037473958332pt}{135.93540078125pt}}
\pgflineto{\pgfpoint{130.21891640625pt}{136.59761354166665pt}}
\pgfpathclose
\pgfusepath{fill,stroke}
\pgfpathmoveto{\pgfpoint{124.95104999999998pt}{117.74891223958333pt}}
\pgflineto{\pgfpoint{123.69144374999999pt}{116.01520911458333pt}}
\pgflineto{\pgfpoint{124.60290208333332pt}{116.67742187499998pt}}
\pgfpathclose
\pgfusepath{fill,stroke}
\pgfpathmoveto{\pgfpoint{124.95104999999998pt}{117.74891223958333pt}}
\pgflineto{\pgfpoint{122.56481380208332pt}{116.01520911458333pt}}
\pgflineto{\pgfpoint{123.69144374999999pt}{116.01520911458333pt}}
\pgfpathclose
\pgfusepath{fill,stroke}
\pgfpathmoveto{\pgfpoint{124.95104999999998pt}{117.74891223958333pt}}
\pgflineto{\pgfpoint{121.65335546874998pt}{116.67742187499998pt}}
\pgflineto{\pgfpoint{122.56481380208332pt}{116.01520911458333pt}}
\pgfpathclose
\pgfusepath{fill,stroke}
\pgfpathmoveto{\pgfpoint{124.95104999999998pt}{117.74891223958333pt}}
\pgflineto{\pgfpoint{121.30521666666665pt}{117.74891223958333pt}}
\pgflineto{\pgfpoint{121.65335546874998pt}{116.67742187499998pt}}
\pgfpathclose
\pgfusepath{fill,stroke}
\pgfpathmoveto{\pgfpoint{124.95104999999998pt}{117.74891223958333pt}}
\pgflineto{\pgfpoint{121.65335546874998pt}{118.82039288194444pt}}
\pgflineto{\pgfpoint{121.30521666666665pt}{117.74891223958333pt}}
\pgfpathclose
\pgfusepath{fill,stroke}
\pgfpathmoveto{\pgfpoint{124.95104999999998pt}{117.74891223958333pt}}
\pgflineto{\pgfpoint{122.56481380208332pt}{119.48260624999999pt}}
\pgflineto{\pgfpoint{121.65335546874998pt}{118.82039288194444pt}}
\pgfpathclose
\pgfusepath{fill,stroke}
\pgfpathmoveto{\pgfpoint{124.95104999999998pt}{117.74891223958333pt}}
\pgflineto{\pgfpoint{123.69144374999999pt}{119.48260624999999pt}}
\pgflineto{\pgfpoint{122.56481380208332pt}{119.48260624999999pt}}
\pgfpathclose
\pgfusepath{fill,stroke}
\pgfpathmoveto{\pgfpoint{124.95104999999998pt}{117.74891223958333pt}}
\pgflineto{\pgfpoint{124.60290208333332pt}{118.82039288194444pt}}
\pgflineto{\pgfpoint{123.69144374999999pt}{119.48260624999999pt}}
\pgfpathclose
\pgfusepath{fill,stroke}
\pgfpathmoveto{\pgfpoint{107.7919269097222pt}{96.07277421875pt}}
\pgflineto{\pgfpoint{106.53232977430555pt}{94.33908081597222pt}}
\pgflineto{\pgfpoint{107.44378810763888pt}{95.00129357638889pt}}
\pgfpathclose
\pgfusepath{fill,stroke}
\pgfpathmoveto{\pgfpoint{107.7919269097222pt}{96.07277421875pt}}
\pgflineto{\pgfpoint{105.4057095486111pt}{94.33908081597222pt}}
\pgflineto{\pgfpoint{106.53232977430555pt}{94.33908081597222pt}}
\pgfpathclose
\pgfusepath{fill,stroke}
\pgfpathmoveto{\pgfpoint{107.7919269097222pt}{96.07277421875pt}}
\pgflineto{\pgfpoint{104.49425121527777pt}{95.00129357638889pt}}
\pgflineto{\pgfpoint{105.4057095486111pt}{94.33908081597222pt}}
\pgfpathclose
\pgfusepath{fill,stroke}
\pgfpathmoveto{\pgfpoint{107.7919269097222pt}{96.07277421875pt}}
\pgflineto{\pgfpoint{104.14610329861111pt}{96.07277421875pt}}
\pgflineto{\pgfpoint{104.49425121527777pt}{95.00129357638889pt}}
\pgfpathclose
\pgfusepath{fill,stroke}
\pgfpathmoveto{\pgfpoint{107.7919269097222pt}{96.07277421875pt}}
\pgflineto{\pgfpoint{104.49425121527777pt}{97.14426458333332pt}}
\pgflineto{\pgfpoint{104.14610329861111pt}{96.07277421875pt}}
\pgfpathclose
\pgfusepath{fill,stroke}
\pgfpathmoveto{\pgfpoint{107.7919269097222pt}{96.07277421875pt}}
\pgflineto{\pgfpoint{105.4057095486111pt}{97.80647734374999pt}}
\pgflineto{\pgfpoint{104.49425121527777pt}{97.14426458333332pt}}
\pgfpathclose
\pgfusepath{fill,stroke}
\pgfpathmoveto{\pgfpoint{107.7919269097222pt}{96.07277421875pt}}
\pgflineto{\pgfpoint{106.53232977430555pt}{97.80647734374999pt}}
\pgflineto{\pgfpoint{105.4057095486111pt}{97.80647734374999pt}}
\pgfpathclose
\pgfusepath{fill,stroke}
\pgfpathmoveto{\pgfpoint{107.7919269097222pt}{96.07277421875pt}}
\pgflineto{\pgfpoint{107.44378810763888pt}{97.14426458333332pt}}
\pgflineto{\pgfpoint{106.53232977430555pt}{97.80647734374999pt}}
\pgfpathclose
\pgfusepath{fill,stroke}
\pgfpathmoveto{\pgfpoint{129.92751223958334pt}{118.72446675347221pt}}
\pgflineto{\pgfpoint{128.66790538194442pt}{116.99077274305554pt}}
\pgflineto{\pgfpoint{129.57936371527776pt}{117.65298550347221pt}}
\pgfpathclose
\pgfusepath{fill,stroke}
\pgfpathmoveto{\pgfpoint{129.92751223958334pt}{118.72446675347221pt}}
\pgflineto{\pgfpoint{127.54128515624998pt}{116.99077274305554pt}}
\pgflineto{\pgfpoint{128.66790538194442pt}{116.99077274305554pt}}
\pgfpathclose
\pgfusepath{fill,stroke}
\pgfpathmoveto{\pgfpoint{129.92751223958334pt}{118.72446675347221pt}}
\pgflineto{\pgfpoint{126.62982682291666pt}{117.65298550347221pt}}
\pgflineto{\pgfpoint{127.54128515624998pt}{116.99077274305554pt}}
\pgfpathclose
\pgfusepath{fill,stroke}
\pgfpathmoveto{\pgfpoint{129.92751223958334pt}{118.72446675347221pt}}
\pgflineto{\pgfpoint{126.28168802083333pt}{118.72446675347221pt}}
\pgflineto{\pgfpoint{126.62982682291666pt}{117.65298550347221pt}}
\pgfpathclose
\pgfusepath{fill,stroke}
\pgfpathmoveto{\pgfpoint{129.92751223958334pt}{118.72446675347221pt}}
\pgflineto{\pgfpoint{126.62982682291666pt}{119.79594739583332pt}}
\pgflineto{\pgfpoint{126.28168802083333pt}{118.72446675347221pt}}
\pgfpathclose
\pgfusepath{fill,stroke}
\pgfpathmoveto{\pgfpoint{129.92751223958334pt}{118.72446675347221pt}}
\pgflineto{\pgfpoint{127.54128515624998pt}{120.45816076388887pt}}
\pgflineto{\pgfpoint{126.62982682291666pt}{119.79594739583332pt}}
\pgfpathclose
\pgfusepath{fill,stroke}
\pgfpathmoveto{\pgfpoint{129.92751223958334pt}{118.72446675347221pt}}
\pgflineto{\pgfpoint{128.66790538194442pt}{120.45816076388887pt}}
\pgflineto{\pgfpoint{127.54128515624998pt}{120.45816076388887pt}}
\pgfpathclose
\pgfusepath{fill,stroke}
\pgfpathmoveto{\pgfpoint{129.92751223958334pt}{118.72446675347221pt}}
\pgflineto{\pgfpoint{129.57936371527776pt}{119.79594739583332pt}}
\pgflineto{\pgfpoint{128.66790538194442pt}{120.45816076388887pt}}
\pgfpathclose
\pgfusepath{fill,stroke}
\pgfpathmoveto{\pgfpoint{106.13223177083333pt}{104.56180546875pt}}
\pgflineto{\pgfpoint{104.87263463541666pt}{102.82811206597222pt}}
\pgflineto{\pgfpoint{105.78409296874999pt}{103.49031571180555pt}}
\pgfpathclose
\pgfusepath{fill,stroke}
\pgfpathmoveto{\pgfpoint{106.13223177083333pt}{104.56180546875pt}}
\pgflineto{\pgfpoint{103.74600468749999pt}{102.82811206597222pt}}
\pgflineto{\pgfpoint{104.87263463541666pt}{102.82811206597222pt}}
\pgfpathclose
\pgfusepath{fill,stroke}
\pgfpathmoveto{\pgfpoint{106.13223177083333pt}{104.56180546875pt}}
\pgflineto{\pgfpoint{102.83454635416665pt}{103.49031571180555pt}}
\pgflineto{\pgfpoint{103.74600468749999pt}{102.82811206597222pt}}
\pgfpathclose
\pgfusepath{fill,stroke}
\pgfpathmoveto{\pgfpoint{106.13223177083333pt}{104.56180546875pt}}
\pgflineto{\pgfpoint{102.48639843749999pt}{104.56180546875pt}}
\pgflineto{\pgfpoint{102.83454635416665pt}{103.49031571180555pt}}
\pgfpathclose
\pgfusepath{fill,stroke}
\pgfpathmoveto{\pgfpoint{106.13223177083333pt}{104.56180546875pt}}
\pgflineto{\pgfpoint{102.83454635416665pt}{105.63329583333332pt}}
\pgflineto{\pgfpoint{102.48639843749999pt}{104.56180546875pt}}
\pgfpathclose
\pgfusepath{fill,stroke}
\pgfpathmoveto{\pgfpoint{106.13223177083333pt}{104.56180546875pt}}
\pgflineto{\pgfpoint{103.74600468749999pt}{106.29549947916665pt}}
\pgflineto{\pgfpoint{102.83454635416665pt}{105.63329583333332pt}}
\pgfpathclose
\pgfusepath{fill,stroke}
\pgfpathmoveto{\pgfpoint{106.13223177083333pt}{104.56180546875pt}}
\pgflineto{\pgfpoint{104.87263463541666pt}{106.29549947916665pt}}
\pgflineto{\pgfpoint{103.74600468749999pt}{106.29549947916665pt}}
\pgfpathclose
\pgfusepath{fill,stroke}
\pgfpathmoveto{\pgfpoint{106.13223177083333pt}{104.56180546875pt}}
\pgflineto{\pgfpoint{105.78409296874999pt}{105.63329583333332pt}}
\pgflineto{\pgfpoint{104.87263463541666pt}{106.29549947916665pt}}
\pgfpathclose
\pgfusepath{fill,stroke}
\pgfpathmoveto{\pgfpoint{102.74653836805554pt}{101.12168645833333pt}}
\pgflineto{\pgfpoint{101.48693211805555pt}{99.38799305555555pt}}
\pgflineto{\pgfpoint{102.39839045138888pt}{100.05020581597222pt}}
\pgfpathclose
\pgfusepath{fill,stroke}
\pgfpathmoveto{\pgfpoint{102.74653836805554pt}{101.12168645833333pt}}
\pgflineto{\pgfpoint{100.36031128472221pt}{99.38799305555555pt}}
\pgflineto{\pgfpoint{101.48693211805555pt}{99.38799305555555pt}}
\pgfpathclose
\pgfusepath{fill,stroke}
\pgfpathmoveto{\pgfpoint{102.74653836805554pt}{101.12168645833333pt}}
\pgflineto{\pgfpoint{99.44885295138889pt}{100.05020581597222pt}}
\pgflineto{\pgfpoint{100.36031128472221pt}{99.38799305555555pt}}
\pgfpathclose
\pgfusepath{fill,stroke}
\pgfpathmoveto{\pgfpoint{102.74653836805554pt}{101.12168645833333pt}}
\pgflineto{\pgfpoint{99.10070503472221pt}{101.12168645833333pt}}
\pgflineto{\pgfpoint{99.44885295138889pt}{100.05020581597222pt}}
\pgfpathclose
\pgfusepath{fill,stroke}
\pgfpathmoveto{\pgfpoint{102.74653836805554pt}{101.12168645833333pt}}
\pgflineto{\pgfpoint{99.44885295138889pt}{102.19316770833332pt}}
\pgflineto{\pgfpoint{99.10070503472221pt}{101.12168645833333pt}}
\pgfpathclose
\pgfusepath{fill,stroke}
\pgfpathmoveto{\pgfpoint{102.74653836805554pt}{101.12168645833333pt}}
\pgflineto{\pgfpoint{100.36031128472221pt}{102.85538046874998pt}}
\pgflineto{\pgfpoint{99.44885295138889pt}{102.19316770833332pt}}
\pgfpathclose
\pgfusepath{fill,stroke}
\pgfpathmoveto{\pgfpoint{102.74653836805554pt}{101.12168645833333pt}}
\pgflineto{\pgfpoint{101.48693211805555pt}{102.85538046874998pt}}
\pgflineto{\pgfpoint{100.36031128472221pt}{102.85538046874998pt}}
\pgfpathclose
\pgfusepath{fill,stroke}
\pgfpathmoveto{\pgfpoint{102.74653836805554pt}{101.12168645833333pt}}
\pgflineto{\pgfpoint{102.39839045138888pt}{102.19316770833332pt}}
\pgflineto{\pgfpoint{101.48693211805555pt}{102.85538046874998pt}}
\pgfpathclose
\pgfusepath{fill,stroke}
\pgfpathmoveto{\pgfpoint{122.37666024305555pt}{109.61929453124999pt}}
\pgflineto{\pgfpoint{121.11705399305553pt}{107.88559140624999pt}}
\pgflineto{\pgfpoint{122.02851232638888pt}{108.54780416666667pt}}
\pgfpathclose
\pgfusepath{fill,stroke}
\pgfpathmoveto{\pgfpoint{122.37666024305555pt}{109.61929453124999pt}}
\pgflineto{\pgfpoint{119.9904337673611pt}{107.88559140624999pt}}
\pgflineto{\pgfpoint{121.11705399305553pt}{107.88559140624999pt}}
\pgfpathclose
\pgfusepath{fill,stroke}
\pgfpathmoveto{\pgfpoint{122.37666024305555pt}{109.61929453124999pt}}
\pgflineto{\pgfpoint{119.07897543402775pt}{108.54780416666667pt}}
\pgflineto{\pgfpoint{119.9904337673611pt}{107.88559140624999pt}}
\pgfpathclose
\pgfusepath{fill,stroke}
\pgfpathmoveto{\pgfpoint{122.37666024305555pt}{109.61929453124999pt}}
\pgflineto{\pgfpoint{118.73082690972221pt}{109.61929453124999pt}}
\pgflineto{\pgfpoint{119.07897543402775pt}{108.54780416666667pt}}
\pgfpathclose
\pgfusepath{fill,stroke}
\pgfpathmoveto{\pgfpoint{122.37666024305555pt}{109.61929453124999pt}}
\pgflineto{\pgfpoint{119.07897543402775pt}{110.6907751736111pt}}
\pgflineto{\pgfpoint{118.73082690972221pt}{109.61929453124999pt}}
\pgfpathclose
\pgfusepath{fill,stroke}
\pgfpathmoveto{\pgfpoint{122.37666024305555pt}{109.61929453124999pt}}
\pgflineto{\pgfpoint{119.9904337673611pt}{111.35298793402777pt}}
\pgflineto{\pgfpoint{119.07897543402775pt}{110.6907751736111pt}}
\pgfpathclose
\pgfusepath{fill,stroke}
\pgfpathmoveto{\pgfpoint{122.37666024305555pt}{109.61929453124999pt}}
\pgflineto{\pgfpoint{121.11705399305553pt}{111.35298793402777pt}}
\pgflineto{\pgfpoint{119.9904337673611pt}{111.35298793402777pt}}
\pgfpathclose
\pgfusepath{fill,stroke}
\pgfpathmoveto{\pgfpoint{122.37666024305555pt}{109.61929453124999pt}}
\pgflineto{\pgfpoint{122.02851232638888pt}{110.6907751736111pt}}
\pgflineto{\pgfpoint{121.11705399305553pt}{111.35298793402777pt}}
\pgfpathclose
\pgfusepath{fill,stroke}
\pgfpathmoveto{\pgfpoint{106.49016145833332pt}{120.09794999999998pt}}
\pgflineto{\pgfpoint{105.23054609375pt}{118.36425598958333pt}}
\pgflineto{\pgfpoint{106.14201354166666pt}{119.02646874999998pt}}
\pgfpathclose
\pgfusepath{fill,stroke}
\pgfpathmoveto{\pgfpoint{106.49016145833332pt}{120.09794999999998pt}}
\pgflineto{\pgfpoint{104.10392526041666pt}{118.36425598958333pt}}
\pgflineto{\pgfpoint{105.23054609375pt}{118.36425598958333pt}}
\pgfpathclose
\pgfusepath{fill,stroke}
\pgfpathmoveto{\pgfpoint{106.49016145833332pt}{120.09794999999998pt}}
\pgflineto{\pgfpoint{103.19246692708333pt}{119.02646874999998pt}}
\pgflineto{\pgfpoint{104.10392526041666pt}{118.36425598958333pt}}
\pgfpathclose
\pgfusepath{fill,stroke}
\pgfpathmoveto{\pgfpoint{106.49016145833332pt}{120.09794999999998pt}}
\pgflineto{\pgfpoint{102.84432812499999pt}{120.09794999999998pt}}
\pgflineto{\pgfpoint{103.19246692708333pt}{119.02646874999998pt}}
\pgfpathclose
\pgfusepath{fill,stroke}
\pgfpathmoveto{\pgfpoint{106.49016145833332pt}{120.09794999999998pt}}
\pgflineto{\pgfpoint{103.19246692708333pt}{121.16943975694444pt}}
\pgflineto{\pgfpoint{102.84432812499999pt}{120.09794999999998pt}}
\pgfpathclose
\pgfusepath{fill,stroke}
\pgfpathmoveto{\pgfpoint{106.49016145833332pt}{120.09794999999998pt}}
\pgflineto{\pgfpoint{104.10392526041666pt}{121.83165312499999pt}}
\pgflineto{\pgfpoint{103.19246692708333pt}{121.16943975694444pt}}
\pgfpathclose
\pgfusepath{fill,stroke}
\pgfpathmoveto{\pgfpoint{106.49016145833332pt}{120.09794999999998pt}}
\pgflineto{\pgfpoint{105.23054609375pt}{121.83165312499999pt}}
\pgflineto{\pgfpoint{104.10392526041666pt}{121.83165312499999pt}}
\pgfpathclose
\pgfusepath{fill,stroke}
\pgfpathmoveto{\pgfpoint{106.49016145833332pt}{120.09794999999998pt}}
\pgflineto{\pgfpoint{106.14201354166666pt}{121.16943975694444pt}}
\pgflineto{\pgfpoint{105.23054609375pt}{121.83165312499999pt}}
\pgfpathclose
\pgfusepath{fill,stroke}
\pgfpathmoveto{\pgfpoint{110.55986605902777pt}{115.4555056423611pt}}
\pgflineto{\pgfpoint{109.3002689236111pt}{113.7218025173611pt}}
\pgflineto{\pgfpoint{110.21172725694444pt}{114.38401527777776pt}}
\pgfpathclose
\pgfusepath{fill,stroke}
\pgfpathmoveto{\pgfpoint{110.55986605902777pt}{115.4555056423611pt}}
\pgflineto{\pgfpoint{108.17364869791666pt}{113.7218025173611pt}}
\pgflineto{\pgfpoint{109.3002689236111pt}{113.7218025173611pt}}
\pgfpathclose
\pgfusepath{fill,stroke}
\pgfpathmoveto{\pgfpoint{110.55986605902777pt}{115.4555056423611pt}}
\pgflineto{\pgfpoint{107.26219036458332pt}{114.38401527777776pt}}
\pgflineto{\pgfpoint{108.17364869791666pt}{113.7218025173611pt}}
\pgfpathclose
\pgfusepath{fill,stroke}
\pgfpathmoveto{\pgfpoint{110.55986605902777pt}{115.4555056423611pt}}
\pgflineto{\pgfpoint{106.91404184027776pt}{115.4555056423611pt}}
\pgflineto{\pgfpoint{107.26219036458332pt}{114.38401527777776pt}}
\pgfpathclose
\pgfusepath{fill,stroke}
\pgfpathmoveto{\pgfpoint{110.55986605902777pt}{115.4555056423611pt}}
\pgflineto{\pgfpoint{107.26219036458332pt}{116.52698628472221pt}}
\pgflineto{\pgfpoint{106.91404184027776pt}{115.4555056423611pt}}
\pgfpathclose
\pgfusepath{fill,stroke}
\pgfpathmoveto{\pgfpoint{110.55986605902777pt}{115.4555056423611pt}}
\pgflineto{\pgfpoint{108.17364869791666pt}{117.18919965277776pt}}
\pgflineto{\pgfpoint{107.26219036458332pt}{116.52698628472221pt}}
\pgfpathclose
\pgfusepath{fill,stroke}
\pgfpathmoveto{\pgfpoint{110.55986605902777pt}{115.4555056423611pt}}
\pgflineto{\pgfpoint{109.3002689236111pt}{117.18919965277776pt}}
\pgflineto{\pgfpoint{108.17364869791666pt}{117.18919965277776pt}}
\pgfpathclose
\pgfusepath{fill,stroke}
\pgfpathmoveto{\pgfpoint{110.55986605902777pt}{115.4555056423611pt}}
\pgflineto{\pgfpoint{110.21172725694444pt}{116.52698628472221pt}}
\pgflineto{\pgfpoint{109.3002689236111pt}{117.18919965277776pt}}
\pgfpathclose
\pgfusepath{fill,stroke}
\pgfpathmoveto{\pgfpoint{108.60588289930554pt}{136.3315302951389pt}}
\pgflineto{\pgfpoint{107.34627664930555pt}{134.5978368923611pt}}
\pgflineto{\pgfpoint{108.25773498263888pt}{135.26004965277778pt}}
\pgfpathclose
\pgfusepath{fill,stroke}
\pgfpathmoveto{\pgfpoint{108.60588289930554pt}{136.3315302951389pt}}
\pgflineto{\pgfpoint{106.21964670138888pt}{134.5978368923611pt}}
\pgflineto{\pgfpoint{107.34627664930555pt}{134.5978368923611pt}}
\pgfpathclose
\pgfusepath{fill,stroke}
\pgfpathmoveto{\pgfpoint{108.60588289930554pt}{136.3315302951389pt}}
\pgflineto{\pgfpoint{105.30818836805554pt}{135.26004965277778pt}}
\pgflineto{\pgfpoint{106.21964670138888pt}{134.5978368923611pt}}
\pgfpathclose
\pgfusepath{fill,stroke}
\pgfpathmoveto{\pgfpoint{108.60588289930554pt}{136.3315302951389pt}}
\pgflineto{\pgfpoint{104.96004956597221pt}{136.3315302951389pt}}
\pgflineto{\pgfpoint{105.30818836805554pt}{135.26004965277778pt}}
\pgfpathclose
\pgfusepath{fill,stroke}
\pgfpathmoveto{\pgfpoint{108.60588289930554pt}{136.3315302951389pt}}
\pgflineto{\pgfpoint{105.30818836805554pt}{137.4030115451389pt}}
\pgflineto{\pgfpoint{104.96004956597221pt}{136.3315302951389pt}}
\pgfpathclose
\pgfusepath{fill,stroke}
\pgfpathmoveto{\pgfpoint{108.60588289930554pt}{136.3315302951389pt}}
\pgflineto{\pgfpoint{106.21964670138888pt}{138.06522430555555pt}}
\pgflineto{\pgfpoint{105.30818836805554pt}{137.4030115451389pt}}
\pgfpathclose
\pgfusepath{fill,stroke}
\pgfpathmoveto{\pgfpoint{108.60588289930554pt}{136.3315302951389pt}}
\pgflineto{\pgfpoint{107.34627664930555pt}{138.06522430555555pt}}
\pgflineto{\pgfpoint{106.21964670138888pt}{138.06522430555555pt}}
\pgfpathclose
\pgfusepath{fill,stroke}
\pgfpathmoveto{\pgfpoint{108.60588289930554pt}{136.3315302951389pt}}
\pgflineto{\pgfpoint{108.25773498263888pt}{137.4030115451389pt}}
\pgflineto{\pgfpoint{107.34627664930555pt}{138.06522430555555pt}}
\pgfpathclose
\pgfusepath{fill,stroke}
\pgfpathmoveto{\pgfpoint{107.75482144097222pt}{138.8944970486111pt}}
\pgflineto{\pgfpoint{106.49521458333332pt}{137.1607939236111pt}}
\pgflineto{\pgfpoint{107.40667291666665pt}{137.82300668402777pt}}
\pgfpathclose
\pgfusepath{fill,stroke}
\pgfpathmoveto{\pgfpoint{107.75482144097222pt}{138.8944970486111pt}}
\pgflineto{\pgfpoint{105.36859435763888pt}{137.1607939236111pt}}
\pgflineto{\pgfpoint{106.49521458333332pt}{137.1607939236111pt}}
\pgfpathclose
\pgfusepath{fill,stroke}
\pgfpathmoveto{\pgfpoint{107.75482144097222pt}{138.8944970486111pt}}
\pgflineto{\pgfpoint{104.45712690972222pt}{137.82300668402777pt}}
\pgflineto{\pgfpoint{105.36859435763888pt}{137.1607939236111pt}}
\pgfpathclose
\pgfusepath{fill,stroke}
\pgfpathmoveto{\pgfpoint{107.75482144097222pt}{138.8944970486111pt}}
\pgflineto{\pgfpoint{104.10898810763888pt}{138.8944970486111pt}}
\pgflineto{\pgfpoint{104.45712690972222pt}{137.82300668402777pt}}
\pgfpathclose
\pgfusepath{fill,stroke}
\pgfpathmoveto{\pgfpoint{107.75482144097222pt}{138.8944970486111pt}}
\pgflineto{\pgfpoint{104.45712690972222pt}{139.9659776909722pt}}
\pgflineto{\pgfpoint{104.10898810763888pt}{138.8944970486111pt}}
\pgfpathclose
\pgfusepath{fill,stroke}
\pgfpathmoveto{\pgfpoint{107.75482144097222pt}{138.8944970486111pt}}
\pgflineto{\pgfpoint{105.36859435763888pt}{140.62819105902776pt}}
\pgflineto{\pgfpoint{104.45712690972222pt}{139.9659776909722pt}}
\pgfpathclose
\pgfusepath{fill,stroke}
\pgfpathmoveto{\pgfpoint{107.75482144097222pt}{138.8944970486111pt}}
\pgflineto{\pgfpoint{106.49521458333332pt}{140.62819105902776pt}}
\pgflineto{\pgfpoint{105.36859435763888pt}{140.62819105902776pt}}
\pgfpathclose
\pgfusepath{fill,stroke}
\pgfpathmoveto{\pgfpoint{107.75482144097222pt}{138.8944970486111pt}}
\pgflineto{\pgfpoint{107.40667291666665pt}{139.9659776909722pt}}
\pgflineto{\pgfpoint{106.49521458333332pt}{140.62819105902776pt}}
\pgfpathclose
\pgfusepath{fill,stroke}
\pgfpathmoveto{\pgfpoint{116.18855512152777pt}{124.00445130208333pt}}
\pgflineto{\pgfpoint{114.9289579861111pt}{122.27075789930555pt}}
\pgflineto{\pgfpoint{115.84041631944444pt}{122.93297065972222pt}}
\pgfpathclose
\pgfusepath{fill,stroke}
\pgfpathmoveto{\pgfpoint{116.18855512152777pt}{124.00445130208333pt}}
\pgflineto{\pgfpoint{113.80233776041666pt}{122.27075789930555pt}}
\pgflineto{\pgfpoint{114.9289579861111pt}{122.27075789930555pt}}
\pgfpathclose
\pgfusepath{fill,stroke}
\pgfpathmoveto{\pgfpoint{116.18855512152777pt}{124.00445130208333pt}}
\pgflineto{\pgfpoint{112.89087031249998pt}{122.93297065972222pt}}
\pgflineto{\pgfpoint{113.80233776041666pt}{122.27075789930555pt}}
\pgfpathclose
\pgfusepath{fill,stroke}
\pgfpathmoveto{\pgfpoint{116.18855512152777pt}{124.00445130208333pt}}
\pgflineto{\pgfpoint{112.54272178819444pt}{124.00445130208333pt}}
\pgflineto{\pgfpoint{112.89087031249998pt}{122.93297065972222pt}}
\pgfpathclose
\pgfusepath{fill,stroke}
\pgfpathmoveto{\pgfpoint{116.18855512152777pt}{124.00445130208333pt}}
\pgflineto{\pgfpoint{112.89087031249998pt}{125.07594166666665pt}}
\pgflineto{\pgfpoint{112.54272178819444pt}{124.00445130208333pt}}
\pgfpathclose
\pgfusepath{fill,stroke}
\pgfpathmoveto{\pgfpoint{116.18855512152777pt}{124.00445130208333pt}}
\pgflineto{\pgfpoint{113.80233776041666pt}{125.73815442708333pt}}
\pgflineto{\pgfpoint{112.89087031249998pt}{125.07594166666665pt}}
\pgfpathclose
\pgfusepath{fill,stroke}
\pgfpathmoveto{\pgfpoint{116.18855512152777pt}{124.00445130208333pt}}
\pgflineto{\pgfpoint{114.9289579861111pt}{125.73815442708333pt}}
\pgflineto{\pgfpoint{113.80233776041666pt}{125.73815442708333pt}}
\pgfpathclose
\pgfusepath{fill,stroke}
\pgfpathmoveto{\pgfpoint{116.18855512152777pt}{124.00445130208333pt}}
\pgflineto{\pgfpoint{115.84041631944444pt}{125.07594166666665pt}}
\pgflineto{\pgfpoint{114.9289579861111pt}{125.73815442708333pt}}
\pgfpathclose
\pgfusepath{fill,stroke}
\end{pgfscope}
\end{pgfpicture}

				%\caption{Translaciones en $x$ y $z$.}
			\end{subfigure}
				\caption[Estimación de la transformación de alineación mediante los marcos de referencia]{\label{fig:cluster}Estimación de la transformación de alineación
				mediante los marcos de referencia de los puntos de cada correspondencia
				entre las capturas \texttt{bun000} y \texttt{bun045}.}
		\end{figure}



	\subsection{Refinamiento}
	Para ajustar las alineaciones iniciales se realiza una segunda alineación utilizando
	el algoritmo de ICP provisto por la biblioteca PCL.
	Se consideran únicamente las áreas solapadas, restringiendo el espacio de
	búsqueda de las correspondencias, y se minimiza la distancia entre los puntos
	de la nube a transformar hacia los planos definidos por las normales en la nube objetivo.

	%@@@
	%\clearpage
	Luego, para reducir el error propagado por cada alineación, se realiza una
	corrección de bucle ajustando la última captura para que correspondiese con la primera,
	y agregando esta transformación a las otras alineaciones ponderándola de forma
	proporcional a su posición en el bucle (algoritmo~\ref{alg:correccion_bucle}).

	\begin{algorithm}
		\begin{algorithmic}[1]
			\Function{Corrección de bucle}{nubes, N}
				\State peso $\gets \frac{1}{N-1}$
				\State error $\gets$ Alineación(desde=nubes[1], hacia=nubes[N])
				\State $[q|t]$ $\gets$ inversa(error)
				\ForAll{$K \in 1:N$}
					\State rotación $\gets$ slerp(K * peso, $q$, Identidad)
					\State translación $\gets$ K * peso * $t$
					\State nubes[K] $\gets$ transformar([rotación|translación], nubes[K])
				\EndFor
			\EndFunction
		\end{algorithmic}
		\caption[Corrección de la propagación del error de alineación]{\label{alg:correccion_bucle}Corrección de la propagación del error de alineación.}
	\end{algorithm}
