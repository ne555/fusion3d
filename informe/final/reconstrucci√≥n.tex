\chapter{Reconstrucción tridimensional}

La reconstrucción tridimensional es un proceso por el cual se combinan diversas mediciones de un objeto
para obtener un modelo tridimensional que lo reproduzca fielmente.
Este proceso puede dividirse en las siguientes etapas:
\begin{enumerate}
	\item Adquisición: en esta etapa se realizarán las mediciones del objeto,
		obteniéndose las posiciones $\{x, y, z\}$ de puntos muestreados sobre su superficie.
	\item Registración: en caso de que el método de adquisición no capture la totalidad del objeto,
		esta etapa establecerá las relaciones entre las medidas parciales
		para ajustarlas a un mismo marco de referencia.
	\item Determinación de la superficie: en esta etapa se estimará la superficie del objeto
		a partir de los puntos muestreados.
	\item Rellenado de huecos: es posible que existan zonas donde
		no se hayan obtenido muestras y, por lo tanto, la superficie presente huecos.
		Dependiendo de la aplicación, puede ser necesario rellenar estos huecos
		estimando nuevos puntos a partir de los lindantes.
		%estimando la superficie mediante los puntos lindantes.
\end{enumerate}

A continuación se describen con más detalle estas etapas y los métodos utilizados en las mismas.

\input{base_de_conocimiento/adquisición}
\input{base_de_conocimiento/registración}
\input{base_de_conocimiento/fusión}
\section{Módulo de rellenado de huecos}
	Al finalizar el algoritmo de fusión se obtuvo una malla triangular a partir
	de la información proveniente de cada vista.
	Sin embargo, esta malla no es cerrada ya que existen zonas que ninguna
	vista pudo capturar y por lo tanto carecen de puntos, produciendo huecos en la misma.

	El módulo de rellenado de huecos se encargará de estimar, de forma automática, la superficie del
	objeto en estas zonas para así obtener finalmente una malla cerrada.

	Se plantearon dos métodos:
	\begin{itemize}
		\item \emph{Advancing front}, que trabaja localmente con los puntos que forman el contorno de cada hueco.
		\item Reconstrucción de Poisson, que trabaja con todos los puntos de la nube a la vez. 
	\end{itemize}


	\subsection{Advancing front}
		En este método, cada hueco se rellenará de forma independiente a los otros,
		utilizando únicamente los puntos que conforman cada contorno para estimar
		las posiciones de los nuevos puntos.
		Tomando en cuenta esas consideraciones, se diseñó el diagrama de clases presentado en la figura~\ref{fig:filling_class},
		cuyas clases principales se describen a continuación:
		\begin{itemize}
			\item {\bfseries Relleno de huecos:} La clase se encarga de estimar
				nuevos puntos en zonas donde se carece de información (huecos)
				y triangularlos para que la \emph{Malla} sea cerrada.
			\item {\bfseries Borde:} Es una colección de puntos ordenados
				que representa un borde de un hueco en la \emph{Malla}.
		\end{itemize}

		\begin{figure}
			\Imagen{uml/hole_filling}
			\caption{\label{fig:filling_class}Diagrama de clases del módulo de registración}
		\end{figure}

		Es posible que dentro de un hueco se observen puntos que no lograron
		conectarse al resto de la malla.  Si bien estas islas nos brindan
		información de la superficie, su presencia dificulta la identificación
		del contorno de cada hueco.  Por esta razón, se eliminaron todas las
		islas al trabajar únicamente con la componente conectada que contenía
		la mayor cantidad de puntos.
		De esta forma, una arista que defina sólo un triángulo formará parte de un hueco,
		y podrá obtenerse el contorno del mismo recorriendo el grafo de conectividades. 

		\TODO{gráficos}
		El rellenado se implementó mediante una variante del método de
		\emph{advancing front}\cite{advance_front}, descripta en el
		algoritmo~\ref{alg:adv_front}.

		Los nuevos puntos insertados son elegidos de forma que los triángulos resultantes sean
		aproximadamente equiláteros, como se detalla en el algoritmo~\ref{alg:new_point}.
		Estos puntos son luego proyectados en un plano de soporte definido
		mediante las normales de los puntos del ángulo candidato.

		En caso de que el nuevo punto cayese cerca de otro ya existente, se utilizará aquel.
		Esto implica dividir en dos el hueco, y cada nueva porción se rellenará de forma independiente.
		\TODO{gráfico}

		\begin{algorithm}
			\begin{algorithmic}[1]
				\Function{Advancing front}{Malla, Contorno}
					\State AF $\gets$ Contorno
					\Repeat
					\State $\alpha = \widehat{PCN} =$ ángulo mínimo(Contorno)
					\If{$\alpha < 75^{\circ}$}
						\State Malla.agregar triángulo(P, C, N)
						\State AF.eliminar punto(C)
					\ElsIf{$\alpha < 135^{\circ}$}
						\State nuevo $\gets$ crear punto(P, C, N, $\alpha/2$)
						\State Malla.agregar punto(nuevo)
						\State Malla.agregar triángulo(nuevo, C, N)
						\State AF.insertar punto(nuevo)
					\ElsIf{$\alpha < 180^{\circ}$}
						\State nuevo $\gets$ crear punto(P, C, N, $\alpha/3$)
						\State Malla.agregar punto(nuevo)
						\State Malla.agregar triángulo(nuevo, C, N)
						\State AF.insertar punto(nuevo)
					\EndIf
					\Until $\mbox{AF} \neq \emptyset$
				\EndFunction
			\end{algorithmic}
			\caption{\label{alg:adv_front}Relleno de huecos mediante el método de \emph{advancing front}.
			Los umbrales fueron elegidos de forma de obtener triángulos con ángulos cercanos a $60^{\circ}$.}
		\end{algorithm}

		\begin{algorithm}
			\begin{algorithmic}[1]
				\Function{crear punto}{P, C, N, $\theta$}
					\State planoA $\gets$ plano(P, C, N)
					\State planoB $\gets \left\{
						\begin{tabular}{l}
							.punto $\gets$ promedio(P, C, N) \\
							.normal $\gets$ promedio(P.normal, C.normal, N.normal)
						\end{tabular}
						\right.$
					\State Q $\gets$ rotar(
						punto = N,
						origen = C,
						\Statex normal = planoA.normal,
						ángulo = $\theta$
						)
					\State \Return proyección(Q, planoB)
				\EndFunction
			\end{algorithmic}
			\caption{\label{alg:new_point}Creación del nuevo punto}
		\end{algorithm}

		Con este método se pueden rellenar agujeros pequeños, obteniéndose una malla bastante regular (figura~\ref{fig:fill_good}).
		Sin embargo, debido a la localidad con la que se generan los nuevos
		puntos, el frente puede diverger o pretender unirse a puntos que no
		forman parte del contorno del hueco, resultando una malla mal formada,
		con aristas que corresponden a más de dos caras (figura~\ref{fig:fill_bad}).
		Para evitar la divergencia es necesario definir una superficie de
		soporte considerando todo el contorno del hueco, de forma de asegurar
		que los nuevos puntos no excedan los límites del hueco.

	\begin{figure}
		\Imagen{img/fill_good}
		\caption{\label{fig:fill_good}Relleno de un hueco pequeño mediante \emph{advancing front}.}
	\end{figure}

	\begin{figure}
		\Imagen{img/fill_bad}
		\caption{\label{fig:fill_bad}Fallo en el algoritmo de \emph{advancing front}. Se intentó completar un triángulo con un punto que no pertenecía al borde.}
		\TODO{cambiar gráfico}
	\end{figure}

	\subsection{Reconstrucción de Poisson}
	%Poisson
	La clase pcl::Poisson provee algoritmos de reconstrucción basados en \cite{Kazhdan:2006:PSR:1281957.1281965}, %cite Poisson surface reconstruction
	siendo el principal parámetro la profundidad del octree utilizado,
	impactando directamente en la resolución de la malla resultante.


	Entonces, se convierte el problema en la resolución de una ecuación de Poisson de la forma:
	\[\Delta\chi \equiv \nabla \cdot\nabla\chi = \nabla \vec{n}\]

	Se realiza una discretización del dominio mediante un octree, ya que sólo
	interesa la solución de la función en la proximidad de la superficie a
	reconstruir.
	Luego se definen funciones de soporte local que aproximen una gaussiana.

	Una vez resuelto el problema en el dominio, se extrae la isosuperficie mediante una variante del método de marching cubes.

	La malla resultante presenta un hueco en la base (figura~\ref{fig:fill_poisson}).
	Ya que los puntos del contorno del hueco pertenecen a un mismo plano, es posible rellenarlo mediante el método de \emph{advancing front}.

	\begin{figure}
		\Imagen{img/fill_poisson}
		\caption{\label{fig:fill_poisson}Reconstrucción de la superficie mediante el método de \emph{Poisson}. Todos los huecos fueron rellenados a excepción de la base.}
	\end{figure}


