\documentclass{pfc}
\usepackage{amsmath}
\usepackage{amssymb}
\usepackage{amsfonts}
\usepackage{mathtools}
\usepackage{pdfpages}
\usepackage{subcaption}

\subtitle{Informe final}

\newcommand{\TODO}[1]{{\color{red}\bfseries#1}}
\newcommand{\Nota}[1]{{\color{blue}\itshape#1}}

\newcommand{\CasoUso}[1]{\bigskip%
{\bfseries Caso de uso: #1}\par}
\newcommand{\IncCU}[1]{{\bfseries Include (#1)}}
\newcommand{\UseCU}[1]{{\bfseries Use (#1)}}
\newcommand{\CUField}[2]{{\bfseries#1:} #2\par}
\newcommand{\CUNormal}{{\bfseries Curso Normal:}\par}

\newenvironment{CasoDeUso}[1]{%
	\bigskip\begin{minipage}{.9\linewidth}
		{\bfseries Caso de uso: #1}\par
}
{%
	\end{minipage}
}


\newcommand{\Requerimiento}[2]{%
	\bigskip%
	{\bfseries #1}\\%
	#2%
}

\usepackage{algorithmicx}
\usepackage{algorithm}
\usepackage[noend]{algpseudocode}
\usepackage{booktabs}


\makeatletter
 \renewcommand{\ALG@name}{Algoritmo}
\makeatother
\renewcommand{\algorithmicfunction}{}

%\newcommand{\Alerta}[1]{{\Huge\bfseries\sffamily#1}}
\newcommand{\NombreItem}[1]{{\bfseries#1:}}

\usepackage{tikz}
\usetikzlibrary{arrows,chains,matrix,positioning,scopes}

\usepackage{environ}
\makeatletter
\newsavebox{\measure@tikzpicture}
\NewEnviron{scaletikzpicturetowidth}[1]{%
  \def\tikz@width{#1}%
  \def\tikzscale{1}\begin{lrbox}{\measure@tikzpicture}%
  \BODY
  \end{lrbox}%
  \pgfmathparse{#1/\wd\measure@tikzpicture}%
  \edef\tikzscale{\pgfmathresult}%
  \BODY
}
\makeatother

\newcommand{\bigO}{\mathcal{O}}
\newcommand{\Real}{\mathbb{R}}
\DeclareMathOperator*{\argmin}{argmin}
\DeclareMathOperator{\atan}{atan}

\graphicspath{{./uml/}{./diagram/}{./graph/}}

\newcommand{\RefImagen}[1]{ (Imagen cortesía de \cite{#1})}

%\includeonly{materiales_y_métodos}
\begin{document}
\frontmatter
%\TODO{revisar tiempos verbales, ahora en pasado}
	\maketitle
	\chapter{Resumen}
Gracias a los avances en sensores de profundidad y técnicas de reconstrucción de superficies,
es posible obtener con gran precisión una nube de puntos que represente
la forma, posición y dimensiones de un objeto cualquiera.
Sin embargo, esta nube de puntos es parcial, ya que solamente refleja
la porción observable desde el punto de captura.
Por esta razón, en este proyecto se propone el desarrollo de una biblioteca de software
que implemente técnicas para combinar estas vistas parciales
y así obtener un modelo tridimensional de todo el objeto,
posibilitando su posterior construcción mediante técnicas de impresión 3D.

\Nota{
	sensores de profundidad y técnicas para determinar la profundidad.

	Hablar de prime-sense, kinect fusion, que requieren que el usuario
	tome las capturas con mucho cuidado

	Nosotros hacemos pocas capturas espaciadas, esa es la gracia
%TODO: motivación continuar lo de pancho
%explayarse en la forma de adquisición
%vistas parciales
}
	%\noindent{\bfseries Palabras clave:} fusión de mallas, imágenes de profundidad, impresión 3D, reconstrucción 3D.


	\tableofcontents
\mainmatter
	\chapter{Introducción}
	\section{Justificación}
	La impresión 3D es un método de fabricación en el cual se deposita el material (como por ejemplo, plástico o metal) en capas para producir un objeto tridimensional.
	Avances en los materiales utilizados han permitido la creación de objetos de calidad comparable a la de métodos tradicionales, con el potencial de personalizar cada producto.
	La impresión 3D ya ha demostrado ser viable para muchas aplicaciones médicas, incluyendo la creación de prótesis e implantes dentales\cite{Schubert159}. %\cite{innovations in 3d printing}
	Sin embargo, para poder realizar estas copias o modificaciones,
	se requiere del modelo geométrico del objecto, el cual puede no estar disponible.
	Para poder obtener estos modelos geométricos,
	se pueden aplicar procesos de ingeniería inversa a objetos ya existentes.
	%Describir los pasos de la ingeniería inversa: adquisición alineación, encuentro de superficie, descripción, segmentación, modelo CAD

	Un proceso crucial en estas técnicas es la adquisición de datos.
	Los distintos métodos de adquisición se diferencian según el fenómeno físico de interacción con la superficie del objeto de interés.
	De esta manera, se pueden clasificar como:
	\begin{itemize}
		\item Métodos táctiles o de contacto, donde sensores en las articulaciones de un brazo robótico determinan las coordenadas relativas de la superficie. Estos son de los más robustos, introduciendo poco ruido, pero también de los más lentos y suelen tener problemas con superficies cóncavas.
		\item Métodos de no-contacto, donde se utiliza luz (métodos ópticos), sonido u ondas electromagnéticas.
	\end{itemize}
	En particular, los métodos ópticos son los más populares con una rápida velocidad de adquisición\cite{Várady97reverseengineering}. %\cite{reverse engineering}.
	Dentro de los métodos ópticos podemos distinguir:
	\begin{itemize}
		\item Métodos activos o de luz estructurada,
			que proyectan patrones de luces conocidos sobre la escena de modo de analizar sus deformaciones;
		\item Métodos pasivos o de visión estereoscópica,
			que utilizan dos o mas cámaras y buscan correspondencias
			entre los puntos de la escena entre los puntos capturados,
			lo cual presenta gran dificultad y por eso son menos usados\cite{Várady97reverseengineering}.%\cite{reverse engineering}
	\end{itemize}

	Un avance reciente en la adquisición de datos mediante luz estructurada consiste en utilizar el sensor Kinect.
	Este dispositivo captura una imagen RGB de 32 bits y resolución de $640 \times 480$ píxeles
	junto con una imagen de profundidad de 16 bits y resolución de $320 \times 240$ píxeles\cite{MatheEstudioKinect}, %\cite{especificaciones kinect}
	procesando ambas se obtiene una nube de puntos 3D a color. %, también llamada imagen RGB-D.
	Una alternativa a Kinect para la adquisición de los mapas de profundidad se presenta en~\cite{Pancho}
	donde se trabajó con un proyector para la emisión de luz estructurada en el espectro visible y cámaras de resolución de $2592 \times 1944$ y $3264 \times 2448$ píxeles,
	obteniéndose resultados con más de dos millones de puntos.

	Sin embargo, los puntos capturados sólo reflejan una porción de la superficie del objeto,
	la parte visible desde la cámara.
	Además, la propia geometría del objeto podría generar obstrucciones, reflejos o sombras
	que imposibilitan la adquisición de datos en esas zonas («huecos»).
	Para solventar este problema, es necesario combinar múltiples capturas
	del objeto en diferentes orientaciones respecto a la cámara, de modo que
	cada parte de la superficie del objeto esté representada en al menos una captura.

	El principal problema para combinar múltiples vistas es
	encontrar las transformaciones de rotación y translación que permitan
	relacionar cada vista parcial, para luego fusionarlas y así reconstruir el objeto.

	%Entre otras investigaciones realizadas en este campo se destacan los siguientes aportes:
	Podemos nombrar algunas investigaciones realizadas en este campo:
	\begin{itemize}
		\item En~\cite{Hassanpour} %\cite{3d reconstruction faces uncontrolled}
	se hace uso de características conocidas de los objetos de estudio (caras y cabezas humanas).
		\item En~\cite{Riegler2017THREEDV} %\cite{learning depth fusion}
	se presenta un método novedoso utilizando redes convolucionales.
		\item En~\cite{Zach08fastand} %\cite{fast and high quaality fusion}ll
	se plantea un método robusto frente a ruido no gaussiano, común en la adquisición.
		\item En~\cite{automatic-3d-model-construction-for-turn-table-sequences} %\cite{model construction turn table}
	se limita las posibles transformaciones a rotaciones sobre un eje fijo.
	\end{itemize}

	Sin embargo, cabe destacar el aporte desarrollado por Microsoft mediante
	KinectFusion, un algoritmo que utiliza el sensor Kinect para reconstruir escenas tridimensionales en tiempo real.
	El usuario mueve el dispositivo por la escena mientras que el algoritmo realiza un seguimiento de la posición y orientación de la cámara. Esa información se combina con el mapa de profundidad para obtener un modelo 3D.
	En~\cite{real-time-3d-reconstruction-using-a-kinect-sensor} %\cite{kinect ed reconstruction}
	se discuten algunas de sus limitaciones:
	\begin{itemize}
		\item utiliza un alto costo computacional, requiriendo de una poderosa GPU;
		\item los movimientos de cámara deben ser lentos y pequeños;
		\item está limitado al dispositivo Kinect;
		\item algunos objetos podrían no aparecer en el mapa de profundidad debido a que absorben o reflejan demasiada luz infrarroja;
		\item no utiliza información de color.
	\end{itemize}

	Una implementación libre del algoritmo de KinectFusion se observa en el módulo KinFu,
	parte de la Point Cloud Library (PCL),
	donde se elimina la dependencia con el dispositivo Kinect,
	presenta algoritmos para trabajar a una mayor escala
	%exigiendo solamente compatibilidad con OpenNI,
	y además se permite la integración de la información de textura a la superficie final reconstruida.

	El uso de otros dispositivos además del Kinect nos permitirá lograr una reconstrucción de mayor precisión o a distinta escala pero podría traer aparejado un aumento en el costo y tiempo de cada adquisición.
	Por esto, es necesario disponer de control sobre los algoritmos de reconstrucción para adecuarlos a los dispositivos de captura y a las características del modelo 3D resultante deseado.

	En este proyecto se analizarán técnicas de fusión de mallas,
	planteando además la posibilidad de utilizar información de textura,
	para lograr un modelo tridimensional de un objeto que permita su posterior replicación
	mediante una impresora 3D.
	Se contará entonces con una herramienta que permitirá el desarrollo de aplicaciones para:
	\begin{itemize}
		\item Reproducción de biomodelos para que estudiantes de medicina y veterinaria realicen prácticas sobre los mismos.
			A este respecto, la Facultad de Ciencias Médicas de la Universidad Nacional del Litoral expresó un interés particular.
			%mirai 3d, empresa argentina, adquiere los biomodelos mediante tomografía axial computarizada (TAC)
		\item Digitalización de esculturas, lo que permitirá salvaguardarlas y facilitará el acceso, aunque indirecto, a las mismas.
			Puede mencionarse el \emph{Digital Michelangelo Project} de la Universidad de Stanford, que tiene como objetivo crear un repositorio de modelos 3D de alta calidad de las esculturas y la arquitectura de Miguel Ángel.
			%catedral de Notre-Dame, escultura de Néstor
		\item Digitalización de ambientes y objetos para utilizarlos en aplicaciones de realidad virtual o de diseño.
	\end{itemize}

	\section{Objetivos}
		\subsection{General}
			Diseñar y desarrollar una biblioteca de funciones que permita
			la reconstrucción de una malla 3D cerrada a partir de mallas de superficies parciales.
		\subsection{Específicos}
		\begin{itemize}
			%Falta un objetivo específico relacionado con el análisis de técnica de fusión de mallas.
			\item Desarrollar algoritmos que determinen las posiciones relativas de cada malla parcial.
			\item Desarrollar algoritmos de fusión de las mallas.
			\item Desarrollar algoritmos para rellenado de huecos producto de obstrucciones.
			\item Evaluar el desempeño de cada bloque desarrollado y realizar ajustes.
		\end{itemize}

	\section{Alcance}
		\subsection{Alcances Funcionales}
		\begin{itemize}
			\item La herramienta deberá generar una malla 3D cerrada (\emph{watertight}).
		\end{itemize}
		\subsection{Alcances No Funcionales}
		\begin{itemize}
			\item La herramienta deberá ser robusta frente al ruido producido por los sensores.
			\item La herramienta no impondrá topologías específicas para los objetos a reconstruirse.
			\item No se pretenderá el funcionamiento de los algoritmos en tiempo real.
			\item La herramienta será de código libre.
			\item La herramienta será multiplataforma.
		\end{itemize}
		\subsection{Supuestos}
		\begin{itemize}
			\item Se contará con un repositorio de mallas tridimensionales con información de textura.
		\end{itemize}

	\chapter{Reconstrucción tridimensional}

La reconstrucción tridimensional es un proceso por el cual se combinan diversas mediciones de un objeto
para obtener un modelo tridimensional que lo reproduzca fielmente.
Este proceso puede dividirse en las siguientes etapas (figura~\ref{fig:proceso_reconstruccion_3d}):
\begin{enumerate}
	\item Adquisición: en esta etapa se realizarán las mediciones del objeto,
		obteniéndose las posiciones $\{x, y, z\}$ de puntos muestreados sobre su superficie.
	\item Registración: en caso de que el método de adquisición no capture la totalidad del objeto,
		esta etapa establecerá las relaciones entre las medidas parciales
		para ajustarlas a un mismo marco de referencia.
	\item Determinación de la superficie: en esta etapa se estimará la superficie del objeto
		a partir de los puntos muestreados.
	\item Rellenado de huecos: es posible que existan zonas donde
		no se hayan obtenido muestras y, por lo tanto, la superficie presente huecos.
		Dependiendo de la aplicación, puede ser necesario rellenar estos huecos
		estimando nuevos puntos a partir de los lindantes.
		%estimando la superficie mediante los puntos lindantes.
\end{enumerate}

\begin{figure}
	\Imagen{img/reconstrucción}
	\caption{\label{fig:proceso_reconstruccion_3d}Proceso de reconstrucción.
	De izquierda a derecha y de arriba a abajo se tiene:
	objeto a reconstruir, adquisición de capturas parciales,
	registración de cada captura a un marco de referencia global,
	determinación de la superficie,
	rellenado de huecos.}
\end{figure}

A continuación se describen con más detalle estas etapas y los métodos utilizados en las mismas.

\section{Método de adquisición: luz estructurada}
\Nota{requerido por preproceso.tex}

\TODO{Adquisición ¿qué es?}

\TODO{métodos ópticos}

\Nota{luz estructurada}
Se proyectan patrones de luces conocidos sobre la escena para que al analizar
sus deformaciones al impactar en sobre las superficies nos permitan calcular la
profundidad de los objetos dentro de la escena.
\Nota{¿muevo lo del ruido del preproceso acá?}

\section{Registración}
Una vez finalizada la etapa de adquisición, se dispone de una nube de puntos
que representa la porción observada del objeto.
Para lograr una reconstrucción total es necesario combinar múltiples capturas
variando la posición relativa cámara-objeto.
Contar con un dispositivo que nos permita definir con precisión
tanto la posición como orientación de la cámara es altamente costoso,
por esta razón, en la etapa de registración se deben estimar las transformaciones que ubiquen cada
captura en un sistema de referencia global, de forma que las zonas comunes encajen perfectamente.


Esta operación se realiza usualmente entre dos capturas,
buscando las transformaciones de translación y rotación que ubiquen
la captura de partida en el marco de referencia de la captura objetivo.
De esta manera, el problema general cuenta con seis grados de libertad.
A continuación se presenta el caso en que
la transformación de alineación es pequeña,
más adelante se discute el caso de incrementos mayores,
y finalmente se extiende el proceso para abarcar $n$ capturas.

%esto va en $N$ capturas
%Para extender este proceso a $n$ capturas, puede alinearse $C_0$ con $C_1$,
%$C_1$ con $C_2$, $C_2$ con $C_3$, etc.,
%sin embargo, el error de registración se propagará en cada paso.


\subsection{Perturbaciones pequeñas: algoritmo iterativo del punto más cercano (ICP)}
Antes de describir este algoritmo, se planteará una versión simplificada del problema
que permitirá establecer ciertas definiciones.

Se supone que se dispone de una nube $A$ a la cual se le aplica
una transformación de translación y rotación arbitraria,
y luego se perturba levemente
las posiciones de los puntos transformados,
obteniéndose la nube $B$.
Se observa que ambas nubes poseen la misma cardinalidad y se pueden establecer las relaciones de origen a destino
$a_j \to b_j$ para cada punto $a_j \in A$.
Entonces, se puede definir el error de alineación como:
\[
	\text{Error} = \frac{1}{N} \sum_{j=1}^n || b_j - T \left(a_j\right) ||
\]
y buscar la transformación $T$ que minimice este error.

Existe una solución cerrada para este problema, obteniéndose la rotación al
calcular los eigenvectores de una matriz simétrica de $4\times4$,
cuyos detalles pueden consultarse en \cite{Horn87closed-formsolution}.

Sin embargo, estas suposiciones son demasiado fuertes.
Al trabajar con dos capturas realizadas desde distintas vistas, el solapamiento no será total,
ya que habrá puntos que serán observables en tan sólo uno de las vistas.
Por lo tanto, primero es necesario determinar cuáles son los puntos comunes a ambas capturas,
y, además, definir las relaciones de origen-destino entre esos puntos comunes.
El algoritmo iterativo del punto más cercano (ICP) supone que las nubes se encuentran
lo suficientemente cerca como para establecer estas correspondencias mediante las coordenadas de los puntos.
El algoritmo se describe de la siguiente manera:
\begin{enumerate}
	\item Obtener las correspondencias:
		Para cada punto $a_j \in A$ buscar el punto más cercano en la nube $B$.
		Si la distancia entre esos puntos supera un umbral $d_{\text{max}}$,
		se considera que $a_j$ no pertenece a la zona común y no es considerado en esta iteración.
	\item Calcular la transformación que alinee los pares de puntos de la zona común.
	\item Aplicar la transformación.
	\item Repetir el proceso hasta que el error de alineación esté por debajo de un umbral $\tau$.
\end{enumerate}%\cite{conf/rss/SegalHT09}
Si bien el algoritmo converge a un mínimo local, puede que este no sea el mínimo global buscado.
Debe cumplirse la suposición de cercanía para obtener una correcta registración.\cite{regBesl92}

%generalized conf/rss/SegalHT09
En \cite{chen-medoni} se presenta una variante del algoritmo de ICP, llamada ICP {punto-a-plano},
que resulta más robusta y precisa al trabajar con nubes de puntos en 2.5D.
Una vez establecida una correspondencia $a_j \to b_j$, si bien no se trata del mismo punto,
se realiza la suposición de que pertenecen al mismo plano.
Por esta razón, podemos decir que conocemos con mucha confianza la posición del punto
a lo largo de su normal, pero no su ubicación en el plano.
Para reflejar esto, se cambia la métrica del error, proyectando el punto transformado sobre la normal
del punto destino (figura~\ref{fig:point_to_plane}).
\[
	\text{Error} = \frac{1}{N} \sum_{j=1}^n || n_j \cdot \left( b_j - T \left(a_j\right) \right) ||
\]

\begin{figure}
	\centering
	\input{diagram/icp_point_to_plane.pdf_tex}
	\caption[Medidas de error punto-a-plano entre dos superficies]{\label{fig:point_to_plane}Medidas de error punto-a-plano entre dos superficies
	(Imagen cortesía de \cite{icp_point_to_plane}).}
\end{figure}


\subsection{Distancias mayores: búsqueda de correspondencias mediante vectores de características}
Cuando la transformación de alineación requerida para alinear las dos nubes de puntos
no es lo suficientemente pequeña, no se obtiene una buena aproximación al establecer las correspondencias
utilizando únicamente las posiciones de los puntos.
Una posibilidad es emplear métodos semiautomáticos, donde el usuario provee de una alineación inicial
de las nubes o determina las correspondencias seleccionando puntos.
Sin embargo, este es un método lento y engorroso,
por lo que surge la necesidad de plantear un algoritmo completamente automático.

Para hallar las correspondencias de manera automática, se calcula un vector de características o \emph{descriptor}
sobre una vecindad de cada punto. Al comparar los descriptores con alguna medida de distancia,
si la distancia es pequeña (cercana a $0$), entonces las vecindades son similares
y podría plantearse una correspondencia entre los puntos.
Un buen descriptor deberá poseer las siguientes características:
\begin{itemize}
	\item Ser discriminante, es decir, la distancia entre los descriptores debe reflejar
		la distancia entre las superficies de las vecindades.
		Esto implica que superficies similares localmente produces descriptores parecidos y
		en superficies diferentes la distancia entre descriptores será grande (figura~\ref{fig:descriptor_matriz_confusion}).
	\item Ser invariante respecto a translaciones y rotaciones.
	\item Ser robusto respecto al ruido de muestreo.
	\item Ser robusto respecto a la densidad de muestreo.
	\item Ser robusto respecto a la ausencia de muestras debido a oclusiones.
\end{itemize}

\begin{figure}
	\centering
	\input{diagram/fpfh_confusión.pdf_tex}
	\caption[Matriz de confusión del descriptor FPFH]{\label{fig:descriptor_matriz_confusion}Matriz de confusión del descriptor FPFH para distintas clases de superficies utilizando la distancia $\chi^2$\RefImagen{RusuDoctoralDissertation}.}
\end{figure}

Además de la elección del descriptor, un parámetro importante a determinar es
el tamaño de la vecindad que éste representará.
En general, la vecindad de un punto $p$ incluye a todos aquellos puntos $q$
que se encuentran dentro de una esfera de radio $r$ centrada en $p$:
\[ |p - q| \leq r \]
Si el valor de $r$ es demasiado pequeño, el descriptor perderá su poder discriminante
o incluso no podrá calcularse al no contar con la cantidad de vecinos mínimos necesarios.
En cambio, si el valor de $r$ es demasiado grande, se corre el riesgo de considerar
dentro de la vecindad puntos pertenecientes a otra superficie, distorsionando el valor del descriptor.
La determinación de un valor adecuado de $r$ limita la posibilidad de obtener un método completamente
automático para el cálculo de los descriptores\cite{RusuDoctoralDissertation}.
 %This issue is of extreme importance and constitutes a limiting factor in the automatic estimation (i.e., without user given thresholds) of a point feature representation.


A continuación se describe el proceso de construcción de dos descriptores
que han mostrado buenos resultados en problemas de registración\cite{Rusu:2009:FPF:1703435.1703733}.

\subsubsection{Estimación de normales}
Para poder describir la geometría de una superficie,
primero debe estimarse su orientación en el sistema de coordenadas, es decir, estimar su normal.
Una opción es aproximar la normal mediante la estimación de un plano tangente a la superficie,
lo que transforma el problema a ajustar un plano a la vecindad de $p$
mediante el método de mínimos cuadrados. 
Representando el plano mediante un punto $x_j$ y un vector normal $n_j$,
la distancia de un punto $q$ de la vecindad a este plano queda definida como
$d = (q_j - x_j) n_j$.
Para minimizar la sumatoria del cuadrado de todas las distancias, se tiene que:
\[x_j = \frac{1}{N} \sum_{k=1}^{N} q^{(k)}_j \]
para los $N$ puntos de la vecindad de $p$.
Y para obtener $n_j$ se analizan los eigenvalores y eigenvectores de la matriz de covarianza de la vecindad:
\[ C_{jk} = \sum_{r=1}{N} (q^{(r)}_j - x_j) (q^{(r)}_k - x_k) \]

Se observa que $C_{jk}$ es una matriz simétrica y por lo tanto todos sus eigenvalores son reales
$\lambda_j \in \Real$.
El eigenvector que corresponde al eigenvalor de menor magnitud corresponde a la normal del plano tangente,
es decir a $+n_j$ o a $-n_j$\cite{10.1109/34.334391}.
Si bien no puede determinarse matemáticamente el signo de $n_j$,
puede resolverse la ambigüedad al considerar que la nube de puntos que se obtiene del proceso de adquisición
es 2.5D y se conoce la ubicación de la cámara $v_j$ en un sistema de referencia local.
Por lo tanto, debe cumplirse que:
\[n_j v_j > 1\]
invirtiendo el signo de $n_j$ en caso de ser necesario.

\subsubsection{Histograma de características del punto (PFH)}
Dado que las normales no son invariantes a rotaciones y no capturan mucho
detalle de la superficie, no presentan un gran poder discriminante.
Por esta razón, no resulta adecuado utilizarlas como único descriptor. 
Sin embargo, al representar las relaciones entre todos los puntos de la vecindad y sus normales,
pueden capturarse con más detalle las variaciones de la superficie,
aumentando la representatividad del descriptor. 
Esta es la idea principal del histograma de características del punto (PFH) presentado en \cite{RusuDoctoralDissertation}.

El vector de características de PFH es un histograma de los valores de tres ángulos
y una distancia calculados para cada par de puntos y sus normales en la vecindad de $p$.
Para asegurar la replicabilidad del descriptor, se define un marco de referencia de Darboux
determinado unívocamente de la siguiente manera:
\begin{itemize}
	\item Dado el par de puntos $p_j$ y $p_k$, se determina el origen $p_s$ del marco de referencia como:
		\[
			\vec{n}_j \cdot (p_k - p_j) > \vec{n}_k \cdot (p_j - p_k)
			\begin{cases}
				\text{Verdadero: }& p_s = p_j \quad p_t = p_k\\
				\text{Falso: }& p_s = p_k \quad p_t = p_j\\
			\end{cases}
		\]
	\item A partir de este origen, se determinan los tres vectores que definen el marco de referencia:
\[
	\begin{cases}
		u =& n_s \\
		v =& u \times \displaystyle\frac{p_t - p_s}{|p_t - p_s|} \\
		w =& u \times v \\
	\end{cases}
\]
\end{itemize}
Y entonces se procede a calcular las siguientes relaciones para estos puntos:
	\begin{align*}
		d        &= |p_t - p_s| \\
		\alpha   &= v \cdot n_t \\
		\phi     &= u \cdot \frac{p_t - p_s}{d}\\
		\theta   &= \atan(w \cdot n_t, u \cdot n_t)
	\end{align*}
Sin embargo, la distancia entre puntos $d$ no es de importancia para escaneos 2.5D,
ya que la distancia entre vecinos se incrementa con la distancia a la cámara,
y suele ser beneficioso omitirla en estos casos \cite{RusuDoctoralDissertation}.

%propio
Para entender el poder discriminante del descriptor PFH, es preciso comprender el significado
de los ángulos calculados entre los pares de puntos (ver figura~\ref{fig:pfh_angulos}).
Considerando el plano tangente a $p_s$, el ángulo $\phi$ mide la separación de la posición de $p_t$ respecto a este plano.
Luego, considerando el plano $wu$, donde se encuentran $p_s$, $p_t$ y $n_s$, y cuya normal es $v$:
\begin{itemize}
	\item el ángulo $\alpha$ mide el desvío de la normal $n_t$ respecto a este plano,
	\item el ángulo $\theta$ mide la distancia entre las proyecciones de las normales sobre este plano.
\end{itemize}
De esta manera, se logra describir con buen detalle las características de la superficie en la vecindad de cada punto.



\begin{figure}
	\centering
	\input{diagram/marco_ref_fpfh.pdf_tex}
	\caption[Representación gráfica del marco de Darboux]{\label{fig:pfh_marco_referencia}Representación gráfica del marco de Darboux y las características calculadas por el descriptor PFH \RefImagen{RusuDoctoralDissertation}.}
\end{figure}

\begin{figure}
	\centering
	\input{diagram/fpfh_meaning.pdf_tex}
	\caption[Representación gráfica de los ángulos $\phi$ y $\theta$ calculados por el descriptor PFH]{\label{fig:pfh_angulos}Representación gráfica de los ángulos $\phi$ y $\theta$ calculados por el descriptor PFH.}
\end{figure}

\subsubsection{PFH rápido (FPFH)}
Debido a que para obtener el descriptor PFH de un punto es necesario calcular
las relaciones entre todos los pares de puntos de su vecindad,
el orden de ejecución para una nube con $n$ puntos es $\bigO(nk^2)$,
donde $k$ es la cantidad de puntos en la vecindad,
y por esta razón el cálculo de este descriptor puede convertirse
en uno de los cuellos de botella del proceso.

Para solventar este problema, en \cite{Rusu:2009:FPF:1703435.1703733} se propone
una simplificación del descriptor PFH, llamada PFH rápido o simplemente FPFH,
que nos permite reducir el tiempo de ejecución a $\bigO(nk)$
manteniendo el poder descriptivo de PFH.

El cálculo de este nuevo descriptor se realiza de la siguiente manera:
\begin{enumerate}
	\item Calcular por cada punto $p$ y sus vecinos las tuplas $\langle \alpha, \phi, \theta \rangle$.
		Llamamos a estas características PFH simplificado (SPFH).
	\item Realizar un promedio ponderado de los valores del SPFH de $p$ y sus vecinos:
		\[
			FPFH(p) = SPFH(p) + \frac{1}{N} \sum_{j=1}^{N} \frac{1}{w_j} SPFH(p_j)
		\]
		donde $w_j$ es una medida de distancia entre $p$ y su vecino.
\end{enumerate}
Las principales diferencias respecto a PFH son:
\begin{itemize}
	\item FPFH no interconecta todos los vecinos de $p$.
	\item FPFH puede incluir puntos fuera del radio $r$ de vecindad, pero a no más de $2r$.
\end{itemize}


\subsubsection{Puntos salientes (\emph{keypoints})}
Al describir las características de un buen descriptor, mencionamos la capacidad discriminante del mismo.
Se presenta una gran diferencia entre descriptores de superficies diferentes,
pero la distancia es cercana a $0$ en superficies parecidas,
y de esta forma se pueden establecer las correspondencias entre los puntos de dos nubes
al comparar sus descriptores.

Al analizar una zona homogénea en una nube, como, por ejemplo, un plano,
todos los puntos presentarán descriptores parecidos.
Surgen problemas al intentar establecer una correspondencia entre estos puntos
y los presentes en la otra nube, ya que todos son candidatos igualmente válidos.

Para solucionar este problema, es necesario identificar puntos salientes (\emph{keypoints}) 
en la nube, para entonces establecer las correspondencias entre los descriptores de estos keypoints.
Si bien un punto cualquiera será considerado como saliente o no dependiendo del detector utilizado,
un buen detector presentará las siguientes características:
\begin{itemize}
	\item Dispersión: un subconjunto pequeño de los puntos de la nube serán considerados como keypoints.
	\item Repetibilidad: un keypoint deberá ser detectado en la misma ubicación
		sobre el objeto sin importar la posición de la cámara al realizar la captura.
	\item Distinguibilidad: la superficie en la vecindad del keypoint
		presentará características únicas que serán capturadas por un descriptor.
\end{itemize}

Como ejemplos de detectores se puede mencionar el detector de esquinas Harris,
utilizado usualmente en imágenes 2D, adaptado al espacio tridimensional
traduciendo cambios en la intensidad de píxeles por ángulos entre las normales de puntos vecinos;
y el detector ISS, que analiza los eigenvalores de la matriz de covarianza en la vecindad de un punto.


\subsection{Registración de múltiples capturas}
Previamente se describieron métodos de registración entre dos capturas,
sin embargo, un objeto difícilmente pueda ser observable completamente desde tan sólo dos posiciones.
Por lo tanto, para lograr la reconstrucción tridimensional se tendrá una lista de capturas
${A_1, A_2, \ldots, A_n}$ correspondientes a distintas posiciones relativas cámara-objeto.
Surge entonces el problema de cómo obtener la lista de transformaciones ${T_1, T_2, \ldots, T_n}$
que las alinee a todas correctamente.

Una opción es utilizar una captura $B$ que presente zonas comunes con todas las capturas $A_j$
y entonces calcular la registración $A_j \to B$.
Esta captura $B$ puede obtenerse mediante un escaneo cilíndrico: el objeto se
ubica sobre una superficie que gira lentamente a una velocidad controlada
mientras que se proyecta una línea vertical sobre el objeto y se miden las
distancias de los puntos que caen sobre esta línea.

Una alternativa es realizar la registración entre pares sucesivos
$A_2 \to A_1$,
$A_3 \to A_2$,
\ldots,
$A_{n} \to A_{n-1}$.
Sin embargo, los errores producidos se propagarán con cada
captura incorporada, siendo especialmente apreciables al completar una
vuelta alrededor del objeto (figura~\ref{fig:error_bucle}).
Para corregir este error, se ajustará la alineación de la captura que cierre el bucle,
propagando luego esta corrección a las demás.


	\begin{figure}
		\Imagen{diagram/error_bucle_inhand}
		\caption[Visualización del error de bucle]{\label{fig:error_bucle}Visualización del error de bucle. Errores de tan sólo $1^{\circ}$
		en cada registración producen una discrepancia considerable
		donde el modelo debería cerrarse (círculo rojo) \RefImagen{5457479}.}
	\end{figure}

\section{Módulo de fusión}
	%¿citas?
	%en el de loop correction (surfel)
	El módulo de fusión tiene como objetivo obtener un modelo que
	describa la geometría del objeto escaneado.
	Para esto, el modelo se inicializa a partir de una captura cualquiera
	y se procesan una a una las capturas correspondientes a las otras vistas.
	En cada nueva vista se agrega información del objeto en zonas que no eran antes visibles
	y además se confirma o refuta la información ya presente en el modelo.
	Al combinar esta información, el módulo de fusión obtiene finalmente una malla
	que represente a la totalidad del objeto.

	\subsection{Método}
	Se utiliza una representación de \emph{surfel} para cada punto similar a la propuesta en \cite{5457479} %ref in-hand scanning
	debido a la facilidad de implementación de las funciones de actualización de los puntos.

	Cada surfel tiene asociado un valor de confianza y las vistas en las que
	fue observado.  El valor de confianza nos indica la probabilidad de que sus
	valores de posición y normal no sean producto de un error de muestreo.
	Este valor se inicializa según el ángulo de su normal respecto a la línea
	de la cámara (figura~\ref{fig:confianza_surfel}).
	%y a su distancia al centro de la captura %terminé usando sólo las normales



	El algoritmo~\ref{alg:surfel} describe el agregado de una nueva vista.
	Por cada punto de la vista se busca qué surfel lo contiene y se actualiza
	su posición y normal ponderando según el nivel de confianza.
	\begin{eqnarray*}
		\hat{P}_k = \frac{\sum_{j} \alpha_j P^{(j)}_k}{\sum_{j} \alpha_j} \\
		\hat{n}_k = \frac{\sum_{j} \alpha_j n^{(j)}_k}{\sum_{j} \alpha_j}
	\end{eqnarray*}
	En caso de que haya caído fuera del dominio de la reconstrucción actual, se
	considera que es un nuevo punto y se lo agrega.
	Si la distancia de proximidad elegida es demasiado pequeña,
	nunca se realizará la actualización ya que todos los puntos serán considerados como nuevos surfels.
	En cambio, si la distancia es muy grande, se considerarán puntos
	que deberían haber sido descartados como atípicos.


	Al terminar de procesar todas las vistas, habrá surfels que hayan sido
	observados desde sólo una posición y tengan un nivel de confianza bajo.
	Estos son considerados como ruido y eliminados de la
	reconstrucción.

	Finalmente, se procede a la obtención de una triangulación de los surfels.
	Para esto se empleó el algoritmo de \emph{Greedy Projection Triangulation} que provee PCL,
	el cual realiza una proyección de la vecindad de un punto en el plano definido por su normal
	y procede a conectar los puntos de modo que los triángulos resultantes
	respeten umbrales de longitud y ángulos definidos por el usuario.

	\begin{figure}
		\Imagen{img/bunny_confianza}
		\caption[Visualización de los valores de confianza de los súrfeles]{\label{fig:confianza_surfel}Visualización de los valores de confianza de los súrfeles para la captura \texttt{bun000}.}
	\end{figure}


	\begin{algorithm}
		\begin{algorithmic}[1]
			\Function{Agregar vista}{vista, reconstrucción}
				\ForAll{$p \in \mbox{vista.puntos}$}
					\State surfel $\gets$ reconstrucción.buscar(p)
					\If {surfel = $\emptyset$}
						\State reconstrucción.agregar(p)
					\Else
						\State surfel.actualizar(p)
					\EndIf
				\EndFor
			\EndFunction
		\end{algorithmic}
		\caption[Actualización de la reconstrucción al agregar una nueva vista]{\label{alg:surfel}Actualización de la reconstrucción al agregar una nueva vista.}
	\end{algorithm}

	\begin{figure}
		\Imagen{img/bun_fusion}

		%\Imagen{img/bun_fusion_conf}
		\caption[Superficie reconstruida]{\label{fig:surface}Superficie reconstruida. En verde se destacan los bordes de los huecos.}
	\end{figure}

\section{Módulo de rellenado de huecos}
	Al finalizar el algoritmo de fusión se obtuvo una malla triangular a partir
	de la información proveniente de cada vista.
	Sin embargo, esta malla no es cerrada ya que existen zonas que ninguna
	vista pudo capturar y por lo tanto carecen de puntos, produciendo huecos en la misma.

	El módulo de rellenado de huecos se encargará de estimar, de forma automática, la superficie del
	objeto en estas zonas para así obtener finalmente una malla cerrada.

	Se plantearon dos métodos:
	\begin{itemize}
		\item \emph{Advancing front}, que trabaja localmente con los puntos que forman el contorno de cada hueco.
		\item Reconstrucción de Poisson, que trabaja con todos los puntos de la nube a la vez. 
	\end{itemize}


	\subsection{Advancing front}
		En este método, cada hueco se rellenará de forma independiente a los otros,
		utilizando únicamente los puntos que conforman cada contorno para estimar
		las posiciones de los nuevos puntos.
		Tomando en cuenta esas consideraciones, se diseñó el diagrama de clases presentado en la figura~\ref{fig:filling_class},
		cuyas clases principales se describen a continuación:
		\TODO{FIX}

		Es posible que dentro de un hueco se observen puntos que no lograron
		conectarse al resto de la malla.  Si bien estas islas nos brindan
		información de la superficie, su presencia dificulta la identificación
		del contorno de cada hueco.  Por esta razón, se eliminaron todas las
		islas al trabajar únicamente con la componente conectada que contenía
		la mayor cantidad de puntos.
		De esta forma, una arista que defina sólo un triángulo formará parte de un hueco,
		y podrá obtenerse el contorno del mismo recorriendo el grafo de conectividades. 

		\TODO{gráficos}
		El rellenado se implementó mediante una variante del método de
		\emph{advancing front}\cite{advance_front}, descripta en el
		algoritmo~\ref{alg:adv_front}.

		Los nuevos puntos insertados son elegidos de forma que los triángulos resultantes sean
		aproximadamente equiláteros, como se detalla en el algoritmo~\ref{alg:new_point}.
		Estos puntos son luego proyectados en un plano de soporte definido
		mediante las normales de los puntos del ángulo candidato.

		En caso de que el nuevo punto cayese cerca de otro ya existente, se utilizará aquel.
		Esto implica dividir en dos el hueco, y cada nueva porción se rellenará de forma independiente.
		\TODO{gráfico}

		\begin{algorithm}
			\begin{algorithmic}[1]
				\Function{Advancing front}{Malla, Contorno}
					\State AF $\gets$ Contorno
					\Repeat
					\State $\alpha = \angle PCN =$ ángulo mínimo(Contorno)
					\If{$\alpha < 75^{\circ}$}
						\State Malla.agregar triángulo(P, C, N)
						\State AF.eliminar punto(C)
					\ElsIf{$\alpha < 135^{\circ}$}
						\State nuevo $\gets$ crear punto(P, C, N, $\alpha/2$)
						\State Malla.agregar punto(nuevo)
						\State Malla.agregar triángulo(nuevo, C, N)
						\State AF.insertar punto(nuevo)
					\ElsIf{$\alpha < 180^{\circ}$}
						\State nuevo $\gets$ crear punto(P, C, N, $\alpha/3$)
						\State Malla.agregar punto(nuevo)
						\State Malla.agregar triángulo(nuevo, C, N)
						\State AF.insertar punto(nuevo)
					\EndIf
					\Until $\mbox{AF} \neq \emptyset$
				\EndFunction
			\end{algorithmic}
			\caption{\label{alg:adv_front}Relleno de huecos mediante el método de \emph{advancing front}.
			Los umbrales fueron elegidos de forma de obtener triángulos con ángulos cercanos a $60^{\circ}$.}
		\end{algorithm}

		\begin{algorithm}
			\begin{algorithmic}[1]
				\Function{crear punto}{P, C, N, $\theta$}
					\State planoA $\gets$ plano(P, C, N)
					\State planoB $\gets \left\{
						\begin{tabular}{l}
							.punto $\gets$ promedio(P, C, N) \\
							.normal $\gets$ promedio(P.normal, C.normal, N.normal)
						\end{tabular}
						\right.$
					\State Q $\gets$ rotar(
						punto = N,
						origen = C,
						\Statex normal = planoA.normal,
						ángulo = $\theta$
						)
					\State \Return proyección(Q, planoB)
				\EndFunction
			\end{algorithmic}
			\caption{\label{alg:new_point}Creación del nuevo punto}
		\end{algorithm}

		Con este método se pueden rellenar agujeros pequeños, obteniéndose una malla bastante regular (figura~\ref{fig:fill_good}).
		Sin embargo, debido a la localidad con la que se generan los nuevos
		puntos, el frente puede diverger o pretender unirse a puntos que no
		forman parte del contorno del hueco, resultando una malla mal formada,
		con aristas que corresponden a más de dos caras (figura~\ref{fig:fill_bad}).
		Para evitar la divergencia es necesario definir una superficie de
		soporte considerando todo el contorno del hueco, de forma de asegurar
		que los nuevos puntos no excedan los límites del hueco.

	\begin{figure}
		\Imagen{img/fill_good}
		\caption{\label{fig:fill_good}Relleno de un hueco pequeño mediante \emph{advancing front}.}
	\end{figure}

	\begin{figure}
		\Imagen{img/fill_bad}
		\caption{\label{fig:fill_bad}Fallo en el algoritmo de \emph{advancing front}. Se intentó completar un triángulo con un punto que no pertenecía al borde.}
		\TODO{cambiar gráfico}
	\end{figure}

	\subsection{Reconstrucción de Poisson}
	\subsubsection{Implementación}
	La clase \texttt{Poisson} de la biblioteca \emph{PCL} implementa este método de reconstrucción,
	imponiendo condiciones de borde Neumann.

	Debido a que sólo es de interés el valor de $\chi$ en los puntos cercanos a
	la superficie, se utiliza un octree para representar esta función. Se
	provee de parámetros para establecer la profundidad del octree, controlando
	de esta manera la resolución de la superficie reconstruida.
	Sin embargo, debe considerarse que el consumo de memoria y el tiempo se incrementan de forma
	cuadrática con la profundidad del octree.
	%la cual se extrae mediante el método de \emph{Marching Cubes}. \TODO{referencia a marching cubes}

	Debido a que no se cuentan con puntos en la base de apoyo de los objetos, el uso de condiciones de borde Neumann
	producirá un estiramiento hacia abajo de la superficie resultante y la presencia de
	un hueco plano en la base (figura~\ref{fig:fill_poisson}.
	Si bien este hueco podrá ser luego rellenado mediante el algoritmo de \emph{advancing front}, no puede solucionarse el estiramiento.



	\begin{figure}
		\Imagen{img/fill_poisson} \TODO{usar dragon up}
		\caption{\label{fig:fill_poisson}Reconstrucción de la superficie mediante el método de \emph{Poisson}. Todos los huecos fueron rellenados a excepción de la base.}
	\end{figure}



	\section{Base de datos}
Uno de los supuestos de este proyecto era contar con un repositorio propio de
mallas tridimensionales.
Para la creación de este repositorio,
se utilizarían los algoritmos de reconstrucción desarrollados en \cite{Pancho},
ubicando al objeto de interés end una base giratoria y realizando capturas
en ángulos espaciados hasta completar una vuelta.
De esta forma, las posiciones de las vistas describirían un círculo centrado en el objeto y
cada captura contendría información de posición ($xyz$) y de textura ($rgb$).
Debido a los tiempos requeridos para calibrar el dispositivo de captura,
este repositorio nunca se materializó,
 por lo que fue necesario buscar otro con características similares.

%\begin{itemize}
%	\item redwood, freibug:
%	rgb y profundidad, pero el movimiento es pequeño y libre
%	(tendría que eliminar intermedios)
%\item middlebury
%	base giratoria, pero sólo RGB
%	(tendría que generar el mapa de profundidad)
%\item stanford
%	base giratoria, nube de puntos, sin textura.
%	Se optó por esta.
%	Se decidió no generar artificialmente los puntos de textura para tener un
%	caso más real.
%\end{itemize}

Se decidió utilizar \emph{The Stanford 3D Scanning Repository}\cite{StanfordScanRep} que brinda
acceso a escaneos tridimensionales y reconstrucciones detalladas para ser
usados en investigación.

Las capturas fueron obtenidas mediante un escáner láser de barrido Cyberware
3030~MS.  Se realizaron escaneos del objeto en diversas posiciones sobre una
base giratoria y luego estas capturas fueron combinadas para producir una única
malla triangular utilizando el método de \emph{zippering} o el de
\emph{volumetric merging}, ambos desarrollados en
Stanford\cite{StanfordScanRep}. \Nota{si no explico esos métodos, volarlos de acá\\}

La base de datos provee un archivo de configuración con las transformaciones de
alineación requeridas por cada captura.
Estas transformaciones fueron obtenidas realizando la registración de cada captura
contra un escaneo cilíndrico del objeto mediante un método semiautomático, donde el usuario
usuario establece una alineación inicial que luego es ajustada mediante un algoritmo
basado en ICP\cite{Turk:1994:ZPM:192161.192241}.

De esta base de datos se utilizaron los modelos
	\texttt{armadillo},
	\texttt{bunny},
	\texttt{dragon},
	\texttt{drill} y
	\texttt{happy},
los cuales presentan distintos niveles de detalles, cantidad de escaneos y niveles de ruido.


\begin{figure}
	\Imagen{example-image-a}
	\caption{\label{fig:stanfod_models}\TODO{Modelos de la base de datos Stanford.}}
\end{figure}


Desgraciadamente, no se cuenta con información de color en estos escaneos.
Se decidió adaptar los algoritmos a esta situación, en lugar de agregar
artificialmente valores de color para los puntos.


\section{Tecnologías}
	%Intro
	A continuación se mencionan las principales herramientas de software
	utilizadas en el desarrollo de programas de reconstrucción tridimensional.

	\subsection{KinectFusion}
	Es el algoritmo desarrollado por Microsoft para lograr reconstrucciones
	tridimensionales utilizando el dispositivo Kinect.

	%kinectfusion_real-time_3d_reconstruction_and_interaction_using_a_moving_depth_camera
	Debido a que uno de sus objetivos era lograr una implementación en tiempo
	real, el algoritmo de registración requiere de poca variación
	de la posición relativa cámara-objeto entre capturas, por lo que no fue utilizado.

	Para realizar la fusión utiliza una variación del método de
	\emph{volumetric merging} sobre GPU.\cite{Izadi:2011:KRR:2047196.2047270}

	%Suposiciones:
	%transforma el sistem en uno lineal, la transformación entre capturas es un incremento pequeño
	%sistema 6x6 (3 translaciones, 3 rotaciones)
	%utiliza todos los puntos porque tiene gpu




	\subsection{Open Source Computer Vision Library (OpenCV)}
	Es una biblioteca de código abierto de visión computacional y aprendizaje
	maquinal.  Cuenta con módulos de procesamiento de imágenes de profundidad y
	registración.

	En un principio se consideró utilizar la información de textura de las
	capturas para poder lograr la registración, pero debido a que la base de
	datos utilizada sólo contenía información geométrica  no se utilizarán las
	funcionalidades de esta biblioteca.

	\subsection{\emph{The Point Cloud Library} (PCL)}
	Es un framework de código abierto multiplataforma para el procesado de
	imágenes 2D/3D y nubes de puntos.
	Provee numerosos algoritmos modernos para reducción de ruido, extracción de
	puntos salientes, cálculo de descriptores, registración, reconstrucción de
	superficies, entre otros.

	La documentación incluye tutoriales para cada módulo de la biblioteca y
	además se cuenta con listas de correos y canales de IRC para brindar
	soporte.

	PCL se encuentra disponible para ser usada en C++.
	Debido al uso intensivo de código templatizado, la compilación del
	código cliente requiere de un tiempo considerable (aproximadamente un minuto).
	Existen proyectos para portarla a Python y Java, pero no se encuentran
	suficientemente avanzados.

	\subsection{CloudCompare, Meshlab}
	Son programas de procesamiento y edición de mallas de puntos 3D.  Presentan
	herramientas de registración semiautomática (a partir de puntos
	seleccionados por el usuario), y cuentan con una implementación del
	algoritmo \emph{Poisson Surface Reconstruction} para reconstrucción de
	superficies.

	Se utilizarán especialmente para visualización y comparación de resultados.



	\chapter{Diseño}
En este capítulo se presentarán los requerimientos identificados para el sistema
y los diagramas de clases definidos para su implementación.

\documentclass{pfc}
\title{Casos de uso}
\author{Walter Bedrij}
\date{\today}

\begin{document}
	\CasoUso{Ingresar lista de mallas}
		\Actor{Programador}
	\CasoUso{Ajustar parámetros}
		\Actor{Programador}
	\CasoUso{Alinear dos mallas}
		\Actor{Programador}
	\CasoUso{Corregir error de bucle}
		\Actor{Programador}
	\CasoUso{Rechazar malla}
		\Actor{Programador}
	\CasoUso{Extraer superficie}
		\Actor{Programador}
	\CasoUso{Rellenar huecos}
		\Actor{Programador}
\end{document}

\section{Especificación de requerimientos}
A partir del análisis de las herramientas de software existentes,
el estudio de la bibliografía relevante y el diagrama de los casos de uso
se definió el siguiente documento de requerimientos

\subsection{Descripción}
	Se dispone de un sistema cámara-superficie giratoria, cuyas posiciones
	se encuentran fijas en el espacio y el eje de giro de la superficie se
	encuentra alineado con el eje vertical del dispositivo de captura.
	El objeto de interés se ubica sobre la superficie giratoria, y se
	realizan capturas a diversos intervalos de giro
	hasta totalizar una vuelta completa (360\textdegree).

	Los algoritmos desarrollados para la registración de las capturas parciales,
	integración de las mallas resultantes y rellenado de huecos tendrán como resultado final
	una superficie cerrada triangulada que represente al objeto.

	Se tendrá como entrada una nube de puntos con valores de posición $\{x, y, z\}$.
	No se dispondrá de información de textura, normales o conectividades.

	\subsubsection{Suposiciones}
		El ángulo máximo entre dos mallas no podrá exceder los 60\textdegree.

		La cámara no se encontrará demasiado elevada respecto a la
		superficie giratoria. En ningún caso deberá superar el punto más alto del objeto.

\subsection{Requerimientos funcionales}
Se identificaron las siguientes funcionalidades para el sistema:

	\Requerimiento
		{Eliminación de puntos atípicos}
		{El sistema debe detectar y eliminar puntos considerados atípicos.}

	\Requerimiento
		{Alineación Inicial}
		{El sistema debe poder calcular una transformación de alineación para dos mallas
		que las acerque lo suficiente como para poder utilizar luego ICP.}

	\Requerimiento
		{Área solapada}
		{El sistema debe poder establecer los puntos en común (o una buena
		aproximación) entre dos mallas ya alineadas burdamente.}

	\Requerimiento
		{Métricas}
		{El sistema debe poder evaluar la calidad de una registración.}

	\Requerimiento
		{Corrección de bucle}
		{El sistema debe corregir el error propagado durante la registración
		una vez que se haya realizado una vuelta con las capturas.}

	\Requerimiento
		{Combinación de nubes}
		{El sistema debe generar una malla de consenso, ajustando los puntos y sus normales
		según la información provista por cada malla de entrada.}

	\Requerimiento
		{Triangulación}
		{El sistema debe poder triangular una nube de puntos tridimensional.}

	\Requerimiento
		{Relleno}
		{El sistema debe disponer de funciones para lograr que una malla sea cerrada. Se
		estimará una superficie en las zonas donde se carezca de
		información.}

\subsection{Requerimientos no funcionales}
	Se identificaron los siguientes requerimientos no funcionales:

	\Requerimiento{Tiempo de ejecución}
	{No se espera una ejecución a tiempo real de los algoritmos implementados.}

	\Requerimiento{Interfaces con software}
	{Las operaciones sobre las mallas y nubes de puntos se realizará
	mediante la \emph{Point Cloud Library} (PCL).
	Debido a esto, se desarrollará en el lenguaje de programación C++.}


	\Requerimiento{Sistemas operativos}
	{El producto desarrollado estará destinado a utilizarse en los sistemas
	operativos Windows y Linux.}

%diagramas de clases

	\documentclass{pfc}
\title{Casos de uso}
\author{Walter Bedrij}
\date{\today}

\begin{document}
	\CasoUso{Ingresar lista de mallas}
		\Actor{Programador}
	\CasoUso{Ajustar parámetros}
		\Actor{Programador}
	\CasoUso{Alinear dos mallas}
		\Actor{Programador}
	\CasoUso{Corregir error de bucle}
		\Actor{Programador}
	\CasoUso{Rechazar malla}
		\Actor{Programador}
	\CasoUso{Extraer superficie}
		\Actor{Programador}
	\CasoUso{Rellenar huecos}
		\Actor{Programador}
\end{document}

\section{Especificación de requerimientos}
A partir del análisis de las herramientas de software existentes,
el estudio de la bibliografía relevante y el diagrama de los casos de uso
se definió el siguiente documento de requerimientos

\subsection{Descripción}
	Se dispone de un sistema cámara-superficie giratoria, cuyas posiciones
	se encuentran fijas en el espacio y el eje de giro de la superficie se
	encuentra alineado con el eje vertical del dispositivo de captura.
	El objeto de interés se ubica sobre la superficie giratoria, y se
	realizan capturas a diversos intervalos de giro
	hasta totalizar una vuelta completa (360\textdegree).

	Los algoritmos desarrollados para la registración de las capturas parciales,
	integración de las mallas resultantes y rellenado de huecos tendrán como resultado final
	una superficie cerrada triangulada que represente al objeto.

	Se tendrá como entrada una nube de puntos con valores de posición $\{x, y, z\}$.
	No se dispondrá de información de textura, normales o conectividades.

	\subsubsection{Suposiciones}
		El ángulo máximo entre dos mallas no podrá exceder los 60\textdegree.

		La cámara no se encontrará demasiado elevada respecto a la
		superficie giratoria. En ningún caso deberá superar el punto más alto del objeto.

\subsection{Requerimientos funcionales}
Se identificaron las siguientes funcionalidades para el sistema:

	\Requerimiento
		{Eliminación de puntos atípicos}
		{El sistema debe detectar y eliminar puntos considerados atípicos.}

	\Requerimiento
		{Alineación Inicial}
		{El sistema debe poder calcular una transformación de alineación para dos mallas
		que las acerque lo suficiente como para poder utilizar luego ICP.}

	\Requerimiento
		{Área solapada}
		{El sistema debe poder establecer los puntos en común (o una buena
		aproximación) entre dos mallas ya alineadas burdamente.}

	\Requerimiento
		{Métricas}
		{El sistema debe poder evaluar la calidad de una registración.}

	\Requerimiento
		{Corrección de bucle}
		{El sistema debe corregir el error propagado durante la registración
		una vez que se haya realizado una vuelta con las capturas.}

	\Requerimiento
		{Combinación de nubes}
		{El sistema debe generar una malla de consenso, ajustando los puntos y sus normales
		según la información provista por cada malla de entrada.}

	\Requerimiento
		{Triangulación}
		{El sistema debe poder triangular una nube de puntos tridimensional.}

	\Requerimiento
		{Relleno}
		{El sistema debe disponer de funciones para lograr que una malla sea cerrada. Se
		estimará una superficie en las zonas donde se carezca de
		información.}

\subsection{Requerimientos no funcionales}
	Se identificaron los siguientes requerimientos no funcionales:

	\Requerimiento{Tiempo de ejecución}
	{No se espera una ejecución a tiempo real de los algoritmos implementados.}

	\Requerimiento{Interfaces con software}
	{Las operaciones sobre las mallas y nubes de puntos se realizará
	mediante la \emph{Point Cloud Library} (PCL).
	Debido a esto, se desarrollará en el lenguaje de programación C++.}


	\Requerimiento{Sistemas operativos}
	{El producto desarrollado estará destinado a utilizarse en los sistemas
	operativos Windows y Linux.}

\chapter{Diseño}
En este capítulo se presentarán los requerimientos identificados para el sistema
y los diagramas de clases definidos para su implementación.

\documentclass{pfc}
\title{Casos de uso}
\author{Walter Bedrij}
\date{\today}

\begin{document}
	\CasoUso{Ingresar lista de mallas}
		\Actor{Programador}
	\CasoUso{Ajustar parámetros}
		\Actor{Programador}
	\CasoUso{Alinear dos mallas}
		\Actor{Programador}
	\CasoUso{Corregir error de bucle}
		\Actor{Programador}
	\CasoUso{Rechazar malla}
		\Actor{Programador}
	\CasoUso{Extraer superficie}
		\Actor{Programador}
	\CasoUso{Rellenar huecos}
		\Actor{Programador}
\end{document}

\section{Especificación de requerimientos}
A partir del análisis de las herramientas de software existentes,
el estudio de la bibliografía relevante y el diagrama de los casos de uso
se definió el siguiente documento de requerimientos

\subsection{Descripción}
	Se dispone de un sistema cámara-superficie giratoria, cuyas posiciones
	se encuentran fijas en el espacio y el eje de giro de la superficie se
	encuentra alineado con el eje vertical del dispositivo de captura.
	El objeto de interés se ubica sobre la superficie giratoria, y se
	realizan capturas a diversos intervalos de giro
	hasta totalizar una vuelta completa (360\textdegree).

	Los algoritmos desarrollados para la registración de las capturas parciales,
	integración de las mallas resultantes y rellenado de huecos tendrán como resultado final
	una superficie cerrada triangulada que represente al objeto.

	Se tendrá como entrada una nube de puntos con valores de posición $\{x, y, z\}$.
	No se dispondrá de información de textura, normales o conectividades.

	\subsubsection{Suposiciones}
		El ángulo máximo entre dos mallas no podrá exceder los 60\textdegree.

		La cámara no se encontrará demasiado elevada respecto a la
		superficie giratoria. En ningún caso deberá superar el punto más alto del objeto.

\subsection{Requerimientos funcionales}
Se identificaron las siguientes funcionalidades para el sistema:

	\Requerimiento
		{Eliminación de puntos atípicos}
		{El sistema debe detectar y eliminar puntos considerados atípicos.}

	\Requerimiento
		{Alineación Inicial}
		{El sistema debe poder calcular una transformación de alineación para dos mallas
		que las acerque lo suficiente como para poder utilizar luego ICP.}

	\Requerimiento
		{Área solapada}
		{El sistema debe poder establecer los puntos en común (o una buena
		aproximación) entre dos mallas ya alineadas burdamente.}

	\Requerimiento
		{Métricas}
		{El sistema debe poder evaluar la calidad de una registración.}

	\Requerimiento
		{Corrección de bucle}
		{El sistema debe corregir el error propagado durante la registración
		una vez que se haya realizado una vuelta con las capturas.}

	\Requerimiento
		{Combinación de nubes}
		{El sistema debe generar una malla de consenso, ajustando los puntos y sus normales
		según la información provista por cada malla de entrada.}

	\Requerimiento
		{Triangulación}
		{El sistema debe poder triangular una nube de puntos tridimensional.}

	\Requerimiento
		{Relleno}
		{El sistema debe disponer de funciones para lograr que una malla sea cerrada. Se
		estimará una superficie en las zonas donde se carezca de
		información.}

\subsection{Requerimientos no funcionales}
	Se identificaron los siguientes requerimientos no funcionales:

	\Requerimiento{Tiempo de ejecución}
	{No se espera una ejecución a tiempo real de los algoritmos implementados.}

	\Requerimiento{Interfaces con software}
	{Las operaciones sobre las mallas y nubes de puntos se realizará
	mediante la \emph{Point Cloud Library} (PCL).
	Debido a esto, se desarrollará en el lenguaje de programación C++.}


	\Requerimiento{Sistemas operativos}
	{El producto desarrollado estará destinado a utilizarse en los sistemas
	operativos Windows y Linux.}

%diagramas de clases

\input{pruebas}


	\chapter{Pruebas y resultados}
	A continuación se detallan las pruebas realizadas y los resultados obtenidos
	por cada módulo desarrollado utilizando como base los modelos
	\texttt{armadillo}, \texttt{bunny}, \texttt{dragon}, \texttt{drill} y \texttt{happy}
	del repositorio de Stanford\cite{StanfordScanRep}.

	\section{Módulo de registración}
	Para la registración se utilizó el método basado en la búsqueda de clúster,
	seguido de un refinamiento mediante ICP y una corrección de bucle.
	Se plantearon dos métodos para evaluar la calidad de cada registración:
	\begin{itemize}
		\item Mediante la comparación entre la transformación calculada y aquella provista por la base de datos (\emph{ground truth}).
		\item Mediante una métrica de \emph{fitness} que se obtiene a partir de la nube transformada y las nubes de entrada.
	\end{itemize}

	Para comparar las alineaciones contra el \emph{ground truth}, se
	observa el efecto de las mismas sobre un punto orientado simulando la
	cámara (figura~\ref{fig:err_reg}). El punto \emph{eye} ($C$) se ubica inicialmente en las coordenadas
	$\{0, -0.1, 0.7\}$ (valores obtenidos de la base de datos), y se
	orienta el vector \emph{target} hacia $-z$ y el \emph{up} hacia $y$.
	El error de posicionamiento es la razón entre la distancia al punto
	de inicio y la distancia al punto obtenido por el \emph{ground truth}.
	\[\text{Error} = \frac{|C'-C_{gt}|}{|C_{gt} - C|}\]
	Los errores de \emph{target} y \emph{up} se corresponden al ángulo formado contra los
	vectores respectivos obtenidos por el \emph{ground truth}.

	\begin{figure}
		\centering
		\input{diagram/error_registration.pdf_tex}
		\caption{\label{fig:err_reg}Comparación entre las transformaciones de alineación.
		Se observa el efecto producido en un punto orientado $C$ que simula la posición de la cámara.}
	\end{figure}

	En el cuadro~\ref{tab:reg_error} se presentan los errores de registración promedio para cada orientación de los modelos.
	En la mayoría de los casos, los errores no superan $1^{\circ}$ en orientación ni $1\%$ en posicionamiento,
	observándose dos excepciones: \texttt{bunny} y \texttt{dragon stand}.
	El aumento en el error del modelo \texttt{bunny} se debe a que la captura \texttt{bun180} presenta una distancia cercana a $90^\circ$,
	superando las restricciones impuestas en este trabajo.
	Sin embargo, en el caso de \texttt{dragon stand} el error refleja una mala
	alineación en la captura~12, debida a una mala selección de los parámetros.
	Mediante un posterior ajuste de los parámetros, en particular, del tamaño de la vecindad para el cálculo de los descriptores, se obtuvo una alineación correcta.

	\begin{table}
	\centering
	\begin{tabular}{l*{3}{c}}
		\toprule
		Modelo                   &    Eye          &    Target (grados)        &    Up (grados)\\
		\midrule
		armadillo\\
		{\Em}back          &     0.0062159   &   0.221725     &    0.15211\\
		{\Em}head          &     0.0036356  &    0.102321     &    0.211231\\
		{\Em}head offset   &     0.0029806  &    0.086309     &    0.229574\\
		{\Em}stand         &     0.0022145  &    0.049612     &    0.105862\\
		{\Em}stand flip    &     0.0045019  &    0.125330     &    0.146033\\
		\midrule
		bunny                   &     0.0104809   &   0.598354     &    0.817185\\
		\midrule
		dragon\\
		{\Em}side             &     0.0070872  &    0.178650     &    0.212932\\
		{\Em}stand            &     0.0536199   &   1.379760     &    0.207754\\
		{\Em}up               &     0.0058265  &    0.139297     &    0.0675651\\
		\midrule
		drill                   &     0.0082317  &    0.238639     &    0.100126\\
		\midrule
		happy\\
		{\Em}back              &     0.0088540  &    0.189885     &    0.207247\\
		{\Em}side              &     0.0072675   &   0.175860     &    0.17525\\
		{\Em}stand             &     0.0050124  &    0.101383     &    0.097800\\
		\bottomrule
	\end{tabular}
	\caption[Errores de registración]{\label{tab:reg_error}Errores de registración.}
\end{table}


	Para evaluar la alineación entre un par de capturas prescindiendo del \emph{ground truth}, y situándonos en un escenario más realista,
	se diseñó una medida de \emph{fitness}.
	Esta medida se define como el porcentaje del área solapada entre las nubes una vez alineadas,
	donde un bajo solapamiento nos indicaría un posible error de alineación, como
	se observa en la figura~\ref{fig:fitness}.

	\TODO{explicar que el cambio de la vecindad lo resuelve}

	\begin{figure}
		\centering
			\begin{tikzpicture}[gnuplot]
%% generated with GNUPLOT 5.4p0 (Lua 5.4; terminal rev. Jun 2020, script rev. 114)
%% Tue 25 Aug 2020 01:55:34 AM -03
\gpmonochromelines
\path (0.000,0.000) rectangle (12.500,8.750);
\gpcolor{color=gp lt color border}
\gpsetlinetype{gp lt border}
\gpsetdashtype{gp dt solid}
\gpsetlinewidth{1.00}
\draw[gp path] (1.320,0.985)--(1.500,0.985);
\draw[gp path] (11.947,0.985)--(11.767,0.985);
\node[gp node right] at (1.136,0.985) {$0$};
\draw[gp path] (1.320,2.476)--(1.500,2.476);
\draw[gp path] (11.947,2.476)--(11.767,2.476);
\node[gp node right] at (1.136,2.476) {$0.2$};
\draw[gp path] (1.320,3.967)--(1.500,3.967);
\draw[gp path] (11.947,3.967)--(11.767,3.967);
\node[gp node right] at (1.136,3.967) {$0.4$};
\draw[gp path] (1.320,5.459)--(1.500,5.459);
\draw[gp path] (11.947,5.459)--(11.767,5.459);
\node[gp node right] at (1.136,5.459) {$0.6$};
\draw[gp path] (1.320,6.950)--(1.500,6.950);
\draw[gp path] (11.947,6.950)--(11.767,6.950);
\node[gp node right] at (1.136,6.950) {$0.8$};
\draw[gp path] (1.320,8.441)--(1.500,8.441);
\draw[gp path] (11.947,8.441)--(11.767,8.441);
\node[gp node right] at (1.136,8.441) {$1$};
\draw[gp path] (1.984,0.985)--(1.984,1.165);
\draw[gp path] (1.984,8.441)--(1.984,8.261);
\node[gp node center] at (1.984,0.677) {1};
\draw[gp path] (2.648,0.985)--(2.648,1.165);
\draw[gp path] (2.648,8.441)--(2.648,8.261);
\node[gp node center] at (2.648,0.677) {2};
\draw[gp path] (3.313,0.985)--(3.313,1.165);
\draw[gp path] (3.313,8.441)--(3.313,8.261);
\node[gp node center] at (3.313,0.677) {3};
\draw[gp path] (3.977,0.985)--(3.977,1.165);
\draw[gp path] (3.977,8.441)--(3.977,8.261);
\node[gp node center] at (3.977,0.677) {4};
\draw[gp path] (4.641,0.985)--(4.641,1.165);
\draw[gp path] (4.641,8.441)--(4.641,8.261);
\node[gp node center] at (4.641,0.677) {5};
\draw[gp path] (5.305,0.985)--(5.305,1.165);
\draw[gp path] (5.305,8.441)--(5.305,8.261);
\node[gp node center] at (5.305,0.677) {6};
\draw[gp path] (5.969,0.985)--(5.969,1.165);
\draw[gp path] (5.969,8.441)--(5.969,8.261);
\node[gp node center] at (5.969,0.677) {7};
\draw[gp path] (6.634,0.985)--(6.634,1.165);
\draw[gp path] (6.634,8.441)--(6.634,8.261);
\node[gp node center] at (6.634,0.677) {8};
\draw[gp path] (7.298,0.985)--(7.298,1.165);
\draw[gp path] (7.298,8.441)--(7.298,8.261);
\node[gp node center] at (7.298,0.677) {9};
\draw[gp path] (7.962,0.985)--(7.962,1.165);
\draw[gp path] (7.962,8.441)--(7.962,8.261);
\node[gp node center] at (7.962,0.677) {10};
\draw[gp path] (8.626,0.985)--(8.626,1.165);
\draw[gp path] (8.626,8.441)--(8.626,8.261);
\node[gp node center] at (8.626,0.677) {11};
\draw[gp path] (9.290,0.985)--(9.290,1.165);
\draw[gp path] (9.290,8.441)--(9.290,8.261);
\node[gp node center] at (9.290,0.677) {12};
\draw[gp path] (9.954,0.985)--(9.954,1.165);
\draw[gp path] (9.954,8.441)--(9.954,8.261);
\node[gp node center] at (9.954,0.677) {13};
\draw[gp path] (10.619,0.985)--(10.619,1.165);
\draw[gp path] (10.619,8.441)--(10.619,8.261);
\node[gp node center] at (10.619,0.677) {14};
\draw[gp path] (11.283,0.985)--(11.283,1.165);
\draw[gp path] (11.283,8.441)--(11.283,8.261);
\node[gp node center] at (11.283,0.677) {0};
\draw[gp path] (1.320,8.441)--(1.320,0.985)--(11.947,0.985)--(11.947,8.441)--cycle;
\node[gp node center,rotate=-270] at (0.292,4.713) {solapamiento};
\node[gp node center] at (6.633,0.215) {captura};
\draw[gp path] (1.873,0.985)--(1.873,8.158)--(2.095,8.158)--(2.095,0.985)--cycle;
\draw[gp path] (2.538,0.985)--(2.538,8.144)--(2.759,8.144)--(2.759,0.985)--cycle;
\draw[gp path] (3.202,0.985)--(3.202,6.102)--(3.423,6.102)--(3.423,0.985)--cycle;
\draw[gp path] (3.866,0.985)--(3.866,6.882)--(4.087,6.882)--(4.087,0.985)--cycle;
\draw[gp path] (4.530,0.985)--(4.530,5.745)--(4.752,5.745)--(4.752,0.985)--cycle;
\draw[gp path] (5.194,0.985)--(5.194,7.078)--(5.416,7.078)--(5.416,0.985)--cycle;
\draw[gp path] (5.859,0.985)--(5.859,7.902)--(6.080,7.902)--(6.080,0.985)--cycle;
\draw[gp path] (6.523,0.985)--(6.523,8.015)--(6.744,8.015)--(6.744,0.985)--cycle;
\draw[gp path] (7.187,0.985)--(7.187,8.047)--(7.408,8.047)--(7.408,0.985)--cycle;
\draw[gp path] (7.851,0.985)--(7.851,7.892)--(8.073,7.892)--(8.073,0.985)--cycle;
\draw[gp path] (8.515,0.985)--(8.515,7.634)--(8.737,7.634)--(8.737,0.985)--cycle;
\draw[gp path] (9.180,0.985)--(9.180,2.163)--(9.401,2.163)--(9.401,0.985)--cycle;
\draw[gp path] (9.844,0.985)--(9.844,5.858)--(10.065,5.858)--(10.065,0.985)--cycle;
\draw[gp path] (10.508,0.985)--(10.508,7.788)--(10.729,7.788)--(10.729,0.985)--cycle;
\draw[gp path] (11.172,0.985)--(11.172,7.869)--(11.394,7.869)--(11.394,0.985)--cycle;
\draw[gp path] (1.320,8.441)--(1.320,0.985)--(11.947,0.985)--(11.947,8.441)--cycle;
%% coordinates of the plot area
\gpdefrectangularnode{gp plot 1}{\pgfpoint{1.320cm}{0.985cm}}{\pgfpoint{11.947cm}{8.441cm}}
\end{tikzpicture}
%% gnuplot variables

		\caption[Métrica de alineación para el modelo \texttt{dragon stand}]{\label{fig:fitness}Métrica de alineación para el modelo \texttt{dragon stand}. El bajo
		porcentaje de solapamiento en la captura 12 se corresponde
		con un error de registración.}
	\end{figure}


	\section{Módulo de fusión}
	%En la fusión se utilizó una distancia de proximidad de $1.5$ veces la resolución de las nubes,
	%y un mínimo de confianza de $0.2$.

	Como medida de error de la fusión se utilizó la distancia entre los puntos de la nube reconstruida
	respecto al punto más cercano en el \emph{ground truth} (cuadro~\ref{tab:fus_error}).
	Esta medición no se realizó para el modelo \texttt{armadillo} debido a que su reconstrucción
	se encontraba a una escala distinta a la de las capturas.
	Nuevamente se destaca el error de \texttt{dragon stand} producto de una mala alineación.

	\begin{table}
	\center
	\begin{tabular}{l*{3}{c}}
		\toprule                                                                  
		Modelo                  &    Error promedio  & Desvío \\ 
		\midrule                                    
		bunny                   &      1.28464       & 0.74131\\
		\midrule                                    
		dragon side             &      1.19651       & 0.69846\\
		dragon stand            &      2.83930       & 2.41398\\
		dragon up               &      1.14363       & 0.88966\\
		\midrule                                    
		drill (contra vrip)     &      1.48515       & 0.96336\\
		drill (contra zip)      &      1.74326       & 1.39883\\
		\midrule                                    
		happy back              &      1.65632       & 1.27056\\
		happy side              &      1.35371       & 1.01163\\
		happy stand             &      1.79513       & 1.25758\\
		\bottomrule                                                               
	\end{tabular}
	\caption{\label{tab:fus_error}Errores en la fusión, normalizados respecto a la resolución de las capturas.}
\end{table}



	En todos los modelos, se observa, además, una inflación/deflación de los objetos
	reconstruidos debida a la propagación del error de alineación.  Así, la
	primera captura coincide casi exactamente, pero el error se incrementa
	a medida que nos alejamos de ella (figura~\ref{fig:fus_happy}).

	\begin{figure}
		\Imagen{img/happy_diff}
		\caption[Medida de error en la fusión]{\label{fig:fus_happy}Diferencia contra el \emph{ground truth} del modelo \texttt{happy}.}
	\end{figure}

	\section{Módulo de rellenado de huecos}
		\subsection{Método de advancing front}
		Al utilizar el método de advancing front sobre la superficie reconstruida de \texttt{bunny},
		se logró el rellenado de agujeros pequeños, obteniéndose una malla regular (figura~\ref{fig:fill_good}).
		Sin embargo, debido a la localidad con la que se generan los nuevos
		puntos, el frente puede diverger o pretender unirse a puntos que no
		forman parte del contorno del hueco, resultando una malla mal formada,
		con aristas que corresponden a más de dos caras (figura~\ref{fig:fill_bad}).
		Por estas razones, el método no resulta adecuado para el rellenado automático.


		\begin{figure}
			\Imagen{img/fill_good}
			\caption[Relleno de un hueco pequeño mediante \emph{advancing front}]
			{\label{fig:fill_good}Relleno de un hueco pequeño mediante \emph{advancing front}.}
		\end{figure}

		\begin{figure}
			\Imagen{img/fill_bad}
			\caption[Fallo en el algoritmo de \emph{advancing front}]
			{\label{fig:fill_bad}Fallo en el algoritmo de \emph{advancing front}.
			Se intentó completar un triángulo con un punto que no pertenecía al borde.}
		\end{figure}

		\subsection{Reconstrucción de Poisson}
		Como se mencionó anteriormente, la reconstrucción de Poisson nos garantiza el
		rellenado de todos los huecos (a excepción de la base) mediante una superficie suave.
		Se procedió, entonces, a una valoración visual de los objetos reconstruidos (figura~\ref{fig:poiss_all}):
		\begin{itemize}
			\item En \texttt{bunny} (figura~\ref{fig:bun_ear}) se observan desperfectos debidos a una mala registración de la captura \texttt{bun180},
				que se encontraba aproximadamente a $90^{\circ}$ respecto a sus vecinos.
			\item En \texttt{drill} (figura~\ref{fig:drill_drops}) se tienen componentes inconexas debido a una mala fusión en una zona de alta curvatura.
			\item En \texttt{dragon}(figura~\ref{fig:dragon_belly}) se observa la creación de un puente entre dos regiones.
				Esta es una de las limitaciones conocidas del método, al no poder incorporar la información de línea de vista de las capturas\cite{Kazhdan:2006:PSR:1281957.1281965}.
			\item En todos los casos, la base de apoyo del objeto presenta una deformación hacia abajo (figura~\ref{fig:base}) con un hueco al final.
		\end{itemize}

		\begin{figure}
			\Imagen{img/models_b}
			\caption[Resultado de las reconstrucciones]{\label{fig:poiss_all}Resultado de las reconstrucciones luego del rellenado de huecos mediante el método de Poisson.
			De izquierda a derecha y de arriba a abajo, los modelos son:
			\texttt{armadillo},
			\texttt{bunny},
			\texttt{dragon},
			\texttt{drill}
			y \texttt{happy}.}
		\end{figure}

		\begin{figure}
			\centering
			\includegraphics[max width=.5\linewidth, max height=.25\textheight, keepaspectratio]
				{img/bunny_ear}
			%\Imagen{img/bunny_ear}
			\caption[Acercamiento a la oreja derecha de \texttt{bunny}]{\label{fig:bun_ear}Acercamiento a la oreja derecha de \texttt{bunny}.}
		\end{figure}

		\begin{figure}
			%\Imagen{img/drill_drops}
			\centering
			\includegraphics[max width=.5\linewidth, max height=.25\textheight, keepaspectratio]
				{img/drill_drops}
			\caption[Acercamiento a la mecha de \texttt{drill}]{\label{fig:drill_drops}Acercamiento a la mecha de \texttt{drill}.}
		\end{figure}

		\begin{figure}
			%\Imagen{img/drill_drops}
			\centering
			\includegraphics[max width=.5\linewidth, max height=.25\textheight, keepaspectratio]
				{img/dragon_belly}
			\caption[Acercamiento al vientre de \texttt{dragon}]{\label{fig:dragon_belly}Acercamiento al vientre de \texttt{dragon}.}
		\end{figure}

		\begin{figure}
			\centering
			\includegraphics[max width=.5\linewidth, max height=.25\textheight, keepaspectratio]
				{img/arma_foot}
			%\Imagen{img/arma_foot}
			\caption[Acercamiento a la base de apoyo de \texttt{armadillo}]{\label{fig:base}Acercamiento a la base de apoyo de \texttt{armadillo}. Se observa un estiramiento hacia abajo debido al uso de condiciones de borde Neumann.}
		\end{figure}

		%trabajo futuro (advancing front)
		%Para evitar la divergencia es necesario definir una superficie de
		%soporte considerando todo el contorno del hueco, de forma de asegurar
		%que los nuevos puntos no excedan los límites del hueco.

	\section{Tiempo de ejecución}
	\TODO{descripción de características de la máquina}

		Durante el proceso de reconstrucción, se observa que la etapa de reconstrucción
		es responsable de la mayor parte del costo computacional, sobre todo al aumentar
		la cantidad de puntos en las capturas (cuadro~\ref{tab:reconstr_time}).
		En el cuadro~\ref{tab:reg_time} se muestran
		los tiempos de ejecución promedio, discriminados en la alineación
		inicial y el refinamiento posterior.
		El orden $\bigO\left(n^2\right)$ de la alineación inicial se debe a la búsqueda
		entre todos los pares de puntos para establecer las correspondencias (figura~\ref{fig:registration_order}).
		%Si bien los tiempos no son considerables, pueden reducirse al mejorar
		%la selección inicial de puntos y realizar la búsqueda de las
		%correspondencias de forma más eficiente.
		\begin{table}
	\centering
	\begin{tabular}{l*{6}{r}}
		\toprule
		Modelo                 & N    & Puntos      &  Registración & Fusión   & Rellenado & Total\\
		\midrule                      
		armadillo\\
		{\Em}back         &   11 &  25e3         &   84.4308        & 10.759  &  8.562  & 103.752\\
		{\Em}head         &   12 &  25e3         &   114.4926       & 12.501  &  9.892  & 136.886\\
		{\Em}head offset  &   11 &  25e3         &   100.7846       & 11.987  &  9.718  & 122.490\\
		{\Em}stand        &   12 &  25e3         &   102.3553       & 12.145  &  9.498  & 123.998\\
		{\Em}stand flip   &   11 &  25e3         &   101.7061       & 12.958  &  9.280  & 123.944\\
		\midrule                      
		bunny                  &   6  &  35e3         &    92.2872       & 11.246  &  11.862 & 115.395\\
		\midrule                      
		dragon\\
		{\Em}side            &   15 &  20e3         &   90.5427        & 14.021  &  8.086  & 112.650\\
		{\Em}stand           &   15 &  30e3         &   180.8010       & 23.485  &  12.680 & 216.966\\
		{\Em}up              &   15 &  30e3         &   164.1075       & 18.469  &  9.669  & 192.245\\
		\midrule                      
		drill                  &   12 &   4e3         &   4.3172         & 1.988   &  4.176  & 10.481 \\
		\midrule                      
		happy\\
		{\Em}back             &   15 &  45e3         &   343.2705       & 29.618  &  7.528  & 380.417\\
		{\Em}side             &   15 &  45e3         &   452.5830       & 32.588  &  8.361  & 493.532\\
		{\Em}stand            &   15 &  75e3         &   907.2930       & 44.570  &  10.929 & 962.792\\
		\bottomrule
	\end{tabular}
	\caption[Tiempo de reconstrucción]{\label{tab:reconstr_time}Tiempo de reconstrucción por cada modelo (en segundos).}
\end{table}



		\begin{table}
	\centering
	\begin{tabular}{l*{4}{c}}
		\toprule
		Modelo                 &  Puntos  &   Inicial (s)  &   ICP  (s)  &  Total  (s)\\
		\midrule
		armadillo back         &  25e3    &       7.05694    &   0.618592  &    7.67553\\
		armadillo head         &  25e3    &       8.94947    &   0.591576  &    9.54105\\
		armadillo head offset  &  25e3    &       8.62241    &   0.539835  &    9.16224\\
		armadillo stand        &  25e3    &       7.98686    &   0.542746  &    8.52961\\
		armadillo stand flip   &  25e3    &       8.69949    &   0.546518  &    9.24601\\
		\midrule
		bunny                  &  35e3    &       14.0323    &   1.348890  &    15.3812\\
		\midrule
		dragon side            &  20e3    &       5.52965    &   0.506529  &    6.03618\\
		dragon stand           &  30e3    &       11.3732    &   0.680195  &    12.0534\\
		dragon up              &  30e3    &       10.3322    &   0.608282  &    10.9405\\
		\midrule
		drill                  &   4e3    &       0.26898    &   0.090796  &    0.35977\\
		\midrule
		happy back             &  45e3    &       21.2702    &   1.614520  &    22.8847\\
		happy side             &  45e3    &       29.1103    &   1.061880  &    30.1722\\
		happy stand            &  75e3    &       59.3281    &   1.158030  &    60.4862\\
		\bottomrule
	\end{tabular}
	\caption{\label{tab:reg_time}Tiempos de ejecución promedio para la
	registración de a pares en los distintos modelos.}
\end{table}


		\begin{figure}
			\centering
			\begin{tikzpicture}[gnuplot]
%% generated with GNUPLOT 5.4p0 (Lua 5.4; terminal rev. Jun 2020, script rev. 114)
%% Thu 27 Aug 2020 10:08:06 PM -03
\gpmonochromelines
\path (0.000,0.000) rectangle (12.500,8.750);
\gpcolor{color=gp lt color border}
\gpsetlinetype{gp lt border}
\gpsetdashtype{gp dt solid}
\gpsetlinewidth{1.00}
\draw[gp path] (1.504,0.985)--(1.684,0.985);
\draw[gp path] (11.947,0.985)--(11.767,0.985);
\node[gp node right] at (1.320,0.985) {$10^{-2}$};
\draw[gp path] (1.504,1.546)--(1.594,1.546);
\draw[gp path] (11.947,1.546)--(11.857,1.546);
\draw[gp path] (1.504,1.874)--(1.594,1.874);
\draw[gp path] (11.947,1.874)--(11.857,1.874);
\draw[gp path] (1.504,2.107)--(1.594,2.107);
\draw[gp path] (11.947,2.107)--(11.857,2.107);
\draw[gp path] (1.504,2.288)--(1.594,2.288);
\draw[gp path] (11.947,2.288)--(11.857,2.288);
\draw[gp path] (1.504,2.435)--(1.594,2.435);
\draw[gp path] (11.947,2.435)--(11.857,2.435);
\draw[gp path] (1.504,2.560)--(1.594,2.560);
\draw[gp path] (11.947,2.560)--(11.857,2.560);
\draw[gp path] (1.504,2.668)--(1.594,2.668);
\draw[gp path] (11.947,2.668)--(11.857,2.668);
\draw[gp path] (1.504,2.764)--(1.594,2.764);
\draw[gp path] (11.947,2.764)--(11.857,2.764);
\draw[gp path] (1.504,2.849)--(1.684,2.849);
\draw[gp path] (11.947,2.849)--(11.767,2.849);
\node[gp node right] at (1.320,2.849) {$10^{-1}$};
\draw[gp path] (1.504,3.410)--(1.594,3.410);
\draw[gp path] (11.947,3.410)--(11.857,3.410);
\draw[gp path] (1.504,3.738)--(1.594,3.738);
\draw[gp path] (11.947,3.738)--(11.857,3.738);
\draw[gp path] (1.504,3.971)--(1.594,3.971);
\draw[gp path] (11.947,3.971)--(11.857,3.971);
\draw[gp path] (1.504,4.152)--(1.594,4.152);
\draw[gp path] (11.947,4.152)--(11.857,4.152);
\draw[gp path] (1.504,4.299)--(1.594,4.299);
\draw[gp path] (11.947,4.299)--(11.857,4.299);
\draw[gp path] (1.504,4.424)--(1.594,4.424);
\draw[gp path] (11.947,4.424)--(11.857,4.424);
\draw[gp path] (1.504,4.532)--(1.594,4.532);
\draw[gp path] (11.947,4.532)--(11.857,4.532);
\draw[gp path] (1.504,4.628)--(1.594,4.628);
\draw[gp path] (11.947,4.628)--(11.857,4.628);
\draw[gp path] (1.504,4.713)--(1.684,4.713);
\draw[gp path] (11.947,4.713)--(11.767,4.713);
\node[gp node right] at (1.320,4.713) {$10^{0}$};
\draw[gp path] (1.504,5.274)--(1.594,5.274);
\draw[gp path] (11.947,5.274)--(11.857,5.274);
\draw[gp path] (1.504,5.602)--(1.594,5.602);
\draw[gp path] (11.947,5.602)--(11.857,5.602);
\draw[gp path] (1.504,5.835)--(1.594,5.835);
\draw[gp path] (11.947,5.835)--(11.857,5.835);
\draw[gp path] (1.504,6.016)--(1.594,6.016);
\draw[gp path] (11.947,6.016)--(11.857,6.016);
\draw[gp path] (1.504,6.163)--(1.594,6.163);
\draw[gp path] (11.947,6.163)--(11.857,6.163);
\draw[gp path] (1.504,6.288)--(1.594,6.288);
\draw[gp path] (11.947,6.288)--(11.857,6.288);
\draw[gp path] (1.504,6.396)--(1.594,6.396);
\draw[gp path] (11.947,6.396)--(11.857,6.396);
\draw[gp path] (1.504,6.492)--(1.594,6.492);
\draw[gp path] (11.947,6.492)--(11.857,6.492);
\draw[gp path] (1.504,6.577)--(1.684,6.577);
\draw[gp path] (11.947,6.577)--(11.767,6.577);
\node[gp node right] at (1.320,6.577) {$10^{1}$};
\draw[gp path] (1.504,7.138)--(1.594,7.138);
\draw[gp path] (11.947,7.138)--(11.857,7.138);
\draw[gp path] (1.504,7.466)--(1.594,7.466);
\draw[gp path] (11.947,7.466)--(11.857,7.466);
\draw[gp path] (1.504,7.699)--(1.594,7.699);
\draw[gp path] (11.947,7.699)--(11.857,7.699);
\draw[gp path] (1.504,7.880)--(1.594,7.880);
\draw[gp path] (11.947,7.880)--(11.857,7.880);
\draw[gp path] (1.504,8.027)--(1.594,8.027);
\draw[gp path] (11.947,8.027)--(11.857,8.027);
\draw[gp path] (1.504,8.152)--(1.594,8.152);
\draw[gp path] (11.947,8.152)--(11.857,8.152);
\draw[gp path] (1.504,8.260)--(1.594,8.260);
\draw[gp path] (11.947,8.260)--(11.857,8.260);
\draw[gp path] (1.504,8.356)--(1.594,8.356);
\draw[gp path] (11.947,8.356)--(11.857,8.356);
\draw[gp path] (1.504,8.441)--(1.684,8.441);
\draw[gp path] (11.947,8.441)--(11.767,8.441);
\node[gp node right] at (1.320,8.441) {$10^{2}$};
\draw[gp path] (1.504,0.985)--(1.504,1.165);
\draw[gp path] (1.504,8.441)--(1.504,8.261);
\node[gp node center] at (1.504,0.677) {$10^{4}$};
\draw[gp path] (4.648,0.985)--(4.648,1.075);
\draw[gp path] (4.648,8.441)--(4.648,8.351);
\draw[gp path] (6.487,0.985)--(6.487,1.075);
\draw[gp path] (6.487,8.441)--(6.487,8.351);
\draw[gp path] (7.791,0.985)--(7.791,1.075);
\draw[gp path] (7.791,8.441)--(7.791,8.351);
\draw[gp path] (8.803,0.985)--(8.803,1.075);
\draw[gp path] (8.803,8.441)--(8.803,8.351);
\draw[gp path] (9.630,0.985)--(9.630,1.075);
\draw[gp path] (9.630,8.441)--(9.630,8.351);
\draw[gp path] (10.329,0.985)--(10.329,1.075);
\draw[gp path] (10.329,8.441)--(10.329,8.351);
\draw[gp path] (10.935,0.985)--(10.935,1.075);
\draw[gp path] (10.935,8.441)--(10.935,8.351);
\draw[gp path] (11.469,0.985)--(11.469,1.075);
\draw[gp path] (11.469,8.441)--(11.469,8.351);
\draw[gp path] (11.947,0.985)--(11.947,1.165);
\draw[gp path] (11.947,8.441)--(11.947,8.261);
\node[gp node center] at (11.947,0.677) {$10^{5}$};
\draw[gp path] (1.504,8.441)--(1.504,0.985)--(11.947,0.985)--(11.947,8.441)--cycle;
\node[gp node center,rotate=-270] at (0.292,4.713) {tiempo};
\node[gp node center] at (6.725,0.215) {puntos};
\gpsetpointsize{4.00}
\gp3point{gp mark 7}{}{(5.660,6.363)}
\gp3point{gp mark 7}{}{(5.660,6.539)}
\gp3point{gp mark 7}{}{(5.660,6.506)}
\gp3point{gp mark 7}{}{(5.660,6.448)}
\gp3point{gp mark 7}{}{(5.660,6.514)}
\gp3point{gp mark 7}{}{(7.186,6.926)}
\gp3point{gp mark 7}{}{(4.648,6.168)}
\gp3point{gp mark 7}{}{(6.487,6.728)}
\gp3point{gp mark 7}{}{(6.487,6.650)}
\gp3point{gp mark 7}{}{(8.325,7.247)}
\gp3point{gp mark 7}{}{(8.325,7.471)}
\gp3point{gp mark 7}{}{(10.642,8.034)}
\gpsetdashtype{gp dt 1}
\draw[gp path] (2.031,2.322)--(2.137,3.390)--(2.242,3.845)--(2.348,4.142)--(2.453,4.364)%
  --(2.559,4.543)--(2.664,4.694)--(2.770,4.825)--(2.875,4.940)--(2.981,5.044)--(3.086,5.139)%
  --(3.192,5.227)--(3.297,5.308)--(3.403,5.384)--(3.508,5.456)--(3.614,5.523)--(3.719,5.588)%
  --(3.825,5.649)--(3.930,5.708)--(4.036,5.765)--(4.141,5.820)--(4.247,5.872)--(4.352,5.923)%
  --(4.458,5.973)--(4.563,6.021)--(4.669,6.068)--(4.774,6.114)--(4.880,6.159)--(4.985,6.202)%
  --(5.090,6.245)--(5.196,6.287)--(5.301,6.329)--(5.407,6.369)--(5.512,6.409)--(5.618,6.448)%
  --(5.723,6.487)--(5.829,6.525)--(5.934,6.563)--(6.040,6.600)--(6.145,6.637)--(6.251,6.674)%
  --(6.356,6.710)--(6.462,6.746)--(6.567,6.781)--(6.673,6.816)--(6.778,6.851)--(6.884,6.885)%
  --(6.989,6.920)--(7.095,6.954)--(7.200,6.988)--(7.306,7.021)--(7.411,7.055)--(7.517,7.088)%
  --(7.622,7.121)--(7.728,7.154)--(7.833,7.187)--(7.939,7.220)--(8.044,7.252)--(8.150,7.285)%
  --(8.255,7.317)--(8.361,7.349)--(8.466,7.382)--(8.571,7.414)--(8.677,7.446)--(8.782,7.478)%
  --(8.888,7.510)--(8.993,7.542)--(9.099,7.573)--(9.204,7.605)--(9.310,7.637)--(9.415,7.669)%
  --(9.521,7.700)--(9.626,7.732)--(9.732,7.764)--(9.837,7.795)--(9.943,7.827)--(10.048,7.859)%
  --(10.154,7.890)--(10.259,7.922)--(10.365,7.954)--(10.470,7.985)--(10.576,8.017)--(10.681,8.049)%
  --(10.787,8.081)--(10.892,8.112)--(10.998,8.144)--(11.103,8.176)--(11.209,8.208)--(11.314,8.240)%
  --(11.420,8.272)--(11.525,8.304)--(11.631,8.336)--(11.736,8.368)--(11.842,8.400)--(11.947,8.432);
\gpsetdashtype{gp dt solid}
\draw[gp path] (1.504,8.441)--(1.504,0.985)--(11.947,0.985)--(11.947,8.441)--cycle;
%% coordinates of the plot area
\gpdefrectangularnode{gp plot 1}{\pgfpoint{1.504cm}{0.985cm}}{\pgfpoint{11.947cm}{8.441cm}}
\end{tikzpicture}
%% gnuplot variables

			\caption{\label{fig:registration_order}Tiempo de ejecución de la registración, ajustado a una función de orden $\bigO\left(n^2\right)$.}
		\end{figure}

	\chapter{Conclusiones y trabajos futuros}
%TODO: introducción

\section{Conclusiones del producto}
%explayarse más
	En este proyecto se realizó el desarrollo de una biblioteca de software para lograr
	la reconstrucción tridimensional de un objeto a partir de capturas de
	vistas parciales.
	Para esto, se dividió el problema en tres módulos: registración, fusión y
	rellenado de huecos, y se implementaron diversos algoritmos.

	A pesar de que las capturas no contenían información de textura,
	el algoritmo de registración fue exitoso en casi todos los casos sin requerir
	ajustes a su conjunto de parámetros. 
	Además, se cuenta con medidas de la calidad de la alineación,
	que permiten detectar fallas durante esta etapa sin requerir de una inspección visual.

	%Al trabajar directamente con las nubes de puntos no se restringió el
	%dispositivo de captura a un hardware en particular.  Sin embargo, las
	%restricciones impuestas de base giratoria y limitar la cantidad de capturas
	%requeridas fueron planteadas considerando una integración futura con
	%el trabajo realizado por \cite{Pancho}.

	Las reconstrucciones fueron obtenidas en tiempos razonables y sin requerir hardware especial.
	Si bien se observa un efecto de «inflación/deflación» debido a la propagación de los errores de registración,
	este se encuentra suficientemente acotado respecto al tamaño del objeto.

	Debido a que se consideró solamente una posición del objeto sobre la base giratoria,
	se presenta una gran cantidad de oclusiones,
	lo que genera la aparición de huecos de tamaño considerable
	en la superficie reconstruida luego de la fusión.
	Aún así, el resultado final es una malla cerrada (a excepción de la base), y la
	superficie estimada en las zonas sin información se une suavemente al resto.

\section{Conclusiones del proceso}
%redacción informal
Se tuvieron problemas al implementar la metodología seleccionada.
El tiempo invertido en la etapa de investigación bibliográfica fue demasiado extenso
y se desperdiciaron recursos al abordar el tratamiento de la información de textura,
que finalmente debió ser descartada al no contar con un repositorio propio.
Además, se produjo un desfasaje temporal entre la adquisición de los conocimientos y la
implementación de los mismos, requiriendo un nuevo análisis.
Estos inconvenientes se hubieran resuelto
al utilizar directamente una metodología incremental en todo el proceso,
con más incrementos de menor tamaño, como ser agregar un primer módulo
de preproceso que contenga la reducción de ruido y la operatoria básica con las
nubes de puntos.

En cuanto al desarrollo, uno de los principales problemas fue la definición de métricas
para evaluar los algoritmos y establecer los niveles de error aceptables en cada etapa.
%Esto se dificulta, además, al considerar que los resultados producidos en una etapa
%serán la entrada de otra, de la cual se desconoce su sensibilidad.
Muchas evaluaciones fueron primeramente visuales, resultando en un proceso lento
que en ocasiones fallaba en detectar errores considerables.

Durante el desarrollo se efectivizó otro de los riesgos identificados para el proyecto:
la falla en los equipos de trabajo requiriendo su reemplazo.
Gracias a las copias de respaldo periódicas, fue posible recuperar fácilmente el trabajo realizado hasta ese momento.
Sin embargo, debido a que el nuevo equipo de trabajo contaba con otro sistema operativo (Clear Linux),
se requirió de un largo proceso de configuración para instalar la biblioteca PCL a partir de sus archivos fuentes.
%Primeramente, se contaba con un sistema operativo Arch Linux, donde la
%instalación de la biblioteca PCL se realiza mediante un script
%\texttt{PKGBUILD} que resuelve las dependencias y configura los módulos.  Fue
%necesario compilar los fuentes, pero fuera del tiempo requerido, no se tuvieron
%mayores inconvenientes.  En el nuevo equipo, se contaba esta vez con un sistema
%Clear Linux.  Ahora la instalación resultó más problemática.  Se requerían
%demasiados recursos de memoria, por lo que el sistema operativo detenía el
%proceso.  Fue necesario un largo proceso de configuración y prueba para lograr
%la instalación exitosa de la biblioteca.

%\subsection{Riesgos efectivizados}
%Ausencia de repositorio de mallas tridimensionales
%(copiar base de datos)
%
%Falla en los equipos de trabajo
%(copiar parte de pcl)
%Meshlab: no se logró instalarlo en el nuevo equipo, se cambió a CloudCompare



\section{Trabajos futuros}
%\TODO{fusión, mejora de la confianza. Confianza según cercanía al borde, promedio de confianza}

En esta sección se describen actividades que excedieron el alcance de este proyecto
y podrían ser abordadas en una etapa posterior.

\begin{itemize}
	\item Ajustar los métodos de registración para combinar escaneos del objeto
		en varias posiciones sobre la base giratoria,
		buscando de esta forma eliminar huecos y reducir la propagación del error de alineación.
		Esto requerirá eliminar la restricción del eje de giro en la registración
		y ajustar el algoritmo de corrección de bucle.
	\item Ajustar los métodos para trabajar con el volumen de puntos generados
		por \cite{Pancho}.
		Es necesario analizar la robustez del algoritmo al submuestreo de la entrada,
		realizar una selección de keypoints de las nubes de entrada
		y utilizar un método más eficiente para la búsqueda de correspondencias.
	\item \TODO{Intentar la paralelización de los métodos desarrollados.}
	\item \TODO{Paso a GPU.}
	\item Implementar métricas de calidad del mallado que consideren las
		características de la impresión 3D.
		En este proyecto solamente se consideraron las condiciones de que la malla
		resultase cerrada y no presente intersecciones consigo misma.
	%\item Mejorar la detección de \emph{outliers} en el módulo de fusión.
	%\item Implementar métodos de suavizado de mallas.
	\item \TODO{Cerrar la base Poisson}
	\item Modificar el método de \emph{advancing front} para que utilice una
		superficie de soporte para establecer la posición de los nuevos puntos,
		asegurando de esta forma la convergencia del método y la suavidad del
		parche generado.
	\item Modificar el método de \emph{advancing front} para que considere las islas.
		Esto requerirá detectar dentro de qué hueco se encuentra cada isla.
\end{itemize}

	%Resumen
	%Introducción
	%   justificación
	%   objetivos
	%   requerimientos
	%Marco teórico
	%	Registración
	%		Keypoints
	%		Features
	%		ICP
	%	Fusión
	%		Mallado
	%	Rellenado y reconstrucción
	%		Poisson
	%		Advancing front
	%Desarrollo
	%	significado de los parámetros
	%Resultados
	%	calidad del mallado
	%	suavizado del mallado
	%	Ajustes para impresión 3D
	%	**Comparación contra otros métodos

	%Conclusiones y trabajos futuros
	%Bibliografía
	%\bibliographystyle{plainnat}
	\bibliographystyle{alpha}
	\bibliography{biblio}
\end{document}
