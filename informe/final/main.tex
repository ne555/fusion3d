\documentclass{pfc}

\subtitle{Informe final}

\newcommand{\TODO}[1]{{\color{red}\bfseries#1}}

\newcommand{\CasoUso}[1]{\bigskip%
{\bfseries Caso de uso: #1}\par}
\newcommand{\IncCU}[1]{{\bfseries Include (#1)}}
\newcommand{\UseCU}[1]{{\bfseries Use (#1)}}
\newcommand{\CUField}[2]{{\bfseries#1:} #2\par}
\newcommand{\CUNormal}{\smallskip{\bfseries Curso Normal:}\par}


\newcommand{\Requerimiento}[2]{%
	\bigskip%
	{\bfseries #1}\\%
	#2%
}

\usepackage{algorithmicx}
\usepackage{algorithm}
\usepackage[noend]{algpseudocode}
\usepackage{booktabs}

\makeatletter
 \renewcommand{\ALG@name}{Algoritmo}
\makeatother
\renewcommand{\algorithmicfunction}{}

%\newcommand{\Alerta}[1]{{\Huge\bfseries\sffamily#1}}
\newcommand{\NombreItem}[1]{{\bfseries#1:}}

%\includeonly{conclusiones}
\begin{document}
\frontmatter
	\maketitle
	\chapter{Resumen}
		Gracias a los avances en los sensores de profundidad,
		es posible obtener una nube de puntos que represente la estructura tridimensional de una escena en el mundo real.
		Sin embargo, para poder reconstruir completamente un objeto, es necesario combinar varias de estas vistas.
		Por esto, en este trabajo se propone el desarrollo de una herramienta
		que permita realizar la fusión de estas vistas,
		obteniendo un modelo tridimensional de un objeto
		y posibilitando su posterior construcción mediante una impresora 3D.

		\noindent{\bfseries Palabras clave:} fusión de mallas, imágenes de profundidad, impresión 3D, reconstrucción 3D.
	\tableofcontents
\mainmatter
	\chapter{Introducción}
	\section{Justificación}
	La impresión 3D es un método de fabricación en el cual se deposita el material (como por ejemplo, plástico o metal) en capas para producir un objeto tridimensional.
	Avances en los materiales utilizados han permitido la creación de objetos de calidad comparable a la de métodos tradicionales, con el potencial de personalizar cada producto.
	La impresión 3D ya ha demostrado ser viable para muchas aplicaciones médicas, incluyendo la creación de prótesis e implantes dentales\cite{Schubert159}. %\cite{innovations in 3d printing}
	Sin embargo, para poder realizar estas copias o modificaciones,
	se requiere del modelo geométrico del objecto, el cual puede no estar disponible.
	Para poder obtener estos modelos geométricos,
	se pueden aplicar procesos de ingeniería inversa a objetos ya existentes.
	%Describir los pasos de la ingeniería inversa: adquisición alineación, encuentro de superficie, descripción, segmentación, modelo CAD

	Un proceso crucial en estas técnicas es la adquisición de datos.
	Los distintos métodos de adquisición se diferencian según el fenómeno físico de interacción con la superficie del objeto de interés.
	De esta manera, se pueden clasificar como:
	\begin{itemize}
		\item Métodos táctiles o de contacto, donde sensores en las articulaciones de un brazo robótico determinan las coordenadas relativas de la superficie. Estos son de los más robustos, introduciendo poco ruido, pero también de los más lentos y suelen tener problemas con superficies cóncavas.
		\item Métodos de no-contacto, donde se utiliza luz (métodos ópticos), sonido u ondas electromagnéticas.
	\end{itemize}
	En particular, los métodos ópticos son los más populares con una rápida velocidad de adquisición\cite{Várady97reverseengineering}. %\cite{reverse engineering}.
	Dentro de los métodos ópticos podemos distinguir:
	\begin{itemize}
		\item Métodos activos o de luz estructurada,
			que proyectan patrones de luces conocidos sobre la escena de modo de analizar sus deformaciones;
		\item Métodos pasivos o de visión estereoscópica,
			que utilizan dos o mas cámaras y buscan correspondencias
			entre los puntos de la escena entre los puntos capturados,
			lo cual presenta gran dificultad y por eso son menos usados\cite{Várady97reverseengineering}.%\cite{reverse engineering}
	\end{itemize}

	Un avance reciente en la adquisición de datos mediante luz estructurada consiste en utilizar el sensor Kinect.
	Este dispositivo captura una imagen RGB de 32 bits y resolución de $640 \times 480$ píxeles
	junto con una imagen de profundidad de 16 bits y resolución de $320 \times 240$ píxeles\cite{MatheEstudioKinect}, %\cite{especificaciones kinect}
	procesando ambas se obtiene una nube de puntos 3D a color. %, también llamada imagen RGB-D.
	Una alternativa a Kinect para la adquisición de los mapas de profundidad se presenta en~\cite{Pancho}
	donde se trabajó con un proyector para la emisión de luz estructurada en el espectro visible y cámaras de resolución de $2592 \times 1944$ y $3264 \times 2448$ píxeles,
	obteniéndose resultados con más de dos millones de puntos.

	Sin embargo, los puntos capturados sólo reflejan una porción de la superficie del objeto,
	la parte visible desde la cámara.
	Además, la propia geometría del objeto podría generar obstrucciones, reflejos o sombras
	que imposibilitan la adquisición de datos en esas zonas («huecos»).
	Para solventar este problema, es necesario combinar múltiples capturas
	del objeto en diferentes orientaciones respecto a la cámara, de modo que
	cada parte de la superficie del objeto esté representada en al menos una captura.

	El principal problema para combinar múltiples vistas es
	encontrar las transformaciones de rotación y translación que permitan
	relacionar cada vista parcial, para luego fusionarlas y así reconstruir el objeto.

	%Entre otras investigaciones realizadas en este campo se destacan los siguientes aportes:
	Podemos nombrar algunas investigaciones realizadas en este campo:
	\begin{itemize}
		\item En~\cite{Hassanpour} %\cite{3d reconstruction faces uncontrolled}
	se hace uso de características conocidas de los objetos de estudio (caras y cabezas humanas).
		\item En~\cite{Riegler2017THREEDV} %\cite{learning depth fusion}
	se presenta un método novedoso utilizando redes convolucionales.
		\item En~\cite{Zach08fastand} %\cite{fast and high quaality fusion}ll
	se plantea un método robusto frente a ruido no gaussiano, común en la adquisición.
		\item En~\cite{automatic-3d-model-construction-for-turn-table-sequences} %\cite{model construction turn table}
	se limita las posibles transformaciones a rotaciones sobre un eje fijo.
	\end{itemize}

	Sin embargo, cabe destacar el aporte desarrollado por Microsoft mediante
	KinectFusion, un algoritmo que utiliza el sensor Kinect para reconstruir escenas tridimensionales en tiempo real.
	El usuario mueve el dispositivo por la escena mientras que el algoritmo realiza un seguimiento de la posición y orientación de la cámara. Esa información se combina con el mapa de profundidad para obtener un modelo 3D.
	En~\cite{real-time-3d-reconstruction-using-a-kinect-sensor} %\cite{kinect ed reconstruction}
	se discuten algunas de sus limitaciones:
	\begin{itemize}
		\item utiliza un alto costo computacional, requiriendo de una poderosa GPU;
		\item los movimientos de cámara deben ser lentos y pequeños;
		\item está limitado al dispositivo Kinect;
		\item algunos objetos podrían no aparecer en el mapa de profundidad debido a que absorben o reflejan demasiada luz infrarroja;
		\item no utiliza información de color.
	\end{itemize}

	Una implementación libre del algoritmo de KinectFusion se observa en el módulo KinFu,
	parte de la Point Cloud Library (PCL),
	donde se elimina la dependencia con el dispositivo Kinect,
	presenta algoritmos para trabajar a una mayor escala
	%exigiendo solamente compatibilidad con OpenNI,
	y además se permite la integración de la información de textura a la superficie final reconstruida.

	El uso de otros dispositivos además del Kinect nos permitirá lograr una reconstrucción de mayor precisión o a distinta escala pero podría traer aparejado un aumento en el costo y tiempo de cada adquisición.
	Por esto, es necesario disponer de control sobre los algoritmos de reconstrucción para adecuarlos a los dispositivos de captura y a las características del modelo 3D resultante deseado.

	En este proyecto se analizarán técnicas de fusión de mallas,
	planteando además la posibilidad de utilizar información de textura,
	para lograr un modelo tridimensional de un objeto que permita su posterior replicación
	mediante una impresora 3D.
	Se contará entonces con una herramienta que permitirá el desarrollo de aplicaciones para:
	\begin{itemize}
		\item Reproducción de biomodelos para que estudiantes de medicina y veterinaria realicen prácticas sobre los mismos.
			A este respecto, la Facultad de Ciencias Médicas de la Universidad Nacional del Litoral expresó un interés particular.
			%mirai 3d, empresa argentina, adquiere los biomodelos mediante tomografía axial computarizada (TAC)
		\item Digitalización de esculturas, lo que permitirá salvaguardarlas y facilitará el acceso, aunque indirecto, a las mismas.
			Puede mencionarse el \emph{Digital Michelangelo Project} de la Universidad de Stanford, que tiene como objetivo crear un repositorio de modelos 3D de alta calidad de las esculturas y la arquitectura de Miguel Ángel.
			%catedral de Notre-Dame, escultura de Néstor
		\item Digitalización de ambientes y objetos para utilizarlos en aplicaciones de realidad virtual o de diseño.
	\end{itemize}

	\section{Objetivos}
		\subsection{General}
			Diseñar y desarrollar una biblioteca de funciones que permita
			la reconstrucción de una malla 3D cerrada a partir de mallas de superficies parciales.
		\subsection{Específicos}
		\begin{itemize}
			%Falta un objetivo específico relacionado con el análisis de técnica de fusión de mallas.
			\item Desarrollar algoritmos que determinen las posiciones relativas de cada malla parcial.
			\item Desarrollar algoritmos de fusión de las mallas.
			\item Desarrollar algoritmos para rellenado de huecos producto de obstrucciones.
			\item Evaluar el desempeño de cada bloque desarrollado y realizar ajustes.
		\end{itemize}

	\section{Alcance}
		\subsection{Alcances Funcionales}
		\begin{itemize}
			\item La herramienta deberá generar una malla 3D cerrada (\emph{watertight}).
		\end{itemize}
		\subsection{Alcances No Funcionales}
		\begin{itemize}
			\item La herramienta deberá ser robusta frente al ruido producido por los sensores.
			\item La herramienta no impondrá topologías específicas para los objetos a reconstruirse.
			\item No se pretenderá el funcionamiento de los algoritmos en tiempo real.
			\item La herramienta será de código libre.
			\item La herramienta será multiplataforma.
		\end{itemize}
		\subsection{Supuestos}
		\begin{itemize}
			\item Se contará con un repositorio de mallas tridimensionales con información de textura.
		\end{itemize}

	\chapter{Marco teórico}
%intro
%%Proceso de reconstrucción general
%Reverse engineering of geometric models
El objetivo final de los sistemas de ingeniería inversa
es lograr realizar un escáner 3D inteligente.
%Diagrama de flujo
1- Obtención de datos
2- Preproceso
3- Segmentación y surface ¿fitting? (la mejor superficie que representa los puntos
4- Obtención modelo CAD
La 1 está resuelta por Pancho
	pero está limitada debido a consideraciones físicas
	se necesita combinar múltiples vistas.

La parte crítica es la 3

	Problemas con la adquisición
		calibración, precisión, oclusión, ruido, datos faltantes, vistas múltiples

		oclusión -> huecos
			Se producen debido a obstrucciones, sombras o reflejos.
			Se pierden los puntos de esas zonas.

		ruido, ¿cuándo eliminarlo?
			se pierden los detalles del objeto

		rellenado de huecos.
			Debido a la naturaleza del proceso de captura,
			la información obtenida cerca de los bordes es poco confiable (¿por qué?)
				(por las reflecciones, en particular malo para los de láser, había un dibujo)
			En algunas situaciones, los huecos no puden solucionarse con otra vista (¿cuándo?) 

		distribución estadística: la captura es una muestra de la población
			¿cómo ajustar la confianza?

		El acabado de la superficie:
			suavidad y recubrimiento (reflexiones, cabello (demasiadas irregulares))


	Organización de la nube de puntos:
		Nubes organizadas: vecindades	

	CAD/CAM conectividad y continuidad de la estructura.
		(no se tienen métricas de calidad para la impresión)
		poisson es suave, ¿pero qué tanto?

	Combinación de vistas:
		diferente resolución, todo el objeto o un detalle
		observar toda la superficie del objeto, no es trivial y puede requerir \emph{feedback}
		de la reconstrucción

	El mayor problema es obtener una registración precisa, es decir,
	encontrar las transformaciones de rotación y translación que relaciona la
	información que proporciona una captura con otra.

	Lo más simple es una base giratoria. (acá parece que tiene información de ángulos)
	Sin embargo, la base del objeto nunca
	será visible, por lo que para obtener un modelo completo del objeto, es
	necesario cambiar su punto de apoyo.

	Se debe asegurar un solapamiento suficiente para poder determinar la alineación.

%zippered
%There are two main issues in creating a single model from multiple
%range images: registration and integration.  Registration refers to
%computing a rigid transformation that brings the points of one range
%image into alignment with the portions of a surface that is shares with
%another range image.  Integration is the process of creating a single
%surface representation from the sample points from two or more
%range images.
	%agregar hole-filling







	Con el fin de identificar requerimientos y obtener el marco teórico
	necesario para el desarrollo del proyecto se analizaron herramientas de
	software utilizadas para la reconstrucción tridimensional y textos
	científicos acordes a la temática.

	El proceso de reconstrucción generalmente se divide en tres etapas:
	\begin{enumerate}
		\item Registración: donde se determinan las transformaciones necesarias
			para llevar cada vista a su correcta posición en un marco de
			referencia global.
		\item Fusión: donde se unifica el aporte de cada vista para obtener una
			superficie que las englobe.
		\item Relleno de huecos: donde se asegura que la superficie global sea
			cerrada, es decir, que encierre un volumen.
	\end{enumerate}


%FIXME: alineación / registración
%\section{Informe bibliográfico}
	%\section{Introducción}

	%	Iterative closest point:
	%		Busca la transformación que minimice el error de alineación
	%		entre los puntos de las mallas
	%		%efficient_variants_of_the_icp_algorithm.txt
	%		%generalized_icp.txt
	%		implementado en PCL

	%zippered
%All the steps needed to digitize an object  that requires up to 10 range scans
%can be performed using our system with five minutes of user interaction and a
%few hours of compute time.

%A range scanner is any device that senses 3D positions on an
%object’s surface and returns an array of distance values.  A range
%image is an m×n grid of distances (range points) that describe a
%surface either in Cartesian coordinates (a height field) or cylindrical
%coordinates, with two of the coordinates being implicitly defined by
%the indices of the grid.


	\section{Registración}


		La registración entre dos nubes de puntos se suele resolver mediante
		alguna de las variantes del algoritmo \emph{Iterative Closest Point (ICP)}.
		Sin embargo, para evitar caer en mínimos locales,
		se debe contar con una buena aproximación inicial.
		Por esto, es necesario desarrollar algoritmos para conseguir esta
		aproximación inicial.\cite{7271006}
		%Registration with the Point Cloud Library A Modular Framework for Aligning in 3-D

	ICP
	%Rusinkiewicz Levoy
	%efficient variants
	%realtime 3d model adquisition

	El algoritmo de ICP se ha convertido en el método dominante para realizar la alineación
	de modelos tridimensionales utilizando únicamente la información de geometría de los mismos.
	El algoritmo 


	Considera simplemente la información de posición (y normales)
	- Selección de puntos mediante submuestreo uniforme o aleatorio
	- Establecer correspondencias entre las nubes de puntos. Par más cercano entre puntos,
		o considerando la «superficie» (point to point, point to plane)
	- Ponderar las correspondencias (ej, uniforme)
	- Rechazar correspondencias para eliminar puntos *anómalos* (umbral de distancia)
	- Minimizar una métrica de error.  (iterar)
	Según cómo se realice cada uno de estos pasos se obtendrá una variante del algoritmo con
	diferente estabilidad, robustez y eficienci.

%A. ICP
%The key concept of the standard ICP algorithm can be
%summarized in two steps:
%1) compute correspondences between the two scans.
%2) compute a transformation which minimizes distance
%between corresponding points.
%Iteratively repeating these two steps typically results in conver-
%gence to the desired transformation. Because we are violating
%the assumption of full overlap, we are forced to add a maximum
%matching threshold d m a x . This threshold accounts for
%the fact that some points will not have any correspondence in
%the second scan (e.g. points which are outside the boundary of
%scan A). In most implementations of ICP, the choice of d m a x
%represents a trade off between convergence and accuracy. A
%low value results in bad convergence (the algorithm becomes
%“short sighted”); a large value causes incorrect correspon-
%dences to pull the final alignment away from the correct value.



		% efficient_variants_of_the_icp_algorithm
		Los algoritmos de registración se pueden dividir en los siguientes pasos:
		\begin{enumerate}
			\item Selección de puntos de la entrada (\emph{keypoints}).
			\item Utilizar descriptores para establecer correspondencias entre los puntos de las nubes.
			\item Rechazar correspondencias para reducir los \emph{outliers}.
			\item Alineación.\cite{conf/3dim/RusinkiewiczL01}
		\end{enumerate}

		En cuanto a la selección de puntos se tienen como opciones:
		\begin{itemize}
			\item Utilizar todos los puntos o realizar un submuestreo uniforme o aleatorio.
			\item Algoritmos basados en procesamiento de imágenes: como harris\cite{Harris88acombined} y brisk\cite{Leutenegger:2011:BBR:2355573.2356277}.
			\item Algoritmos específicos para puntos 3D: como ISS\cite{ISS}.
			\item Búsqueda de puntos cuyos descriptores sean persistentes a varias escalas: se requiere de formas eficientes de calcular los descriptores, como FPFH\cite{Rusu:2009:FPF:1703435.1703733}.
		\end{itemize}

		%Descriptores:
			Por cada \emph{keypoint} se calculará un descriptor que nos
			permitirá determinar las correspondencias entre las dos nubes de
			puntos.
			Un descriptor es una representación compacta
			de la vecindad de un punto,
			siendo importante además establecer el límite de esta vecindad.

			Para puntos en el espacio, los descriptores utilizan las
			posiciones relativas de los vecinos o los ángulos entre sus
			normales, ponderándolos según la distancia al punto de interés.
			Por ejemplo: el descriptor 3DSC define un histograma tridimensional
			cuantizando la distancia, azimut y elevación entre los puntos;
			el descriptor FPFH realiza algo similar con las normales.

			Ciertos descriptores, como ISS y SHOT, establecen un marco de
			referencia que permite obtener una estimación de la transformación
			entre las nubes a partir de solamente dos puntos.

		\subsection{Corrección de bucle}
			Debido a que la registración se hace de a pares sucesivos, el error
			se acumula con cada nueva malla.  Si se tiene una lista de capturas
			que completan una vuelta sobre el objeto, el error de registración
			acumulado para la última malla de la lista podrá ser apreciado
			especialmente en los bordes de esta respecto a la primera malla.

			Es posible corregir este error al perturbar la última registración
			y luego propagar esta perturbación en las alineaciones anteriores.
			Estas perturbaciones provocan una deformación de la malla, por lo
			cual
			%in-hand_scanning_with_online_loop_closure
			%FIXME: describir el algoritmo
			en \cite{5457479}
			se plantea un algoritmo para el cálculo y la propagación de las
			mismas, de forma que la deformación sea ``lo más rígida posible''.

	\section{Fusión}
		Una vez alineadas las superficies, éstas deben combinarse en una única malla resultante.

		\subsection{Volumetric merge}
		Se divide el espacio en un arreglo de vóxeles, que contienen la
		distancia con signo desde el centro del vóxel a la superficie (la
		distancia será positiva si el vóxel se encuentra en la parte exterior, y
		negativa si se encuentra en el interior).
		De esta forma,
		se tiene definida de forma implícita a la
		superficie donde la distancia es 0, y puede extraerse con, por ejemplo, \emph{marching cubes}.

		En cada nueva registración se actualizan las distancias de los
		vóxeles mediante raycasting,
		desde la posición estimada de la cámara hacia la superficie,
		realizando un promedio ponderado con los valores anteriormente calculados.\cite{Curless:1996:VMB:237170.237269} %a_volumetric_method_for_building_complex_models_from_range_images.txt
		Debido a que sólo nos interesan las zonas donde la distancia es cercana a 0 es posible utilizar una representación rala para los vóxeles.\cite{Steinbrucker:2013:LMS:2586117.2586926} % large-scale_multi-resolution_surface_reconstruction_from_rgb-d_sequences

		%PCL provee la clase TSDFVolume para las operaciones sobre el arreglo de vóxeles.

		\subsection{Zippered}
		%zippered_polygon_meshes_from_range_images.txt
		Obtiene como resultado una malla poligonal.

		El algoritmo trabaja en dos pasos:
		\begin{enumerate}
			\item Aproximación de la topología: se reduce el área solapada
				entre las mallas seleccionando una de las representaciones para
				cada punto en común.
			\item Refinado: se ajusta la posición de cada punto moviéndolo
				según su normal mediante un promedio ponderado dependiente del
				nivel de \emph{confianza} que posee el punto en cada vista.\cite{Turk:1994:ZPM:192161.192241}
		\end{enumerate}

		\subsection{Representación de surfel}
		%in-hand_scanning_with_online_loop_closure.txt
		En lugar de realizar una triangulación, en cada punto se establece un disco de radio variable orientado según la normal.
		Esto facilita la actualización, el agregado y la eliminación de puntos en cada nueva registración y además permite detectar ciertos \emph{outliers}.\cite{5457479}

		Sin embargo, se requiere un post-procesamiento para determinar las conectividades de los puntos.

		\subsection{Poisson Surface Reconstruction}
		%poisson_surface_reconstruction
		La reconstrucción se realiza considerando los puntos y normales de
		todas las nubes a la vez, es decir, no realiza un promedio ponderado como los métodos anteriores.

		Se define una función indicadora $\chi$ que recibe el valor 1 para puntos
		dentro del modelo y 0 para puntos en el exterior.
		La superficie del objeto queda determinada por la frontera entre estos
		valores, y el gradiente de la función indicadora se corresponde con las
		normales de los puntos del objeto $\vec{n}$.

		El problema consiste en encontrar la función $\chi$ cuyo gradiente
		aproxime el campo vectorial definido por las normales. Aplicando el
		operador de divergencia, se obtiene un problema de Poisson: calcular la
		función escalar $\chi$ cuyo laplaciano equivale a la divergencia del
		campo vectorial $\vec{n}$.

		\[\Delta\chi \equiv \nabla \cdot\nabla\chi = \nabla \vec{n}\]

		Utilizando funciones de soporte local para aproximar la solución, se obtiene un sistema lineal ralo bien condicionado.

		El algoritmo rellena huecos automáticamente, pero puede resultar demasiado agresivo, uniendo porciones que deberían permanecer separadas.\cite{Kazhdan:2006:PSR:1281957.1281965}


	\section{Relleno de huecos}
		Los huecos son regiones que ninguna de las vistas logró capturar.
		La existencia de huecos elimina la propiedad de \emph{watertight} de la
		malla, la cual es necesaria para su impresión 3D.

		El método de fusión \emph{volumetric merge} rellena huecos
		automáticamente al definirlos como la frontera entre vóxeles externos y
		aquellos nunca alcanzados por el raycasting.

		En el caso de una malla poligonal,
		pueden identificarse fácilmente los vértices que limitan el hueco
		como aquellos que definen una arista que corresponde a sólo un elemento.
		Entonces puede operarse solamente en la vecindad del hueco.

		Una opción es convertir la superficie en una
		representación volumétrica y luego realizar la difusión de la función
		de distancia en la zona del hueco.
		Este proceso se itera hasta que no se detectan cambios significativos en la superficie.\cite{fillingholes}
		%filling_holes_in_comple_surfaces_using_volumetric_diffusion

		Otra opción es transformar el problema al de una interpolación.
		\begin{itemize}
			\item Por cada hueco se ajusta un plano a los puntos del borde y
				sus vecinos.
			\item Se proyectan los puntos a este plano, obteniéndose un mapa de
				altura del contorno del hueco.
			\item Utilizando el algoritmo de \emph{Moving Least Squares} se
				ajusta una superficie a este mapa de altura.
			\item Se realiza un muestreo sobre esta superficie para obtener
				puntos que rellenen el hueco.
		\end{itemize}
		De esta forma el parche de reconstrucción se unirá suavemente a la malla original.\cite{Filling_holes_on_locally_smooth_surfaces}
			%Filling holes on locally smooth surfaces reconstructed from point clouds



\section{Base de datos}
Uno de los supuestos de este proyecto era contar con un repositorio propio de
mallas tridimensionales.
Para la creación de este repositorio,
se utilizarían los algoritmos de reconstrucción desarrollados en \cite{Pancho},
ubicando al objeto de interés end una base giratoria y realizando capturas
en ángulos espaciados hasta completar una vuelta.
De esta forma, las posiciones de las vistas describirían un círculo centrado en el objeto y
cada captura contendría información de posición ($xyz$) y de textura ($rgb$).
Debido a los tiempos requeridos para calibrar el dispositivo de captura,
este repositorio nunca se materializó,
 por lo que fue necesario buscar otro con características similares.

%\begin{itemize}
%	\item redwood, freibug:
%	rgb y profundidad, pero el movimiento es pequeño y libre
%	(tendría que eliminar intermedios)
%\item middlebury
%	base giratoria, pero sólo RGB
%	(tendría que generar el mapa de profundidad)
%\item stanford
%	base giratoria, nube de puntos, sin textura.
%	Se optó por esta.
%	Se decidió no generar artificialmente los puntos de textura para tener un
%	caso más real.
%\end{itemize}

Se decidió utilizar \emph{The Stanford 3D Scanning Repository}\cite{StanfordScanRep} que brinda
acceso a escaneos tridimensionales y reconstrucciones detalladas para ser
usados en investigación.

Las capturas fueron obtenidas mediante un escáner láser de barrido Cyberware
3030~MS.  Se realizaron escaneos del objeto en diversas posiciones sobre una
base giratoria y luego estas capturas fueron combinadas para producir una única
malla triangular utilizando el método de \emph{zippering} o el de
\emph{volumetric merging}, ambos desarrollados en
Stanford\cite{StanfordScanRep}. \Nota{si no explico esos métodos, volarlos de acá\\}

La base de datos provee un archivo de configuración con las transformaciones de
alineación requeridas por cada captura.
Estas transformaciones fueron obtenidas realizando la registración de cada captura
contra un escaneo cilíndrico del objeto mediante un método semiautomático, donde el usuario
usuario establece una alineación inicial que luego es ajustada mediante un algoritmo
basado en ICP\cite{Turk:1994:ZPM:192161.192241}.

De esta base de datos se utilizaron los modelos
	\texttt{armadillo},
	\texttt{bunny},
	\texttt{dragon},
	\texttt{drill} y
	\texttt{happy},
los cuales presentan distintos niveles de detalles, cantidad de escaneos y niveles de ruido.


\begin{figure}
	\Imagen{example-image-a}
	\caption{\label{fig:stanfod_models}\TODO{Modelos de la base de datos Stanford.}}
\end{figure}


Desgraciadamente, no se cuenta con información de color en estos escaneos.
Se decidió adaptar los algoritmos a esta situación, en lugar de agregar
artificialmente valores de color para los puntos.


\section{Tecnologías}
	%Intro
	A continuación se mencionan las principales herramientas de software
	utilizadas en el desarrollo de programas de reconstrucción tridimensional.

	\subsection{KinectFusion}
	Es el algoritmo desarrollado por Microsoft para lograr reconstrucciones
	tridimensionales utilizando el dispositivo Kinect.

	%kinectfusion_real-time_3d_reconstruction_and_interaction_using_a_moving_depth_camera
	Debido a que uno de sus objetivos era lograr una implementación en tiempo
	real, el algoritmo de registración requiere de poca variación
	de la posición relativa cámara-objeto entre capturas, por lo que no fue utilizado.

	Para realizar la fusión utiliza una variación del método de
	\emph{volumetric merging} sobre GPU.\cite{Izadi:2011:KRR:2047196.2047270}

	%Suposiciones:
	%transforma el sistem en uno lineal, la transformación entre capturas es un incremento pequeño
	%sistema 6x6 (3 translaciones, 3 rotaciones)
	%utiliza todos los puntos porque tiene gpu




	\subsection{Open Source Computer Vision Library (OpenCV)}
	Es una biblioteca de código abierto de visión computacional y aprendizaje
	maquinal.  Cuenta con módulos de procesamiento de imágenes de profundidad y
	registración.

	En un principio se consideró utilizar la información de textura de las
	capturas para poder lograr la registración, pero debido a que la base de
	datos utilizada sólo contenía información geométrica  no se utilizarán las
	funcionalidades de esta biblioteca.

	\subsection{\emph{The Point Cloud Library} (PCL)}
	Es un framework de código abierto multiplataforma para el procesado de
	imágenes 2D/3D y nubes de puntos.
	Provee numerosos algoritmos modernos para reducción de ruido, extracción de
	puntos salientes, cálculo de descriptores, registración, reconstrucción de
	superficies, entre otros.

	La documentación incluye tutoriales para cada módulo de la biblioteca y
	además se cuenta con listas de correos y canales de IRC para brindar
	soporte.

	PCL se encuentra disponible para ser usada en C++.
	Debido al uso intensivo de código templatizado, la compilación del
	código cliente requiere de un tiempo considerable (aproximadamente un minuto).
	Existen proyectos para portarla a Python y Java, pero no se encuentran
	suficientemente avanzados.

	\subsection{CloudCompare, Meshlab}
	Son programas de procesamiento y edición de mallas de puntos 3D.  Presentan
	herramientas de registración semiautomática (a partir de puntos
	seleccionados por el usuario), y cuentan con una implementación del
	algoritmo \emph{Poisson Surface Reconstruction} para reconstrucción de
	superficies.

	Se utilizarán especialmente para visualización y comparación de resultados.



	\documentclass{pfc}
\title{Casos de uso}
\author{Walter Bedrij}
\date{\today}

\begin{document}
	\CasoUso{Ingresar lista de mallas}
		\Actor{Programador}
	\CasoUso{Ajustar parámetros}
		\Actor{Programador}
	\CasoUso{Alinear dos mallas}
		\Actor{Programador}
	\CasoUso{Corregir error de bucle}
		\Actor{Programador}
	\CasoUso{Rechazar malla}
		\Actor{Programador}
	\CasoUso{Extraer superficie}
		\Actor{Programador}
	\CasoUso{Rellenar huecos}
		\Actor{Programador}
\end{document}

\section{Especificación de requerimientos}
A partir del análisis de las herramientas de software existentes,
el estudio de la bibliografía relevante y el diagrama de los casos de uso
se definió el siguiente documento de requerimientos

\subsection{Descripción}
	Se dispone de un sistema cámara-superficie giratoria, cuyas posiciones
	se encuentran fijas en el espacio y el eje de giro de la superficie se
	encuentra alineado con el eje vertical del dispositivo de captura.
	El objeto de interés se ubica sobre la superficie giratoria, y se
	realizan capturas a diversos intervalos de giro
	hasta totalizar una vuelta completa (360\textdegree).

	Los algoritmos desarrollados para la registración de las capturas parciales,
	integración de las mallas resultantes y rellenado de huecos tendrán como resultado final
	una superficie cerrada triangulada que represente al objeto.

	Se tendrá como entrada una nube de puntos con valores de posición $\{x, y, z\}$.
	No se dispondrá de información de textura, normales o conectividades.

	\subsubsection{Suposiciones}
		El ángulo máximo entre dos mallas no podrá exceder los 60\textdegree.

		La cámara no se encontrará demasiado elevada respecto a la
		superficie giratoria. En ningún caso deberá superar el punto más alto del objeto.

\subsection{Requerimientos funcionales}
Se identificaron las siguientes funcionalidades para el sistema:

	\Requerimiento
		{Eliminación de puntos atípicos}
		{El sistema debe detectar y eliminar puntos considerados atípicos.}

	\Requerimiento
		{Alineación Inicial}
		{El sistema debe poder calcular una transformación de alineación para dos mallas
		que las acerque lo suficiente como para poder utilizar luego ICP.}

	\Requerimiento
		{Área solapada}
		{El sistema debe poder establecer los puntos en común (o una buena
		aproximación) entre dos mallas ya alineadas burdamente.}

	\Requerimiento
		{Métricas}
		{El sistema debe poder evaluar la calidad de una registración.}

	\Requerimiento
		{Corrección de bucle}
		{El sistema debe corregir el error propagado durante la registración
		una vez que se haya realizado una vuelta con las capturas.}

	\Requerimiento
		{Combinación de nubes}
		{El sistema debe generar una malla de consenso, ajustando los puntos y sus normales
		según la información provista por cada malla de entrada.}

	\Requerimiento
		{Triangulación}
		{El sistema debe poder triangular una nube de puntos tridimensional.}

	\Requerimiento
		{Relleno}
		{El sistema debe disponer de funciones para lograr que una malla sea cerrada. Se
		estimará una superficie en las zonas donde se carezca de
		información.}

\subsection{Requerimientos no funcionales}
	Se identificaron los siguientes requerimientos no funcionales:

	\Requerimiento{Tiempo de ejecución}
	{No se espera una ejecución a tiempo real de los algoritmos implementados.}

	\Requerimiento{Interfaces con software}
	{Las operaciones sobre las mallas y nubes de puntos se realizará
	mediante la \emph{Point Cloud Library} (PCL).
	Debido a esto, se desarrollará en el lenguaje de programación C++.}


	\Requerimiento{Sistemas operativos}
	{El producto desarrollado estará destinado a utilizarse en los sistemas
	operativos Windows y Linux.}

\chapter{Diseño}
En este capítulo se presentarán los requerimientos identificados para el sistema
y los diagramas de clases definidos para su implementación.

\documentclass{pfc}
\title{Casos de uso}
\author{Walter Bedrij}
\date{\today}

\begin{document}
	\CasoUso{Ingresar lista de mallas}
		\Actor{Programador}
	\CasoUso{Ajustar parámetros}
		\Actor{Programador}
	\CasoUso{Alinear dos mallas}
		\Actor{Programador}
	\CasoUso{Corregir error de bucle}
		\Actor{Programador}
	\CasoUso{Rechazar malla}
		\Actor{Programador}
	\CasoUso{Extraer superficie}
		\Actor{Programador}
	\CasoUso{Rellenar huecos}
		\Actor{Programador}
\end{document}

\section{Especificación de requerimientos}
A partir del análisis de las herramientas de software existentes,
el estudio de la bibliografía relevante y el diagrama de los casos de uso
se definió el siguiente documento de requerimientos

\subsection{Descripción}
	Se dispone de un sistema cámara-superficie giratoria, cuyas posiciones
	se encuentran fijas en el espacio y el eje de giro de la superficie se
	encuentra alineado con el eje vertical del dispositivo de captura.
	El objeto de interés se ubica sobre la superficie giratoria, y se
	realizan capturas a diversos intervalos de giro
	hasta totalizar una vuelta completa (360\textdegree).

	Los algoritmos desarrollados para la registración de las capturas parciales,
	integración de las mallas resultantes y rellenado de huecos tendrán como resultado final
	una superficie cerrada triangulada que represente al objeto.

	Se tendrá como entrada una nube de puntos con valores de posición $\{x, y, z\}$.
	No se dispondrá de información de textura, normales o conectividades.

	\subsubsection{Suposiciones}
		El ángulo máximo entre dos mallas no podrá exceder los 60\textdegree.

		La cámara no se encontrará demasiado elevada respecto a la
		superficie giratoria. En ningún caso deberá superar el punto más alto del objeto.

\subsection{Requerimientos funcionales}
Se identificaron las siguientes funcionalidades para el sistema:

	\Requerimiento
		{Eliminación de puntos atípicos}
		{El sistema debe detectar y eliminar puntos considerados atípicos.}

	\Requerimiento
		{Alineación Inicial}
		{El sistema debe poder calcular una transformación de alineación para dos mallas
		que las acerque lo suficiente como para poder utilizar luego ICP.}

	\Requerimiento
		{Área solapada}
		{El sistema debe poder establecer los puntos en común (o una buena
		aproximación) entre dos mallas ya alineadas burdamente.}

	\Requerimiento
		{Métricas}
		{El sistema debe poder evaluar la calidad de una registración.}

	\Requerimiento
		{Corrección de bucle}
		{El sistema debe corregir el error propagado durante la registración
		una vez que se haya realizado una vuelta con las capturas.}

	\Requerimiento
		{Combinación de nubes}
		{El sistema debe generar una malla de consenso, ajustando los puntos y sus normales
		según la información provista por cada malla de entrada.}

	\Requerimiento
		{Triangulación}
		{El sistema debe poder triangular una nube de puntos tridimensional.}

	\Requerimiento
		{Relleno}
		{El sistema debe disponer de funciones para lograr que una malla sea cerrada. Se
		estimará una superficie en las zonas donde se carezca de
		información.}

\subsection{Requerimientos no funcionales}
	Se identificaron los siguientes requerimientos no funcionales:

	\Requerimiento{Tiempo de ejecución}
	{No se espera una ejecución a tiempo real de los algoritmos implementados.}

	\Requerimiento{Interfaces con software}
	{Las operaciones sobre las mallas y nubes de puntos se realizará
	mediante la \emph{Point Cloud Library} (PCL).
	Debido a esto, se desarrollará en el lenguaje de programación C++.}


	\Requerimiento{Sistemas operativos}
	{El producto desarrollado estará destinado a utilizarse en los sistemas
	operativos Windows y Linux.}

%diagramas de clases

\input{pruebas}

	\chapter{Conclusiones y trabajos futuros}
%TODO: introducción

\section{Conclusiones del producto}
%explayarse más
	En este proyecto se realizó el desarrollo de una biblioteca de software para lograr
	la reconstrucción tridimensional de un objeto a partir de capturas de
	vistas parciales.
	Para esto, se dividió el problema en tres módulos: registración, fusión y
	rellenado de huecos, y se implementaron diversos algoritmos.

	A pesar de que las capturas no contenían información de textura,
	el algoritmo de registración fue exitoso en casi todos los casos sin requerir
	ajustes a su conjunto de parámetros. 
	Además, se cuenta con medidas de la calidad de la alineación,
	que permiten detectar fallas durante esta etapa sin requerir de una inspección visual.

	%Al trabajar directamente con las nubes de puntos no se restringió el
	%dispositivo de captura a un hardware en particular.  Sin embargo, las
	%restricciones impuestas de base giratoria y limitar la cantidad de capturas
	%requeridas fueron planteadas considerando una integración futura con
	%el trabajo realizado por \cite{Pancho}.

	Las reconstrucciones fueron obtenidas en tiempos razonables y sin requerir hardware especial.
	Si bien se observa un efecto de «inflación/deflación» debido a la propagación de los errores de registración,
	este se encuentra suficientemente acotado respecto al tamaño del objeto.

	Debido a que se consideró solamente una posición del objeto sobre la base giratoria,
	se presenta una gran cantidad de oclusiones,
	lo que genera la aparición de huecos de tamaño considerable
	en la superficie reconstruida luego de la fusión.
	Aún así, el resultado final es una malla cerrada (a excepción de la base), y la
	superficie estimada en las zonas sin información se une suavemente al resto.

\section{Conclusiones del proceso}
%redacción informal
Se tuvieron problemas al implementar la metodología seleccionada.
El tiempo invertido en la etapa de investigación bibliográfica fue demasiado extenso
y se desperdiciaron recursos al abordar el tratamiento de la información de textura,
que finalmente debió ser descartada al no contar con un repositorio propio.
Además, se produjo un desfasaje temporal entre la adquisición de los conocimientos y la
implementación de los mismos, requiriendo un nuevo análisis.
Estos inconvenientes se hubieran resuelto
al utilizar directamente una metodología incremental en todo el proceso,
con más incrementos de menor tamaño, como ser agregar un primer módulo
de preproceso que contenga la reducción de ruido y la operatoria básica con las
nubes de puntos.

En cuanto al desarrollo, uno de los principales problemas fue la definición de métricas
para evaluar los algoritmos y establecer los niveles de error aceptables en cada etapa.
%Esto se dificulta, además, al considerar que los resultados producidos en una etapa
%serán la entrada de otra, de la cual se desconoce su sensibilidad.
Muchas evaluaciones fueron primeramente visuales, resultando en un proceso lento
que en ocasiones fallaba en detectar errores considerables.

Durante el desarrollo se efectivizó otro de los riesgos identificados para el proyecto:
la falla en los equipos de trabajo requiriendo su reemplazo.
Gracias a las copias de respaldo periódicas, fue posible recuperar fácilmente el trabajo realizado hasta ese momento.
Sin embargo, debido a que el nuevo equipo de trabajo contaba con otro sistema operativo (Clear Linux),
se requirió de un largo proceso de configuración para instalar la biblioteca PCL a partir de sus archivos fuentes.
%Primeramente, se contaba con un sistema operativo Arch Linux, donde la
%instalación de la biblioteca PCL se realiza mediante un script
%\texttt{PKGBUILD} que resuelve las dependencias y configura los módulos.  Fue
%necesario compilar los fuentes, pero fuera del tiempo requerido, no se tuvieron
%mayores inconvenientes.  En el nuevo equipo, se contaba esta vez con un sistema
%Clear Linux.  Ahora la instalación resultó más problemática.  Se requerían
%demasiados recursos de memoria, por lo que el sistema operativo detenía el
%proceso.  Fue necesario un largo proceso de configuración y prueba para lograr
%la instalación exitosa de la biblioteca.

%\subsection{Riesgos efectivizados}
%Ausencia de repositorio de mallas tridimensionales
%(copiar base de datos)
%
%Falla en los equipos de trabajo
%(copiar parte de pcl)
%Meshlab: no se logró instalarlo en el nuevo equipo, se cambió a CloudCompare



\section{Trabajos futuros}
%\TODO{fusión, mejora de la confianza. Confianza según cercanía al borde, promedio de confianza}

En esta sección se describen actividades que excedieron el alcance de este proyecto
y podrían ser abordadas en una etapa posterior.

\begin{itemize}
	\item Ajustar los métodos de registración para combinar escaneos del objeto
		en varias posiciones sobre la base giratoria,
		buscando de esta forma eliminar huecos y reducir la propagación del error de alineación.
		Esto requerirá eliminar la restricción del eje de giro en la registración
		y ajustar el algoritmo de corrección de bucle.
	\item Ajustar los métodos para trabajar con el volumen de puntos generados
		por \cite{Pancho}.
		Es necesario analizar la robustez del algoritmo al submuestreo de la entrada,
		realizar una selección de keypoints de las nubes de entrada
		y utilizar un método más eficiente para la búsqueda de correspondencias.
	\item \TODO{Intentar la paralelización de los métodos desarrollados.}
	\item \TODO{Paso a GPU.}
	\item Implementar métricas de calidad del mallado que consideren las
		características de la impresión 3D.
		En este proyecto solamente se consideraron las condiciones de que la malla
		resultase cerrada y no presente intersecciones consigo misma.
	%\item Mejorar la detección de \emph{outliers} en el módulo de fusión.
	%\item Implementar métodos de suavizado de mallas.
	\item \TODO{Cerrar la base Poisson}
	\item Modificar el método de \emph{advancing front} para que utilice una
		superficie de soporte para establecer la posición de los nuevos puntos,
		asegurando de esta forma la convergencia del método y la suavidad del
		parche generado.
	\item Modificar el método de \emph{advancing front} para que considere las islas.
		Esto requerirá detectar dentro de qué hueco se encuentra cada isla.
\end{itemize}

	%Resumen
	%Introducción
	%   justificación
	%   objetivos
	%   requerimientos
	%Marco teórico
	%	Registración
	%		Keypoints
	%		Features
	%		ICP
	%	Fusión
	%		Mallado
	%	Rellenado y reconstrucción
	%		Poisson
	%		Advancing front
	%Desarrollo
	%	significado de los parámetros
	%Resultados
	%	calidad del mallado
	%	suavizado del mallado
	%	Ajustes para impresión 3D
	%	**Comparación contra otros métodos

	%Conclusiones y trabajos futuros
	%Bibliografía
	\bibliographystyle{alpha}
	\bibliography{biblio}
\end{document}
