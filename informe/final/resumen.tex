\chapter{Resumen}
Gracias a los avances en las técticas para determinar la profundidad
utilizando sistemas de visión estereoscópica o sistemas de luz estructurada,
es posible obtener con gran precisión una nube de puntos que represente
la forma, posición y dimensiones de un objeto cualquiera.
Sin embargo, esta nube de puntos es parcial, ya que solamente refleja
la porción observable desde el punto de captura.
Por esta razón, en este proyecto se propone el desarrollo de una biblioteca de software
que implemente técnicas para alinear y combinar estas vistas parciales
y así obtener un modelo tridimensional de todo el objeto,
posibilitando su posterior construcción mediante técnicas de impresión 3D.

\noindent{\bfseries Palabras clave:} fusión de mallas, imágenes de profundidad, impresión 3D, reconstrucción 3D.
