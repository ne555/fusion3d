\chapter{Resumen}
Gracias a los avances en sensores de profundidad y técnicas de reconstrucción de superficies,
es posible obtener con gran precisión una nube de puntos que represente
la forma, posición y dimensiones de un objeto cualquiera.
Sin embargo, esta nube de puntos es parcial, ya que solamente refleja
la porción observable desde el punto de captura.
Por esta razón, en este proyecto se propone el desarrollo de una biblioteca de software
que implemente técnicas para combinar estas vistas parciales
y así obtener un modelo tridimensional de todo el objeto,
posibilitando su posterior construcción mediante técnicas de impresión 3D.

\Nota{
	sensores de profundidad y técnicas para determinar la profundidad.

	Hablar de prime-sense, kinect fusion, que requieren que el usuario
	tome las capturas con mucho cuidado

	Nosotros hacemos pocas capturas espaciadas, esa es la gracia
%TODO: motivación continuar lo de pancho
%explayarse en la forma de adquisición
%vistas parciales
}
	%\noindent{\bfseries Palabras clave:} fusión de mallas, imágenes de profundidad, impresión 3D, reconstrucción 3D.
