\section{Diagrama de clases}
El sistema se dividió en cuatro módulos principales:
\begin{enumerate}
	\item Preproceso:
		donde se realiza una reducción de ruido a las nubes de entrada
		antes de ser procesadas por las siguientes etapas.
	\item Registración:
		donde se obtienen las transformaciones de rotación y translación
		que lleven a cada captura a un sistema global
		de forma que las zonas comunes encajen perfectamente.
	\item Fusión:
		donde se combina la información presente en cada captura
		para obtener finalmente una malla que represente la totalidad del objeto.
	\item Rellenado de huecos:
		donde se estima la superficie del objeto en las zonas sin información.
\end{enumerate}

A continuación se presenta el diagrama de clases del sistema y una breve descripción de las clases más importantes.

\begin{figure}[h]
	\Imagen{uml/diagrama_de_clases}
	\caption{\label{fig:diagrama_de_clases}Diagrama de clases del sistema.}
\end{figure}

\begin{itemize}
	\item {\bfseries Nube:}
		Representa una captura del objeto desde una posición de la cámara en particular.
		Es una colección de \emph{Puntos} sin organización. Clase provista por PCL.
	\item {\bfseries Punto:} contiene las coordenadas $\{x, y, z\}$ obtenidas por el
		dispositivo de captura. Sus normales se estimarán durante el preproceso.

		%alineación
	\item {\bfseries Registración:} se encarga de obtener la \emph{Transformación} que
		permita alinear dos \emph{Nubes} entre sí.  Para esto establece
		correspondencias entre los \emph{Anclajes}.
	\item {\bfseries Anclaje:} a partir de puntos salientes de \emph{Nube} calcula
		\emph{descriptores}  que permitan asociarlos y un
		\emph{marco de referencia} para obtener una estimación de la
		\emph{Transformación}.
	\item {\bfseries Transformación:} representa una transformación de rotación y translación
		que será aplicada a una \emph{Nube} para alinearla.

		%fusión
	\item {\bfseries Fusión:} une las \emph{Nubes} ya alineadas para
		obtener una superficie global que represente al objeto.
		Esto supone corregir la posición de los puntos, descartar
		aquellos considerados como ruido y triangular la superficie.
	\item {\bfseries Malla:} triangulación de la \emph{Nube}
		representada por un grafo de conectividades (\emph{DCEL}).
	\item {\bfseries DCEL:} relaciona las caras, aristas y vértices de una superficie mediante
		una estructura de medias aritas.

		%rellenado de huecos
	\item {\bfseries Rellenado de huecos:} La clase se encarga de estimar
		nuevos puntos en zonas donde se carece de información (huecos)
		y triangularlos para que la \emph{Malla} sea cerrada.
	\item {\bfseries Borde:} Es una colección de puntos ordenados
		que representa un borde de un hueco en la \emph{Malla}.
\end{itemize}
