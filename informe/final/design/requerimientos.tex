\section{Especificación de requerimientos}
A partir del análisis de las herramientas de software existentes,
el estudio de la bibliografía relevante y el diagrama de los casos de uso
se definió el siguiente documento de requerimientos

\subsection{Descripción}
	Se dispone de un sistema cámara-superficie giratoria, cuyas posiciones
	se encuentran fijas en el espacio y el eje de giro de la superficie se
	encuentra alineado con el eje vertical del dispositivo de captura.
	El objeto de interés se ubica sobre la superficie giratoria, y se
	realizan capturas a diversos intervalos de giro
	hasta totalizar una vuelta completa (360\textdegree).

	Los algoritmos desarrollados para la registración de las capturas parciales,
	integración de las mallas resultantes y rellenado de huecos tendrán como resultado final
	una superficie cerrada triangulada que represente al objeto.

	Se tendrá como entrada una nube de puntos con valores de posición $\{x, y, z\}$.
	No se dispondrá de información de textura, normales o conectividades.

	\subsubsection{Suposiciones}
		El ángulo máximo entre dos mallas no podrá exceder los 60\textdegree.

		La cámara no se encontrará demasiado elevada respecto a la
		superficie giratoria. En ningún caso deberá superar el punto más alto del objeto.

\subsection{Requerimientos funcionales}
Se identificaron las siguientes funcionalidades para el sistema:

	\Requerimiento
		{Eliminación de puntos atípicos}
		{El sistema debe detectar y eliminar puntos considerados atípicos.}

	\Requerimiento
		{Alineación Inicial}
		{El sistema debe poder calcular una transformación de alineación para dos mallas
		que las acerque lo suficiente como para poder utilizar luego ICP.}

	\Requerimiento
		{Área solapada}
		{El sistema debe poder establecer los puntos en común (o una buena
		aproximación) entre dos mallas ya alineadas burdamente.}

	\Requerimiento
		{Métricas}
		{El sistema debe poder evaluar la calidad de una registración.}

	\Requerimiento
		{Corrección de bucle}
		{El sistema debe corregir el error propagado durante la registración
		una vez que se haya realizado una vuelta con las capturas.}

	\Requerimiento
		{Combinación de nubes}
		{El sistema debe generar una malla de consenso, ajustando los puntos y sus normales
		según la información provista por cada malla de entrada.}

	\Requerimiento
		{Triangulación}
		{El sistema debe poder triangular una nube de puntos tridimensional.}

	\Requerimiento
		{Relleno}
		{El sistema debe disponer de funciones para lograr que una malla sea cerrada. Se
		estimará una superficie en las zonas donde se carezca de
		información.}

\subsection{Requerimientos no funcionales}
	Se identificaron los siguientes requerimientos no funcionales:

	\Requerimiento{Tiempo de ejecución}
	{No se espera una ejecución a tiempo real de los algoritmos implementados.}

	\Requerimiento{Interfaces con software}
	{Las operaciones sobre las mallas y nubes de puntos se realizará
	mediante la \emph{Point Cloud Library} (PCL).
	Debido a esto, se desarrollará en el lenguaje de programación C++.}


	\Requerimiento{Sistemas operativos}
	{El producto desarrollado estará destinado a utilizarse en los sistemas
	operativos Windows y Linux.}
