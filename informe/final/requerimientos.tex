\chapter{Especificación de requerimientos}

	\section{Introducción}
	El presente documento tiene como propósito definir las especificaciones
	funcionales y no funcionales para el desarrollo de una biblioteca de
	software que proveerá implementaciones de técnicas de registración, fusión
	y relleno de huecos de mallas tridimensionales.

	\section{Descripción}
		Se dispone de un sistema cámara-superficie giratoria, cuyas posiciones
		se encuentran fijas en el espacio y el eje de giro de la superficie se
		encuentra alineado con el eje vertical del dispositivo de captura.
		El objeto de interés se ubicará sobre la superficie giratoria, y se
		realizarán capturas a intervalos de giro regulares
		hasta totalizar una vuelta completa (360\textdegree).

		A partir de estas suposiciones, se desarrollarán algoritmos de
		registración de las capturas parciales,
		fusión de las mallas resultantes
		y relleno de huecos para conseguir una superficie cerrada.

		Las entradas serán nubes de puntos con valores de posición (x,y,z).
		No se dispondrá de información de textura, normales o conectividades.

		Como salida se tendrá una superficie cerrada triangulada.

		\subsection{Interfaces con software}
			Se utilizará la \emph{Point Cloud Library} (PCL)
			para realizar las operaciones sobre las mallas y nubes de puntos.
			Debido a esto, se utilizará el lenguaje de programación C++.


		\subsection{Suposiciones}
		%Se tiene un ambiente controlado, es la forma de trabajo diseñada
			El ángulo máximo entre dos mallas no podrá exceder los 60\textdegree.

			La cámara no se encontrará demasiado elevada respecto a la
			superficie giratoria. En ningún caso deberá superar el punto más alto del objeto. %[PANCHO]
			\clearpage

	\section{Requerimientos funcionales}
		El sistema deberá proveer las siguientes funcionalidades:

		\Requerimiento
			{Outliers}
			{Se debe disponer de funciones para la detección y eliminación de
			puntos considerados outliers.}

		\Requerimiento
			{Registración Inicial}
			{Se debe disponer de funciones que dadas dos mallas calculen una
			transformación que las acerque lo suficiente como para poder
			utilizar luego ICP.}

		\Requerimiento
			{Área solapada}
			{Se debe disponer de funciones que establezcan los puntos en común (o una buena
			aproximación) entre dos mallas ya alineadas burdamente.}

		\Requerimiento
			{Métricas}
			{Se debe disponer de funciones para evaluar la calidad de una registración.}

		\Requerimiento
			{Corrección de bucle}
			{Se debe disponer de funciones para corregir el error propagado durante la registración
			una vez que se haya realizado una vuelta con las capturas.}

		\Requerimiento
			{Combinación de nubes}
			{Se debe disponer de funciones para ajustar los puntos y sus normales,
			según la información provista por cada malla.}

		\Requerimiento
			{Triangulación}
			{Se debe disponer de funciones para obtener una triangulación dada una nube de puntos tridimensional.}

		\Requerimiento
			{Relleno}
			{Se debe disponer de funciones para lograr que una triangulación sea cerrada. Se
			estimará una superficie en las zonas donde se carezca de
			información.}

	\section{Requerimientos no funcionales}
		Se identificaron los siguientes requerimientos no funcionales:
	
		\Requerimiento{Tiempo de ejecución}
		{No se espera una ejecución a tiempo real de los algoritmos implementados.}

		\Requerimiento{Lenguaje de programación}
		{El producto se desarrollará en el lenguaje C++.}

		\Requerimiento{Sistemas operativos}
		{El producto desarrollado estará destinado a utilizarse en los sistemas
		operativos Windows y Linux.}
