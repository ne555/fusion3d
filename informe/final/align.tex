\section{Módulo de registración}
	%¿Qué es la registración?
	%mover al marco teórico
	Este módulo se encargará de obtener las transformaciones de rotación y translación
	que lleven a cada vista a un sistema global de forma
	que las zonas comunes encajen perfectamente.



	%real-time 3d model adquisition
	El algoritmo de \emph{Iterative Closest Point} (ICP) se ha convertido en el
	método dominante para realizar la registración de modelos tridimensionales
	utilizando únicamente la información de geometría de los mismos.
	%generalized icp
	La idea central de este algoritmo se puede resumir en realizar
	iterativamente los siguientes pasos hasta lograr la convergencia.
	\begin{enumerate}
		\item Definir correspondencias entre dos capturas.
		\item Calcular una transformación que minimice la distancia entre estas correspondencias.
	\end{enumerate}
	%a method for registration
	Sin embargo, a pesar de que está garantizado que el algoritmo convergirá a un mínimo local,
	este puede no ser el mínimo global buscado. Entonces, es necesario proveer una alineación inicial
	lo suficientemente cercana para obtener una correcta registración.

	Por esta razón, y considerando que la distancia entre capturas sucesivas es
	considerables, se plantearán algoritmos para obtener una alineación inicial
	con cada par de capturas para luego refinar la registración mediante ICP.


	Debido a que las coordenadas espaciales de un punto no nos suministran suficiente
	información para poder identificarlo en otra vista, es necesario utilizar
	sus relaciones con otros puntos en una vecindad.
	Así, podrá describirse al punto mediante las posiciones relativas, la
	densidad o la orientación, para luego corresponderlo con el más parecido en
	la otra vista.
	Sin embargo, los puntos en zonas homogéneas de la nube serán descriptos de
	forma muy similar, con lo que se corre el riesgo de realizar
	correspondencias equívocas que producirán una alineación errónea.

	Para solventar este último problema, se puede tomar un subconjunto de los
	puntos de entrada, aquellos puntos que por sus características de vecindad
	sea más probable que tengan un descriptor único (\emph{keypoints}).

	Entonces, los métodos utilizados en la alineación inicial seguirán los siguientes pasos:
	\begin{enumerate}
		\item Selección de puntos de la entrada (\emph{keypoints}).
		\item Cálculo de descriptores y determinación de correspondencias.
		\item Filtrado de correspondencias.
		\item Alineación.
	\end{enumerate}
	Variaciones en la ejecución de estos pasos nos permitirán implementar diversos métodos.


	Utilizando la primera captura para definir el sistema de coordenadas
	global, cada nueva vista se alineará con la anterior hasta completar una
	vuelta sobre el objeto.  Sin embargo, debido a que la registración se
	realiza de a pares sucesivos, los errores producidos se propagarán con cada
	captura incorporada, siendo especialmente apreciable al completar una
	vuelta alrededor del objeto (figura~\ref{fig:error_bucle}).
	Por esta razón, se incorporará un algoritmo de corrección de bucle al proceso de registración
	(figura~\ref{fig:flow_registracion}).

	\begin{figure}
		\Imagen{example-image-a}
		\caption{\label{fig:error_bucle}\TODO{imagen error bucle, ISS}}
	\end{figure}

	\begin{figure}
		\Imagen{example-image-a}
		\caption{\label{fig:flow_registracion}\TODO{Diagrama de flujo de la registración.}}
	\end{figure}

	\section{Diagrama de clases}
		En la figura~\ref{fig:align_class} se presentan las clases principales y sus interacciones.
		A continuación se presenta una breve descripción de las mismas.
		\begin{figure}
			\Imagen{uml/align.pdf}
			\caption{\label{fig:align_class}Diagrama de clases del módulo de registración}
		\end{figure}

		\begin{itemize}
			\item {\bfseries Registración:} se encarga de obtener la \emph{Transformación} que
				permita alinear dos \emph{Nubes} entre sí.  Para esto establece
				correspondencias entre los \emph{Anclaje}.
			\item {\bfseries Anclaje:} a partir de puntos salientes de \emph{Nube} calcula
				\emph{descriptores}  que permitan asociarlos y un
				\emph{marco\_de\_referencia} para obtener una estimación de la
				\emph{Transformación}.
			\item {\bfseries Nube:} representa una vista del objeto que se desea alinear.
				Es una colección de \emph{Puntos} sin organización. Clase
				provista por PCL.
			\item {\bfseries Punto:} contiene las coordenadas $xyz$ obtenidas por el
				dispositivo de captura. El algoritmo estimará las normales.
			\item {\bfseries Transformación:} representa una transformación rígida
				(rotación y translación) que será aplicada a una \emph{Nube}
				para alinearla.
		\end{itemize}


	\section{Método A}
		\subsection{Selección de keypoints}
			Basándonos en los resultados obtenidos por \cite{ISS},
			se consideró utilizar el algoritmo de detección de keypoints basado en \emph{Intrinsic Shape Signatures} (algoritmo~\ref{alg:iss}),
			el cual se haya implementado en PCL en la clase \texttt{ISSKeypoint3D}, permitiendo
			definir el radio de la esfera y el nivel de disimilitud.

			%ver bien el problema
			Sin embargo, no pudieron encontrarse los parámetros adecuados.
			Al observar las zonas comunes, eran pocos los keypoints que
			realmente se encontraban lo suficientemente cerca como para generar
			una correspondencia válida (figura~\ref{fig:iss_key}).


			\begin{figure}
				\Imagen{img/iss_happy}
				\caption{\label{fig:iss_key}Keypoints ISS calculados en las vistas \texttt{happy\_0} (verde) y \texttt{happy\_24} (rojo).
				A la derecha se seleccionaron aquellos cuyo par más cercano se encontraban a menos de 4 veces la resolución de la nube.
				}
			\end{figure}

			\begin{algorithm}
				\begin{algorithmic}[1]
					\Function{ISS Keypoints}{nube}
						\State keypoints $\gets\emptyset$
						\ForAll{$p \in \mbox{nube.puntos}$}
							\State vecinos $\gets$ obtener puntos cercanos(nube, p, $r_1$)
							\State m $\gets$ matriz de covarianza(vecinos)
							\State $\lambda$ = eigenvalores(m)
							\If{$\lambda_1/\lambda_2 > \mbox{umbral}_1$ and $\lambda_2/\lambda_3 > \mbox{umbral}_2$}
								\State keypoints.insert(p)
							\EndIf
						\EndFor
						\State\Return Non-Max Suppression(keypoints, $r_2$)
					\EndFunction
				\end{algorithmic}
				\caption{\label{alg:iss}Determinación de los keypoints mediante ISS}
			\end{algorithm}


			%A partir de acá ref Rusu FPFH
			Se procedió entonces a cambiar el método de selección de keypoints, eligiendo ahora un análisis de persistencia multiescala\cite{Rusu:2009:FPF:1703435.1703733}.
			\begin{enumerate}
				\item Por cada punto de la nube se calcula su descriptor para distintos tamaños de vecindad (escala).
				\item A partir de todos los descriptores en todas las escalas se estima una distribución gaussiana que los aproxime.
				\item Los keypoints quedan definidos como aquellos puntos cuyos descriptores se encuentran alejados de la media.
			\end{enumerate}
			El algoritmo se encuentra implementado en PCL en la clase
			\texttt{Multiscale\-Feature\-Persistence} permitiendo ajustar las
			escalas a utilizar, el umbral para ser considerado saliente y la
			función descriptora a utilizar.

			Debido a que es necesario calcular un descriptor para cada punto, y
			además en diferentes escalas, se eligió utilizar el método
			\emph{Fast Point Feature Histograms} (FPFH) para su construcción,
			el cual es lineal en la cantidad de puntos de la vecindad.
			Este método calcula un histograma de los ángulos entre las normales del punto y sus vecinos\cite{Rusu:2009:FPF:1703435.1703733}.

			Los keypoints se encontraban ahora agrupados, formando líneas en zonas de cambio brusco de curvatura (figura~\ref{fig:multiscale_key}).

			\begin{figure}
				\Imagen{img/multiscale_happy}
				\caption{\label{fig:multiscale_key}Keypoints FPFH de persistencia multiescala calculados en las vistas \texttt{happy\_0} (verde) y \texttt{happy\_24} (rojo).}
			\end{figure}




		\subsection{Alineación}
			Se utilizó el algoritmo de \emph{sample consensus initial alignment (SAC-IA)} para la alineación, el cual consiste en:
			\begin{enumerate}
				\item Seleccionar al azar \emph{m} puntos de la nube A
				\item Por cada punto, buscar aquellos con descriptores similares en B y seleccionar uno al azar.
				\item Calcular la transformación definida por estos puntos y
					sus correspondencias. Calcular, además, una medida del
					error de transformación.
				\item Repetir varias veces y devolver aquella transformación que produjo el menor error.
			\end{enumerate}

			\subsection{Resultados}
			Mediante este método, se obtuvieron buenos resultados en la mayoría de las
			capturas de \texttt{happy}
			donde los ángulos eran cercanos a $25^{\circ}$.
			Sin embargo, algunas alineaciones presentaban problemas de deslizamiento.

			En el caso de \texttt{bunny}, cuyos ángulos eran cercanos a $45^{\circ}$,
			se obtuvo una buena alineación en el caso de las capturas 315--000 y 000--045,
			sin embargo, para los otros casos las correspondencias fueron completamente erróneas (figura~\ref{fig:align_sac}).

			%gráficos
			\begin{figure}
				%\Imagen{img/sac_45_90}
				\Imagen{img/bun_sac_270_315}
				\caption{\label{fig:align_sac}Ejemplo de fallo del algoritmo de \emph{sample consensus} producto de malas correspondencias
				en la alineación de las vistas
				\texttt{bun270} (verde) y \texttt{bun315} (rojo).}
				%\texttt{bun045} (verde) y \texttt{bun090} (rojo).}
			\end{figure}


	%El que tengo ahora
	\section{Método B}
		En este caso no se seleccionaron keypoints, simplemente se realizó un submuestreo de
		los puntos de la nube para reducir el costo computacional.
		Además del descriptor, en cada punto se estableció un marco de referencia utilizando los
		eigenvectores de la matriz de covarianza de la vecindad del punto. Este
		marco de referencia nos permite establecer
		la transformación de alineación considerando solamente dos puntos\cite{ISS}.

		Para establecer las correspondencias se utilizó el descriptor FPFH
		comparando los histogramas mediante la distancia $\chi^2$.  Luego se
		procedió a eliminar correspondencias erróneas utilizando los marcos de
		referencia y las suposiciones de ubicación de la cámara en la obtención
		de las capturas.  Por esto, se descartan aquellas correspondencias que
		requieren un movimiento en $y$ excesivo o una rotación sobre un eje no
		vertical. 

		Cada correspondencia entonces define un ángulo de giro $\theta$ sobre
		el eje $y$ y una translación en el plano $xz$.  Se observará entonces
		una agrupación de los parámetros de estas transformaciones y, mediante el
		algoritmo de k-means, se buscará el centroide del clúster más grande (figura~\ref{fig:cluster}).
		%gráfico
		\begin{figure}
			\Imagen{img/cluster_trans}
			\caption{\label{fig:cluster}Translaciones estimadas entre cada par
			de puntos de las correspondencias.}
		\end{figure}

		\subsection{Resultados}
			Se realizaron pruebas sobre los objetos \texttt{happy} y \texttt{bunny}.

			En el caso de \texttt{bunny} se ignoró la captura
			\texttt{bun180.ply} ya que presentaba un salto de aproximadamente
			$90^{\circ}$ respecto a su anterior y siguiente.  Las otras
			capturas presentaban un desvío no mayor a $2^{\circ}$ respecto al
			\emph{ground truth} (figura~\ref{fig:clust_bunny}).

			Para \texttt{happy} se tuvo un desvío medio de poco más de
			$2^{\circ}$, con un máximo de $6^{\circ}$ respecto al \emph{ground truth} (figura~\ref{fig:clust_happy}).

			De esta forma, se logró acercar las capturas lo suficiente como
			para intentar alinearlas por ICP.

		\begin{figure}
			\Imagen{img/cluster_bunny}
			\caption{\label{fig:clust_bunny}Diferencias absolutas entre la rotación estimada y el \emph{ground truth} para el objeto \texttt{bunny}.}
		\end{figure}

		\begin{figure}
			\Imagen{img/cluster_happy}
			\caption{\label{fig:clust_happy}Diferencias absolutas entre la rotación estimada y el \emph{ground truth} para el objeto \texttt{happy}.}
		\end{figure}

		\begin{figure}
			\Imagen{img/bun_clust_270_315}
			\caption{\label{fig:clust_bun_good}Alineación exitosa entre las capturas \texttt{bun270} (verde) y \texttt{bun316} (rojo) mediante el uso de marcos de referencia en cada punto.}
		\end{figure}


	\section{Refinamiento}
	Una vez obtenida la alineación inicial, se procedió a realizar una segunda
	alineación mediante ICP.
	Se consideraron sólo las áreas solapadas y se restringió el
	espacio de búsqueda de las correspondencias.

	Luego, para reducir el error propagado por cada alineación, se propuso una
	corrección de bucle.
	Se ajustó la última captura para que correspondiese con la primera, y
	esta transformación se agregó a las otras alineaciones con un peso
	proporcional a su posición en el bucle.


	%informes de prueba

	%casos de uso/funcionalidades resueltas
