\chapter{Pruebas y resultados}
	A continuación se detallan las pruebas realizadas y los resultados obtenidos
	por cada módulo desarrollado utilizando como base los modelos
	\texttt{armadillo}, \texttt{bunny}, \texttt{dragon}, \texttt{drill} y \texttt{happy}
	del repositorio de Stanford\cite{StanfordScanRep}.

	\section{Módulo de registración}
	Para la registración se utilizó el método basado en la búsqueda de clúster,
	seguido de un refinamiento mediante ICP y una corrección de bucle.
	Se plantearon dos métodos para evaluar la calidad de cada registración:
	\begin{itemize}
		\item Mediante la comparación entre la transformación calculada y aquella provista por la base de datos (\emph{ground truth}).
		\item Mediante una métrica de \emph{fitness} que se obtiene a partir de la nube transformada y las nubes de entrada.
	\end{itemize}

	Para comparar las alineaciones contra el \emph{ground truth}, se
	observa el efecto de las mismas sobre un punto orientado simulando la
	cámara (figura~\ref{fig:err_reg}). El punto \emph{eye} ($C$) se ubica inicialmente en las coordenadas
	$\{0, -0.1, 0.7\}$ (valores obtenidos de la base de datos), y se
	orienta el vector \emph{target} hacia $-z$ y el \emph{up} hacia $y$.
	El error de posicionamiento es la razón entre la distancia al punto
	de inicio y la distancia al punto obtenido por el \emph{ground truth}.
	\[\text{Error} = \frac{|C'-C_{gt}|}{|C_{gt} - C|}\]
	Los errores de \emph{target} y \emph{up} se corresponden al ángulo formado contra los
	vectores respectivos obtenidos por el \emph{ground truth}.

	\begin{figure}
		\centering
		\input{diagram/error_registration.pdf_tex}
		\caption{\label{fig:err_reg}Comparación entre las transformaciones de alineación.
		Se observa el efecto producido en un punto orientado $C$ que simula la posición de la cámara.}
	\end{figure}

	En el cuadro~\ref{tab:reg_error} se presentan los errores de registración promedio para cada orientación de los modelos.
	En la mayoría de los casos, los errores no superan $1^{\circ}$ en orientación ni $1\%$ en posicionamiento,
	observándose dos excepciones: \texttt{bunny} y \texttt{dragon stand}.
	El aumento en el error del modelo \texttt{bunny} se debe a que la captura \texttt{bun180} presenta una distancia cercana a $90^\circ$,
	superando las restricciones impuestas en este trabajo.
	Sin embargo, en el caso de \texttt{dragon stand} el error refleja una mala
	alineación en la captura~12, debida a una mala selección de los parámetros.
	Mediante un posterior ajuste de los parámetros, en particular, del tamaño de la vecindad para el cálculo de los descriptores, se obtuvo una alineación correcta.

	\begin{table}
	\centering
	\begin{tabular}{l*{3}{c}}
		\toprule
		Modelo                   &    Eye          &    Target (grados)        &    Up (grados)\\
		\midrule
		armadillo\\
		{\Em}back          &     0.0062159   &   0.221725     &    0.15211\\
		{\Em}head          &     0.0036356  &    0.102321     &    0.211231\\
		{\Em}head offset   &     0.0029806  &    0.086309     &    0.229574\\
		{\Em}stand         &     0.0022145  &    0.049612     &    0.105862\\
		{\Em}stand flip    &     0.0045019  &    0.125330     &    0.146033\\
		\midrule
		bunny                   &     0.0104809   &   0.598354     &    0.817185\\
		\midrule
		dragon\\
		{\Em}side             &     0.0070872  &    0.178650     &    0.212932\\
		{\Em}stand            &     0.0536199   &   1.379760     &    0.207754\\
		{\Em}up               &     0.0058265  &    0.139297     &    0.0675651\\
		\midrule
		drill                   &     0.0082317  &    0.238639     &    0.100126\\
		\midrule
		happy\\
		{\Em}back              &     0.0088540  &    0.189885     &    0.207247\\
		{\Em}side              &     0.0072675   &   0.175860     &    0.17525\\
		{\Em}stand             &     0.0050124  &    0.101383     &    0.097800\\
		\bottomrule
	\end{tabular}
	\caption[Errores de registración]{\label{tab:reg_error}Errores de registración.}
\end{table}


	Para evaluar la alineación entre un par de capturas prescindiendo del \emph{ground truth}, y situándonos en un escenario más realista,
	se diseñó una medida de \emph{fitness}.
	Esta medida se define como el porcentaje del área solapada entre las nubes una vez alineadas,
	donde un bajo solapamiento nos indicaría un posible error de alineación, como
	se observa en la figura~\ref{fig:fitness}.

	\begin{figure}
		\centering
			\begin{tikzpicture}[gnuplot]
%% generated with GNUPLOT 5.4p0 (Lua 5.4; terminal rev. Jun 2020, script rev. 114)
%% Tue 25 Aug 2020 01:55:34 AM -03
\gpmonochromelines
\path (0.000,0.000) rectangle (12.500,8.750);
\gpcolor{color=gp lt color border}
\gpsetlinetype{gp lt border}
\gpsetdashtype{gp dt solid}
\gpsetlinewidth{1.00}
\draw[gp path] (1.320,0.985)--(1.500,0.985);
\draw[gp path] (11.947,0.985)--(11.767,0.985);
\node[gp node right] at (1.136,0.985) {$0$};
\draw[gp path] (1.320,2.476)--(1.500,2.476);
\draw[gp path] (11.947,2.476)--(11.767,2.476);
\node[gp node right] at (1.136,2.476) {$0.2$};
\draw[gp path] (1.320,3.967)--(1.500,3.967);
\draw[gp path] (11.947,3.967)--(11.767,3.967);
\node[gp node right] at (1.136,3.967) {$0.4$};
\draw[gp path] (1.320,5.459)--(1.500,5.459);
\draw[gp path] (11.947,5.459)--(11.767,5.459);
\node[gp node right] at (1.136,5.459) {$0.6$};
\draw[gp path] (1.320,6.950)--(1.500,6.950);
\draw[gp path] (11.947,6.950)--(11.767,6.950);
\node[gp node right] at (1.136,6.950) {$0.8$};
\draw[gp path] (1.320,8.441)--(1.500,8.441);
\draw[gp path] (11.947,8.441)--(11.767,8.441);
\node[gp node right] at (1.136,8.441) {$1$};
\draw[gp path] (1.984,0.985)--(1.984,1.165);
\draw[gp path] (1.984,8.441)--(1.984,8.261);
\node[gp node center] at (1.984,0.677) {1};
\draw[gp path] (2.648,0.985)--(2.648,1.165);
\draw[gp path] (2.648,8.441)--(2.648,8.261);
\node[gp node center] at (2.648,0.677) {2};
\draw[gp path] (3.313,0.985)--(3.313,1.165);
\draw[gp path] (3.313,8.441)--(3.313,8.261);
\node[gp node center] at (3.313,0.677) {3};
\draw[gp path] (3.977,0.985)--(3.977,1.165);
\draw[gp path] (3.977,8.441)--(3.977,8.261);
\node[gp node center] at (3.977,0.677) {4};
\draw[gp path] (4.641,0.985)--(4.641,1.165);
\draw[gp path] (4.641,8.441)--(4.641,8.261);
\node[gp node center] at (4.641,0.677) {5};
\draw[gp path] (5.305,0.985)--(5.305,1.165);
\draw[gp path] (5.305,8.441)--(5.305,8.261);
\node[gp node center] at (5.305,0.677) {6};
\draw[gp path] (5.969,0.985)--(5.969,1.165);
\draw[gp path] (5.969,8.441)--(5.969,8.261);
\node[gp node center] at (5.969,0.677) {7};
\draw[gp path] (6.634,0.985)--(6.634,1.165);
\draw[gp path] (6.634,8.441)--(6.634,8.261);
\node[gp node center] at (6.634,0.677) {8};
\draw[gp path] (7.298,0.985)--(7.298,1.165);
\draw[gp path] (7.298,8.441)--(7.298,8.261);
\node[gp node center] at (7.298,0.677) {9};
\draw[gp path] (7.962,0.985)--(7.962,1.165);
\draw[gp path] (7.962,8.441)--(7.962,8.261);
\node[gp node center] at (7.962,0.677) {10};
\draw[gp path] (8.626,0.985)--(8.626,1.165);
\draw[gp path] (8.626,8.441)--(8.626,8.261);
\node[gp node center] at (8.626,0.677) {11};
\draw[gp path] (9.290,0.985)--(9.290,1.165);
\draw[gp path] (9.290,8.441)--(9.290,8.261);
\node[gp node center] at (9.290,0.677) {12};
\draw[gp path] (9.954,0.985)--(9.954,1.165);
\draw[gp path] (9.954,8.441)--(9.954,8.261);
\node[gp node center] at (9.954,0.677) {13};
\draw[gp path] (10.619,0.985)--(10.619,1.165);
\draw[gp path] (10.619,8.441)--(10.619,8.261);
\node[gp node center] at (10.619,0.677) {14};
\draw[gp path] (11.283,0.985)--(11.283,1.165);
\draw[gp path] (11.283,8.441)--(11.283,8.261);
\node[gp node center] at (11.283,0.677) {0};
\draw[gp path] (1.320,8.441)--(1.320,0.985)--(11.947,0.985)--(11.947,8.441)--cycle;
\node[gp node center,rotate=-270] at (0.292,4.713) {solapamiento};
\node[gp node center] at (6.633,0.215) {captura};
\draw[gp path] (1.873,0.985)--(1.873,8.158)--(2.095,8.158)--(2.095,0.985)--cycle;
\draw[gp path] (2.538,0.985)--(2.538,8.144)--(2.759,8.144)--(2.759,0.985)--cycle;
\draw[gp path] (3.202,0.985)--(3.202,6.102)--(3.423,6.102)--(3.423,0.985)--cycle;
\draw[gp path] (3.866,0.985)--(3.866,6.882)--(4.087,6.882)--(4.087,0.985)--cycle;
\draw[gp path] (4.530,0.985)--(4.530,5.745)--(4.752,5.745)--(4.752,0.985)--cycle;
\draw[gp path] (5.194,0.985)--(5.194,7.078)--(5.416,7.078)--(5.416,0.985)--cycle;
\draw[gp path] (5.859,0.985)--(5.859,7.902)--(6.080,7.902)--(6.080,0.985)--cycle;
\draw[gp path] (6.523,0.985)--(6.523,8.015)--(6.744,8.015)--(6.744,0.985)--cycle;
\draw[gp path] (7.187,0.985)--(7.187,8.047)--(7.408,8.047)--(7.408,0.985)--cycle;
\draw[gp path] (7.851,0.985)--(7.851,7.892)--(8.073,7.892)--(8.073,0.985)--cycle;
\draw[gp path] (8.515,0.985)--(8.515,7.634)--(8.737,7.634)--(8.737,0.985)--cycle;
\draw[gp path] (9.180,0.985)--(9.180,2.163)--(9.401,2.163)--(9.401,0.985)--cycle;
\draw[gp path] (9.844,0.985)--(9.844,5.858)--(10.065,5.858)--(10.065,0.985)--cycle;
\draw[gp path] (10.508,0.985)--(10.508,7.788)--(10.729,7.788)--(10.729,0.985)--cycle;
\draw[gp path] (11.172,0.985)--(11.172,7.869)--(11.394,7.869)--(11.394,0.985)--cycle;
\draw[gp path] (1.320,8.441)--(1.320,0.985)--(11.947,0.985)--(11.947,8.441)--cycle;
%% coordinates of the plot area
\gpdefrectangularnode{gp plot 1}{\pgfpoint{1.320cm}{0.985cm}}{\pgfpoint{11.947cm}{8.441cm}}
\end{tikzpicture}
%% gnuplot variables

		\caption[Métrica de alineación para el modelo \texttt{dragon stand}]{\label{fig:fitness}Métrica de alineación para el modelo \texttt{dragon stand}. El bajo
		porcentaje de solapamiento en la captura 12 se corresponde
		con un error de registración.}
	\end{figure}


	\section{Módulo de fusión}

