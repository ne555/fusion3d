\chapter{Pruebas y resultados}
	A continuación se detallan las pruebas realizadas y los resultados obtenidos
	por cada módulo desarrollado utilizando como base los modelos
	\texttt{armadillo}, \texttt{bunny}, \texttt{dragon}, \texttt{drill} y \texttt{happy}
	del repositorio de Stanford\cite{StanfordScanRep}.

	\section{Módulo de registración}
	Para la registración se utilizó el método basado en la búsqueda de clúster,
	seguido de un refinamiento mediante ICP y una corrección de bucle.
	Se plantearon dos métodos para evaluar la calidad de cada registración:
	\begin{itemize}
		\item Mediante la comparación entre la transformación calculada y aquella provista por la base de datos (\emph{ground truth});
		\item Mediante una métrica de \emph{fitness} que se obtiene a partir de la nube transformada y las nubes de entrada.
	\end{itemize}

	Para comparar las alineaciones contra el \emph{ground truth}, se
	observa el efecto de las mismas sobre un punto orientado simulando la
	cámara (figura~\ref{fig:err_reg}). El punto \emph{eye} ($C$) se ubica inicialmente en las coordenadas
	$\{0, -0.1, 0.7\}$ (valores obtenidos de la base de datos), y se
	orienta el vector \emph{target} hacia $-z$ y el \emph{up} hacia $y$.
	El error de posicionamiento es la razón entre la distancia al punto
	de inicio y la distancia al punto obtenido por el \emph{ground truth}.
	\[\text{Error} = \frac{|C'-C_{gt}|}{|C_{gt} - C|}\]
	Los errores de \emph{target} y \emph{up} se corresponden al ángulo contra los
	vectores correspondientes obtenidos por el \emph{ground truth}.

	\begin{figure}
		\centering
		\input{diagram/error_registration.pdf_tex}
		\caption{\label{fig:err_reg}Comparación entre las transformaciones de alineación.
		Se observa el efecto producido en un punto orientado $C$ que simula la posición de la cámara.}
	\end{figure}

	Los errores de la registración se observan en el
	cuadro~\ref{tab:reg_error}.  Los errores no superan $1^{\circ}$ en
	orientación ni $1\%$ en posicionamiento.  Se da una excepción en el
	caso de \texttt{dragon stand}, debido a una mala alineación en la
	captura 12 (figura~\ref{fig:fitness}).

	\begin{table}
	\centering
	\begin{tabular}{l*{3}{c}}
		\toprule
		Modelo                   &    Eye          &    Target (grados)        &    Up (grados)\\
		\midrule
		armadillo\\
		{\Em}back          &     0.0062159   &   0.221725     &    0.15211\\
		{\Em}head          &     0.0036356  &    0.102321     &    0.211231\\
		{\Em}head offset   &     0.0029806  &    0.086309     &    0.229574\\
		{\Em}stand         &     0.0022145  &    0.049612     &    0.105862\\
		{\Em}stand flip    &     0.0045019  &    0.125330     &    0.146033\\
		\midrule
		bunny                   &     0.0104809   &   0.598354     &    0.817185\\
		\midrule
		dragon\\
		{\Em}side             &     0.0070872  &    0.178650     &    0.212932\\
		{\Em}stand            &     0.0536199   &   1.379760     &    0.207754\\
		{\Em}up               &     0.0058265  &    0.139297     &    0.0675651\\
		\midrule
		drill                   &     0.0082317  &    0.238639     &    0.100126\\
		\midrule
		happy\\
		{\Em}back              &     0.0088540  &    0.189885     &    0.207247\\
		{\Em}side              &     0.0072675   &   0.175860     &    0.17525\\
		{\Em}stand             &     0.0050124  &    0.101383     &    0.097800\\
		\bottomrule
	\end{tabular}
	\caption[Errores de registración]{\label{tab:reg_error}Errores de registración.}
\end{table}

