no quiero leer todo eso, algunas consideraciones
	Expandir fpfh
	Expandir iss
	Explicar por qué no se usó lo que no se usó


%Proceso de reconstrucción general
%Reverse engineering of geometric models
El objetivo final de los sistemas de ingeniería inversa
es lograr realizar un escáner 3D inteligente.
%Diagrama de flujo
1- Obtención de datos
2- Preproceso
3- Segmentación y surface ¿fitting? (la mejor superficie que representa los puntos
4- Obtención modelo CAD
La 1 está resuelta por Pancho
	pero está limitada debido a consideraciones físicas
	se necesita combinar múltiples vistas.

La parte crítica es la 3

	Problemas con la adquisición
		calibración, precisión, oclusión, ruido, datos faltantes, vistas múltiples

		oclusión -> huecos
			Se producen debido a obstrucciones, sombras o reflejos.
			Se pierden los puntos de esas zonas.

		ruido, ¿cuándo eliminarlo?
			se pierden los detalles del objeto

		rellenado de huecos.
			Debido a la naturaleza del proceso de captura,
			la información obtenida cerca de los bordes es poco confiable (¿por qué?)
				(por las reflecciones, en particular malo para los de láser, había un dibujo)
			En algunas situaciones, los huecos no puden solucionarse con otra vista (¿cuándo?) 

		distribución estadística: la captura es una muestra de la población
			¿cómo ajustar la confianza?

		El acabado de la superficie:
			suavidad y recubrimiento (reflexiones, cabello (demasiadas irregulares))


	Organización de la nube de puntos:
		Nubes organizadas: vecindades	

	CAD/CAM conectividad y continuidad de la estructura.
		(no se tienen métricas de calidad para la impresión)
		poisson es suave, ¿pero qué tanto?

	Combinación de vistas:
		diferente resolución, todo el objeto o un detalle
		observar toda la superficie del objeto, no es trivial y puede requerir \emph{feedback}
		de la reconstrucción

	El mayor problema es obtener una registración precisa, es decir,
	encontrar las transformaciones de rotación y translación que relaciona la
	información que proporciona una captura con otra.

	Lo más simple es una base giratoria. (acá parece que tiene información de ángulos)
	Sin embargo, la base del objeto nunca
	será visible, por lo que para obtener un modelo completo del objeto, es
	necesario cambiar su punto de apoyo.

	Se debe asegurar un solapamiento suficiente para poder determinar la alineación.

%zippered
%There are two main issues in creating a single model from multiple
%range images: registration and integration.  Registration refers to
%computing a rigid transformation that brings the points of one range
%image into alignment with the portions of a surface that is shares with
%another range image.  Integration is the process of creating a single
%surface representation from the sample points from two or more
%range images.
	%agregar hole-filling







	Con el fin de identificar requerimientos y obtener el marco teórico
	necesario para el desarrollo del proyecto se analizaron herramientas de
	software utilizadas para la reconstrucción tridimensional y textos
	científicos acordes a la temática.

	El proceso de reconstrucción generalmente se divide en tres etapas:
	\begin{enumerate}
		\item Registración: donde se determinan las transformaciones necesarias
			para llevar cada vista a su correcta posición en un marco de
			referencia global.
		\item Fusión: donde se unifica el aporte de cada vista para obtener una
			superficie que las englobe.
		\item Relleno de huecos: donde se asegura que la superficie global sea
			cerrada, es decir, que encierre un volumen.
	\end{enumerate}


%FIXME: alineación / registración
%\section{Informe bibliográfico}
	%\section{Introducción}

	%	Iterative closest point:
	%		Busca la transformación que minimice el error de alineación
	%		entre los puntos de las mallas
	%		%efficient_variants_of_the_icp_algorithm.txt
	%		%generalized_icp.txt
	%		implementado en PCL

	%zippered
%All the steps needed to digitize an object  that requires up to 10 range scans
%can be performed using our system with five minutes of user interaction and a
%few hours of compute time.

%A range scanner is any device that senses 3D positions on an
%object’s surface and returns an array of distance values.  A range
%image is an m×n grid of distances (range points) that describe a
%surface either in Cartesian coordinates (a height field) or cylindrical
%coordinates, with two of the coordinates being implicitly defined by
%the indices of the grid.


	\section{Registración}


		La registración entre dos nubes de puntos se suele resolver mediante
		alguna de las variantes del algoritmo \emph{Iterative Closest Point (ICP)}.
		Sin embargo, para evitar caer en mínimos locales,
		se debe contar con una buena aproximación inicial.
		Por esto, es necesario desarrollar algoritmos para conseguir esta
		aproximación inicial.\cite{7271006}
		%Registration with the Point Cloud Library A Modular Framework for Aligning in 3-D

	ICP
	%Rusinkiewicz Levoy
	%efficient variants
	%realtime 3d model adquisition

	El algoritmo de ICP se ha convertido en el método dominante para realizar la alineación
	de modelos tridimensionales utilizando únicamente la información de geometría de los mismos.
	El algoritmo 


	Considera simplemente la información de posición (y normales)
	- Selección de puntos mediante submuestreo uniforme o aleatorio
	- Establecer correspondencias entre las nubes de puntos. Par más cercano entre puntos,
		o considerando la «superficie» (point to point, point to plane)
	- Ponderar las correspondencias (ej, uniforme)
	- Rechazar correspondencias para eliminar puntos *anómalos* (umbral de distancia)
	- Minimizar una métrica de error.  (iterar)
	Según cómo se realice cada uno de estos pasos se obtendrá una variante del algoritmo con
	diferente estabilidad, robustez y eficienci.

%A. ICP
%The key concept of the standard ICP algorithm can be
%summarized in two steps:
%1) compute correspondences between the two scans.
%2) compute a transformation which minimizes distance
%between corresponding points.
%Iteratively repeating these two steps typically results in conver-
%gence to the desired transformation. Because we are violating
%the assumption of full overlap, we are forced to add a maximum
%matching threshold d m a x . This threshold accounts for
%the fact that some points will not have any correspondence in
%the second scan (e.g. points which are outside the boundary of
%scan A). In most implementations of ICP, the choice of d m a x
%represents a trade off between convergence and accuracy. A
%low value results in bad convergence (the algorithm becomes
%“short sighted”); a large value causes incorrect correspon-
%dences to pull the final alignment away from the correct value.



		% efficient_variants_of_the_icp_algorithm
		Los algoritmos de registración se pueden dividir en los siguientes pasos:
		\begin{enumerate}
			\item Selección de puntos de la entrada (\emph{keypoints}).
			\item Utilizar descriptores para establecer correspondencias entre los puntos de las nubes.
			\item Rechazar correspondencias para reducir los \emph{outliers}.
			\item Alineación.\cite{conf/3dim/RusinkiewiczL01}
		\end{enumerate}

		En cuanto a la selección de puntos se tienen como opciones:
		\begin{itemize}
			\item Utilizar todos los puntos o realizar un submuestreo uniforme o aleatorio.
			\item Algoritmos basados en procesamiento de imágenes: como harris\cite{Harris88acombined} y brisk\cite{Leutenegger:2011:BBR:2355573.2356277}.
			\item Algoritmos específicos para puntos 3D: como ISS\cite{ISS}.
			\item Búsqueda de puntos cuyos descriptores sean persistentes a varias escalas: se requiere de formas eficientes de calcular los descriptores, como FPFH\cite{Rusu:2009:FPF:1703435.1703733}.
		\end{itemize}

		%Descriptores:
			Por cada \emph{keypoint} se calculará un descriptor que nos
			permitirá determinar las correspondencias entre las dos nubes de
			puntos.
			Un descriptor es una representación compacta
			de la vecindad de un punto,
			siendo importante además establecer el límite de esta vecindad.

			Para puntos en el espacio, los descriptores utilizan las
			posiciones relativas de los vecinos o los ángulos entre sus
			normales, ponderándolos según la distancia al punto de interés.
			Por ejemplo: el descriptor 3DSC define un histograma tridimensional
			cuantizando la distancia, azimut y elevación entre los puntos;
			el descriptor FPFH realiza algo similar con las normales.

			Ciertos descriptores, como ISS y SHOT, establecen un marco de
			referencia que permite obtener una estimación de la transformación
			entre las nubes a partir de solamente dos puntos.

		\subsection{Corrección de bucle}
			Debido a que la registración se hace de a pares sucesivos, el error
			se acumula con cada nueva malla.  Si se tiene una lista de capturas
			que completan una vuelta sobre el objeto, el error de registración
			acumulado para la última malla de la lista podrá ser apreciado
			especialmente en los bordes de esta respecto a la primera malla.

			Es posible corregir este error al perturbar la última registración
			y luego propagar esta perturbación en las alineaciones anteriores.
			Estas perturbaciones provocan una deformación de la malla, por lo
			cual
			%in-hand_scanning_with_online_loop_closure
			%FIXME: describir el algoritmo
			en \cite{5457479}
			se plantea un algoritmo para el cálculo y la propagación de las
			mismas, de forma que la deformación sea ``lo más rígida posible''.

	\section{Fusión}
		Una vez alineadas las superficies, éstas deben combinarse en una única malla resultante.

		\subsection{Volumetric merge}
		Se divide el espacio en un arreglo de vóxeles, que contienen la
		distancia con signo desde el centro del vóxel a la superficie (la
		distancia será positiva si el vóxel se encuentra en la parte exterior, y
		negativa si se encuentra en el interior).
		De esta forma,
		se tiene definida de forma implícita a la
		superficie donde la distancia es 0, y puede extraerse con, por ejemplo, \emph{marching cubes}.

		En cada nueva registración se actualizan las distancias de los
		vóxeles mediante raycasting,
		desde la posición estimada de la cámara hacia la superficie,
		realizando un promedio ponderado con los valores anteriormente calculados.\cite{Curless:1996:VMB:237170.237269} %a_volumetric_method_for_building_complex_models_from_range_images.txt
		Debido a que sólo nos interesan las zonas donde la distancia es cercana a 0 es posible utilizar una representación rala para los vóxeles.\cite{Steinbrucker:2013:LMS:2586117.2586926} % large-scale_multi-resolution_surface_reconstruction_from_rgb-d_sequences

		%PCL provee la clase TSDFVolume para las operaciones sobre el arreglo de vóxeles.

		\subsection{Zippered}
		%zippered_polygon_meshes_from_range_images.txt
		Obtiene como resultado una malla poligonal.

		El algoritmo trabaja en dos pasos:
		\begin{enumerate}
			\item Aproximación de la topología: se reduce el área solapada
				entre las mallas seleccionando una de las representaciones para
				cada punto en común.
			\item Refinado: se ajusta la posición de cada punto moviéndolo
				según su normal mediante un promedio ponderado dependiente del
				nivel de \emph{confianza} que posee el punto en cada vista.\cite{Turk:1994:ZPM:192161.192241}
		\end{enumerate}

		\subsection{Representación de surfel}
		%in-hand_scanning_with_online_loop_closure.txt
		En lugar de realizar una triangulación, en cada punto se establece un disco de radio variable orientado según la normal.
		Esto facilita la actualización, el agregado y la eliminación de puntos en cada nueva registración y además permite detectar ciertos \emph{outliers}.\cite{5457479}

		Sin embargo, se requiere un post-procesamiento para determinar las conectividades de los puntos.

		\subsection{Poisson Surface Reconstruction}
		%poisson_surface_reconstruction
		La reconstrucción se realiza considerando los puntos y normales de
		todas las nubes a la vez, es decir, no realiza un promedio ponderado como los métodos anteriores.

		Se define una función indicadora $\chi$ que recibe el valor 1 para puntos
		dentro del modelo y 0 para puntos en el exterior.
		La superficie del objeto queda determinada por la frontera entre estos
		valores, y el gradiente de la función indicadora se corresponde con las
		normales de los puntos del objeto $\vec{n}$.

		El problema consiste en encontrar la función $\chi$ cuyo gradiente
		aproxime el campo vectorial definido por las normales. Aplicando el
		operador de divergencia, se obtiene un problema de Poisson: calcular la
		función escalar $\chi$ cuyo laplaciano equivale a la divergencia del
		campo vectorial $\vec{n}$.

		\[\Delta\chi \equiv \nabla \cdot\nabla\chi = \nabla \vec{n}\]

		Utilizando funciones de soporte local para aproximar la solución, se obtiene un sistema lineal ralo bien condicionado.

		El algoritmo rellena huecos automáticamente, pero puede resultar demasiado agresivo, uniendo porciones que deberían permanecer separadas.\cite{Kazhdan:2006:PSR:1281957.1281965}


	\section{Relleno de huecos}
		Los huecos son regiones que ninguna de las vistas logró capturar.
		La existencia de huecos elimina la propiedad de \emph{watertight} de la
		malla, la cual es necesaria para su impresión 3D.

		El método de fusión \emph{volumetric merge} rellena huecos
		automáticamente al definirlos como la frontera entre vóxeles externos y
		aquellos nunca alcanzados por el raycasting.

		En el caso de una malla poligonal,
		pueden identificarse fácilmente los vértices que limitan el hueco
		como aquellos que definen una arista que corresponde a sólo un elemento.
		Entonces puede operarse solamente en la vecindad del hueco.

		Una opción es convertir la superficie en una
		representación volumétrica y luego realizar la difusión de la función
		de distancia en la zona del hueco.
		Este proceso se itera hasta que no se detectan cambios significativos en la superficie.\cite{fillingholes}
		%filling_holes_in_comple_surfaces_using_volumetric_diffusion

		Otra opción es transformar el problema al de una interpolación.
		\begin{itemize}
			\item Por cada hueco se ajusta un plano a los puntos del borde y
				sus vecinos.
			\item Se proyectan los puntos a este plano, obteniéndose un mapa de
				altura del contorno del hueco.
			\item Utilizando el algoritmo de \emph{Moving Least Squares} se
				ajusta una superficie a este mapa de altura.
			\item Se realiza un muestreo sobre esta superficie para obtener
				puntos que rellenen el hueco.
		\end{itemize}
		De esta forma el parche de reconstrucción se unirá suavemente a la malla original.\cite{Filling_holes_on_locally_smooth_surfaces}
			%Filling holes on locally smooth surfaces reconstructed from point clouds


