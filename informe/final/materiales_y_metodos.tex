\chapter{Materiales y métodos}

\section{Base de datos}
Uno de los supuestos de este proyecto era contar con un repositorio propio de
mallas tridimensionales.
Para la creación de este repositorio,
se utilizarían los algoritmos de reconstrucción desarrollados en \TODO{cite{Pancho}},
ubicando al objeto de interés de una base giratoria y realizando capturas 
en ángulos espaciados hasta completar una vuelta.
De esta forma, las posiciones de las vistas describirían un círculo centrado en el objeto y
cada captura contendría información de posición ($xyz$) y de textura ($rgb$), 

Por cuestiones externas, este repositorio nunca se materializó, por lo que fue necesario
la búsqueda de otro con características similares.

%\begin{itemize}
%	\item redwood, freibug:
%	rgb y profundidad, pero el movimiento es pequeño y libre
%	(tendría que eliminar intermedios)
%\item middlebury
%	base giratoria, pero sólo RGB
%	(tendría que generar el mapa de profundidad)
%\item stanford
%	base giratoria, nube de puntos, sin textura.
%	Se optó por esta.
%	Se decidió no generar artificialmente los puntos de textura para tener un
%	caso más real.
%\end{itemize}

Se decidió utilizar \emph{The Stanford 3D Scanning Repository}, que brinda
acceso a escaneos tridimensionales y reconstrucciones detalladas para ser
usados en investigación.

Las capturas fueron obtenidas mediante un escáner láser de barrido Cyberware
3030~MS.  Se realizaron escaneos del objeto en diversas posiciones sobre una
base giratoria y luego estas capturas fueron combinadas para producir una única
malla triangular utilizando el método de \emph{zippering} o bien
\emph{volumetric merging}, ambos métodos desarrollados en
Stanford.\TODO{cite{StanfordScanRep}}

Para resolver la registración se utilizó un método semi-automático,
donde el usuario establece una alineación inicial que luego se ajustará mediante un algoritmo
de ICP modificado. \TODO{cite{zipper}} \TODO{explicar ICP}.
Esta transformación final se provee en un archivo de configuración.

De esta base de datos se utilizaron los modelos
	\textt{armadillo},
	\textt{bunny},
	\textt{dragon},
	\textt{drill} y
	\textt{happy},
los cuales presentan distintos niveles de detalles, cantidad de escaneos y niveles de ruido.



Desgraciadamente, no se cuenta con información de textura de los modelos.
Se decidió adaptar los algoritmos a esta situación, en lugar de agregar
artificialmente valores de color para los puntos.


\section{Tecnologías}
\subsection{\emph{The Point Cloud Library} (PCL)}
Para realizar operaciones sobre las nubes de puntos.
