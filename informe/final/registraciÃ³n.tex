\section{Módulo de registración}
	%¿Qué es la registración?
	%mover al marco teórico
	Este módulo se encargará de obtener las transformaciones de rotación y translación
	que lleven cada vista a un sistema global de forma
	que las zonas comunes encajen perfectamente.



	El algoritmo de \emph{Iterative Closest Point} (ICP) se ha convertido en el
	método dominante para realizar la registración de modelos tridimensionales
	utilizando únicamente la información de geometría de los mismos. \cite{Rusinkiewicz02real-time3d}
	La idea central de este algoritmo se puede resumir en realizar
	iterativamente los siguientes pasos hasta lograr la convergencia.
	\begin{enumerate}
		\item Definir correspondencias entre dos capturas.
		\item Calcular una transformación que minimice la distancia entre estas correspondencias. \cite{conf/rss/SegalHT09}
	\end{enumerate}
	Sin embargo, a pesar de que está garantizado que el algoritmo convergirá a un mínimo local,
	este puede no ser el mínimo global buscado. Entonces, es necesario proveer una alineación inicial
	lo suficientemente cercana para obtener una correcta registración.\cite{regBesl92}

	Por esta razón, y considerando que la distancia entre capturas sucesivas es
	considerable, se plantearán algoritmos para obtener una alineación inicial
	con cada par de capturas para luego refinar la registración mediante ICP.


	Debido a que las coordenadas espaciales de un punto no nos suministran suficiente
	información para poder identificarlo en otra vista, es necesario utilizar
	sus relaciones con otros puntos en una vecindad.
	Así, podrá describirse al punto mediante las posiciones relativas, la
	densidad o la orientación, para luego corresponderlo con el más parecido en
	la otra vista.
	Sin embargo, puntos en zonas homogéneas de la nube serán descriptos de
	forma muy similar, con lo que se corre el riesgo de realizar
	correspondencias equívocas que producirán una alineación errónea.
	\TODO{pero yo uso todos los puntos y funca, entonces no tengo problemas de homogeneidad ¿?}

	Para solventar este último problema, se puede tomar un subconjunto de los
	puntos de entrada, aquellos puntos que por sus características de vecindad
	sea más probable que tengan un descriptor único (\emph{keypoints}).





	Entonces, los métodos utilizados en la alineación inicial seguirán los siguientes pasos:
	\begin{enumerate}
		\item Selección de puntos de la entrada.
		\item Cálculo de descriptores y determinación de correspondencias.
		\item Filtrado de correspondencias.
		\item Estimación de la transformación de alineación.
	\end{enumerate}
	Variaciones en la ejecución de estos pasos nos permitirán implementar diversos métodos.


	Utilizando la primera captura para definir el sistema de coordenadas
	global, cada nueva vista se alineará con la anterior hasta completar una
	vuelta sobre el objeto.  Sin embargo, debido a que la registración se
	realiza de a pares sucesivos, los errores producidos se propagarán con cada
	captura incorporada, siendo especialmente apreciable al completar una
	vuelta alrededor del objeto (figura~\ref{fig:error_bucle}).
	Por esta razón, se incorporará un algoritmo de corrección de bucle al proceso de registración
	(figura~\ref{fig:flow_registracion}).

	\begin{figure}
		\Imagen{diagram/error_bucle_inhand}
		\caption{\label{fig:error_bucle}Visualización del error de bucle. Errores de tan sólo $1^{\circ}$
		en cada registración producen una discrepancia considerable
		donde el modelo debería cerrarse (círculo rojo).}
	\end{figure}

	\begin{figure}
		\Imagen{uml/registration_flow.pdf}
		\caption{\label{fig:flow_registracion}Diagrama de flujo de la registración.}
	\end{figure}

	\subsection{Diagrama de clases}
		Para implementar los distintos algoritmos de alineación inicial se utilizó como base
		el diseño de clases presentado en la figura~\ref{fig:align_class}.
		A continuación se describirán las clases principales y sus interacciones.
		\begin{figure}
			\Imagen{uml/align.pdf}
			\caption{\label{fig:align_class}Diagrama de clases del módulo de registración.}
		\end{figure}

		\begin{itemize}
			\item {\bfseries Registración:} se encarga de obtener la \emph{Transformación} que
				permita alinear dos \emph{Nubes} entre sí.  Para esto establece
				correspondencias entre los \emph{Anclaje}.
			\item {\bfseries Anclaje:} a partir de puntos salientes de \emph{Nube} calcula
				\emph{descriptores}  que permitan asociarlos y un
				\emph{marco de referencia} para obtener una estimación de la
				\emph{Transformación}.
			\item {\bfseries Nube:} representa una vista del objeto que se desea alinear.
				Es una colección de \emph{Puntos} sin organización. Clase
				provista por PCL.
			\item {\bfseries Punto:} contiene las coordenadas $xyz$ obtenidas por el
				dispositivo de captura. El algoritmo estimará las normales.
			\item {\bfseries Transformación:} representa una transformación rígida
				(rotación y translación) que será aplicada a una \emph{Nube}
				para alinearla.
		\end{itemize}


	\subsection{Alineación mediante \emph{sample consensus}}
		\subsubsection{Selección de keypoints}
			Basándonos en los resultados obtenidos por \cite{ISS},
			se consideró utilizar el algoritmo de detección de keypoints basado en \emph{Intrinsic Shape Signatures} (algoritmo~\ref{alg:iss}),
			el cual se halla implementado en PCL en la clase \texttt{ISSKeypoint3D}, permitiendo
			definir el radio de la vecindad y el nivel de disimilitud de los eigenvalores.


			%ver bien el problema
			Sin embargo, no pudieron encontrarse los parámetros adecuados.
			Al observar las zonas comunes, eran pocos los keypoints que
			realmente se encontraban lo suficientemente cerca como para generar
			una correspondencia válida (figura~\ref{fig:iss_key}).


			\begin{figure}
				\centering
				\begin{subfigure}{\linewidth}
					\Imagen{img/iss_happy}
					\caption{\label{fig:iss_key}Keypoints ISS}
				\end{subfigure}

				\begin{subfigure}{\linewidth}
					\Imagen{img/multiscale_happy}
					\caption{\label{fig:multiscale_key}Keypoints FPFH de persistencia multiescala.}
				\end{subfigure}
				\caption{\label{fig:keypoints}Visualización de los keypoints calculados en las vistas
					\texttt{happy\_0} (verde) y \texttt{happy\_24} (rojo).
					A la derecha se seleccionaron aquellos cuyo par
					más cercano se encontraban a menos de 4 veces la resolución
					de la nube.}
			\end{figure}

			\begin{algorithm}
				\begin{algorithmic}[1]
					\Function{ISS Keypoints}{nube, $r_1$, $\mbox{umbral}_1$, $\mbox{umbral}_2$, $r_2$}
						\State keypoints $\gets\emptyset$
						\ForAll{$p \in \mbox{nube.puntos}$}
							\State vecinos $\gets$ obtener puntos cercanos(nube, p, $r_1$)
							\State m $\gets$ matriz de covarianza(vecinos)
							\State $\lambda$ $gets$ eigenvalores(m)
							\If{$\lambda_1/\lambda_2 > \mbox{umbral}_1$ and $\lambda_2/\lambda_3 > \mbox{umbral}_2$}
								\State keypoints.insert(p)
							\EndIf
						\EndFor
						\State\Return Non-Max Suppression(keypoints, $r_2$)
					\EndFunction
				\end{algorithmic}
				\caption{\label{alg:iss}Determinación de los keypoints mediante ISS}
			\end{algorithm}


			%A partir de acá ref Rusu FPFH
			Se procedió entonces a cambiar el método de selección de keypoints, eligiendo ahora un análisis de persistencia multiescala\cite{Rusu:2009:FPF:1703435.1703733}.
			\begin{enumerate}
				\item Por cada punto de la nube se calcula su descriptor para distintos tamaños de vecindad (escala).
				\item A partir de todos los descriptores en todas las escalas se estima una distribución gaussiana que los aproxime.
				\item Los keypoints quedan definidos como aquellos puntos cuyos descriptores se encuentran alejados de la media.
			\end{enumerate}
			El algoritmo se encuentra implementado en PCL en la clase
			\texttt{Multiscale\-Feature\-Persistence} permitiendo ajustar las
			escalas a utilizar, el umbral para ser considerado saliente y la
			función descriptora a utilizar.

			Debido a que es necesario calcular un descriptor para cada punto, y
			además en diferentes escalas, se eligió utilizar el método
			\emph{Fast Point Feature Histograms} (FPFH) para su construcción,
			el cual es lineal en la cantidad de puntos de la vecindad.
			Este método calcula un histograma de los ángulos entre las normales del punto y sus vecinos\cite{Rusu:2009:FPF:1703435.1703733}.

			Los keypoints se encontraban ahora agrupados, formando líneas en zonas de cambio brusco de curvatura (figura~\ref{fig:multiscale_key}).





		\subsubsection{Estimación de la transformación}
			Se utilizó el algoritmo de \emph{sample consensus initial alignment
			(SAC-IA)} para obtener la transformación de alineación. Este algoritmo
			consiste en:
			\begin{enumerate}
				\item Seleccionar al azar \emph{m} puntos de la nube A
				\item Por cada punto, buscar aquellos con descriptores similares en B y seleccionar uno al azar.
				\item Calcular la transformación definida por estos puntos y
					sus correspondencias. Calcular, además, una medida del
					error de transformación.
				\item Repetir varias veces y devolver aquella transformación que produjo el menor error.\cite{Rusu:2009:FPF:1703435.1703733}
			\end{enumerate}


	%El que tengo ahora
	\subsection{Alineación mediante búsqueda de clúster}
		En este caso no se seleccionaron keypoints, simplemente se realizó un submuestreo de
		los puntos de la nube para reducir el costo computacional.
		Además del descriptor, en cada punto se estableció un marco de referencia
		que nos permite estimar
		una transformación de alineación considerando solamente dos puntos\cite{ISS}.

		Este marco de referencia se calcula mediante la matriz de covarianza de la vecindad del punto:
		\begin{itemize}
			\item Se computan los eigenvalores ${\lambda_1, \lambda_2, \lambda_3}$ en orden decreciente y sus eigenvectores correspondientes
				$e^{1}, e^{2}, e^{3}$.
			\item Debido a que $e^{3}$ representa la normal del punto, se ajusta su sentido para que coincida con el del eje $z$.
			\item Los otros ejes se definen mediante $e^{1}$ y $e^{1} \times e^{3}$.
		\end{itemize}
		Se tiene entonces una ambigüedad en el marco de referencia según el sentido que se le asigne a $e^{(1)}$ (figura~\ref{fig:marco_referencia_iss}),
		la cual se resolverá al definir un eje de giro para la alineación.

		\begin{figure}
			\Imagen{diagram/marco_referencia_iss}
			\caption{\label{fig:marco_referencia_iss}Marcos de referencia ISS. Se observa una ambigüedad equivalente a un giro de $180^{\circ}$ sobre el eje $e^{3}$.}
		\end{figure}

		Para establecer las correspondencias se utilizó el descriptor FPFH
		comparando los histogramas mediante la distancia $\chi^2$.
		\[ \chi^2 = \sum_{j=1}^{N} \frac{\left(a_j - b_j\right)^2}{a_j + b_j} \]
		Luego se
		procedió a eliminar correspondencias erróneas utilizando los marcos de
		referencia y las suposiciones de ubicación de la cámara en la obtención
		de las capturas.  Por esto, se descartan aquellas correspondencias que
		requieren un movimiento en $y$ excesivo o una rotación sobre un eje no
		vertical. 

		Cada correspondencia entonces define un ángulo de giro $\theta$ sobre
		el eje $y$ y una translación en el plano $xz$.  Se observará entonces
		una agrupación de los parámetros de estas transformaciones y, mediante el
		algoritmo de k-means, se buscará el centroide del clúster más grande (figura~\ref{fig:cluster}).
		\begin{figure}
			\resizebox{.9\linewidth}{!}{% Title: gl2ps_renderer figure
% Creator: GL2PS 1.4.0, (C) 1999-2017 C. Geuzaine
% For: Octave
% CreationDate: Wed Feb 26 12:22:37 2020
\begin{pgfpicture}
\color[rgb]{1.000000,1.000000,1.000000}
\pgfpathrectanglecorners{\pgfpoint{0pt}{0pt}}{\pgfpoint{576pt}{432pt}}
\pgfusepath{fill}
\begin{pgfscope}
\pgfpathrectangle{\pgfpoint{0pt}{0pt}}{\pgfpoint{576pt}{432pt}}
\pgfusepath{fill}
\pgfpathrectangle{\pgfpoint{0pt}{0pt}}{\pgfpoint{576pt}{432pt}}
\pgfusepath{clip}
\pgfpathmoveto{\pgfpoint{122.040253pt}{399.599915pt}}
\pgflineto{\pgfpoint{474.120056pt}{47.519958pt}}
\pgflineto{\pgfpoint{122.040253pt}{47.519958pt}}
\pgfpathclose
\pgfusepath{fill,stroke}
\pgfpathmoveto{\pgfpoint{122.040253pt}{399.599915pt}}
\pgflineto{\pgfpoint{474.120056pt}{399.599915pt}}
\pgflineto{\pgfpoint{474.120056pt}{47.519958pt}}
\pgfpathclose
\pgfusepath{fill,stroke}
\color[rgb]{0.150000,0.150000,0.150000}
\pgfsetlinewidth{0.500000pt}
\pgfpathmoveto{\pgfpoint{188.261169pt}{51.984970pt}}
\pgflineto{\pgfpoint{188.261169pt}{47.519958pt}}
\pgfusepath{stroke}
\pgfpathmoveto{\pgfpoint{275.526398pt}{51.984970pt}}
\pgflineto{\pgfpoint{275.526398pt}{47.519958pt}}
\pgfusepath{stroke}
\pgfpathmoveto{\pgfpoint{362.791626pt}{51.984970pt}}
\pgflineto{\pgfpoint{362.791626pt}{47.519958pt}}
\pgfusepath{stroke}
\pgfpathmoveto{\pgfpoint{450.056702pt}{51.984970pt}}
\pgflineto{\pgfpoint{450.056702pt}{47.519958pt}}
\pgfusepath{stroke}
{
\pgftransformshift{\pgfpoint{188.261154pt}{40.018234pt}}
\pgfnode{rectangle}{north}{\fontsize{10}{0}\selectfont\textcolor[rgb]{0.15,0.15,0.15}{{-0.054}}}{}{\pgfusepath{discard}}}
{
\pgftransformshift{\pgfpoint{275.526337pt}{40.018234pt}}
\pgfnode{rectangle}{north}{\fontsize{10}{0}\selectfont\textcolor[rgb]{0.15,0.15,0.15}{{-0.052}}}{}{\pgfusepath{discard}}}
{
\pgftransformshift{\pgfpoint{362.791534pt}{40.018234pt}}
\pgfnode{rectangle}{north}{\fontsize{10}{0}\selectfont\textcolor[rgb]{0.15,0.15,0.15}{{-0.05}}}{}{\pgfusepath{discard}}}
{
\pgftransformshift{\pgfpoint{450.056549pt}{40.018234pt}}
\pgfnode{rectangle}{north}{\fontsize{10}{0}\selectfont\textcolor[rgb]{0.15,0.15,0.15}{{-0.048}}}{}{\pgfusepath{discard}}}
\pgfpathmoveto{\pgfpoint{126.505096pt}{47.519958pt}}
\pgflineto{\pgfpoint{122.040253pt}{47.519958pt}}
\pgfusepath{stroke}
\pgfpathmoveto{\pgfpoint{126.505096pt}{117.935951pt}}
\pgflineto{\pgfpoint{122.040253pt}{117.935951pt}}
\pgfusepath{stroke}
\pgfpathmoveto{\pgfpoint{126.505096pt}{188.351944pt}}
\pgflineto{\pgfpoint{122.040253pt}{188.351944pt}}
\pgfusepath{stroke}
\pgfpathmoveto{\pgfpoint{126.505096pt}{258.767944pt}}
\pgflineto{\pgfpoint{122.040253pt}{258.767944pt}}
\pgfusepath{stroke}
\pgfpathmoveto{\pgfpoint{126.505096pt}{329.183990pt}}
\pgflineto{\pgfpoint{122.040253pt}{329.183990pt}}
\pgfusepath{stroke}
\pgfpathmoveto{\pgfpoint{126.505096pt}{399.599915pt}}
\pgflineto{\pgfpoint{122.040253pt}{399.599915pt}}
\pgfusepath{stroke}
{
\pgftransformshift{\pgfpoint{117.038879pt}{47.519897pt}}
\pgfnode{rectangle}{east}{\fontsize{10}{0}\selectfont\textcolor[rgb]{0.15,0.15,0.15}{{-0.013}}}{}{\pgfusepath{discard}}}
{
\pgftransformshift{\pgfpoint{117.038879pt}{117.935921pt}}
\pgfnode{rectangle}{east}{\fontsize{10}{0}\selectfont\textcolor[rgb]{0.15,0.15,0.15}{{-0.012}}}{}{\pgfusepath{discard}}}
{
\pgftransformshift{\pgfpoint{117.038879pt}{188.351944pt}}
\pgfnode{rectangle}{east}{\fontsize{10}{0}\selectfont\textcolor[rgb]{0.15,0.15,0.15}{{-0.011}}}{}{\pgfusepath{discard}}}
{
\pgftransformshift{\pgfpoint{117.038879pt}{258.767944pt}}
\pgfnode{rectangle}{east}{\fontsize{10}{0}\selectfont\textcolor[rgb]{0.15,0.15,0.15}{{-0.01}}}{}{\pgfusepath{discard}}}
{
\pgftransformshift{\pgfpoint{117.038879pt}{329.183990pt}}
\pgfnode{rectangle}{east}{\fontsize{10}{0}\selectfont\textcolor[rgb]{0.15,0.15,0.15}{{-0.009}}}{}{\pgfusepath{discard}}}
{
\pgftransformshift{\pgfpoint{117.038879pt}{399.599976pt}}
\pgfnode{rectangle}{east}{\fontsize{10}{0}\selectfont\textcolor[rgb]{0.15,0.15,0.15}{{-0.008}}}{}{\pgfusepath{discard}}}
\pgfsetrectcap
\pgfsetdash{{16pt}{0pt}}{0pt}
\pgfpathmoveto{\pgfpoint{474.120056pt}{47.519958pt}}
\pgflineto{\pgfpoint{122.040253pt}{47.519958pt}}
\pgfusepath{stroke}
\pgfpathmoveto{\pgfpoint{122.040253pt}{399.599915pt}}
\pgflineto{\pgfpoint{122.040253pt}{47.519958pt}}
\pgfusepath{stroke}
{
\pgftransformshift{\pgfpoint{298.080170pt}{27.018234pt}}
\pgfnode{rectangle}{north}{\fontsize{11}{0}\selectfont\textcolor[rgb]{0.15,0.15,0.15}{{x}}}{}{\pgfusepath{discard}}}
{
\pgftransformshift{\pgfpoint{82.038910pt}{223.559998pt}}
\pgftransformrotate{90.000000}{\pgfnode{rectangle}{south}{\fontsize{11}{0}\selectfont\textcolor[rgb]{0.15,0.15,0.15}{{z}}}{}{\pgfusepath{discard}}}}
\pgfsetlinewidth{0.01pt}
\pgfsetroundcap
\pgfsetroundjoin
\color[rgb]{0.000000,0.000000,0.000000}
\pgfpathmoveto{\pgfpoint{204.390198pt}{253.184006pt}}
\pgflineto{\pgfpoint{202.317261pt}{250.330841pt}}
\pgflineto{\pgfpoint{203.817261pt}{251.420654pt}}
\pgfpathclose
\pgfusepath{fill,stroke}
\pgfpathmoveto{\pgfpoint{204.390198pt}{253.184006pt}}
\pgflineto{\pgfpoint{200.463165pt}{250.330841pt}}
\pgflineto{\pgfpoint{202.317261pt}{250.330841pt}}
\pgfpathclose
\pgfusepath{fill,stroke}
\pgfpathmoveto{\pgfpoint{204.390198pt}{253.184006pt}}
\pgflineto{\pgfpoint{198.963165pt}{251.420654pt}}
\pgflineto{\pgfpoint{200.463165pt}{250.330841pt}}
\pgfpathclose
\pgfusepath{fill,stroke}
\pgfpathmoveto{\pgfpoint{204.390198pt}{253.184006pt}}
\pgflineto{\pgfpoint{198.390213pt}{253.184006pt}}
\pgflineto{\pgfpoint{198.963165pt}{251.420654pt}}
\pgfpathclose
\pgfusepath{fill,stroke}
\pgfpathmoveto{\pgfpoint{204.390198pt}{253.184006pt}}
\pgflineto{\pgfpoint{198.963165pt}{254.947357pt}}
\pgflineto{\pgfpoint{198.390213pt}{253.184006pt}}
\pgfpathclose
\pgfusepath{fill,stroke}
\pgfpathmoveto{\pgfpoint{204.390198pt}{253.184006pt}}
\pgflineto{\pgfpoint{200.463165pt}{256.037170pt}}
\pgflineto{\pgfpoint{198.963165pt}{254.947357pt}}
\pgfpathclose
\pgfusepath{fill,stroke}
\pgfpathmoveto{\pgfpoint{204.390198pt}{253.184006pt}}
\pgflineto{\pgfpoint{202.317261pt}{256.037170pt}}
\pgflineto{\pgfpoint{200.463165pt}{256.037170pt}}
\pgfpathclose
\pgfusepath{fill,stroke}
\pgfpathmoveto{\pgfpoint{204.390198pt}{253.184006pt}}
\pgflineto{\pgfpoint{203.817261pt}{254.947357pt}}
\pgflineto{\pgfpoint{202.317261pt}{256.037170pt}}
\pgfpathclose
\pgfusepath{fill,stroke}
\pgfpathmoveto{\pgfpoint{226.525024pt}{258.000458pt}}
\pgflineto{\pgfpoint{224.452072pt}{255.147308pt}}
\pgflineto{\pgfpoint{225.952072pt}{256.237122pt}}
\pgfpathclose
\pgfusepath{fill,stroke}
\pgfpathmoveto{\pgfpoint{226.525024pt}{258.000458pt}}
\pgflineto{\pgfpoint{222.597961pt}{255.147308pt}}
\pgflineto{\pgfpoint{224.452072pt}{255.147308pt}}
\pgfpathclose
\pgfusepath{fill,stroke}
\pgfpathmoveto{\pgfpoint{226.525024pt}{258.000458pt}}
\pgflineto{\pgfpoint{221.097961pt}{256.237122pt}}
\pgflineto{\pgfpoint{222.597961pt}{255.147308pt}}
\pgfpathclose
\pgfusepath{fill,stroke}
\pgfpathmoveto{\pgfpoint{226.525024pt}{258.000458pt}}
\pgflineto{\pgfpoint{220.525024pt}{258.000458pt}}
\pgflineto{\pgfpoint{221.097961pt}{256.237122pt}}
\pgfpathclose
\pgfusepath{fill,stroke}
\pgfpathmoveto{\pgfpoint{226.525024pt}{258.000458pt}}
\pgflineto{\pgfpoint{221.097961pt}{259.763824pt}}
\pgflineto{\pgfpoint{220.525024pt}{258.000458pt}}
\pgfpathclose
\pgfusepath{fill,stroke}
\pgfpathmoveto{\pgfpoint{226.525024pt}{258.000458pt}}
\pgflineto{\pgfpoint{222.597961pt}{260.853638pt}}
\pgflineto{\pgfpoint{221.097961pt}{259.763824pt}}
\pgfpathclose
\pgfusepath{fill,stroke}
\pgfpathmoveto{\pgfpoint{226.525024pt}{258.000458pt}}
\pgflineto{\pgfpoint{224.452072pt}{260.853638pt}}
\pgflineto{\pgfpoint{222.597961pt}{260.853638pt}}
\pgfpathclose
\pgfusepath{fill,stroke}
\pgfpathmoveto{\pgfpoint{226.525024pt}{258.000458pt}}
\pgflineto{\pgfpoint{225.952072pt}{259.763824pt}}
\pgflineto{\pgfpoint{224.452072pt}{260.853638pt}}
\pgfpathclose
\pgfusepath{fill,stroke}
\pgfpathmoveto{\pgfpoint{221.241119pt}{260.096741pt}}
\pgflineto{\pgfpoint{219.168152pt}{257.243591pt}}
\pgflineto{\pgfpoint{220.668152pt}{258.333405pt}}
\pgfpathclose
\pgfusepath{fill,stroke}
\pgfpathmoveto{\pgfpoint{221.241119pt}{260.096741pt}}
\pgflineto{\pgfpoint{217.314056pt}{257.243591pt}}
\pgflineto{\pgfpoint{219.168152pt}{257.243591pt}}
\pgfpathclose
\pgfusepath{fill,stroke}
\pgfpathmoveto{\pgfpoint{221.241119pt}{260.096741pt}}
\pgflineto{\pgfpoint{215.814056pt}{258.333405pt}}
\pgflineto{\pgfpoint{217.314056pt}{257.243591pt}}
\pgfpathclose
\pgfusepath{fill,stroke}
\pgfpathmoveto{\pgfpoint{221.241119pt}{260.096741pt}}
\pgflineto{\pgfpoint{215.241119pt}{260.096741pt}}
\pgflineto{\pgfpoint{215.814056pt}{258.333405pt}}
\pgfpathclose
\pgfusepath{fill,stroke}
\pgfpathmoveto{\pgfpoint{221.241119pt}{260.096741pt}}
\pgflineto{\pgfpoint{215.814056pt}{261.860107pt}}
\pgflineto{\pgfpoint{215.241119pt}{260.096741pt}}
\pgfpathclose
\pgfusepath{fill,stroke}
\pgfpathmoveto{\pgfpoint{221.241119pt}{260.096741pt}}
\pgflineto{\pgfpoint{217.314056pt}{262.949921pt}}
\pgflineto{\pgfpoint{215.814056pt}{261.860107pt}}
\pgfpathclose
\pgfusepath{fill,stroke}
\pgfpathmoveto{\pgfpoint{221.241119pt}{260.096741pt}}
\pgflineto{\pgfpoint{219.168152pt}{262.949921pt}}
\pgflineto{\pgfpoint{217.314056pt}{262.949921pt}}
\pgfpathclose
\pgfusepath{fill,stroke}
\pgfpathmoveto{\pgfpoint{221.241119pt}{260.096741pt}}
\pgflineto{\pgfpoint{220.668152pt}{261.860107pt}}
\pgflineto{\pgfpoint{219.168152pt}{262.949921pt}}
\pgfpathclose
\pgfusepath{fill,stroke}
\pgfpathmoveto{\pgfpoint{236.988129pt}{252.261566pt}}
\pgflineto{\pgfpoint{234.915176pt}{249.408386pt}}
\pgflineto{\pgfpoint{236.415176pt}{250.498199pt}}
\pgfpathclose
\pgfusepath{fill,stroke}
\pgfpathmoveto{\pgfpoint{236.988129pt}{252.261566pt}}
\pgflineto{\pgfpoint{233.061066pt}{249.408386pt}}
\pgflineto{\pgfpoint{234.915176pt}{249.408386pt}}
\pgfpathclose
\pgfusepath{fill,stroke}
\pgfpathmoveto{\pgfpoint{236.988129pt}{252.261566pt}}
\pgflineto{\pgfpoint{231.561066pt}{250.498199pt}}
\pgflineto{\pgfpoint{233.061066pt}{249.408386pt}}
\pgfpathclose
\pgfusepath{fill,stroke}
\pgfpathmoveto{\pgfpoint{236.988129pt}{252.261566pt}}
\pgflineto{\pgfpoint{230.988129pt}{252.261566pt}}
\pgflineto{\pgfpoint{231.561066pt}{250.498199pt}}
\pgfpathclose
\pgfusepath{fill,stroke}
\pgfpathmoveto{\pgfpoint{236.988129pt}{252.261566pt}}
\pgflineto{\pgfpoint{231.561066pt}{254.024918pt}}
\pgflineto{\pgfpoint{230.988129pt}{252.261566pt}}
\pgfpathclose
\pgfusepath{fill,stroke}
\pgfpathmoveto{\pgfpoint{236.988129pt}{252.261566pt}}
\pgflineto{\pgfpoint{233.061066pt}{255.114731pt}}
\pgflineto{\pgfpoint{231.561066pt}{254.024918pt}}
\pgfpathclose
\pgfusepath{fill,stroke}
\pgfpathmoveto{\pgfpoint{236.988129pt}{252.261566pt}}
\pgflineto{\pgfpoint{234.915176pt}{255.114731pt}}
\pgflineto{\pgfpoint{233.061066pt}{255.114731pt}}
\pgfpathclose
\pgfusepath{fill,stroke}
\pgfpathmoveto{\pgfpoint{236.988129pt}{252.261566pt}}
\pgflineto{\pgfpoint{236.415176pt}{254.024918pt}}
\pgflineto{\pgfpoint{234.915176pt}{255.114731pt}}
\pgfpathclose
\pgfusepath{fill,stroke}
\pgfpathmoveto{\pgfpoint{228.296494pt}{255.289444pt}}
\pgflineto{\pgfpoint{226.223541pt}{252.436279pt}}
\pgflineto{\pgfpoint{227.723541pt}{253.526077pt}}
\pgfpathclose
\pgfusepath{fill,stroke}
\pgfpathmoveto{\pgfpoint{228.296494pt}{255.289444pt}}
\pgflineto{\pgfpoint{224.369446pt}{252.436279pt}}
\pgflineto{\pgfpoint{226.223541pt}{252.436279pt}}
\pgfpathclose
\pgfusepath{fill,stroke}
\pgfpathmoveto{\pgfpoint{228.296494pt}{255.289444pt}}
\pgflineto{\pgfpoint{222.869446pt}{253.526077pt}}
\pgflineto{\pgfpoint{224.369446pt}{252.436279pt}}
\pgfpathclose
\pgfusepath{fill,stroke}
\pgfpathmoveto{\pgfpoint{228.296494pt}{255.289444pt}}
\pgflineto{\pgfpoint{222.296494pt}{255.289444pt}}
\pgflineto{\pgfpoint{222.869446pt}{253.526077pt}}
\pgfpathclose
\pgfusepath{fill,stroke}
\pgfpathmoveto{\pgfpoint{228.296494pt}{255.289444pt}}
\pgflineto{\pgfpoint{222.869446pt}{257.052795pt}}
\pgflineto{\pgfpoint{222.296494pt}{255.289444pt}}
\pgfpathclose
\pgfusepath{fill,stroke}
\pgfpathmoveto{\pgfpoint{228.296494pt}{255.289444pt}}
\pgflineto{\pgfpoint{224.369446pt}{258.142609pt}}
\pgflineto{\pgfpoint{222.869446pt}{257.052795pt}}
\pgfpathclose
\pgfusepath{fill,stroke}
\pgfpathmoveto{\pgfpoint{228.296494pt}{255.289444pt}}
\pgflineto{\pgfpoint{226.223541pt}{258.142609pt}}
\pgflineto{\pgfpoint{224.369446pt}{258.142609pt}}
\pgfpathclose
\pgfusepath{fill,stroke}
\pgfpathmoveto{\pgfpoint{228.296494pt}{255.289444pt}}
\pgflineto{\pgfpoint{227.723541pt}{257.052795pt}}
\pgflineto{\pgfpoint{226.223541pt}{258.142609pt}}
\pgfpathclose
\pgfusepath{fill,stroke}
\pgfpathmoveto{\pgfpoint{199.411743pt}{249.951904pt}}
\pgflineto{\pgfpoint{197.338776pt}{247.098740pt}}
\pgflineto{\pgfpoint{198.838791pt}{248.188553pt}}
\pgfpathclose
\pgfusepath{fill,stroke}
\pgfpathmoveto{\pgfpoint{199.411743pt}{249.951904pt}}
\pgflineto{\pgfpoint{195.484680pt}{247.098740pt}}
\pgflineto{\pgfpoint{197.338776pt}{247.098740pt}}
\pgfpathclose
\pgfusepath{fill,stroke}
\pgfpathmoveto{\pgfpoint{199.411743pt}{249.951904pt}}
\pgflineto{\pgfpoint{193.984680pt}{248.188553pt}}
\pgflineto{\pgfpoint{195.484680pt}{247.098740pt}}
\pgfpathclose
\pgfusepath{fill,stroke}
\pgfpathmoveto{\pgfpoint{199.411743pt}{249.951904pt}}
\pgflineto{\pgfpoint{193.411743pt}{249.951904pt}}
\pgflineto{\pgfpoint{193.984680pt}{248.188553pt}}
\pgfpathclose
\pgfusepath{fill,stroke}
\pgfpathmoveto{\pgfpoint{199.411743pt}{249.951904pt}}
\pgflineto{\pgfpoint{193.984680pt}{251.715271pt}}
\pgflineto{\pgfpoint{193.411743pt}{249.951904pt}}
\pgfpathclose
\pgfusepath{fill,stroke}
\pgfpathmoveto{\pgfpoint{199.411743pt}{249.951904pt}}
\pgflineto{\pgfpoint{195.484680pt}{252.805084pt}}
\pgflineto{\pgfpoint{193.984680pt}{251.715271pt}}
\pgfpathclose
\pgfusepath{fill,stroke}
\pgfpathmoveto{\pgfpoint{199.411743pt}{249.951904pt}}
\pgflineto{\pgfpoint{197.338776pt}{252.805084pt}}
\pgflineto{\pgfpoint{195.484680pt}{252.805084pt}}
\pgfpathclose
\pgfusepath{fill,stroke}
\pgfpathmoveto{\pgfpoint{199.411743pt}{249.951904pt}}
\pgflineto{\pgfpoint{198.838791pt}{251.715271pt}}
\pgflineto{\pgfpoint{197.338776pt}{252.805084pt}}
\pgfpathclose
\pgfusepath{fill,stroke}
\pgfpathmoveto{\pgfpoint{248.515839pt}{248.437973pt}}
\pgflineto{\pgfpoint{246.442886pt}{245.584808pt}}
\pgflineto{\pgfpoint{247.942886pt}{246.674622pt}}
\pgfpathclose
\pgfusepath{fill,stroke}
\pgfpathmoveto{\pgfpoint{248.515839pt}{248.437973pt}}
\pgflineto{\pgfpoint{244.588776pt}{245.584808pt}}
\pgflineto{\pgfpoint{246.442886pt}{245.584808pt}}
\pgfpathclose
\pgfusepath{fill,stroke}
\pgfpathmoveto{\pgfpoint{248.515839pt}{248.437973pt}}
\pgflineto{\pgfpoint{243.088776pt}{246.674622pt}}
\pgflineto{\pgfpoint{244.588776pt}{245.584808pt}}
\pgfpathclose
\pgfusepath{fill,stroke}
\pgfpathmoveto{\pgfpoint{248.515839pt}{248.437973pt}}
\pgflineto{\pgfpoint{242.515839pt}{248.437973pt}}
\pgflineto{\pgfpoint{243.088776pt}{246.674622pt}}
\pgfpathclose
\pgfusepath{fill,stroke}
\pgfpathmoveto{\pgfpoint{248.515839pt}{248.437973pt}}
\pgflineto{\pgfpoint{243.088776pt}{250.201324pt}}
\pgflineto{\pgfpoint{242.515839pt}{248.437973pt}}
\pgfpathclose
\pgfusepath{fill,stroke}
\pgfpathmoveto{\pgfpoint{248.515839pt}{248.437973pt}}
\pgflineto{\pgfpoint{244.588776pt}{251.291138pt}}
\pgflineto{\pgfpoint{243.088776pt}{250.201324pt}}
\pgfpathclose
\pgfusepath{fill,stroke}
\pgfpathmoveto{\pgfpoint{248.515839pt}{248.437973pt}}
\pgflineto{\pgfpoint{246.442886pt}{251.291138pt}}
\pgflineto{\pgfpoint{244.588776pt}{251.291138pt}}
\pgfpathclose
\pgfusepath{fill,stroke}
\pgfpathmoveto{\pgfpoint{248.515839pt}{248.437973pt}}
\pgflineto{\pgfpoint{247.942886pt}{250.201324pt}}
\pgflineto{\pgfpoint{246.442886pt}{251.291138pt}}
\pgfpathclose
\pgfusepath{fill,stroke}
\pgfpathmoveto{\pgfpoint{229.513855pt}{247.888733pt}}
\pgflineto{\pgfpoint{227.440903pt}{245.035568pt}}
\pgflineto{\pgfpoint{228.940903pt}{246.125366pt}}
\pgfpathclose
\pgfusepath{fill,stroke}
\pgfpathmoveto{\pgfpoint{229.513855pt}{247.888733pt}}
\pgflineto{\pgfpoint{225.586792pt}{245.035568pt}}
\pgflineto{\pgfpoint{227.440903pt}{245.035568pt}}
\pgfpathclose
\pgfusepath{fill,stroke}
\pgfpathmoveto{\pgfpoint{229.513855pt}{247.888733pt}}
\pgflineto{\pgfpoint{224.086792pt}{246.125366pt}}
\pgflineto{\pgfpoint{225.586792pt}{245.035568pt}}
\pgfpathclose
\pgfusepath{fill,stroke}
\pgfpathmoveto{\pgfpoint{229.513855pt}{247.888733pt}}
\pgflineto{\pgfpoint{223.513855pt}{247.888733pt}}
\pgflineto{\pgfpoint{224.086792pt}{246.125366pt}}
\pgfpathclose
\pgfusepath{fill,stroke}
\pgfpathmoveto{\pgfpoint{229.513855pt}{247.888733pt}}
\pgflineto{\pgfpoint{224.086792pt}{249.652084pt}}
\pgflineto{\pgfpoint{223.513855pt}{247.888733pt}}
\pgfpathclose
\pgfusepath{fill,stroke}
\pgfpathmoveto{\pgfpoint{229.513855pt}{247.888733pt}}
\pgflineto{\pgfpoint{225.586792pt}{250.741898pt}}
\pgflineto{\pgfpoint{224.086792pt}{249.652084pt}}
\pgfpathclose
\pgfusepath{fill,stroke}
\pgfpathmoveto{\pgfpoint{229.513855pt}{247.888733pt}}
\pgflineto{\pgfpoint{227.440903pt}{250.741898pt}}
\pgflineto{\pgfpoint{225.586792pt}{250.741898pt}}
\pgfpathclose
\pgfusepath{fill,stroke}
\pgfpathmoveto{\pgfpoint{229.513855pt}{247.888733pt}}
\pgflineto{\pgfpoint{228.940903pt}{249.652084pt}}
\pgflineto{\pgfpoint{227.440903pt}{250.741898pt}}
\pgfpathclose
\pgfusepath{fill,stroke}
\pgfpathmoveto{\pgfpoint{235.177353pt}{231.819794pt}}
\pgflineto{\pgfpoint{233.104401pt}{228.966629pt}}
\pgflineto{\pgfpoint{234.604401pt}{230.056442pt}}
\pgfpathclose
\pgfusepath{fill,stroke}
\pgfpathmoveto{\pgfpoint{235.177353pt}{231.819794pt}}
\pgflineto{\pgfpoint{231.250305pt}{228.966629pt}}
\pgflineto{\pgfpoint{233.104401pt}{228.966629pt}}
\pgfpathclose
\pgfusepath{fill,stroke}
\pgfpathmoveto{\pgfpoint{235.177353pt}{231.819794pt}}
\pgflineto{\pgfpoint{229.750305pt}{230.056442pt}}
\pgflineto{\pgfpoint{231.250305pt}{228.966629pt}}
\pgfpathclose
\pgfusepath{fill,stroke}
\pgfpathmoveto{\pgfpoint{235.177353pt}{231.819794pt}}
\pgflineto{\pgfpoint{229.177353pt}{231.819794pt}}
\pgflineto{\pgfpoint{229.750305pt}{230.056442pt}}
\pgfpathclose
\pgfusepath{fill,stroke}
\pgfpathmoveto{\pgfpoint{235.177353pt}{231.819794pt}}
\pgflineto{\pgfpoint{229.750305pt}{233.583160pt}}
\pgflineto{\pgfpoint{229.177353pt}{231.819794pt}}
\pgfpathclose
\pgfusepath{fill,stroke}
\pgfpathmoveto{\pgfpoint{235.177353pt}{231.819794pt}}
\pgflineto{\pgfpoint{231.250305pt}{234.672974pt}}
\pgflineto{\pgfpoint{229.750305pt}{233.583160pt}}
\pgfpathclose
\pgfusepath{fill,stroke}
\pgfpathmoveto{\pgfpoint{235.177353pt}{231.819794pt}}
\pgflineto{\pgfpoint{233.104401pt}{234.672974pt}}
\pgflineto{\pgfpoint{231.250305pt}{234.672974pt}}
\pgfpathclose
\pgfusepath{fill,stroke}
\pgfpathmoveto{\pgfpoint{235.177353pt}{231.819794pt}}
\pgflineto{\pgfpoint{234.604401pt}{233.583160pt}}
\pgflineto{\pgfpoint{233.104401pt}{234.672974pt}}
\pgfpathclose
\pgfusepath{fill,stroke}
\pgfpathmoveto{\pgfpoint{228.623734pt}{245.248123pt}}
\pgflineto{\pgfpoint{226.550781pt}{242.394958pt}}
\pgflineto{\pgfpoint{228.050781pt}{243.484772pt}}
\pgfpathclose
\pgfusepath{fill,stroke}
\pgfpathmoveto{\pgfpoint{228.623734pt}{245.248123pt}}
\pgflineto{\pgfpoint{224.696686pt}{242.394958pt}}
\pgflineto{\pgfpoint{226.550781pt}{242.394958pt}}
\pgfpathclose
\pgfusepath{fill,stroke}
\pgfpathmoveto{\pgfpoint{228.623734pt}{245.248123pt}}
\pgflineto{\pgfpoint{223.196686pt}{243.484772pt}}
\pgflineto{\pgfpoint{224.696686pt}{242.394958pt}}
\pgfpathclose
\pgfusepath{fill,stroke}
\pgfpathmoveto{\pgfpoint{228.623734pt}{245.248123pt}}
\pgflineto{\pgfpoint{222.623734pt}{245.248123pt}}
\pgflineto{\pgfpoint{223.196686pt}{243.484772pt}}
\pgfpathclose
\pgfusepath{fill,stroke}
\pgfpathmoveto{\pgfpoint{228.623734pt}{245.248123pt}}
\pgflineto{\pgfpoint{223.196686pt}{247.011475pt}}
\pgflineto{\pgfpoint{222.623734pt}{245.248123pt}}
\pgfpathclose
\pgfusepath{fill,stroke}
\pgfpathmoveto{\pgfpoint{228.623734pt}{245.248123pt}}
\pgflineto{\pgfpoint{224.696686pt}{248.101288pt}}
\pgflineto{\pgfpoint{223.196686pt}{247.011475pt}}
\pgfpathclose
\pgfusepath{fill,stroke}
\pgfpathmoveto{\pgfpoint{228.623734pt}{245.248123pt}}
\pgflineto{\pgfpoint{226.550781pt}{248.101288pt}}
\pgflineto{\pgfpoint{224.696686pt}{248.101288pt}}
\pgfpathclose
\pgfusepath{fill,stroke}
\pgfpathmoveto{\pgfpoint{228.623734pt}{245.248123pt}}
\pgflineto{\pgfpoint{228.050781pt}{247.011475pt}}
\pgflineto{\pgfpoint{226.550781pt}{248.101288pt}}
\pgfpathclose
\pgfusepath{fill,stroke}
\pgfpathmoveto{\pgfpoint{217.671967pt}{249.233673pt}}
\pgflineto{\pgfpoint{215.599030pt}{246.380508pt}}
\pgflineto{\pgfpoint{217.099030pt}{247.470322pt}}
\pgfpathclose
\pgfusepath{fill,stroke}
\pgfpathmoveto{\pgfpoint{217.671967pt}{249.233673pt}}
\pgflineto{\pgfpoint{213.744934pt}{246.380508pt}}
\pgflineto{\pgfpoint{215.599030pt}{246.380508pt}}
\pgfpathclose
\pgfusepath{fill,stroke}
\pgfpathmoveto{\pgfpoint{217.671967pt}{249.233673pt}}
\pgflineto{\pgfpoint{212.244934pt}{247.470322pt}}
\pgflineto{\pgfpoint{213.744934pt}{246.380508pt}}
\pgfpathclose
\pgfusepath{fill,stroke}
\pgfpathmoveto{\pgfpoint{217.671967pt}{249.233673pt}}
\pgflineto{\pgfpoint{211.671982pt}{249.233673pt}}
\pgflineto{\pgfpoint{212.244934pt}{247.470322pt}}
\pgfpathclose
\pgfusepath{fill,stroke}
\pgfpathmoveto{\pgfpoint{217.671967pt}{249.233673pt}}
\pgflineto{\pgfpoint{212.244934pt}{250.997040pt}}
\pgflineto{\pgfpoint{211.671982pt}{249.233673pt}}
\pgfpathclose
\pgfusepath{fill,stroke}
\pgfpathmoveto{\pgfpoint{217.671967pt}{249.233673pt}}
\pgflineto{\pgfpoint{213.744934pt}{252.086853pt}}
\pgflineto{\pgfpoint{212.244934pt}{250.997040pt}}
\pgfpathclose
\pgfusepath{fill,stroke}
\pgfpathmoveto{\pgfpoint{217.671967pt}{249.233673pt}}
\pgflineto{\pgfpoint{215.599030pt}{252.086853pt}}
\pgflineto{\pgfpoint{213.744934pt}{252.086853pt}}
\pgfpathclose
\pgfusepath{fill,stroke}
\pgfpathmoveto{\pgfpoint{217.671967pt}{249.233673pt}}
\pgflineto{\pgfpoint{217.099030pt}{250.997040pt}}
\pgflineto{\pgfpoint{215.599030pt}{252.086853pt}}
\pgfpathclose
\pgfusepath{fill,stroke}
\pgfpathmoveto{\pgfpoint{289.796600pt}{177.289642pt}}
\pgflineto{\pgfpoint{287.723663pt}{174.436478pt}}
\pgflineto{\pgfpoint{289.223663pt}{175.526291pt}}
\pgfpathclose
\pgfusepath{fill,stroke}
\pgfpathmoveto{\pgfpoint{289.796600pt}{177.289642pt}}
\pgflineto{\pgfpoint{285.869537pt}{174.436478pt}}
\pgflineto{\pgfpoint{287.723663pt}{174.436478pt}}
\pgfpathclose
\pgfusepath{fill,stroke}
\pgfpathmoveto{\pgfpoint{289.796600pt}{177.289642pt}}
\pgflineto{\pgfpoint{284.369537pt}{175.526291pt}}
\pgflineto{\pgfpoint{285.869537pt}{174.436478pt}}
\pgfpathclose
\pgfusepath{fill,stroke}
\pgfpathmoveto{\pgfpoint{289.796600pt}{177.289642pt}}
\pgflineto{\pgfpoint{283.796600pt}{177.289642pt}}
\pgflineto{\pgfpoint{284.369537pt}{175.526291pt}}
\pgfpathclose
\pgfusepath{fill,stroke}
\pgfpathmoveto{\pgfpoint{289.796600pt}{177.289642pt}}
\pgflineto{\pgfpoint{284.369537pt}{179.053009pt}}
\pgflineto{\pgfpoint{283.796600pt}{177.289642pt}}
\pgfpathclose
\pgfusepath{fill,stroke}
\pgfpathmoveto{\pgfpoint{289.796600pt}{177.289642pt}}
\pgflineto{\pgfpoint{285.869537pt}{180.142822pt}}
\pgflineto{\pgfpoint{284.369537pt}{179.053009pt}}
\pgfpathclose
\pgfusepath{fill,stroke}
\pgfpathmoveto{\pgfpoint{289.796600pt}{177.289642pt}}
\pgflineto{\pgfpoint{287.723663pt}{180.142822pt}}
\pgflineto{\pgfpoint{285.869537pt}{180.142822pt}}
\pgfpathclose
\pgfusepath{fill,stroke}
\pgfpathmoveto{\pgfpoint{289.796600pt}{177.289642pt}}
\pgflineto{\pgfpoint{289.223663pt}{179.053009pt}}
\pgflineto{\pgfpoint{287.723663pt}{180.142822pt}}
\pgfpathclose
\pgfusepath{fill,stroke}
\pgfpathmoveto{\pgfpoint{241.368820pt}{246.367737pt}}
\pgflineto{\pgfpoint{239.295868pt}{243.514572pt}}
\pgflineto{\pgfpoint{240.795868pt}{244.604385pt}}
\pgfpathclose
\pgfusepath{fill,stroke}
\pgfpathmoveto{\pgfpoint{241.368820pt}{246.367737pt}}
\pgflineto{\pgfpoint{237.441772pt}{243.514572pt}}
\pgflineto{\pgfpoint{239.295868pt}{243.514572pt}}
\pgfpathclose
\pgfusepath{fill,stroke}
\pgfpathmoveto{\pgfpoint{241.368820pt}{246.367737pt}}
\pgflineto{\pgfpoint{235.941772pt}{244.604385pt}}
\pgflineto{\pgfpoint{237.441772pt}{243.514572pt}}
\pgfpathclose
\pgfusepath{fill,stroke}
\pgfpathmoveto{\pgfpoint{241.368820pt}{246.367737pt}}
\pgflineto{\pgfpoint{235.368820pt}{246.367737pt}}
\pgflineto{\pgfpoint{235.941772pt}{244.604385pt}}
\pgfpathclose
\pgfusepath{fill,stroke}
\pgfpathmoveto{\pgfpoint{241.368820pt}{246.367737pt}}
\pgflineto{\pgfpoint{235.941772pt}{248.131104pt}}
\pgflineto{\pgfpoint{235.368820pt}{246.367737pt}}
\pgfpathclose
\pgfusepath{fill,stroke}
\pgfpathmoveto{\pgfpoint{241.368820pt}{246.367737pt}}
\pgflineto{\pgfpoint{237.441772pt}{249.220917pt}}
\pgflineto{\pgfpoint{235.941772pt}{248.131104pt}}
\pgfpathclose
\pgfusepath{fill,stroke}
\pgfpathmoveto{\pgfpoint{241.368820pt}{246.367737pt}}
\pgflineto{\pgfpoint{239.295868pt}{249.220917pt}}
\pgflineto{\pgfpoint{237.441772pt}{249.220917pt}}
\pgfpathclose
\pgfusepath{fill,stroke}
\pgfpathmoveto{\pgfpoint{241.368820pt}{246.367737pt}}
\pgflineto{\pgfpoint{240.795868pt}{248.131104pt}}
\pgflineto{\pgfpoint{239.295868pt}{249.220917pt}}
\pgfpathclose
\pgfusepath{fill,stroke}
\pgfpathmoveto{\pgfpoint{229.736374pt}{242.473740pt}}
\pgflineto{\pgfpoint{227.663422pt}{239.620575pt}}
\pgflineto{\pgfpoint{229.163422pt}{240.710388pt}}
\pgfpathclose
\pgfusepath{fill,stroke}
\pgfpathmoveto{\pgfpoint{229.736374pt}{242.473740pt}}
\pgflineto{\pgfpoint{225.809326pt}{239.620575pt}}
\pgflineto{\pgfpoint{227.663422pt}{239.620575pt}}
\pgfpathclose
\pgfusepath{fill,stroke}
\pgfpathmoveto{\pgfpoint{229.736374pt}{242.473740pt}}
\pgflineto{\pgfpoint{224.309326pt}{240.710388pt}}
\pgflineto{\pgfpoint{225.809326pt}{239.620575pt}}
\pgfpathclose
\pgfusepath{fill,stroke}
\pgfpathmoveto{\pgfpoint{229.736374pt}{242.473740pt}}
\pgflineto{\pgfpoint{223.736374pt}{242.473740pt}}
\pgflineto{\pgfpoint{224.309326pt}{240.710388pt}}
\pgfpathclose
\pgfusepath{fill,stroke}
\pgfpathmoveto{\pgfpoint{229.736374pt}{242.473740pt}}
\pgflineto{\pgfpoint{224.309326pt}{244.237091pt}}
\pgflineto{\pgfpoint{223.736374pt}{242.473740pt}}
\pgfpathclose
\pgfusepath{fill,stroke}
\pgfpathmoveto{\pgfpoint{229.736374pt}{242.473740pt}}
\pgflineto{\pgfpoint{225.809326pt}{245.326904pt}}
\pgflineto{\pgfpoint{224.309326pt}{244.237091pt}}
\pgfpathclose
\pgfusepath{fill,stroke}
\pgfpathmoveto{\pgfpoint{229.736374pt}{242.473740pt}}
\pgflineto{\pgfpoint{227.663422pt}{245.326904pt}}
\pgflineto{\pgfpoint{225.809326pt}{245.326904pt}}
\pgfpathclose
\pgfusepath{fill,stroke}
\pgfpathmoveto{\pgfpoint{229.736374pt}{242.473740pt}}
\pgflineto{\pgfpoint{229.163422pt}{244.237091pt}}
\pgflineto{\pgfpoint{227.663422pt}{245.326904pt}}
\pgfpathclose
\pgfusepath{fill,stroke}
\pgfpathmoveto{\pgfpoint{218.858765pt}{247.346527pt}}
\pgflineto{\pgfpoint{216.785828pt}{244.493362pt}}
\pgflineto{\pgfpoint{218.285828pt}{245.583176pt}}
\pgfpathclose
\pgfusepath{fill,stroke}
\pgfpathmoveto{\pgfpoint{218.858765pt}{247.346527pt}}
\pgflineto{\pgfpoint{214.931732pt}{244.493362pt}}
\pgflineto{\pgfpoint{216.785828pt}{244.493362pt}}
\pgfpathclose
\pgfusepath{fill,stroke}
\pgfpathmoveto{\pgfpoint{218.858765pt}{247.346527pt}}
\pgflineto{\pgfpoint{213.431732pt}{245.583176pt}}
\pgflineto{\pgfpoint{214.931732pt}{244.493362pt}}
\pgfpathclose
\pgfusepath{fill,stroke}
\pgfpathmoveto{\pgfpoint{218.858765pt}{247.346527pt}}
\pgflineto{\pgfpoint{212.858780pt}{247.346527pt}}
\pgflineto{\pgfpoint{213.431732pt}{245.583176pt}}
\pgfpathclose
\pgfusepath{fill,stroke}
\pgfpathmoveto{\pgfpoint{218.858765pt}{247.346527pt}}
\pgflineto{\pgfpoint{213.431732pt}{249.109894pt}}
\pgflineto{\pgfpoint{212.858780pt}{247.346527pt}}
\pgfpathclose
\pgfusepath{fill,stroke}
\pgfpathmoveto{\pgfpoint{218.858765pt}{247.346527pt}}
\pgflineto{\pgfpoint{214.931732pt}{250.199707pt}}
\pgflineto{\pgfpoint{213.431732pt}{249.109894pt}}
\pgfpathclose
\pgfusepath{fill,stroke}
\pgfpathmoveto{\pgfpoint{218.858765pt}{247.346527pt}}
\pgflineto{\pgfpoint{216.785828pt}{250.199707pt}}
\pgflineto{\pgfpoint{214.931732pt}{250.199707pt}}
\pgfpathclose
\pgfusepath{fill,stroke}
\pgfpathmoveto{\pgfpoint{218.858765pt}{247.346527pt}}
\pgflineto{\pgfpoint{218.285828pt}{249.109894pt}}
\pgflineto{\pgfpoint{216.785828pt}{250.199707pt}}
\pgfpathclose
\pgfusepath{fill,stroke}
\pgfpathmoveto{\pgfpoint{218.452988pt}{247.212738pt}}
\pgflineto{\pgfpoint{216.380035pt}{244.359573pt}}
\pgflineto{\pgfpoint{217.880035pt}{245.449387pt}}
\pgfpathclose
\pgfusepath{fill,stroke}
\pgfpathmoveto{\pgfpoint{218.452988pt}{247.212738pt}}
\pgflineto{\pgfpoint{214.525940pt}{244.359573pt}}
\pgflineto{\pgfpoint{216.380035pt}{244.359573pt}}
\pgfpathclose
\pgfusepath{fill,stroke}
\pgfpathmoveto{\pgfpoint{218.452988pt}{247.212738pt}}
\pgflineto{\pgfpoint{213.025940pt}{245.449387pt}}
\pgflineto{\pgfpoint{214.525940pt}{244.359573pt}}
\pgfpathclose
\pgfusepath{fill,stroke}
\pgfpathmoveto{\pgfpoint{218.452988pt}{247.212738pt}}
\pgflineto{\pgfpoint{212.452988pt}{247.212738pt}}
\pgflineto{\pgfpoint{213.025940pt}{245.449387pt}}
\pgfpathclose
\pgfusepath{fill,stroke}
\pgfpathmoveto{\pgfpoint{218.452988pt}{247.212738pt}}
\pgflineto{\pgfpoint{213.025940pt}{248.976089pt}}
\pgflineto{\pgfpoint{212.452988pt}{247.212738pt}}
\pgfpathclose
\pgfusepath{fill,stroke}
\pgfpathmoveto{\pgfpoint{218.452988pt}{247.212738pt}}
\pgflineto{\pgfpoint{214.525940pt}{250.065903pt}}
\pgflineto{\pgfpoint{213.025940pt}{248.976089pt}}
\pgfpathclose
\pgfusepath{fill,stroke}
\pgfpathmoveto{\pgfpoint{218.452988pt}{247.212738pt}}
\pgflineto{\pgfpoint{216.380035pt}{250.065903pt}}
\pgflineto{\pgfpoint{214.525940pt}{250.065903pt}}
\pgfpathclose
\pgfusepath{fill,stroke}
\pgfpathmoveto{\pgfpoint{218.452988pt}{247.212738pt}}
\pgflineto{\pgfpoint{217.880035pt}{248.976089pt}}
\pgflineto{\pgfpoint{216.380035pt}{250.065903pt}}
\pgfpathclose
\pgfusepath{fill,stroke}
\pgfpathmoveto{\pgfpoint{250.143326pt}{235.636337pt}}
\pgflineto{\pgfpoint{248.070374pt}{232.783157pt}}
\pgflineto{\pgfpoint{249.570374pt}{233.872971pt}}
\pgfpathclose
\pgfusepath{fill,stroke}
\pgfpathmoveto{\pgfpoint{250.143326pt}{235.636337pt}}
\pgflineto{\pgfpoint{246.216278pt}{232.783157pt}}
\pgflineto{\pgfpoint{248.070374pt}{232.783157pt}}
\pgfpathclose
\pgfusepath{fill,stroke}
\pgfpathmoveto{\pgfpoint{250.143326pt}{235.636337pt}}
\pgflineto{\pgfpoint{244.716278pt}{233.872971pt}}
\pgflineto{\pgfpoint{246.216278pt}{232.783157pt}}
\pgfpathclose
\pgfusepath{fill,stroke}
\pgfpathmoveto{\pgfpoint{250.143326pt}{235.636337pt}}
\pgflineto{\pgfpoint{244.143326pt}{235.636337pt}}
\pgflineto{\pgfpoint{244.716278pt}{233.872971pt}}
\pgfpathclose
\pgfusepath{fill,stroke}
\pgfpathmoveto{\pgfpoint{250.143326pt}{235.636337pt}}
\pgflineto{\pgfpoint{244.716278pt}{237.399689pt}}
\pgflineto{\pgfpoint{244.143326pt}{235.636337pt}}
\pgfpathclose
\pgfusepath{fill,stroke}
\pgfpathmoveto{\pgfpoint{250.143326pt}{235.636337pt}}
\pgflineto{\pgfpoint{246.216278pt}{238.489502pt}}
\pgflineto{\pgfpoint{244.716278pt}{237.399689pt}}
\pgfpathclose
\pgfusepath{fill,stroke}
\pgfpathmoveto{\pgfpoint{250.143326pt}{235.636337pt}}
\pgflineto{\pgfpoint{248.070374pt}{238.489502pt}}
\pgflineto{\pgfpoint{246.216278pt}{238.489502pt}}
\pgfpathclose
\pgfusepath{fill,stroke}
\pgfpathmoveto{\pgfpoint{250.143326pt}{235.636337pt}}
\pgflineto{\pgfpoint{249.570374pt}{237.399689pt}}
\pgflineto{\pgfpoint{248.070374pt}{238.489502pt}}
\pgfpathclose
\pgfusepath{fill,stroke}
\pgfpathmoveto{\pgfpoint{252.167877pt}{225.200699pt}}
\pgflineto{\pgfpoint{250.094940pt}{222.347519pt}}
\pgflineto{\pgfpoint{251.594940pt}{223.437332pt}}
\pgfpathclose
\pgfusepath{fill,stroke}
\pgfpathmoveto{\pgfpoint{252.167877pt}{225.200699pt}}
\pgflineto{\pgfpoint{248.240829pt}{222.347519pt}}
\pgflineto{\pgfpoint{250.094940pt}{222.347519pt}}
\pgfpathclose
\pgfusepath{fill,stroke}
\pgfpathmoveto{\pgfpoint{252.167877pt}{225.200699pt}}
\pgflineto{\pgfpoint{246.740829pt}{223.437332pt}}
\pgflineto{\pgfpoint{248.240829pt}{222.347519pt}}
\pgfpathclose
\pgfusepath{fill,stroke}
\pgfpathmoveto{\pgfpoint{252.167877pt}{225.200699pt}}
\pgflineto{\pgfpoint{246.167877pt}{225.200699pt}}
\pgflineto{\pgfpoint{246.740829pt}{223.437332pt}}
\pgfpathclose
\pgfusepath{fill,stroke}
\pgfpathmoveto{\pgfpoint{252.167877pt}{225.200699pt}}
\pgflineto{\pgfpoint{246.740829pt}{226.964050pt}}
\pgflineto{\pgfpoint{246.167877pt}{225.200699pt}}
\pgfpathclose
\pgfusepath{fill,stroke}
\pgfpathmoveto{\pgfpoint{252.167877pt}{225.200699pt}}
\pgflineto{\pgfpoint{248.240829pt}{228.053864pt}}
\pgflineto{\pgfpoint{246.740829pt}{226.964050pt}}
\pgfpathclose
\pgfusepath{fill,stroke}
\pgfpathmoveto{\pgfpoint{252.167877pt}{225.200699pt}}
\pgflineto{\pgfpoint{250.094940pt}{228.053864pt}}
\pgflineto{\pgfpoint{248.240829pt}{228.053864pt}}
\pgfpathclose
\pgfusepath{fill,stroke}
\pgfpathmoveto{\pgfpoint{252.167877pt}{225.200699pt}}
\pgflineto{\pgfpoint{251.594940pt}{226.964050pt}}
\pgflineto{\pgfpoint{250.094940pt}{228.053864pt}}
\pgfpathclose
\pgfusepath{fill,stroke}
\pgfpathmoveto{\pgfpoint{183.559998pt}{251.148987pt}}
\pgflineto{\pgfpoint{181.487061pt}{248.295807pt}}
\pgflineto{\pgfpoint{182.987061pt}{249.385620pt}}
\pgfpathclose
\pgfusepath{fill,stroke}
\pgfpathmoveto{\pgfpoint{183.559998pt}{251.148987pt}}
\pgflineto{\pgfpoint{179.632965pt}{248.295807pt}}
\pgflineto{\pgfpoint{181.487061pt}{248.295807pt}}
\pgfpathclose
\pgfusepath{fill,stroke}
\pgfpathmoveto{\pgfpoint{183.559998pt}{251.148987pt}}
\pgflineto{\pgfpoint{178.132965pt}{249.385620pt}}
\pgflineto{\pgfpoint{179.632965pt}{248.295807pt}}
\pgfpathclose
\pgfusepath{fill,stroke}
\pgfpathmoveto{\pgfpoint{183.559998pt}{251.148987pt}}
\pgflineto{\pgfpoint{177.560013pt}{251.148987pt}}
\pgflineto{\pgfpoint{178.132965pt}{249.385620pt}}
\pgfpathclose
\pgfusepath{fill,stroke}
\pgfpathmoveto{\pgfpoint{183.559998pt}{251.148987pt}}
\pgflineto{\pgfpoint{178.132965pt}{252.912338pt}}
\pgflineto{\pgfpoint{177.560013pt}{251.148987pt}}
\pgfpathclose
\pgfusepath{fill,stroke}
\pgfpathmoveto{\pgfpoint{183.559998pt}{251.148987pt}}
\pgflineto{\pgfpoint{179.632965pt}{254.002151pt}}
\pgflineto{\pgfpoint{178.132965pt}{252.912338pt}}
\pgfpathclose
\pgfusepath{fill,stroke}
\pgfpathmoveto{\pgfpoint{183.559998pt}{251.148987pt}}
\pgflineto{\pgfpoint{181.487061pt}{254.002151pt}}
\pgflineto{\pgfpoint{179.632965pt}{254.002151pt}}
\pgfpathclose
\pgfusepath{fill,stroke}
\pgfpathmoveto{\pgfpoint{183.559998pt}{251.148987pt}}
\pgflineto{\pgfpoint{182.987061pt}{252.912338pt}}
\pgflineto{\pgfpoint{181.487061pt}{254.002151pt}}
\pgfpathclose
\pgfusepath{fill,stroke}
\pgfpathmoveto{\pgfpoint{270.990967pt}{182.084976pt}}
\pgflineto{\pgfpoint{268.917999pt}{179.231812pt}}
\pgflineto{\pgfpoint{270.417999pt}{180.321625pt}}
\pgfpathclose
\pgfusepath{fill,stroke}
\pgfpathmoveto{\pgfpoint{270.990967pt}{182.084976pt}}
\pgflineto{\pgfpoint{267.063904pt}{179.231812pt}}
\pgflineto{\pgfpoint{268.917999pt}{179.231812pt}}
\pgfpathclose
\pgfusepath{fill,stroke}
\pgfpathmoveto{\pgfpoint{270.990967pt}{182.084976pt}}
\pgflineto{\pgfpoint{265.563904pt}{180.321625pt}}
\pgflineto{\pgfpoint{267.063904pt}{179.231812pt}}
\pgfpathclose
\pgfusepath{fill,stroke}
\pgfpathmoveto{\pgfpoint{270.990967pt}{182.084976pt}}
\pgflineto{\pgfpoint{264.990967pt}{182.084976pt}}
\pgflineto{\pgfpoint{265.563904pt}{180.321625pt}}
\pgfpathclose
\pgfusepath{fill,stroke}
\pgfpathmoveto{\pgfpoint{270.990967pt}{182.084976pt}}
\pgflineto{\pgfpoint{265.563904pt}{183.848328pt}}
\pgflineto{\pgfpoint{264.990967pt}{182.084976pt}}
\pgfpathclose
\pgfusepath{fill,stroke}
\pgfpathmoveto{\pgfpoint{270.990967pt}{182.084976pt}}
\pgflineto{\pgfpoint{267.063904pt}{184.938141pt}}
\pgflineto{\pgfpoint{265.563904pt}{183.848328pt}}
\pgfpathclose
\pgfusepath{fill,stroke}
\pgfpathmoveto{\pgfpoint{270.990967pt}{182.084976pt}}
\pgflineto{\pgfpoint{268.917999pt}{184.938141pt}}
\pgflineto{\pgfpoint{267.063904pt}{184.938141pt}}
\pgfpathclose
\pgfusepath{fill,stroke}
\pgfpathmoveto{\pgfpoint{270.990967pt}{182.084976pt}}
\pgflineto{\pgfpoint{270.417999pt}{183.848328pt}}
\pgflineto{\pgfpoint{268.917999pt}{184.938141pt}}
\pgfpathclose
\pgfusepath{fill,stroke}
\pgfpathmoveto{\pgfpoint{432.401001pt}{93.895973pt}}
\pgflineto{\pgfpoint{430.328033pt}{91.042809pt}}
\pgflineto{\pgfpoint{431.828033pt}{92.132614pt}}
\pgfpathclose
\pgfusepath{fill,stroke}
\pgfpathmoveto{\pgfpoint{432.401001pt}{93.895973pt}}
\pgflineto{\pgfpoint{428.473938pt}{91.042809pt}}
\pgflineto{\pgfpoint{430.328033pt}{91.042809pt}}
\pgfpathclose
\pgfusepath{fill,stroke}
\pgfpathmoveto{\pgfpoint{432.401001pt}{93.895973pt}}
\pgflineto{\pgfpoint{426.973938pt}{92.132614pt}}
\pgflineto{\pgfpoint{428.473938pt}{91.042809pt}}
\pgfpathclose
\pgfusepath{fill,stroke}
\pgfpathmoveto{\pgfpoint{432.401001pt}{93.895973pt}}
\pgflineto{\pgfpoint{426.401001pt}{93.895973pt}}
\pgflineto{\pgfpoint{426.973938pt}{92.132614pt}}
\pgfpathclose
\pgfusepath{fill,stroke}
\pgfpathmoveto{\pgfpoint{432.401001pt}{93.895973pt}}
\pgflineto{\pgfpoint{426.973938pt}{95.659332pt}}
\pgflineto{\pgfpoint{426.401001pt}{93.895973pt}}
\pgfpathclose
\pgfusepath{fill,stroke}
\pgfpathmoveto{\pgfpoint{432.401001pt}{93.895973pt}}
\pgflineto{\pgfpoint{428.473938pt}{96.749146pt}}
\pgflineto{\pgfpoint{426.973938pt}{95.659332pt}}
\pgfpathclose
\pgfusepath{fill,stroke}
\pgfpathmoveto{\pgfpoint{432.401001pt}{93.895973pt}}
\pgflineto{\pgfpoint{430.328033pt}{96.749146pt}}
\pgflineto{\pgfpoint{428.473938pt}{96.749146pt}}
\pgfpathclose
\pgfusepath{fill,stroke}
\pgfpathmoveto{\pgfpoint{432.401001pt}{93.895973pt}}
\pgflineto{\pgfpoint{431.828033pt}{95.659332pt}}
\pgflineto{\pgfpoint{430.328033pt}{96.749146pt}}
\pgfpathclose
\pgfusepath{fill,stroke}
\pgfpathmoveto{\pgfpoint{393.367279pt}{143.257599pt}}
\pgflineto{\pgfpoint{391.294312pt}{140.404419pt}}
\pgflineto{\pgfpoint{392.794312pt}{141.494232pt}}
\pgfpathclose
\pgfusepath{fill,stroke}
\pgfpathmoveto{\pgfpoint{393.367279pt}{143.257599pt}}
\pgflineto{\pgfpoint{389.440216pt}{140.404419pt}}
\pgflineto{\pgfpoint{391.294312pt}{140.404419pt}}
\pgfpathclose
\pgfusepath{fill,stroke}
\pgfpathmoveto{\pgfpoint{393.367279pt}{143.257599pt}}
\pgflineto{\pgfpoint{387.940216pt}{141.494232pt}}
\pgflineto{\pgfpoint{389.440216pt}{140.404419pt}}
\pgfpathclose
\pgfusepath{fill,stroke}
\pgfpathmoveto{\pgfpoint{393.367279pt}{143.257599pt}}
\pgflineto{\pgfpoint{387.367279pt}{143.257599pt}}
\pgflineto{\pgfpoint{387.940216pt}{141.494232pt}}
\pgfpathclose
\pgfusepath{fill,stroke}
\pgfpathmoveto{\pgfpoint{393.367279pt}{143.257599pt}}
\pgflineto{\pgfpoint{387.940216pt}{145.020950pt}}
\pgflineto{\pgfpoint{387.367279pt}{143.257599pt}}
\pgfpathclose
\pgfusepath{fill,stroke}
\pgfpathmoveto{\pgfpoint{393.367279pt}{143.257599pt}}
\pgflineto{\pgfpoint{389.440216pt}{146.110748pt}}
\pgflineto{\pgfpoint{387.940216pt}{145.020950pt}}
\pgfpathclose
\pgfusepath{fill,stroke}
\pgfpathmoveto{\pgfpoint{393.367279pt}{143.257599pt}}
\pgflineto{\pgfpoint{391.294312pt}{146.110748pt}}
\pgflineto{\pgfpoint{389.440216pt}{146.110748pt}}
\pgfpathclose
\pgfusepath{fill,stroke}
\pgfpathmoveto{\pgfpoint{393.367279pt}{143.257599pt}}
\pgflineto{\pgfpoint{392.794312pt}{145.020950pt}}
\pgflineto{\pgfpoint{391.294312pt}{146.110748pt}}
\pgfpathclose
\pgfusepath{fill,stroke}
\pgfpathmoveto{\pgfpoint{394.942413pt}{150.918854pt}}
\pgflineto{\pgfpoint{392.869446pt}{148.065689pt}}
\pgflineto{\pgfpoint{394.369446pt}{149.155487pt}}
\pgfpathclose
\pgfusepath{fill,stroke}
\pgfpathmoveto{\pgfpoint{394.942413pt}{150.918854pt}}
\pgflineto{\pgfpoint{391.015350pt}{148.065689pt}}
\pgflineto{\pgfpoint{392.869446pt}{148.065689pt}}
\pgfpathclose
\pgfusepath{fill,stroke}
\pgfpathmoveto{\pgfpoint{394.942413pt}{150.918854pt}}
\pgflineto{\pgfpoint{389.515350pt}{149.155487pt}}
\pgflineto{\pgfpoint{391.015350pt}{148.065689pt}}
\pgfpathclose
\pgfusepath{fill,stroke}
\pgfpathmoveto{\pgfpoint{394.942413pt}{150.918854pt}}
\pgflineto{\pgfpoint{388.942413pt}{150.918854pt}}
\pgflineto{\pgfpoint{389.515350pt}{149.155487pt}}
\pgfpathclose
\pgfusepath{fill,stroke}
\pgfpathmoveto{\pgfpoint{394.942413pt}{150.918854pt}}
\pgflineto{\pgfpoint{389.515350pt}{152.682220pt}}
\pgflineto{\pgfpoint{388.942413pt}{150.918854pt}}
\pgfpathclose
\pgfusepath{fill,stroke}
\pgfpathmoveto{\pgfpoint{394.942413pt}{150.918854pt}}
\pgflineto{\pgfpoint{391.015350pt}{153.772018pt}}
\pgflineto{\pgfpoint{389.515350pt}{152.682220pt}}
\pgfpathclose
\pgfusepath{fill,stroke}
\pgfpathmoveto{\pgfpoint{394.942413pt}{150.918854pt}}
\pgflineto{\pgfpoint{392.869446pt}{153.772018pt}}
\pgflineto{\pgfpoint{391.015350pt}{153.772018pt}}
\pgfpathclose
\pgfusepath{fill,stroke}
\pgfpathmoveto{\pgfpoint{394.942413pt}{150.918854pt}}
\pgflineto{\pgfpoint{394.369446pt}{152.682220pt}}
\pgflineto{\pgfpoint{392.869446pt}{153.772018pt}}
\pgfpathclose
\pgfusepath{fill,stroke}
\pgfpathmoveto{\pgfpoint{236.089294pt}{244.881958pt}}
\pgflineto{\pgfpoint{234.016342pt}{242.028793pt}}
\pgflineto{\pgfpoint{235.516327pt}{243.118607pt}}
\pgfpathclose
\pgfusepath{fill,stroke}
\pgfpathmoveto{\pgfpoint{236.089294pt}{244.881958pt}}
\pgflineto{\pgfpoint{232.162231pt}{242.028793pt}}
\pgflineto{\pgfpoint{234.016342pt}{242.028793pt}}
\pgfpathclose
\pgfusepath{fill,stroke}
\pgfpathmoveto{\pgfpoint{236.089294pt}{244.881958pt}}
\pgflineto{\pgfpoint{230.662231pt}{243.118607pt}}
\pgflineto{\pgfpoint{232.162231pt}{242.028793pt}}
\pgfpathclose
\pgfusepath{fill,stroke}
\pgfpathmoveto{\pgfpoint{236.089294pt}{244.881958pt}}
\pgflineto{\pgfpoint{230.089279pt}{244.881958pt}}
\pgflineto{\pgfpoint{230.662231pt}{243.118607pt}}
\pgfpathclose
\pgfusepath{fill,stroke}
\pgfpathmoveto{\pgfpoint{236.089294pt}{244.881958pt}}
\pgflineto{\pgfpoint{230.662231pt}{246.645309pt}}
\pgflineto{\pgfpoint{230.089279pt}{244.881958pt}}
\pgfpathclose
\pgfusepath{fill,stroke}
\pgfpathmoveto{\pgfpoint{236.089294pt}{244.881958pt}}
\pgflineto{\pgfpoint{232.162231pt}{247.735123pt}}
\pgflineto{\pgfpoint{230.662231pt}{246.645309pt}}
\pgfpathclose
\pgfusepath{fill,stroke}
\pgfpathmoveto{\pgfpoint{236.089294pt}{244.881958pt}}
\pgflineto{\pgfpoint{234.016342pt}{247.735123pt}}
\pgflineto{\pgfpoint{232.162231pt}{247.735123pt}}
\pgfpathclose
\pgfusepath{fill,stroke}
\pgfpathmoveto{\pgfpoint{236.089294pt}{244.881958pt}}
\pgflineto{\pgfpoint{235.516327pt}{246.645309pt}}
\pgflineto{\pgfpoint{234.016342pt}{247.735123pt}}
\pgfpathclose
\pgfusepath{fill,stroke}
\pgfpathmoveto{\pgfpoint{477.119995pt}{344.550873pt}}
\pgflineto{\pgfpoint{475.047058pt}{341.697723pt}}
\pgflineto{\pgfpoint{476.547058pt}{342.787537pt}}
\pgfpathclose
\pgfusepath{fill,stroke}
\pgfpathmoveto{\pgfpoint{477.119995pt}{344.550873pt}}
\pgflineto{\pgfpoint{473.192932pt}{341.697723pt}}
\pgflineto{\pgfpoint{475.047058pt}{341.697723pt}}
\pgfpathclose
\pgfusepath{fill,stroke}
\pgfpathmoveto{\pgfpoint{477.119995pt}{344.550873pt}}
\pgflineto{\pgfpoint{471.692932pt}{342.787537pt}}
\pgflineto{\pgfpoint{473.192932pt}{341.697723pt}}
\pgfpathclose
\pgfusepath{fill,stroke}
\pgfpathmoveto{\pgfpoint{477.119995pt}{344.550873pt}}
\pgflineto{\pgfpoint{471.119995pt}{344.550873pt}}
\pgflineto{\pgfpoint{471.692932pt}{342.787537pt}}
\pgfpathclose
\pgfusepath{fill,stroke}
\pgfpathmoveto{\pgfpoint{477.119995pt}{344.550873pt}}
\pgflineto{\pgfpoint{471.692932pt}{346.314240pt}}
\pgflineto{\pgfpoint{471.119995pt}{344.550873pt}}
\pgfpathclose
\pgfusepath{fill,stroke}
\pgfpathmoveto{\pgfpoint{477.119995pt}{344.550873pt}}
\pgflineto{\pgfpoint{473.192932pt}{347.404053pt}}
\pgflineto{\pgfpoint{471.692932pt}{346.314240pt}}
\pgfpathclose
\pgfusepath{fill,stroke}
\pgfpathmoveto{\pgfpoint{477.119995pt}{344.550873pt}}
\pgflineto{\pgfpoint{475.047058pt}{347.404053pt}}
\pgflineto{\pgfpoint{473.192932pt}{347.404053pt}}
\pgfpathclose
\pgfusepath{fill,stroke}
\pgfpathmoveto{\pgfpoint{477.119995pt}{344.550873pt}}
\pgflineto{\pgfpoint{476.547058pt}{346.314240pt}}
\pgflineto{\pgfpoint{475.047058pt}{347.404053pt}}
\pgfpathclose
\pgfusepath{fill,stroke}
\pgfpathmoveto{\pgfpoint{257.495422pt}{209.244446pt}}
\pgflineto{\pgfpoint{255.422470pt}{206.391266pt}}
\pgflineto{\pgfpoint{256.922485pt}{207.481079pt}}
\pgfpathclose
\pgfusepath{fill,stroke}
\pgfpathmoveto{\pgfpoint{257.495422pt}{209.244446pt}}
\pgflineto{\pgfpoint{253.568375pt}{206.391266pt}}
\pgflineto{\pgfpoint{255.422470pt}{206.391266pt}}
\pgfpathclose
\pgfusepath{fill,stroke}
\pgfpathmoveto{\pgfpoint{257.495422pt}{209.244446pt}}
\pgflineto{\pgfpoint{252.068375pt}{207.481079pt}}
\pgflineto{\pgfpoint{253.568375pt}{206.391266pt}}
\pgfpathclose
\pgfusepath{fill,stroke}
\pgfpathmoveto{\pgfpoint{257.495422pt}{209.244446pt}}
\pgflineto{\pgfpoint{251.495422pt}{209.244446pt}}
\pgflineto{\pgfpoint{252.068375pt}{207.481079pt}}
\pgfpathclose
\pgfusepath{fill,stroke}
\pgfpathmoveto{\pgfpoint{257.495422pt}{209.244446pt}}
\pgflineto{\pgfpoint{252.068375pt}{211.007797pt}}
\pgflineto{\pgfpoint{251.495422pt}{209.244446pt}}
\pgfpathclose
\pgfusepath{fill,stroke}
\pgfpathmoveto{\pgfpoint{257.495422pt}{209.244446pt}}
\pgflineto{\pgfpoint{253.568375pt}{212.097610pt}}
\pgflineto{\pgfpoint{252.068375pt}{211.007797pt}}
\pgfpathclose
\pgfusepath{fill,stroke}
\pgfpathmoveto{\pgfpoint{257.495422pt}{209.244446pt}}
\pgflineto{\pgfpoint{255.422470pt}{212.097610pt}}
\pgflineto{\pgfpoint{253.568375pt}{212.097610pt}}
\pgfpathclose
\pgfusepath{fill,stroke}
\pgfpathmoveto{\pgfpoint{257.495422pt}{209.244446pt}}
\pgflineto{\pgfpoint{256.922485pt}{211.007797pt}}
\pgflineto{\pgfpoint{255.422470pt}{212.097610pt}}
\pgfpathclose
\pgfusepath{fill,stroke}
\pgfpathmoveto{\pgfpoint{352.095215pt}{171.712708pt}}
\pgflineto{\pgfpoint{350.022278pt}{168.859528pt}}
\pgflineto{\pgfpoint{351.522278pt}{169.949341pt}}
\pgfpathclose
\pgfusepath{fill,stroke}
\pgfpathmoveto{\pgfpoint{352.095215pt}{171.712708pt}}
\pgflineto{\pgfpoint{348.168152pt}{168.859528pt}}
\pgflineto{\pgfpoint{350.022278pt}{168.859528pt}}
\pgfpathclose
\pgfusepath{fill,stroke}
\pgfpathmoveto{\pgfpoint{352.095215pt}{171.712708pt}}
\pgflineto{\pgfpoint{346.668152pt}{169.949341pt}}
\pgflineto{\pgfpoint{348.168152pt}{168.859528pt}}
\pgfpathclose
\pgfusepath{fill,stroke}
\pgfpathmoveto{\pgfpoint{352.095215pt}{171.712708pt}}
\pgflineto{\pgfpoint{346.095215pt}{171.712708pt}}
\pgflineto{\pgfpoint{346.668152pt}{169.949341pt}}
\pgfpathclose
\pgfusepath{fill,stroke}
\pgfpathmoveto{\pgfpoint{352.095215pt}{171.712708pt}}
\pgflineto{\pgfpoint{346.668152pt}{173.476059pt}}
\pgflineto{\pgfpoint{346.095215pt}{171.712708pt}}
\pgfpathclose
\pgfusepath{fill,stroke}
\pgfpathmoveto{\pgfpoint{352.095215pt}{171.712708pt}}
\pgflineto{\pgfpoint{348.168152pt}{174.565872pt}}
\pgflineto{\pgfpoint{346.668152pt}{173.476059pt}}
\pgfpathclose
\pgfusepath{fill,stroke}
\pgfpathmoveto{\pgfpoint{352.095215pt}{171.712708pt}}
\pgflineto{\pgfpoint{350.022278pt}{174.565872pt}}
\pgflineto{\pgfpoint{348.168152pt}{174.565872pt}}
\pgfpathclose
\pgfusepath{fill,stroke}
\pgfpathmoveto{\pgfpoint{352.095215pt}{171.712708pt}}
\pgflineto{\pgfpoint{351.522278pt}{173.476059pt}}
\pgflineto{\pgfpoint{350.022278pt}{174.565872pt}}
\pgfpathclose
\pgfusepath{fill,stroke}
\pgfpathmoveto{\pgfpoint{353.962677pt}{180.838608pt}}
\pgflineto{\pgfpoint{351.889740pt}{177.985443pt}}
\pgflineto{\pgfpoint{353.389740pt}{179.075256pt}}
\pgfpathclose
\pgfusepath{fill,stroke}
\pgfpathmoveto{\pgfpoint{353.962677pt}{180.838608pt}}
\pgflineto{\pgfpoint{350.035645pt}{177.985443pt}}
\pgflineto{\pgfpoint{351.889740pt}{177.985443pt}}
\pgfpathclose
\pgfusepath{fill,stroke}
\pgfpathmoveto{\pgfpoint{353.962677pt}{180.838608pt}}
\pgflineto{\pgfpoint{348.535645pt}{179.075256pt}}
\pgflineto{\pgfpoint{350.035645pt}{177.985443pt}}
\pgfpathclose
\pgfusepath{fill,stroke}
\pgfpathmoveto{\pgfpoint{353.962677pt}{180.838608pt}}
\pgflineto{\pgfpoint{347.962677pt}{180.838608pt}}
\pgflineto{\pgfpoint{348.535645pt}{179.075256pt}}
\pgfpathclose
\pgfusepath{fill,stroke}
\pgfpathmoveto{\pgfpoint{353.962677pt}{180.838608pt}}
\pgflineto{\pgfpoint{348.535645pt}{182.601959pt}}
\pgflineto{\pgfpoint{347.962677pt}{180.838608pt}}
\pgfpathclose
\pgfusepath{fill,stroke}
\pgfpathmoveto{\pgfpoint{353.962677pt}{180.838608pt}}
\pgflineto{\pgfpoint{350.035645pt}{183.691772pt}}
\pgflineto{\pgfpoint{348.535645pt}{182.601959pt}}
\pgfpathclose
\pgfusepath{fill,stroke}
\pgfpathmoveto{\pgfpoint{353.962677pt}{180.838608pt}}
\pgflineto{\pgfpoint{351.889740pt}{183.691772pt}}
\pgflineto{\pgfpoint{350.035645pt}{183.691772pt}}
\pgfpathclose
\pgfusepath{fill,stroke}
\pgfpathmoveto{\pgfpoint{353.962677pt}{180.838608pt}}
\pgflineto{\pgfpoint{353.389740pt}{182.601959pt}}
\pgflineto{\pgfpoint{351.889740pt}{183.691772pt}}
\pgfpathclose
\pgfusepath{fill,stroke}
\pgfpathmoveto{\pgfpoint{231.464218pt}{235.868713pt}}
\pgflineto{\pgfpoint{229.391266pt}{233.015533pt}}
\pgflineto{\pgfpoint{230.891266pt}{234.105347pt}}
\pgfpathclose
\pgfusepath{fill,stroke}
\pgfpathmoveto{\pgfpoint{231.464218pt}{235.868713pt}}
\pgflineto{\pgfpoint{227.537155pt}{233.015533pt}}
\pgflineto{\pgfpoint{229.391266pt}{233.015533pt}}
\pgfpathclose
\pgfusepath{fill,stroke}
\pgfpathmoveto{\pgfpoint{231.464218pt}{235.868713pt}}
\pgflineto{\pgfpoint{226.037170pt}{234.105347pt}}
\pgflineto{\pgfpoint{227.537155pt}{233.015533pt}}
\pgfpathclose
\pgfusepath{fill,stroke}
\pgfpathmoveto{\pgfpoint{231.464218pt}{235.868713pt}}
\pgflineto{\pgfpoint{225.464203pt}{235.868713pt}}
\pgflineto{\pgfpoint{226.037170pt}{234.105347pt}}
\pgfpathclose
\pgfusepath{fill,stroke}
\pgfpathmoveto{\pgfpoint{231.464218pt}{235.868713pt}}
\pgflineto{\pgfpoint{226.037170pt}{237.632065pt}}
\pgflineto{\pgfpoint{225.464203pt}{235.868713pt}}
\pgfpathclose
\pgfusepath{fill,stroke}
\pgfpathmoveto{\pgfpoint{231.464218pt}{235.868713pt}}
\pgflineto{\pgfpoint{227.537155pt}{238.721878pt}}
\pgflineto{\pgfpoint{226.037170pt}{237.632065pt}}
\pgfpathclose
\pgfusepath{fill,stroke}
\pgfpathmoveto{\pgfpoint{231.464218pt}{235.868713pt}}
\pgflineto{\pgfpoint{229.391266pt}{238.721878pt}}
\pgflineto{\pgfpoint{227.537155pt}{238.721878pt}}
\pgfpathclose
\pgfusepath{fill,stroke}
\pgfpathmoveto{\pgfpoint{231.464218pt}{235.868713pt}}
\pgflineto{\pgfpoint{230.891266pt}{237.632065pt}}
\pgflineto{\pgfpoint{229.391266pt}{238.721878pt}}
\pgfpathclose
\pgfusepath{fill,stroke}
\pgfpathmoveto{\pgfpoint{258.970215pt}{230.538223pt}}
\pgflineto{\pgfpoint{256.897247pt}{227.685059pt}}
\pgflineto{\pgfpoint{258.397247pt}{228.774872pt}}
\pgfpathclose
\pgfusepath{fill,stroke}
\pgfpathmoveto{\pgfpoint{258.970215pt}{230.538223pt}}
\pgflineto{\pgfpoint{255.043152pt}{227.685059pt}}
\pgflineto{\pgfpoint{256.897247pt}{227.685059pt}}
\pgfpathclose
\pgfusepath{fill,stroke}
\pgfpathmoveto{\pgfpoint{258.970215pt}{230.538223pt}}
\pgflineto{\pgfpoint{253.543152pt}{228.774872pt}}
\pgflineto{\pgfpoint{255.043152pt}{227.685059pt}}
\pgfpathclose
\pgfusepath{fill,stroke}
\pgfpathmoveto{\pgfpoint{258.970215pt}{230.538223pt}}
\pgflineto{\pgfpoint{252.970200pt}{230.538223pt}}
\pgflineto{\pgfpoint{253.543152pt}{228.774872pt}}
\pgfpathclose
\pgfusepath{fill,stroke}
\pgfpathmoveto{\pgfpoint{258.970215pt}{230.538223pt}}
\pgflineto{\pgfpoint{253.543152pt}{232.301575pt}}
\pgflineto{\pgfpoint{252.970200pt}{230.538223pt}}
\pgfpathclose
\pgfusepath{fill,stroke}
\pgfpathmoveto{\pgfpoint{258.970215pt}{230.538223pt}}
\pgflineto{\pgfpoint{255.043152pt}{233.391388pt}}
\pgflineto{\pgfpoint{253.543152pt}{232.301575pt}}
\pgfpathclose
\pgfusepath{fill,stroke}
\pgfpathmoveto{\pgfpoint{258.970215pt}{230.538223pt}}
\pgflineto{\pgfpoint{256.897247pt}{233.391388pt}}
\pgflineto{\pgfpoint{255.043152pt}{233.391388pt}}
\pgfpathclose
\pgfusepath{fill,stroke}
\pgfpathmoveto{\pgfpoint{258.970215pt}{230.538223pt}}
\pgflineto{\pgfpoint{258.397247pt}{232.301575pt}}
\pgflineto{\pgfpoint{256.897247pt}{233.391388pt}}
\pgfpathclose
\pgfusepath{fill,stroke}
\pgfpathmoveto{\pgfpoint{197.129730pt}{273.138519pt}}
\pgflineto{\pgfpoint{195.056793pt}{270.285339pt}}
\pgflineto{\pgfpoint{196.556793pt}{271.375153pt}}
\pgfpathclose
\pgfusepath{fill,stroke}
\pgfpathmoveto{\pgfpoint{197.129730pt}{273.138519pt}}
\pgflineto{\pgfpoint{193.202698pt}{270.285339pt}}
\pgflineto{\pgfpoint{195.056793pt}{270.285339pt}}
\pgfpathclose
\pgfusepath{fill,stroke}
\pgfpathmoveto{\pgfpoint{197.129730pt}{273.138519pt}}
\pgflineto{\pgfpoint{191.702698pt}{271.375153pt}}
\pgflineto{\pgfpoint{193.202698pt}{270.285339pt}}
\pgfpathclose
\pgfusepath{fill,stroke}
\pgfpathmoveto{\pgfpoint{197.129730pt}{273.138519pt}}
\pgflineto{\pgfpoint{191.129745pt}{273.138519pt}}
\pgflineto{\pgfpoint{191.702698pt}{271.375153pt}}
\pgfpathclose
\pgfusepath{fill,stroke}
\pgfpathmoveto{\pgfpoint{197.129730pt}{273.138519pt}}
\pgflineto{\pgfpoint{191.702698pt}{274.901855pt}}
\pgflineto{\pgfpoint{191.129745pt}{273.138519pt}}
\pgfpathclose
\pgfusepath{fill,stroke}
\pgfpathmoveto{\pgfpoint{197.129730pt}{273.138519pt}}
\pgflineto{\pgfpoint{193.202698pt}{275.991669pt}}
\pgflineto{\pgfpoint{191.702698pt}{274.901855pt}}
\pgfpathclose
\pgfusepath{fill,stroke}
\pgfpathmoveto{\pgfpoint{197.129730pt}{273.138519pt}}
\pgflineto{\pgfpoint{195.056793pt}{275.991669pt}}
\pgflineto{\pgfpoint{193.202698pt}{275.991669pt}}
\pgfpathclose
\pgfusepath{fill,stroke}
\pgfpathmoveto{\pgfpoint{197.129730pt}{273.138519pt}}
\pgflineto{\pgfpoint{196.556793pt}{274.901855pt}}
\pgflineto{\pgfpoint{195.056793pt}{275.991669pt}}
\pgfpathclose
\pgfusepath{fill,stroke}
\pgfpathmoveto{\pgfpoint{384.086609pt}{197.034302pt}}
\pgflineto{\pgfpoint{382.013672pt}{194.181122pt}}
\pgflineto{\pgfpoint{383.513672pt}{195.270935pt}}
\pgfpathclose
\pgfusepath{fill,stroke}
\pgfpathmoveto{\pgfpoint{384.086609pt}{197.034302pt}}
\pgflineto{\pgfpoint{380.159546pt}{194.181122pt}}
\pgflineto{\pgfpoint{382.013672pt}{194.181122pt}}
\pgfpathclose
\pgfusepath{fill,stroke}
\pgfpathmoveto{\pgfpoint{384.086609pt}{197.034302pt}}
\pgflineto{\pgfpoint{378.659546pt}{195.270935pt}}
\pgflineto{\pgfpoint{380.159546pt}{194.181122pt}}
\pgfpathclose
\pgfusepath{fill,stroke}
\pgfpathmoveto{\pgfpoint{384.086609pt}{197.034302pt}}
\pgflineto{\pgfpoint{378.086609pt}{197.034302pt}}
\pgflineto{\pgfpoint{378.659546pt}{195.270935pt}}
\pgfpathclose
\pgfusepath{fill,stroke}
\pgfpathmoveto{\pgfpoint{384.086609pt}{197.034302pt}}
\pgflineto{\pgfpoint{378.659546pt}{198.797653pt}}
\pgflineto{\pgfpoint{378.086609pt}{197.034302pt}}
\pgfpathclose
\pgfusepath{fill,stroke}
\pgfpathmoveto{\pgfpoint{384.086609pt}{197.034302pt}}
\pgflineto{\pgfpoint{380.159546pt}{199.887466pt}}
\pgflineto{\pgfpoint{378.659546pt}{198.797653pt}}
\pgfpathclose
\pgfusepath{fill,stroke}
\pgfpathmoveto{\pgfpoint{384.086609pt}{197.034302pt}}
\pgflineto{\pgfpoint{382.013672pt}{199.887466pt}}
\pgflineto{\pgfpoint{380.159546pt}{199.887466pt}}
\pgfpathclose
\pgfusepath{fill,stroke}
\pgfpathmoveto{\pgfpoint{384.086609pt}{197.034302pt}}
\pgflineto{\pgfpoint{383.513672pt}{198.797653pt}}
\pgflineto{\pgfpoint{382.013672pt}{199.887466pt}}
\pgfpathclose
\pgfusepath{fill,stroke}
\pgfpathmoveto{\pgfpoint{225.839981pt}{243.910217pt}}
\pgflineto{\pgfpoint{223.767029pt}{241.057053pt}}
\pgflineto{\pgfpoint{225.267029pt}{242.146866pt}}
\pgfpathclose
\pgfusepath{fill,stroke}
\pgfpathmoveto{\pgfpoint{225.839981pt}{243.910217pt}}
\pgflineto{\pgfpoint{221.912933pt}{241.057053pt}}
\pgflineto{\pgfpoint{223.767029pt}{241.057053pt}}
\pgfpathclose
\pgfusepath{fill,stroke}
\pgfpathmoveto{\pgfpoint{225.839981pt}{243.910217pt}}
\pgflineto{\pgfpoint{220.412933pt}{242.146866pt}}
\pgflineto{\pgfpoint{221.912933pt}{241.057053pt}}
\pgfpathclose
\pgfusepath{fill,stroke}
\pgfpathmoveto{\pgfpoint{225.839981pt}{243.910217pt}}
\pgflineto{\pgfpoint{219.839981pt}{243.910217pt}}
\pgflineto{\pgfpoint{220.412933pt}{242.146866pt}}
\pgfpathclose
\pgfusepath{fill,stroke}
\pgfpathmoveto{\pgfpoint{225.839981pt}{243.910217pt}}
\pgflineto{\pgfpoint{220.412933pt}{245.673584pt}}
\pgflineto{\pgfpoint{219.839981pt}{243.910217pt}}
\pgfpathclose
\pgfusepath{fill,stroke}
\pgfpathmoveto{\pgfpoint{225.839981pt}{243.910217pt}}
\pgflineto{\pgfpoint{221.912933pt}{246.763397pt}}
\pgflineto{\pgfpoint{220.412933pt}{245.673584pt}}
\pgfpathclose
\pgfusepath{fill,stroke}
\pgfpathmoveto{\pgfpoint{225.839981pt}{243.910217pt}}
\pgflineto{\pgfpoint{223.767029pt}{246.763397pt}}
\pgflineto{\pgfpoint{221.912933pt}{246.763397pt}}
\pgfpathclose
\pgfusepath{fill,stroke}
\pgfpathmoveto{\pgfpoint{225.839981pt}{243.910217pt}}
\pgflineto{\pgfpoint{225.267029pt}{245.673584pt}}
\pgflineto{\pgfpoint{223.767029pt}{246.763397pt}}
\pgfpathclose
\pgfusepath{fill,stroke}
\pgfpathmoveto{\pgfpoint{201.737350pt}{271.245026pt}}
\pgflineto{\pgfpoint{199.664398pt}{268.391846pt}}
\pgflineto{\pgfpoint{201.164398pt}{269.481659pt}}
\pgfpathclose
\pgfusepath{fill,stroke}
\pgfpathmoveto{\pgfpoint{201.737350pt}{271.245026pt}}
\pgflineto{\pgfpoint{197.810303pt}{268.391846pt}}
\pgflineto{\pgfpoint{199.664398pt}{268.391846pt}}
\pgfpathclose
\pgfusepath{fill,stroke}
\pgfpathmoveto{\pgfpoint{201.737350pt}{271.245026pt}}
\pgflineto{\pgfpoint{196.310303pt}{269.481659pt}}
\pgflineto{\pgfpoint{197.810303pt}{268.391846pt}}
\pgfpathclose
\pgfusepath{fill,stroke}
\pgfpathmoveto{\pgfpoint{201.737350pt}{271.245026pt}}
\pgflineto{\pgfpoint{195.737366pt}{271.245026pt}}
\pgflineto{\pgfpoint{196.310303pt}{269.481659pt}}
\pgfpathclose
\pgfusepath{fill,stroke}
\pgfpathmoveto{\pgfpoint{201.737350pt}{271.245026pt}}
\pgflineto{\pgfpoint{196.310303pt}{273.008362pt}}
\pgflineto{\pgfpoint{195.737366pt}{271.245026pt}}
\pgfpathclose
\pgfusepath{fill,stroke}
\pgfpathmoveto{\pgfpoint{201.737350pt}{271.245026pt}}
\pgflineto{\pgfpoint{197.810303pt}{274.098175pt}}
\pgflineto{\pgfpoint{196.310303pt}{273.008362pt}}
\pgfpathclose
\pgfusepath{fill,stroke}
\pgfpathmoveto{\pgfpoint{201.737350pt}{271.245026pt}}
\pgflineto{\pgfpoint{199.664398pt}{274.098175pt}}
\pgflineto{\pgfpoint{197.810303pt}{274.098175pt}}
\pgfpathclose
\pgfusepath{fill,stroke}
\pgfpathmoveto{\pgfpoint{201.737350pt}{271.245026pt}}
\pgflineto{\pgfpoint{201.164398pt}{273.008362pt}}
\pgflineto{\pgfpoint{199.664398pt}{274.098175pt}}
\pgfpathclose
\pgfusepath{fill,stroke}
\pgfpathmoveto{\pgfpoint{194.062378pt}{283.754425pt}}
\pgflineto{\pgfpoint{191.989426pt}{280.901245pt}}
\pgflineto{\pgfpoint{193.489426pt}{281.991058pt}}
\pgfpathclose
\pgfusepath{fill,stroke}
\pgfpathmoveto{\pgfpoint{194.062378pt}{283.754425pt}}
\pgflineto{\pgfpoint{190.135315pt}{280.901245pt}}
\pgflineto{\pgfpoint{191.989426pt}{280.901245pt}}
\pgfpathclose
\pgfusepath{fill,stroke}
\pgfpathmoveto{\pgfpoint{194.062378pt}{283.754425pt}}
\pgflineto{\pgfpoint{188.635315pt}{281.991058pt}}
\pgflineto{\pgfpoint{190.135315pt}{280.901245pt}}
\pgfpathclose
\pgfusepath{fill,stroke}
\pgfpathmoveto{\pgfpoint{194.062378pt}{283.754425pt}}
\pgflineto{\pgfpoint{188.062378pt}{283.754425pt}}
\pgflineto{\pgfpoint{188.635315pt}{281.991058pt}}
\pgfpathclose
\pgfusepath{fill,stroke}
\pgfpathmoveto{\pgfpoint{194.062378pt}{283.754425pt}}
\pgflineto{\pgfpoint{188.635315pt}{285.517761pt}}
\pgflineto{\pgfpoint{188.062378pt}{283.754425pt}}
\pgfpathclose
\pgfusepath{fill,stroke}
\pgfpathmoveto{\pgfpoint{194.062378pt}{283.754425pt}}
\pgflineto{\pgfpoint{190.135315pt}{286.607574pt}}
\pgflineto{\pgfpoint{188.635315pt}{285.517761pt}}
\pgfpathclose
\pgfusepath{fill,stroke}
\pgfpathmoveto{\pgfpoint{194.062378pt}{283.754425pt}}
\pgflineto{\pgfpoint{191.989426pt}{286.607574pt}}
\pgflineto{\pgfpoint{190.135315pt}{286.607574pt}}
\pgfpathclose
\pgfusepath{fill,stroke}
\pgfpathmoveto{\pgfpoint{194.062378pt}{283.754425pt}}
\pgflineto{\pgfpoint{193.489426pt}{285.517761pt}}
\pgflineto{\pgfpoint{191.989426pt}{286.607574pt}}
\pgfpathclose
\pgfusepath{fill,stroke}
\pgfpathmoveto{\pgfpoint{214.617676pt}{249.346344pt}}
\pgflineto{\pgfpoint{212.544724pt}{246.493179pt}}
\pgflineto{\pgfpoint{214.044724pt}{247.582977pt}}
\pgfpathclose
\pgfusepath{fill,stroke}
\pgfpathmoveto{\pgfpoint{214.617676pt}{249.346344pt}}
\pgflineto{\pgfpoint{210.690628pt}{246.493179pt}}
\pgflineto{\pgfpoint{212.544724pt}{246.493179pt}}
\pgfpathclose
\pgfusepath{fill,stroke}
\pgfpathmoveto{\pgfpoint{214.617676pt}{249.346344pt}}
\pgflineto{\pgfpoint{209.190613pt}{247.582977pt}}
\pgflineto{\pgfpoint{210.690628pt}{246.493179pt}}
\pgfpathclose
\pgfusepath{fill,stroke}
\pgfpathmoveto{\pgfpoint{214.617676pt}{249.346344pt}}
\pgflineto{\pgfpoint{208.617676pt}{249.346344pt}}
\pgflineto{\pgfpoint{209.190613pt}{247.582977pt}}
\pgfpathclose
\pgfusepath{fill,stroke}
\pgfpathmoveto{\pgfpoint{214.617676pt}{249.346344pt}}
\pgflineto{\pgfpoint{209.190613pt}{251.109695pt}}
\pgflineto{\pgfpoint{208.617676pt}{249.346344pt}}
\pgfpathclose
\pgfusepath{fill,stroke}
\pgfpathmoveto{\pgfpoint{214.617676pt}{249.346344pt}}
\pgflineto{\pgfpoint{210.690628pt}{252.199509pt}}
\pgflineto{\pgfpoint{209.190613pt}{251.109695pt}}
\pgfpathclose
\pgfusepath{fill,stroke}
\pgfpathmoveto{\pgfpoint{214.617676pt}{249.346344pt}}
\pgflineto{\pgfpoint{212.544724pt}{252.199509pt}}
\pgflineto{\pgfpoint{210.690628pt}{252.199509pt}}
\pgfpathclose
\pgfusepath{fill,stroke}
\pgfpathmoveto{\pgfpoint{214.617676pt}{249.346344pt}}
\pgflineto{\pgfpoint{214.044724pt}{251.109695pt}}
\pgflineto{\pgfpoint{212.544724pt}{252.199509pt}}
\pgfpathclose
\pgfusepath{fill,stroke}
\pgfpathmoveto{\pgfpoint{221.664337pt}{247.670441pt}}
\pgflineto{\pgfpoint{219.591400pt}{244.817276pt}}
\pgflineto{\pgfpoint{221.091400pt}{245.907089pt}}
\pgfpathclose
\pgfusepath{fill,stroke}
\pgfpathmoveto{\pgfpoint{221.664337pt}{247.670441pt}}
\pgflineto{\pgfpoint{217.737289pt}{244.817276pt}}
\pgflineto{\pgfpoint{219.591400pt}{244.817276pt}}
\pgfpathclose
\pgfusepath{fill,stroke}
\pgfpathmoveto{\pgfpoint{221.664337pt}{247.670441pt}}
\pgflineto{\pgfpoint{216.237289pt}{245.907089pt}}
\pgflineto{\pgfpoint{217.737289pt}{244.817276pt}}
\pgfpathclose
\pgfusepath{fill,stroke}
\pgfpathmoveto{\pgfpoint{221.664337pt}{247.670441pt}}
\pgflineto{\pgfpoint{215.664337pt}{247.670441pt}}
\pgflineto{\pgfpoint{216.237289pt}{245.907089pt}}
\pgfpathclose
\pgfusepath{fill,stroke}
\pgfpathmoveto{\pgfpoint{221.664337pt}{247.670441pt}}
\pgflineto{\pgfpoint{216.237289pt}{249.433807pt}}
\pgflineto{\pgfpoint{215.664337pt}{247.670441pt}}
\pgfpathclose
\pgfusepath{fill,stroke}
\pgfpathmoveto{\pgfpoint{221.664337pt}{247.670441pt}}
\pgflineto{\pgfpoint{217.737289pt}{250.523621pt}}
\pgflineto{\pgfpoint{216.237289pt}{249.433807pt}}
\pgfpathclose
\pgfusepath{fill,stroke}
\pgfpathmoveto{\pgfpoint{221.664337pt}{247.670441pt}}
\pgflineto{\pgfpoint{219.591400pt}{250.523621pt}}
\pgflineto{\pgfpoint{217.737289pt}{250.523621pt}}
\pgfpathclose
\pgfusepath{fill,stroke}
\pgfpathmoveto{\pgfpoint{221.664337pt}{247.670441pt}}
\pgflineto{\pgfpoint{221.091400pt}{249.433807pt}}
\pgflineto{\pgfpoint{219.591400pt}{250.523621pt}}
\pgfpathclose
\pgfusepath{fill,stroke}
\pgfpathmoveto{\pgfpoint{207.252502pt}{262.388092pt}}
\pgflineto{\pgfpoint{205.179565pt}{259.534912pt}}
\pgflineto{\pgfpoint{206.679565pt}{260.624756pt}}
\pgfpathclose
\pgfusepath{fill,stroke}
\pgfpathmoveto{\pgfpoint{207.252502pt}{262.388092pt}}
\pgflineto{\pgfpoint{203.325470pt}{259.534912pt}}
\pgflineto{\pgfpoint{205.179565pt}{259.534912pt}}
\pgfpathclose
\pgfusepath{fill,stroke}
\pgfpathmoveto{\pgfpoint{207.252502pt}{262.388092pt}}
\pgflineto{\pgfpoint{201.825455pt}{260.624756pt}}
\pgflineto{\pgfpoint{203.325470pt}{259.534912pt}}
\pgfpathclose
\pgfusepath{fill,stroke}
\pgfpathmoveto{\pgfpoint{207.252502pt}{262.388092pt}}
\pgflineto{\pgfpoint{201.252502pt}{262.388092pt}}
\pgflineto{\pgfpoint{201.825455pt}{260.624756pt}}
\pgfpathclose
\pgfusepath{fill,stroke}
\pgfpathmoveto{\pgfpoint{207.252502pt}{262.388092pt}}
\pgflineto{\pgfpoint{201.825455pt}{264.151459pt}}
\pgflineto{\pgfpoint{201.252502pt}{262.388092pt}}
\pgfpathclose
\pgfusepath{fill,stroke}
\pgfpathmoveto{\pgfpoint{207.252502pt}{262.388092pt}}
\pgflineto{\pgfpoint{203.325470pt}{265.241272pt}}
\pgflineto{\pgfpoint{201.825455pt}{264.151459pt}}
\pgfpathclose
\pgfusepath{fill,stroke}
\pgfpathmoveto{\pgfpoint{207.252502pt}{262.388092pt}}
\pgflineto{\pgfpoint{205.179565pt}{265.241272pt}}
\pgflineto{\pgfpoint{203.325470pt}{265.241272pt}}
\pgfpathclose
\pgfusepath{fill,stroke}
\pgfpathmoveto{\pgfpoint{207.252502pt}{262.388092pt}}
\pgflineto{\pgfpoint{206.679565pt}{264.151459pt}}
\pgflineto{\pgfpoint{205.179565pt}{265.241272pt}}
\pgfpathclose
\pgfusepath{fill,stroke}
\pgfpathmoveto{\pgfpoint{216.258270pt}{246.860657pt}}
\pgflineto{\pgfpoint{214.185333pt}{244.007492pt}}
\pgflineto{\pgfpoint{215.685333pt}{245.097290pt}}
\pgfpathclose
\pgfusepath{fill,stroke}
\pgfpathmoveto{\pgfpoint{216.258270pt}{246.860657pt}}
\pgflineto{\pgfpoint{212.331238pt}{244.007492pt}}
\pgflineto{\pgfpoint{214.185333pt}{244.007492pt}}
\pgfpathclose
\pgfusepath{fill,stroke}
\pgfpathmoveto{\pgfpoint{216.258270pt}{246.860657pt}}
\pgflineto{\pgfpoint{210.831223pt}{245.097290pt}}
\pgflineto{\pgfpoint{212.331238pt}{244.007492pt}}
\pgfpathclose
\pgfusepath{fill,stroke}
\pgfpathmoveto{\pgfpoint{216.258270pt}{246.860657pt}}
\pgflineto{\pgfpoint{210.258270pt}{246.860657pt}}
\pgflineto{\pgfpoint{210.831223pt}{245.097290pt}}
\pgfpathclose
\pgfusepath{fill,stroke}
\pgfpathmoveto{\pgfpoint{216.258270pt}{246.860657pt}}
\pgflineto{\pgfpoint{210.831223pt}{248.624008pt}}
\pgflineto{\pgfpoint{210.258270pt}{246.860657pt}}
\pgfpathclose
\pgfusepath{fill,stroke}
\pgfpathmoveto{\pgfpoint{216.258270pt}{246.860657pt}}
\pgflineto{\pgfpoint{212.331238pt}{249.713821pt}}
\pgflineto{\pgfpoint{210.831223pt}{248.624008pt}}
\pgfpathclose
\pgfusepath{fill,stroke}
\pgfpathmoveto{\pgfpoint{216.258270pt}{246.860657pt}}
\pgflineto{\pgfpoint{214.185333pt}{249.713821pt}}
\pgflineto{\pgfpoint{212.331238pt}{249.713821pt}}
\pgfpathclose
\pgfusepath{fill,stroke}
\pgfpathmoveto{\pgfpoint{216.258270pt}{246.860657pt}}
\pgflineto{\pgfpoint{215.685333pt}{248.624008pt}}
\pgflineto{\pgfpoint{214.185333pt}{249.713821pt}}
\pgfpathclose
\pgfusepath{fill,stroke}
\pgfpathmoveto{\pgfpoint{235.622406pt}{234.798401pt}}
\pgflineto{\pgfpoint{233.549454pt}{231.945236pt}}
\pgflineto{\pgfpoint{235.049454pt}{233.035049pt}}
\pgfpathclose
\pgfusepath{fill,stroke}
\pgfpathmoveto{\pgfpoint{235.622406pt}{234.798401pt}}
\pgflineto{\pgfpoint{231.695343pt}{231.945236pt}}
\pgflineto{\pgfpoint{233.549454pt}{231.945236pt}}
\pgfpathclose
\pgfusepath{fill,stroke}
\pgfpathmoveto{\pgfpoint{235.622406pt}{234.798401pt}}
\pgflineto{\pgfpoint{230.195343pt}{233.035049pt}}
\pgflineto{\pgfpoint{231.695343pt}{231.945236pt}}
\pgfpathclose
\pgfusepath{fill,stroke}
\pgfpathmoveto{\pgfpoint{235.622406pt}{234.798401pt}}
\pgflineto{\pgfpoint{229.622406pt}{234.798401pt}}
\pgflineto{\pgfpoint{230.195343pt}{233.035049pt}}
\pgfpathclose
\pgfusepath{fill,stroke}
\pgfpathmoveto{\pgfpoint{235.622406pt}{234.798401pt}}
\pgflineto{\pgfpoint{230.195343pt}{236.561752pt}}
\pgflineto{\pgfpoint{229.622406pt}{234.798401pt}}
\pgfpathclose
\pgfusepath{fill,stroke}
\pgfpathmoveto{\pgfpoint{235.622406pt}{234.798401pt}}
\pgflineto{\pgfpoint{231.695343pt}{237.651566pt}}
\pgflineto{\pgfpoint{230.195343pt}{236.561752pt}}
\pgfpathclose
\pgfusepath{fill,stroke}
\pgfpathmoveto{\pgfpoint{235.622406pt}{234.798401pt}}
\pgflineto{\pgfpoint{233.549454pt}{237.651566pt}}
\pgflineto{\pgfpoint{231.695343pt}{237.651566pt}}
\pgfpathclose
\pgfusepath{fill,stroke}
\pgfpathmoveto{\pgfpoint{235.622406pt}{234.798401pt}}
\pgflineto{\pgfpoint{235.049454pt}{236.561752pt}}
\pgflineto{\pgfpoint{233.549454pt}{237.651566pt}}
\pgfpathclose
\pgfusepath{fill,stroke}
\pgfpathmoveto{\pgfpoint{208.766556pt}{250.712402pt}}
\pgflineto{\pgfpoint{206.693604pt}{247.859238pt}}
\pgflineto{\pgfpoint{208.193604pt}{248.949036pt}}
\pgfpathclose
\pgfusepath{fill,stroke}
\pgfpathmoveto{\pgfpoint{208.766556pt}{250.712402pt}}
\pgflineto{\pgfpoint{204.839508pt}{247.859238pt}}
\pgflineto{\pgfpoint{206.693604pt}{247.859238pt}}
\pgfpathclose
\pgfusepath{fill,stroke}
\pgfpathmoveto{\pgfpoint{208.766556pt}{250.712402pt}}
\pgflineto{\pgfpoint{203.339508pt}{248.949036pt}}
\pgflineto{\pgfpoint{204.839508pt}{247.859238pt}}
\pgfpathclose
\pgfusepath{fill,stroke}
\pgfpathmoveto{\pgfpoint{208.766556pt}{250.712402pt}}
\pgflineto{\pgfpoint{202.766571pt}{250.712402pt}}
\pgflineto{\pgfpoint{203.339508pt}{248.949036pt}}
\pgfpathclose
\pgfusepath{fill,stroke}
\pgfpathmoveto{\pgfpoint{208.766556pt}{250.712402pt}}
\pgflineto{\pgfpoint{203.339508pt}{252.475769pt}}
\pgflineto{\pgfpoint{202.766571pt}{250.712402pt}}
\pgfpathclose
\pgfusepath{fill,stroke}
\pgfpathmoveto{\pgfpoint{208.766556pt}{250.712402pt}}
\pgflineto{\pgfpoint{204.839508pt}{253.565567pt}}
\pgflineto{\pgfpoint{203.339508pt}{252.475769pt}}
\pgfpathclose
\pgfusepath{fill,stroke}
\pgfpathmoveto{\pgfpoint{208.766556pt}{250.712402pt}}
\pgflineto{\pgfpoint{206.693604pt}{253.565567pt}}
\pgflineto{\pgfpoint{204.839508pt}{253.565567pt}}
\pgfpathclose
\pgfusepath{fill,stroke}
\pgfpathmoveto{\pgfpoint{208.766556pt}{250.712402pt}}
\pgflineto{\pgfpoint{208.193604pt}{252.475769pt}}
\pgflineto{\pgfpoint{206.693604pt}{253.565567pt}}
\pgfpathclose
\pgfusepath{fill,stroke}
\pgfpathmoveto{\pgfpoint{218.950394pt}{226.221725pt}}
\pgflineto{\pgfpoint{216.877441pt}{223.368561pt}}
\pgflineto{\pgfpoint{218.377441pt}{224.458374pt}}
\pgfpathclose
\pgfusepath{fill,stroke}
\pgfpathmoveto{\pgfpoint{218.950394pt}{226.221725pt}}
\pgflineto{\pgfpoint{215.023346pt}{223.368561pt}}
\pgflineto{\pgfpoint{216.877441pt}{223.368561pt}}
\pgfpathclose
\pgfusepath{fill,stroke}
\pgfpathmoveto{\pgfpoint{218.950394pt}{226.221725pt}}
\pgflineto{\pgfpoint{213.523346pt}{224.458374pt}}
\pgflineto{\pgfpoint{215.023346pt}{223.368561pt}}
\pgfpathclose
\pgfusepath{fill,stroke}
\pgfpathmoveto{\pgfpoint{218.950394pt}{226.221725pt}}
\pgflineto{\pgfpoint{212.950394pt}{226.221725pt}}
\pgflineto{\pgfpoint{213.523346pt}{224.458374pt}}
\pgfpathclose
\pgfusepath{fill,stroke}
\pgfpathmoveto{\pgfpoint{218.950394pt}{226.221725pt}}
\pgflineto{\pgfpoint{213.523346pt}{227.985077pt}}
\pgflineto{\pgfpoint{212.950394pt}{226.221725pt}}
\pgfpathclose
\pgfusepath{fill,stroke}
\pgfpathmoveto{\pgfpoint{218.950394pt}{226.221725pt}}
\pgflineto{\pgfpoint{215.023346pt}{229.074890pt}}
\pgflineto{\pgfpoint{213.523346pt}{227.985077pt}}
\pgfpathclose
\pgfusepath{fill,stroke}
\pgfpathmoveto{\pgfpoint{218.950394pt}{226.221725pt}}
\pgflineto{\pgfpoint{216.877441pt}{229.074890pt}}
\pgflineto{\pgfpoint{215.023346pt}{229.074890pt}}
\pgfpathclose
\pgfusepath{fill,stroke}
\pgfpathmoveto{\pgfpoint{218.950394pt}{226.221725pt}}
\pgflineto{\pgfpoint{218.377441pt}{227.985077pt}}
\pgflineto{\pgfpoint{216.877441pt}{229.074890pt}}
\pgfpathclose
\pgfusepath{fill,stroke}
\pgfpathmoveto{\pgfpoint{197.282440pt}{288.294830pt}}
\pgflineto{\pgfpoint{195.209503pt}{285.441650pt}}
\pgflineto{\pgfpoint{196.709503pt}{286.531464pt}}
\pgfpathclose
\pgfusepath{fill,stroke}
\pgfpathmoveto{\pgfpoint{197.282440pt}{288.294830pt}}
\pgflineto{\pgfpoint{193.355408pt}{285.441650pt}}
\pgflineto{\pgfpoint{195.209503pt}{285.441650pt}}
\pgfpathclose
\pgfusepath{fill,stroke}
\pgfpathmoveto{\pgfpoint{197.282440pt}{288.294830pt}}
\pgflineto{\pgfpoint{191.855408pt}{286.531464pt}}
\pgflineto{\pgfpoint{193.355408pt}{285.441650pt}}
\pgfpathclose
\pgfusepath{fill,stroke}
\pgfpathmoveto{\pgfpoint{197.282440pt}{288.294830pt}}
\pgflineto{\pgfpoint{191.282455pt}{288.294830pt}}
\pgflineto{\pgfpoint{191.855408pt}{286.531464pt}}
\pgfpathclose
\pgfusepath{fill,stroke}
\pgfpathmoveto{\pgfpoint{197.282440pt}{288.294830pt}}
\pgflineto{\pgfpoint{191.855408pt}{290.058167pt}}
\pgflineto{\pgfpoint{191.282455pt}{288.294830pt}}
\pgfpathclose
\pgfusepath{fill,stroke}
\pgfpathmoveto{\pgfpoint{197.282440pt}{288.294830pt}}
\pgflineto{\pgfpoint{193.355408pt}{291.147980pt}}
\pgflineto{\pgfpoint{191.855408pt}{290.058167pt}}
\pgfpathclose
\pgfusepath{fill,stroke}
\pgfpathmoveto{\pgfpoint{197.282440pt}{288.294830pt}}
\pgflineto{\pgfpoint{195.209503pt}{291.147980pt}}
\pgflineto{\pgfpoint{193.355408pt}{291.147980pt}}
\pgfpathclose
\pgfusepath{fill,stroke}
\pgfpathmoveto{\pgfpoint{197.282440pt}{288.294830pt}}
\pgflineto{\pgfpoint{196.709503pt}{290.058167pt}}
\pgflineto{\pgfpoint{195.209503pt}{291.147980pt}}
\pgfpathclose
\pgfusepath{fill,stroke}
\pgfpathmoveto{\pgfpoint{238.628693pt}{216.574738pt}}
\pgflineto{\pgfpoint{236.555740pt}{213.721573pt}}
\pgflineto{\pgfpoint{238.055740pt}{214.811386pt}}
\pgfpathclose
\pgfusepath{fill,stroke}
\pgfpathmoveto{\pgfpoint{238.628693pt}{216.574738pt}}
\pgflineto{\pgfpoint{234.701630pt}{213.721573pt}}
\pgflineto{\pgfpoint{236.555740pt}{213.721573pt}}
\pgfpathclose
\pgfusepath{fill,stroke}
\pgfpathmoveto{\pgfpoint{238.628693pt}{216.574738pt}}
\pgflineto{\pgfpoint{233.201630pt}{214.811386pt}}
\pgflineto{\pgfpoint{234.701630pt}{213.721573pt}}
\pgfpathclose
\pgfusepath{fill,stroke}
\pgfpathmoveto{\pgfpoint{238.628693pt}{216.574738pt}}
\pgflineto{\pgfpoint{232.628693pt}{216.574738pt}}
\pgflineto{\pgfpoint{233.201630pt}{214.811386pt}}
\pgfpathclose
\pgfusepath{fill,stroke}
\pgfpathmoveto{\pgfpoint{238.628693pt}{216.574738pt}}
\pgflineto{\pgfpoint{233.201630pt}{218.338089pt}}
\pgflineto{\pgfpoint{232.628693pt}{216.574738pt}}
\pgfpathclose
\pgfusepath{fill,stroke}
\pgfpathmoveto{\pgfpoint{238.628693pt}{216.574738pt}}
\pgflineto{\pgfpoint{234.701630pt}{219.427902pt}}
\pgflineto{\pgfpoint{233.201630pt}{218.338089pt}}
\pgfpathclose
\pgfusepath{fill,stroke}
\pgfpathmoveto{\pgfpoint{238.628693pt}{216.574738pt}}
\pgflineto{\pgfpoint{236.555740pt}{219.427902pt}}
\pgflineto{\pgfpoint{234.701630pt}{219.427902pt}}
\pgfpathclose
\pgfusepath{fill,stroke}
\pgfpathmoveto{\pgfpoint{238.628693pt}{216.574738pt}}
\pgflineto{\pgfpoint{238.055740pt}{218.338089pt}}
\pgflineto{\pgfpoint{236.555740pt}{219.427902pt}}
\pgfpathclose
\pgfusepath{fill,stroke}
\pgfpathmoveto{\pgfpoint{224.024872pt}{225.454193pt}}
\pgflineto{\pgfpoint{221.951935pt}{222.601028pt}}
\pgflineto{\pgfpoint{223.451920pt}{223.690842pt}}
\pgfpathclose
\pgfusepath{fill,stroke}
\pgfpathmoveto{\pgfpoint{224.024872pt}{225.454193pt}}
\pgflineto{\pgfpoint{220.097809pt}{222.601028pt}}
\pgflineto{\pgfpoint{221.951935pt}{222.601028pt}}
\pgfpathclose
\pgfusepath{fill,stroke}
\pgfpathmoveto{\pgfpoint{224.024872pt}{225.454193pt}}
\pgflineto{\pgfpoint{218.597824pt}{223.690842pt}}
\pgflineto{\pgfpoint{220.097809pt}{222.601028pt}}
\pgfpathclose
\pgfusepath{fill,stroke}
\pgfpathmoveto{\pgfpoint{224.024872pt}{225.454193pt}}
\pgflineto{\pgfpoint{218.024872pt}{225.454193pt}}
\pgflineto{\pgfpoint{218.597824pt}{223.690842pt}}
\pgfpathclose
\pgfusepath{fill,stroke}
\pgfpathmoveto{\pgfpoint{224.024872pt}{225.454193pt}}
\pgflineto{\pgfpoint{218.597824pt}{227.217545pt}}
\pgflineto{\pgfpoint{218.024872pt}{225.454193pt}}
\pgfpathclose
\pgfusepath{fill,stroke}
\pgfpathmoveto{\pgfpoint{224.024872pt}{225.454193pt}}
\pgflineto{\pgfpoint{220.097809pt}{228.307358pt}}
\pgflineto{\pgfpoint{218.597824pt}{227.217545pt}}
\pgfpathclose
\pgfusepath{fill,stroke}
\pgfpathmoveto{\pgfpoint{224.024872pt}{225.454193pt}}
\pgflineto{\pgfpoint{221.951935pt}{228.307358pt}}
\pgflineto{\pgfpoint{220.097809pt}{228.307358pt}}
\pgfpathclose
\pgfusepath{fill,stroke}
\pgfpathmoveto{\pgfpoint{224.024872pt}{225.454193pt}}
\pgflineto{\pgfpoint{223.451920pt}{227.217545pt}}
\pgflineto{\pgfpoint{221.951935pt}{228.307358pt}}
\pgfpathclose
\pgfusepath{fill,stroke}
\pgfpathmoveto{\pgfpoint{228.610657pt}{206.462982pt}}
\pgflineto{\pgfpoint{226.537704pt}{203.609818pt}}
\pgflineto{\pgfpoint{228.037704pt}{204.699631pt}}
\pgfpathclose
\pgfusepath{fill,stroke}
\pgfpathmoveto{\pgfpoint{228.610657pt}{206.462982pt}}
\pgflineto{\pgfpoint{224.683594pt}{203.609818pt}}
\pgflineto{\pgfpoint{226.537704pt}{203.609818pt}}
\pgfpathclose
\pgfusepath{fill,stroke}
\pgfpathmoveto{\pgfpoint{228.610657pt}{206.462982pt}}
\pgflineto{\pgfpoint{223.183594pt}{204.699631pt}}
\pgflineto{\pgfpoint{224.683594pt}{203.609818pt}}
\pgfpathclose
\pgfusepath{fill,stroke}
\pgfpathmoveto{\pgfpoint{228.610657pt}{206.462982pt}}
\pgflineto{\pgfpoint{222.610657pt}{206.462982pt}}
\pgflineto{\pgfpoint{223.183594pt}{204.699631pt}}
\pgfpathclose
\pgfusepath{fill,stroke}
\pgfpathmoveto{\pgfpoint{228.610657pt}{206.462982pt}}
\pgflineto{\pgfpoint{223.183594pt}{208.226334pt}}
\pgflineto{\pgfpoint{222.610657pt}{206.462982pt}}
\pgfpathclose
\pgfusepath{fill,stroke}
\pgfpathmoveto{\pgfpoint{228.610657pt}{206.462982pt}}
\pgflineto{\pgfpoint{224.683594pt}{209.316147pt}}
\pgflineto{\pgfpoint{223.183594pt}{208.226334pt}}
\pgfpathclose
\pgfusepath{fill,stroke}
\pgfpathmoveto{\pgfpoint{228.610657pt}{206.462982pt}}
\pgflineto{\pgfpoint{226.537704pt}{209.316147pt}}
\pgflineto{\pgfpoint{224.683594pt}{209.316147pt}}
\pgfpathclose
\pgfusepath{fill,stroke}
\pgfpathmoveto{\pgfpoint{228.610657pt}{206.462982pt}}
\pgflineto{\pgfpoint{228.037704pt}{208.226334pt}}
\pgflineto{\pgfpoint{226.537704pt}{209.316147pt}}
\pgfpathclose
\pgfusepath{fill,stroke}
\pgfpathmoveto{\pgfpoint{133.962875pt}{247.078949pt}}
\pgflineto{\pgfpoint{131.889923pt}{244.225769pt}}
\pgflineto{\pgfpoint{133.389923pt}{245.315582pt}}
\pgfpathclose
\pgfusepath{fill,stroke}
\pgfpathmoveto{\pgfpoint{133.962875pt}{247.078949pt}}
\pgflineto{\pgfpoint{130.035812pt}{244.225769pt}}
\pgflineto{\pgfpoint{131.889923pt}{244.225769pt}}
\pgfpathclose
\pgfusepath{fill,stroke}
\pgfpathmoveto{\pgfpoint{133.962875pt}{247.078949pt}}
\pgflineto{\pgfpoint{128.535812pt}{245.315582pt}}
\pgflineto{\pgfpoint{130.035812pt}{244.225769pt}}
\pgfpathclose
\pgfusepath{fill,stroke}
\pgfpathmoveto{\pgfpoint{133.962875pt}{247.078949pt}}
\pgflineto{\pgfpoint{127.962860pt}{247.078949pt}}
\pgflineto{\pgfpoint{128.535812pt}{245.315582pt}}
\pgfpathclose
\pgfusepath{fill,stroke}
\pgfpathmoveto{\pgfpoint{133.962875pt}{247.078949pt}}
\pgflineto{\pgfpoint{128.535812pt}{248.842300pt}}
\pgflineto{\pgfpoint{127.962860pt}{247.078949pt}}
\pgfpathclose
\pgfusepath{fill,stroke}
\pgfpathmoveto{\pgfpoint{133.962875pt}{247.078949pt}}
\pgflineto{\pgfpoint{130.035812pt}{249.932114pt}}
\pgflineto{\pgfpoint{128.535812pt}{248.842300pt}}
\pgfpathclose
\pgfusepath{fill,stroke}
\pgfpathmoveto{\pgfpoint{133.962875pt}{247.078949pt}}
\pgflineto{\pgfpoint{131.889923pt}{249.932114pt}}
\pgflineto{\pgfpoint{130.035812pt}{249.932114pt}}
\pgfpathclose
\pgfusepath{fill,stroke}
\pgfpathmoveto{\pgfpoint{133.962875pt}{247.078949pt}}
\pgflineto{\pgfpoint{133.389923pt}{248.842300pt}}
\pgflineto{\pgfpoint{131.889923pt}{249.932114pt}}
\pgfpathclose
\pgfusepath{fill,stroke}
\pgfpathmoveto{\pgfpoint{179.694153pt}{234.622360pt}}
\pgflineto{\pgfpoint{177.621216pt}{231.769180pt}}
\pgflineto{\pgfpoint{179.121216pt}{232.858994pt}}
\pgfpathclose
\pgfusepath{fill,stroke}
\pgfpathmoveto{\pgfpoint{179.694153pt}{234.622360pt}}
\pgflineto{\pgfpoint{175.767120pt}{231.769180pt}}
\pgflineto{\pgfpoint{177.621216pt}{231.769180pt}}
\pgfpathclose
\pgfusepath{fill,stroke}
\pgfpathmoveto{\pgfpoint{179.694153pt}{234.622360pt}}
\pgflineto{\pgfpoint{174.267120pt}{232.858994pt}}
\pgflineto{\pgfpoint{175.767120pt}{231.769180pt}}
\pgfpathclose
\pgfusepath{fill,stroke}
\pgfpathmoveto{\pgfpoint{179.694153pt}{234.622360pt}}
\pgflineto{\pgfpoint{173.694168pt}{234.622360pt}}
\pgflineto{\pgfpoint{174.267120pt}{232.858994pt}}
\pgfpathclose
\pgfusepath{fill,stroke}
\pgfpathmoveto{\pgfpoint{179.694153pt}{234.622360pt}}
\pgflineto{\pgfpoint{174.267120pt}{236.385712pt}}
\pgflineto{\pgfpoint{173.694168pt}{234.622360pt}}
\pgfpathclose
\pgfusepath{fill,stroke}
\pgfpathmoveto{\pgfpoint{179.694153pt}{234.622360pt}}
\pgflineto{\pgfpoint{175.767120pt}{237.475525pt}}
\pgflineto{\pgfpoint{174.267120pt}{236.385712pt}}
\pgfpathclose
\pgfusepath{fill,stroke}
\pgfpathmoveto{\pgfpoint{179.694153pt}{234.622360pt}}
\pgflineto{\pgfpoint{177.621216pt}{237.475525pt}}
\pgflineto{\pgfpoint{175.767120pt}{237.475525pt}}
\pgfpathclose
\pgfusepath{fill,stroke}
\pgfpathmoveto{\pgfpoint{179.694153pt}{234.622360pt}}
\pgflineto{\pgfpoint{179.121216pt}{236.385712pt}}
\pgflineto{\pgfpoint{177.621216pt}{237.475525pt}}
\pgfpathclose
\pgfusepath{fill,stroke}
\pgfpathmoveto{\pgfpoint{204.250580pt}{218.426666pt}}
\pgflineto{\pgfpoint{202.177628pt}{215.573502pt}}
\pgflineto{\pgfpoint{203.677628pt}{216.663315pt}}
\pgfpathclose
\pgfusepath{fill,stroke}
\pgfpathmoveto{\pgfpoint{204.250580pt}{218.426666pt}}
\pgflineto{\pgfpoint{200.323532pt}{215.573502pt}}
\pgflineto{\pgfpoint{202.177628pt}{215.573502pt}}
\pgfpathclose
\pgfusepath{fill,stroke}
\pgfpathmoveto{\pgfpoint{204.250580pt}{218.426666pt}}
\pgflineto{\pgfpoint{198.823532pt}{216.663315pt}}
\pgflineto{\pgfpoint{200.323532pt}{215.573502pt}}
\pgfpathclose
\pgfusepath{fill,stroke}
\pgfpathmoveto{\pgfpoint{204.250580pt}{218.426666pt}}
\pgflineto{\pgfpoint{198.250580pt}{218.426666pt}}
\pgflineto{\pgfpoint{198.823532pt}{216.663315pt}}
\pgfpathclose
\pgfusepath{fill,stroke}
\pgfpathmoveto{\pgfpoint{204.250580pt}{218.426666pt}}
\pgflineto{\pgfpoint{198.823532pt}{220.190033pt}}
\pgflineto{\pgfpoint{198.250580pt}{218.426666pt}}
\pgfpathclose
\pgfusepath{fill,stroke}
\pgfpathmoveto{\pgfpoint{204.250580pt}{218.426666pt}}
\pgflineto{\pgfpoint{200.323532pt}{221.279846pt}}
\pgflineto{\pgfpoint{198.823532pt}{220.190033pt}}
\pgfpathclose
\pgfusepath{fill,stroke}
\pgfpathmoveto{\pgfpoint{204.250580pt}{218.426666pt}}
\pgflineto{\pgfpoint{202.177628pt}{221.279846pt}}
\pgflineto{\pgfpoint{200.323532pt}{221.279846pt}}
\pgfpathclose
\pgfusepath{fill,stroke}
\pgfpathmoveto{\pgfpoint{204.250580pt}{218.426666pt}}
\pgflineto{\pgfpoint{203.677628pt}{220.190033pt}}
\pgflineto{\pgfpoint{202.177628pt}{221.279846pt}}
\pgfpathclose
\pgfusepath{fill,stroke}
\pgfpathmoveto{\pgfpoint{232.184158pt}{210.575287pt}}
\pgflineto{\pgfpoint{230.111206pt}{207.722122pt}}
\pgflineto{\pgfpoint{231.611206pt}{208.811935pt}}
\pgfpathclose
\pgfusepath{fill,stroke}
\pgfpathmoveto{\pgfpoint{232.184158pt}{210.575287pt}}
\pgflineto{\pgfpoint{228.257111pt}{207.722122pt}}
\pgflineto{\pgfpoint{230.111206pt}{207.722122pt}}
\pgfpathclose
\pgfusepath{fill,stroke}
\pgfpathmoveto{\pgfpoint{232.184158pt}{210.575287pt}}
\pgflineto{\pgfpoint{226.757111pt}{208.811935pt}}
\pgflineto{\pgfpoint{228.257111pt}{207.722122pt}}
\pgfpathclose
\pgfusepath{fill,stroke}
\pgfpathmoveto{\pgfpoint{232.184158pt}{210.575287pt}}
\pgflineto{\pgfpoint{226.184158pt}{210.575287pt}}
\pgflineto{\pgfpoint{226.757111pt}{208.811935pt}}
\pgfpathclose
\pgfusepath{fill,stroke}
\pgfpathmoveto{\pgfpoint{232.184158pt}{210.575287pt}}
\pgflineto{\pgfpoint{226.757111pt}{212.338638pt}}
\pgflineto{\pgfpoint{226.184158pt}{210.575287pt}}
\pgfpathclose
\pgfusepath{fill,stroke}
\pgfpathmoveto{\pgfpoint{232.184158pt}{210.575287pt}}
\pgflineto{\pgfpoint{228.257111pt}{213.428452pt}}
\pgflineto{\pgfpoint{226.757111pt}{212.338638pt}}
\pgfpathclose
\pgfusepath{fill,stroke}
\pgfpathmoveto{\pgfpoint{232.184158pt}{210.575287pt}}
\pgflineto{\pgfpoint{230.111206pt}{213.428452pt}}
\pgflineto{\pgfpoint{228.257111pt}{213.428452pt}}
\pgfpathclose
\pgfusepath{fill,stroke}
\pgfpathmoveto{\pgfpoint{232.184158pt}{210.575287pt}}
\pgflineto{\pgfpoint{231.611206pt}{212.338638pt}}
\pgflineto{\pgfpoint{230.111206pt}{213.428452pt}}
\pgfpathclose
\pgfusepath{fill,stroke}
\pgfpathmoveto{\pgfpoint{215.974655pt}{218.842133pt}}
\pgflineto{\pgfpoint{213.901703pt}{215.988968pt}}
\pgflineto{\pgfpoint{215.401703pt}{217.078781pt}}
\pgfpathclose
\pgfusepath{fill,stroke}
\pgfpathmoveto{\pgfpoint{215.974655pt}{218.842133pt}}
\pgflineto{\pgfpoint{212.047607pt}{215.988968pt}}
\pgflineto{\pgfpoint{213.901703pt}{215.988968pt}}
\pgfpathclose
\pgfusepath{fill,stroke}
\pgfpathmoveto{\pgfpoint{215.974655pt}{218.842133pt}}
\pgflineto{\pgfpoint{210.547607pt}{217.078781pt}}
\pgflineto{\pgfpoint{212.047607pt}{215.988968pt}}
\pgfpathclose
\pgfusepath{fill,stroke}
\pgfpathmoveto{\pgfpoint{215.974655pt}{218.842133pt}}
\pgflineto{\pgfpoint{209.974670pt}{218.842133pt}}
\pgflineto{\pgfpoint{210.547607pt}{217.078781pt}}
\pgfpathclose
\pgfusepath{fill,stroke}
\pgfpathmoveto{\pgfpoint{215.974655pt}{218.842133pt}}
\pgflineto{\pgfpoint{210.547607pt}{220.605484pt}}
\pgflineto{\pgfpoint{209.974670pt}{218.842133pt}}
\pgfpathclose
\pgfusepath{fill,stroke}
\pgfpathmoveto{\pgfpoint{215.974655pt}{218.842133pt}}
\pgflineto{\pgfpoint{212.047607pt}{221.695297pt}}
\pgflineto{\pgfpoint{210.547607pt}{220.605484pt}}
\pgfpathclose
\pgfusepath{fill,stroke}
\pgfpathmoveto{\pgfpoint{215.974655pt}{218.842133pt}}
\pgflineto{\pgfpoint{213.901703pt}{221.695297pt}}
\pgflineto{\pgfpoint{212.047607pt}{221.695297pt}}
\pgfpathclose
\pgfusepath{fill,stroke}
\pgfpathmoveto{\pgfpoint{215.974655pt}{218.842133pt}}
\pgflineto{\pgfpoint{215.401703pt}{220.605484pt}}
\pgflineto{\pgfpoint{213.901703pt}{221.695297pt}}
\pgfpathclose
\pgfusepath{fill,stroke}
\pgfpathmoveto{\pgfpoint{137.597458pt}{320.024994pt}}
\pgflineto{\pgfpoint{135.524506pt}{317.171814pt}}
\pgflineto{\pgfpoint{137.024506pt}{318.261627pt}}
\pgfpathclose
\pgfusepath{fill,stroke}
\pgfpathmoveto{\pgfpoint{137.597458pt}{320.024994pt}}
\pgflineto{\pgfpoint{133.670395pt}{317.171814pt}}
\pgflineto{\pgfpoint{135.524506pt}{317.171814pt}}
\pgfpathclose
\pgfusepath{fill,stroke}
\pgfpathmoveto{\pgfpoint{137.597458pt}{320.024994pt}}
\pgflineto{\pgfpoint{132.170410pt}{318.261627pt}}
\pgflineto{\pgfpoint{133.670395pt}{317.171814pt}}
\pgfpathclose
\pgfusepath{fill,stroke}
\pgfpathmoveto{\pgfpoint{137.597458pt}{320.024994pt}}
\pgflineto{\pgfpoint{131.597443pt}{320.024994pt}}
\pgflineto{\pgfpoint{132.170410pt}{318.261627pt}}
\pgfpathclose
\pgfusepath{fill,stroke}
\pgfpathmoveto{\pgfpoint{137.597458pt}{320.024994pt}}
\pgflineto{\pgfpoint{132.170410pt}{321.788330pt}}
\pgflineto{\pgfpoint{131.597443pt}{320.024994pt}}
\pgfpathclose
\pgfusepath{fill,stroke}
\pgfpathmoveto{\pgfpoint{137.597458pt}{320.024994pt}}
\pgflineto{\pgfpoint{133.670395pt}{322.878174pt}}
\pgflineto{\pgfpoint{132.170410pt}{321.788330pt}}
\pgfpathclose
\pgfusepath{fill,stroke}
\pgfpathmoveto{\pgfpoint{137.597458pt}{320.024994pt}}
\pgflineto{\pgfpoint{135.524506pt}{322.878174pt}}
\pgflineto{\pgfpoint{133.670395pt}{322.878174pt}}
\pgfpathclose
\pgfusepath{fill,stroke}
\pgfpathmoveto{\pgfpoint{137.597458pt}{320.024994pt}}
\pgflineto{\pgfpoint{137.024506pt}{321.788330pt}}
\pgflineto{\pgfpoint{135.524506pt}{322.878174pt}}
\pgfpathclose
\pgfusepath{fill,stroke}
\pgfpathmoveto{\pgfpoint{213.251984pt}{238.910690pt}}
\pgflineto{\pgfpoint{211.179047pt}{236.057526pt}}
\pgflineto{\pgfpoint{212.679047pt}{237.147339pt}}
\pgfpathclose
\pgfusepath{fill,stroke}
\pgfpathmoveto{\pgfpoint{213.251984pt}{238.910690pt}}
\pgflineto{\pgfpoint{209.324936pt}{236.057526pt}}
\pgflineto{\pgfpoint{211.179047pt}{236.057526pt}}
\pgfpathclose
\pgfusepath{fill,stroke}
\pgfpathmoveto{\pgfpoint{213.251984pt}{238.910690pt}}
\pgflineto{\pgfpoint{207.824936pt}{237.147339pt}}
\pgflineto{\pgfpoint{209.324936pt}{236.057526pt}}
\pgfpathclose
\pgfusepath{fill,stroke}
\pgfpathmoveto{\pgfpoint{213.251984pt}{238.910690pt}}
\pgflineto{\pgfpoint{207.251984pt}{238.910690pt}}
\pgflineto{\pgfpoint{207.824936pt}{237.147339pt}}
\pgfpathclose
\pgfusepath{fill,stroke}
\pgfpathmoveto{\pgfpoint{213.251984pt}{238.910690pt}}
\pgflineto{\pgfpoint{207.824936pt}{240.674042pt}}
\pgflineto{\pgfpoint{207.251984pt}{238.910690pt}}
\pgfpathclose
\pgfusepath{fill,stroke}
\pgfpathmoveto{\pgfpoint{213.251984pt}{238.910690pt}}
\pgflineto{\pgfpoint{209.324936pt}{241.763855pt}}
\pgflineto{\pgfpoint{207.824936pt}{240.674042pt}}
\pgfpathclose
\pgfusepath{fill,stroke}
\pgfpathmoveto{\pgfpoint{213.251984pt}{238.910690pt}}
\pgflineto{\pgfpoint{211.179047pt}{241.763855pt}}
\pgflineto{\pgfpoint{209.324936pt}{241.763855pt}}
\pgfpathclose
\pgfusepath{fill,stroke}
\pgfpathmoveto{\pgfpoint{213.251984pt}{238.910690pt}}
\pgflineto{\pgfpoint{212.679047pt}{240.674042pt}}
\pgflineto{\pgfpoint{211.179047pt}{241.763855pt}}
\pgfpathclose
\pgfusepath{fill,stroke}
\pgfpathmoveto{\pgfpoint{200.118561pt}{255.338745pt}}
\pgflineto{\pgfpoint{198.045624pt}{252.485580pt}}
\pgflineto{\pgfpoint{199.545624pt}{253.575394pt}}
\pgfpathclose
\pgfusepath{fill,stroke}
\pgfpathmoveto{\pgfpoint{200.118561pt}{255.338745pt}}
\pgflineto{\pgfpoint{196.191528pt}{252.485580pt}}
\pgflineto{\pgfpoint{198.045624pt}{252.485580pt}}
\pgfpathclose
\pgfusepath{fill,stroke}
\pgfpathmoveto{\pgfpoint{200.118561pt}{255.338745pt}}
\pgflineto{\pgfpoint{194.691528pt}{253.575394pt}}
\pgflineto{\pgfpoint{196.191528pt}{252.485580pt}}
\pgfpathclose
\pgfusepath{fill,stroke}
\pgfpathmoveto{\pgfpoint{200.118561pt}{255.338745pt}}
\pgflineto{\pgfpoint{194.118576pt}{255.338745pt}}
\pgflineto{\pgfpoint{194.691528pt}{253.575394pt}}
\pgfpathclose
\pgfusepath{fill,stroke}
\pgfpathmoveto{\pgfpoint{200.118561pt}{255.338745pt}}
\pgflineto{\pgfpoint{194.691528pt}{257.102112pt}}
\pgflineto{\pgfpoint{194.118576pt}{255.338745pt}}
\pgfpathclose
\pgfusepath{fill,stroke}
\pgfpathmoveto{\pgfpoint{200.118561pt}{255.338745pt}}
\pgflineto{\pgfpoint{196.191528pt}{258.191925pt}}
\pgflineto{\pgfpoint{194.691528pt}{257.102112pt}}
\pgfpathclose
\pgfusepath{fill,stroke}
\pgfpathmoveto{\pgfpoint{200.118561pt}{255.338745pt}}
\pgflineto{\pgfpoint{198.045624pt}{258.191925pt}}
\pgflineto{\pgfpoint{196.191528pt}{258.191925pt}}
\pgfpathclose
\pgfusepath{fill,stroke}
\pgfpathmoveto{\pgfpoint{200.118561pt}{255.338745pt}}
\pgflineto{\pgfpoint{199.545624pt}{257.102112pt}}
\pgflineto{\pgfpoint{198.045624pt}{258.191925pt}}
\pgfpathclose
\pgfusepath{fill,stroke}
\pgfpathmoveto{\pgfpoint{232.795013pt}{217.863342pt}}
\pgflineto{\pgfpoint{230.722076pt}{215.010178pt}}
\pgflineto{\pgfpoint{232.222076pt}{216.099991pt}}
\pgfpathclose
\pgfusepath{fill,stroke}
\pgfpathmoveto{\pgfpoint{232.795013pt}{217.863342pt}}
\pgflineto{\pgfpoint{228.867966pt}{215.010178pt}}
\pgflineto{\pgfpoint{230.722076pt}{215.010178pt}}
\pgfpathclose
\pgfusepath{fill,stroke}
\pgfpathmoveto{\pgfpoint{232.795013pt}{217.863342pt}}
\pgflineto{\pgfpoint{227.367966pt}{216.099991pt}}
\pgflineto{\pgfpoint{228.867966pt}{215.010178pt}}
\pgfpathclose
\pgfusepath{fill,stroke}
\pgfpathmoveto{\pgfpoint{232.795013pt}{217.863342pt}}
\pgflineto{\pgfpoint{226.795013pt}{217.863342pt}}
\pgflineto{\pgfpoint{227.367966pt}{216.099991pt}}
\pgfpathclose
\pgfusepath{fill,stroke}
\pgfpathmoveto{\pgfpoint{232.795013pt}{217.863342pt}}
\pgflineto{\pgfpoint{227.367966pt}{219.626694pt}}
\pgflineto{\pgfpoint{226.795013pt}{217.863342pt}}
\pgfpathclose
\pgfusepath{fill,stroke}
\pgfpathmoveto{\pgfpoint{232.795013pt}{217.863342pt}}
\pgflineto{\pgfpoint{228.867966pt}{220.716507pt}}
\pgflineto{\pgfpoint{227.367966pt}{219.626694pt}}
\pgfpathclose
\pgfusepath{fill,stroke}
\pgfpathmoveto{\pgfpoint{232.795013pt}{217.863342pt}}
\pgflineto{\pgfpoint{230.722076pt}{220.716507pt}}
\pgflineto{\pgfpoint{228.867966pt}{220.716507pt}}
\pgfpathclose
\pgfusepath{fill,stroke}
\pgfpathmoveto{\pgfpoint{232.795013pt}{217.863342pt}}
\pgflineto{\pgfpoint{232.222076pt}{219.626694pt}}
\pgflineto{\pgfpoint{230.722076pt}{220.716507pt}}
\pgfpathclose
\pgfusepath{fill,stroke}
\pgfpathmoveto{\pgfpoint{178.053589pt}{274.591187pt}}
\pgflineto{\pgfpoint{175.980652pt}{271.738037pt}}
\pgflineto{\pgfpoint{177.480652pt}{272.827820pt}}
\pgfpathclose
\pgfusepath{fill,stroke}
\pgfpathmoveto{\pgfpoint{178.053589pt}{274.591187pt}}
\pgflineto{\pgfpoint{174.126556pt}{271.738037pt}}
\pgflineto{\pgfpoint{175.980652pt}{271.738037pt}}
\pgfpathclose
\pgfusepath{fill,stroke}
\pgfpathmoveto{\pgfpoint{178.053589pt}{274.591187pt}}
\pgflineto{\pgfpoint{172.626541pt}{272.827820pt}}
\pgflineto{\pgfpoint{174.126556pt}{271.738037pt}}
\pgfpathclose
\pgfusepath{fill,stroke}
\pgfpathmoveto{\pgfpoint{178.053589pt}{274.591187pt}}
\pgflineto{\pgfpoint{172.053589pt}{274.591187pt}}
\pgflineto{\pgfpoint{172.626541pt}{272.827820pt}}
\pgfpathclose
\pgfusepath{fill,stroke}
\pgfpathmoveto{\pgfpoint{178.053589pt}{274.591187pt}}
\pgflineto{\pgfpoint{172.626541pt}{276.354553pt}}
\pgflineto{\pgfpoint{172.053589pt}{274.591187pt}}
\pgfpathclose
\pgfusepath{fill,stroke}
\pgfpathmoveto{\pgfpoint{178.053589pt}{274.591187pt}}
\pgflineto{\pgfpoint{174.126556pt}{277.444336pt}}
\pgflineto{\pgfpoint{172.626541pt}{276.354553pt}}
\pgfpathclose
\pgfusepath{fill,stroke}
\pgfpathmoveto{\pgfpoint{178.053589pt}{274.591187pt}}
\pgflineto{\pgfpoint{175.980652pt}{277.444336pt}}
\pgflineto{\pgfpoint{174.126556pt}{277.444336pt}}
\pgfpathclose
\pgfusepath{fill,stroke}
\pgfpathmoveto{\pgfpoint{178.053589pt}{274.591187pt}}
\pgflineto{\pgfpoint{177.480652pt}{276.354553pt}}
\pgflineto{\pgfpoint{175.980652pt}{277.444336pt}}
\pgfpathclose
\pgfusepath{fill,stroke}
\pgfpathmoveto{\pgfpoint{228.968445pt}{226.707581pt}}
\pgflineto{\pgfpoint{226.895493pt}{223.854416pt}}
\pgflineto{\pgfpoint{228.395493pt}{224.944229pt}}
\pgfpathclose
\pgfusepath{fill,stroke}
\pgfpathmoveto{\pgfpoint{228.968445pt}{226.707581pt}}
\pgflineto{\pgfpoint{225.041397pt}{223.854416pt}}
\pgflineto{\pgfpoint{226.895493pt}{223.854416pt}}
\pgfpathclose
\pgfusepath{fill,stroke}
\pgfpathmoveto{\pgfpoint{228.968445pt}{226.707581pt}}
\pgflineto{\pgfpoint{223.541397pt}{224.944229pt}}
\pgflineto{\pgfpoint{225.041397pt}{223.854416pt}}
\pgfpathclose
\pgfusepath{fill,stroke}
\pgfpathmoveto{\pgfpoint{228.968445pt}{226.707581pt}}
\pgflineto{\pgfpoint{222.968445pt}{226.707581pt}}
\pgflineto{\pgfpoint{223.541397pt}{224.944229pt}}
\pgfpathclose
\pgfusepath{fill,stroke}
\pgfpathmoveto{\pgfpoint{228.968445pt}{226.707581pt}}
\pgflineto{\pgfpoint{223.541397pt}{228.470947pt}}
\pgflineto{\pgfpoint{222.968445pt}{226.707581pt}}
\pgfpathclose
\pgfusepath{fill,stroke}
\pgfpathmoveto{\pgfpoint{228.968445pt}{226.707581pt}}
\pgflineto{\pgfpoint{225.041397pt}{229.560760pt}}
\pgflineto{\pgfpoint{223.541397pt}{228.470947pt}}
\pgfpathclose
\pgfusepath{fill,stroke}
\pgfpathmoveto{\pgfpoint{228.968445pt}{226.707581pt}}
\pgflineto{\pgfpoint{226.895493pt}{229.560760pt}}
\pgflineto{\pgfpoint{225.041397pt}{229.560760pt}}
\pgfpathclose
\pgfusepath{fill,stroke}
\pgfpathmoveto{\pgfpoint{228.968445pt}{226.707581pt}}
\pgflineto{\pgfpoint{228.395493pt}{228.470947pt}}
\pgflineto{\pgfpoint{226.895493pt}{229.560760pt}}
\pgfpathclose
\pgfusepath{fill,stroke}
\pgfpathmoveto{\pgfpoint{182.203033pt}{204.118164pt}}
\pgflineto{\pgfpoint{180.130066pt}{201.264984pt}}
\pgflineto{\pgfpoint{181.630081pt}{202.354797pt}}
\pgfpathclose
\pgfusepath{fill,stroke}
\pgfpathmoveto{\pgfpoint{182.203033pt}{204.118164pt}}
\pgflineto{\pgfpoint{178.275970pt}{201.264984pt}}
\pgflineto{\pgfpoint{180.130066pt}{201.264984pt}}
\pgfpathclose
\pgfusepath{fill,stroke}
\pgfpathmoveto{\pgfpoint{182.203033pt}{204.118164pt}}
\pgflineto{\pgfpoint{176.775970pt}{202.354797pt}}
\pgflineto{\pgfpoint{178.275970pt}{201.264984pt}}
\pgfpathclose
\pgfusepath{fill,stroke}
\pgfpathmoveto{\pgfpoint{182.203033pt}{204.118164pt}}
\pgflineto{\pgfpoint{176.203033pt}{204.118164pt}}
\pgflineto{\pgfpoint{176.775970pt}{202.354797pt}}
\pgfpathclose
\pgfusepath{fill,stroke}
\pgfpathmoveto{\pgfpoint{182.203033pt}{204.118164pt}}
\pgflineto{\pgfpoint{176.775970pt}{205.881516pt}}
\pgflineto{\pgfpoint{176.203033pt}{204.118164pt}}
\pgfpathclose
\pgfusepath{fill,stroke}
\pgfpathmoveto{\pgfpoint{182.203033pt}{204.118164pt}}
\pgflineto{\pgfpoint{178.275970pt}{206.971329pt}}
\pgflineto{\pgfpoint{176.775970pt}{205.881516pt}}
\pgfpathclose
\pgfusepath{fill,stroke}
\pgfpathmoveto{\pgfpoint{182.203033pt}{204.118164pt}}
\pgflineto{\pgfpoint{180.130066pt}{206.971329pt}}
\pgflineto{\pgfpoint{178.275970pt}{206.971329pt}}
\pgfpathclose
\pgfusepath{fill,stroke}
\pgfpathmoveto{\pgfpoint{182.203033pt}{204.118164pt}}
\pgflineto{\pgfpoint{181.630081pt}{205.881516pt}}
\pgflineto{\pgfpoint{180.130066pt}{206.971329pt}}
\pgfpathclose
\pgfusepath{fill,stroke}
\pgfpathmoveto{\pgfpoint{180.187195pt}{269.298004pt}}
\pgflineto{\pgfpoint{178.114258pt}{266.444824pt}}
\pgflineto{\pgfpoint{179.614258pt}{267.534637pt}}
\pgfpathclose
\pgfusepath{fill,stroke}
\pgfpathmoveto{\pgfpoint{180.187195pt}{269.298004pt}}
\pgflineto{\pgfpoint{176.260162pt}{266.444824pt}}
\pgflineto{\pgfpoint{178.114258pt}{266.444824pt}}
\pgfpathclose
\pgfusepath{fill,stroke}
\pgfpathmoveto{\pgfpoint{180.187195pt}{269.298004pt}}
\pgflineto{\pgfpoint{174.760162pt}{267.534637pt}}
\pgflineto{\pgfpoint{176.260162pt}{266.444824pt}}
\pgfpathclose
\pgfusepath{fill,stroke}
\pgfpathmoveto{\pgfpoint{180.187195pt}{269.298004pt}}
\pgflineto{\pgfpoint{174.187210pt}{269.298004pt}}
\pgflineto{\pgfpoint{174.760162pt}{267.534637pt}}
\pgfpathclose
\pgfusepath{fill,stroke}
\pgfpathmoveto{\pgfpoint{180.187195pt}{269.298004pt}}
\pgflineto{\pgfpoint{174.760162pt}{271.061340pt}}
\pgflineto{\pgfpoint{174.187210pt}{269.298004pt}}
\pgfpathclose
\pgfusepath{fill,stroke}
\pgfpathmoveto{\pgfpoint{180.187195pt}{269.298004pt}}
\pgflineto{\pgfpoint{176.260162pt}{272.151154pt}}
\pgflineto{\pgfpoint{174.760162pt}{271.061340pt}}
\pgfpathclose
\pgfusepath{fill,stroke}
\pgfpathmoveto{\pgfpoint{180.187195pt}{269.298004pt}}
\pgflineto{\pgfpoint{178.114258pt}{272.151154pt}}
\pgflineto{\pgfpoint{176.260162pt}{272.151154pt}}
\pgfpathclose
\pgfusepath{fill,stroke}
\pgfpathmoveto{\pgfpoint{180.187195pt}{269.298004pt}}
\pgflineto{\pgfpoint{179.614258pt}{271.061340pt}}
\pgflineto{\pgfpoint{178.114258pt}{272.151154pt}}
\pgfpathclose
\pgfusepath{fill,stroke}
\pgfpathmoveto{\pgfpoint{217.148376pt}{281.919373pt}}
\pgflineto{\pgfpoint{215.075424pt}{279.066193pt}}
\pgflineto{\pgfpoint{216.575424pt}{280.156006pt}}
\pgfpathclose
\pgfusepath{fill,stroke}
\pgfpathmoveto{\pgfpoint{217.148376pt}{281.919373pt}}
\pgflineto{\pgfpoint{213.221329pt}{279.066193pt}}
\pgflineto{\pgfpoint{215.075424pt}{279.066193pt}}
\pgfpathclose
\pgfusepath{fill,stroke}
\pgfpathmoveto{\pgfpoint{217.148376pt}{281.919373pt}}
\pgflineto{\pgfpoint{211.721313pt}{280.156006pt}}
\pgflineto{\pgfpoint{213.221329pt}{279.066193pt}}
\pgfpathclose
\pgfusepath{fill,stroke}
\pgfpathmoveto{\pgfpoint{217.148376pt}{281.919373pt}}
\pgflineto{\pgfpoint{211.148376pt}{281.919373pt}}
\pgflineto{\pgfpoint{211.721313pt}{280.156006pt}}
\pgfpathclose
\pgfusepath{fill,stroke}
\pgfpathmoveto{\pgfpoint{217.148376pt}{281.919373pt}}
\pgflineto{\pgfpoint{211.721313pt}{283.682739pt}}
\pgflineto{\pgfpoint{211.148376pt}{281.919373pt}}
\pgfpathclose
\pgfusepath{fill,stroke}
\pgfpathmoveto{\pgfpoint{217.148376pt}{281.919373pt}}
\pgflineto{\pgfpoint{213.221329pt}{284.772522pt}}
\pgflineto{\pgfpoint{211.721313pt}{283.682739pt}}
\pgfpathclose
\pgfusepath{fill,stroke}
\pgfpathmoveto{\pgfpoint{217.148376pt}{281.919373pt}}
\pgflineto{\pgfpoint{215.075424pt}{284.772522pt}}
\pgflineto{\pgfpoint{213.221329pt}{284.772522pt}}
\pgfpathclose
\pgfusepath{fill,stroke}
\pgfpathmoveto{\pgfpoint{217.148376pt}{281.919373pt}}
\pgflineto{\pgfpoint{216.575424pt}{283.682739pt}}
\pgflineto{\pgfpoint{215.075424pt}{284.772522pt}}
\pgfpathclose
\pgfusepath{fill,stroke}
\pgfpathmoveto{\pgfpoint{175.291656pt}{283.002350pt}}
\pgflineto{\pgfpoint{173.218689pt}{280.149200pt}}
\pgflineto{\pgfpoint{174.718704pt}{281.239014pt}}
\pgfpathclose
\pgfusepath{fill,stroke}
\pgfpathmoveto{\pgfpoint{175.291656pt}{283.002350pt}}
\pgflineto{\pgfpoint{171.364594pt}{280.149200pt}}
\pgflineto{\pgfpoint{173.218689pt}{280.149200pt}}
\pgfpathclose
\pgfusepath{fill,stroke}
\pgfpathmoveto{\pgfpoint{175.291656pt}{283.002350pt}}
\pgflineto{\pgfpoint{169.864594pt}{281.239014pt}}
\pgflineto{\pgfpoint{171.364594pt}{280.149200pt}}
\pgfpathclose
\pgfusepath{fill,stroke}
\pgfpathmoveto{\pgfpoint{175.291656pt}{283.002350pt}}
\pgflineto{\pgfpoint{169.291656pt}{283.002350pt}}
\pgflineto{\pgfpoint{169.864594pt}{281.239014pt}}
\pgfpathclose
\pgfusepath{fill,stroke}
\pgfpathmoveto{\pgfpoint{175.291656pt}{283.002350pt}}
\pgflineto{\pgfpoint{169.864594pt}{284.765717pt}}
\pgflineto{\pgfpoint{169.291656pt}{283.002350pt}}
\pgfpathclose
\pgfusepath{fill,stroke}
\pgfpathmoveto{\pgfpoint{175.291656pt}{283.002350pt}}
\pgflineto{\pgfpoint{171.364594pt}{285.855530pt}}
\pgflineto{\pgfpoint{169.864594pt}{284.765717pt}}
\pgfpathclose
\pgfusepath{fill,stroke}
\pgfpathmoveto{\pgfpoint{175.291656pt}{283.002350pt}}
\pgflineto{\pgfpoint{173.218689pt}{285.855530pt}}
\pgflineto{\pgfpoint{171.364594pt}{285.855530pt}}
\pgfpathclose
\pgfusepath{fill,stroke}
\pgfpathmoveto{\pgfpoint{175.291656pt}{283.002350pt}}
\pgflineto{\pgfpoint{174.718704pt}{284.765717pt}}
\pgflineto{\pgfpoint{173.218689pt}{285.855530pt}}
\pgfpathclose
\pgfusepath{fill,stroke}
\pgfpathmoveto{\pgfpoint{207.994263pt}{250.867325pt}}
\pgflineto{\pgfpoint{205.921295pt}{248.014160pt}}
\pgflineto{\pgfpoint{207.421310pt}{249.103973pt}}
\pgfpathclose
\pgfusepath{fill,stroke}
\pgfpathmoveto{\pgfpoint{207.994263pt}{250.867325pt}}
\pgflineto{\pgfpoint{204.067200pt}{248.014160pt}}
\pgflineto{\pgfpoint{205.921295pt}{248.014160pt}}
\pgfpathclose
\pgfusepath{fill,stroke}
\pgfpathmoveto{\pgfpoint{207.994263pt}{250.867325pt}}
\pgflineto{\pgfpoint{202.567200pt}{249.103973pt}}
\pgflineto{\pgfpoint{204.067200pt}{248.014160pt}}
\pgfpathclose
\pgfusepath{fill,stroke}
\pgfpathmoveto{\pgfpoint{207.994263pt}{250.867325pt}}
\pgflineto{\pgfpoint{201.994263pt}{250.867325pt}}
\pgflineto{\pgfpoint{202.567200pt}{249.103973pt}}
\pgfpathclose
\pgfusepath{fill,stroke}
\pgfpathmoveto{\pgfpoint{207.994263pt}{250.867325pt}}
\pgflineto{\pgfpoint{202.567200pt}{252.630676pt}}
\pgflineto{\pgfpoint{201.994263pt}{250.867325pt}}
\pgfpathclose
\pgfusepath{fill,stroke}
\pgfpathmoveto{\pgfpoint{207.994263pt}{250.867325pt}}
\pgflineto{\pgfpoint{204.067200pt}{253.720490pt}}
\pgflineto{\pgfpoint{202.567200pt}{252.630676pt}}
\pgfpathclose
\pgfusepath{fill,stroke}
\pgfpathmoveto{\pgfpoint{207.994263pt}{250.867325pt}}
\pgflineto{\pgfpoint{205.921295pt}{253.720490pt}}
\pgflineto{\pgfpoint{204.067200pt}{253.720490pt}}
\pgfpathclose
\pgfusepath{fill,stroke}
\pgfpathmoveto{\pgfpoint{207.994263pt}{250.867325pt}}
\pgflineto{\pgfpoint{207.421310pt}{252.630676pt}}
\pgflineto{\pgfpoint{205.921295pt}{253.720490pt}}
\pgfpathclose
\pgfusepath{fill,stroke}
\pgfpathmoveto{\pgfpoint{232.812469pt}{220.011032pt}}
\pgflineto{\pgfpoint{230.739532pt}{217.157867pt}}
\pgflineto{\pgfpoint{232.239532pt}{218.247681pt}}
\pgfpathclose
\pgfusepath{fill,stroke}
\pgfpathmoveto{\pgfpoint{232.812469pt}{220.011032pt}}
\pgflineto{\pgfpoint{228.885422pt}{217.157867pt}}
\pgflineto{\pgfpoint{230.739532pt}{217.157867pt}}
\pgfpathclose
\pgfusepath{fill,stroke}
\pgfpathmoveto{\pgfpoint{232.812469pt}{220.011032pt}}
\pgflineto{\pgfpoint{227.385422pt}{218.247681pt}}
\pgflineto{\pgfpoint{228.885422pt}{217.157867pt}}
\pgfpathclose
\pgfusepath{fill,stroke}
\pgfpathmoveto{\pgfpoint{232.812469pt}{220.011032pt}}
\pgflineto{\pgfpoint{226.812469pt}{220.011032pt}}
\pgflineto{\pgfpoint{227.385422pt}{218.247681pt}}
\pgfpathclose
\pgfusepath{fill,stroke}
\pgfpathmoveto{\pgfpoint{232.812469pt}{220.011032pt}}
\pgflineto{\pgfpoint{227.385422pt}{221.774384pt}}
\pgflineto{\pgfpoint{226.812469pt}{220.011032pt}}
\pgfpathclose
\pgfusepath{fill,stroke}
\pgfpathmoveto{\pgfpoint{232.812469pt}{220.011032pt}}
\pgflineto{\pgfpoint{228.885422pt}{222.864197pt}}
\pgflineto{\pgfpoint{227.385422pt}{221.774384pt}}
\pgfpathclose
\pgfusepath{fill,stroke}
\pgfpathmoveto{\pgfpoint{232.812469pt}{220.011032pt}}
\pgflineto{\pgfpoint{230.739532pt}{222.864197pt}}
\pgflineto{\pgfpoint{228.885422pt}{222.864197pt}}
\pgfpathclose
\pgfusepath{fill,stroke}
\pgfpathmoveto{\pgfpoint{232.812469pt}{220.011032pt}}
\pgflineto{\pgfpoint{232.239532pt}{221.774384pt}}
\pgflineto{\pgfpoint{230.739532pt}{222.864197pt}}
\pgfpathclose
\pgfusepath{fill,stroke}
\pgfpathmoveto{\pgfpoint{241.242279pt}{211.582230pt}}
\pgflineto{\pgfpoint{239.169342pt}{208.729065pt}}
\pgflineto{\pgfpoint{240.669342pt}{209.818878pt}}
\pgfpathclose
\pgfusepath{fill,stroke}
\pgfpathmoveto{\pgfpoint{241.242279pt}{211.582230pt}}
\pgflineto{\pgfpoint{237.315231pt}{208.729065pt}}
\pgflineto{\pgfpoint{239.169342pt}{208.729065pt}}
\pgfpathclose
\pgfusepath{fill,stroke}
\pgfpathmoveto{\pgfpoint{241.242279pt}{211.582230pt}}
\pgflineto{\pgfpoint{235.815231pt}{209.818878pt}}
\pgflineto{\pgfpoint{237.315231pt}{208.729065pt}}
\pgfpathclose
\pgfusepath{fill,stroke}
\pgfpathmoveto{\pgfpoint{241.242279pt}{211.582230pt}}
\pgflineto{\pgfpoint{235.242279pt}{211.582230pt}}
\pgflineto{\pgfpoint{235.815231pt}{209.818878pt}}
\pgfpathclose
\pgfusepath{fill,stroke}
\pgfpathmoveto{\pgfpoint{241.242279pt}{211.582230pt}}
\pgflineto{\pgfpoint{235.815231pt}{213.345596pt}}
\pgflineto{\pgfpoint{235.242279pt}{211.582230pt}}
\pgfpathclose
\pgfusepath{fill,stroke}
\pgfpathmoveto{\pgfpoint{241.242279pt}{211.582230pt}}
\pgflineto{\pgfpoint{237.315231pt}{214.435410pt}}
\pgflineto{\pgfpoint{235.815231pt}{213.345596pt}}
\pgfpathclose
\pgfusepath{fill,stroke}
\pgfpathmoveto{\pgfpoint{241.242279pt}{211.582230pt}}
\pgflineto{\pgfpoint{239.169342pt}{214.435410pt}}
\pgflineto{\pgfpoint{237.315231pt}{214.435410pt}}
\pgfpathclose
\pgfusepath{fill,stroke}
\pgfpathmoveto{\pgfpoint{241.242279pt}{211.582230pt}}
\pgflineto{\pgfpoint{240.669342pt}{213.345596pt}}
\pgflineto{\pgfpoint{239.169342pt}{214.435410pt}}
\pgfpathclose
\pgfusepath{fill,stroke}
\pgfpathmoveto{\pgfpoint{237.502991pt}{215.955063pt}}
\pgflineto{\pgfpoint{235.430038pt}{213.101898pt}}
\pgflineto{\pgfpoint{236.930038pt}{214.191711pt}}
\pgfpathclose
\pgfusepath{fill,stroke}
\pgfpathmoveto{\pgfpoint{237.502991pt}{215.955063pt}}
\pgflineto{\pgfpoint{233.575928pt}{213.101898pt}}
\pgflineto{\pgfpoint{235.430038pt}{213.101898pt}}
\pgfpathclose
\pgfusepath{fill,stroke}
\pgfpathmoveto{\pgfpoint{237.502991pt}{215.955063pt}}
\pgflineto{\pgfpoint{232.075928pt}{214.191711pt}}
\pgflineto{\pgfpoint{233.575928pt}{213.101898pt}}
\pgfpathclose
\pgfusepath{fill,stroke}
\pgfpathmoveto{\pgfpoint{237.502991pt}{215.955063pt}}
\pgflineto{\pgfpoint{231.502991pt}{215.955063pt}}
\pgflineto{\pgfpoint{232.075928pt}{214.191711pt}}
\pgfpathclose
\pgfusepath{fill,stroke}
\pgfpathmoveto{\pgfpoint{237.502991pt}{215.955063pt}}
\pgflineto{\pgfpoint{232.075928pt}{217.718430pt}}
\pgflineto{\pgfpoint{231.502991pt}{215.955063pt}}
\pgfpathclose
\pgfusepath{fill,stroke}
\pgfpathmoveto{\pgfpoint{237.502991pt}{215.955063pt}}
\pgflineto{\pgfpoint{233.575928pt}{218.808243pt}}
\pgflineto{\pgfpoint{232.075928pt}{217.718430pt}}
\pgfpathclose
\pgfusepath{fill,stroke}
\pgfpathmoveto{\pgfpoint{237.502991pt}{215.955063pt}}
\pgflineto{\pgfpoint{235.430038pt}{218.808243pt}}
\pgflineto{\pgfpoint{233.575928pt}{218.808243pt}}
\pgfpathclose
\pgfusepath{fill,stroke}
\pgfpathmoveto{\pgfpoint{237.502991pt}{215.955063pt}}
\pgflineto{\pgfpoint{236.930038pt}{217.718430pt}}
\pgflineto{\pgfpoint{235.430038pt}{218.808243pt}}
\pgfpathclose
\pgfusepath{fill,stroke}
\pgfpathmoveto{\pgfpoint{202.199844pt}{207.300934pt}}
\pgflineto{\pgfpoint{200.126892pt}{204.447769pt}}
\pgflineto{\pgfpoint{201.626892pt}{205.537582pt}}
\pgfpathclose
\pgfusepath{fill,stroke}
\pgfpathmoveto{\pgfpoint{202.199844pt}{207.300934pt}}
\pgflineto{\pgfpoint{198.272797pt}{204.447769pt}}
\pgflineto{\pgfpoint{200.126892pt}{204.447769pt}}
\pgfpathclose
\pgfusepath{fill,stroke}
\pgfpathmoveto{\pgfpoint{202.199844pt}{207.300934pt}}
\pgflineto{\pgfpoint{196.772797pt}{205.537582pt}}
\pgflineto{\pgfpoint{198.272797pt}{204.447769pt}}
\pgfpathclose
\pgfusepath{fill,stroke}
\pgfpathmoveto{\pgfpoint{202.199844pt}{207.300934pt}}
\pgflineto{\pgfpoint{196.199844pt}{207.300934pt}}
\pgflineto{\pgfpoint{196.772797pt}{205.537582pt}}
\pgfpathclose
\pgfusepath{fill,stroke}
\pgfpathmoveto{\pgfpoint{202.199844pt}{207.300934pt}}
\pgflineto{\pgfpoint{196.772797pt}{209.064301pt}}
\pgflineto{\pgfpoint{196.199844pt}{207.300934pt}}
\pgfpathclose
\pgfusepath{fill,stroke}
\pgfpathmoveto{\pgfpoint{202.199844pt}{207.300934pt}}
\pgflineto{\pgfpoint{198.272797pt}{210.154114pt}}
\pgflineto{\pgfpoint{196.772797pt}{209.064301pt}}
\pgfpathclose
\pgfusepath{fill,stroke}
\pgfpathmoveto{\pgfpoint{202.199844pt}{207.300934pt}}
\pgflineto{\pgfpoint{200.126892pt}{210.154114pt}}
\pgflineto{\pgfpoint{198.272797pt}{210.154114pt}}
\pgfpathclose
\pgfusepath{fill,stroke}
\pgfpathmoveto{\pgfpoint{202.199844pt}{207.300934pt}}
\pgflineto{\pgfpoint{201.626892pt}{209.064301pt}}
\pgflineto{\pgfpoint{200.126892pt}{210.154114pt}}
\pgfpathclose
\pgfusepath{fill,stroke}
\pgfpathmoveto{\pgfpoint{198.848846pt}{260.025635pt}}
\pgflineto{\pgfpoint{196.775909pt}{257.172455pt}}
\pgflineto{\pgfpoint{198.275909pt}{258.262268pt}}
\pgfpathclose
\pgfusepath{fill,stroke}
\pgfpathmoveto{\pgfpoint{198.848846pt}{260.025635pt}}
\pgflineto{\pgfpoint{194.921814pt}{257.172455pt}}
\pgflineto{\pgfpoint{196.775909pt}{257.172455pt}}
\pgfpathclose
\pgfusepath{fill,stroke}
\pgfpathmoveto{\pgfpoint{198.848846pt}{260.025635pt}}
\pgflineto{\pgfpoint{193.421814pt}{258.262268pt}}
\pgflineto{\pgfpoint{194.921814pt}{257.172455pt}}
\pgfpathclose
\pgfusepath{fill,stroke}
\pgfpathmoveto{\pgfpoint{198.848846pt}{260.025635pt}}
\pgflineto{\pgfpoint{192.848862pt}{260.025635pt}}
\pgflineto{\pgfpoint{193.421814pt}{258.262268pt}}
\pgfpathclose
\pgfusepath{fill,stroke}
\pgfpathmoveto{\pgfpoint{198.848846pt}{260.025635pt}}
\pgflineto{\pgfpoint{193.421814pt}{261.789001pt}}
\pgflineto{\pgfpoint{192.848862pt}{260.025635pt}}
\pgfpathclose
\pgfusepath{fill,stroke}
\pgfpathmoveto{\pgfpoint{198.848846pt}{260.025635pt}}
\pgflineto{\pgfpoint{194.921814pt}{262.878815pt}}
\pgflineto{\pgfpoint{193.421814pt}{261.789001pt}}
\pgfpathclose
\pgfusepath{fill,stroke}
\pgfpathmoveto{\pgfpoint{198.848846pt}{260.025635pt}}
\pgflineto{\pgfpoint{196.775909pt}{262.878815pt}}
\pgflineto{\pgfpoint{194.921814pt}{262.878815pt}}
\pgfpathclose
\pgfusepath{fill,stroke}
\pgfpathmoveto{\pgfpoint{198.848846pt}{260.025635pt}}
\pgflineto{\pgfpoint{198.275909pt}{261.789001pt}}
\pgflineto{\pgfpoint{196.775909pt}{262.878815pt}}
\pgfpathclose
\pgfusepath{fill,stroke}
\pgfpathmoveto{\pgfpoint{191.710602pt}{260.576294pt}}
\pgflineto{\pgfpoint{189.637634pt}{257.723114pt}}
\pgflineto{\pgfpoint{191.137650pt}{258.812927pt}}
\pgfpathclose
\pgfusepath{fill,stroke}
\pgfpathmoveto{\pgfpoint{191.710602pt}{260.576294pt}}
\pgflineto{\pgfpoint{187.783539pt}{257.723114pt}}
\pgflineto{\pgfpoint{189.637634pt}{257.723114pt}}
\pgfpathclose
\pgfusepath{fill,stroke}
\pgfpathmoveto{\pgfpoint{191.710602pt}{260.576294pt}}
\pgflineto{\pgfpoint{186.283539pt}{258.812927pt}}
\pgflineto{\pgfpoint{187.783539pt}{257.723114pt}}
\pgfpathclose
\pgfusepath{fill,stroke}
\pgfpathmoveto{\pgfpoint{191.710602pt}{260.576294pt}}
\pgflineto{\pgfpoint{185.710602pt}{260.576294pt}}
\pgflineto{\pgfpoint{186.283539pt}{258.812927pt}}
\pgfpathclose
\pgfusepath{fill,stroke}
\pgfpathmoveto{\pgfpoint{191.710602pt}{260.576294pt}}
\pgflineto{\pgfpoint{186.283539pt}{262.339630pt}}
\pgflineto{\pgfpoint{185.710602pt}{260.576294pt}}
\pgfpathclose
\pgfusepath{fill,stroke}
\pgfpathmoveto{\pgfpoint{191.710602pt}{260.576294pt}}
\pgflineto{\pgfpoint{187.783539pt}{263.429443pt}}
\pgflineto{\pgfpoint{186.283539pt}{262.339630pt}}
\pgfpathclose
\pgfusepath{fill,stroke}
\pgfpathmoveto{\pgfpoint{191.710602pt}{260.576294pt}}
\pgflineto{\pgfpoint{189.637634pt}{263.429443pt}}
\pgflineto{\pgfpoint{187.783539pt}{263.429443pt}}
\pgfpathclose
\pgfusepath{fill,stroke}
\pgfpathmoveto{\pgfpoint{191.710602pt}{260.576294pt}}
\pgflineto{\pgfpoint{191.137650pt}{262.339630pt}}
\pgflineto{\pgfpoint{189.637634pt}{263.429443pt}}
\pgfpathclose
\pgfusepath{fill,stroke}
\pgfpathmoveto{\pgfpoint{229.653473pt}{219.285751pt}}
\pgflineto{\pgfpoint{227.580536pt}{216.432587pt}}
\pgflineto{\pgfpoint{229.080536pt}{217.522400pt}}
\pgfpathclose
\pgfusepath{fill,stroke}
\pgfpathmoveto{\pgfpoint{229.653473pt}{219.285751pt}}
\pgflineto{\pgfpoint{225.726425pt}{216.432587pt}}
\pgflineto{\pgfpoint{227.580536pt}{216.432587pt}}
\pgfpathclose
\pgfusepath{fill,stroke}
\pgfpathmoveto{\pgfpoint{229.653473pt}{219.285751pt}}
\pgflineto{\pgfpoint{224.226425pt}{217.522400pt}}
\pgflineto{\pgfpoint{225.726425pt}{216.432587pt}}
\pgfpathclose
\pgfusepath{fill,stroke}
\pgfpathmoveto{\pgfpoint{229.653473pt}{219.285751pt}}
\pgflineto{\pgfpoint{223.653473pt}{219.285751pt}}
\pgflineto{\pgfpoint{224.226425pt}{217.522400pt}}
\pgfpathclose
\pgfusepath{fill,stroke}
\pgfpathmoveto{\pgfpoint{229.653473pt}{219.285751pt}}
\pgflineto{\pgfpoint{224.226425pt}{221.049103pt}}
\pgflineto{\pgfpoint{223.653473pt}{219.285751pt}}
\pgfpathclose
\pgfusepath{fill,stroke}
\pgfpathmoveto{\pgfpoint{229.653473pt}{219.285751pt}}
\pgflineto{\pgfpoint{225.726425pt}{222.138916pt}}
\pgflineto{\pgfpoint{224.226425pt}{221.049103pt}}
\pgfpathclose
\pgfusepath{fill,stroke}
\pgfpathmoveto{\pgfpoint{229.653473pt}{219.285751pt}}
\pgflineto{\pgfpoint{227.580536pt}{222.138916pt}}
\pgflineto{\pgfpoint{225.726425pt}{222.138916pt}}
\pgfpathclose
\pgfusepath{fill,stroke}
\pgfpathmoveto{\pgfpoint{229.653473pt}{219.285751pt}}
\pgflineto{\pgfpoint{229.080536pt}{221.049103pt}}
\pgflineto{\pgfpoint{227.580536pt}{222.138916pt}}
\pgfpathclose
\pgfusepath{fill,stroke}
\pgfpathmoveto{\pgfpoint{222.554443pt}{202.421112pt}}
\pgflineto{\pgfpoint{220.481491pt}{199.567932pt}}
\pgflineto{\pgfpoint{221.981491pt}{200.657745pt}}
\pgfpathclose
\pgfusepath{fill,stroke}
\pgfpathmoveto{\pgfpoint{222.554443pt}{202.421112pt}}
\pgflineto{\pgfpoint{218.627380pt}{199.567932pt}}
\pgflineto{\pgfpoint{220.481491pt}{199.567932pt}}
\pgfpathclose
\pgfusepath{fill,stroke}
\pgfpathmoveto{\pgfpoint{222.554443pt}{202.421112pt}}
\pgflineto{\pgfpoint{217.127380pt}{200.657745pt}}
\pgflineto{\pgfpoint{218.627380pt}{199.567932pt}}
\pgfpathclose
\pgfusepath{fill,stroke}
\pgfpathmoveto{\pgfpoint{222.554443pt}{202.421112pt}}
\pgflineto{\pgfpoint{216.554443pt}{202.421112pt}}
\pgflineto{\pgfpoint{217.127380pt}{200.657745pt}}
\pgfpathclose
\pgfusepath{fill,stroke}
\pgfpathmoveto{\pgfpoint{222.554443pt}{202.421112pt}}
\pgflineto{\pgfpoint{217.127380pt}{204.184464pt}}
\pgflineto{\pgfpoint{216.554443pt}{202.421112pt}}
\pgfpathclose
\pgfusepath{fill,stroke}
\pgfpathmoveto{\pgfpoint{222.554443pt}{202.421112pt}}
\pgflineto{\pgfpoint{218.627380pt}{205.274277pt}}
\pgflineto{\pgfpoint{217.127380pt}{204.184464pt}}
\pgfpathclose
\pgfusepath{fill,stroke}
\pgfpathmoveto{\pgfpoint{222.554443pt}{202.421112pt}}
\pgflineto{\pgfpoint{220.481491pt}{205.274277pt}}
\pgflineto{\pgfpoint{218.627380pt}{205.274277pt}}
\pgfpathclose
\pgfusepath{fill,stroke}
\pgfpathmoveto{\pgfpoint{222.554443pt}{202.421112pt}}
\pgflineto{\pgfpoint{221.981491pt}{204.184464pt}}
\pgflineto{\pgfpoint{220.481491pt}{205.274277pt}}
\pgfpathclose
\pgfusepath{fill,stroke}
\pgfpathmoveto{\pgfpoint{196.959595pt}{219.771622pt}}
\pgflineto{\pgfpoint{194.886627pt}{216.918442pt}}
\pgflineto{\pgfpoint{196.386642pt}{218.008255pt}}
\pgfpathclose
\pgfusepath{fill,stroke}
\pgfpathmoveto{\pgfpoint{196.959595pt}{219.771622pt}}
\pgflineto{\pgfpoint{193.032532pt}{216.918442pt}}
\pgflineto{\pgfpoint{194.886627pt}{216.918442pt}}
\pgfpathclose
\pgfusepath{fill,stroke}
\pgfpathmoveto{\pgfpoint{196.959595pt}{219.771622pt}}
\pgflineto{\pgfpoint{191.532532pt}{218.008255pt}}
\pgflineto{\pgfpoint{193.032532pt}{216.918442pt}}
\pgfpathclose
\pgfusepath{fill,stroke}
\pgfpathmoveto{\pgfpoint{196.959595pt}{219.771622pt}}
\pgflineto{\pgfpoint{190.959595pt}{219.771622pt}}
\pgflineto{\pgfpoint{191.532532pt}{218.008255pt}}
\pgfpathclose
\pgfusepath{fill,stroke}
\pgfpathmoveto{\pgfpoint{196.959595pt}{219.771622pt}}
\pgflineto{\pgfpoint{191.532532pt}{221.534973pt}}
\pgflineto{\pgfpoint{190.959595pt}{219.771622pt}}
\pgfpathclose
\pgfusepath{fill,stroke}
\pgfpathmoveto{\pgfpoint{196.959595pt}{219.771622pt}}
\pgflineto{\pgfpoint{193.032532pt}{222.624786pt}}
\pgflineto{\pgfpoint{191.532532pt}{221.534973pt}}
\pgfpathclose
\pgfusepath{fill,stroke}
\pgfpathmoveto{\pgfpoint{196.959595pt}{219.771622pt}}
\pgflineto{\pgfpoint{194.886627pt}{222.624786pt}}
\pgflineto{\pgfpoint{193.032532pt}{222.624786pt}}
\pgfpathclose
\pgfusepath{fill,stroke}
\pgfpathmoveto{\pgfpoint{196.959595pt}{219.771622pt}}
\pgflineto{\pgfpoint{196.386642pt}{221.534973pt}}
\pgflineto{\pgfpoint{194.886627pt}{222.624786pt}}
\pgfpathclose
\pgfusepath{fill,stroke}
\pgfpathmoveto{\pgfpoint{211.410690pt}{229.418610pt}}
\pgflineto{\pgfpoint{209.337738pt}{226.565430pt}}
\pgflineto{\pgfpoint{210.837738pt}{227.655258pt}}
\pgfpathclose
\pgfusepath{fill,stroke}
\pgfpathmoveto{\pgfpoint{211.410690pt}{229.418610pt}}
\pgflineto{\pgfpoint{207.483643pt}{226.565430pt}}
\pgflineto{\pgfpoint{209.337738pt}{226.565430pt}}
\pgfpathclose
\pgfusepath{fill,stroke}
\pgfpathmoveto{\pgfpoint{211.410690pt}{229.418610pt}}
\pgflineto{\pgfpoint{205.983643pt}{227.655258pt}}
\pgflineto{\pgfpoint{207.483643pt}{226.565430pt}}
\pgfpathclose
\pgfusepath{fill,stroke}
\pgfpathmoveto{\pgfpoint{211.410690pt}{229.418610pt}}
\pgflineto{\pgfpoint{205.410706pt}{229.418610pt}}
\pgflineto{\pgfpoint{205.983643pt}{227.655258pt}}
\pgfpathclose
\pgfusepath{fill,stroke}
\pgfpathmoveto{\pgfpoint{211.410690pt}{229.418610pt}}
\pgflineto{\pgfpoint{205.983643pt}{231.181961pt}}
\pgflineto{\pgfpoint{205.410706pt}{229.418610pt}}
\pgfpathclose
\pgfusepath{fill,stroke}
\pgfpathmoveto{\pgfpoint{211.410690pt}{229.418610pt}}
\pgflineto{\pgfpoint{207.483643pt}{232.271774pt}}
\pgflineto{\pgfpoint{205.983643pt}{231.181961pt}}
\pgfpathclose
\pgfusepath{fill,stroke}
\pgfpathmoveto{\pgfpoint{211.410690pt}{229.418610pt}}
\pgflineto{\pgfpoint{209.337738pt}{232.271774pt}}
\pgflineto{\pgfpoint{207.483643pt}{232.271774pt}}
\pgfpathclose
\pgfusepath{fill,stroke}
\pgfpathmoveto{\pgfpoint{211.410690pt}{229.418610pt}}
\pgflineto{\pgfpoint{210.837738pt}{231.181961pt}}
\pgflineto{\pgfpoint{209.337738pt}{232.271774pt}}
\pgfpathclose
\pgfusepath{fill,stroke}
\pgfpathmoveto{\pgfpoint{196.339996pt}{261.698029pt}}
\pgflineto{\pgfpoint{194.267044pt}{258.844849pt}}
\pgflineto{\pgfpoint{195.767044pt}{259.934662pt}}
\pgfpathclose
\pgfusepath{fill,stroke}
\pgfpathmoveto{\pgfpoint{196.339996pt}{261.698029pt}}
\pgflineto{\pgfpoint{192.412949pt}{258.844849pt}}
\pgflineto{\pgfpoint{194.267044pt}{258.844849pt}}
\pgfpathclose
\pgfusepath{fill,stroke}
\pgfpathmoveto{\pgfpoint{196.339996pt}{261.698029pt}}
\pgflineto{\pgfpoint{190.912933pt}{259.934662pt}}
\pgflineto{\pgfpoint{192.412949pt}{258.844849pt}}
\pgfpathclose
\pgfusepath{fill,stroke}
\pgfpathmoveto{\pgfpoint{196.339996pt}{261.698029pt}}
\pgflineto{\pgfpoint{190.339996pt}{261.698029pt}}
\pgflineto{\pgfpoint{190.912933pt}{259.934662pt}}
\pgfpathclose
\pgfusepath{fill,stroke}
\pgfpathmoveto{\pgfpoint{196.339996pt}{261.698029pt}}
\pgflineto{\pgfpoint{190.912933pt}{263.461365pt}}
\pgflineto{\pgfpoint{190.339996pt}{261.698029pt}}
\pgfpathclose
\pgfusepath{fill,stroke}
\pgfpathmoveto{\pgfpoint{196.339996pt}{261.698029pt}}
\pgflineto{\pgfpoint{192.412949pt}{264.551178pt}}
\pgflineto{\pgfpoint{190.912933pt}{263.461365pt}}
\pgfpathclose
\pgfusepath{fill,stroke}
\pgfpathmoveto{\pgfpoint{196.339996pt}{261.698029pt}}
\pgflineto{\pgfpoint{194.267044pt}{264.551178pt}}
\pgflineto{\pgfpoint{192.412949pt}{264.551178pt}}
\pgfpathclose
\pgfusepath{fill,stroke}
\pgfpathmoveto{\pgfpoint{196.339996pt}{261.698029pt}}
\pgflineto{\pgfpoint{195.767044pt}{263.461365pt}}
\pgflineto{\pgfpoint{194.267044pt}{264.551178pt}}
\pgfpathclose
\pgfusepath{fill,stroke}
\pgfpathmoveto{\pgfpoint{205.258514pt}{205.075790pt}}
\pgflineto{\pgfpoint{203.185547pt}{202.222626pt}}
\pgflineto{\pgfpoint{204.685562pt}{203.312439pt}}
\pgfpathclose
\pgfusepath{fill,stroke}
\pgfpathmoveto{\pgfpoint{205.258514pt}{205.075790pt}}
\pgflineto{\pgfpoint{201.331451pt}{202.222626pt}}
\pgflineto{\pgfpoint{203.185547pt}{202.222626pt}}
\pgfpathclose
\pgfusepath{fill,stroke}
\pgfpathmoveto{\pgfpoint{205.258514pt}{205.075790pt}}
\pgflineto{\pgfpoint{199.831451pt}{203.312439pt}}
\pgflineto{\pgfpoint{201.331451pt}{202.222626pt}}
\pgfpathclose
\pgfusepath{fill,stroke}
\pgfpathmoveto{\pgfpoint{205.258514pt}{205.075790pt}}
\pgflineto{\pgfpoint{199.258514pt}{205.075790pt}}
\pgflineto{\pgfpoint{199.831451pt}{203.312439pt}}
\pgfpathclose
\pgfusepath{fill,stroke}
\pgfpathmoveto{\pgfpoint{205.258514pt}{205.075790pt}}
\pgflineto{\pgfpoint{199.831451pt}{206.839157pt}}
\pgflineto{\pgfpoint{199.258514pt}{205.075790pt}}
\pgfpathclose
\pgfusepath{fill,stroke}
\pgfpathmoveto{\pgfpoint{205.258514pt}{205.075790pt}}
\pgflineto{\pgfpoint{201.331451pt}{207.928970pt}}
\pgflineto{\pgfpoint{199.831451pt}{206.839157pt}}
\pgfpathclose
\pgfusepath{fill,stroke}
\pgfpathmoveto{\pgfpoint{205.258514pt}{205.075790pt}}
\pgflineto{\pgfpoint{203.185547pt}{207.928970pt}}
\pgflineto{\pgfpoint{201.331451pt}{207.928970pt}}
\pgfpathclose
\pgfusepath{fill,stroke}
\pgfpathmoveto{\pgfpoint{205.258514pt}{205.075790pt}}
\pgflineto{\pgfpoint{204.685562pt}{206.839157pt}}
\pgflineto{\pgfpoint{203.185547pt}{207.928970pt}}
\pgfpathclose
\pgfusepath{fill,stroke}
\pgfpathmoveto{\pgfpoint{223.431458pt}{232.059204pt}}
\pgflineto{\pgfpoint{221.358505pt}{229.206024pt}}
\pgflineto{\pgfpoint{222.858505pt}{230.295837pt}}
\pgfpathclose
\pgfusepath{fill,stroke}
\pgfpathmoveto{\pgfpoint{223.431458pt}{232.059204pt}}
\pgflineto{\pgfpoint{219.504395pt}{229.206024pt}}
\pgflineto{\pgfpoint{221.358505pt}{229.206024pt}}
\pgfpathclose
\pgfusepath{fill,stroke}
\pgfpathmoveto{\pgfpoint{223.431458pt}{232.059204pt}}
\pgflineto{\pgfpoint{218.004395pt}{230.295837pt}}
\pgflineto{\pgfpoint{219.504395pt}{229.206024pt}}
\pgfpathclose
\pgfusepath{fill,stroke}
\pgfpathmoveto{\pgfpoint{223.431458pt}{232.059204pt}}
\pgflineto{\pgfpoint{217.431458pt}{232.059204pt}}
\pgflineto{\pgfpoint{218.004395pt}{230.295837pt}}
\pgfpathclose
\pgfusepath{fill,stroke}
\pgfpathmoveto{\pgfpoint{223.431458pt}{232.059204pt}}
\pgflineto{\pgfpoint{218.004395pt}{233.822556pt}}
\pgflineto{\pgfpoint{217.431458pt}{232.059204pt}}
\pgfpathclose
\pgfusepath{fill,stroke}
\pgfpathmoveto{\pgfpoint{223.431458pt}{232.059204pt}}
\pgflineto{\pgfpoint{219.504395pt}{234.912369pt}}
\pgflineto{\pgfpoint{218.004395pt}{233.822556pt}}
\pgfpathclose
\pgfusepath{fill,stroke}
\pgfpathmoveto{\pgfpoint{223.431458pt}{232.059204pt}}
\pgflineto{\pgfpoint{221.358505pt}{234.912369pt}}
\pgflineto{\pgfpoint{219.504395pt}{234.912369pt}}
\pgfpathclose
\pgfusepath{fill,stroke}
\pgfpathmoveto{\pgfpoint{223.431458pt}{232.059204pt}}
\pgflineto{\pgfpoint{222.858505pt}{233.822556pt}}
\pgflineto{\pgfpoint{221.358505pt}{234.912369pt}}
\pgfpathclose
\pgfusepath{fill,stroke}
\pgfpathmoveto{\pgfpoint{256.094818pt}{231.305756pt}}
\pgflineto{\pgfpoint{254.021881pt}{228.452576pt}}
\pgflineto{\pgfpoint{255.521881pt}{229.542404pt}}
\pgfpathclose
\pgfusepath{fill,stroke}
\pgfpathmoveto{\pgfpoint{256.094818pt}{231.305756pt}}
\pgflineto{\pgfpoint{252.167770pt}{228.452576pt}}
\pgflineto{\pgfpoint{254.021881pt}{228.452576pt}}
\pgfpathclose
\pgfusepath{fill,stroke}
\pgfpathmoveto{\pgfpoint{256.094818pt}{231.305756pt}}
\pgflineto{\pgfpoint{250.667770pt}{229.542404pt}}
\pgflineto{\pgfpoint{252.167770pt}{228.452576pt}}
\pgfpathclose
\pgfusepath{fill,stroke}
\pgfpathmoveto{\pgfpoint{256.094818pt}{231.305756pt}}
\pgflineto{\pgfpoint{250.094818pt}{231.305756pt}}
\pgflineto{\pgfpoint{250.667770pt}{229.542404pt}}
\pgfpathclose
\pgfusepath{fill,stroke}
\pgfpathmoveto{\pgfpoint{256.094818pt}{231.305756pt}}
\pgflineto{\pgfpoint{250.667770pt}{233.069107pt}}
\pgflineto{\pgfpoint{250.094818pt}{231.305756pt}}
\pgfpathclose
\pgfusepath{fill,stroke}
\pgfpathmoveto{\pgfpoint{256.094818pt}{231.305756pt}}
\pgflineto{\pgfpoint{252.167770pt}{234.158920pt}}
\pgflineto{\pgfpoint{250.667770pt}{233.069107pt}}
\pgfpathclose
\pgfusepath{fill,stroke}
\pgfpathmoveto{\pgfpoint{256.094818pt}{231.305756pt}}
\pgflineto{\pgfpoint{254.021881pt}{234.158920pt}}
\pgflineto{\pgfpoint{252.167770pt}{234.158920pt}}
\pgfpathclose
\pgfusepath{fill,stroke}
\pgfpathmoveto{\pgfpoint{256.094818pt}{231.305756pt}}
\pgflineto{\pgfpoint{255.521881pt}{233.069107pt}}
\pgflineto{\pgfpoint{254.021881pt}{234.158920pt}}
\pgfpathclose
\pgfusepath{fill,stroke}
\pgfpathmoveto{\pgfpoint{214.604584pt}{211.525894pt}}
\pgflineto{\pgfpoint{212.531647pt}{208.672729pt}}
\pgflineto{\pgfpoint{214.031647pt}{209.762543pt}}
\pgfpathclose
\pgfusepath{fill,stroke}
\pgfpathmoveto{\pgfpoint{214.604584pt}{211.525894pt}}
\pgflineto{\pgfpoint{210.677551pt}{208.672729pt}}
\pgflineto{\pgfpoint{212.531647pt}{208.672729pt}}
\pgfpathclose
\pgfusepath{fill,stroke}
\pgfpathmoveto{\pgfpoint{214.604584pt}{211.525894pt}}
\pgflineto{\pgfpoint{209.177551pt}{209.762543pt}}
\pgflineto{\pgfpoint{210.677551pt}{208.672729pt}}
\pgfpathclose
\pgfusepath{fill,stroke}
\pgfpathmoveto{\pgfpoint{214.604584pt}{211.525894pt}}
\pgflineto{\pgfpoint{208.604599pt}{211.525894pt}}
\pgflineto{\pgfpoint{209.177551pt}{209.762543pt}}
\pgfpathclose
\pgfusepath{fill,stroke}
\pgfpathmoveto{\pgfpoint{214.604584pt}{211.525894pt}}
\pgflineto{\pgfpoint{209.177551pt}{213.289246pt}}
\pgflineto{\pgfpoint{208.604599pt}{211.525894pt}}
\pgfpathclose
\pgfusepath{fill,stroke}
\pgfpathmoveto{\pgfpoint{214.604584pt}{211.525894pt}}
\pgflineto{\pgfpoint{210.677551pt}{214.379059pt}}
\pgflineto{\pgfpoint{209.177551pt}{213.289246pt}}
\pgfpathclose
\pgfusepath{fill,stroke}
\pgfpathmoveto{\pgfpoint{214.604584pt}{211.525894pt}}
\pgflineto{\pgfpoint{212.531647pt}{214.379059pt}}
\pgflineto{\pgfpoint{210.677551pt}{214.379059pt}}
\pgfpathclose
\pgfusepath{fill,stroke}
\pgfpathmoveto{\pgfpoint{214.604584pt}{211.525894pt}}
\pgflineto{\pgfpoint{214.031647pt}{213.289246pt}}
\pgflineto{\pgfpoint{212.531647pt}{214.379059pt}}
\pgfpathclose
\pgfusepath{fill,stroke}
\pgfpathmoveto{\pgfpoint{209.307602pt}{226.390717pt}}
\pgflineto{\pgfpoint{207.234650pt}{223.537552pt}}
\pgflineto{\pgfpoint{208.734650pt}{224.627365pt}}
\pgfpathclose
\pgfusepath{fill,stroke}
\pgfpathmoveto{\pgfpoint{209.307602pt}{226.390717pt}}
\pgflineto{\pgfpoint{205.380554pt}{223.537552pt}}
\pgflineto{\pgfpoint{207.234650pt}{223.537552pt}}
\pgfpathclose
\pgfusepath{fill,stroke}
\pgfpathmoveto{\pgfpoint{209.307602pt}{226.390717pt}}
\pgflineto{\pgfpoint{203.880554pt}{224.627365pt}}
\pgflineto{\pgfpoint{205.380554pt}{223.537552pt}}
\pgfpathclose
\pgfusepath{fill,stroke}
\pgfpathmoveto{\pgfpoint{209.307602pt}{226.390717pt}}
\pgflineto{\pgfpoint{203.307617pt}{226.390717pt}}
\pgflineto{\pgfpoint{203.880554pt}{224.627365pt}}
\pgfpathclose
\pgfusepath{fill,stroke}
\pgfpathmoveto{\pgfpoint{209.307602pt}{226.390717pt}}
\pgflineto{\pgfpoint{203.880554pt}{228.154068pt}}
\pgflineto{\pgfpoint{203.307617pt}{226.390717pt}}
\pgfpathclose
\pgfusepath{fill,stroke}
\pgfpathmoveto{\pgfpoint{209.307602pt}{226.390717pt}}
\pgflineto{\pgfpoint{205.380554pt}{229.243896pt}}
\pgflineto{\pgfpoint{203.880554pt}{228.154068pt}}
\pgfpathclose
\pgfusepath{fill,stroke}
\pgfpathmoveto{\pgfpoint{209.307602pt}{226.390717pt}}
\pgflineto{\pgfpoint{207.234650pt}{229.243896pt}}
\pgflineto{\pgfpoint{205.380554pt}{229.243896pt}}
\pgfpathclose
\pgfusepath{fill,stroke}
\pgfpathmoveto{\pgfpoint{209.307602pt}{226.390717pt}}
\pgflineto{\pgfpoint{208.734650pt}{228.154068pt}}
\pgflineto{\pgfpoint{207.234650pt}{229.243896pt}}
\pgfpathclose
\pgfusepath{fill,stroke}
\pgfpathmoveto{\pgfpoint{197.256287pt}{190.309570pt}}
\pgflineto{\pgfpoint{195.183334pt}{187.456406pt}}
\pgflineto{\pgfpoint{196.683350pt}{188.546219pt}}
\pgfpathclose
\pgfusepath{fill,stroke}
\pgfpathmoveto{\pgfpoint{197.256287pt}{190.309570pt}}
\pgflineto{\pgfpoint{193.329239pt}{187.456406pt}}
\pgflineto{\pgfpoint{195.183334pt}{187.456406pt}}
\pgfpathclose
\pgfusepath{fill,stroke}
\pgfpathmoveto{\pgfpoint{197.256287pt}{190.309570pt}}
\pgflineto{\pgfpoint{191.829239pt}{188.546219pt}}
\pgflineto{\pgfpoint{193.329239pt}{187.456406pt}}
\pgfpathclose
\pgfusepath{fill,stroke}
\pgfpathmoveto{\pgfpoint{197.256287pt}{190.309570pt}}
\pgflineto{\pgfpoint{191.256287pt}{190.309570pt}}
\pgflineto{\pgfpoint{191.829239pt}{188.546219pt}}
\pgfpathclose
\pgfusepath{fill,stroke}
\pgfpathmoveto{\pgfpoint{197.256287pt}{190.309570pt}}
\pgflineto{\pgfpoint{191.829239pt}{192.072937pt}}
\pgflineto{\pgfpoint{191.256287pt}{190.309570pt}}
\pgfpathclose
\pgfusepath{fill,stroke}
\pgfpathmoveto{\pgfpoint{197.256287pt}{190.309570pt}}
\pgflineto{\pgfpoint{193.329239pt}{193.162750pt}}
\pgflineto{\pgfpoint{191.829239pt}{192.072937pt}}
\pgfpathclose
\pgfusepath{fill,stroke}
\pgfpathmoveto{\pgfpoint{197.256287pt}{190.309570pt}}
\pgflineto{\pgfpoint{195.183334pt}{193.162750pt}}
\pgflineto{\pgfpoint{193.329239pt}{193.162750pt}}
\pgfpathclose
\pgfusepath{fill,stroke}
\pgfpathmoveto{\pgfpoint{197.256287pt}{190.309570pt}}
\pgflineto{\pgfpoint{196.683350pt}{192.072937pt}}
\pgflineto{\pgfpoint{195.183334pt}{193.162750pt}}
\pgfpathclose
\pgfusepath{fill,stroke}
\pgfpathmoveto{\pgfpoint{205.110138pt}{231.319839pt}}
\pgflineto{\pgfpoint{203.037186pt}{228.466675pt}}
\pgflineto{\pgfpoint{204.537201pt}{229.556488pt}}
\pgfpathclose
\pgfusepath{fill,stroke}
\pgfpathmoveto{\pgfpoint{205.110138pt}{231.319839pt}}
\pgflineto{\pgfpoint{201.183090pt}{228.466675pt}}
\pgflineto{\pgfpoint{203.037186pt}{228.466675pt}}
\pgfpathclose
\pgfusepath{fill,stroke}
\pgfpathmoveto{\pgfpoint{205.110138pt}{231.319839pt}}
\pgflineto{\pgfpoint{199.683090pt}{229.556488pt}}
\pgflineto{\pgfpoint{201.183090pt}{228.466675pt}}
\pgfpathclose
\pgfusepath{fill,stroke}
\pgfpathmoveto{\pgfpoint{205.110138pt}{231.319839pt}}
\pgflineto{\pgfpoint{199.110138pt}{231.319839pt}}
\pgflineto{\pgfpoint{199.683090pt}{229.556488pt}}
\pgfpathclose
\pgfusepath{fill,stroke}
\pgfpathmoveto{\pgfpoint{205.110138pt}{231.319839pt}}
\pgflineto{\pgfpoint{199.683090pt}{233.083191pt}}
\pgflineto{\pgfpoint{199.110138pt}{231.319839pt}}
\pgfpathclose
\pgfusepath{fill,stroke}
\pgfpathmoveto{\pgfpoint{205.110138pt}{231.319839pt}}
\pgflineto{\pgfpoint{201.183090pt}{234.173004pt}}
\pgflineto{\pgfpoint{199.683090pt}{233.083191pt}}
\pgfpathclose
\pgfusepath{fill,stroke}
\pgfpathmoveto{\pgfpoint{205.110138pt}{231.319839pt}}
\pgflineto{\pgfpoint{203.037186pt}{234.173004pt}}
\pgflineto{\pgfpoint{201.183090pt}{234.173004pt}}
\pgfpathclose
\pgfusepath{fill,stroke}
\pgfpathmoveto{\pgfpoint{205.110138pt}{231.319839pt}}
\pgflineto{\pgfpoint{204.537201pt}{233.083191pt}}
\pgflineto{\pgfpoint{203.037186pt}{234.173004pt}}
\pgfpathclose
\pgfusepath{fill,stroke}
\pgfpathmoveto{\pgfpoint{212.641129pt}{205.251846pt}}
\pgflineto{\pgfpoint{210.568176pt}{202.398682pt}}
\pgflineto{\pgfpoint{212.068176pt}{203.488495pt}}
\pgfpathclose
\pgfusepath{fill,stroke}
\pgfpathmoveto{\pgfpoint{212.641129pt}{205.251846pt}}
\pgflineto{\pgfpoint{208.714081pt}{202.398682pt}}
\pgflineto{\pgfpoint{210.568176pt}{202.398682pt}}
\pgfpathclose
\pgfusepath{fill,stroke}
\pgfpathmoveto{\pgfpoint{212.641129pt}{205.251846pt}}
\pgflineto{\pgfpoint{207.214081pt}{203.488495pt}}
\pgflineto{\pgfpoint{208.714081pt}{202.398682pt}}
\pgfpathclose
\pgfusepath{fill,stroke}
\pgfpathmoveto{\pgfpoint{212.641129pt}{205.251846pt}}
\pgflineto{\pgfpoint{206.641144pt}{205.251846pt}}
\pgflineto{\pgfpoint{207.214081pt}{203.488495pt}}
\pgfpathclose
\pgfusepath{fill,stroke}
\pgfpathmoveto{\pgfpoint{212.641129pt}{205.251846pt}}
\pgflineto{\pgfpoint{207.214081pt}{207.015198pt}}
\pgflineto{\pgfpoint{206.641144pt}{205.251846pt}}
\pgfpathclose
\pgfusepath{fill,stroke}
\pgfpathmoveto{\pgfpoint{212.641129pt}{205.251846pt}}
\pgflineto{\pgfpoint{208.714081pt}{208.105011pt}}
\pgflineto{\pgfpoint{207.214081pt}{207.015198pt}}
\pgfpathclose
\pgfusepath{fill,stroke}
\pgfpathmoveto{\pgfpoint{212.641129pt}{205.251846pt}}
\pgflineto{\pgfpoint{210.568176pt}{208.105011pt}}
\pgflineto{\pgfpoint{208.714081pt}{208.105011pt}}
\pgfpathclose
\pgfusepath{fill,stroke}
\pgfpathmoveto{\pgfpoint{212.641129pt}{205.251846pt}}
\pgflineto{\pgfpoint{212.068176pt}{207.015198pt}}
\pgflineto{\pgfpoint{210.568176pt}{208.105011pt}}
\pgfpathclose
\pgfusepath{fill,stroke}
\pgfpathmoveto{\pgfpoint{201.170105pt}{222.989624pt}}
\pgflineto{\pgfpoint{199.097168pt}{220.136459pt}}
\pgflineto{\pgfpoint{200.597168pt}{221.226273pt}}
\pgfpathclose
\pgfusepath{fill,stroke}
\pgfpathmoveto{\pgfpoint{201.170105pt}{222.989624pt}}
\pgflineto{\pgfpoint{197.243073pt}{220.136459pt}}
\pgflineto{\pgfpoint{199.097168pt}{220.136459pt}}
\pgfpathclose
\pgfusepath{fill,stroke}
\pgfpathmoveto{\pgfpoint{201.170105pt}{222.989624pt}}
\pgflineto{\pgfpoint{195.743073pt}{221.226273pt}}
\pgflineto{\pgfpoint{197.243073pt}{220.136459pt}}
\pgfpathclose
\pgfusepath{fill,stroke}
\pgfpathmoveto{\pgfpoint{201.170105pt}{222.989624pt}}
\pgflineto{\pgfpoint{195.170120pt}{222.989624pt}}
\pgflineto{\pgfpoint{195.743073pt}{221.226273pt}}
\pgfpathclose
\pgfusepath{fill,stroke}
\pgfpathmoveto{\pgfpoint{201.170105pt}{222.989624pt}}
\pgflineto{\pgfpoint{195.743073pt}{224.752975pt}}
\pgflineto{\pgfpoint{195.170120pt}{222.989624pt}}
\pgfpathclose
\pgfusepath{fill,stroke}
\pgfpathmoveto{\pgfpoint{201.170105pt}{222.989624pt}}
\pgflineto{\pgfpoint{197.243073pt}{225.842789pt}}
\pgflineto{\pgfpoint{195.743073pt}{224.752975pt}}
\pgfpathclose
\pgfusepath{fill,stroke}
\pgfpathmoveto{\pgfpoint{201.170105pt}{222.989624pt}}
\pgflineto{\pgfpoint{199.097168pt}{225.842789pt}}
\pgflineto{\pgfpoint{197.243073pt}{225.842789pt}}
\pgfpathclose
\pgfusepath{fill,stroke}
\pgfpathmoveto{\pgfpoint{201.170105pt}{222.989624pt}}
\pgflineto{\pgfpoint{200.597168pt}{224.752975pt}}
\pgflineto{\pgfpoint{199.097168pt}{225.842789pt}}
\pgfpathclose
\pgfusepath{fill,stroke}
\pgfpathmoveto{\pgfpoint{177.394714pt}{227.686371pt}}
\pgflineto{\pgfpoint{175.321777pt}{224.833191pt}}
\pgflineto{\pgfpoint{176.821777pt}{225.923004pt}}
\pgfpathclose
\pgfusepath{fill,stroke}
\pgfpathmoveto{\pgfpoint{177.394714pt}{227.686371pt}}
\pgflineto{\pgfpoint{173.467682pt}{224.833191pt}}
\pgflineto{\pgfpoint{175.321777pt}{224.833191pt}}
\pgfpathclose
\pgfusepath{fill,stroke}
\pgfpathmoveto{\pgfpoint{177.394714pt}{227.686371pt}}
\pgflineto{\pgfpoint{171.967682pt}{225.923004pt}}
\pgflineto{\pgfpoint{173.467682pt}{224.833191pt}}
\pgfpathclose
\pgfusepath{fill,stroke}
\pgfpathmoveto{\pgfpoint{177.394714pt}{227.686371pt}}
\pgflineto{\pgfpoint{171.394730pt}{227.686371pt}}
\pgflineto{\pgfpoint{171.967682pt}{225.923004pt}}
\pgfpathclose
\pgfusepath{fill,stroke}
\pgfpathmoveto{\pgfpoint{177.394714pt}{227.686371pt}}
\pgflineto{\pgfpoint{171.967682pt}{229.449722pt}}
\pgflineto{\pgfpoint{171.394730pt}{227.686371pt}}
\pgfpathclose
\pgfusepath{fill,stroke}
\pgfpathmoveto{\pgfpoint{177.394714pt}{227.686371pt}}
\pgflineto{\pgfpoint{173.467682pt}{230.539536pt}}
\pgflineto{\pgfpoint{171.967682pt}{229.449722pt}}
\pgfpathclose
\pgfusepath{fill,stroke}
\pgfpathmoveto{\pgfpoint{177.394714pt}{227.686371pt}}
\pgflineto{\pgfpoint{175.321777pt}{230.539536pt}}
\pgflineto{\pgfpoint{173.467682pt}{230.539536pt}}
\pgfpathclose
\pgfusepath{fill,stroke}
\pgfpathmoveto{\pgfpoint{177.394714pt}{227.686371pt}}
\pgflineto{\pgfpoint{176.821777pt}{229.449722pt}}
\pgflineto{\pgfpoint{175.321777pt}{230.539536pt}}
\pgfpathclose
\pgfusepath{fill,stroke}
\pgfpathmoveto{\pgfpoint{180.745697pt}{238.100906pt}}
\pgflineto{\pgfpoint{178.672760pt}{235.247742pt}}
\pgflineto{\pgfpoint{180.172760pt}{236.337555pt}}
\pgfpathclose
\pgfusepath{fill,stroke}
\pgfpathmoveto{\pgfpoint{180.745697pt}{238.100906pt}}
\pgflineto{\pgfpoint{176.818665pt}{235.247742pt}}
\pgflineto{\pgfpoint{178.672760pt}{235.247742pt}}
\pgfpathclose
\pgfusepath{fill,stroke}
\pgfpathmoveto{\pgfpoint{180.745697pt}{238.100906pt}}
\pgflineto{\pgfpoint{175.318665pt}{236.337555pt}}
\pgflineto{\pgfpoint{176.818665pt}{235.247742pt}}
\pgfpathclose
\pgfusepath{fill,stroke}
\pgfpathmoveto{\pgfpoint{180.745697pt}{238.100906pt}}
\pgflineto{\pgfpoint{174.745712pt}{238.100906pt}}
\pgflineto{\pgfpoint{175.318665pt}{236.337555pt}}
\pgfpathclose
\pgfusepath{fill,stroke}
\pgfpathmoveto{\pgfpoint{180.745697pt}{238.100906pt}}
\pgflineto{\pgfpoint{175.318665pt}{239.864258pt}}
\pgflineto{\pgfpoint{174.745712pt}{238.100906pt}}
\pgfpathclose
\pgfusepath{fill,stroke}
\pgfpathmoveto{\pgfpoint{180.745697pt}{238.100906pt}}
\pgflineto{\pgfpoint{176.818665pt}{240.954071pt}}
\pgflineto{\pgfpoint{175.318665pt}{239.864258pt}}
\pgfpathclose
\pgfusepath{fill,stroke}
\pgfpathmoveto{\pgfpoint{180.745697pt}{238.100906pt}}
\pgflineto{\pgfpoint{178.672760pt}{240.954071pt}}
\pgflineto{\pgfpoint{176.818665pt}{240.954071pt}}
\pgfpathclose
\pgfusepath{fill,stroke}
\pgfpathmoveto{\pgfpoint{180.745697pt}{238.100906pt}}
\pgflineto{\pgfpoint{180.172760pt}{239.864258pt}}
\pgflineto{\pgfpoint{178.672760pt}{240.954071pt}}
\pgfpathclose
\pgfusepath{fill,stroke}
\pgfpathmoveto{\pgfpoint{208.513489pt}{226.813217pt}}
\pgflineto{\pgfpoint{206.440552pt}{223.960037pt}}
\pgflineto{\pgfpoint{207.940552pt}{225.049850pt}}
\pgfpathclose
\pgfusepath{fill,stroke}
\pgfpathmoveto{\pgfpoint{208.513489pt}{226.813217pt}}
\pgflineto{\pgfpoint{204.586456pt}{223.960037pt}}
\pgflineto{\pgfpoint{206.440552pt}{223.960037pt}}
\pgfpathclose
\pgfusepath{fill,stroke}
\pgfpathmoveto{\pgfpoint{208.513489pt}{226.813217pt}}
\pgflineto{\pgfpoint{203.086456pt}{225.049850pt}}
\pgflineto{\pgfpoint{204.586456pt}{223.960037pt}}
\pgfpathclose
\pgfusepath{fill,stroke}
\pgfpathmoveto{\pgfpoint{208.513489pt}{226.813217pt}}
\pgflineto{\pgfpoint{202.513504pt}{226.813217pt}}
\pgflineto{\pgfpoint{203.086456pt}{225.049850pt}}
\pgfpathclose
\pgfusepath{fill,stroke}
\pgfpathmoveto{\pgfpoint{208.513489pt}{226.813217pt}}
\pgflineto{\pgfpoint{203.086456pt}{228.576569pt}}
\pgflineto{\pgfpoint{202.513504pt}{226.813217pt}}
\pgfpathclose
\pgfusepath{fill,stroke}
\pgfpathmoveto{\pgfpoint{208.513489pt}{226.813217pt}}
\pgflineto{\pgfpoint{204.586456pt}{229.666382pt}}
\pgflineto{\pgfpoint{203.086456pt}{228.576569pt}}
\pgfpathclose
\pgfusepath{fill,stroke}
\pgfpathmoveto{\pgfpoint{208.513489pt}{226.813217pt}}
\pgflineto{\pgfpoint{206.440552pt}{229.666382pt}}
\pgflineto{\pgfpoint{204.586456pt}{229.666382pt}}
\pgfpathclose
\pgfusepath{fill,stroke}
\pgfpathmoveto{\pgfpoint{208.513489pt}{226.813217pt}}
\pgflineto{\pgfpoint{207.940552pt}{228.576569pt}}
\pgflineto{\pgfpoint{206.440552pt}{229.666382pt}}
\pgfpathclose
\pgfusepath{fill,stroke}
\pgfpathmoveto{\pgfpoint{214.691864pt}{203.209778pt}}
\pgflineto{\pgfpoint{212.618912pt}{200.356613pt}}
\pgflineto{\pgfpoint{214.118927pt}{201.446426pt}}
\pgfpathclose
\pgfusepath{fill,stroke}
\pgfpathmoveto{\pgfpoint{214.691864pt}{203.209778pt}}
\pgflineto{\pgfpoint{210.764816pt}{200.356613pt}}
\pgflineto{\pgfpoint{212.618912pt}{200.356613pt}}
\pgfpathclose
\pgfusepath{fill,stroke}
\pgfpathmoveto{\pgfpoint{214.691864pt}{203.209778pt}}
\pgflineto{\pgfpoint{209.264816pt}{201.446426pt}}
\pgflineto{\pgfpoint{210.764816pt}{200.356613pt}}
\pgfpathclose
\pgfusepath{fill,stroke}
\pgfpathmoveto{\pgfpoint{214.691864pt}{203.209778pt}}
\pgflineto{\pgfpoint{208.691864pt}{203.209778pt}}
\pgflineto{\pgfpoint{209.264816pt}{201.446426pt}}
\pgfpathclose
\pgfusepath{fill,stroke}
\pgfpathmoveto{\pgfpoint{214.691864pt}{203.209778pt}}
\pgflineto{\pgfpoint{209.264816pt}{204.973129pt}}
\pgflineto{\pgfpoint{208.691864pt}{203.209778pt}}
\pgfpathclose
\pgfusepath{fill,stroke}
\pgfpathmoveto{\pgfpoint{214.691864pt}{203.209778pt}}
\pgflineto{\pgfpoint{210.764816pt}{206.062943pt}}
\pgflineto{\pgfpoint{209.264816pt}{204.973129pt}}
\pgfpathclose
\pgfusepath{fill,stroke}
\pgfpathmoveto{\pgfpoint{214.691864pt}{203.209778pt}}
\pgflineto{\pgfpoint{212.618912pt}{206.062943pt}}
\pgflineto{\pgfpoint{210.764816pt}{206.062943pt}}
\pgfpathclose
\pgfusepath{fill,stroke}
\pgfpathmoveto{\pgfpoint{214.691864pt}{203.209778pt}}
\pgflineto{\pgfpoint{214.118927pt}{204.973129pt}}
\pgflineto{\pgfpoint{212.618912pt}{206.062943pt}}
\pgfpathclose
\pgfusepath{fill,stroke}
\pgfpathmoveto{\pgfpoint{218.775864pt}{212.427216pt}}
\pgflineto{\pgfpoint{216.702911pt}{209.574051pt}}
\pgflineto{\pgfpoint{218.202911pt}{210.663864pt}}
\pgfpathclose
\pgfusepath{fill,stroke}
\pgfpathmoveto{\pgfpoint{218.775864pt}{212.427216pt}}
\pgflineto{\pgfpoint{214.848816pt}{209.574051pt}}
\pgflineto{\pgfpoint{216.702911pt}{209.574051pt}}
\pgfpathclose
\pgfusepath{fill,stroke}
\pgfpathmoveto{\pgfpoint{218.775864pt}{212.427216pt}}
\pgflineto{\pgfpoint{213.348816pt}{210.663864pt}}
\pgflineto{\pgfpoint{214.848816pt}{209.574051pt}}
\pgfpathclose
\pgfusepath{fill,stroke}
\pgfpathmoveto{\pgfpoint{218.775864pt}{212.427216pt}}
\pgflineto{\pgfpoint{212.775864pt}{212.427216pt}}
\pgflineto{\pgfpoint{213.348816pt}{210.663864pt}}
\pgfpathclose
\pgfusepath{fill,stroke}
\pgfpathmoveto{\pgfpoint{218.775864pt}{212.427216pt}}
\pgflineto{\pgfpoint{213.348816pt}{214.190582pt}}
\pgflineto{\pgfpoint{212.775864pt}{212.427216pt}}
\pgfpathclose
\pgfusepath{fill,stroke}
\pgfpathmoveto{\pgfpoint{218.775864pt}{212.427216pt}}
\pgflineto{\pgfpoint{214.848816pt}{215.280396pt}}
\pgflineto{\pgfpoint{213.348816pt}{214.190582pt}}
\pgfpathclose
\pgfusepath{fill,stroke}
\pgfpathmoveto{\pgfpoint{218.775864pt}{212.427216pt}}
\pgflineto{\pgfpoint{216.702911pt}{215.280396pt}}
\pgflineto{\pgfpoint{214.848816pt}{215.280396pt}}
\pgfpathclose
\pgfusepath{fill,stroke}
\pgfpathmoveto{\pgfpoint{218.775864pt}{212.427216pt}}
\pgflineto{\pgfpoint{218.202911pt}{214.190582pt}}
\pgflineto{\pgfpoint{216.702911pt}{215.280396pt}}
\pgfpathclose
\pgfusepath{fill,stroke}
\pgfpathmoveto{\pgfpoint{239.793686pt}{178.873993pt}}
\pgflineto{\pgfpoint{237.720734pt}{176.020828pt}}
\pgflineto{\pgfpoint{239.220734pt}{177.110641pt}}
\pgfpathclose
\pgfusepath{fill,stroke}
\pgfpathmoveto{\pgfpoint{239.793686pt}{178.873993pt}}
\pgflineto{\pgfpoint{235.866638pt}{176.020828pt}}
\pgflineto{\pgfpoint{237.720734pt}{176.020828pt}}
\pgfpathclose
\pgfusepath{fill,stroke}
\pgfpathmoveto{\pgfpoint{239.793686pt}{178.873993pt}}
\pgflineto{\pgfpoint{234.366638pt}{177.110641pt}}
\pgflineto{\pgfpoint{235.866638pt}{176.020828pt}}
\pgfpathclose
\pgfusepath{fill,stroke}
\pgfpathmoveto{\pgfpoint{239.793686pt}{178.873993pt}}
\pgflineto{\pgfpoint{233.793686pt}{178.873993pt}}
\pgflineto{\pgfpoint{234.366638pt}{177.110641pt}}
\pgfpathclose
\pgfusepath{fill,stroke}
\pgfpathmoveto{\pgfpoint{239.793686pt}{178.873993pt}}
\pgflineto{\pgfpoint{234.366638pt}{180.637360pt}}
\pgflineto{\pgfpoint{233.793686pt}{178.873993pt}}
\pgfpathclose
\pgfusepath{fill,stroke}
\pgfpathmoveto{\pgfpoint{239.793686pt}{178.873993pt}}
\pgflineto{\pgfpoint{235.866638pt}{181.727173pt}}
\pgflineto{\pgfpoint{234.366638pt}{180.637360pt}}
\pgfpathclose
\pgfusepath{fill,stroke}
\pgfpathmoveto{\pgfpoint{239.793686pt}{178.873993pt}}
\pgflineto{\pgfpoint{237.720734pt}{181.727173pt}}
\pgflineto{\pgfpoint{235.866638pt}{181.727173pt}}
\pgfpathclose
\pgfusepath{fill,stroke}
\pgfpathmoveto{\pgfpoint{239.793686pt}{178.873993pt}}
\pgflineto{\pgfpoint{239.220734pt}{180.637360pt}}
\pgflineto{\pgfpoint{237.720734pt}{181.727173pt}}
\pgfpathclose
\pgfusepath{fill,stroke}
\pgfpathmoveto{\pgfpoint{155.569717pt}{281.206055pt}}
\pgflineto{\pgfpoint{153.496765pt}{278.352905pt}}
\pgflineto{\pgfpoint{154.996765pt}{279.442688pt}}
\pgfpathclose
\pgfusepath{fill,stroke}
\pgfpathmoveto{\pgfpoint{155.569717pt}{281.206055pt}}
\pgflineto{\pgfpoint{151.642670pt}{278.352905pt}}
\pgflineto{\pgfpoint{153.496765pt}{278.352905pt}}
\pgfpathclose
\pgfusepath{fill,stroke}
\pgfpathmoveto{\pgfpoint{155.569717pt}{281.206055pt}}
\pgflineto{\pgfpoint{150.142670pt}{279.442688pt}}
\pgflineto{\pgfpoint{151.642670pt}{278.352905pt}}
\pgfpathclose
\pgfusepath{fill,stroke}
\pgfpathmoveto{\pgfpoint{155.569717pt}{281.206055pt}}
\pgflineto{\pgfpoint{149.569717pt}{281.206055pt}}
\pgflineto{\pgfpoint{150.142670pt}{279.442688pt}}
\pgfpathclose
\pgfusepath{fill,stroke}
\pgfpathmoveto{\pgfpoint{155.569717pt}{281.206055pt}}
\pgflineto{\pgfpoint{150.142670pt}{282.969421pt}}
\pgflineto{\pgfpoint{149.569717pt}{281.206055pt}}
\pgfpathclose
\pgfusepath{fill,stroke}
\pgfpathmoveto{\pgfpoint{155.569717pt}{281.206055pt}}
\pgflineto{\pgfpoint{151.642670pt}{284.059204pt}}
\pgflineto{\pgfpoint{150.142670pt}{282.969421pt}}
\pgfpathclose
\pgfusepath{fill,stroke}
\pgfpathmoveto{\pgfpoint{155.569717pt}{281.206055pt}}
\pgflineto{\pgfpoint{153.496765pt}{284.059204pt}}
\pgflineto{\pgfpoint{151.642670pt}{284.059204pt}}
\pgfpathclose
\pgfusepath{fill,stroke}
\pgfpathmoveto{\pgfpoint{155.569717pt}{281.206055pt}}
\pgflineto{\pgfpoint{154.996765pt}{282.969421pt}}
\pgflineto{\pgfpoint{153.496765pt}{284.059204pt}}
\pgfpathclose
\pgfusepath{fill,stroke}
\pgfpathmoveto{\pgfpoint{293.457397pt}{204.547684pt}}
\pgflineto{\pgfpoint{291.384460pt}{201.694504pt}}
\pgflineto{\pgfpoint{292.884460pt}{202.784332pt}}
\pgfpathclose
\pgfusepath{fill,stroke}
\pgfpathmoveto{\pgfpoint{293.457397pt}{204.547684pt}}
\pgflineto{\pgfpoint{289.530365pt}{201.694504pt}}
\pgflineto{\pgfpoint{291.384460pt}{201.694504pt}}
\pgfpathclose
\pgfusepath{fill,stroke}
\pgfpathmoveto{\pgfpoint{293.457397pt}{204.547684pt}}
\pgflineto{\pgfpoint{288.030365pt}{202.784332pt}}
\pgflineto{\pgfpoint{289.530365pt}{201.694504pt}}
\pgfpathclose
\pgfusepath{fill,stroke}
\pgfpathmoveto{\pgfpoint{293.457397pt}{204.547684pt}}
\pgflineto{\pgfpoint{287.457397pt}{204.547684pt}}
\pgflineto{\pgfpoint{288.030365pt}{202.784332pt}}
\pgfpathclose
\pgfusepath{fill,stroke}
\pgfpathmoveto{\pgfpoint{293.457397pt}{204.547684pt}}
\pgflineto{\pgfpoint{288.030365pt}{206.311035pt}}
\pgflineto{\pgfpoint{287.457397pt}{204.547684pt}}
\pgfpathclose
\pgfusepath{fill,stroke}
\pgfpathmoveto{\pgfpoint{293.457397pt}{204.547684pt}}
\pgflineto{\pgfpoint{289.530365pt}{207.400848pt}}
\pgflineto{\pgfpoint{288.030365pt}{206.311035pt}}
\pgfpathclose
\pgfusepath{fill,stroke}
\pgfpathmoveto{\pgfpoint{293.457397pt}{204.547684pt}}
\pgflineto{\pgfpoint{291.384460pt}{207.400848pt}}
\pgflineto{\pgfpoint{289.530365pt}{207.400848pt}}
\pgfpathclose
\pgfusepath{fill,stroke}
\pgfpathmoveto{\pgfpoint{293.457397pt}{204.547684pt}}
\pgflineto{\pgfpoint{292.884460pt}{206.311035pt}}
\pgflineto{\pgfpoint{291.384460pt}{207.400848pt}}
\pgfpathclose
\pgfusepath{fill,stroke}
\pgfpathmoveto{\pgfpoint{163.698486pt}{268.265717pt}}
\pgflineto{\pgfpoint{161.625519pt}{265.412537pt}}
\pgflineto{\pgfpoint{163.125534pt}{266.502350pt}}
\pgfpathclose
\pgfusepath{fill,stroke}
\pgfpathmoveto{\pgfpoint{163.698486pt}{268.265717pt}}
\pgflineto{\pgfpoint{159.771439pt}{265.412537pt}}
\pgflineto{\pgfpoint{161.625519pt}{265.412537pt}}
\pgfpathclose
\pgfusepath{fill,stroke}
\pgfpathmoveto{\pgfpoint{163.698486pt}{268.265717pt}}
\pgflineto{\pgfpoint{158.271423pt}{266.502350pt}}
\pgflineto{\pgfpoint{159.771439pt}{265.412537pt}}
\pgfpathclose
\pgfusepath{fill,stroke}
\pgfpathmoveto{\pgfpoint{163.698486pt}{268.265717pt}}
\pgflineto{\pgfpoint{157.698486pt}{268.265717pt}}
\pgflineto{\pgfpoint{158.271423pt}{266.502350pt}}
\pgfpathclose
\pgfusepath{fill,stroke}
\pgfpathmoveto{\pgfpoint{163.698486pt}{268.265717pt}}
\pgflineto{\pgfpoint{158.271423pt}{270.029083pt}}
\pgflineto{\pgfpoint{157.698486pt}{268.265717pt}}
\pgfpathclose
\pgfusepath{fill,stroke}
\pgfpathmoveto{\pgfpoint{163.698486pt}{268.265717pt}}
\pgflineto{\pgfpoint{159.771439pt}{271.118896pt}}
\pgflineto{\pgfpoint{158.271423pt}{270.029083pt}}
\pgfpathclose
\pgfusepath{fill,stroke}
\pgfpathmoveto{\pgfpoint{163.698486pt}{268.265717pt}}
\pgflineto{\pgfpoint{161.625519pt}{271.118896pt}}
\pgflineto{\pgfpoint{159.771439pt}{271.118896pt}}
\pgfpathclose
\pgfusepath{fill,stroke}
\pgfpathmoveto{\pgfpoint{163.698486pt}{268.265717pt}}
\pgflineto{\pgfpoint{163.125534pt}{270.029083pt}}
\pgflineto{\pgfpoint{161.625519pt}{271.118896pt}}
\pgfpathclose
\pgfusepath{fill,stroke}
\pgfpathmoveto{\pgfpoint{181.984879pt}{192.006577pt}}
\pgflineto{\pgfpoint{179.911926pt}{189.153412pt}}
\pgflineto{\pgfpoint{181.411926pt}{190.243225pt}}
\pgfpathclose
\pgfusepath{fill,stroke}
\pgfpathmoveto{\pgfpoint{181.984879pt}{192.006577pt}}
\pgflineto{\pgfpoint{178.057831pt}{189.153412pt}}
\pgflineto{\pgfpoint{179.911926pt}{189.153412pt}}
\pgfpathclose
\pgfusepath{fill,stroke}
\pgfpathmoveto{\pgfpoint{181.984879pt}{192.006577pt}}
\pgflineto{\pgfpoint{176.557831pt}{190.243225pt}}
\pgflineto{\pgfpoint{178.057831pt}{189.153412pt}}
\pgfpathclose
\pgfusepath{fill,stroke}
\pgfpathmoveto{\pgfpoint{181.984879pt}{192.006577pt}}
\pgflineto{\pgfpoint{175.984894pt}{192.006577pt}}
\pgflineto{\pgfpoint{176.557831pt}{190.243225pt}}
\pgfpathclose
\pgfusepath{fill,stroke}
\pgfpathmoveto{\pgfpoint{181.984879pt}{192.006577pt}}
\pgflineto{\pgfpoint{176.557831pt}{193.769928pt}}
\pgflineto{\pgfpoint{175.984894pt}{192.006577pt}}
\pgfpathclose
\pgfusepath{fill,stroke}
\pgfpathmoveto{\pgfpoint{181.984879pt}{192.006577pt}}
\pgflineto{\pgfpoint{178.057831pt}{194.859756pt}}
\pgflineto{\pgfpoint{176.557831pt}{193.769928pt}}
\pgfpathclose
\pgfusepath{fill,stroke}
\pgfpathmoveto{\pgfpoint{181.984879pt}{192.006577pt}}
\pgflineto{\pgfpoint{179.911926pt}{194.859756pt}}
\pgflineto{\pgfpoint{178.057831pt}{194.859756pt}}
\pgfpathclose
\pgfusepath{fill,stroke}
\pgfpathmoveto{\pgfpoint{181.984879pt}{192.006577pt}}
\pgflineto{\pgfpoint{181.411926pt}{193.769928pt}}
\pgflineto{\pgfpoint{179.911926pt}{194.859756pt}}
\pgfpathclose
\pgfusepath{fill,stroke}
\pgfpathmoveto{\pgfpoint{209.089447pt}{198.928482pt}}
\pgflineto{\pgfpoint{207.016479pt}{196.075317pt}}
\pgflineto{\pgfpoint{208.516495pt}{197.165131pt}}
\pgfpathclose
\pgfusepath{fill,stroke}
\pgfpathmoveto{\pgfpoint{209.089447pt}{198.928482pt}}
\pgflineto{\pgfpoint{205.162384pt}{196.075317pt}}
\pgflineto{\pgfpoint{207.016479pt}{196.075317pt}}
\pgfpathclose
\pgfusepath{fill,stroke}
\pgfpathmoveto{\pgfpoint{209.089447pt}{198.928482pt}}
\pgflineto{\pgfpoint{203.662384pt}{197.165131pt}}
\pgflineto{\pgfpoint{205.162384pt}{196.075317pt}}
\pgfpathclose
\pgfusepath{fill,stroke}
\pgfpathmoveto{\pgfpoint{209.089447pt}{198.928482pt}}
\pgflineto{\pgfpoint{203.089447pt}{198.928482pt}}
\pgflineto{\pgfpoint{203.662384pt}{197.165131pt}}
\pgfpathclose
\pgfusepath{fill,stroke}
\pgfpathmoveto{\pgfpoint{209.089447pt}{198.928482pt}}
\pgflineto{\pgfpoint{203.662384pt}{200.691833pt}}
\pgflineto{\pgfpoint{203.089447pt}{198.928482pt}}
\pgfpathclose
\pgfusepath{fill,stroke}
\pgfpathmoveto{\pgfpoint{209.089447pt}{198.928482pt}}
\pgflineto{\pgfpoint{205.162384pt}{201.781647pt}}
\pgflineto{\pgfpoint{203.662384pt}{200.691833pt}}
\pgfpathclose
\pgfusepath{fill,stroke}
\pgfpathmoveto{\pgfpoint{209.089447pt}{198.928482pt}}
\pgflineto{\pgfpoint{207.016479pt}{201.781647pt}}
\pgflineto{\pgfpoint{205.162384pt}{201.781647pt}}
\pgfpathclose
\pgfusepath{fill,stroke}
\pgfpathmoveto{\pgfpoint{209.089447pt}{198.928482pt}}
\pgflineto{\pgfpoint{208.516495pt}{200.691833pt}}
\pgflineto{\pgfpoint{207.016479pt}{201.781647pt}}
\pgfpathclose
\pgfusepath{fill,stroke}
\pgfpathmoveto{\pgfpoint{228.104507pt}{195.738617pt}}
\pgflineto{\pgfpoint{226.031555pt}{192.885452pt}}
\pgflineto{\pgfpoint{227.531555pt}{193.975266pt}}
\pgfpathclose
\pgfusepath{fill,stroke}
\pgfpathmoveto{\pgfpoint{228.104507pt}{195.738617pt}}
\pgflineto{\pgfpoint{224.177460pt}{192.885452pt}}
\pgflineto{\pgfpoint{226.031555pt}{192.885452pt}}
\pgfpathclose
\pgfusepath{fill,stroke}
\pgfpathmoveto{\pgfpoint{228.104507pt}{195.738617pt}}
\pgflineto{\pgfpoint{222.677460pt}{193.975266pt}}
\pgflineto{\pgfpoint{224.177460pt}{192.885452pt}}
\pgfpathclose
\pgfusepath{fill,stroke}
\pgfpathmoveto{\pgfpoint{228.104507pt}{195.738617pt}}
\pgflineto{\pgfpoint{222.104507pt}{195.738617pt}}
\pgflineto{\pgfpoint{222.677460pt}{193.975266pt}}
\pgfpathclose
\pgfusepath{fill,stroke}
\pgfpathmoveto{\pgfpoint{228.104507pt}{195.738617pt}}
\pgflineto{\pgfpoint{222.677460pt}{197.501968pt}}
\pgflineto{\pgfpoint{222.104507pt}{195.738617pt}}
\pgfpathclose
\pgfusepath{fill,stroke}
\pgfpathmoveto{\pgfpoint{228.104507pt}{195.738617pt}}
\pgflineto{\pgfpoint{224.177460pt}{198.591797pt}}
\pgflineto{\pgfpoint{222.677460pt}{197.501968pt}}
\pgfpathclose
\pgfusepath{fill,stroke}
\pgfpathmoveto{\pgfpoint{228.104507pt}{195.738617pt}}
\pgflineto{\pgfpoint{226.031555pt}{198.591797pt}}
\pgflineto{\pgfpoint{224.177460pt}{198.591797pt}}
\pgfpathclose
\pgfusepath{fill,stroke}
\pgfpathmoveto{\pgfpoint{228.104507pt}{195.738617pt}}
\pgflineto{\pgfpoint{227.531555pt}{197.501968pt}}
\pgflineto{\pgfpoint{226.031555pt}{198.591797pt}}
\pgfpathclose
\pgfusepath{fill,stroke}
\pgfpathmoveto{\pgfpoint{221.808319pt}{195.210526pt}}
\pgflineto{\pgfpoint{219.735382pt}{192.357361pt}}
\pgflineto{\pgfpoint{221.235382pt}{193.447174pt}}
\pgfpathclose
\pgfusepath{fill,stroke}
\pgfpathmoveto{\pgfpoint{221.808319pt}{195.210526pt}}
\pgflineto{\pgfpoint{217.881287pt}{192.357361pt}}
\pgflineto{\pgfpoint{219.735382pt}{192.357361pt}}
\pgfpathclose
\pgfusepath{fill,stroke}
\pgfpathmoveto{\pgfpoint{221.808319pt}{195.210526pt}}
\pgflineto{\pgfpoint{216.381287pt}{193.447174pt}}
\pgflineto{\pgfpoint{217.881287pt}{192.357361pt}}
\pgfpathclose
\pgfusepath{fill,stroke}
\pgfpathmoveto{\pgfpoint{221.808319pt}{195.210526pt}}
\pgflineto{\pgfpoint{215.808334pt}{195.210526pt}}
\pgflineto{\pgfpoint{216.381287pt}{193.447174pt}}
\pgfpathclose
\pgfusepath{fill,stroke}
\pgfpathmoveto{\pgfpoint{221.808319pt}{195.210526pt}}
\pgflineto{\pgfpoint{216.381287pt}{196.973877pt}}
\pgflineto{\pgfpoint{215.808334pt}{195.210526pt}}
\pgfpathclose
\pgfusepath{fill,stroke}
\pgfpathmoveto{\pgfpoint{221.808319pt}{195.210526pt}}
\pgflineto{\pgfpoint{217.881287pt}{198.063690pt}}
\pgflineto{\pgfpoint{216.381287pt}{196.973877pt}}
\pgfpathclose
\pgfusepath{fill,stroke}
\pgfpathmoveto{\pgfpoint{221.808319pt}{195.210526pt}}
\pgflineto{\pgfpoint{219.735382pt}{198.063690pt}}
\pgflineto{\pgfpoint{217.881287pt}{198.063690pt}}
\pgfpathclose
\pgfusepath{fill,stroke}
\pgfpathmoveto{\pgfpoint{221.808319pt}{195.210526pt}}
\pgflineto{\pgfpoint{221.235382pt}{196.973877pt}}
\pgflineto{\pgfpoint{219.735382pt}{198.063690pt}}
\pgfpathclose
\pgfusepath{fill,stroke}
\pgfpathmoveto{\pgfpoint{222.506470pt}{204.315292pt}}
\pgflineto{\pgfpoint{220.433502pt}{201.462128pt}}
\pgflineto{\pgfpoint{221.933502pt}{202.551941pt}}
\pgfpathclose
\pgfusepath{fill,stroke}
\pgfpathmoveto{\pgfpoint{222.506470pt}{204.315292pt}}
\pgflineto{\pgfpoint{218.579407pt}{201.462128pt}}
\pgflineto{\pgfpoint{220.433502pt}{201.462128pt}}
\pgfpathclose
\pgfusepath{fill,stroke}
\pgfpathmoveto{\pgfpoint{222.506470pt}{204.315292pt}}
\pgflineto{\pgfpoint{217.079407pt}{202.551941pt}}
\pgflineto{\pgfpoint{218.579407pt}{201.462128pt}}
\pgfpathclose
\pgfusepath{fill,stroke}
\pgfpathmoveto{\pgfpoint{222.506470pt}{204.315292pt}}
\pgflineto{\pgfpoint{216.506454pt}{204.315292pt}}
\pgflineto{\pgfpoint{217.079407pt}{202.551941pt}}
\pgfpathclose
\pgfusepath{fill,stroke}
\pgfpathmoveto{\pgfpoint{222.506470pt}{204.315292pt}}
\pgflineto{\pgfpoint{217.079407pt}{206.078644pt}}
\pgflineto{\pgfpoint{216.506454pt}{204.315292pt}}
\pgfpathclose
\pgfusepath{fill,stroke}
\pgfpathmoveto{\pgfpoint{222.506470pt}{204.315292pt}}
\pgflineto{\pgfpoint{218.579407pt}{207.168457pt}}
\pgflineto{\pgfpoint{217.079407pt}{206.078644pt}}
\pgfpathclose
\pgfusepath{fill,stroke}
\pgfpathmoveto{\pgfpoint{222.506470pt}{204.315292pt}}
\pgflineto{\pgfpoint{220.433502pt}{207.168457pt}}
\pgflineto{\pgfpoint{218.579407pt}{207.168457pt}}
\pgfpathclose
\pgfusepath{fill,stroke}
\pgfpathmoveto{\pgfpoint{222.506470pt}{204.315292pt}}
\pgflineto{\pgfpoint{221.933502pt}{206.078644pt}}
\pgflineto{\pgfpoint{220.433502pt}{207.168457pt}}
\pgfpathclose
\pgfusepath{fill,stroke}
\pgfpathmoveto{\pgfpoint{211.934280pt}{195.055603pt}}
\pgflineto{\pgfpoint{209.861328pt}{192.202423pt}}
\pgflineto{\pgfpoint{211.361328pt}{193.292236pt}}
\pgfpathclose
\pgfusepath{fill,stroke}
\pgfpathmoveto{\pgfpoint{211.934280pt}{195.055603pt}}
\pgflineto{\pgfpoint{208.007233pt}{192.202423pt}}
\pgflineto{\pgfpoint{209.861328pt}{192.202423pt}}
\pgfpathclose
\pgfusepath{fill,stroke}
\pgfpathmoveto{\pgfpoint{211.934280pt}{195.055603pt}}
\pgflineto{\pgfpoint{206.507233pt}{193.292236pt}}
\pgflineto{\pgfpoint{208.007233pt}{192.202423pt}}
\pgfpathclose
\pgfusepath{fill,stroke}
\pgfpathmoveto{\pgfpoint{211.934280pt}{195.055603pt}}
\pgflineto{\pgfpoint{205.934280pt}{195.055603pt}}
\pgflineto{\pgfpoint{206.507233pt}{193.292236pt}}
\pgfpathclose
\pgfusepath{fill,stroke}
\pgfpathmoveto{\pgfpoint{211.934280pt}{195.055603pt}}
\pgflineto{\pgfpoint{206.507233pt}{196.818954pt}}
\pgflineto{\pgfpoint{205.934280pt}{195.055603pt}}
\pgfpathclose
\pgfusepath{fill,stroke}
\pgfpathmoveto{\pgfpoint{211.934280pt}{195.055603pt}}
\pgflineto{\pgfpoint{208.007233pt}{197.908768pt}}
\pgflineto{\pgfpoint{206.507233pt}{196.818954pt}}
\pgfpathclose
\pgfusepath{fill,stroke}
\pgfpathmoveto{\pgfpoint{211.934280pt}{195.055603pt}}
\pgflineto{\pgfpoint{209.861328pt}{197.908768pt}}
\pgflineto{\pgfpoint{208.007233pt}{197.908768pt}}
\pgfpathclose
\pgfusepath{fill,stroke}
\pgfpathmoveto{\pgfpoint{211.934280pt}{195.055603pt}}
\pgflineto{\pgfpoint{211.361328pt}{196.818954pt}}
\pgflineto{\pgfpoint{209.861328pt}{197.908768pt}}
\pgfpathclose
\pgfusepath{fill,stroke}
\pgfpathmoveto{\pgfpoint{174.737503pt}{200.026962pt}}
\pgflineto{\pgfpoint{172.664551pt}{197.173798pt}}
\pgflineto{\pgfpoint{174.164551pt}{198.263611pt}}
\pgfpathclose
\pgfusepath{fill,stroke}
\pgfpathmoveto{\pgfpoint{174.737503pt}{200.026962pt}}
\pgflineto{\pgfpoint{170.810455pt}{197.173798pt}}
\pgflineto{\pgfpoint{172.664551pt}{197.173798pt}}
\pgfpathclose
\pgfusepath{fill,stroke}
\pgfpathmoveto{\pgfpoint{174.737503pt}{200.026962pt}}
\pgflineto{\pgfpoint{169.310455pt}{198.263611pt}}
\pgflineto{\pgfpoint{170.810455pt}{197.173798pt}}
\pgfpathclose
\pgfusepath{fill,stroke}
\pgfpathmoveto{\pgfpoint{174.737503pt}{200.026962pt}}
\pgflineto{\pgfpoint{168.737518pt}{200.026962pt}}
\pgflineto{\pgfpoint{169.310455pt}{198.263611pt}}
\pgfpathclose
\pgfusepath{fill,stroke}
\pgfpathmoveto{\pgfpoint{174.737503pt}{200.026962pt}}
\pgflineto{\pgfpoint{169.310455pt}{201.790329pt}}
\pgflineto{\pgfpoint{168.737518pt}{200.026962pt}}
\pgfpathclose
\pgfusepath{fill,stroke}
\pgfpathmoveto{\pgfpoint{174.737503pt}{200.026962pt}}
\pgflineto{\pgfpoint{170.810455pt}{202.880142pt}}
\pgflineto{\pgfpoint{169.310455pt}{201.790329pt}}
\pgfpathclose
\pgfusepath{fill,stroke}
\pgfpathmoveto{\pgfpoint{174.737503pt}{200.026962pt}}
\pgflineto{\pgfpoint{172.664551pt}{202.880142pt}}
\pgflineto{\pgfpoint{170.810455pt}{202.880142pt}}
\pgfpathclose
\pgfusepath{fill,stroke}
\pgfpathmoveto{\pgfpoint{174.737503pt}{200.026962pt}}
\pgflineto{\pgfpoint{174.164551pt}{201.790329pt}}
\pgflineto{\pgfpoint{172.664551pt}{202.880142pt}}
\pgfpathclose
\pgfusepath{fill,stroke}
\pgfpathmoveto{\pgfpoint{143.592575pt}{266.274353pt}}
\pgflineto{\pgfpoint{141.519623pt}{263.421173pt}}
\pgflineto{\pgfpoint{143.019623pt}{264.510986pt}}
\pgfpathclose
\pgfusepath{fill,stroke}
\pgfpathmoveto{\pgfpoint{143.592575pt}{266.274353pt}}
\pgflineto{\pgfpoint{139.665512pt}{263.421173pt}}
\pgflineto{\pgfpoint{141.519623pt}{263.421173pt}}
\pgfpathclose
\pgfusepath{fill,stroke}
\pgfpathmoveto{\pgfpoint{143.592575pt}{266.274353pt}}
\pgflineto{\pgfpoint{138.165527pt}{264.510986pt}}
\pgflineto{\pgfpoint{139.665512pt}{263.421173pt}}
\pgfpathclose
\pgfusepath{fill,stroke}
\pgfpathmoveto{\pgfpoint{143.592575pt}{266.274353pt}}
\pgflineto{\pgfpoint{137.592560pt}{266.274353pt}}
\pgflineto{\pgfpoint{138.165527pt}{264.510986pt}}
\pgfpathclose
\pgfusepath{fill,stroke}
\pgfpathmoveto{\pgfpoint{143.592575pt}{266.274353pt}}
\pgflineto{\pgfpoint{138.165527pt}{268.037720pt}}
\pgflineto{\pgfpoint{137.592560pt}{266.274353pt}}
\pgfpathclose
\pgfusepath{fill,stroke}
\pgfpathmoveto{\pgfpoint{143.592575pt}{266.274353pt}}
\pgflineto{\pgfpoint{139.665512pt}{269.127502pt}}
\pgflineto{\pgfpoint{138.165527pt}{268.037720pt}}
\pgfpathclose
\pgfusepath{fill,stroke}
\pgfpathmoveto{\pgfpoint{143.592575pt}{266.274353pt}}
\pgflineto{\pgfpoint{141.519623pt}{269.127502pt}}
\pgflineto{\pgfpoint{139.665512pt}{269.127502pt}}
\pgfpathclose
\pgfusepath{fill,stroke}
\pgfpathmoveto{\pgfpoint{143.592575pt}{266.274353pt}}
\pgflineto{\pgfpoint{143.019623pt}{268.037720pt}}
\pgflineto{\pgfpoint{141.519623pt}{269.127502pt}}
\pgfpathclose
\pgfusepath{fill,stroke}
\pgfpathmoveto{\pgfpoint{181.539825pt}{196.309021pt}}
\pgflineto{\pgfpoint{179.466888pt}{193.455841pt}}
\pgflineto{\pgfpoint{180.966888pt}{194.545654pt}}
\pgfpathclose
\pgfusepath{fill,stroke}
\pgfpathmoveto{\pgfpoint{181.539825pt}{196.309021pt}}
\pgflineto{\pgfpoint{177.612793pt}{193.455841pt}}
\pgflineto{\pgfpoint{179.466888pt}{193.455841pt}}
\pgfpathclose
\pgfusepath{fill,stroke}
\pgfpathmoveto{\pgfpoint{181.539825pt}{196.309021pt}}
\pgflineto{\pgfpoint{176.112778pt}{194.545654pt}}
\pgflineto{\pgfpoint{177.612793pt}{193.455841pt}}
\pgfpathclose
\pgfusepath{fill,stroke}
\pgfpathmoveto{\pgfpoint{181.539825pt}{196.309021pt}}
\pgflineto{\pgfpoint{175.539825pt}{196.309021pt}}
\pgflineto{\pgfpoint{176.112778pt}{194.545654pt}}
\pgfpathclose
\pgfusepath{fill,stroke}
\pgfpathmoveto{\pgfpoint{181.539825pt}{196.309021pt}}
\pgflineto{\pgfpoint{176.112778pt}{198.072372pt}}
\pgflineto{\pgfpoint{175.539825pt}{196.309021pt}}
\pgfpathclose
\pgfusepath{fill,stroke}
\pgfpathmoveto{\pgfpoint{181.539825pt}{196.309021pt}}
\pgflineto{\pgfpoint{177.612793pt}{199.162186pt}}
\pgflineto{\pgfpoint{176.112778pt}{198.072372pt}}
\pgfpathclose
\pgfusepath{fill,stroke}
\pgfpathmoveto{\pgfpoint{181.539825pt}{196.309021pt}}
\pgflineto{\pgfpoint{179.466888pt}{199.162186pt}}
\pgflineto{\pgfpoint{177.612793pt}{199.162186pt}}
\pgfpathclose
\pgfusepath{fill,stroke}
\pgfpathmoveto{\pgfpoint{181.539825pt}{196.309021pt}}
\pgflineto{\pgfpoint{180.966888pt}{198.072372pt}}
\pgflineto{\pgfpoint{179.466888pt}{199.162186pt}}
\pgfpathclose
\pgfusepath{fill,stroke}
\pgfpathmoveto{\pgfpoint{207.156509pt}{198.132782pt}}
\pgflineto{\pgfpoint{205.083557pt}{195.279602pt}}
\pgflineto{\pgfpoint{206.583557pt}{196.369431pt}}
\pgfpathclose
\pgfusepath{fill,stroke}
\pgfpathmoveto{\pgfpoint{207.156509pt}{198.132782pt}}
\pgflineto{\pgfpoint{203.229462pt}{195.279602pt}}
\pgflineto{\pgfpoint{205.083557pt}{195.279602pt}}
\pgfpathclose
\pgfusepath{fill,stroke}
\pgfpathmoveto{\pgfpoint{207.156509pt}{198.132782pt}}
\pgflineto{\pgfpoint{201.729462pt}{196.369431pt}}
\pgflineto{\pgfpoint{203.229462pt}{195.279602pt}}
\pgfpathclose
\pgfusepath{fill,stroke}
\pgfpathmoveto{\pgfpoint{207.156509pt}{198.132782pt}}
\pgflineto{\pgfpoint{201.156525pt}{198.132782pt}}
\pgflineto{\pgfpoint{201.729462pt}{196.369431pt}}
\pgfpathclose
\pgfusepath{fill,stroke}
\pgfpathmoveto{\pgfpoint{207.156509pt}{198.132782pt}}
\pgflineto{\pgfpoint{201.729462pt}{199.896133pt}}
\pgflineto{\pgfpoint{201.156525pt}{198.132782pt}}
\pgfpathclose
\pgfusepath{fill,stroke}
\pgfpathmoveto{\pgfpoint{207.156509pt}{198.132782pt}}
\pgflineto{\pgfpoint{203.229462pt}{200.985947pt}}
\pgflineto{\pgfpoint{201.729462pt}{199.896133pt}}
\pgfpathclose
\pgfusepath{fill,stroke}
\pgfpathmoveto{\pgfpoint{207.156509pt}{198.132782pt}}
\pgflineto{\pgfpoint{205.083557pt}{200.985947pt}}
\pgflineto{\pgfpoint{203.229462pt}{200.985947pt}}
\pgfpathclose
\pgfusepath{fill,stroke}
\pgfpathmoveto{\pgfpoint{207.156509pt}{198.132782pt}}
\pgflineto{\pgfpoint{206.583557pt}{199.896133pt}}
\pgflineto{\pgfpoint{205.083557pt}{200.985947pt}}
\pgfpathclose
\pgfusepath{fill,stroke}
\pgfpathmoveto{\pgfpoint{140.852448pt}{185.838150pt}}
\pgflineto{\pgfpoint{138.779510pt}{182.984985pt}}
\pgflineto{\pgfpoint{140.279495pt}{184.074799pt}}
\pgfpathclose
\pgfusepath{fill,stroke}
\pgfpathmoveto{\pgfpoint{140.852448pt}{185.838150pt}}
\pgflineto{\pgfpoint{136.925385pt}{182.984985pt}}
\pgflineto{\pgfpoint{138.779510pt}{182.984985pt}}
\pgfpathclose
\pgfusepath{fill,stroke}
\pgfpathmoveto{\pgfpoint{140.852448pt}{185.838150pt}}
\pgflineto{\pgfpoint{135.425400pt}{184.074799pt}}
\pgflineto{\pgfpoint{136.925385pt}{182.984985pt}}
\pgfpathclose
\pgfusepath{fill,stroke}
\pgfpathmoveto{\pgfpoint{140.852448pt}{185.838150pt}}
\pgflineto{\pgfpoint{134.852448pt}{185.838150pt}}
\pgflineto{\pgfpoint{135.425400pt}{184.074799pt}}
\pgfpathclose
\pgfusepath{fill,stroke}
\pgfpathmoveto{\pgfpoint{140.852448pt}{185.838150pt}}
\pgflineto{\pgfpoint{135.425400pt}{187.601501pt}}
\pgflineto{\pgfpoint{134.852448pt}{185.838150pt}}
\pgfpathclose
\pgfusepath{fill,stroke}
\pgfpathmoveto{\pgfpoint{140.852448pt}{185.838150pt}}
\pgflineto{\pgfpoint{136.925385pt}{188.691315pt}}
\pgflineto{\pgfpoint{135.425400pt}{187.601501pt}}
\pgfpathclose
\pgfusepath{fill,stroke}
\pgfpathmoveto{\pgfpoint{140.852448pt}{185.838150pt}}
\pgflineto{\pgfpoint{138.779510pt}{188.691315pt}}
\pgflineto{\pgfpoint{136.925385pt}{188.691315pt}}
\pgfpathclose
\pgfusepath{fill,stroke}
\pgfpathmoveto{\pgfpoint{140.852448pt}{185.838150pt}}
\pgflineto{\pgfpoint{140.279495pt}{187.601501pt}}
\pgflineto{\pgfpoint{138.779510pt}{188.691315pt}}
\pgfpathclose
\pgfusepath{fill,stroke}
\pgfpathmoveto{\pgfpoint{185.004272pt}{195.013336pt}}
\pgflineto{\pgfpoint{182.931305pt}{192.160172pt}}
\pgflineto{\pgfpoint{184.431320pt}{193.249985pt}}
\pgfpathclose
\pgfusepath{fill,stroke}
\pgfpathmoveto{\pgfpoint{185.004272pt}{195.013336pt}}
\pgflineto{\pgfpoint{181.077209pt}{192.160172pt}}
\pgflineto{\pgfpoint{182.931305pt}{192.160172pt}}
\pgfpathclose
\pgfusepath{fill,stroke}
\pgfpathmoveto{\pgfpoint{185.004272pt}{195.013336pt}}
\pgflineto{\pgfpoint{179.577209pt}{193.249985pt}}
\pgflineto{\pgfpoint{181.077209pt}{192.160172pt}}
\pgfpathclose
\pgfusepath{fill,stroke}
\pgfpathmoveto{\pgfpoint{185.004272pt}{195.013336pt}}
\pgflineto{\pgfpoint{179.004272pt}{195.013336pt}}
\pgflineto{\pgfpoint{179.577209pt}{193.249985pt}}
\pgfpathclose
\pgfusepath{fill,stroke}
\pgfpathmoveto{\pgfpoint{185.004272pt}{195.013336pt}}
\pgflineto{\pgfpoint{179.577209pt}{196.776703pt}}
\pgflineto{\pgfpoint{179.004272pt}{195.013336pt}}
\pgfpathclose
\pgfusepath{fill,stroke}
\pgfpathmoveto{\pgfpoint{185.004272pt}{195.013336pt}}
\pgflineto{\pgfpoint{181.077209pt}{197.866516pt}}
\pgflineto{\pgfpoint{179.577209pt}{196.776703pt}}
\pgfpathclose
\pgfusepath{fill,stroke}
\pgfpathmoveto{\pgfpoint{185.004272pt}{195.013336pt}}
\pgflineto{\pgfpoint{182.931305pt}{197.866516pt}}
\pgflineto{\pgfpoint{181.077209pt}{197.866516pt}}
\pgfpathclose
\pgfusepath{fill,stroke}
\pgfpathmoveto{\pgfpoint{185.004272pt}{195.013336pt}}
\pgflineto{\pgfpoint{184.431320pt}{196.776703pt}}
\pgflineto{\pgfpoint{182.931305pt}{197.866516pt}}
\pgfpathclose
\pgfusepath{fill,stroke}
\pgfpathmoveto{\pgfpoint{125.040024pt}{198.259552pt}}
\pgflineto{\pgfpoint{122.967072pt}{195.406372pt}}
\pgflineto{\pgfpoint{124.467072pt}{196.496185pt}}
\pgfpathclose
\pgfusepath{fill,stroke}
\pgfpathmoveto{\pgfpoint{125.040024pt}{198.259552pt}}
\pgflineto{\pgfpoint{121.112961pt}{195.406372pt}}
\pgflineto{\pgfpoint{122.967072pt}{195.406372pt}}
\pgfpathclose
\pgfusepath{fill,stroke}
\pgfpathmoveto{\pgfpoint{125.040024pt}{198.259552pt}}
\pgflineto{\pgfpoint{119.612961pt}{196.496185pt}}
\pgflineto{\pgfpoint{121.112961pt}{195.406372pt}}
\pgfpathclose
\pgfusepath{fill,stroke}
\pgfpathmoveto{\pgfpoint{125.040024pt}{198.259552pt}}
\pgflineto{\pgfpoint{119.040009pt}{198.259552pt}}
\pgflineto{\pgfpoint{119.612961pt}{196.496185pt}}
\pgfpathclose
\pgfusepath{fill,stroke}
\pgfpathmoveto{\pgfpoint{125.040024pt}{198.259552pt}}
\pgflineto{\pgfpoint{119.612961pt}{200.022903pt}}
\pgflineto{\pgfpoint{119.040009pt}{198.259552pt}}
\pgfpathclose
\pgfusepath{fill,stroke}
\pgfpathmoveto{\pgfpoint{125.040024pt}{198.259552pt}}
\pgflineto{\pgfpoint{121.112961pt}{201.112717pt}}
\pgflineto{\pgfpoint{119.612961pt}{200.022903pt}}
\pgfpathclose
\pgfusepath{fill,stroke}
\pgfpathmoveto{\pgfpoint{125.040024pt}{198.259552pt}}
\pgflineto{\pgfpoint{122.967072pt}{201.112717pt}}
\pgflineto{\pgfpoint{121.112961pt}{201.112717pt}}
\pgfpathclose
\pgfusepath{fill,stroke}
\pgfpathmoveto{\pgfpoint{125.040024pt}{198.259552pt}}
\pgflineto{\pgfpoint{124.467072pt}{200.022903pt}}
\pgflineto{\pgfpoint{122.967072pt}{201.112717pt}}
\pgfpathclose
\pgfusepath{fill,stroke}
\pgfpathmoveto{\pgfpoint{249.580475pt}{165.523132pt}}
\pgflineto{\pgfpoint{247.507538pt}{162.669968pt}}
\pgflineto{\pgfpoint{249.007538pt}{163.759781pt}}
\pgfpathclose
\pgfusepath{fill,stroke}
\pgfpathmoveto{\pgfpoint{249.580475pt}{165.523132pt}}
\pgflineto{\pgfpoint{245.653427pt}{162.669968pt}}
\pgflineto{\pgfpoint{247.507538pt}{162.669968pt}}
\pgfpathclose
\pgfusepath{fill,stroke}
\pgfpathmoveto{\pgfpoint{249.580475pt}{165.523132pt}}
\pgflineto{\pgfpoint{244.153427pt}{163.759781pt}}
\pgflineto{\pgfpoint{245.653427pt}{162.669968pt}}
\pgfpathclose
\pgfusepath{fill,stroke}
\pgfpathmoveto{\pgfpoint{249.580475pt}{165.523132pt}}
\pgflineto{\pgfpoint{243.580475pt}{165.523132pt}}
\pgflineto{\pgfpoint{244.153427pt}{163.759781pt}}
\pgfpathclose
\pgfusepath{fill,stroke}
\pgfpathmoveto{\pgfpoint{249.580475pt}{165.523132pt}}
\pgflineto{\pgfpoint{244.153427pt}{167.286499pt}}
\pgflineto{\pgfpoint{243.580475pt}{165.523132pt}}
\pgfpathclose
\pgfusepath{fill,stroke}
\pgfpathmoveto{\pgfpoint{249.580475pt}{165.523132pt}}
\pgflineto{\pgfpoint{245.653427pt}{168.376312pt}}
\pgflineto{\pgfpoint{244.153427pt}{167.286499pt}}
\pgfpathclose
\pgfusepath{fill,stroke}
\pgfpathmoveto{\pgfpoint{249.580475pt}{165.523132pt}}
\pgflineto{\pgfpoint{247.507538pt}{168.376312pt}}
\pgflineto{\pgfpoint{245.653427pt}{168.376312pt}}
\pgfpathclose
\pgfusepath{fill,stroke}
\pgfpathmoveto{\pgfpoint{249.580475pt}{165.523132pt}}
\pgflineto{\pgfpoint{249.007538pt}{167.286499pt}}
\pgflineto{\pgfpoint{247.507538pt}{168.376312pt}}
\pgfpathclose
\pgfusepath{fill,stroke}
\pgfpathmoveto{\pgfpoint{203.129242pt}{215.159363pt}}
\pgflineto{\pgfpoint{201.056274pt}{212.306198pt}}
\pgflineto{\pgfpoint{202.556290pt}{213.396011pt}}
\pgfpathclose
\pgfusepath{fill,stroke}
\pgfpathmoveto{\pgfpoint{203.129242pt}{215.159363pt}}
\pgflineto{\pgfpoint{199.202179pt}{212.306198pt}}
\pgflineto{\pgfpoint{201.056274pt}{212.306198pt}}
\pgfpathclose
\pgfusepath{fill,stroke}
\pgfpathmoveto{\pgfpoint{203.129242pt}{215.159363pt}}
\pgflineto{\pgfpoint{197.702179pt}{213.396011pt}}
\pgflineto{\pgfpoint{199.202179pt}{212.306198pt}}
\pgfpathclose
\pgfusepath{fill,stroke}
\pgfpathmoveto{\pgfpoint{203.129242pt}{215.159363pt}}
\pgflineto{\pgfpoint{197.129242pt}{215.159363pt}}
\pgflineto{\pgfpoint{197.702179pt}{213.396011pt}}
\pgfpathclose
\pgfusepath{fill,stroke}
\pgfpathmoveto{\pgfpoint{203.129242pt}{215.159363pt}}
\pgflineto{\pgfpoint{197.702179pt}{216.922714pt}}
\pgflineto{\pgfpoint{197.129242pt}{215.159363pt}}
\pgfpathclose
\pgfusepath{fill,stroke}
\pgfpathmoveto{\pgfpoint{203.129242pt}{215.159363pt}}
\pgflineto{\pgfpoint{199.202179pt}{218.012543pt}}
\pgflineto{\pgfpoint{197.702179pt}{216.922714pt}}
\pgfpathclose
\pgfusepath{fill,stroke}
\pgfpathmoveto{\pgfpoint{203.129242pt}{215.159363pt}}
\pgflineto{\pgfpoint{201.056274pt}{218.012543pt}}
\pgflineto{\pgfpoint{199.202179pt}{218.012543pt}}
\pgfpathclose
\pgfusepath{fill,stroke}
\pgfpathmoveto{\pgfpoint{203.129242pt}{215.159363pt}}
\pgflineto{\pgfpoint{202.556290pt}{216.922714pt}}
\pgflineto{\pgfpoint{201.056274pt}{218.012543pt}}
\pgfpathclose
\pgfusepath{fill,stroke}
\pgfpathmoveto{\pgfpoint{160.953979pt}{192.133347pt}}
\pgflineto{\pgfpoint{158.881027pt}{189.280182pt}}
\pgflineto{\pgfpoint{160.381027pt}{190.369995pt}}
\pgfpathclose
\pgfusepath{fill,stroke}
\pgfpathmoveto{\pgfpoint{160.953979pt}{192.133347pt}}
\pgflineto{\pgfpoint{157.026932pt}{189.280182pt}}
\pgflineto{\pgfpoint{158.881027pt}{189.280182pt}}
\pgfpathclose
\pgfusepath{fill,stroke}
\pgfpathmoveto{\pgfpoint{160.953979pt}{192.133347pt}}
\pgflineto{\pgfpoint{155.526917pt}{190.369995pt}}
\pgflineto{\pgfpoint{157.026932pt}{189.280182pt}}
\pgfpathclose
\pgfusepath{fill,stroke}
\pgfpathmoveto{\pgfpoint{160.953979pt}{192.133347pt}}
\pgflineto{\pgfpoint{154.953979pt}{192.133347pt}}
\pgflineto{\pgfpoint{155.526917pt}{190.369995pt}}
\pgfpathclose
\pgfusepath{fill,stroke}
\pgfpathmoveto{\pgfpoint{160.953979pt}{192.133347pt}}
\pgflineto{\pgfpoint{155.526917pt}{193.896698pt}}
\pgflineto{\pgfpoint{154.953979pt}{192.133347pt}}
\pgfpathclose
\pgfusepath{fill,stroke}
\pgfpathmoveto{\pgfpoint{160.953979pt}{192.133347pt}}
\pgflineto{\pgfpoint{157.026932pt}{194.986511pt}}
\pgflineto{\pgfpoint{155.526917pt}{193.896698pt}}
\pgfpathclose
\pgfusepath{fill,stroke}
\pgfpathmoveto{\pgfpoint{160.953979pt}{192.133347pt}}
\pgflineto{\pgfpoint{158.881027pt}{194.986511pt}}
\pgflineto{\pgfpoint{157.026932pt}{194.986511pt}}
\pgfpathclose
\pgfusepath{fill,stroke}
\pgfpathmoveto{\pgfpoint{160.953979pt}{192.133347pt}}
\pgflineto{\pgfpoint{160.381027pt}{193.896698pt}}
\pgflineto{\pgfpoint{158.881027pt}{194.986511pt}}
\pgfpathclose
\pgfusepath{fill,stroke}
\pgfpathmoveto{\pgfpoint{198.194382pt}{186.500061pt}}
\pgflineto{\pgfpoint{196.121429pt}{183.646896pt}}
\pgflineto{\pgfpoint{197.621429pt}{184.736710pt}}
\pgfpathclose
\pgfusepath{fill,stroke}
\pgfpathmoveto{\pgfpoint{198.194382pt}{186.500061pt}}
\pgflineto{\pgfpoint{194.267334pt}{183.646896pt}}
\pgflineto{\pgfpoint{196.121429pt}{183.646896pt}}
\pgfpathclose
\pgfusepath{fill,stroke}
\pgfpathmoveto{\pgfpoint{198.194382pt}{186.500061pt}}
\pgflineto{\pgfpoint{192.767334pt}{184.736710pt}}
\pgflineto{\pgfpoint{194.267334pt}{183.646896pt}}
\pgfpathclose
\pgfusepath{fill,stroke}
\pgfpathmoveto{\pgfpoint{198.194382pt}{186.500061pt}}
\pgflineto{\pgfpoint{192.194397pt}{186.500061pt}}
\pgflineto{\pgfpoint{192.767334pt}{184.736710pt}}
\pgfpathclose
\pgfusepath{fill,stroke}
\pgfpathmoveto{\pgfpoint{198.194382pt}{186.500061pt}}
\pgflineto{\pgfpoint{192.767334pt}{188.263412pt}}
\pgflineto{\pgfpoint{192.194397pt}{186.500061pt}}
\pgfpathclose
\pgfusepath{fill,stroke}
\pgfpathmoveto{\pgfpoint{198.194382pt}{186.500061pt}}
\pgflineto{\pgfpoint{194.267334pt}{189.353226pt}}
\pgflineto{\pgfpoint{192.767334pt}{188.263412pt}}
\pgfpathclose
\pgfusepath{fill,stroke}
\pgfpathmoveto{\pgfpoint{198.194382pt}{186.500061pt}}
\pgflineto{\pgfpoint{196.121429pt}{189.353226pt}}
\pgflineto{\pgfpoint{194.267334pt}{189.353226pt}}
\pgfpathclose
\pgfusepath{fill,stroke}
\pgfpathmoveto{\pgfpoint{198.194382pt}{186.500061pt}}
\pgflineto{\pgfpoint{197.621429pt}{188.263412pt}}
\pgflineto{\pgfpoint{196.121429pt}{189.353226pt}}
\pgfpathclose
\pgfusepath{fill,stroke}
\pgfpathmoveto{\pgfpoint{168.851471pt}{198.203201pt}}
\pgflineto{\pgfpoint{166.778534pt}{195.350037pt}}
\pgflineto{\pgfpoint{168.278534pt}{196.439850pt}}
\pgfpathclose
\pgfusepath{fill,stroke}
\pgfpathmoveto{\pgfpoint{168.851471pt}{198.203201pt}}
\pgflineto{\pgfpoint{164.924438pt}{195.350037pt}}
\pgflineto{\pgfpoint{166.778534pt}{195.350037pt}}
\pgfpathclose
\pgfusepath{fill,stroke}
\pgfpathmoveto{\pgfpoint{168.851471pt}{198.203201pt}}
\pgflineto{\pgfpoint{163.424438pt}{196.439850pt}}
\pgflineto{\pgfpoint{164.924438pt}{195.350037pt}}
\pgfpathclose
\pgfusepath{fill,stroke}
\pgfpathmoveto{\pgfpoint{168.851471pt}{198.203201pt}}
\pgflineto{\pgfpoint{162.851486pt}{198.203201pt}}
\pgflineto{\pgfpoint{163.424438pt}{196.439850pt}}
\pgfpathclose
\pgfusepath{fill,stroke}
\pgfpathmoveto{\pgfpoint{168.851471pt}{198.203201pt}}
\pgflineto{\pgfpoint{163.424438pt}{199.966553pt}}
\pgflineto{\pgfpoint{162.851486pt}{198.203201pt}}
\pgfpathclose
\pgfusepath{fill,stroke}
\pgfpathmoveto{\pgfpoint{168.851471pt}{198.203201pt}}
\pgflineto{\pgfpoint{164.924438pt}{201.056366pt}}
\pgflineto{\pgfpoint{163.424438pt}{199.966553pt}}
\pgfpathclose
\pgfusepath{fill,stroke}
\pgfpathmoveto{\pgfpoint{168.851471pt}{198.203201pt}}
\pgflineto{\pgfpoint{166.778534pt}{201.056366pt}}
\pgflineto{\pgfpoint{164.924438pt}{201.056366pt}}
\pgfpathclose
\pgfusepath{fill,stroke}
\pgfpathmoveto{\pgfpoint{168.851471pt}{198.203201pt}}
\pgflineto{\pgfpoint{168.278534pt}{199.966553pt}}
\pgflineto{\pgfpoint{166.778534pt}{201.056366pt}}
\pgfpathclose
\pgfusepath{fill,stroke}
\pgfpathmoveto{\pgfpoint{214.342804pt}{178.430389pt}}
\pgflineto{\pgfpoint{212.269867pt}{175.577209pt}}
\pgflineto{\pgfpoint{213.769867pt}{176.667023pt}}
\pgfpathclose
\pgfusepath{fill,stroke}
\pgfpathmoveto{\pgfpoint{214.342804pt}{178.430389pt}}
\pgflineto{\pgfpoint{210.415771pt}{175.577209pt}}
\pgflineto{\pgfpoint{212.269867pt}{175.577209pt}}
\pgfpathclose
\pgfusepath{fill,stroke}
\pgfpathmoveto{\pgfpoint{214.342804pt}{178.430389pt}}
\pgflineto{\pgfpoint{208.915756pt}{176.667023pt}}
\pgflineto{\pgfpoint{210.415771pt}{175.577209pt}}
\pgfpathclose
\pgfusepath{fill,stroke}
\pgfpathmoveto{\pgfpoint{214.342804pt}{178.430389pt}}
\pgflineto{\pgfpoint{208.342804pt}{178.430389pt}}
\pgflineto{\pgfpoint{208.915756pt}{176.667023pt}}
\pgfpathclose
\pgfusepath{fill,stroke}
\pgfpathmoveto{\pgfpoint{214.342804pt}{178.430389pt}}
\pgflineto{\pgfpoint{208.915756pt}{180.193741pt}}
\pgflineto{\pgfpoint{208.342804pt}{178.430389pt}}
\pgfpathclose
\pgfusepath{fill,stroke}
\pgfpathmoveto{\pgfpoint{214.342804pt}{178.430389pt}}
\pgflineto{\pgfpoint{210.415771pt}{181.283554pt}}
\pgflineto{\pgfpoint{208.915756pt}{180.193741pt}}
\pgfpathclose
\pgfusepath{fill,stroke}
\pgfpathmoveto{\pgfpoint{214.342804pt}{178.430389pt}}
\pgflineto{\pgfpoint{212.269867pt}{181.283554pt}}
\pgflineto{\pgfpoint{210.415771pt}{181.283554pt}}
\pgfpathclose
\pgfusepath{fill,stroke}
\pgfpathmoveto{\pgfpoint{214.342804pt}{178.430389pt}}
\pgflineto{\pgfpoint{213.769867pt}{180.193741pt}}
\pgflineto{\pgfpoint{212.269867pt}{181.283554pt}}
\pgfpathclose
\pgfusepath{fill,stroke}
\pgfpathmoveto{\pgfpoint{298.758759pt}{152.742615pt}}
\pgflineto{\pgfpoint{296.685791pt}{149.889450pt}}
\pgflineto{\pgfpoint{298.185791pt}{150.979263pt}}
\pgfpathclose
\pgfusepath{fill,stroke}
\pgfpathmoveto{\pgfpoint{298.758759pt}{152.742615pt}}
\pgflineto{\pgfpoint{294.831696pt}{149.889450pt}}
\pgflineto{\pgfpoint{296.685791pt}{149.889450pt}}
\pgfpathclose
\pgfusepath{fill,stroke}
\pgfpathmoveto{\pgfpoint{298.758759pt}{152.742615pt}}
\pgflineto{\pgfpoint{293.331696pt}{150.979263pt}}
\pgflineto{\pgfpoint{294.831696pt}{149.889450pt}}
\pgfpathclose
\pgfusepath{fill,stroke}
\pgfpathmoveto{\pgfpoint{298.758759pt}{152.742615pt}}
\pgflineto{\pgfpoint{292.758759pt}{152.742615pt}}
\pgflineto{\pgfpoint{293.331696pt}{150.979263pt}}
\pgfpathclose
\pgfusepath{fill,stroke}
\pgfpathmoveto{\pgfpoint{298.758759pt}{152.742615pt}}
\pgflineto{\pgfpoint{293.331696pt}{154.505981pt}}
\pgflineto{\pgfpoint{292.758759pt}{152.742615pt}}
\pgfpathclose
\pgfusepath{fill,stroke}
\pgfpathmoveto{\pgfpoint{298.758759pt}{152.742615pt}}
\pgflineto{\pgfpoint{294.831696pt}{155.595795pt}}
\pgflineto{\pgfpoint{293.331696pt}{154.505981pt}}
\pgfpathclose
\pgfusepath{fill,stroke}
\pgfpathmoveto{\pgfpoint{298.758759pt}{152.742615pt}}
\pgflineto{\pgfpoint{296.685791pt}{155.595795pt}}
\pgflineto{\pgfpoint{294.831696pt}{155.595795pt}}
\pgfpathclose
\pgfusepath{fill,stroke}
\pgfpathmoveto{\pgfpoint{298.758759pt}{152.742615pt}}
\pgflineto{\pgfpoint{298.185791pt}{154.505981pt}}
\pgflineto{\pgfpoint{296.685791pt}{155.595795pt}}
\pgfpathclose
\pgfusepath{fill,stroke}
\pgfpathmoveto{\pgfpoint{220.560440pt}{209.265549pt}}
\pgflineto{\pgfpoint{218.487488pt}{206.412384pt}}
\pgflineto{\pgfpoint{219.987488pt}{207.502197pt}}
\pgfpathclose
\pgfusepath{fill,stroke}
\pgfpathmoveto{\pgfpoint{220.560440pt}{209.265549pt}}
\pgflineto{\pgfpoint{216.633392pt}{206.412384pt}}
\pgflineto{\pgfpoint{218.487488pt}{206.412384pt}}
\pgfpathclose
\pgfusepath{fill,stroke}
\pgfpathmoveto{\pgfpoint{220.560440pt}{209.265549pt}}
\pgflineto{\pgfpoint{215.133392pt}{207.502197pt}}
\pgflineto{\pgfpoint{216.633392pt}{206.412384pt}}
\pgfpathclose
\pgfusepath{fill,stroke}
\pgfpathmoveto{\pgfpoint{220.560440pt}{209.265549pt}}
\pgflineto{\pgfpoint{214.560440pt}{209.265549pt}}
\pgflineto{\pgfpoint{215.133392pt}{207.502197pt}}
\pgfpathclose
\pgfusepath{fill,stroke}
\pgfpathmoveto{\pgfpoint{220.560440pt}{209.265549pt}}
\pgflineto{\pgfpoint{215.133392pt}{211.028915pt}}
\pgflineto{\pgfpoint{214.560440pt}{209.265549pt}}
\pgfpathclose
\pgfusepath{fill,stroke}
\pgfpathmoveto{\pgfpoint{220.560440pt}{209.265549pt}}
\pgflineto{\pgfpoint{216.633392pt}{212.118729pt}}
\pgflineto{\pgfpoint{215.133392pt}{211.028915pt}}
\pgfpathclose
\pgfusepath{fill,stroke}
\pgfpathmoveto{\pgfpoint{220.560440pt}{209.265549pt}}
\pgflineto{\pgfpoint{218.487488pt}{212.118729pt}}
\pgflineto{\pgfpoint{216.633392pt}{212.118729pt}}
\pgfpathclose
\pgfusepath{fill,stroke}
\pgfpathmoveto{\pgfpoint{220.560440pt}{209.265549pt}}
\pgflineto{\pgfpoint{219.987488pt}{211.028915pt}}
\pgflineto{\pgfpoint{218.487488pt}{212.118729pt}}
\pgfpathclose
\pgfusepath{fill,stroke}
\pgfpathmoveto{\pgfpoint{217.344727pt}{205.357468pt}}
\pgflineto{\pgfpoint{215.271759pt}{202.504303pt}}
\pgflineto{\pgfpoint{216.771774pt}{203.594116pt}}
\pgfpathclose
\pgfusepath{fill,stroke}
\pgfpathmoveto{\pgfpoint{217.344727pt}{205.357468pt}}
\pgflineto{\pgfpoint{213.417664pt}{202.504303pt}}
\pgflineto{\pgfpoint{215.271759pt}{202.504303pt}}
\pgfpathclose
\pgfusepath{fill,stroke}
\pgfpathmoveto{\pgfpoint{217.344727pt}{205.357468pt}}
\pgflineto{\pgfpoint{211.917664pt}{203.594116pt}}
\pgflineto{\pgfpoint{213.417664pt}{202.504303pt}}
\pgfpathclose
\pgfusepath{fill,stroke}
\pgfpathmoveto{\pgfpoint{217.344727pt}{205.357468pt}}
\pgflineto{\pgfpoint{211.344727pt}{205.357468pt}}
\pgflineto{\pgfpoint{211.917664pt}{203.594116pt}}
\pgfpathclose
\pgfusepath{fill,stroke}
\pgfpathmoveto{\pgfpoint{217.344727pt}{205.357468pt}}
\pgflineto{\pgfpoint{211.917664pt}{207.120819pt}}
\pgflineto{\pgfpoint{211.344727pt}{205.357468pt}}
\pgfpathclose
\pgfusepath{fill,stroke}
\pgfpathmoveto{\pgfpoint{217.344727pt}{205.357468pt}}
\pgflineto{\pgfpoint{213.417664pt}{208.210632pt}}
\pgflineto{\pgfpoint{211.917664pt}{207.120819pt}}
\pgfpathclose
\pgfusepath{fill,stroke}
\pgfpathmoveto{\pgfpoint{217.344727pt}{205.357468pt}}
\pgflineto{\pgfpoint{215.271759pt}{208.210632pt}}
\pgflineto{\pgfpoint{213.417664pt}{208.210632pt}}
\pgfpathclose
\pgfusepath{fill,stroke}
\pgfpathmoveto{\pgfpoint{217.344727pt}{205.357468pt}}
\pgflineto{\pgfpoint{216.771774pt}{207.120819pt}}
\pgflineto{\pgfpoint{215.271759pt}{208.210632pt}}
\pgfpathclose
\pgfusepath{fill,stroke}
\pgfpathmoveto{\pgfpoint{183.869812pt}{200.048080pt}}
\pgflineto{\pgfpoint{181.796860pt}{197.194916pt}}
\pgflineto{\pgfpoint{183.296860pt}{198.284729pt}}
\pgfpathclose
\pgfusepath{fill,stroke}
\pgfpathmoveto{\pgfpoint{183.869812pt}{200.048080pt}}
\pgflineto{\pgfpoint{179.942749pt}{197.194916pt}}
\pgflineto{\pgfpoint{181.796860pt}{197.194916pt}}
\pgfpathclose
\pgfusepath{fill,stroke}
\pgfpathmoveto{\pgfpoint{183.869812pt}{200.048080pt}}
\pgflineto{\pgfpoint{178.442749pt}{198.284729pt}}
\pgflineto{\pgfpoint{179.942749pt}{197.194916pt}}
\pgfpathclose
\pgfusepath{fill,stroke}
\pgfpathmoveto{\pgfpoint{183.869812pt}{200.048080pt}}
\pgflineto{\pgfpoint{177.869812pt}{200.048080pt}}
\pgflineto{\pgfpoint{178.442749pt}{198.284729pt}}
\pgfpathclose
\pgfusepath{fill,stroke}
\pgfpathmoveto{\pgfpoint{183.869812pt}{200.048080pt}}
\pgflineto{\pgfpoint{178.442749pt}{201.811432pt}}
\pgflineto{\pgfpoint{177.869812pt}{200.048080pt}}
\pgfpathclose
\pgfusepath{fill,stroke}
\pgfpathmoveto{\pgfpoint{183.869812pt}{200.048080pt}}
\pgflineto{\pgfpoint{179.942749pt}{202.901245pt}}
\pgflineto{\pgfpoint{178.442749pt}{201.811432pt}}
\pgfpathclose
\pgfusepath{fill,stroke}
\pgfpathmoveto{\pgfpoint{183.869812pt}{200.048080pt}}
\pgflineto{\pgfpoint{181.796860pt}{202.901245pt}}
\pgflineto{\pgfpoint{179.942749pt}{202.901245pt}}
\pgfpathclose
\pgfusepath{fill,stroke}
\pgfpathmoveto{\pgfpoint{183.869812pt}{200.048080pt}}
\pgflineto{\pgfpoint{183.296860pt}{201.811432pt}}
\pgflineto{\pgfpoint{181.796860pt}{202.901245pt}}
\pgfpathclose
\pgfusepath{fill,stroke}
\pgfpathmoveto{\pgfpoint{186.365601pt}{189.809601pt}}
\pgflineto{\pgfpoint{184.292633pt}{186.956436pt}}
\pgflineto{\pgfpoint{185.792648pt}{188.046249pt}}
\pgfpathclose
\pgfusepath{fill,stroke}
\pgfpathmoveto{\pgfpoint{186.365601pt}{189.809601pt}}
\pgflineto{\pgfpoint{182.438538pt}{186.956436pt}}
\pgflineto{\pgfpoint{184.292633pt}{186.956436pt}}
\pgfpathclose
\pgfusepath{fill,stroke}
\pgfpathmoveto{\pgfpoint{186.365601pt}{189.809601pt}}
\pgflineto{\pgfpoint{180.938538pt}{188.046249pt}}
\pgflineto{\pgfpoint{182.438538pt}{186.956436pt}}
\pgfpathclose
\pgfusepath{fill,stroke}
\pgfpathmoveto{\pgfpoint{186.365601pt}{189.809601pt}}
\pgflineto{\pgfpoint{180.365601pt}{189.809601pt}}
\pgflineto{\pgfpoint{180.938538pt}{188.046249pt}}
\pgfpathclose
\pgfusepath{fill,stroke}
\pgfpathmoveto{\pgfpoint{186.365601pt}{189.809601pt}}
\pgflineto{\pgfpoint{180.938538pt}{191.572968pt}}
\pgflineto{\pgfpoint{180.365601pt}{189.809601pt}}
\pgfpathclose
\pgfusepath{fill,stroke}
\pgfpathmoveto{\pgfpoint{186.365601pt}{189.809601pt}}
\pgflineto{\pgfpoint{182.438538pt}{192.662781pt}}
\pgflineto{\pgfpoint{180.938538pt}{191.572968pt}}
\pgfpathclose
\pgfusepath{fill,stroke}
\pgfpathmoveto{\pgfpoint{186.365601pt}{189.809601pt}}
\pgflineto{\pgfpoint{184.292633pt}{192.662781pt}}
\pgflineto{\pgfpoint{182.438538pt}{192.662781pt}}
\pgfpathclose
\pgfusepath{fill,stroke}
\pgfpathmoveto{\pgfpoint{186.365601pt}{189.809601pt}}
\pgflineto{\pgfpoint{185.792648pt}{191.572968pt}}
\pgflineto{\pgfpoint{184.292633pt}{192.662781pt}}
\pgfpathclose
\pgfusepath{fill,stroke}
\pgfpathmoveto{\pgfpoint{206.013336pt}{224.510620pt}}
\pgflineto{\pgfpoint{203.940384pt}{221.657440pt}}
\pgflineto{\pgfpoint{205.440384pt}{222.747253pt}}
\pgfpathclose
\pgfusepath{fill,stroke}
\pgfpathmoveto{\pgfpoint{206.013336pt}{224.510620pt}}
\pgflineto{\pgfpoint{202.086288pt}{221.657440pt}}
\pgflineto{\pgfpoint{203.940384pt}{221.657440pt}}
\pgfpathclose
\pgfusepath{fill,stroke}
\pgfpathmoveto{\pgfpoint{206.013336pt}{224.510620pt}}
\pgflineto{\pgfpoint{200.586288pt}{222.747253pt}}
\pgflineto{\pgfpoint{202.086288pt}{221.657440pt}}
\pgfpathclose
\pgfusepath{fill,stroke}
\pgfpathmoveto{\pgfpoint{206.013336pt}{224.510620pt}}
\pgflineto{\pgfpoint{200.013336pt}{224.510620pt}}
\pgflineto{\pgfpoint{200.586288pt}{222.747253pt}}
\pgfpathclose
\pgfusepath{fill,stroke}
\pgfpathmoveto{\pgfpoint{206.013336pt}{224.510620pt}}
\pgflineto{\pgfpoint{200.586288pt}{226.273972pt}}
\pgflineto{\pgfpoint{200.013336pt}{224.510620pt}}
\pgfpathclose
\pgfusepath{fill,stroke}
\pgfpathmoveto{\pgfpoint{206.013336pt}{224.510620pt}}
\pgflineto{\pgfpoint{202.086288pt}{227.363785pt}}
\pgflineto{\pgfpoint{200.586288pt}{226.273972pt}}
\pgfpathclose
\pgfusepath{fill,stroke}
\pgfpathmoveto{\pgfpoint{206.013336pt}{224.510620pt}}
\pgflineto{\pgfpoint{203.940384pt}{227.363785pt}}
\pgflineto{\pgfpoint{202.086288pt}{227.363785pt}}
\pgfpathclose
\pgfusepath{fill,stroke}
\pgfpathmoveto{\pgfpoint{206.013336pt}{224.510620pt}}
\pgflineto{\pgfpoint{205.440384pt}{226.273972pt}}
\pgflineto{\pgfpoint{203.940384pt}{227.363785pt}}
\pgfpathclose
\pgfusepath{fill,stroke}
\pgfpathmoveto{\pgfpoint{182.918610pt}{234.178726pt}}
\pgflineto{\pgfpoint{180.845673pt}{231.325546pt}}
\pgflineto{\pgfpoint{182.345673pt}{232.415359pt}}
\pgfpathclose
\pgfusepath{fill,stroke}
\pgfpathmoveto{\pgfpoint{182.918610pt}{234.178726pt}}
\pgflineto{\pgfpoint{178.991577pt}{231.325546pt}}
\pgflineto{\pgfpoint{180.845673pt}{231.325546pt}}
\pgfpathclose
\pgfusepath{fill,stroke}
\pgfpathmoveto{\pgfpoint{182.918610pt}{234.178726pt}}
\pgflineto{\pgfpoint{177.491562pt}{232.415359pt}}
\pgflineto{\pgfpoint{178.991577pt}{231.325546pt}}
\pgfpathclose
\pgfusepath{fill,stroke}
\pgfpathmoveto{\pgfpoint{182.918610pt}{234.178726pt}}
\pgflineto{\pgfpoint{176.918610pt}{234.178726pt}}
\pgflineto{\pgfpoint{177.491562pt}{232.415359pt}}
\pgfpathclose
\pgfusepath{fill,stroke}
\pgfpathmoveto{\pgfpoint{182.918610pt}{234.178726pt}}
\pgflineto{\pgfpoint{177.491562pt}{235.942078pt}}
\pgflineto{\pgfpoint{176.918610pt}{234.178726pt}}
\pgfpathclose
\pgfusepath{fill,stroke}
\pgfpathmoveto{\pgfpoint{182.918610pt}{234.178726pt}}
\pgflineto{\pgfpoint{178.991577pt}{237.031891pt}}
\pgflineto{\pgfpoint{177.491562pt}{235.942078pt}}
\pgfpathclose
\pgfusepath{fill,stroke}
\pgfpathmoveto{\pgfpoint{182.918610pt}{234.178726pt}}
\pgflineto{\pgfpoint{180.845673pt}{237.031891pt}}
\pgflineto{\pgfpoint{178.991577pt}{237.031891pt}}
\pgfpathclose
\pgfusepath{fill,stroke}
\pgfpathmoveto{\pgfpoint{182.918610pt}{234.178726pt}}
\pgflineto{\pgfpoint{182.345673pt}{235.942078pt}}
\pgflineto{\pgfpoint{180.845673pt}{237.031891pt}}
\pgfpathclose
\pgfusepath{fill,stroke}
\pgfpathmoveto{\pgfpoint{246.622192pt}{183.014465pt}}
\pgflineto{\pgfpoint{244.549240pt}{180.161301pt}}
\pgflineto{\pgfpoint{246.049240pt}{181.251114pt}}
\pgfpathclose
\pgfusepath{fill,stroke}
\pgfpathmoveto{\pgfpoint{246.622192pt}{183.014465pt}}
\pgflineto{\pgfpoint{242.695129pt}{180.161301pt}}
\pgflineto{\pgfpoint{244.549240pt}{180.161301pt}}
\pgfpathclose
\pgfusepath{fill,stroke}
\pgfpathmoveto{\pgfpoint{246.622192pt}{183.014465pt}}
\pgflineto{\pgfpoint{241.195129pt}{181.251114pt}}
\pgflineto{\pgfpoint{242.695129pt}{180.161301pt}}
\pgfpathclose
\pgfusepath{fill,stroke}
\pgfpathmoveto{\pgfpoint{246.622192pt}{183.014465pt}}
\pgflineto{\pgfpoint{240.622192pt}{183.014465pt}}
\pgflineto{\pgfpoint{241.195129pt}{181.251114pt}}
\pgfpathclose
\pgfusepath{fill,stroke}
\pgfpathmoveto{\pgfpoint{246.622192pt}{183.014465pt}}
\pgflineto{\pgfpoint{241.195129pt}{184.777832pt}}
\pgflineto{\pgfpoint{240.622192pt}{183.014465pt}}
\pgfpathclose
\pgfusepath{fill,stroke}
\pgfpathmoveto{\pgfpoint{246.622192pt}{183.014465pt}}
\pgflineto{\pgfpoint{242.695129pt}{185.867645pt}}
\pgflineto{\pgfpoint{241.195129pt}{184.777832pt}}
\pgfpathclose
\pgfusepath{fill,stroke}
\pgfpathmoveto{\pgfpoint{246.622192pt}{183.014465pt}}
\pgflineto{\pgfpoint{244.549240pt}{185.867645pt}}
\pgflineto{\pgfpoint{242.695129pt}{185.867645pt}}
\pgfpathclose
\pgfusepath{fill,stroke}
\pgfpathmoveto{\pgfpoint{246.622192pt}{183.014465pt}}
\pgflineto{\pgfpoint{246.049240pt}{184.777832pt}}
\pgflineto{\pgfpoint{244.549240pt}{185.867645pt}}
\pgfpathclose
\pgfusepath{fill,stroke}
\pgfpathmoveto{\pgfpoint{174.122284pt}{229.418610pt}}
\pgflineto{\pgfpoint{172.049347pt}{226.565430pt}}
\pgflineto{\pgfpoint{173.549347pt}{227.655258pt}}
\pgfpathclose
\pgfusepath{fill,stroke}
\pgfpathmoveto{\pgfpoint{174.122284pt}{229.418610pt}}
\pgflineto{\pgfpoint{170.195251pt}{226.565430pt}}
\pgflineto{\pgfpoint{172.049347pt}{226.565430pt}}
\pgfpathclose
\pgfusepath{fill,stroke}
\pgfpathmoveto{\pgfpoint{174.122284pt}{229.418610pt}}
\pgflineto{\pgfpoint{168.695236pt}{227.655258pt}}
\pgflineto{\pgfpoint{170.195251pt}{226.565430pt}}
\pgfpathclose
\pgfusepath{fill,stroke}
\pgfpathmoveto{\pgfpoint{174.122284pt}{229.418610pt}}
\pgflineto{\pgfpoint{168.122284pt}{229.418610pt}}
\pgflineto{\pgfpoint{168.695236pt}{227.655258pt}}
\pgfpathclose
\pgfusepath{fill,stroke}
\pgfpathmoveto{\pgfpoint{174.122284pt}{229.418610pt}}
\pgflineto{\pgfpoint{168.695236pt}{231.181961pt}}
\pgflineto{\pgfpoint{168.122284pt}{229.418610pt}}
\pgfpathclose
\pgfusepath{fill,stroke}
\pgfpathmoveto{\pgfpoint{174.122284pt}{229.418610pt}}
\pgflineto{\pgfpoint{170.195251pt}{232.271774pt}}
\pgflineto{\pgfpoint{168.695236pt}{231.181961pt}}
\pgfpathclose
\pgfusepath{fill,stroke}
\pgfpathmoveto{\pgfpoint{174.122284pt}{229.418610pt}}
\pgflineto{\pgfpoint{172.049347pt}{232.271774pt}}
\pgflineto{\pgfpoint{170.195251pt}{232.271774pt}}
\pgfpathclose
\pgfusepath{fill,stroke}
\pgfpathmoveto{\pgfpoint{174.122284pt}{229.418610pt}}
\pgflineto{\pgfpoint{173.549347pt}{231.181961pt}}
\pgflineto{\pgfpoint{172.049347pt}{232.271774pt}}
\pgfpathclose
\pgfusepath{fill,stroke}
\pgfpathmoveto{\pgfpoint{216.362991pt}{203.632263pt}}
\pgflineto{\pgfpoint{214.290039pt}{200.779099pt}}
\pgflineto{\pgfpoint{215.790039pt}{201.868912pt}}
\pgfpathclose
\pgfusepath{fill,stroke}
\pgfpathmoveto{\pgfpoint{216.362991pt}{203.632263pt}}
\pgflineto{\pgfpoint{212.435944pt}{200.779099pt}}
\pgflineto{\pgfpoint{214.290039pt}{200.779099pt}}
\pgfpathclose
\pgfusepath{fill,stroke}
\pgfpathmoveto{\pgfpoint{216.362991pt}{203.632263pt}}
\pgflineto{\pgfpoint{210.935944pt}{201.868912pt}}
\pgflineto{\pgfpoint{212.435944pt}{200.779099pt}}
\pgfpathclose
\pgfusepath{fill,stroke}
\pgfpathmoveto{\pgfpoint{216.362991pt}{203.632263pt}}
\pgflineto{\pgfpoint{210.362991pt}{203.632263pt}}
\pgflineto{\pgfpoint{210.935944pt}{201.868912pt}}
\pgfpathclose
\pgfusepath{fill,stroke}
\pgfpathmoveto{\pgfpoint{216.362991pt}{203.632263pt}}
\pgflineto{\pgfpoint{210.935944pt}{205.395630pt}}
\pgflineto{\pgfpoint{210.362991pt}{203.632263pt}}
\pgfpathclose
\pgfusepath{fill,stroke}
\pgfpathmoveto{\pgfpoint{216.362991pt}{203.632263pt}}
\pgflineto{\pgfpoint{212.435944pt}{206.485443pt}}
\pgflineto{\pgfpoint{210.935944pt}{205.395630pt}}
\pgfpathclose
\pgfusepath{fill,stroke}
\pgfpathmoveto{\pgfpoint{216.362991pt}{203.632263pt}}
\pgflineto{\pgfpoint{214.290039pt}{206.485443pt}}
\pgflineto{\pgfpoint{212.435944pt}{206.485443pt}}
\pgfpathclose
\pgfusepath{fill,stroke}
\pgfpathmoveto{\pgfpoint{216.362991pt}{203.632263pt}}
\pgflineto{\pgfpoint{215.790039pt}{205.395630pt}}
\pgflineto{\pgfpoint{214.290039pt}{206.485443pt}}
\pgfpathclose
\pgfusepath{fill,stroke}
\pgfpathmoveto{\pgfpoint{190.907745pt}{202.597153pt}}
\pgflineto{\pgfpoint{188.834808pt}{199.743988pt}}
\pgflineto{\pgfpoint{190.334808pt}{200.833801pt}}
\pgfpathclose
\pgfusepath{fill,stroke}
\pgfpathmoveto{\pgfpoint{190.907745pt}{202.597153pt}}
\pgflineto{\pgfpoint{186.980713pt}{199.743988pt}}
\pgflineto{\pgfpoint{188.834808pt}{199.743988pt}}
\pgfpathclose
\pgfusepath{fill,stroke}
\pgfpathmoveto{\pgfpoint{190.907745pt}{202.597153pt}}
\pgflineto{\pgfpoint{185.480698pt}{200.833801pt}}
\pgflineto{\pgfpoint{186.980713pt}{199.743988pt}}
\pgfpathclose
\pgfusepath{fill,stroke}
\pgfpathmoveto{\pgfpoint{190.907745pt}{202.597153pt}}
\pgflineto{\pgfpoint{184.907745pt}{202.597153pt}}
\pgflineto{\pgfpoint{185.480698pt}{200.833801pt}}
\pgfpathclose
\pgfusepath{fill,stroke}
\pgfpathmoveto{\pgfpoint{190.907745pt}{202.597153pt}}
\pgflineto{\pgfpoint{185.480698pt}{204.360504pt}}
\pgflineto{\pgfpoint{184.907745pt}{202.597153pt}}
\pgfpathclose
\pgfusepath{fill,stroke}
\pgfpathmoveto{\pgfpoint{190.907745pt}{202.597153pt}}
\pgflineto{\pgfpoint{186.980713pt}{205.450317pt}}
\pgflineto{\pgfpoint{185.480698pt}{204.360504pt}}
\pgfpathclose
\pgfusepath{fill,stroke}
\pgfpathmoveto{\pgfpoint{190.907745pt}{202.597153pt}}
\pgflineto{\pgfpoint{188.834808pt}{205.450317pt}}
\pgflineto{\pgfpoint{186.980713pt}{205.450317pt}}
\pgfpathclose
\pgfusepath{fill,stroke}
\pgfpathmoveto{\pgfpoint{190.907745pt}{202.597153pt}}
\pgflineto{\pgfpoint{190.334808pt}{204.360504pt}}
\pgflineto{\pgfpoint{188.834808pt}{205.450317pt}}
\pgfpathclose
\pgfusepath{fill,stroke}
\pgfpathmoveto{\pgfpoint{185.065338pt}{198.062347pt}}
\pgflineto{\pgfpoint{182.992401pt}{195.209183pt}}
\pgflineto{\pgfpoint{184.492401pt}{196.298996pt}}
\pgfpathclose
\pgfusepath{fill,stroke}
\pgfpathmoveto{\pgfpoint{185.065338pt}{198.062347pt}}
\pgflineto{\pgfpoint{181.138306pt}{195.209183pt}}
\pgflineto{\pgfpoint{182.992401pt}{195.209183pt}}
\pgfpathclose
\pgfusepath{fill,stroke}
\pgfpathmoveto{\pgfpoint{185.065338pt}{198.062347pt}}
\pgflineto{\pgfpoint{179.638290pt}{196.298996pt}}
\pgflineto{\pgfpoint{181.138306pt}{195.209183pt}}
\pgfpathclose
\pgfusepath{fill,stroke}
\pgfpathmoveto{\pgfpoint{185.065338pt}{198.062347pt}}
\pgflineto{\pgfpoint{179.065338pt}{198.062347pt}}
\pgflineto{\pgfpoint{179.638290pt}{196.298996pt}}
\pgfpathclose
\pgfusepath{fill,stroke}
\pgfpathmoveto{\pgfpoint{185.065338pt}{198.062347pt}}
\pgflineto{\pgfpoint{179.638290pt}{199.825714pt}}
\pgflineto{\pgfpoint{179.065338pt}{198.062347pt}}
\pgfpathclose
\pgfusepath{fill,stroke}
\pgfpathmoveto{\pgfpoint{185.065338pt}{198.062347pt}}
\pgflineto{\pgfpoint{181.138306pt}{200.915527pt}}
\pgflineto{\pgfpoint{179.638290pt}{199.825714pt}}
\pgfpathclose
\pgfusepath{fill,stroke}
\pgfpathmoveto{\pgfpoint{185.065338pt}{198.062347pt}}
\pgflineto{\pgfpoint{182.992401pt}{200.915527pt}}
\pgflineto{\pgfpoint{181.138306pt}{200.915527pt}}
\pgfpathclose
\pgfusepath{fill,stroke}
\pgfpathmoveto{\pgfpoint{185.065338pt}{198.062347pt}}
\pgflineto{\pgfpoint{184.492401pt}{199.825714pt}}
\pgflineto{\pgfpoint{182.992401pt}{200.915527pt}}
\pgfpathclose
\pgfusepath{fill,stroke}
\pgfpathmoveto{\pgfpoint{216.376099pt}{221.947479pt}}
\pgflineto{\pgfpoint{214.303131pt}{219.094299pt}}
\pgflineto{\pgfpoint{215.803131pt}{220.184113pt}}
\pgfpathclose
\pgfusepath{fill,stroke}
\pgfpathmoveto{\pgfpoint{216.376099pt}{221.947479pt}}
\pgflineto{\pgfpoint{212.449036pt}{219.094299pt}}
\pgflineto{\pgfpoint{214.303131pt}{219.094299pt}}
\pgfpathclose
\pgfusepath{fill,stroke}
\pgfpathmoveto{\pgfpoint{216.376099pt}{221.947479pt}}
\pgflineto{\pgfpoint{210.949036pt}{220.184113pt}}
\pgflineto{\pgfpoint{212.449036pt}{219.094299pt}}
\pgfpathclose
\pgfusepath{fill,stroke}
\pgfpathmoveto{\pgfpoint{216.376099pt}{221.947479pt}}
\pgflineto{\pgfpoint{210.376099pt}{221.947479pt}}
\pgflineto{\pgfpoint{210.949036pt}{220.184113pt}}
\pgfpathclose
\pgfusepath{fill,stroke}
\pgfpathmoveto{\pgfpoint{216.376099pt}{221.947479pt}}
\pgflineto{\pgfpoint{210.949036pt}{223.710831pt}}
\pgflineto{\pgfpoint{210.376099pt}{221.947479pt}}
\pgfpathclose
\pgfusepath{fill,stroke}
\pgfpathmoveto{\pgfpoint{216.376099pt}{221.947479pt}}
\pgflineto{\pgfpoint{212.449036pt}{224.800644pt}}
\pgflineto{\pgfpoint{210.949036pt}{223.710831pt}}
\pgfpathclose
\pgfusepath{fill,stroke}
\pgfpathmoveto{\pgfpoint{216.376099pt}{221.947479pt}}
\pgflineto{\pgfpoint{214.303131pt}{224.800644pt}}
\pgflineto{\pgfpoint{212.449036pt}{224.800644pt}}
\pgfpathclose
\pgfusepath{fill,stroke}
\pgfpathmoveto{\pgfpoint{216.376099pt}{221.947479pt}}
\pgflineto{\pgfpoint{215.803131pt}{223.710831pt}}
\pgflineto{\pgfpoint{214.303131pt}{224.800644pt}}
\pgfpathclose
\pgfusepath{fill,stroke}
\pgfpathmoveto{\pgfpoint{205.633728pt}{193.781067pt}}
\pgflineto{\pgfpoint{203.560776pt}{190.927887pt}}
\pgflineto{\pgfpoint{205.060776pt}{192.017700pt}}
\pgfpathclose
\pgfusepath{fill,stroke}
\pgfpathmoveto{\pgfpoint{205.633728pt}{193.781067pt}}
\pgflineto{\pgfpoint{201.706665pt}{190.927887pt}}
\pgflineto{\pgfpoint{203.560776pt}{190.927887pt}}
\pgfpathclose
\pgfusepath{fill,stroke}
\pgfpathmoveto{\pgfpoint{205.633728pt}{193.781067pt}}
\pgflineto{\pgfpoint{200.206665pt}{192.017700pt}}
\pgflineto{\pgfpoint{201.706665pt}{190.927887pt}}
\pgfpathclose
\pgfusepath{fill,stroke}
\pgfpathmoveto{\pgfpoint{205.633728pt}{193.781067pt}}
\pgflineto{\pgfpoint{199.633728pt}{193.781067pt}}
\pgflineto{\pgfpoint{200.206665pt}{192.017700pt}}
\pgfpathclose
\pgfusepath{fill,stroke}
\pgfpathmoveto{\pgfpoint{205.633728pt}{193.781067pt}}
\pgflineto{\pgfpoint{200.206665pt}{195.544418pt}}
\pgflineto{\pgfpoint{199.633728pt}{193.781067pt}}
\pgfpathclose
\pgfusepath{fill,stroke}
\pgfpathmoveto{\pgfpoint{205.633728pt}{193.781067pt}}
\pgflineto{\pgfpoint{201.706665pt}{196.634232pt}}
\pgflineto{\pgfpoint{200.206665pt}{195.544418pt}}
\pgfpathclose
\pgfusepath{fill,stroke}
\pgfpathmoveto{\pgfpoint{205.633728pt}{193.781067pt}}
\pgflineto{\pgfpoint{203.560776pt}{196.634232pt}}
\pgflineto{\pgfpoint{201.706665pt}{196.634232pt}}
\pgfpathclose
\pgfusepath{fill,stroke}
\pgfpathmoveto{\pgfpoint{205.633728pt}{193.781067pt}}
\pgflineto{\pgfpoint{205.060776pt}{195.544418pt}}
\pgflineto{\pgfpoint{203.560776pt}{196.634232pt}}
\pgfpathclose
\pgfusepath{fill,stroke}
\pgfpathmoveto{\pgfpoint{177.394714pt}{158.108337pt}}
\pgflineto{\pgfpoint{175.321777pt}{155.255173pt}}
\pgflineto{\pgfpoint{176.821777pt}{156.344986pt}}
\pgfpathclose
\pgfusepath{fill,stroke}
\pgfpathmoveto{\pgfpoint{177.394714pt}{158.108337pt}}
\pgflineto{\pgfpoint{173.467682pt}{155.255173pt}}
\pgflineto{\pgfpoint{175.321777pt}{155.255173pt}}
\pgfpathclose
\pgfusepath{fill,stroke}
\pgfpathmoveto{\pgfpoint{177.394714pt}{158.108337pt}}
\pgflineto{\pgfpoint{171.967682pt}{156.344986pt}}
\pgflineto{\pgfpoint{173.467682pt}{155.255173pt}}
\pgfpathclose
\pgfusepath{fill,stroke}
\pgfpathmoveto{\pgfpoint{177.394714pt}{158.108337pt}}
\pgflineto{\pgfpoint{171.394730pt}{158.108337pt}}
\pgflineto{\pgfpoint{171.967682pt}{156.344986pt}}
\pgfpathclose
\pgfusepath{fill,stroke}
\pgfpathmoveto{\pgfpoint{177.394714pt}{158.108337pt}}
\pgflineto{\pgfpoint{171.967682pt}{159.871704pt}}
\pgflineto{\pgfpoint{171.394730pt}{158.108337pt}}
\pgfpathclose
\pgfusepath{fill,stroke}
\pgfpathmoveto{\pgfpoint{177.394714pt}{158.108337pt}}
\pgflineto{\pgfpoint{173.467682pt}{160.961517pt}}
\pgflineto{\pgfpoint{171.967682pt}{159.871704pt}}
\pgfpathclose
\pgfusepath{fill,stroke}
\pgfpathmoveto{\pgfpoint{177.394714pt}{158.108337pt}}
\pgflineto{\pgfpoint{175.321777pt}{160.961517pt}}
\pgflineto{\pgfpoint{173.467682pt}{160.961517pt}}
\pgfpathclose
\pgfusepath{fill,stroke}
\pgfpathmoveto{\pgfpoint{177.394714pt}{158.108337pt}}
\pgflineto{\pgfpoint{176.821777pt}{159.871704pt}}
\pgflineto{\pgfpoint{175.321777pt}{160.961517pt}}
\pgfpathclose
\pgfusepath{fill,stroke}
\pgfpathmoveto{\pgfpoint{213.823563pt}{195.386551pt}}
\pgflineto{\pgfpoint{211.750610pt}{192.533386pt}}
\pgflineto{\pgfpoint{213.250610pt}{193.623199pt}}
\pgfpathclose
\pgfusepath{fill,stroke}
\pgfpathmoveto{\pgfpoint{213.823563pt}{195.386551pt}}
\pgflineto{\pgfpoint{209.896515pt}{192.533386pt}}
\pgflineto{\pgfpoint{211.750610pt}{192.533386pt}}
\pgfpathclose
\pgfusepath{fill,stroke}
\pgfpathmoveto{\pgfpoint{213.823563pt}{195.386551pt}}
\pgflineto{\pgfpoint{208.396515pt}{193.623199pt}}
\pgflineto{\pgfpoint{209.896515pt}{192.533386pt}}
\pgfpathclose
\pgfusepath{fill,stroke}
\pgfpathmoveto{\pgfpoint{213.823563pt}{195.386551pt}}
\pgflineto{\pgfpoint{207.823578pt}{195.386551pt}}
\pgflineto{\pgfpoint{208.396515pt}{193.623199pt}}
\pgfpathclose
\pgfusepath{fill,stroke}
\pgfpathmoveto{\pgfpoint{213.823563pt}{195.386551pt}}
\pgflineto{\pgfpoint{208.396515pt}{197.149902pt}}
\pgflineto{\pgfpoint{207.823578pt}{195.386551pt}}
\pgfpathclose
\pgfusepath{fill,stroke}
\pgfpathmoveto{\pgfpoint{213.823563pt}{195.386551pt}}
\pgflineto{\pgfpoint{209.896515pt}{198.239716pt}}
\pgflineto{\pgfpoint{208.396515pt}{197.149902pt}}
\pgfpathclose
\pgfusepath{fill,stroke}
\pgfpathmoveto{\pgfpoint{213.823563pt}{195.386551pt}}
\pgflineto{\pgfpoint{211.750610pt}{198.239716pt}}
\pgflineto{\pgfpoint{209.896515pt}{198.239716pt}}
\pgfpathclose
\pgfusepath{fill,stroke}
\pgfpathmoveto{\pgfpoint{213.823563pt}{195.386551pt}}
\pgflineto{\pgfpoint{213.250610pt}{197.149902pt}}
\pgflineto{\pgfpoint{211.750610pt}{198.239716pt}}
\pgfpathclose
\pgfusepath{fill,stroke}
\pgfpathmoveto{\pgfpoint{174.663330pt}{172.078857pt}}
\pgflineto{\pgfpoint{172.590393pt}{169.225693pt}}
\pgflineto{\pgfpoint{174.090393pt}{170.315491pt}}
\pgfpathclose
\pgfusepath{fill,stroke}
\pgfpathmoveto{\pgfpoint{174.663330pt}{172.078857pt}}
\pgflineto{\pgfpoint{170.736282pt}{169.225693pt}}
\pgflineto{\pgfpoint{172.590393pt}{169.225693pt}}
\pgfpathclose
\pgfusepath{fill,stroke}
\pgfpathmoveto{\pgfpoint{174.663330pt}{172.078857pt}}
\pgflineto{\pgfpoint{169.236282pt}{170.315491pt}}
\pgflineto{\pgfpoint{170.736282pt}{169.225693pt}}
\pgfpathclose
\pgfusepath{fill,stroke}
\pgfpathmoveto{\pgfpoint{174.663330pt}{172.078857pt}}
\pgflineto{\pgfpoint{168.663330pt}{172.078857pt}}
\pgflineto{\pgfpoint{169.236282pt}{170.315491pt}}
\pgfpathclose
\pgfusepath{fill,stroke}
\pgfpathmoveto{\pgfpoint{174.663330pt}{172.078857pt}}
\pgflineto{\pgfpoint{169.236282pt}{173.842224pt}}
\pgflineto{\pgfpoint{168.663330pt}{172.078857pt}}
\pgfpathclose
\pgfusepath{fill,stroke}
\pgfpathmoveto{\pgfpoint{174.663330pt}{172.078857pt}}
\pgflineto{\pgfpoint{170.736282pt}{174.932022pt}}
\pgflineto{\pgfpoint{169.236282pt}{173.842224pt}}
\pgfpathclose
\pgfusepath{fill,stroke}
\pgfpathmoveto{\pgfpoint{174.663330pt}{172.078857pt}}
\pgflineto{\pgfpoint{172.590393pt}{174.932022pt}}
\pgflineto{\pgfpoint{170.736282pt}{174.932022pt}}
\pgfpathclose
\pgfusepath{fill,stroke}
\pgfpathmoveto{\pgfpoint{174.663330pt}{172.078857pt}}
\pgflineto{\pgfpoint{174.090393pt}{173.842224pt}}
\pgflineto{\pgfpoint{172.590393pt}{174.932022pt}}
\pgfpathclose
\pgfusepath{fill,stroke}
\pgfpathmoveto{\pgfpoint{169.091446pt}{166.417404pt}}
\pgflineto{\pgfpoint{167.018494pt}{163.564240pt}}
\pgflineto{\pgfpoint{168.518494pt}{164.654053pt}}
\pgfpathclose
\pgfusepath{fill,stroke}
\pgfpathmoveto{\pgfpoint{169.091446pt}{166.417404pt}}
\pgflineto{\pgfpoint{165.164398pt}{163.564240pt}}
\pgflineto{\pgfpoint{167.018494pt}{163.564240pt}}
\pgfpathclose
\pgfusepath{fill,stroke}
\pgfpathmoveto{\pgfpoint{169.091446pt}{166.417404pt}}
\pgflineto{\pgfpoint{163.664398pt}{164.654053pt}}
\pgflineto{\pgfpoint{165.164398pt}{163.564240pt}}
\pgfpathclose
\pgfusepath{fill,stroke}
\pgfpathmoveto{\pgfpoint{169.091446pt}{166.417404pt}}
\pgflineto{\pgfpoint{163.091446pt}{166.417404pt}}
\pgflineto{\pgfpoint{163.664398pt}{164.654053pt}}
\pgfpathclose
\pgfusepath{fill,stroke}
\pgfpathmoveto{\pgfpoint{169.091446pt}{166.417404pt}}
\pgflineto{\pgfpoint{163.664398pt}{168.180756pt}}
\pgflineto{\pgfpoint{163.091446pt}{166.417404pt}}
\pgfpathclose
\pgfusepath{fill,stroke}
\pgfpathmoveto{\pgfpoint{169.091446pt}{166.417404pt}}
\pgflineto{\pgfpoint{165.164398pt}{169.270569pt}}
\pgflineto{\pgfpoint{163.664398pt}{168.180756pt}}
\pgfpathclose
\pgfusepath{fill,stroke}
\pgfpathmoveto{\pgfpoint{169.091446pt}{166.417404pt}}
\pgflineto{\pgfpoint{167.018494pt}{169.270569pt}}
\pgflineto{\pgfpoint{165.164398pt}{169.270569pt}}
\pgfpathclose
\pgfusepath{fill,stroke}
\pgfpathmoveto{\pgfpoint{169.091446pt}{166.417404pt}}
\pgflineto{\pgfpoint{168.518494pt}{168.180756pt}}
\pgflineto{\pgfpoint{167.018494pt}{169.270569pt}}
\pgfpathclose
\pgfusepath{fill,stroke}
\pgfpathmoveto{\pgfpoint{201.397018pt}{180.402039pt}}
\pgflineto{\pgfpoint{199.324066pt}{177.548859pt}}
\pgflineto{\pgfpoint{200.824066pt}{178.638672pt}}
\pgfpathclose
\pgfusepath{fill,stroke}
\pgfpathmoveto{\pgfpoint{201.397018pt}{180.402039pt}}
\pgflineto{\pgfpoint{197.469971pt}{177.548859pt}}
\pgflineto{\pgfpoint{199.324066pt}{177.548859pt}}
\pgfpathclose
\pgfusepath{fill,stroke}
\pgfpathmoveto{\pgfpoint{201.397018pt}{180.402039pt}}
\pgflineto{\pgfpoint{195.969971pt}{178.638672pt}}
\pgflineto{\pgfpoint{197.469971pt}{177.548859pt}}
\pgfpathclose
\pgfusepath{fill,stroke}
\pgfpathmoveto{\pgfpoint{201.397018pt}{180.402039pt}}
\pgflineto{\pgfpoint{195.397018pt}{180.402039pt}}
\pgflineto{\pgfpoint{195.969971pt}{178.638672pt}}
\pgfpathclose
\pgfusepath{fill,stroke}
\pgfpathmoveto{\pgfpoint{201.397018pt}{180.402039pt}}
\pgflineto{\pgfpoint{195.969971pt}{182.165390pt}}
\pgflineto{\pgfpoint{195.397018pt}{180.402039pt}}
\pgfpathclose
\pgfusepath{fill,stroke}
\pgfpathmoveto{\pgfpoint{201.397018pt}{180.402039pt}}
\pgflineto{\pgfpoint{197.469971pt}{183.255203pt}}
\pgflineto{\pgfpoint{195.969971pt}{182.165390pt}}
\pgfpathclose
\pgfusepath{fill,stroke}
\pgfpathmoveto{\pgfpoint{201.397018pt}{180.402039pt}}
\pgflineto{\pgfpoint{199.324066pt}{183.255203pt}}
\pgflineto{\pgfpoint{197.469971pt}{183.255203pt}}
\pgfpathclose
\pgfusepath{fill,stroke}
\pgfpathmoveto{\pgfpoint{201.397018pt}{180.402039pt}}
\pgflineto{\pgfpoint{200.824066pt}{182.165390pt}}
\pgflineto{\pgfpoint{199.324066pt}{183.255203pt}}
\pgfpathclose
\pgfusepath{fill,stroke}
\pgfpathmoveto{\pgfpoint{175.252380pt}{197.646912pt}}
\pgflineto{\pgfpoint{173.179413pt}{194.793747pt}}
\pgflineto{\pgfpoint{174.679428pt}{195.883560pt}}
\pgfpathclose
\pgfusepath{fill,stroke}
\pgfpathmoveto{\pgfpoint{175.252380pt}{197.646912pt}}
\pgflineto{\pgfpoint{171.325317pt}{194.793747pt}}
\pgflineto{\pgfpoint{173.179413pt}{194.793747pt}}
\pgfpathclose
\pgfusepath{fill,stroke}
\pgfpathmoveto{\pgfpoint{175.252380pt}{197.646912pt}}
\pgflineto{\pgfpoint{169.825317pt}{195.883560pt}}
\pgflineto{\pgfpoint{171.325317pt}{194.793747pt}}
\pgfpathclose
\pgfusepath{fill,stroke}
\pgfpathmoveto{\pgfpoint{175.252380pt}{197.646912pt}}
\pgflineto{\pgfpoint{169.252380pt}{197.646912pt}}
\pgflineto{\pgfpoint{169.825317pt}{195.883560pt}}
\pgfpathclose
\pgfusepath{fill,stroke}
\pgfpathmoveto{\pgfpoint{175.252380pt}{197.646912pt}}
\pgflineto{\pgfpoint{169.825317pt}{199.410278pt}}
\pgflineto{\pgfpoint{169.252380pt}{197.646912pt}}
\pgfpathclose
\pgfusepath{fill,stroke}
\pgfpathmoveto{\pgfpoint{175.252380pt}{197.646912pt}}
\pgflineto{\pgfpoint{171.325317pt}{200.500092pt}}
\pgflineto{\pgfpoint{169.825317pt}{199.410278pt}}
\pgfpathclose
\pgfusepath{fill,stroke}
\pgfpathmoveto{\pgfpoint{175.252380pt}{197.646912pt}}
\pgflineto{\pgfpoint{173.179413pt}{200.500092pt}}
\pgflineto{\pgfpoint{171.325317pt}{200.500092pt}}
\pgfpathclose
\pgfusepath{fill,stroke}
\pgfpathmoveto{\pgfpoint{175.252380pt}{197.646912pt}}
\pgflineto{\pgfpoint{174.679428pt}{199.410278pt}}
\pgflineto{\pgfpoint{173.179413pt}{200.500092pt}}
\pgfpathclose
\pgfusepath{fill,stroke}
\pgfpathmoveto{\pgfpoint{181.949951pt}{190.006775pt}}
\pgflineto{\pgfpoint{179.877014pt}{187.153595pt}}
\pgflineto{\pgfpoint{181.377014pt}{188.243408pt}}
\pgfpathclose
\pgfusepath{fill,stroke}
\pgfpathmoveto{\pgfpoint{181.949951pt}{190.006775pt}}
\pgflineto{\pgfpoint{178.022919pt}{187.153595pt}}
\pgflineto{\pgfpoint{179.877014pt}{187.153595pt}}
\pgfpathclose
\pgfusepath{fill,stroke}
\pgfpathmoveto{\pgfpoint{181.949951pt}{190.006775pt}}
\pgflineto{\pgfpoint{176.522919pt}{188.243408pt}}
\pgflineto{\pgfpoint{178.022919pt}{187.153595pt}}
\pgfpathclose
\pgfusepath{fill,stroke}
\pgfpathmoveto{\pgfpoint{181.949951pt}{190.006775pt}}
\pgflineto{\pgfpoint{175.949966pt}{190.006775pt}}
\pgflineto{\pgfpoint{176.522919pt}{188.243408pt}}
\pgfpathclose
\pgfusepath{fill,stroke}
\pgfpathmoveto{\pgfpoint{181.949951pt}{190.006775pt}}
\pgflineto{\pgfpoint{176.522919pt}{191.770126pt}}
\pgflineto{\pgfpoint{175.949966pt}{190.006775pt}}
\pgfpathclose
\pgfusepath{fill,stroke}
\pgfpathmoveto{\pgfpoint{181.949951pt}{190.006775pt}}
\pgflineto{\pgfpoint{178.022919pt}{192.859940pt}}
\pgflineto{\pgfpoint{176.522919pt}{191.770126pt}}
\pgfpathclose
\pgfusepath{fill,stroke}
\pgfpathmoveto{\pgfpoint{181.949951pt}{190.006775pt}}
\pgflineto{\pgfpoint{179.877014pt}{192.859940pt}}
\pgflineto{\pgfpoint{178.022919pt}{192.859940pt}}
\pgfpathclose
\pgfusepath{fill,stroke}
\pgfpathmoveto{\pgfpoint{181.949951pt}{190.006775pt}}
\pgflineto{\pgfpoint{181.377014pt}{191.770126pt}}
\pgflineto{\pgfpoint{179.877014pt}{192.859940pt}}
\pgfpathclose
\pgfusepath{fill,stroke}
\pgfpathmoveto{\pgfpoint{178.734253pt}{224.362747pt}}
\pgflineto{\pgfpoint{176.661301pt}{221.509583pt}}
\pgflineto{\pgfpoint{178.161301pt}{222.599396pt}}
\pgfpathclose
\pgfusepath{fill,stroke}
\pgfpathmoveto{\pgfpoint{178.734253pt}{224.362747pt}}
\pgflineto{\pgfpoint{174.807190pt}{221.509583pt}}
\pgflineto{\pgfpoint{176.661301pt}{221.509583pt}}
\pgfpathclose
\pgfusepath{fill,stroke}
\pgfpathmoveto{\pgfpoint{178.734253pt}{224.362747pt}}
\pgflineto{\pgfpoint{173.307190pt}{222.599396pt}}
\pgflineto{\pgfpoint{174.807190pt}{221.509583pt}}
\pgfpathclose
\pgfusepath{fill,stroke}
\pgfpathmoveto{\pgfpoint{178.734253pt}{224.362747pt}}
\pgflineto{\pgfpoint{172.734253pt}{224.362747pt}}
\pgflineto{\pgfpoint{173.307190pt}{222.599396pt}}
\pgfpathclose
\pgfusepath{fill,stroke}
\pgfpathmoveto{\pgfpoint{178.734253pt}{224.362747pt}}
\pgflineto{\pgfpoint{173.307190pt}{226.126099pt}}
\pgflineto{\pgfpoint{172.734253pt}{224.362747pt}}
\pgfpathclose
\pgfusepath{fill,stroke}
\pgfpathmoveto{\pgfpoint{178.734253pt}{224.362747pt}}
\pgflineto{\pgfpoint{174.807190pt}{227.215912pt}}
\pgflineto{\pgfpoint{173.307190pt}{226.126099pt}}
\pgfpathclose
\pgfusepath{fill,stroke}
\pgfpathmoveto{\pgfpoint{178.734253pt}{224.362747pt}}
\pgflineto{\pgfpoint{176.661301pt}{227.215912pt}}
\pgflineto{\pgfpoint{174.807190pt}{227.215912pt}}
\pgfpathclose
\pgfusepath{fill,stroke}
\pgfpathmoveto{\pgfpoint{178.734253pt}{224.362747pt}}
\pgflineto{\pgfpoint{178.161301pt}{226.126099pt}}
\pgflineto{\pgfpoint{176.661301pt}{227.215912pt}}
\pgfpathclose
\pgfusepath{fill,stroke}
\pgfpathmoveto{\pgfpoint{177.333649pt}{228.580658pt}}
\pgflineto{\pgfpoint{175.260696pt}{225.727478pt}}
\pgflineto{\pgfpoint{176.760696pt}{226.817291pt}}
\pgfpathclose
\pgfusepath{fill,stroke}
\pgfpathmoveto{\pgfpoint{177.333649pt}{228.580658pt}}
\pgflineto{\pgfpoint{173.406601pt}{225.727478pt}}
\pgflineto{\pgfpoint{175.260696pt}{225.727478pt}}
\pgfpathclose
\pgfusepath{fill,stroke}
\pgfpathmoveto{\pgfpoint{177.333649pt}{228.580658pt}}
\pgflineto{\pgfpoint{171.906586pt}{226.817291pt}}
\pgflineto{\pgfpoint{173.406601pt}{225.727478pt}}
\pgfpathclose
\pgfusepath{fill,stroke}
\pgfpathmoveto{\pgfpoint{177.333649pt}{228.580658pt}}
\pgflineto{\pgfpoint{171.333649pt}{228.580658pt}}
\pgflineto{\pgfpoint{171.906586pt}{226.817291pt}}
\pgfpathclose
\pgfusepath{fill,stroke}
\pgfpathmoveto{\pgfpoint{177.333649pt}{228.580658pt}}
\pgflineto{\pgfpoint{171.906586pt}{230.344009pt}}
\pgflineto{\pgfpoint{171.333649pt}{228.580658pt}}
\pgfpathclose
\pgfusepath{fill,stroke}
\pgfpathmoveto{\pgfpoint{177.333649pt}{228.580658pt}}
\pgflineto{\pgfpoint{173.406601pt}{231.433823pt}}
\pgflineto{\pgfpoint{171.906586pt}{230.344009pt}}
\pgfpathclose
\pgfusepath{fill,stroke}
\pgfpathmoveto{\pgfpoint{177.333649pt}{228.580658pt}}
\pgflineto{\pgfpoint{175.260696pt}{231.433823pt}}
\pgflineto{\pgfpoint{173.406601pt}{231.433823pt}}
\pgfpathclose
\pgfusepath{fill,stroke}
\pgfpathmoveto{\pgfpoint{177.333649pt}{228.580658pt}}
\pgflineto{\pgfpoint{176.760696pt}{230.344009pt}}
\pgflineto{\pgfpoint{175.260696pt}{231.433823pt}}
\pgfpathclose
\pgfusepath{fill,stroke}
\pgfpathmoveto{\pgfpoint{191.213165pt}{204.075897pt}}
\pgflineto{\pgfpoint{189.140228pt}{201.222733pt}}
\pgflineto{\pgfpoint{190.640228pt}{202.312546pt}}
\pgfpathclose
\pgfusepath{fill,stroke}
\pgfpathmoveto{\pgfpoint{191.213165pt}{204.075897pt}}
\pgflineto{\pgfpoint{187.286133pt}{201.222733pt}}
\pgflineto{\pgfpoint{189.140228pt}{201.222733pt}}
\pgfpathclose
\pgfusepath{fill,stroke}
\pgfpathmoveto{\pgfpoint{191.213165pt}{204.075897pt}}
\pgflineto{\pgfpoint{185.786118pt}{202.312546pt}}
\pgflineto{\pgfpoint{187.286133pt}{201.222733pt}}
\pgfpathclose
\pgfusepath{fill,stroke}
\pgfpathmoveto{\pgfpoint{191.213165pt}{204.075897pt}}
\pgflineto{\pgfpoint{185.213165pt}{204.075897pt}}
\pgflineto{\pgfpoint{185.786118pt}{202.312546pt}}
\pgfpathclose
\pgfusepath{fill,stroke}
\pgfpathmoveto{\pgfpoint{191.213165pt}{204.075897pt}}
\pgflineto{\pgfpoint{185.786118pt}{205.839264pt}}
\pgflineto{\pgfpoint{185.213165pt}{204.075897pt}}
\pgfpathclose
\pgfusepath{fill,stroke}
\pgfpathmoveto{\pgfpoint{191.213165pt}{204.075897pt}}
\pgflineto{\pgfpoint{187.286133pt}{206.929077pt}}
\pgflineto{\pgfpoint{185.786118pt}{205.839264pt}}
\pgfpathclose
\pgfusepath{fill,stroke}
\pgfpathmoveto{\pgfpoint{191.213165pt}{204.075897pt}}
\pgflineto{\pgfpoint{189.140228pt}{206.929077pt}}
\pgflineto{\pgfpoint{187.286133pt}{206.929077pt}}
\pgfpathclose
\pgfusepath{fill,stroke}
\pgfpathmoveto{\pgfpoint{191.213165pt}{204.075897pt}}
\pgflineto{\pgfpoint{190.640228pt}{205.839264pt}}
\pgflineto{\pgfpoint{189.140228pt}{206.929077pt}}
\pgfpathclose
\pgfusepath{fill,stroke}
\end{pgfscope}
\end{pgfpicture}
}
			\caption{\label{fig:cluster}Translaciones estimadas entre cada par
			de puntos de las correspondencias.}
		\end{figure}

		\section{Resultados}
		%\TODO{revisar}

			Se realizaron pruebas preliminares sobre los objetos \texttt{happy} y \texttt{bunny}.
			Las capturas de \texttt{happy} se realizaron a intervalos de ángulos equiespaciados (cercanos a $24^{\circ}$) y
			presentan zonas suaves en la base, barriga y espalda.
			En cambio los ángulos en \texttt{bunny} son más variados y espaciados, yendo desde $35^{\circ}$ a $90^{\circ}$,
			y se observa una superficie más irregular por todo el cuerpo (figura~\ref{fig:stanford_models}).%\TODO{Sacar foto a los modelos}


			En el caso de \texttt{bunny} se presentan saltos cercanos a $90^{\circ}$ al realizar las alineaciones
			contra la captura \texttt{bun180}. Si bien este salto supera las restricciones impuestas,
			en ningún caso se produjo una diferencia mayor a $5^{\circ}$ respecto al \emph{ground truth} (figura~\ref{fig:clust_bunny}).

			Para \texttt{happy} se obtuvieron resultados similares a pesar de que en este
			caso los ángulos eran más cercanos (aproximadamente $24^{\circ}$).  Las
			diferencias respecto al \emph{ground truth} no superaban los $5^{\circ}$,
			teniendo un error promedio de $2^{\circ}$ (figura~\ref{fig:clust_happy}).

			De esta forma, se logró acercar las capturas lo suficiente como
			para intentar alinearlas por ICP.

		\subsubsection{Resultados} %ver de eliminar
			Mediante este método, se obtuvieron buenos resultados en la mayoría de las
			capturas de \texttt{happy}
			donde los ángulos eran cercanos a $25^{\circ}$ (figura~\ref{fig:sac_angles}).

			\begin{figure}
				\TODO{unir con el de clustering}
				\Imagen{example-image-a}
				\caption{\label{fig:sac_angles}\TODO{Diferencias entre la rotación estimada mediante \emph{sample consensus} y el \emph{ground truth} para el objeto \texttt{happy}}}
			\end{figure}
			%Sin embargo, algunas alineaciones presentaban problemas de deslizamiento. \TODO{mostrar alguno}
			%era ISS donde se veía eso, ya no lo tengo

			En el caso de \texttt{bunny}, cuyos ángulos eran cercanos a $45^{\circ}$,
			se obtuvo una buena alineación en el caso de las capturas 315--000 y 000--045,
			sin embargo, para los otros casos las correspondencias fueron completamente erróneas (figura~\ref{fig:align_sac}).


		\begin{figure}
			%\Imagen{img/cluster_bunny}
			\resizebox{.9\linewidth}{!}{% Title: gl2ps_renderer figure
% Creator: GL2PS 1.4.0, (C) 1999-2017 C. Geuzaine
% For: Octave
% CreationDate: Wed Feb 26 13:03:30 2020
\begin{pgfpicture}
\color[rgb]{1.000000,1.000000,1.000000}
\pgfpathrectanglecorners{\pgfpoint{0pt}{0pt}}{\pgfpoint{576pt}{432pt}}
\pgfusepath{fill}
\begin{pgfscope}
\pgfpathrectangle{\pgfpoint{0pt}{0pt}}{\pgfpoint{576pt}{432pt}}
\pgfusepath{fill}
\pgfpathrectangle{\pgfpoint{0pt}{0pt}}{\pgfpoint{576pt}{432pt}}
\pgfusepath{clip}
\pgfpathmoveto{\pgfpoint{74.880020pt}{399.599976pt}}
\pgflineto{\pgfpoint{521.280029pt}{47.519989pt}}
\pgflineto{\pgfpoint{74.880020pt}{47.519989pt}}
\pgfpathclose
\pgfusepath{fill,stroke}
\pgfpathmoveto{\pgfpoint{74.880020pt}{399.599976pt}}
\pgflineto{\pgfpoint{521.280029pt}{399.599976pt}}
\pgflineto{\pgfpoint{521.280029pt}{47.519989pt}}
\pgfpathclose
\pgfusepath{fill,stroke}
\color[rgb]{0.150000,0.150000,0.150000}
\pgfsetlinewidth{0.500000pt}
\pgfpathmoveto{\pgfpoint{74.880020pt}{51.985016pt}}
\pgflineto{\pgfpoint{74.880020pt}{47.519989pt}}
\pgfusepath{stroke}
\pgfpathmoveto{\pgfpoint{74.880020pt}{395.134979pt}}
\pgflineto{\pgfpoint{74.880020pt}{399.599976pt}}
\pgfusepath{stroke}
\pgfpathmoveto{\pgfpoint{164.160019pt}{51.985016pt}}
\pgflineto{\pgfpoint{164.160019pt}{47.519989pt}}
\pgfusepath{stroke}
\pgfpathmoveto{\pgfpoint{164.160019pt}{395.134979pt}}
\pgflineto{\pgfpoint{164.160019pt}{399.599976pt}}
\pgfusepath{stroke}
\pgfpathmoveto{\pgfpoint{253.440018pt}{51.985016pt}}
\pgflineto{\pgfpoint{253.440018pt}{47.519989pt}}
\pgfusepath{stroke}
\pgfpathmoveto{\pgfpoint{253.440018pt}{395.134979pt}}
\pgflineto{\pgfpoint{253.440018pt}{399.599976pt}}
\pgfusepath{stroke}
\pgfpathmoveto{\pgfpoint{342.720032pt}{51.985016pt}}
\pgflineto{\pgfpoint{342.720032pt}{47.519989pt}}
\pgfusepath{stroke}
\pgfpathmoveto{\pgfpoint{342.720032pt}{395.134979pt}}
\pgflineto{\pgfpoint{342.720032pt}{399.599976pt}}
\pgfusepath{stroke}
\pgfpathmoveto{\pgfpoint{432.000000pt}{51.985016pt}}
\pgflineto{\pgfpoint{432.000000pt}{47.519989pt}}
\pgfusepath{stroke}
\pgfpathmoveto{\pgfpoint{432.000000pt}{395.134979pt}}
\pgflineto{\pgfpoint{432.000000pt}{399.599976pt}}
\pgfusepath{stroke}
\pgfpathmoveto{\pgfpoint{521.280029pt}{51.985016pt}}
\pgflineto{\pgfpoint{521.280029pt}{47.519989pt}}
\pgfusepath{stroke}
\pgfpathmoveto{\pgfpoint{521.280029pt}{395.134979pt}}
\pgflineto{\pgfpoint{521.280029pt}{399.599976pt}}
\pgfusepath{stroke}
{
\pgftransformshift{\pgfpoint{74.880005pt}{40.018295pt}}
\pgfnode{rectangle}{north}{\fontsize{10}{0}\selectfont\textcolor[rgb]{0.15,0.15,0.15}{{45}}}{}{\pgfusepath{discard}}}
{
\pgftransformshift{\pgfpoint{164.160004pt}{40.018295pt}}
\pgfnode{rectangle}{north}{\fontsize{10}{0}\selectfont\textcolor[rgb]{0.15,0.15,0.15}{{90}}}{}{\pgfusepath{discard}}}
{
\pgftransformshift{\pgfpoint{253.440002pt}{40.018295pt}}
\pgfnode{rectangle}{north}{\fontsize{10}{0}\selectfont\textcolor[rgb]{0.15,0.15,0.15}{{180}}}{}{\pgfusepath{discard}}}
{
\pgftransformshift{\pgfpoint{342.720001pt}{40.018295pt}}
\pgfnode{rectangle}{north}{\fontsize{10}{0}\selectfont\textcolor[rgb]{0.15,0.15,0.15}{{270}}}{}{\pgfusepath{discard}}}
{
\pgftransformshift{\pgfpoint{432.000000pt}{40.018295pt}}
\pgfnode{rectangle}{north}{\fontsize{10}{0}\selectfont\textcolor[rgb]{0.15,0.15,0.15}{{315}}}{}{\pgfusepath{discard}}}
{
\pgftransformshift{\pgfpoint{521.280029pt}{40.018295pt}}
\pgfnode{rectangle}{north}{\fontsize{10}{0}\selectfont\textcolor[rgb]{0.15,0.15,0.15}{{0}}}{}{\pgfusepath{discard}}}
\pgfpathmoveto{\pgfpoint{79.348038pt}{47.519989pt}}
\pgflineto{\pgfpoint{74.880020pt}{47.519989pt}}
\pgfusepath{stroke}
\pgfpathmoveto{\pgfpoint{516.812012pt}{47.519989pt}}
\pgflineto{\pgfpoint{521.280029pt}{47.519989pt}}
\pgfusepath{stroke}
\pgfpathmoveto{\pgfpoint{79.348038pt}{117.935997pt}}
\pgflineto{\pgfpoint{74.880020pt}{117.935997pt}}
\pgfusepath{stroke}
\pgfpathmoveto{\pgfpoint{516.812012pt}{117.935997pt}}
\pgflineto{\pgfpoint{521.280029pt}{117.935997pt}}
\pgfusepath{stroke}
\pgfpathmoveto{\pgfpoint{79.348038pt}{188.351990pt}}
\pgflineto{\pgfpoint{74.880020pt}{188.351990pt}}
\pgfusepath{stroke}
\pgfpathmoveto{\pgfpoint{516.812012pt}{188.351990pt}}
\pgflineto{\pgfpoint{521.280029pt}{188.351990pt}}
\pgfusepath{stroke}
\pgfpathmoveto{\pgfpoint{79.348038pt}{258.768005pt}}
\pgflineto{\pgfpoint{74.880020pt}{258.768005pt}}
\pgfusepath{stroke}
\pgfpathmoveto{\pgfpoint{516.812012pt}{258.768005pt}}
\pgflineto{\pgfpoint{521.280029pt}{258.768005pt}}
\pgfusepath{stroke}
\pgfpathmoveto{\pgfpoint{79.348038pt}{329.183990pt}}
\pgflineto{\pgfpoint{74.880020pt}{329.183990pt}}
\pgfusepath{stroke}
\pgfpathmoveto{\pgfpoint{516.812012pt}{329.183990pt}}
\pgflineto{\pgfpoint{521.280029pt}{329.183990pt}}
\pgfusepath{stroke}
\pgfpathmoveto{\pgfpoint{79.348038pt}{399.599976pt}}
\pgflineto{\pgfpoint{74.880020pt}{399.599976pt}}
\pgfusepath{stroke}
\pgfpathmoveto{\pgfpoint{516.812012pt}{399.599976pt}}
\pgflineto{\pgfpoint{521.280029pt}{399.599976pt}}
\pgfusepath{stroke}
{
\pgftransformshift{\pgfpoint{69.875519pt}{47.519989pt}}
\pgfnode{rectangle}{east}{\fontsize{10}{0}\selectfont\textcolor[rgb]{0.15,0.15,0.15}{{0}}}{}{\pgfusepath{discard}}}
{
\pgftransformshift{\pgfpoint{69.875519pt}{117.935989pt}}
\pgfnode{rectangle}{east}{\fontsize{10}{0}\selectfont\textcolor[rgb]{0.15,0.15,0.15}{{1}}}{}{\pgfusepath{discard}}}
{
\pgftransformshift{\pgfpoint{69.875519pt}{188.351990pt}}
\pgfnode{rectangle}{east}{\fontsize{10}{0}\selectfont\textcolor[rgb]{0.15,0.15,0.15}{{2}}}{}{\pgfusepath{discard}}}
{
\pgftransformshift{\pgfpoint{69.875519pt}{258.768005pt}}
\pgfnode{rectangle}{east}{\fontsize{10}{0}\selectfont\textcolor[rgb]{0.15,0.15,0.15}{{3}}}{}{\pgfusepath{discard}}}
{
\pgftransformshift{\pgfpoint{69.875519pt}{329.183990pt}}
\pgfnode{rectangle}{east}{\fontsize{10}{0}\selectfont\textcolor[rgb]{0.15,0.15,0.15}{{4}}}{}{\pgfusepath{discard}}}
{
\pgftransformshift{\pgfpoint{69.875519pt}{399.599976pt}}
\pgfnode{rectangle}{east}{\fontsize{10}{0}\selectfont\textcolor[rgb]{0.15,0.15,0.15}{{5}}}{}{\pgfusepath{discard}}}
\pgfsetrectcap
\pgfsetdash{{16pt}{0pt}}{0pt}
\pgfpathmoveto{\pgfpoint{521.280029pt}{47.519989pt}}
\pgflineto{\pgfpoint{74.880020pt}{47.519989pt}}
\pgfusepath{stroke}
\pgfpathmoveto{\pgfpoint{521.280029pt}{399.599976pt}}
\pgflineto{\pgfpoint{74.880020pt}{399.599976pt}}
\pgfusepath{stroke}
\pgfpathmoveto{\pgfpoint{74.880020pt}{399.599976pt}}
\pgflineto{\pgfpoint{74.880020pt}{47.519989pt}}
\pgfusepath{stroke}
\pgfpathmoveto{\pgfpoint{521.280029pt}{399.599976pt}}
\pgflineto{\pgfpoint{521.280029pt}{47.519989pt}}
\pgfusepath{stroke}
{
\pgftransformshift{\pgfpoint{298.079987pt}{27.018295pt}}
\pgfnode{rectangle}{north}{\fontsize{11}{0}\selectfont\textcolor[rgb]{0.15,0.15,0.15}{{captura}}}{}{\pgfusepath{discard}}}
{
\pgftransformshift{\pgfpoint{58.875519pt}{223.559998pt}}
\pgftransformrotate{90.000000}{\pgfnode{rectangle}{south}{\fontsize{11}{0}\selectfont\textcolor[rgb]{0.15,0.15,0.15}{{ángulo (grados)}}}{}{\pgfusepath{discard}}}}
\color[rgb]{0.000000,0.000000,0.000000}
\pgfsetbuttcap
\pgfsetroundjoin
\pgfsetdash{}{0pt}
\pgfpathmoveto{\pgfpoint{164.160019pt}{94.410004pt}}
\pgflineto{\pgfpoint{74.880020pt}{84.727814pt}}
\pgfusepath{stroke}
\pgfpathmoveto{\pgfpoint{253.440018pt}{338.880310pt}}
\pgflineto{\pgfpoint{164.160019pt}{94.410004pt}}
\pgfusepath{stroke}
\pgfpathmoveto{\pgfpoint{342.720032pt}{63.103058pt}}
\pgflineto{\pgfpoint{253.440018pt}{338.880310pt}}
\pgfusepath{stroke}
\pgfpathmoveto{\pgfpoint{432.000000pt}{193.823318pt}}
\pgflineto{\pgfpoint{342.720032pt}{63.103058pt}}
\pgfusepath{stroke}
\pgfpathmoveto{\pgfpoint{521.280029pt}{160.699646pt}}
\pgflineto{\pgfpoint{432.000000pt}{193.823318pt}}
\pgfusepath{stroke}
\pgfsetrectcap
\pgfsetmiterjoin
\pgfpathmoveto{\pgfpoint{77.880005pt}{84.727814pt}}
\pgflineto{\pgfpoint{71.879990pt}{84.727814pt}}
\pgfusepath{stroke}
\pgfpathmoveto{\pgfpoint{74.880005pt}{81.727814pt}}
\pgflineto{\pgfpoint{74.880005pt}{87.727814pt}}
\pgfusepath{stroke}
\pgfpathmoveto{\pgfpoint{77.001312pt}{82.606491pt}}
\pgflineto{\pgfpoint{72.758667pt}{86.849136pt}}
\pgfusepath{stroke}
\pgfpathmoveto{\pgfpoint{77.001312pt}{86.849136pt}}
\pgflineto{\pgfpoint{72.758667pt}{82.606491pt}}
\pgfusepath{stroke}
\pgfpathmoveto{\pgfpoint{167.160004pt}{94.410004pt}}
\pgflineto{\pgfpoint{161.160004pt}{94.410004pt}}
\pgfusepath{stroke}
\pgfpathmoveto{\pgfpoint{164.160004pt}{91.410004pt}}
\pgflineto{\pgfpoint{164.160004pt}{97.410011pt}}
\pgfusepath{stroke}
\pgfpathmoveto{\pgfpoint{166.281311pt}{92.288681pt}}
\pgflineto{\pgfpoint{162.038666pt}{96.531326pt}}
\pgfusepath{stroke}
\pgfpathmoveto{\pgfpoint{166.281311pt}{96.531326pt}}
\pgflineto{\pgfpoint{162.038666pt}{92.288681pt}}
\pgfusepath{stroke}
\pgfpathmoveto{\pgfpoint{256.440002pt}{338.880280pt}}
\pgflineto{\pgfpoint{250.440002pt}{338.880280pt}}
\pgfusepath{stroke}
\pgfpathmoveto{\pgfpoint{253.440002pt}{335.880280pt}}
\pgflineto{\pgfpoint{253.440002pt}{341.880280pt}}
\pgfusepath{stroke}
\pgfpathmoveto{\pgfpoint{255.561310pt}{336.758972pt}}
\pgflineto{\pgfpoint{251.318680pt}{341.001617pt}}
\pgfusepath{stroke}
\pgfpathmoveto{\pgfpoint{255.561310pt}{341.001617pt}}
\pgflineto{\pgfpoint{251.318680pt}{336.758972pt}}
\pgfusepath{stroke}
\pgfpathmoveto{\pgfpoint{345.720032pt}{63.103058pt}}
\pgflineto{\pgfpoint{339.720032pt}{63.103058pt}}
\pgfusepath{stroke}
\pgfpathmoveto{\pgfpoint{342.720032pt}{60.103058pt}}
\pgflineto{\pgfpoint{342.720032pt}{66.103073pt}}
\pgfusepath{stroke}
\pgfpathmoveto{\pgfpoint{344.841339pt}{60.981750pt}}
\pgflineto{\pgfpoint{340.598694pt}{65.224396pt}}
\pgfusepath{stroke}
\pgfpathmoveto{\pgfpoint{344.841339pt}{65.224396pt}}
\pgflineto{\pgfpoint{340.598694pt}{60.981750pt}}
\pgfusepath{stroke}
\pgfpathmoveto{\pgfpoint{435.000000pt}{193.823318pt}}
\pgflineto{\pgfpoint{429.000000pt}{193.823318pt}}
\pgfusepath{stroke}
\pgfpathmoveto{\pgfpoint{432.000000pt}{190.823318pt}}
\pgflineto{\pgfpoint{432.000000pt}{196.823318pt}}
\pgfusepath{stroke}
\pgfpathmoveto{\pgfpoint{434.121338pt}{191.701996pt}}
\pgflineto{\pgfpoint{429.878662pt}{195.944641pt}}
\pgfusepath{stroke}
\pgfpathmoveto{\pgfpoint{434.121338pt}{195.944641pt}}
\pgflineto{\pgfpoint{429.878662pt}{191.701996pt}}
\pgfusepath{stroke}
\pgfpathmoveto{\pgfpoint{524.280029pt}{160.699646pt}}
\pgflineto{\pgfpoint{518.280029pt}{160.699646pt}}
\pgfusepath{stroke}
\pgfpathmoveto{\pgfpoint{521.280029pt}{157.699646pt}}
\pgflineto{\pgfpoint{521.280029pt}{163.699646pt}}
\pgfusepath{stroke}
\pgfpathmoveto{\pgfpoint{523.401367pt}{158.578308pt}}
\pgflineto{\pgfpoint{519.158691pt}{162.820953pt}}
\pgfusepath{stroke}
\pgfpathmoveto{\pgfpoint{523.401367pt}{162.820953pt}}
\pgflineto{\pgfpoint{519.158691pt}{158.578308pt}}
\pgfusepath{stroke}
\end{pgfscope}
\end{pgfpicture}
}
			\caption{\label{fig:clust_bunny}Diferencias absolutas entre la rotación estimada y el \emph{ground truth} para el objeto \texttt{bunny}.}
		\end{figure}

		\begin{figure}
			%\Imagen{img/cluster_happy}
			\resizebox{.9\linewidth}{!}{% Title: gl2ps_renderer figure
% Creator: GL2PS 1.4.0, (C) 1999-2017 C. Geuzaine
% For: Octave
% CreationDate: Wed Feb 26 13:15:21 2020
\begin{pgfpicture}
\color[rgb]{1.000000,1.000000,1.000000}
\pgfpathrectanglecorners{\pgfpoint{0pt}{0pt}}{\pgfpoint{576pt}{432pt}}
\pgfusepath{fill}
\begin{pgfscope}
\pgfpathrectangle{\pgfpoint{0pt}{0pt}}{\pgfpoint{576pt}{432pt}}
\pgfusepath{fill}
\pgfpathrectangle{\pgfpoint{0pt}{0pt}}{\pgfpoint{576pt}{432pt}}
\pgfusepath{clip}
\pgfpathmoveto{\pgfpoint{74.880005pt}{399.599976pt}}
\pgflineto{\pgfpoint{521.279968pt}{47.519989pt}}
\pgflineto{\pgfpoint{74.880005pt}{47.519989pt}}
\pgfpathclose
\pgfusepath{fill,stroke}
\pgfpathmoveto{\pgfpoint{74.880005pt}{399.599976pt}}
\pgflineto{\pgfpoint{521.279968pt}{399.599976pt}}
\pgflineto{\pgfpoint{521.279968pt}{47.519989pt}}
\pgfpathclose
\pgfusepath{fill,stroke}
\color[rgb]{0.150000,0.150000,0.150000}
\pgfsetlinewidth{0.500000pt}
\pgfpathmoveto{\pgfpoint{106.765701pt}{51.985016pt}}
\pgflineto{\pgfpoint{106.765701pt}{47.519989pt}}
\pgfusepath{stroke}
\pgfpathmoveto{\pgfpoint{106.765701pt}{395.135010pt}}
\pgflineto{\pgfpoint{106.765701pt}{399.599976pt}}
\pgfusepath{stroke}
\pgfpathmoveto{\pgfpoint{170.537140pt}{51.985016pt}}
\pgflineto{\pgfpoint{170.537140pt}{47.519989pt}}
\pgfusepath{stroke}
\pgfpathmoveto{\pgfpoint{170.537140pt}{395.135010pt}}
\pgflineto{\pgfpoint{170.537140pt}{399.599976pt}}
\pgfusepath{stroke}
\pgfpathmoveto{\pgfpoint{234.308563pt}{51.985016pt}}
\pgflineto{\pgfpoint{234.308563pt}{47.519989pt}}
\pgfusepath{stroke}
\pgfpathmoveto{\pgfpoint{234.308563pt}{395.135010pt}}
\pgflineto{\pgfpoint{234.308563pt}{399.599976pt}}
\pgfusepath{stroke}
\pgfpathmoveto{\pgfpoint{298.079987pt}{51.985016pt}}
\pgflineto{\pgfpoint{298.079987pt}{47.519989pt}}
\pgfusepath{stroke}
\pgfpathmoveto{\pgfpoint{298.079987pt}{395.135010pt}}
\pgflineto{\pgfpoint{298.079987pt}{399.599976pt}}
\pgfusepath{stroke}
\pgfpathmoveto{\pgfpoint{361.851440pt}{51.985016pt}}
\pgflineto{\pgfpoint{361.851440pt}{47.519989pt}}
\pgfusepath{stroke}
\pgfpathmoveto{\pgfpoint{361.851440pt}{395.135010pt}}
\pgflineto{\pgfpoint{361.851440pt}{399.599976pt}}
\pgfusepath{stroke}
\pgfpathmoveto{\pgfpoint{425.622864pt}{51.985016pt}}
\pgflineto{\pgfpoint{425.622864pt}{47.519989pt}}
\pgfusepath{stroke}
\pgfpathmoveto{\pgfpoint{425.622864pt}{395.135010pt}}
\pgflineto{\pgfpoint{425.622864pt}{399.599976pt}}
\pgfusepath{stroke}
\pgfpathmoveto{\pgfpoint{489.394257pt}{51.985016pt}}
\pgflineto{\pgfpoint{489.394257pt}{47.519989pt}}
\pgfusepath{stroke}
\pgfpathmoveto{\pgfpoint{489.394257pt}{395.135010pt}}
\pgflineto{\pgfpoint{489.394257pt}{399.599976pt}}
\pgfusepath{stroke}
{
\pgftransformshift{\pgfpoint{106.765701pt}{40.018295pt}}
\pgfnode{rectangle}{north}{\fontsize{10}{0}\selectfont\textcolor[rgb]{0.15,0.15,0.15}{{2}}}{}{\pgfusepath{discard}}}
{
\pgftransformshift{\pgfpoint{170.537140pt}{40.018295pt}}
\pgfnode{rectangle}{north}{\fontsize{10}{0}\selectfont\textcolor[rgb]{0.15,0.15,0.15}{{4}}}{}{\pgfusepath{discard}}}
{
\pgftransformshift{\pgfpoint{234.308563pt}{40.018295pt}}
\pgfnode{rectangle}{north}{\fontsize{10}{0}\selectfont\textcolor[rgb]{0.15,0.15,0.15}{{6}}}{}{\pgfusepath{discard}}}
{
\pgftransformshift{\pgfpoint{298.079987pt}{40.018295pt}}
\pgfnode{rectangle}{north}{\fontsize{10}{0}\selectfont\textcolor[rgb]{0.15,0.15,0.15}{{8}}}{}{\pgfusepath{discard}}}
{
\pgftransformshift{\pgfpoint{361.851410pt}{40.018295pt}}
\pgfnode{rectangle}{north}{\fontsize{10}{0}\selectfont\textcolor[rgb]{0.15,0.15,0.15}{{10}}}{}{\pgfusepath{discard}}}
{
\pgftransformshift{\pgfpoint{425.622864pt}{40.018295pt}}
\pgfnode{rectangle}{north}{\fontsize{10}{0}\selectfont\textcolor[rgb]{0.15,0.15,0.15}{{12}}}{}{\pgfusepath{discard}}}
{
\pgftransformshift{\pgfpoint{489.394257pt}{40.018295pt}}
\pgfnode{rectangle}{north}{\fontsize{10}{0}\selectfont\textcolor[rgb]{0.15,0.15,0.15}{{14}}}{}{\pgfusepath{discard}}}
\pgfpathmoveto{\pgfpoint{79.347992pt}{119.599602pt}}
\pgflineto{\pgfpoint{74.880005pt}{119.599602pt}}
\pgfusepath{stroke}
\pgfpathmoveto{\pgfpoint{516.812012pt}{119.599602pt}}
\pgflineto{\pgfpoint{521.279968pt}{119.599602pt}}
\pgfusepath{stroke}
\pgfpathmoveto{\pgfpoint{79.347992pt}{192.628571pt}}
\pgflineto{\pgfpoint{74.880005pt}{192.628571pt}}
\pgfusepath{stroke}
\pgfpathmoveto{\pgfpoint{516.812012pt}{192.628571pt}}
\pgflineto{\pgfpoint{521.279968pt}{192.628571pt}}
\pgfusepath{stroke}
\pgfpathmoveto{\pgfpoint{79.347992pt}{265.657562pt}}
\pgflineto{\pgfpoint{74.880005pt}{265.657562pt}}
\pgfusepath{stroke}
\pgfpathmoveto{\pgfpoint{516.812012pt}{265.657562pt}}
\pgflineto{\pgfpoint{521.279968pt}{265.657562pt}}
\pgfusepath{stroke}
\pgfpathmoveto{\pgfpoint{79.347992pt}{338.686523pt}}
\pgflineto{\pgfpoint{74.880005pt}{338.686523pt}}
\pgfusepath{stroke}
\pgfpathmoveto{\pgfpoint{516.812012pt}{338.686523pt}}
\pgflineto{\pgfpoint{521.279968pt}{338.686523pt}}
\pgfusepath{stroke}
{
\pgftransformshift{\pgfpoint{69.875504pt}{119.599602pt}}
\pgfnode{rectangle}{east}{\fontsize{10}{0}\selectfont\textcolor[rgb]{0.15,0.15,0.15}{{1}}}{}{\pgfusepath{discard}}}
{
\pgftransformshift{\pgfpoint{69.875504pt}{192.628571pt}}
\pgfnode{rectangle}{east}{\fontsize{10}{0}\selectfont\textcolor[rgb]{0.15,0.15,0.15}{{2}}}{}{\pgfusepath{discard}}}
{
\pgftransformshift{\pgfpoint{69.875504pt}{265.657562pt}}
\pgfnode{rectangle}{east}{\fontsize{10}{0}\selectfont\textcolor[rgb]{0.15,0.15,0.15}{{3}}}{}{\pgfusepath{discard}}}
{
\pgftransformshift{\pgfpoint{69.875504pt}{338.686523pt}}
\pgfnode{rectangle}{east}{\fontsize{10}{0}\selectfont\textcolor[rgb]{0.15,0.15,0.15}{{4}}}{}{\pgfusepath{discard}}}
\pgfsetrectcap
\pgfsetdash{{16pt}{0pt}}{0pt}
\pgfpathmoveto{\pgfpoint{521.279968pt}{47.519989pt}}
\pgflineto{\pgfpoint{74.880005pt}{47.519989pt}}
\pgfusepath{stroke}
\pgfpathmoveto{\pgfpoint{521.279968pt}{399.599976pt}}
\pgflineto{\pgfpoint{74.880005pt}{399.599976pt}}
\pgfusepath{stroke}
\pgfpathmoveto{\pgfpoint{74.880005pt}{399.599976pt}}
\pgflineto{\pgfpoint{74.880005pt}{47.519989pt}}
\pgfusepath{stroke}
\pgfpathmoveto{\pgfpoint{521.279968pt}{399.599976pt}}
\pgflineto{\pgfpoint{521.279968pt}{47.519989pt}}
\pgfusepath{stroke}
{
\pgftransformshift{\pgfpoint{298.079987pt}{27.018295pt}}
\pgfnode{rectangle}{north}{\fontsize{11}{0}\selectfont\textcolor[rgb]{0.15,0.15,0.15}{{captura}}}{}{\pgfusepath{discard}}}
{
\pgftransformshift{\pgfpoint{58.875488pt}{223.559998pt}}
\pgftransformrotate{90.000000}{\pgfnode{rectangle}{south}{\fontsize{11}{0}\selectfont\textcolor[rgb]{0.15,0.15,0.15}{{ángulo (grados)}}}{}{\pgfusepath{discard}}}}
\color[rgb]{0.000000,0.000000,0.000000}
\pgfsetbuttcap
\pgfsetroundjoin
\pgfsetdash{}{0pt}
\pgfpathmoveto{\pgfpoint{106.765701pt}{104.614059pt}}
\pgflineto{\pgfpoint{74.880005pt}{78.732574pt}}
\pgfusepath{stroke}
\pgfpathmoveto{\pgfpoint{138.651413pt}{124.864990pt}}
\pgflineto{\pgfpoint{106.765701pt}{104.614059pt}}
\pgfusepath{stroke}
\pgfpathmoveto{\pgfpoint{170.537140pt}{347.625275pt}}
\pgflineto{\pgfpoint{138.651413pt}{124.864990pt}}
\pgfusepath{stroke}
\pgfpathmoveto{\pgfpoint{202.422852pt}{47.519989pt}}
\pgflineto{\pgfpoint{170.537140pt}{347.625275pt}}
\pgfusepath{stroke}
\pgfpathmoveto{\pgfpoint{234.308563pt}{248.838989pt}}
\pgflineto{\pgfpoint{202.422852pt}{47.519989pt}}
\pgfusepath{stroke}
\pgfpathmoveto{\pgfpoint{266.194275pt}{156.296661pt}}
\pgflineto{\pgfpoint{234.308563pt}{248.838989pt}}
\pgfusepath{stroke}
\pgfpathmoveto{\pgfpoint{298.079987pt}{52.624725pt}}
\pgflineto{\pgfpoint{266.194275pt}{156.296661pt}}
\pgfusepath{stroke}
\pgfpathmoveto{\pgfpoint{329.965698pt}{67.464203pt}}
\pgflineto{\pgfpoint{298.079987pt}{52.624725pt}}
\pgfusepath{stroke}
\pgfpathmoveto{\pgfpoint{361.851440pt}{204.094131pt}}
\pgflineto{\pgfpoint{329.965698pt}{67.464203pt}}
\pgfusepath{stroke}
\pgfpathmoveto{\pgfpoint{393.737122pt}{260.771912pt}}
\pgflineto{\pgfpoint{361.851440pt}{204.094131pt}}
\pgfusepath{stroke}
\pgfpathmoveto{\pgfpoint{425.622864pt}{399.599976pt}}
\pgflineto{\pgfpoint{393.737122pt}{260.771912pt}}
\pgfusepath{stroke}
\pgfpathmoveto{\pgfpoint{457.508575pt}{187.253632pt}}
\pgflineto{\pgfpoint{425.622864pt}{399.599976pt}}
\pgfusepath{stroke}
\pgfpathmoveto{\pgfpoint{489.394257pt}{313.433136pt}}
\pgflineto{\pgfpoint{457.508575pt}{187.253632pt}}
\pgfusepath{stroke}
\pgfpathmoveto{\pgfpoint{521.279968pt}{398.219757pt}}
\pgflineto{\pgfpoint{489.394257pt}{313.433136pt}}
\pgfusepath{stroke}
\pgfsetrectcap
\pgfsetmiterjoin
\pgfpathmoveto{\pgfpoint{77.880005pt}{78.732574pt}}
\pgflineto{\pgfpoint{71.879990pt}{78.732574pt}}
\pgfusepath{stroke}
\pgfpathmoveto{\pgfpoint{74.880005pt}{75.732574pt}}
\pgflineto{\pgfpoint{74.880005pt}{81.732574pt}}
\pgfusepath{stroke}
\pgfpathmoveto{\pgfpoint{77.001312pt}{76.611252pt}}
\pgflineto{\pgfpoint{72.758667pt}{80.853897pt}}
\pgfusepath{stroke}
\pgfpathmoveto{\pgfpoint{77.001312pt}{80.853897pt}}
\pgflineto{\pgfpoint{72.758667pt}{76.611252pt}}
\pgfusepath{stroke}
\pgfpathmoveto{\pgfpoint{109.765732pt}{104.614029pt}}
\pgflineto{\pgfpoint{103.765717pt}{104.614029pt}}
\pgfusepath{stroke}
\pgfpathmoveto{\pgfpoint{106.765717pt}{101.614029pt}}
\pgflineto{\pgfpoint{106.765717pt}{107.614029pt}}
\pgfusepath{stroke}
\pgfpathmoveto{\pgfpoint{108.887039pt}{102.492706pt}}
\pgflineto{\pgfpoint{104.644394pt}{106.735352pt}}
\pgfusepath{stroke}
\pgfpathmoveto{\pgfpoint{108.887039pt}{106.735352pt}}
\pgflineto{\pgfpoint{104.644394pt}{102.492706pt}}
\pgfusepath{stroke}
\pgfpathmoveto{\pgfpoint{141.651428pt}{124.864983pt}}
\pgflineto{\pgfpoint{135.651428pt}{124.864983pt}}
\pgfusepath{stroke}
\pgfpathmoveto{\pgfpoint{138.651428pt}{121.864983pt}}
\pgflineto{\pgfpoint{138.651428pt}{127.864983pt}}
\pgfusepath{stroke}
\pgfpathmoveto{\pgfpoint{140.772751pt}{122.743660pt}}
\pgflineto{\pgfpoint{136.530106pt}{126.986305pt}}
\pgfusepath{stroke}
\pgfpathmoveto{\pgfpoint{140.772751pt}{126.986305pt}}
\pgflineto{\pgfpoint{136.530106pt}{122.743660pt}}
\pgfusepath{stroke}
\pgfpathmoveto{\pgfpoint{173.537140pt}{347.625305pt}}
\pgflineto{\pgfpoint{167.537155pt}{347.625305pt}}
\pgfusepath{stroke}
\pgfpathmoveto{\pgfpoint{170.537140pt}{344.625275pt}}
\pgflineto{\pgfpoint{170.537140pt}{350.625305pt}}
\pgfusepath{stroke}
\pgfpathmoveto{\pgfpoint{172.658478pt}{345.503967pt}}
\pgflineto{\pgfpoint{168.415833pt}{349.746613pt}}
\pgfusepath{stroke}
\pgfpathmoveto{\pgfpoint{172.658478pt}{349.746613pt}}
\pgflineto{\pgfpoint{168.415833pt}{345.503967pt}}
\pgfusepath{stroke}
\pgfpathmoveto{\pgfpoint{205.422852pt}{47.519974pt}}
\pgflineto{\pgfpoint{199.422852pt}{47.519974pt}}
\pgfusepath{stroke}
\pgfpathmoveto{\pgfpoint{202.422852pt}{44.519974pt}}
\pgflineto{\pgfpoint{202.422852pt}{50.519989pt}}
\pgfusepath{stroke}
\pgfpathmoveto{\pgfpoint{204.544174pt}{45.398651pt}}
\pgflineto{\pgfpoint{200.301529pt}{49.641296pt}}
\pgfusepath{stroke}
\pgfpathmoveto{\pgfpoint{204.544174pt}{49.641296pt}}
\pgflineto{\pgfpoint{200.301529pt}{45.398651pt}}
\pgfusepath{stroke}
\pgfpathmoveto{\pgfpoint{237.308578pt}{248.838959pt}}
\pgflineto{\pgfpoint{231.308578pt}{248.838959pt}}
\pgfusepath{stroke}
\pgfpathmoveto{\pgfpoint{234.308578pt}{245.838959pt}}
\pgflineto{\pgfpoint{234.308578pt}{251.838974pt}}
\pgfusepath{stroke}
\pgfpathmoveto{\pgfpoint{236.429901pt}{246.717651pt}}
\pgflineto{\pgfpoint{232.187256pt}{250.960281pt}}
\pgfusepath{stroke}
\pgfpathmoveto{\pgfpoint{236.429901pt}{250.960281pt}}
\pgflineto{\pgfpoint{232.187256pt}{246.717651pt}}
\pgfusepath{stroke}
\pgfpathmoveto{\pgfpoint{269.194275pt}{156.296661pt}}
\pgflineto{\pgfpoint{263.194275pt}{156.296661pt}}
\pgfusepath{stroke}
\pgfpathmoveto{\pgfpoint{266.194275pt}{153.296661pt}}
\pgflineto{\pgfpoint{266.194275pt}{159.296661pt}}
\pgfusepath{stroke}
\pgfpathmoveto{\pgfpoint{268.315613pt}{154.175323pt}}
\pgflineto{\pgfpoint{264.072968pt}{158.417969pt}}
\pgfusepath{stroke}
\pgfpathmoveto{\pgfpoint{268.315613pt}{158.417969pt}}
\pgflineto{\pgfpoint{264.072968pt}{154.175323pt}}
\pgfusepath{stroke}
\pgfpathmoveto{\pgfpoint{301.079987pt}{52.624741pt}}
\pgflineto{\pgfpoint{295.079987pt}{52.624741pt}}
\pgfusepath{stroke}
\pgfpathmoveto{\pgfpoint{298.079987pt}{49.624741pt}}
\pgflineto{\pgfpoint{298.079987pt}{55.624741pt}}
\pgfusepath{stroke}
\pgfpathmoveto{\pgfpoint{300.201324pt}{50.503418pt}}
\pgflineto{\pgfpoint{295.958679pt}{54.746063pt}}
\pgfusepath{stroke}
\pgfpathmoveto{\pgfpoint{300.201324pt}{54.746063pt}}
\pgflineto{\pgfpoint{295.958679pt}{50.503418pt}}
\pgfusepath{stroke}
\pgfpathmoveto{\pgfpoint{332.965759pt}{67.464203pt}}
\pgflineto{\pgfpoint{326.965759pt}{67.464203pt}}
\pgfusepath{stroke}
\pgfpathmoveto{\pgfpoint{329.965759pt}{64.464203pt}}
\pgflineto{\pgfpoint{329.965759pt}{70.464203pt}}
\pgfusepath{stroke}
\pgfpathmoveto{\pgfpoint{332.087067pt}{65.342880pt}}
\pgflineto{\pgfpoint{327.844421pt}{69.585526pt}}
\pgfusepath{stroke}
\pgfpathmoveto{\pgfpoint{332.087067pt}{69.585526pt}}
\pgflineto{\pgfpoint{327.844421pt}{65.342880pt}}
\pgfusepath{stroke}
\pgfpathmoveto{\pgfpoint{364.851440pt}{204.094131pt}}
\pgflineto{\pgfpoint{358.851440pt}{204.094131pt}}
\pgfusepath{stroke}
\pgfpathmoveto{\pgfpoint{361.851440pt}{201.094131pt}}
\pgflineto{\pgfpoint{361.851440pt}{207.094131pt}}
\pgfusepath{stroke}
\pgfpathmoveto{\pgfpoint{363.972778pt}{201.972809pt}}
\pgflineto{\pgfpoint{359.730103pt}{206.215454pt}}
\pgfusepath{stroke}
\pgfpathmoveto{\pgfpoint{363.972778pt}{206.215454pt}}
\pgflineto{\pgfpoint{359.730103pt}{201.972809pt}}
\pgfusepath{stroke}
\pgfpathmoveto{\pgfpoint{396.737152pt}{260.771912pt}}
\pgflineto{\pgfpoint{390.737152pt}{260.771912pt}}
\pgfusepath{stroke}
\pgfpathmoveto{\pgfpoint{393.737152pt}{257.771912pt}}
\pgflineto{\pgfpoint{393.737152pt}{263.771912pt}}
\pgfusepath{stroke}
\pgfpathmoveto{\pgfpoint{395.858490pt}{258.650574pt}}
\pgflineto{\pgfpoint{391.615845pt}{262.893219pt}}
\pgfusepath{stroke}
\pgfpathmoveto{\pgfpoint{395.858490pt}{262.893219pt}}
\pgflineto{\pgfpoint{391.615845pt}{258.650574pt}}
\pgfusepath{stroke}
\pgfpathmoveto{\pgfpoint{428.622864pt}{399.600006pt}}
\pgflineto{\pgfpoint{422.622864pt}{399.600006pt}}
\pgfusepath{stroke}
\pgfpathmoveto{\pgfpoint{425.622864pt}{396.600006pt}}
\pgflineto{\pgfpoint{425.622864pt}{402.600006pt}}
\pgfusepath{stroke}
\pgfpathmoveto{\pgfpoint{427.744202pt}{397.478699pt}}
\pgflineto{\pgfpoint{423.501556pt}{401.721313pt}}
\pgfusepath{stroke}
\pgfpathmoveto{\pgfpoint{427.744202pt}{401.721313pt}}
\pgflineto{\pgfpoint{423.501556pt}{397.478699pt}}
\pgfusepath{stroke}
\pgfpathmoveto{\pgfpoint{460.508575pt}{187.253632pt}}
\pgflineto{\pgfpoint{454.508545pt}{187.253632pt}}
\pgfusepath{stroke}
\pgfpathmoveto{\pgfpoint{457.508575pt}{184.253632pt}}
\pgflineto{\pgfpoint{457.508575pt}{190.253632pt}}
\pgfusepath{stroke}
\pgfpathmoveto{\pgfpoint{459.629883pt}{185.132324pt}}
\pgflineto{\pgfpoint{455.387268pt}{189.374954pt}}
\pgfusepath{stroke}
\pgfpathmoveto{\pgfpoint{459.629883pt}{189.374954pt}}
\pgflineto{\pgfpoint{455.387268pt}{185.132324pt}}
\pgfusepath{stroke}
\pgfpathmoveto{\pgfpoint{492.394287pt}{313.433105pt}}
\pgflineto{\pgfpoint{486.394287pt}{313.433105pt}}
\pgfusepath{stroke}
\pgfpathmoveto{\pgfpoint{489.394287pt}{310.433105pt}}
\pgflineto{\pgfpoint{489.394287pt}{316.433105pt}}
\pgfusepath{stroke}
\pgfpathmoveto{\pgfpoint{491.515625pt}{311.311798pt}}
\pgflineto{\pgfpoint{487.272980pt}{315.554443pt}}
\pgfusepath{stroke}
\pgfpathmoveto{\pgfpoint{491.515625pt}{315.554443pt}}
\pgflineto{\pgfpoint{487.272980pt}{311.311798pt}}
\pgfusepath{stroke}
\pgfpathmoveto{\pgfpoint{524.280029pt}{398.219757pt}}
\pgflineto{\pgfpoint{518.280029pt}{398.219757pt}}
\pgfusepath{stroke}
\pgfpathmoveto{\pgfpoint{521.280029pt}{395.219757pt}}
\pgflineto{\pgfpoint{521.280029pt}{401.219757pt}}
\pgfusepath{stroke}
\pgfpathmoveto{\pgfpoint{523.401367pt}{396.098450pt}}
\pgflineto{\pgfpoint{519.158691pt}{400.341064pt}}
\pgfusepath{stroke}
\pgfpathmoveto{\pgfpoint{523.401367pt}{400.341064pt}}
\pgflineto{\pgfpoint{519.158691pt}{396.098450pt}}
\pgfusepath{stroke}
\end{pgfscope}
\end{pgfpicture}
}
			\caption{\label{fig:clust_happy}Diferencias absolutas entre la rotación estimada y el \emph{ground truth} para el objeto \texttt{happy}.}
		\end{figure}

		\begin{figure}
			\centering
			\begin{subfigure}{.8\linewidth}
				\Imagen{img/bun_sac_270_315}
				\caption{\label{fig:align_sac}Ejemplo de fallo del algoritmo de \emph{sample consensus} producto de malas correspondencias.}
			\end{subfigure}
			\begin{subfigure}{.8\linewidth}
				\Imagen{img/bun_clust_270_315}
				\caption{\label{fig:clust_bun_good}Alineación exitosa mediante el uso de marcos de referencia ISS.}
			\end{subfigure}
			\caption{Alineación entre las capturas \texttt{bun270} (verde) y \texttt{bun315} (rojo).}
		\end{figure}


	\subsection{Refinamiento}
	Para ajustar las alineaciones iniciales se procedió a realizar una segunda alineación utilizando
	el algoritmo de ICP provisto por la biblioteca PCL.
	Se consideraron únicamente las áreas solapadas, restringiendo el espacio de
	búsqueda de las correspondencias, minimizando la distancia entre los puntos
	de la nube a transformar hacia los planos definidos por las normales en la nube objetivo.

	%@@@
	\clearpage
	Luego, para reducir el error propagado por cada alineación, se propuso una
	corrección de bucle
	ajustando la última captura para que correspondiese con la primera,
	y agregando esta transformación a las otras alineaciones ponderándola de forma
	proporcional a su posición en el bucle (algoritmo~\ref{alg:correccion_bucle}).

	\begin{algorithm}
		\begin{algorithmic}[1]
			\Function{Corrección de bucle}{nubes, N}
				\State peso $\gets \frac{1}{N-1}$
				\State error $\gets$ Alineación(desde=nubes[1], hacia=nubes[N])
				\State $[q|t]$ $\gets$ inversa(error)
				\ForAll{$K \in 1:N$}
					\State rotación $\gets$ slerp(K * peso, $q$, Identidad)
					\State translación $\gets$ K * peso * $t$
					\State nubes[K] $\gets$ transformar([rotación|translación], nubes[K])
				\EndFor
			\EndFunction
		\end{algorithmic}
		\caption{\label{alg:correccion_bucle}Corrección de la propagación del error de alineación.}
	\end{algorithm}
