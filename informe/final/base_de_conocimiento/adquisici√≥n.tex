\section{Adquisición}
%Un proceso crucial en la reconstrucción tridimensional es la adquisición de datos.
Los distintos métodos de adquisición se diferencian de acuerdo al fenómeno físico de interacción
con la superficie del objeto de interés. De esta manera, se pueden clasificar como:
	\begin{itemize}
		\item Métodos táctiles o de contacto, donde sensores en las articulaciones de un brazo robótico determinan las coordenadas relativas de la superficie. Estos son de los más robustos, introduciendo poco ruido, pero también de los más lentos y suelen tener problemas con superficies cóncavas.
		\item Métodos de no-contacto, donde se utiliza luz (métodos ópticos), sonido u ondas electromagnéticas.
	\end{itemize}

En particular, los métodos ópticos son los más populares con una rápida velocidad de adquisición\cite{Várady97reverseengineering}. %\cite{reverse engineering}.
Dentro de los métodos ópticos podemos distinguir:
\begin{itemize}
	\item Métodos pasivos o de visión estereoscópica,
		que utilizan dos o más cámaras y buscan correspondencias
		entre los puntos de la escena capturados en cada fotografía.
	\item Métodos activos o de luz estructurada,
		que proyectan patrones de luces conocidos sobre la escena de modo de analizar sus deformaciones.
\end{itemize}
Identificar los patrones lumínicos es un problema más simple que el de
encontrar las correspondencias entre imágenes, por lo que los métodos de luz
estructuradas resultan más rápidos y robustos\cite{Pancho}.

%FIXME
%volarlo todo y decir que los métodos ópticos me dan una nube de puntos 2.5D,
%orientada en el sistema de referencia de la cámara
A partir de los métodos ópticos se obtendrá como resultado una imagen de profundidad, donde el valor de
cada píxel representa la distancia entre el punto observado y la cámara, lo cual se conoce como imagen 2.5D.
Conociendo los parámetros intrínsecos $K$ de la cámara,
puede obtenerse para cada píxel $\{X, Y, d\}$ sus coordenadas $\{x, y, z\}$ en el espacio:
\[
	\left(\begin{matrix}
		x \\ y \\ z \\
	\end{matrix}\right) = 
	K^{-1}
	\left(\begin{matrix}
		X \\ Y \\ d \\
	\end{matrix}\right)
\]
Estas son coordenadas locales, definidas en un sistema de referencia centrado en la cámara.


Se tendrá entonces una \emph{nube de puntos}, es decir, una colección de puntos
sin información de conectividad entre ellos.
%Se observa, además, que no existen dos puntos con las mismas coordenadas $\{x, y\}$, es decir, la nube de puntos es 2.5D.
Esta nube de puntos será una representación parcial de la superficie del objeto,
ya que se corresponderá únicamente con la porción observable desde la cámara.


Si bien este proyecto no abarca la etapa de adquisición,
no es posible ignorar el método de adquisición,
ya que este influye en las características de la nube de puntos resultante.
Por esta razón, se supondrá que las capturas provienen de métodos ópticos de luz estructurada.


\subsection{Ruido de adquisición}
Es imposible obtener la posición exacta de los puntos muestreados sobre la superficie del objeto.
Se tendrán distorsiones debidas a particularidades de la escena, del objeto, de los
dispositivos utilizados y del método empleado.
En los escáneres de luz estructurada dos fuentes de error son particularmente relevantes:
\begin{itemize}
	\item \emph{Ángulo de incidencia excesivo:}
		El rayo de luz proyectado impacta en una porción de la superficie del objeto
		que es casi paralela a su dirección.
		Entonces, el sensor capta
		una versión estirada y con menor intensidad del patrón, %(figura~\ref{fig:error_distorsion}),
		lo que agrega incertidumbre en la posición de los puntos.
	\item \emph{Reflejo parcial:}
		Se produce cuando una porción del patrón no incide completamente en el objeto (figura~\ref{fig:error_adquisicion}).
		Debido a que el método de triangulación supone que todo el ancho de la línea impactó en el objeto,
		se estima una posición incorrecta del punto muestreado.
		Esto resulta en bordes distorsionados y en posiciones más alejadas que las reales.\cite{Turk:1994:ZPM:192161.192241}
\end{itemize}

%\begin{figure}
%	\Imagen{foo}
%	\caption{\label{fig:error_distorsion}El patrón a identificar se encuentra estirado debido a que los haces de luz inciden con un ángulo excesivo sobre la superficie.}
%\end{figure}
\begin{figure}
		\Imagen{diagram/error_adquisición_borde}
		\caption[Error debido a un reflejo parcial]{\label{fig:error_adquisicion}Error en la posición estimada debido a un reflejo parcial del patrón de luz.}
\end{figure}

\endinput
