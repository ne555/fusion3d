\section{Registración}
Al finalizar la etapa de adquisición se obtuvo una nube de puntos
que representaba la porción observada del objeto.
Para lograr una reconstrucción total del objeto es necesario combinar múltiples capturas
variando la posición relativa cámara-objeto.
Contar con un dispositivo que nos permita definir con precisión
tanto la posición como orientación de la cámara es altamente costoso,
por esta razón, en la etapa de registración se calcularán las transformaciones que ubiquen cada
captura en un sistema de referencia global, de forma que las zonas comunes encajen perfectamente.


La operación de registración se realizará siempre entre dos capturas,
buscando las transformaciones de translación y rotación que ubiquen
la captura de partida en el marco de referencia de la captura objetivo.
De esta manera, el problema general cuenta con seis grados de libertad.
Primero se discutirá el caso donde la transformación de alineación sea pequeña,
luego se permitirán incrementos mayores,
y finalmente se extenderá el proceso para abarcar $n$ capturas.

%esto va en $N$ capturas
%Para extender este proceso a $n$ capturas, puede alinearse $C_0$ con $C_1$,
%$C_1$ con $C_2$, $C_2$ con $C_3$, etc.,
%sin embargo, el error de registración se propagará en cada paso.


\subsection{Perturbaciones pequeñas: algoritmo iterativo del punto más cercano (ICP)}
Antes de describir este algoritmo, se planteará una versión simplificada del problema
que nos permitirá establecer ciertas definiciones.

Supongamos que a una nube $A$ le aplicamos una transformación de translación y rotación arbitraria,
y luego aplicamos una pequeña perturbación a las posiciones de los puntos transformados,
obteniendo así la nube $B$.
Se observa ambas nubes poseen la misma cardinalidad y podemos establecer las relaciones de origen a destino
$a_j \to b_j$ para cada punto $a_j \in A$.
Se puede definir el error de alineación como
\[
	\text{Error} = \frac{1}{N} \sum_{j=1}^n || b_j - T \left(a_j\right) ||
\]
y se buscará la transformación $T$ que minimice este error.
Existe una solución cerrada para este problema, obteniéndose la rotación al
calcular los eigenvectores de una matriz simétrica de $4\times4$,
cuyos detalles pueden consultarse en \cite{Horn87closed-formsolution}.

Sin embargo, estas suposiciones son demasiado fuertes.
Al trabajar con dos capturas realizadas desde distintas vistas, el solapamiento no será total,
ya que habrá puntos que serán observables en tan sólo uno de las vistas,
por lo que primero es necesario determinar cuáles son los puntos comunes.
Además, se desconoce para los puntos de la nube $A$ cuál es su posición en $B$.
El algoritmo iterativo del punto más cercano (ICP) supone que las nubes se encuentran
lo suficientemente cerca como para establecer estas correspondencias mediante las coordenadas de los puntos.
El algoritmo se describe de la siguiente manera:
\begin{enumerate}
	\item Obtener las correspondencias:
		Para cada punto $a_j \in A$ buscar el punto más cercano en la nube $B$.
		Si la distancia entre esos puntos supera un umbral $d_{\text{max}}$,
		se considera que $a_j$ no pertenece a la zona común y no es considerado en esta iteración.
	\item Calcular la transformación que alinee los pares de puntos de la zona común.
	\item Aplicar la transformación.
	\item Repetir el proceso hasta que el error de alineación esté por debajo de un umbral $\tau$.
\end{enumerate}%\cite{conf/rss/SegalHT09}
Si bien el algoritmo convergirá a un mínimo local, puede que este no sea el mínimo global buscado.
Debe cumplirse la suposición de cercanía para obtener una correcta registración.\cite{regBesl92}

%generalized conf/rss/SegalHT09
En \cite{chen-medoni} se presenta una variante del algoritmo de ICP llamada ICP punto-a-plano,
que resulta más robusta y precisa al trabajar con nubes de puntos en 2.5D. \TODO{definir 2.5D, que es lo que tenemos}
Una vez establecida una correspondencia $a_j \to b_j$, si bien no se trata del mismo punto,
se realiza la suposición de que pertenecen al mismo plano.
Por esta razón, podemos decir que conocemos con mucha confianza la posición del punto
a lo largo de su normal, pero no su ubicación en el plano.
Para reflejar esto, se cambia la métrica del error, proyectando el punto transformado sobre la normal
del punto destino (figura~\ref{fig:point_to_plane}).
\[
	\text{Error} = \frac{1}{N} \sum_{j=1}^n || n_j \cdot \left( b_j - T \left(a_j\right) \right) ||
\]

\begin{figure}
	\centering
	\input{diagram/icp_point_to_plane.pdf_tex}
	\caption{\label{fig:point_to_plane}Medidas de error punto-a-plano entre dos superficies
	(Imagen cortesía de \cite{icp_point_to_plane}).}
\end{figure}

\endinput
