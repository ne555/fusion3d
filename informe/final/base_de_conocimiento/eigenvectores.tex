Se hará una relación entre el cálculo de los eigenvalores y eigenvectores de la matriz de covarianza en la vecindad
de un punto y la estimación de la normal en ese punto.

Definición de eigenvalor, eigenvector

\[ M_{jk} v_k = \lambda v_j \]

$\lambda$ es un eigenvalor
$v_j$ es su eigenvector correspondiente \Nota{¿era así?}

¿Cómo se calcula?

\[ M_{jk} - \lambda \delta_{jk} = 0 \]


Estimación de la normal de un punto
Matriz de covarianza
se resuelven los eigenvalores, el menor da el eigenvector normal
No se puede resolver el signo de los eigenvectores, por lo que se tiene ambigüedad.

Análisis de componentes principales (PCA)
En este método se busca proyectar los datos a un nuevo sistemas de coordenadas
de forma que la mayor varianza de los datos se proyecte sobre el primer eje,
la segunda mayor varianza sobre el segundo eje, y así sucesivamente.
De esta forma, se puede reducir el orden al 


%https://blog.clairvoyantsoft.com/eigen-decomposition-and-pca-c50f4ca15501
So, PCA is a method that:
	* Measures how each variable is associated with one another using a Covariance matrix
	* Understands the directions of the spread of our data using Eigenvectors
	* Brings out the relative importance of these directions using Eigenvalues

menor eigenvector corresponde a la normal


For the covariance or correlation matrix, the eigenvectors correspond to
principal components and the eigenvalues to the variance explained by the
principal components.

%semantic 3d
%4.3 Surface Normals and Curvature Estimates
The problem of determining the normal to a point on
the surface is approximated by the problem of estimating the normal of a plane tangent to the
surface, which in turn becomes a least-square plane fitting estimation problem in P
k [Sha98].

In general, because there is no mathematical way to solve for the sign of⃗n, the orientation of
the normal computed via Principal Component Analysis (PCA) as shown above is ambiguous,
and not consistently oriented over a point cloud dataset P.


Though many different normal estimation methods exist (see [KAWB09] for a recent com-
parison for normal estimation in 3D range data), the simplest method is based on the first order
3D plane fitting as proposed by [BC94].
%Jens Berkmann and Terry Caelli.
%Computation of Surface Geometry and Segmentation Using Covariance Techniques.
%IEEE Transactions on Pattern Analysis and Machine Intelligence (PAMI), 16(11):1114–1116, November 1994.
