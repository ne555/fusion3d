La reconstrucción tridimensional es un proceso por el cual se combinan diversas mediciones de un objeto
para obtener un modelo tridimensional que lo reproduzca fielmente.
Este proceso puede dividirse en las siguientes etapas:
\begin{enumerate}
	\item Adquisición: en esta etapa se realizarán las mediciones del objeto,
		obteniéndose las posiciones $\{x, y, z\}$ de puntos muestreados sobre su superficie.
	\item Registración: en caso de que el método de adquisición no capture la totalidad del objeto,
		esta etapa establecerá las relaciones entre las medidas parciales
		para ajustarlas a un mismo marco de referencia.
	\item Determinación de la superficie: en esta etapa se estimará la superficie del objeto
		a partir de los puntos muestreados.
	\item Rellenado de huecos: es posible que existan zonas donde
		no se hayan obtenido muestras y, por lo tanto, la superficie presente huecos.
		Dependiendo de la aplicación, puede ser necesario rellenar estos huecos
		estimando la superficie mediante los puntos lindantes.
\end{enumerate}

A continuación se describirán con más detalle estas etapas y los métodos utilizados en las mismas.

\input{base_de_conocimiento/adquisición}
\input{base_de_conocimiento/registración}
\input{base_de_conocimiento/fusión}
