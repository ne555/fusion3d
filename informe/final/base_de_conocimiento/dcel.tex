%computational geometry deberg
%2.2
\section{The doubly-connected edge list DCEL}
representación de grafos/mallas mediante medias aristas

Se busca buscar una representación apropiada para una subdivisión.
Lo que importa es la clase de operaciones que pretendemos realizar.
un simple grafo donde los nodos sean los vértices o las aristas \Nota{¿se puede?}
no nos permite de forma simple obtener bordes de las regiones.
Nos interesa incorporar información topológica, qué segmentos bordean qué región.

\Nota{La definición de subdivisión, cara, vértice y arista que da es medio complicada}

Operaciones:
-¿qué cara contiene un punto dado? Es demasiado para esta representación
Buscamos cosas más locales:
- recorrer el contorno de una cara
- ir de una cara a otra vecina
- recorrer las aristas que inciden en un vértice

Almacenamos cada cara, arista y vértice de la subdivisión.
Además de la información topológica (\Nota{¿conectividades?})
y geométrica podemos almacenar otras propiedades del dominio.

Para recorrer una cara, cada arista almacena un puntero a la siguiente arista (y es conveniente mantener el anterior también)
Una arista en general limita dos caras, mantenemos punteros a esas.
Cambio: en lugar de una arista, tenemos dos semiaristas (pares gemelos),
cada semiarista limita solamente una cara.
se orientan para recorrer en sentido antihorario (la cara cae a la izquierda)
Como están orientadas tenemos vértices de inicio y fin.

una cara necesita solamente un puntero a una media-arista

Agujeros:
\Nota{revisar bien}

class edge{
	edge *next, *previous;
};

Tres contenedores
	vértices: coordenadas, puntero a 1 media arista que sale de aquí
	caras: puntero a 1 media-arista que limita su borde
	medias-aristas:
		vértice de origen, (no hace falta el destino, lo saco del gemelo/complementario)
		puntero a su media-arista gemela,
		cara que delimita (queda a su izquierda)
		punteros a media-arista siguiente y anterior en el recorrido de la cara

conclusión:
	se almacena información de tamaño constante para vértices y medias-aristas
	una cara requiere más si se aceptan agujeros y puntos aislados dentro
	O(n) de almacenamiento

operaciones
	recorrer el borde de una cara
	obtener todas las aristas que inciden en un vértice
	obtener los vecinos de un vértice
	encontrar aristas de borde \Nota{borde de la malla}
	encontrar vértices de borde

En muchas aplicaciones, las caras no tienen un interés especial
implementación, el grafo debe ser conectado (y triangulado)
