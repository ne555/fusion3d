\section{Definiciones}
\begin{description}
	\item [Imagen]
		Una imagen representa los valores que adquiere una función $f$
		en puntos discretos (píxeles) ordenados en una grilla rectangular.
	\item [Imagen de profundidad]
		En una imagen de profundidad, los valores de cada píxel se corresponden
		con la distancia que existe entre la cámara y los objetos de la escena.
	\item [Imagen 2.5D]
		\TODO{Algo respecto a poder proyectar en $z=0$}
	\item [Nube de puntos]
		Una nube de puntos es una colección de puntos cuyas coordenadas espaciales
		están definidas respecto a un sistema de referencia fijo.
	\item [Malla]
		Una malla representa una superficie al establecer conectividades entre
		los puntos de una nube y definir de esta forma caras y aristas.
	\item [Captura]
		Una captura es la adquisición de las coordenadas $\{x, y, z\}$ de los puntos de una nube
		por medio de sensores y técnicas de procesamiento adecuado.
	\item [Vista]
		Una vista es una captura realizada a partir de cierta posición sensor-escena específica.
	\item [Vecindad]
		Para un punto $p$, su vecindad son aquellos puntos que se encuentran
		dentro de una esfera de radio $r$ centrada e $p$. Es decir, aquellos
		puntos $q$ en los que se satisface que
		\[ |q - p| \leq r \]
\end{description}

\section{Adquisición de la nube de puntos}
Al definir el origen del sistema de referencia centrado en el dispositivo de captura,
las coordenadas de cada punto de la nube representan la distancia entre el sensor y la superficie
de la cual fue muestreado el punto.
Para lograr medir esta distancia, se puede hacer uso de sistemas de luz estructurada,
donde un patrón conocido es proyectado en la escena y se analizan sus deformaciones.

Debido a las características geométricas de los objetos que componen la escena,
no se podrá obtener una representación de todas las superficies a partir de tan sólo
una vista. Es necesario entonces combinar varias vistas para lograr capturar
esas superficies que permanecían ocultas. Sin embargo, debido a que
cada captura se encuentra en un sistema de referencia distinto,
esta combinación no es trivial, debiendo determinarse las transformaciones que lleven
cada vista a un sistema de referencia global.

El proceso de obtener estas transformaciones se denomina registración.
