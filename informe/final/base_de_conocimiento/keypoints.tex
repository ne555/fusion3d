\section{\emph{Keypoints}}
Los \emph{keypoints}, también llamados puntos de interés o puntos salientes, son puntos de una nube que son
estables,
descriptivos
e identificables.
La idea es que sean reproducible sen distintas vistas, entonces se intentarán establecer las correspondencias entre los \emph{keypoints} utilizando los descriptores locales.

Esto se realiza por dos razones:
- reducir la cantidad de puntos a trabajar
- evitar correspondencias erróneas en zonas homogéneas, es decir, donde los
valores de los descriptores son similares.


ralos
repetibles
distintivos


%r2rnn.pdf
%Saliency
%Saliency is a measure of local significance in a surface:
%salient points/regions are those whose properties
%are unlike most of their neighbours. They are used
%for key point/region detection, often in conjunction
%with features (Section 4.1.2). They reduce the size
%of correspondence space, potential mismatches and
%obtain more reliable coarse correspondences. These
%saliency measures include: differential properties [56],
%rareness [48], scale space analysis on geometry [17]
%and features [48], size and curvature based on visual
%saliency [53], multi-scale slippage [57] and maximally
%stable extremal regions [58].
