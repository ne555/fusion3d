\section{Determinación de la superficie}
Una vez alineadas todas las capturas en un mismo sistema de referencia,
debemos combinar estas nubes de puntos para obtener un único modelo
que describa la geometría del objeto escaneado.
Cada captura presentará nueva información en las zonas
que no pudieron ser observadas desde otras posiciones,
y confirmará o refutará la información ya presente en las zonas comunes a otras capturas.

A continuación se presentarán métodos para lograr la integración de las distintas capturas.

%zippered
\subsection{\emph{Mesh zippering}}
Este método combina dos mallas poligonales en una única superficie,
obteniéndose el objeto completo al agregar las capturas una a una sobre la superficie final.
El proceso consta de los siguientes pasos:
\begin{enumerate}
	\item Triangulación de la captura.
		Debido a que este método trabaja con mallas, cada nube de puntos debe ser primero triangulada.
	\item Eliminación de superficies redundantes.
		Se eliminan los triángulos de ambas mallas que se encuentren completamente dentro del área solapada.
		De esta manera, los únicos triángulos que presentarán un solapamiento
		serán aquellos pertenecientes a los bordes de las mallas.
	\item \emph{Clipping}.
		Los triángulos de la malla A se recortan por el borde de la malla B.
		Es decir, se eliminan las porciones de los triángulos que caen dentro de la otra malla.
\end{enumerate}

De esta forma se obtiene una malla triangular que representa burdamente la topología del objeto.
Para ajustar los detalles de la superficie, se realiza un proceso de refinamiento,
donde cada vértice de la malla final es perturbado por un promedio ponderado
de los vértices de los triángulos que caen cerca de él en las capturas originales\cite{Turk:1994:ZPM:192161.192241}.


%volumetric (ocupar el del kinect)
\subsection{Fusión volumétrica (\emph{volumetric merge})}
En este método, cada captura se representa de forma volumétrica,
definiendo un volumen 3D que mapea las coordenadas $\{x, y, z\}$ de los puntos a posiciones $\{w, h, d\}$
de una grilla de vóxeles.
En cada uno de estos vóxeles se codifica una función de distancia a la superficie,
siendo positiva si el punto se encuentra adelante de la superficie, y negativa si se encuentra después.
De esta manera, la superficie queda definida de forma implícita en el cruce por cero,
donde los valores cambian de signo, y debe ser extraída
mediante un algoritmo de \emph{marching cubes} o similar.

En cada nueva registración se actualizan las distancias de los vóxeles mediante raycasting,
desde la posición estimada de la cámara hacia la superficie,
realizando un promedio ponderado con los valores anteriormente calculados.\cite{Curless:1996:VMB:237170.237269} %a_volumetric_method_for_building_complex_models_from_range_images.txt
Debido a que sólo nos interesan las zonas donde la distancia es cercana a 0
es posible utilizar una representación rala para los vóxeles.\cite{Steinbrucker:2013:LMS:2586117.2586926} % large-scale_multi-resolution_surface_reconstruction_from_rgb-d_sequences


%surfel
\subsection{Representación de surfel}
En este método, la superficie se representa como un conjunto de discos orientados o elementos de superficie (\emph{surfels}).
Cada uno de estos \emph{surfels} posee una posición $p_j$,
un vector normal $n_j$, un radio $r$ y una medida de confianza de visibilidad $v$,
que nos indica cuántas capturas confirmaron las características del \emph{surfel}.
Esta representación facilita la actualización, agregado o eliminación de vértices y además nos permite detectar ciertos puntos atípicos\cite{5457479}.

El modelo se actualiza con cada nueva captura, comparándose la nueva información
con la ya existente en el modelo y realizando las operaciones correspondientes:
\begin{itemize}
	\item Se crean nuevos \emph{surfels} para las porciones que el modelo actual no puede explicar.
	\item Se ajustan las posiciones y confianza de aquellos que están en concordancia.
	\item Se eliminan los \emph{surfels} con baja confianza y que presentan conflictos con la nueva información.
\end{itemize}

Al finalizar el proceso, se utilizan las posiciones y normales de los \emph{surfels} para realizar una
triangulación y así obtener una malla que represente al objeto.
