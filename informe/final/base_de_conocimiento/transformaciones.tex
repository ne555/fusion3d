\subsection{Transformaciones}
Una transformación es una función $f(x) \rightarrow y$.
En particular nos interesan aquellas transformaciones donde
tanto la imagen como el dominio se encuentran en el espacio tridimensional, es decir, $\Real^3 \rightarrow \Real^3$.

Nos limitaremos a transformaciones que preservan las distancias entre los puntos
y los ángulos entre los vectores, manteniéndose siempre en un
sistema de referencia de mano derecha.
Es decir, nos limitaremos a rotaciones y translaciones.

Estas transformaciones pueden representarse en el espacio proyectivo mediante una matriz de $4\times4$.
\[ T_{jk} a_k = b_j \]
\[
	T_{jk} = \left[\begin{matrix}
		x'_1 & y'_1 & z'_1 & t_x \\
		x'_2 & y'_2 & z'_2 & t_y \\
		x'_3 & y'_3 & z'_3 & t_z \\
		0   &  0  &  0  &  1  \\
	\end{matrix}\right] =
	\left[
		\begin{matrix}
			x'_j & y'_j & z'_j & t_j \\
			0 & 0 & 0 & 1 \\
		\end{matrix}
	\right]
\]
Donde los vectores $x'_j, y'_j, z'_j$ son los vectores resultantes de aplicar la transformación a los ejes coordenados,
y el vector $t_j$ es la translación que sufre el origen.

Si se tienen dos transformaciones $A_{jk}$ y $B_{jk}$, la transformación que pasa desde $A_{jk}$ hacia $B_{jk}$ es
\begin{align*}
	A_{jk} x_k &= a_k \\
	B_{jk} x_k &= b_k \\
	C_{jk} a_k &= b_k \\
%
	C_{jr} A_{rk} x_k &= B_{jk} x_k \\
%
	C_{jr} A_{rk} &= B_{jk} \\
	C_{jk} &= B_{jr} A^{-1}_{rk}
\end{align*}
\Nota{¿sigue siendo ortogonal?}


\endinput

Transformación rígida

Matrix transformación lineal $3\times3$

Matriz proyectiva transformación afín $4\times4$

Translación

Cálculo de la matriz de transformación,
transformación de los ejes.

T_{jk} x_{k} = {x'}_{j}
T_{jk} y_{k} = {y'}_{j}
T_{jk} z_{k} = {z'}_{j}

Realizando la multiplicación
{x'}_j = T_{j1}
{y'}_j = T_{j2}
{z'}_j = T_{j3}

Es decir, la matriz de transformación tiene por columnas los vectores de la base en el espacio transformado

Si se tienen dos transformaciones A y B, la transformación que pasa desde A hacia B es
B_jk x_k = b_k
A_jk x_k = a_k
C_jk a_k = b_k
C_jk = ¿?


C_jr A_rk x_k = B_jk x_k

C_jr A_rk = B_jk

post-multiplicando por la inversa de A
y dado que es ortogonal A^-1_jk = A_kj
A_rj A_rk = I_jk (por definición de ortogonal) \Nota{¿o de ortonormal?}

C_jk = B_jr A_kr
\Nota{en el código lo hice al revés, C_jk = A_jr B_kr, es decir, la transformación inversa}


Extracción del angle-axis a partir de la matriz de rotación
\Nota{es un lío, se puede, pero no vale la pena explicarlo}


Muchas rotaciones pueden representarse por una sola. \TODO{Eso explicar}
usar la transformación a cuaternión
multiplicar cuaterniones da otro cuaternión
un cuaternión representa una rotación
\Nota{¿se mantiene unitario? ¿cuándo representaba escala?}
