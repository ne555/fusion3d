Matrix transformación lineal $3\times3$

Matriz proyectiva transformación afín $4\times4$

Translación

Cálculo de la matriz de transformación,
transformación de los ejes.

T_{jk} x_{k} = {x'}_{j}
T_{jk} y_{k} = {y'}_{j}
T_{jk} z_{k} = {z'}_{j}

Realizando la multiplicación
{x'}_j = T_{j1}
{y'}_j = T_{j2}
{z'}_j = T_{j3}

Es decir, la matriz de transformación tiene por columnas los vectores de la base en el espacio transformado

Si se tienen dos transformaciones A y B, la transformación que pasa desde A hacia B es
B_jk x_k = b_k
A_jk x_k = a_k
C_jk a_k = b_k
C_jk = ¿?


C_jr A_rk x_k = B_jk x_k

C_jr A_rk = B_jk

post-multiplicando por la inversa de A
y dado que es ortogonal A^-1_jk = A_kj
A_rj A_rk = I_jk (por definición de ortogonal) \Nota{¿o de ortonormal?}

C_jk = B_jr A_kr
\Nota{en el código lo hice al revés, C_jk = A_jr B_kr, es decir, la transformación inversa}


Extracción del angle-axis a partir de la matriz de rotación
\Nota{es un lío, se puede, pero no vale la pena explicarlo}


Muchas rotaciones pueden representarse por una sola. \TODO{Eso explicar}
usar la transformación a cuaternión
multiplicar cuaterniones da otro cuaternión
un cuaternión representa una rotación
\Nota{¿se mantiene unitario? ¿cuándo representaba escala?}
