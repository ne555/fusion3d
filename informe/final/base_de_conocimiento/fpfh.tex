%semantic 3d object maps (es basicamente la pcl)
Features:

Buscamos que el descriptor sea invariante respecto a translaciones y rotaciones,
con el fin de identificarlo en ambas vistas.

Point Feature Histogram (PFH)
Representa la relación entre los puntos de la vecindad y sus normales de superficie estimadas.
Intenta describir las variaciones de la superficie en la región.
Es altamente de la calidad de la estimación de las normales.

Todos los pares de puntos en la vecindad.

Dados dos puntos y sus normales, se define un sistema de referencia
y se calculan los ángulos entre las normales de los puntos y los ejes.

sistema de referencia:
u, es la normal en el punto A
v es ortogonal al plano definido por u y la translación entre los puntos
w es ortogonal a u y v (producto vectorial)

ángulos:
$\alpha$ es el ángulo de la normal en B contra el eje v
$\phi$ es el ángulo entre el eje u y el vector translación
$\theta$ es el ángulo entre la proyección de la normal en B sobre el plano uw, y el eje u

\begin{eqnarray}
	\alpha = v \cdot n_B \\
	\phi = u \cdot \frac{P_B - P_A}{|P_B - P_A|}
	\theta = \atan(w \cdot n_B, u \cdot n_B)
\end{eqnarray}

Significado gráfico de los valores

\TODO{gráfico: plano vectores {n_A, B-A}, proyección de n_B}

\alpha mide el desvío de las normales fuera de este plano, hacia arriba o abajo
\theta mide el desvío de las normales en la proyección del plano, es el ángulo entre n_A y la proyección de n_B
\phi mide la posición de los vecinos respecto al plano tangente, determina qué tan cóncava o convexa es la superficie



\section{\emph{Fast Point Feature Histograms} (FPFH)}
Diferencias respecto a PFH
\begin{itemize}
	\item FPFH No ve todos los vecinos del punto, pueden perderse pares importantes.
	\item FPFH puede salir fuera de la esfera de vecindad
	\item \Nota{no entiendo, ¿suma dos veces?}
	\item Reducción considerable en la complejidad
	\item histograma decorrelacionado
\end{itemize}

\Nota{duda ¿distribuye la muestra en las cubetas cercanas o sólo suma en la que cae?}
