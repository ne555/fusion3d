%semantic 3d object maps (es basicamente la pcl)
\section{Descriptores}
Para lograr encontrar la transformación de alineación entre dos vistas,
primeramente debemos hallar las correspondencias entre los puntos de ambas nubes.
Es decir, por cada punto de la nube $A$ identificar su posición en la nube $B$.
\Nota{no están todos los puntos y tampoco necesito todos los del área solapada}

Surge entonces la necesidad de definir un descriptor para cada punto.  Este
descriptor es un vector de características de la vecindad del punto, eligiendo
estas características de forma que sean invariantes respecto a translaciones y
rotaciones y robustas respecto a perturbaciones de los puntos de la vecindad y
el agregado o eliminación de puntos de la vecindad.

\Nota{Va devuelta.}
Un descriptor es un vector de características.
Esas características se calculan en la vecindad del punto.
Se pide que sea discriminante.
Se pida invarianza respecto a transformaciones de rotación y translación, porque justamente queremos encontrar correspondencias entre capturas rotadas y trasladadas.
Se pide robustez respecto a ruido de muestreo.
Se pide robustez respecto a oclusiones.
Se pide robustez respecto a la densidad del muestreo.
Con robustez quiero decir que la distancia no varíe demasiado respecto a esas perturbaciones.


\Nota{sigue}
Entonces, para comparar dos puntos se calcula la distancia entre sus descriptores.
Si esta distancia es cercana a $0$, los puntos son similares y existe la
posibilidad de que se correspondan.
Considerando las vecindades de los puntos se obtiene una estimación de la superficie,
superficies parecidas darán descriptores similares.


Vecindades

- obtener los k vecinos más cercanos. \Nota{No}
- obtener todos los vecinos que se encuentran a menos de una distancia r.
\Nota{Sí} De esta manera, se utiliza siempre el mismo tamaño de porción de
superficie, sin importar la cantidad de puntos muestreados o su posición y
ángulo respecto al dispositivo de captura.

Problema, ¿cómo definir el tamaño de la vecindad? ¿cómo hacerlo de forma completamente automática?
Si la vecindad es muy pequeña, no será lo suficientemente discriminante.
En cambio, si es muy grande, se considerarán puntos pertenecientes a otra superficie, distorsionando el descriptor.


Valores atípicos

Antes de calcular el vector de características, debe determinarse si los puntos
de la vecindad representad adecuadamente la superficie muestreada.

Muchos descriptores requieren un mínimo de vecinos para poder calcularse.
Debido a variaciones en la densidad de muestreo, características de interacción
entre la superficie y el dispositivo de captura o oclusiones, es posible que
existan puntos que no cumplan con esta restricción.
Eliminar estos puntos disminuye el costo de procesamiento.


Estimación de la normal y la curvatura

Se aproxima la normal en el punto estimando un plano tangente a la superficie
mediante el método de mínimos cuadrados.
El plano se representa como un punto $x_j$ y un vector normal $n_j$, la distancia de un punto
al plano será
\[ d = (p_j - x_j) n_j \]

Los valores de $x_j$ y $n_j$ se calcularan de forma de minimizar el promedio de la distancia al considerar todos los puntos de la vecindad.
De esta forma, tomando el centroide de los puntos de la vecindad
\[x_j = \frac{1}{N}\sum_{j=1}{N} p_j\]
y calculando los eigenvectores de la matriz de covarianza.
Se tiene ambigüedad en el sentido del vector, lo cual puede resolverse orientándolo hacia el dispositivo de captura.
\TODO{seguir}

\subsection{\emph{Point Feature Histogram} (PFH)}
Representa la relación entre los puntos de la vecindad y sus normales de superficie estimadas.
Intenta describir las variaciones de la superficie en la región.
Es altamente sensible a la calidad de la estimación de las normales.

Todos los pares de puntos en la vecindad.

Dados dos puntos y sus normales, se define un sistema de referencia
y se calculan los ángulos entre las normales de los puntos y los ejes.

sistema de referencia:
u, es la normal en el punto A
v es ortogonal al plano definido por u y la translación entre los puntos
w es ortogonal a u y v (producto vectorial)

ángulos:
$\alpha$ es el ángulo de la normal en B contra el eje v
$\phi$ es el ángulo entre el eje u y el vector translación
$\theta$ es el ángulo entre la proyección de la normal en B sobre el plano uw, y el eje u

\begin{align*}
	\alpha &= v \cdot n_B \\
	\phi &= u \cdot \frac{P_B - P_A}{|P_B - P_A|}\\
	\theta &= \atan(w \cdot n_B, u \cdot n_B)
\end{align*}

Significado gráfico de los valores

\TODO{gráfico: plano vectores {n\_A, B-A}, proyección de n\_B}

$\alpha$ mide el desvío de las normales fuera de este plano, hacia arriba o abajo
$\theta$ mide el desvío de las normales en la proyección del plano, es el ángulo entre $n_A$ y la proyección de $n_B$
$\phi$ mide la posición de los vecinos respecto al plano tangente, determina qué tan cóncava o convexa es la superficie



\subsection{\emph{Fast Point Feature Histograms} (FPFH)}
Diferencias respecto a PFH
\begin{itemize}
	\item FPFH No ve todos los vecinos del punto, pueden perderse pares importantes.
	\item FPFH puede salir fuera de la esfera de vecindad
	\item \Nota{no entiendo, ¿suma dos veces?}
	\item Reducción considerable en la complejidad
	\item histograma decorrelacionado
\end{itemize}

\Nota{duda ¿distribuye la muestra en las cubetas cercanas o sólo suma en la que cae?}
