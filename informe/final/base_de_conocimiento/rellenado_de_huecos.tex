\section{Rellenado de huecos}
Los huecos son regiones que ninguna de las vistas logró capturar,
por lo que se carece de información de la superficie en esas zonas.
La existencia de huecos elimina la propiedad de \emph{watertight} de la malla,
la cual es necesaria para su impresión 3D,
por lo tanto, se deben estimar los valores de la superficie en los huecos.

%Idealmente, los algoritmos de rellenado de huecos deberían poseer las siguientes propiedades:
%\begin{itemize}
%	\item Robustez: se deben rellenar huecos de formas arbitrarias.
%	\item Eficiencia: se deben rellenar huecos grandes en tiempos razonables.
%	\item Precisión: la superficie reconstruida debe estimar correctamente los puntos faltantes.
%\end{itemize}
%\Nota{¿cómo mido la precisión?}

Las características de los huecos dependerán no sólo de la geometría del objeto y las
posiciones desde donde se realizaron las captura, sino también del método usado
para integrar las capturas.
Por ejemplo, el método de fusión volumétrica nos presenta una superficie sin huecos.


A continuación se presentarán métodos para el rellenado los huecos de forma de conseguir una malla cerrada.

\subsection{Identificación de los huecos}
Antes de poder rellenar cada hueco es necesario identificar los vértices y aristas que los delimitan.
Este proceso se facilita al utilizar una representación
de lista de medias aristas (DCEL) para describir la malla.

En esta representación, se almacena cada cara, arista y vértice de la malla.
Cada arista se divide en dos medias aristas gemelas, que comparten los mismos vértices
pero poseen orientaciones opuestas.
Cada media arista posee punteros \emph{siguiente} y \emph{anterior},
formando la triada que delimita una cara (aquella que queda a la izquierda al recorrerla).

Si una media arista no delimita una cara, entonces forma parte del borde de la malla.
Siguiendo los punteros, se puede recorrer este borde, obteniéndose su contorno.
Como la malla que reconstruye el objeto debe ser cerrada, no puede presentar aristas de borde,
por lo cual estos bordes nos identifican los huecos.
Debe hacerse la salvedad de que lo anterior es cierto únicamente cuando la malla conste
de una sola componente conectada.
En caso de que la malla conste de varias porciones desconectadas,
se dificulta la identificación del borde del hueco al estar formado por el contorno de diversas mallas.



\subsection{Advancing front}
Los algoritmos de advancing front construyen una malla del dominio a partir de su contorno.
El proceso puede ser resumido de la siguiente manera:
\begin{enumerate}
	\item Inicialización del frente.
	\item Análisis del frente para definir un candidato.
	\item Creación de nuevos elementos a partir del candidato.
	\item Actualización del frente.
	\item Repetición de los pasos 2--4 hasta vaciar el frente.
\end{enumerate}

Podemos utilizar estas técnicas para el rellenado de huecos
inicializando el frente a partir del contorno de cada hueco
y rellenando hacia su interior.
Los diversos métodos se diferenciarán según la elección del próximo candidato
y la determinación de las posiciones de los nuevos puntos.
En este último punto debe considerarse que el nuevo punto debe caer dentro del dominio y
los nuevos triángulos no deben ser extremadamente aplanados.

Los detalles de implementación serán discutidos en un capítulo posterior.

\subsection{Reconstrucción de Poisson}
En este método el rellenado de huecos se aborda como un problema de
reconstrucción de la superficie a partir de puntos orientados.  Por esta
razón, se consideran todos los puntos a la vez sin consideraciones
especiales de pertenencia o no a los contornos de los huecos.
Además, se convierte el problema de la reconstrucción a la resolución de
una ecuación de Poisson.

En esta ecuación de Poisson, se calcula una función indicadora $\chi$
que toma el valor 1 en los puntos que se encuentran dentro del modelo
y 0 en los puntos que se encuentran fuera.
De esta manera, extrayendo una isosuperficie apropiada se obtiene la frontera del modelo.

Para poder plantear la ecuación, debe notarse que el gradiente de la función indicadora $\nabla \chi$
es 0 en casi todos los puntos, excepto en aquellos cerca de la superficie,
donde es equivalente a las normales de los puntos muestreados ($\vec{V}$), pero en sentido contrario (figura~\ref{fig:poiss_planteo}).
Entonces, se busca calcular la función $\chi$ cuyo gradiente aproxime de la mejor forma posible el
campo vectorial $\vec{V}$.
Aplicando  el operador de divergencia se obtiene la ecuación de Poisson
\[ \Delta\chi \equiv \nabla \cdot \nabla \chi = \nabla\vec{V} \]

\begin{figure}
	\Imagen{foo}
	\caption{\label{fig:poiss_planteo}Ilustración del método de reconstrucción de Poisson.
	Tomando como entrada una nube de puntos con sus normales $\vec{V}$ se define el gradiente de la función
	indicadora $\nabla \chi$. Mediante la ecuación de Poisson se calculará el valor de $\chi$,
	y al extraer una isosuperfice se obtendrá la superficie del objeto $\partial M$. 
	}
\end{figure}

Entre las ventajas de esta formulación podemos nombrar:
\begin{itemize}
	\item La superficie reconstruida es una superficie suave y cerrada.
	\item Los sistemas de Poisson son resistentes al ruido de entrada.
	\item El uso de funciones de soporte local para aproximar $\chi$,
		produce un sistema lineal bien condicionado y ralo.\cite{Kazhdan:2006:PSR:1281957.1281965}
\end{itemize}

	\subsubsection{Condiciones de borde}
	Para definir el sistema lineal, es necesario especificar el
	comportamiento de la función a lo largo del contorno del dominio de
	integración.
	Tomando en consideración la definición de la función indicadora $\chi$,
	se pueden establecer fácilmente condiciones de borde Dirichlet ($\chi = 0$)
	o también condiciones de borde Neumann ($\nabla \chi = 0$).

	Si se utilizan condiciones Dirichlet, se asegurará que la superficie resultante sea cerrada.
	En cambio, en caso de que se tengan regiones sin información (huecos) de tamaño
	considerable, usar condiciones Neumann permitirá que la superficie se
	extienda fuera del dominio de integración, cruzando el borde de forma
	ortogonal (figura~\ref{fig:poisson_boundary})\cite{Kazhdan_screenedpoisson}.
	\begin{figure}
		\Imagen{diagram/poisson_boundary_conditions}
		\caption{\label{fig:poisson_boundary}Reconstrucción del modelo Ángel (izquierda) usando condiciones de borde Dirichlet (centro) y Neumann (derecha).}
	\end{figure}


	\subsubsection{Extracción de la isosuperficie}
	Para poder obtener la superficie del modelo $\partial M$, es necesario
	definir un isovalor de $\chi$ para así obtener la isosuperficie correspondiente.
	Para aproximar las posiciones de los puntos de entrada, se utiliza como isovalor
	el promedio de los valores de $\chi$ en esos puntos.

	\[\partial M \equiv {q \in \mathbb{R}^3 | \chi(q) = \gamma} \quad \mbox{con}
		\quad \gamma = \frac{1}{N} \sum \chi(p) \]

	Esta elección de isovalor presenta la propiedad de que un escalado de $\chi$ no cambia la isosuperficie.
