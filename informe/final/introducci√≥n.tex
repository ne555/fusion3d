\chapter{Introducción}
%zippered
%This paper presents a method of combining multiple views of an
%object, captured by a range scanner, and assembling these views into
%one unbroken polygonal surface.  Applications for such a method
%include:
%• Digitizing complex objects for animation and visual simulation.
%• Digitizing the shape of a found object such as an archaeological
%artifact for measurement and for dissemination to the scientific
%• Digitizing human external anatomy for surgical planning,
%remote consultation or the compilation of anatomical atlases.
%• Digitizing the shape of a damaged machine part to help create
%a replacement.

	\section{Justificación}
	La impresión 3D es un método de fabricación en el cual se deposita el material
	(como por ejemplo, plástico o metal) en capas para producir un objeto tridimensional.
	Avances en los materiales utilizados han permitido la creación de objetos
	de calidad comparable a la de métodos tradicionales,
	demostrando su viabilidad para muchas aplicaciones médicas,
	incluyendo la creación de prótesis e implantes dentales\cite{Schubert159}. %\cite{innovations in 3d printing}
	%con el potencial de personalizar cada producto.
	Los modelos geométricos que desean imprimirse pueden obtenerse aplicando procesos de
	reconstrucción tridimensional (figura~\ref{fig:ingenieria_inversa}) a objetos ya existentes.

	\begin{figure}
		\centering
		\begin{subfigure}{.3\linewidth}
			\Imagen{img/face-02}
			\caption{\label{fig:cara}Objeto}
		\end{subfigure}
		\begin{subfigure}{.3\linewidth}
			\Imagen{img/geometric_model}
			\caption{\label{fig:cara_3d}Modelo geométrico}
		\end{subfigure}
		\begin{subfigure}{.3\linewidth}
			\Imagen{img/print_3d}
			\caption{\label{fig:cara_print_3d}Impresión 3D}
		\end{subfigure}
		\caption{\label{fig:ingenieria_inversa}Reconstrucción tridimensional.
		A partir de un objeto (\ref{fig:cara}) se obtiene su modelo geométrico (\ref{fig:cara_3d})
		para lograr su posterior impresión 3D (\ref{fig:cara_print_3d}).}
	\end{figure}

	Entre las aplicaciones de la reconstrucción tridimensional podemos nombrar:
	\begin{itemize}
		\item Reconstrucción de órganos para la planificación de cirugías
			y la realización de prácticas por parte de estudiantes de ciencias médicas y veterinarias.
			En particular, la empresa argentina \emph{mirai}\footnote{\url{https://www.modelosmedicos.com/}} realiza estos biomodelos basándose en imágenes de resonancia magnética y tomografía computarizada.
			%mirai 3d, empresa argentina, adquiere los biomodelos mediante tomografía axial computarizada (TAC)
			\ImagenInline{img/corazon_1}
		\item Digitalización de obras de arte para salvaguardar el patrimonio cultural,
			ayudar en su restauración e incrementar la accesibilidad a las mismas.
			A este respecto, puede mencionarse el \emph{Digital Michelangelo Project} de la Universidad de Stanford, que tiene como objetivo crear un repositorio de modelos 3D de alta calidad de las esculturas y la arquitectura de Miguel Ángel.
			\ImagenInline{img/estatua_leon}
		\item Realización de prototipos y modificaciones a productos.
			\ImagenInline{img/anteojos}
		\item Simulación de propiedades físicas sobre los objetos.
			\ImagenInline{img/bunny_heat}
		\item Digitalización de ambientes y objetos para utilizarlos en aplicaciones de realidad virtual o de diseño.
			\ImagenInline{img/room}
	\end{itemize}

	Uno de los métodos para la obtención de los modelos geométricos
	se basa en la utilización de dispositivos de luz estructurada,
	como ser el dispositivo Kinect\cite{MatheEstudioKinect},
	para realizar mediciones sobre el objeto.
	Debido a que los puntos capturados sólo reflejan la porción de la superficie del objeto visible desde la cámara
	y a que la propia geometría del objeto podría generar obstrucciones, reflejos o sombras
	que imposibilitan la adquisición de datos en esas zonas («huecos»),
	es necesario combinar múltiples capturas
	del objeto en diferentes orientaciones respecto a la cámara, de modo que
	cada parte de la superficie del objeto esté representada en al menos una captura (figura~\ref{fig:kinect_reconstruction}).

	\begin{figure}
		\Imagen{img/loop_multiple_view}
		\caption{\label{fig:kinect_reconstruction}Medición de las características geométricas de los objetos mediante múltiples capturas utilizando cámaras de profundidad.}
	\end{figure}

	%Describir los pasos de la ingeniería inversa: adquisición alineación, encuentro de superficie, descripción, segmentación, modelo CAD
	El principal problema para combinar las múltiples vistas es
	encontrar las transformaciones de rotación y translación que permitan
	relacionar cada vista parcial, para luego fusionarlas y así reconstruir el objeto.
	

	Para esto existen métodos automáticos que requieren que la posición del dispositivo de captura
	entre las distintas capturas no haya variado de forma significativa\cite{regBesl92},
	lo cual genera que el proceso de adquisición de datos sea engorroso,
	requiriendo de una supervisión constante para asegurar que el movimiento del dispositivo de captura no sea excesivo
	y donde un fallo puede exigir el reinicio del proceso.
	Este es el caso de \emph{KinectFusion}, un algoritmo desarrollado por Microsoft que utiliza el sensor Kinect para reconstruir escenas tridimensionales en tiempo real,
	pero que presenta las siguientes limitaciones:
	\begin{itemize}
		\item utiliza un alto costo computacional, requiriendo de una poderosa GPU;
		\item los movimientos de cámara deben ser lentos y pequeños;
		\item está limitado al dispositivo Kinect;
		\item algunos objetos podrían no aparecer en el mapa de profundidad debido a que absorben o reflejan demasiada luz infrarroja;
		\item no utiliza información de color.\cite{real-time-3d-reconstruction-using-a-kinect-sensor}
	\end{itemize}

	Existen, además, métodos manuales o semiautomáticos donde se permiten distancias mayores entre las posiciones del dispositivo de captura.
	Por ejemplo, el usuario posiciona de forma interactiva las capturas\cite{Turk:1994:ZPM:192161.192241}
	o selecciona iterativamente puntos coincidentes entre ellas\cite{LocalChapterEvents:ItalChap:ItalianChapConf2008:129-136}.

	Con el fin de evitar o limitar el uso de los recursos humanos,
	en este proyecto se analizarán técnicas de fusión de mallas,
	obtenidas permitiendo mayor distancia entre capturas,
	para lograr un modelo tridimensional de un objeto que permita su posterior replicación
	mediante una impresora 3D.



	\section{Objetivos}
		\subsection{General}
			Diseñar y desarrollar una biblioteca de funciones que permita
			la reconstrucción de una malla 3D cerrada a partir de mallas de superficies parciales.
		\subsection{Específicos}
		\begin{itemize}
			%Falta un objetivo específico relacionado con el análisis de técnica de fusión de mallas.
			\item Desarrollar algoritmos que determinen las posiciones relativas de cada malla parcial.
			\item Desarrollar algoritmos de fusión de las mallas.
			\item Desarrollar algoritmos para rellenado de huecos producto de obstrucciones.
			\item Evaluar el desempeño de cada bloque desarrollado y realizar ajustes.
		\end{itemize}

	\section{Alcance}
	\begin{itemize}
		\item Para posibilitar la posterior impresión 3D,
			la malla resultante deberá ser una malla cerrada (\emph{watertight}).
		\item No se impondrán topologías específicas a los objetos a reconstruir.
		\item Los algoritmos desarrollados deberán ser robustos frente al ruido producido por los sensores.
		\item No se pretenderá el funcionamiento de los algoritmos en tiempo real.
		\item La biblioteca será de código libre y multiplataforma.
	\end{itemize}
