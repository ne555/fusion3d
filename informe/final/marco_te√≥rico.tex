En este capítulo se presenta una introducción a los conceptos necesarios para
comprender el desarrollo del proyecto.
\TODO{\ldots\\}


\input{base_de_conocimiento/reconstrucción}

%\section{Definiciones}
\begin{description}
	\item [Imagen]
		Una imagen representa los valores que adquiere una función $f$
		en puntos discretos (píxeles) ordenados en una grilla rectangular.
	\item [Imagen 2.5D o imagen de profundidad]
		Representación bidimensional de puntos en el espacio, donde cada píxel
		almacena el valor de la profundidad $z$.
		\Nota{Aunque faltaría obtener los parámetros intrínsecos de la cámara\ldots,
		o sea, ¿qué coordenadas tiene el punto? no lo sé, tengo $z$, ¿pero $\{x,y\}$?
		}
	\item [Nube de puntos]
		Una nube de puntos es una colección de puntos cuyas coordenadas espaciales
		están definidas respecto a un sistema de referencia fijo.
	\item [Malla]
		Una malla representa una superficie al establecer conectividades entre
		los puntos de una nube y definir de esta forma caras y aristas.
	\item [Captura]
		Una captura es la adquisición de las coordenadas $\{x, y, z\}$ de los puntos de una nube
		por medio de sensores y técnicas de procesamiento adecuado.
		\TODO{z apunta al ojo}
	\item [Vista]
		Una vista es una captura realizada a partir de cierta posición sensor-escena específica.
	\item [Vecindad]
		Para un punto $p$, su vecindad son aquellos puntos que se encuentran
		dentro de una esfera de radio $r$ centrada e $p$. Es decir, aquellos
		puntos $q$ en los que se satisface que
		\[ |q - p| \leq r \]
\end{description}

\section{Adquisición de la nube de puntos}
Al definir el origen del sistema de referencia centrado en el dispositivo de captura,
las coordenadas de cada punto de la nube representan la distancia entre el sensor y la superficie
de la cual fue muestreado el punto.
Para lograr medir esta distancia, se puede hacer uso de sistemas de luz estructurada,
donde un patrón conocido es proyectado en la escena y se analizan sus deformaciones.

Debido a las características geométricas de los objetos que componen la escena,
no se podrá obtener una representación de todas las superficies a partir de tan sólo
una vista. Es necesario entonces combinar varias vistas para lograr capturar
esas superficies que permanecían ocultas. Sin embargo, debido a que
cada captura se encuentra en un sistema de referencia distinto,
esta combinación no es trivial, debiendo determinarse las transformaciones que lleven
cada vista a un sistema de referencia global.

El proceso de obtener estas transformaciones se denomina registración.

%\section{\emph{Iterative Closest Point} (ICP)}
%generalized icp
%darpa grand challenge

%icp está en a method for registration of 3d shapes
Dado un cuaternión $q$ y una translación $t$
se puede definir el vector de registración como $T = [q|t]$.
Tomando el conjunto de puntos $P = {p_j}$ a ser alineados
con el modelo $X = {x_j}$ ambos de la misma cardinalidad
y correspondiendo de la forma $p_j \rightarrow x_j$.
\Nota{se conocen todas las correspondencias de entrada}
La función objetivo a minimizar se define como
\[ f(T) = \frac{1}{N} \sum || x_j - q p_j - t || \]

Se calcula la matriz de covarianza cruzada de los conjuntos $P$ y $X$
\[ \Sigma_{px} = \frac{1}{N} p_j x_j - \mu_p \mu_x \]

Los componentes cíclicos \Nota{¿?} de la matriz $A_{jk} = \Sigma_{jk} - \Sigma_{kj}$
forman el vector $\Delta = [A_{23} A_{31} A_{12}]$
que arma la matriz simétrica de $4\times4$
\[
	Q = \left[
		traza(\Sigma) \Delta
		\Delta'   \quad  algo
		\right]
	\]

El eigenvector correspondiente al mayor eigenvalor es el cuaternión que produce la rotación óptima.
Para la translación
\[t = \mu_x - q \mu_p \]
\Nota{Supongo que cualquier punto funcionaría, total tenemos las correspondencias.
Nope, los puntos no coinciden exactamente, usar la media}

\Nota{Ahora define ICP}
Con cualquier P, busca los más cercanos en X, así establece las correspondencias,
cumpliéndose lo de la cardinalidad.

Buscar la mejor alineación de P en X.

Distancia entre un punto y el modelo
\[ d(p, X) = \min_{x \in X} ||x-p|| \]

Algoritmo

Dado $P$ con $N_p$ puntos y el modelo $X$

Inicialización $P_0 = P$, $T = \mbox{Identidad}$
\begin{enumerate}
	\item Obtener las correspondencias, es decir, para cada $p_j$ el punto más cercano en $X$.
		\Nota{¿qué pasa si no es recíproco o hay varios?}
	\item Calcular la registración óptima con la matriz mágica $Q$ \Nota{fórmula cerrada en point-to-point euclídeo}
	\item Aplicar la registración
	\item Iterar. Cortar cuando el error cuadrático medio baje de un umbral
		$\tau \sqrt{tr\left(\Sigma_x\right)}$ (adimensional, $\tau$ está en proporción del tamaño del modelo)
\end{enumerate}


Desventajas

susceptible a outliers \TODO{traducir}



%generalized icp
Problemas:
	supone que hay un solapamiento total entre las nubes a ser solapadas,
	requiere que los puntos provengan de una superficie geométrica conocida (no medido)

En realidad el solapamiento es parcial, \Nota{y con más razón en nuestro caso}.
Se resuelve con un umbral de cercanía.

Lo de superficie conocida jode con diferentes discretizaciones,
no se tiene una correspondencia exacta después de la convergencia. Resuelto por point-to-plane


En resumidas cuentas, ICP hace
\begin{enumerate}
	\item Calcula correspondencias
	\item Calcula la transformación que minimiza la distancia entre las correspondencias
\end{enumerate}

%object modeling 
%Introduce variante point-to-plane
\subsection{Point-to-plane}
Utiliza información de la normal de las superficies.
Es una variante más robusta y precisa de ICP al considerar imágenes 2.5D.
Dado el error de registración $ \mbox{error}_j = T_{jk} p_k - x_j$
En lugar de encontrar $T$ que minimice la norma del error, minimiza la proyección sobre la normal de la superficie objetivo
\[ \argmin_T \left\{ w_k n_j \mbox{error}_j \right\} \]

\subsection{Plane-to-plane}
Mejorar point-to-plane y aumentar la simetría del modelo \Nota{¿? ¿por qué me interesa eso?}
Generalized ICP toma información de ambas capturas

La clave de point-to-plane es que los puntos fueron muestreados 2.5D, no son arbitrarios.
Es un 2-manifold en 3-space \Nota{¿qué?}

Supone que los puntos están localmente en un plano. No muestreamos el mismísimo punto.
Cada punto sólo presenta una restricción a lo largo de su normal.
Alta covarianza en el plano, muy baja en la normal; es decir, conocemos con mucha confianza
la posición a lo largo de la normal, pero no estamos seguros de dónde está en el plano.

Eliminar correspondencias que den normales inconsistentes.
Puede verse como restricciones de movimiento para cada correspondencia.
Las correspondencia erróneas forman restricciones muy débiles en el alineamiento global \Nota{No entendí}

\Nota{y eso es todo lo que tenía para decir acerca de esto}

\TODO{estimación de la normal mediante matriz de covarianza}
%semantic 3d object maps (es basicamente la pcl)
\section{Descriptores}
Para lograr encontrar la transformación de alineación entre dos vistas,
primeramente debemos hallar las correspondencias entre los puntos de ambas nubes.
Es decir, por cada punto de la nube $A$ identificar su posición en la nube $B$.
\Nota{no están todos los puntos y tampoco necesito todos los del área solapada}

Surge entonces la necesidad de definir un descriptor para cada punto.  Este
descriptor es un vector de características de la vecindad del punto, eligiendo
estas características de forma que sean invariantes respecto a translaciones y
rotaciones y robustas respecto a perturbaciones de los puntos de la vecindad y
el agregado o eliminación de puntos de la vecindad.

\Nota{Va devuelta.}
Un descriptor es un vector de características.
Esas características se calculan en la vecindad del punto.
Se pide que sea discriminante.
Se pida invarianza respecto a transformaciones de rotación y translación, porque justamente queremos encontrar correspondencias entre capturas rotadas y trasladadas.
Se pide robustez respecto a ruido de muestreo.
Se pide robustez respecto a oclusiones.
Se pide robustez respecto a la densidad del muestreo.
Con robustez quiero decir que la distancia no varíe demasiado respecto a esas perturbaciones.


\Nota{sigue}
Entonces, para comparar dos puntos se calcula la distancia entre sus descriptores.
Si esta distancia es cercana a $0$, los puntos son similares y existe la
posibilidad de que se correspondan.
Considerando las vecindades de los puntos se obtiene una estimación de la superficie,
superficies parecidas darán descriptores similares.


Vecindades

- obtener los k vecinos más cercanos. \Nota{No}
- obtener todos los vecinos que se encuentran a menos de una distancia r.
\Nota{Sí} De esta manera, se utiliza siempre el mismo tamaño de porción de
superficie, sin importar la cantidad de puntos muestreados o su posición y
ángulo respecto al dispositivo de captura.

Problema, ¿cómo definir el tamaño de la vecindad? ¿cómo hacerlo de forma completamente automática?
Si la vecindad es muy pequeña, no será lo suficientemente discriminante.
En cambio, si es muy grande, se considerarán puntos pertenecientes a otra superficie, distorsionando el descriptor.


Valores atípicos

Antes de calcular el vector de características, debe determinarse si los puntos
de la vecindad representad adecuadamente la superficie muestreada.

Muchos descriptores requieren un mínimo de vecinos para poder calcularse.
Debido a variaciones en la densidad de muestreo, características de interacción
entre la superficie y el dispositivo de captura o oclusiones, es posible que
existan puntos que no cumplan con esta restricción.
Eliminar estos puntos disminuye el costo de procesamiento.


Estimación de la normal y la curvatura

Se aproxima la normal en el punto estimando un plano tangente a la superficie
mediante el método de mínimos cuadrados.
El plano se representa como un punto $x_j$ y un vector normal $n_j$, la distancia de un punto
al plano será
\[ d = (p_j - x_j) n_j \]

Los valores de $x_j$ y $n_j$ se calcularan de forma de minimizar el promedio de la distancia al considerar todos los puntos de la vecindad.
De esta forma, tomando el centroide de los puntos de la vecindad
\[x_j = \frac{1}{N}\sum_{j=1}{N} p_j\]
y calculando los eigenvectores de la matriz de covarianza.
Se tiene ambigüedad en el sentido del vector, lo cual puede resolverse orientándolo hacia el dispositivo de captura.
\TODO{seguir}

\subsection{\emph{Point Feature Histogram} (PFH)}
Representa la relación entre los puntos de la vecindad y sus normales de superficie estimadas.
Intenta describir las variaciones de la superficie en la región.
Es altamente sensible a la calidad de la estimación de las normales.

Todos los pares de puntos en la vecindad.

Dados dos puntos y sus normales, se define un sistema de referencia
y se calculan los ángulos entre las normales de los puntos y los ejes.

sistema de referencia:
u, es la normal en el punto A
v es ortogonal al plano definido por u y la translación entre los puntos
w es ortogonal a u y v (producto vectorial)

ángulos:
$\alpha$ es el ángulo de la normal en B contra el eje v
$\phi$ es el ángulo entre el eje u y el vector translación
$\theta$ es el ángulo entre la proyección de la normal en B sobre el plano uw, y el eje u

\begin{align*}
	\alpha &= v \cdot n_B \\
	\phi &= u \cdot \frac{P_B - P_A}{|P_B - P_A|}\\
	\theta &= \atan(w \cdot n_B, u \cdot n_B)
\end{align*}

Significado gráfico de los valores

\TODO{gráfico: plano vectores {n\_A, B-A}, proyección de n\_B}

$\alpha$ mide el desvío de las normales fuera de este plano, hacia arriba o abajo
$\theta$ mide el desvío de las normales en la proyección del plano, es el ángulo entre $n_A$ y la proyección de $n_B$
$\phi$ mide la posición de los vecinos respecto al plano tangente, determina qué tan cóncava o convexa es la superficie



\subsection{\emph{Fast Point Feature Histograms} (FPFH)}
Diferencias respecto a PFH
\begin{itemize}
	\item FPFH No ve todos los vecinos del punto, pueden perderse pares importantes.
	\item FPFH puede salir fuera de la esfera de vecindad
	\item \Nota{no entiendo, ¿suma dos veces?}
	\item Reducción considerable en la complejidad
	\item histograma decorrelacionado
\end{itemize}

\Nota{duda ¿distribuye la muestra en las cubetas cercanas o sólo suma en la que cae?}

\section{Álgebra}
Matrix transformación lineal $3\times3$

Matriz proyectiva transformación afín $4\times4$

Translación

Cálculo de la matriz de transformación,
transformación de los ejes.

T_{jk} x_{k} = {x'}_{j}
T_{jk} y_{k} = {y'}_{j}
T_{jk} z_{k} = {z'}_{j}

Realizando la multiplicación
{x'}_j = T_{j1}
{y'}_j = T_{j2}
{z'}_j = T_{j3}

Es decir, la matriz de transformación tiene por columnas los vectores de la base en el espacio transformado

Si se tienen dos transformaciones A y B, la transformación que pasa desde A hacia B es
B_jk x_k = b_k
A_jk x_k = a_k
C_jk a_k = b_k
C_jk = ¿?


C_jr A_rk x_k = B_jk x_k

C_jr A_rk = B_jk

post-multiplicando por la inversa de A
y dado que es ortogonal A^-1_jk = A_kj
A_rj A_rk = I_jk (por definición de ortogonal) \Nota{¿o de ortonormal?}

C_jk = B_jr A_kr
\Nota{en el código lo hice al revés, C_jk = A_jr B_kr, es decir, la transformación inversa}


Extracción del angle-axis a partir de la matriz de rotación
\Nota{es un lío, se puede, pero no vale la pena explicarlo}


Muchas rotaciones pueden representarse por una sola. \TODO{Eso explicar}
usar la transformación a cuaternión
multiplicar cuaterniones da otro cuaternión
un cuaternión representa una rotación
\Nota{¿se mantiene unitario? ¿cuándo representaba escala?}


\TODO{Fusión}

\TODO{Rellenado de huecos}

%%Proceso de reconstrucción general
%%Reverse engineering of geometric models
%El objetivo final de los sistemas de ingeniería inversa
%es lograr realizar un escáner 3D inteligente.
%%Diagrama de flujo
%1- Obtención de datos
%2- Preproceso
%3- Segmentación y surface ¿fitting? (la mejor superficie que representa los puntos
%4- Obtención modelo CAD
%
%%Prefiero este
%1- Adquisición de datos
%2- Preproceso
%3- Registración
%4- Integración
%	Advancing front
%		4a- Fusión
%		4b- Rellenado de huecos
%	Poisson
%
%La 1 está resuelta por Pancho
%	pero está limitada debido a consideraciones físicas
%	se necesita combinar múltiples vistas.




