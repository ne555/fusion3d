\chapter{Uso de la biblioteca}
En este apéndice se presenta una reseña de la biblioteca desarrollada.
Se explica brevemente la instalación, compilación y el uso de las funciones fundamentales
que provee cada módulo de la misma.

	\section{Instalación y compilación}
	La biblioteca se encuentra contenida en archivos de cabecera, los cuales serán incluidos por los fuentes del proyecto.
	Se requiere la instalación de las siguientes dependencias:
	\begin{itemize}
		\item \texttt{PCL} versión 1.9 (\url{https://pointclouds.org/downloads/}).
		\item \texttt{delaunator-cpp} versión 0.4 (\url{https://github.com/delfrrr/delaunator-cpp}).
		\item \texttt{DKM} (\url{https://github.com/genbattle/dkm}).
	\end{itemize}

	Para la compilación se recomienda el uso de la herramienta \texttt{cmake} con el siguiente esqueleto para su archivo de configuración \texttt{CMakeLists.txt}:
	\inputminted{cmake}{code/CMakeLists.txt}

	\section{Representación de las capturas}
		La clase \verb|cloud_with_transformation|
			agrupa la información de posición y normales de cada punto,
			y la transformación de alineación.
			Durante la registración, las posiciones y normales se mantienen en dos nubes separadas,
			sin embargo, los módulos de fusión y rellenado de huecos
			requieren esta información integrada en una sola nube.
			Para convertir a este segundo formato se provee la función 
			\mintinline{cpp}{join_cloud_and_normal(nube)}.
		%\item \verb|alignment|: realiza la registración de dos nubes mediante el método de búsqueda de clústeres.
		%\item \verb|fusion|: realiza la fusión de las nubes una vez alineadas.
		%\item \verb|advancing_front|: realiza el rellenado de los huecos mediante el algoritmo basado en advancing front.

	\section{Módulo de preproceso}
	\begin{itemize}
		\item Carga la nube desde un archivo de disco y la prepara para ser usada por el módulo de registración
	\inputminted{cpp}{code/preproceso.cpp}
	\end{itemize}

	\section{Módulo de registración}
	\begin{itemize}
		\item Alineación inicial de dos nubes
			\inputminted{cpp}{code/pairwise_reg.cpp}

		\item Ajuste mediante el algoritmo de ICP
			\inputminted{cpp}{code/pairwise_icp.cpp}

		\item Corrección de bucle.
			Se realiza sobre un vector que contiene todas las capturas realizadas alrededor del objecto.
			\inputminted{cpp}{code/loop.cpp}

		\item Medición de la calidad de la registración
			\inputminted{cpp}{code/fitness.cpp}
	\end{itemize}

	\section{Módulo de fusión}
	\begin{itemize}
		\item Fusión de las capturas alineadas y triangulación de la nube resultante
			\inputminted{cpp}{code/fusion.cpp}
	\end{itemize}

	\section{Rellenado de huecos}
	\begin{itemize}
		\item Detección de huecos
			\inputminted{cpp}{code/holes.cpp}

		\item Rellenado mediante el método de advancing front.
			\emph{¡Advertencia! El algoritmo falla excepto en huecos pequeños o planos.
			No se recomienda su uso}
			\inputminted{cpp}{code/adv_front.cpp}
	\end{itemize}

	\section{Reconstrucción de Poisson}
	\begin{itemize}
		\item Utilización del algoritmo de reconstrucción de Poisson implementado en la biblioteca PCL
			\inputminted{cpp}{code/poisson.cpp}
	\end{itemize}
