\section{Módulo de rellenado de huecos}
	Al finalizar el algoritmo de fusión se obtuvo una malla triangular a partir
	de la información proveniente de cada vista.
	Sin embargo, esta malla no es cerrada ya que existen zonas que ninguna
	vista pudo capturar y por lo tanto carecen de puntos, produciendo huecos en la misma.
	Además, es posible observar la presencia de islas, es decir, puntos dentro de los huecos que no lograron conectarse con el resto de la malla.

	El módulo de rellenado de huecos se encargará de estimar, de forma automática, la superficie del
	objeto en estas zonas para así obtener finalmente una malla cerrada.

	\subsection{Diagrama de clases}
		En la figura~\ref{fig:filling_class} se presentan las clases principales y sus interacciones.
		A continuación se presenta una breve descripción de las mismas.
		\begin{figure}
			\Imagen{example-image-a}
			%\Imagen{uml/hole_filling.pdf}
			\caption{\label{fig:filling_class}Diagrama de clases del módulo de registración}
		\end{figure}

		\begin{itemize}
			\item {\bfseries Relleno de huecos:} La clase se encarga de estimar
				nuevos puntos en zonas donde se carece de información (huecos)
				y triangularlos para que la \emph{Malla} sea cerrada.
			\item {\bfseries Borde:} Es una colección de puntos ordenados
				que representa un borde de un hueco en la \emph{Malla}.
		\end{itemize}

	\subsection{Advancing front}
		Con el fin de simplificar la identificación de los huecos primeramente se eliminan
		todas las islas al quedarse únicamente con la componente conectada que
		contiene la mayor cantidad de puntos.
		De esta forma, una arista que defina sólo un triángulo formará parte de un hueco,
		y podrá obtenerse el contorno del mismo recorriendo el grafo de conectividades. 

		Para realizar el rellenado se implementó una variante del método de \emph{advancing front},
		\cite{advance_front}
		descripta en el algoritmo~\ref{alg:adv_front}. Los nuevos puntos
		insertados son elegidos de forma que los triángulos resultantes sean
		aproximadamente equiláteros, como se detalla en el algoritmo~\ref{alg:new_point}.
		Si el nuevo punto cae cerca de alguno ya existente, se utiliza aquel,
		actualizando de forma apropiada el advancing front.

		\begin{algorithm}
			\begin{algorithmic}[1]
				\Function{Advancing front}{Malla, Contorno}
					\State AF $\gets$ Contorno
					\Repeat
					\State $\alpha = \widehat{PCN} =$ ángulo mínimo(Contorno)
					\If{$\alpha < 75^{\circ}$}
						\State Malla.agregar triángulo(P, C, N)
						\State AF.eliminar punto(C)
					\ElsIf{$\alpha < 135^{\circ}$}
						\State nuevo $\gets$ crear punto(P, C, N, $\alpha/2$)
						\State Malla.agregar punto(nuevo)
						\State Malla.agregar triángulo(nuevo, C, N)
						\State AF.insertar punto(nuevo)
					\ElsIf{$\alpha < 180^{\circ}$}
						\State nuevo $\gets$ crear punto(P, C, N, $\alpha/3$)
						\State Malla.agregar punto(nuevo)
						\State Malla.agregar triángulo(nuevo, C, N)
						\State AF.insertar punto(nuevo)
					\EndIf
					\Until $\mbox{AF} \neq \emptyset$
				\EndFunction
			\end{algorithmic}
			\caption{\label{alg:adv_front}Relleno de huecos mediante el método de \emph{advancing front}}
		\end{algorithm}

		\begin{algorithm}
			\begin{algorithmic}[1]
				\Function{crear punto}{P, C, N, $\theta$}
					\State planoA $\gets$ plano(P, C, N)
					\State planoB $\gets \left\{
						\begin{tabular}{l}
							.punto $\gets$ promedio(P, C, N) \\
							.normal $\gets$ promedio(P.normal, C.normal, N.normal)
						\end{tabular}
						\right.$
					\State Q $\gets$ rotar(
						punto = N,
						origen = C,
						\Statex normal = planoA.normal,
						ángulo = $\theta$
						)
					\State \Return proyección(Q, planoB)
				\EndFunction
			\end{algorithmic}
			\caption{\label{alg:new_point}Creación del nuevo punto}
		\end{algorithm}

		Con este método se pueden rellenar agujeros pequeños, obteniéndose una malla bastante regular (figura~\ref{fig:fill_good}).
		Sin embargo, debido a la localidad con la que se generan los nuevos
		puntos, el frente puede diverger o pretender unirse a puntos que no
		forman parte del contorno del hueco, resultando una malla mal formada,
		con aristas que corresponden a más de dos caras (figura~\ref{fig:fill_bad}).
		Para evitar la divergencia es necesario definir una superficie de
		soporte que asegure que los nuevos puntos no excedan los límites del
		hueco.

	\begin{figure}
		\Imagen{img/fill_good}
		\caption{\label{fig:fill_good}Relleno de un hueco pequeño mediante \emph{advancing front}.}
	\end{figure}

	\begin{figure}
		\Imagen{img/fill_bad}
		\caption{\label{fig:fill_bad}Fallo en el algoritmo de \emph{advancing front}. Se intentó completar un triángulo con un punto que no pertenecía al borde.}
	\end{figure}

	\subsection{Reconstrucción de Poisson}
	%Poisson
	La clase pcl::Poisson provee algoritmos de reconstrucción basados en \cite{Kazhdan:2006:PSR:1281957.1281965}, %cite Poisson surface reconstruction
	siendo el principal parámetro la profundidad del octree utilizado,
	impactando directamente en la resolución de la malla resultante.


	Entonces, se convierte el problema en la resolución de una ecuación de Poisson de la forma:
	\[\Delta\chi \equiv \nabla \cdot\nabla\chi = \nabla \vec{n}\]

	Se realiza una discretización del dominio mediante un octree, ya que sólo
	interesa la solución de la función en la proximidad de la superficie a
	reconstruir.
	Luego se definen funciones de soporte local que aproximen una gaussiana.

	Una vez resuelto el problema en el dominio, se extrae la isosuperficie mediante una variante del método de marching cubes.

	La malla resultante presenta un hueco en la base (figura~\ref{fig:fill_poisson}).
	Ya que los puntos del contorno del hueco pertenecen a un mismo plano, es posible rellenarlo mediante el método de \emph{advancing front}.

	\begin{figure}
		\Imagen{img/fill_poisson}
		\caption{\label{fig:fill_poisson}Reconstrucción de la superficie mediante el método de \emph{Poisson}. Todos los huecos fueron rellenados a excepción de la base.}
	\end{figure}

