\section{Módulo de rellenado de huecos}
	Al finalizar el algoritmo de fusión se obtuvo una malla triangular a partir
	de la información proveniente de cada vista.
	Sin embargo, esta malla no es cerrada ya que existen zonas que ninguna
	vista pudo capturar y por lo tanto carecen de puntos, produciendo huecos en la misma.

	El módulo de rellenado de huecos se encargará de estimar, de forma automática, la superficie del
	objeto en estas zonas para así obtener finalmente una malla cerrada.

	Se plantearon dos métodos:
	\begin{itemize}
		\item \emph{Advancing front}, que trabaja localmente con los puntos que forman el contorno de cada hueco.
		\item Reconstrucción de Poisson, que trabaja con todos los puntos de la nube a la vez. 
	\end{itemize}


	\subsection{Advancing front}
		En este método, cada hueco se rellenará de forma independiente a los otros,
		utilizando únicamente los puntos que conforman cada contorno para estimar
		las posiciones de los nuevos puntos.
		Tomando en cuenta esas consideraciones, se diseñó el diagrama de clases presentado en la figura~\ref{fig:filling_class},
		cuyas clases principales se describen a continuación:
		\begin{itemize}
			\item {\bfseries Relleno de huecos:} La clase se encarga de estimar
				nuevos puntos en zonas donde se carece de información (huecos)
				y triangularlos para que la \emph{Malla} sea cerrada.
			\item {\bfseries Borde:} Es una colección de puntos ordenados
				que representa un borde de un hueco en la \emph{Malla}.
		\end{itemize}

		\begin{figure}
			\Imagen{uml/hole_filling}
			\caption{\label{fig:filling_class}Diagrama de clases del módulo de registración}
		\end{figure}

		Es posible que dentro de un hueco se observen puntos que no lograron
		conectarse al resto de la malla.  Si bien estas islas nos brindan
		información de la superficie, su presencia dificulta la identificación
		del contorno de cada hueco.  Por esta razón, se eliminaron todas las
		islas al trabajar únicamente con la componente conectada que contenía
		la mayor cantidad de puntos.
		De esta forma, una arista que defina sólo un triángulo formará parte de un hueco,
		y podrá obtenerse el contorno del mismo recorriendo el grafo de conectividades. 

		\TODO{gráficos}
		El rellenado se implementó mediante una variante del método de
		\emph{advancing front}\cite{advance_front}, descripta en el
		algoritmo~\ref{alg:adv_front}.

		Los nuevos puntos insertados son elegidos de forma que los triángulos resultantes sean
		aproximadamente equiláteros, como se detalla en el algoritmo~\ref{alg:new_point}.
		Estos puntos son luego proyectados en un plano de soporte definido
		mediante las normales de los puntos del ángulo candidato.

		En caso de que el nuevo punto cayese cerca de otro ya existente, se utilizará aquel.
		Esto implica dividir en dos el hueco, y cada nueva porción se rellenará de forma independiente.
		\TODO{gráfico}

		\begin{algorithm}
			\begin{algorithmic}[1]
				\Function{Advancing front}{Malla, Contorno}
					\State AF $\gets$ Contorno
					\Repeat
					\State $\alpha = \widehat{PCN} =$ ángulo mínimo(Contorno)
					\If{$\alpha < 75^{\circ}$}
						\State Malla.agregar triángulo(P, C, N)
						\State AF.eliminar punto(C)
					\ElsIf{$\alpha < 135^{\circ}$}
						\State nuevo $\gets$ crear punto(P, C, N, $\alpha/2$)
						\State Malla.agregar punto(nuevo)
						\State Malla.agregar triángulo(nuevo, C, N)
						\State AF.insertar punto(nuevo)
					\ElsIf{$\alpha < 180^{\circ}$}
						\State nuevo $\gets$ crear punto(P, C, N, $\alpha/3$)
						\State Malla.agregar punto(nuevo)
						\State Malla.agregar triángulo(nuevo, C, N)
						\State AF.insertar punto(nuevo)
					\EndIf
					\Until $\mbox{AF} \neq \emptyset$
				\EndFunction
			\end{algorithmic}
			\caption{\label{alg:adv_front}Relleno de huecos mediante el método de \emph{advancing front}.
			Los umbrales fueron elegidos de forma de obtener triángulos con ángulos cercanos a $60^{\circ}$.}
		\end{algorithm}

		\begin{algorithm}
			\begin{algorithmic}[1]
				\Function{crear punto}{P, C, N, $\theta$}
					\State planoA $\gets$ plano(P, C, N)
					\State planoB $\gets \left\{
						\begin{tabular}{l}
							.punto $\gets$ promedio(P, C, N) \\
							.normal $\gets$ promedio(P.normal, C.normal, N.normal)
						\end{tabular}
						\right.$
					\State Q $\gets$ rotar(
						punto = N,
						origen = C,
						\Statex normal = planoA.normal,
						ángulo = $\theta$
						)
					\State \Return proyección(Q, planoB)
				\EndFunction
			\end{algorithmic}
			\caption{\label{alg:new_point}Creación del nuevo punto}
		\end{algorithm}

		Con este método se pueden rellenar agujeros pequeños, obteniéndose una malla bastante regular (figura~\ref{fig:fill_good}).
		Sin embargo, debido a la localidad con la que se generan los nuevos
		puntos, el frente puede diverger o pretender unirse a puntos que no
		forman parte del contorno del hueco, resultando una malla mal formada,
		con aristas que corresponden a más de dos caras (figura~\ref{fig:fill_bad}).
		Para evitar la divergencia es necesario definir una superficie de
		soporte considerando todo el contorno del hueco, de forma de asegurar
		que los nuevos puntos no excedan los límites del hueco.

	\begin{figure}
		\Imagen{img/fill_good}
		\caption{\label{fig:fill_good}Relleno de un hueco pequeño mediante \emph{advancing front}.}
	\end{figure}

	\begin{figure}
		\Imagen{img/fill_bad}
		\caption{\label{fig:fill_bad}Fallo en el algoritmo de \emph{advancing front}. Se intentó completar un triángulo con un punto que no pertenecía al borde.}
		\TODO{cambiar gráfico}
	\end{figure}

	\subsection{Reconstrucción de Poisson}
	\TODO{¿dónde cito? casi todo es de \cite{Kazhdan:2006:PSR:1281957.1281965}}
	En este método el rellenado de huecos se aborda como un problema de
	reconstrucción de la superficie a partir de puntos orientados.  Por esta
	razón, se consideran todos los puntos a la vez sin consideraciones
	especiales de pertenencia o no a los contornos de los huecos.
	Además, se convierte el problema de la reconstrucción a la resolución de
	una ecuación de Poisson.

	En esta ecuación de Poisson, se calculará una función indicadora $\chi$
	que tomará el valor 1 en los puntos que se encuentran dentro del modelo
	y 0 en los puntos que se encuentran fuera.
	De esta manera, extrayendo una isosuperficie apropiada se obtendrá la frontera del modelo.

	Para poder plantear la ecuación, debe notarse que el gradiente de la función indicadora $\nabla \chi$
	es 0 en casi todos los puntos, excepto en aquellos cerca de la superficie,
	donde es equivalente a las normales de los puntos muestreados ($\vec{V}$), pero en sentido contrario (figura~\ref{fig:poiss_planteo}).
	Entonces, se busca calcular la función $\chi$ cuyo gradiente aproxime de la mejor forma posible el
	campo vectorial $\vec{V}$.
	Aplicando  el operador de divergencia se obtiene la ecuación de Poisson
	\[ \Delta\chi \equiv \nabla \cdot \nabla \chi = \nabla\vec{V} \]

	\begin{figure}
		\Imagen{example-image-a}
		\caption{\label{fig:poiss_planteo}Ilustración del método de reconstrucción de Poisson.
		Tomando como entrada una nube de puntos con sus normales $\vec{V}$ se define el gradiente de la función
		indicadora $\nabla \chi$. Mediante la ecuación de Poisson se calculará el valor de $\chi$,
		y al extraer una isosuperfice se obtendrá la superficie del objeto $\partial M$. 
		}
	\end{figure}

	Entre las ventajas de esta formulación podemos nombrar:
	\begin{itemize}
		\item La superficie reconstruida es una superficie suave y cerrada.
		\item Los sistemas de Poisson son resistentes al ruido de entrada.
		\item El uso de funciones de soporte local para aproximar $\chi$,
			produce un sistema lineal bien condicionado y ralo.\cite{Kazhdan:2006:PSR:1281957.1281965}
	\end{itemize}

		\subsubsection{Condiciones de borde}
		Para definir el sistema lineal, es necesario especificar el
		comportamiento de la función a lo largo del contorno del dominio de
		integración.
		Tomando en consideración la definición de la función indicadora $\chi$,
		se pueden establecer fácilmente condiciones de borde Dirichlet ($\chi = 0$)
		o también condiciones de borde Neumann ($\nabla \chi = 0$).

		Si se utilizan condiciones Dirichlet, se asegurará que la superficie resultante sea cerrada.
		En cambio, en caso de que se tengan regiones sin información (huecos) de tamaño
		considerable, usar condiciones Neumann permitirá que la superficie se
		extienda fuera del dominio de integración, cruzando el borde de forma
		ortogonal (figura~\ref{fig:poisson_boundary})\cite{Kazhdan_screenedpoisson}.
		\begin{figure}
			\Imagen{diagram/poisson_boundary_conditions}
			\caption{\label{fig:poisson_boundary}Reconstrucción del modelo Ángel (izquierda) usando condiciones de borde Dirichlet (centro) y Neumann (derecha).}
		\end{figure}


		\subsubsection{Extracción de la isosuperficie}
		Para poder obtener la superficie del modelo $\partial M$, primeramente es necesario
		definir un isovalor de $\chi$ para así obtener la isosuperficie correspondiente.
		Para aproximar las posiciones de los puntos de entrada, se elegirá como isovalor
		el promedio de los valores de $\chi$ en esos puntos.

		\[\partial M \equiv {q \in \mathbb{R}^3 | \chi(q) = \gamma} \quad \mbox{con}
			\quad \gamma = \frac{1}{N} \sum \chi(p) \]




	\subsubsection{Implementación}
	La clase \texttt{Poisson} de la biblioteca \emph{PCL} implementa este método de reconstrucción,
	imponiendo condiciones de borde Neumann.

	Debido a que sólo es de interés el valor de $\chi$ en los puntos cercanos a
	la superficie, se utiliza un octree para representar esta función. Se
	provee de parámetros para establecer la profundidad del octree, controlando
	de esta manera la resolución de la superficie reconstruida.
	Sin embargo, debe considerarse que el consumo de memoria y el tiempo se incrementan de forma
	cuadrática con la profundidad del octree.
	%la cual se extrae mediante el método de \emph{Marching Cubes}. \TODO{referencia a marching cubes}

	Debido a que no se cuentan con puntos en la base de apoyo de los objetos, el uso de condiciones de borde Neumann
	producirá un estiramiento hacia abajo de la superficie resultante y la presencia de
	un hueco plano en la base (figura~\ref{fig:fill_poisson}.
	Si bien este hueco podrá ser luego rellenado mediante el algoritmo de \emph{advancing front}, no puede solucionarse el estiramiento.



	\begin{figure}
		\Imagen{img/fill_poisson} \TODO{usar dragon up}
		\caption{\label{fig:fill_poisson}Reconstrucción de la superficie mediante el método de \emph{Poisson}. Todos los huecos fueron rellenados a excepción de la base.}
	\end{figure}

