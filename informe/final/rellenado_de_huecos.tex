\section{Módulo de rellenado de huecos}
	Al finalizar el algoritmo de fusión se obtiene una malla triangular a partir
	de la información proveniente de cada vista.
	Sin embargo, esta malla no es cerrada ya que existen zonas que ninguna
	vista pudo capturar y por lo tanto carecen de puntos, produciendo huecos en la misma.

	El módulo de rellenado de huecos se encarga de estimar, de forma automática, la superficie del
	objeto en estas zonas para así obtener finalmente una malla cerrada.

	Se plantearon dos métodos:
	\begin{itemize}
		\item Un algoritmo de \emph{advancing front}, que trabaja localmente con los puntos que forman el contorno de cada hueco.
		\item Reconstrucción de Poisson, que trabaja con todos los puntos de la nube a la vez. 
	\end{itemize}


	\subsection{\label{part:adv_font_implementación}Advancing front}
		Se implementó una variante del método propuesto por \cite{advance_front},
		la cual se halla descripta en el algoritmo~\ref{alg:adv_front}.
		Cada hueco se rellena de forma independiente a los otros,
		utilizando únicamente los puntos que conforman cada contorno para estimar
		las posiciones de los nuevos puntos.

		Es posible que dentro de un hueco se observen islas, es decir,
		elementos que no lograron conectarse al resto de la malla.
		Si bien estas islas nos brindan información de la superficie,
		su presencia dificulta la identificación del contorno de cada hueco.
		Por esta razón, se eliminan todas las islas,
		quedándose únicamente con la componente conectada que contiene
		la mayor cantidad de puntos.
		De esta forma, una arista que defina sólo un triángulo forma parte de un hueco,
		y puede obtenerse el contorno del mismo recorriendo el grafo de conectividades. 

		El frente se inicializa a partir del contorno del hueco, y se recorre el borde
		calculando los ángulos entre las aristas a fin de encontrar los tres puntos
		que definen el menor ángulo $\alpha$. Según el valor de $\alpha$, se determina
		si insertar o no un nuevo punto (figura~\ref{fig:af_triangle}).
		Los nuevos puntos insertados son elegidos de forma que los triángulos resultantes sean
		aproximadamente equiláteros, como se detalla en el algoritmo~\ref{alg:new_point}.
		Estos puntos son luego proyectados en un plano de soporte definido
		mediante las normales de los puntos del ángulo candidato.

		En caso de que el nuevo punto cayese cerca de otro ya existente, se utiliza este último.
		Esto implica dividir en dos el hueco, y rellenar cada nueva porción de forma independiente (figura~\ref{fig:af_split}).

		\begin{figure}
			\centering
			\input{diagram/af_new.pdf_tex}
			\caption[Creación de los nuevos triángulos mediante un algoritmo de advancing front]{\label{fig:af_triangle}Creación de los nuevos triángulos
			mediante un algoritmo de advancing front. Dependiendo del valor del
			ángulo candidato $\alpha$ se utilizan los puntos existentes
			(izquierda), o se crea un nuevo punto en $\alpha/2$ (centro) o
			$\alpha/3$ (derecha).}
		\end{figure}

		\begin{figure}
			\centering
			%\Imagen{diagram/af_split.pdf}
			\input{diagram/af_split.pdf_tex}
			\caption[División del frente en dos]{\label{fig:af_split}División del frente en dos
			al utilizar un punto de la malla para crear el nuevo triángulo.}
		\end{figure}

		\begin{algorithm}
			\begin{algorithmic}[1]
				\Function{Advancing front}{Malla, Contorno}
					\State AF $\gets$ Contorno
					\Repeat
					\State $\alpha = \angle PCN =$ ángulo mínimo(Contorno)
					\If{$\alpha < 75^{\circ}$}
						\State Malla.agregar triángulo(P, C, N)
						\State AF.eliminar punto(C)
					\ElsIf{$\alpha < 135^{\circ}$}
						\State Q $\gets$ crear punto(P, C, N, $\alpha/2$)
						\State Malla.agregar punto(Q)
						\State Malla.agregar triángulo(Q, C, N)
						\State AF.insertar punto(Q)
					\ElsIf{$\alpha < 180^{\circ}$}
						\State Q $\gets$ crear punto(P, C, N, $\alpha/3$)
						\State Malla.agregar punto(Q)
						\State Malla.agregar triángulo(Q, C, N)
						\State AF.insertar punto(Q)
					\EndIf
					\Until $\mbox{AF} \neq \emptyset$
				\EndFunction
			\end{algorithmic}
			\caption[Relleno de huecos mediante el método de \emph{advancing front}]{\label{alg:adv_front}Relleno de huecos mediante el método de \emph{advancing front}.
			Los umbrales fueron elegidos de forma de obtener triángulos con ángulos cercanos a $60^{\circ}$.}
		\end{algorithm}

		\begin{algorithm}
			\begin{algorithmic}[1]
				\Function{crear punto}{P, C, N, $\theta$}
					\State planoA $\gets$ plano(P, C, N)
					\State planoB $\gets \left\{
						\begin{tabular}{l}
							.punto $\gets$ promedio(P, C, N) \\
							.normal $\gets$ promedio(P.normal, C.normal, N.normal)
						\end{tabular}
						\right.$
					\State Q $\gets$ rotar(
						punto = N,
						origen = C,
						\Statex normal = planoA.normal,
						ángulo = $\theta$
						)
					\State \Return proyección(Q, planoB)
				\EndFunction
			\end{algorithmic}
			\caption[Creación del nuevo punto]{\label{alg:new_point}Creación del nuevo punto}
		\end{algorithm}


	\subsection{Reconstrucción de Poisson}
	\subsubsection{Implementación}
	La clase \texttt{Poisson} de la biblioteca \emph{PCL} implementa este método de reconstrucción
	imponiendo condiciones de borde Neumann.

	Debido a que sólo es de interés el valor de $\chi$ en los puntos cercanos a
	la superficie, se utiliza un octree para representar esta función. Se
	provee de parámetros para establecer la profundidad del octree, controlando
	de esta manera la resolución de la superficie reconstruida.
	Sin embargo, debe considerarse que el consumo de memoria y el tiempo se incrementan de forma
	cuadrática con la resolución, observándose un incremento en un factor de 4 por cada aumento de la
	profundidad del octree\cite{Kazhdan:2006:PSR:1281957.1281965}.


	Debido a que no se cuentan con puntos en la base de apoyo de los objetos,
	el uso de condiciones de borde Neumann
	produce un estiramiento hacia abajo de la superficie resultante y la presencia de
	un hueco plano en la base (figura~\ref{fig:fill_poisson}.

	\begin{figure}
		\Imagen{img/bun_poisson}
		\caption[Reconstrucción de la superficie mediante el método de \emph{Poisson}]{\label{fig:fill_poisson}Reconstrucción de la superficie mediante el método de \emph{Poisson}. Todos los huecos fueron rellenados a excepción del presente en la base de apoyo (destacado en verde), donde se observa, además, un estiramiento debido a las condiciones de borde Neumann.}
	\end{figure}

