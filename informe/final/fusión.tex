\section{Módulo de fusión}
	%¿citas?
	%en el de loop correction (surfel)
	El módulo de fusión tiene como objetivo obtener un modelo que
	describa la geometría del objeto escaneado.
	Para esto, el modelo se inicializa a partir de una captura cualquiera
	y se procesan una a una las capturas correspondientes a las otras vistas.
	En cada nueva vista se agrega información del objeto en zonas que no eran antes visibles
	y además se confirma o refuta la información ya presente en el modelo.
	Al combinar esta información, el módulo de fusión obtiene finalmente una malla
	que represente a la totalidad del objeto.

	\subsection{Método}
	Se utiliza una representación de \emph{surfel} para cada punto similar a la propuesta en \cite{5457479} %ref in-hand scanning
	debido a la facilidad de implementación de las funciones de actualización de los puntos.

	Cada surfel tiene asociado un valor de confianza y las vistas en las que
	fue observado.  El valor de confianza nos indica la probabilidad de que sus
	valores de posición y normal no sean producto de un error de muestreo.
	Este valor se inicializa según el ángulo de su normal respecto a la línea
	de la cámara (figura~\ref{fig:confianza_surfel}).
	%y a su distancia al centro de la captura %terminé usando sólo las normales



	El algoritmo~\ref{alg:surfel} describe el agregado de una nueva vista.
	Por cada punto de la vista se busca qué surfel lo contiene y se actualiza
	su posición y normal ponderando según el nivel de confianza.
	\begin{eqnarray*}
		\hat{P}_k = \frac{\sum_{j} \alpha_j P^{(j)}_k}{\sum_{j} \alpha_j} \\
		\hat{n}_k = \frac{\sum_{j} \alpha_j n^{(j)}_k}{\sum_{j} \alpha_j}
	\end{eqnarray*}
	En caso de que haya caído fuera del dominio de la reconstrucción actual, se
	considera que es un nuevo punto y se lo agrega.
	Si la distancia de proximidad elegida es demasiado pequeña,
	nunca se realizará la actualización ya que todos los puntos serán considerados como nuevos surfels.
	En cambio, si la distancia es muy grande, se considerarán puntos
	que deberían haber sido descartados como atípicos.


	Al terminar de procesar todas las vistas, habrá surfels que hayan sido
	observados desde sólo una posición y tengan un nivel de confianza bajo.
	Estos son considerados como ruido y eliminados de la
	reconstrucción.

	Finalmente, se procede a la obtención de una triangulación de los surfels.
	Para esto se empleó el algoritmo de \emph{Greedy Projection Triangulation} que provee PCL,
	el cual realiza una proyección de la vecindad de un punto en el plano definido por su normal
	y procede a conectar los puntos de modo que los triángulos resultantes
	respeten umbrales de longitud y ángulos definidos por el usuario.

	\begin{figure}
		\Imagen{img/bunny_confianza}
		\caption[Visualización de los valores de confianza de los súrfeles]{\label{fig:confianza_surfel}Visualización de los valores de confianza de los súrfeles para la captura \texttt{bun000}.}
	\end{figure}


	\begin{algorithm}
		\begin{algorithmic}[1]
			\Function{Agregar vista}{vista, reconstrucción}
				\ForAll{$p \in \mbox{vista.puntos}$}
					\State surfel $\gets$ reconstrucción.buscar(p)
					\If {surfel = $\emptyset$}
						\State reconstrucción.agregar(p)
					\Else
						\State surfel.actualizar(p)
					\EndIf
				\EndFor
			\EndFunction
		\end{algorithmic}
		\caption[Actualización de la reconstrucción al agregar una nueva vista]{\label{alg:surfel}Actualización de la reconstrucción al agregar una nueva vista.}
	\end{algorithm}

	\begin{figure}
		\Imagen{img/bun_fusion}

		%\Imagen{img/bun_fusion_conf}
		\caption[Superficie reconstruida]{\label{fig:surface}Superficie reconstruida. En verde se destacan los bordes de los huecos.}
	\end{figure}
