\section{Módulo de fusión}
	%¿citas?
	%en el de loop correction (surfel)
	Cada nueva vista agrega información del objeto en zonas que no eran antes visibles
	y además confirma o refuta la información ya presente en las zonas comunes a las otras vistas.
	El módulo de fusión se encarga de combinar esta información para obtener
	finalmente una malla que represente la porción observada del objeto.

	%diagramas de clase
	\subsection{Diagrama de clases}
		En la figura~\ref{fig:fusion_class} se presentan las clases principales y sus interacciones.
		A continuación se presenta una breve descripción de las más importantes.
		\begin{figure}
			\Imagen{uml/fusion.pdf}
			\caption{\label{fig:fusion_class}Diagrama de clases del módulo de fusión}
		\end{figure}

		\begin{itemize}
			\item {\bfseries Fusión:} une las \emph{Nubes} ya alineadas para
				obtener una superficie global que represente al objeto.
				Esto supone corregir la posición de los puntos, descartar
				aquellos considerados como ruido y triangular la superficie.
			\item {\bfseries Malla:} triangulación de la \emph{Nube} representada por un grafo de conectividades (\emph{DCEL}).
		\end{itemize}

	%informes de prueba

	%casos de uso/funcionalidades resueltas
	\subsection{Método}
	Se utiliza una representación de \emph{surfel} para cada punto similar a la propuesta en \cite{5457479} %ref in-hand scanning
	debido a la facilidad de implementación de las funciones de actualización de los puntos.

	Cada surfel tiene asociado un valor de confianza y las vistas en las que
	fue observado.  El valor de confianza nos indica la probabilidad de que sus
	valores de posición y normal no sean producto de un error de muestreo.
	Este valor se inicializa según el ángulo de su normal respecto a la línea
	de la cámara y a su distancia al centro de la captura.


	El algoritmo~\ref{alg:surfel} describe el agregado de una nueva vista.
	Por cada punto de la vista se busca qué surfel lo contiene y se actualiza
	su posición y normal ponderando según el nivel de confianza.
	En caso de que haya caído fuera del dominio de la reconstrucción actual, se
	considera que es un nuevo punto y se lo agrega. \TODO{fórmula de actualización}

	\begin{algorithm}
		\begin{algorithmic}[1]
			\Function{Agregar vista}{vista, reconstrucción}
				\ForAll{$p \in \mbox{vista.puntos}$}
					\State surfel $\gets$ reconstrucción.buscar(p)
					\If surfel = $\emptyset$
						\State reconstrucción.agregar(p)
					\Else
						\State surfel.actualizar(p)
					\EndIf
				\EndFor
			\EndFunction
		\end{algorithmic}
		\caption{\label{alg:surfel}Actualización de la reconstrucción al agregar una nueva vista.}
	\end{algorithm}

	Al terminar de procesar todas las vistas, habrá surfels que hayan sido
	observados desde sólo una posición y tengan un nivel de confianza bajo.
	Estos son considerados como ruido y eliminados de la
	reconstrucción.

	Finalmente, se procede a la obtención de una triangulación de los surfels.
	Para esto se utiliza el algoritmo de \emph{Greedy Projection Triangulation} que provee PCL.

	\begin{figure}
		\Imagen{img/bun_reconst}
		\caption{\label{fig:surface}Superficie reconstruída.}
	\end{figure}
